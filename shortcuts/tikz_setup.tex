\usetikzlibrary{matrix}

\usetikzlibrary{cd}

% In order to remove 1 pixel line at end of equal arrows.
\usetikzlibrary{nfold}

% From quiver:
% `pathmorphing` is necessary to draw squiggly arrows.
\usetikzlibrary{decorations.pathmorphing}

% From quiver:
% `calc` is necessary to draw curved arrows.
\usetikzlibrary{calc}

% Externalize TikZ diagrams to save compilation time.
% NOTE: Doesn't work with tikz-cd.
% \usetikzlibrary{external}
% \tikzexternalize[prefix=tikz/]

% \usepackage{memoize}
% \mmzset{memo dir}

% Marius Thaule TikZ matrix.
\newcommand{\diagram}[3]{\matrix[ampersand replacement = \&] (#1) [matrix of math nodes,row
  sep={#2},column sep={#3},text height=1.5ex,text
  depth=0.25ex]}


% Set style of TikZ pictures to tikz-cd style.
\tikzset{every picture/.append style={commutative diagrams/every diagram}}
\tikzset{math/.style = {commutative diagrams/every arrow,
  commutative diagrams/every label,
  execute at begin node=\(, execute at end node=\)}
}

% Shortcuts to tikz-cd styles.
\tikzset{hook/.style = {commutative diagrams/hook}}
\tikzset{dashed/.style = {commutative diagrams/dashed}}
\tikzset{two heads/.style = {commutative diagrams/two heads}}
\tikzset{equal/.style = {commutative diagrams/equal, nfold}}
\tikzset{squiggly/.style = {commutative diagrams/squiggly}}
\tikzset{marking/.style = {commutative diagrams/marking}}

% From quiver:
% A TikZ style for curved arrows of a fixed height, due to AndréC.
\tikzset{curve/.style={settings={#1},to path={(\tikztostart)
    .. controls ($(\tikztostart)!\pv{pos}!(\tikztotarget)!\pv{height}!270:(\tikztotarget)$)
    and ($(\tikztostart)!1-\pv{pos}!(\tikztotarget)!\pv{height}!270:(\tikztotarget)$)
    .. (\tikztotarget)\tikztonodes}},
    settings/.code={\tikzset{quiver/.cd,#1}
        \def\pv##1{\pgfkeysvalueof{/tikz/quiver/##1}}},
    quiver/.cd,pos/.initial=0.35,height/.initial=0
}

% Needed for suspension style.
\usetikzlibrary{decorations.markings}

% Marks the arrows with a suspension style.
\tikzset{
  suspension/.style = {postaction = decorate,
      decoration = {
          markings,
          mark = at position 0.3 with {\draw[-] (0,-0.075) -- (0,0.075);}
      },
  },
}
    