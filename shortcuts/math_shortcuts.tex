% Blackboard shortcuts
\newcommand{\Fb}{{\mathbb{F}}}
\newcommand{\Nb}{{\mathbb{N}}}
\newcommand{\Qb}{{\mathbb{Q}}}
\newcommand{\Rb}{{\mathbb{R}}}
\newcommand{\Zb}{{\mathbb{Z}}}

% Caligraphy shortcuts
\newcommand{\Ac}{{\mathcal{A}}}
\newcommand{\Bc}{{\mathcal{B}}}
\newcommand{\Cc}{{\mathcal{C}}}
\newcommand{\Ic}{{\mathcal{I}}}
\newcommand{\Kc}{{\mathcal{K}}}
\newcommand{\Mc}{{\mathcal{M}}}
\newcommand{\Nc}{{\mathcal{N}}}
\newcommand{\Pc}{{\mathcal{P}}}
\newcommand{\Tc}{{\mathcal{T}}}

% Set management shortcuts
\newcommand{\intersect}{\mathop{\cap}\limits}
\newcommand{\union}{\mathop{\cup}\limits}
\newcommand{\directsum}{\mathop{\oplus}\limits}

% \newcommand{\Stmod}[1]{\stablemod\tuple{ #1 }}

% Shorthands
% \newcommand{\abs}[1]{ \lvert #1 \rvert }
% \newcommand{\set}[1]{ \left\{ #1 \right\} }
% \newcommand{\tuple}[1]{ \left( #1 \right) }
% \newcommand{\toda}[1]{ \langle #1 \rangle }
% \newcommand{\class}[1]{ \left[ #1 \right] }
% \newcommand{\massey}[1]{ \langle \! \langle #1 \rangle \! \rangle }

% SRC: https://tex.stackexchange.com/questions/94410/easily-change-behavior-of-declarepaireddelimiter
% \NewDocumentCommand\xDeclarePairedDelimiter{mmm}
%  {%
%   \NewDocumentCommand#1{som}{%
%    \IfNoValueTF{##2}
%     {\IfBooleanTF{##1}{#2##3#3}{\mleft#2##3\mright#3}}
%     {\mathopen{##2#2}##3\mathclose{##2#3}}%
%   }%
%  }
% \xDeclarePairedDelimiter{\set}{\lbrace}{\rbrace}

% Swaps * and non * version of \DeclarePairedDelimiter, but doesnt work.
% \newcommand\swapifbranches[3]{#1{#3}{#2}}
% \makeatletter
% \MHInternalSyntaxOn
% \patchcmd{\DeclarePairedDelimiterX}{\@ifstar}{\swapifbranches\@ifstar}{}{}
% \MHInternalSyntaxOff
% \makeatother

% Double barackets: https://tex.stackexchange.com/questions/79657/how-to-get-double-angle-bracket-without-using-mnsymbol-package
% Trenge Mathtools 1.30, eg har 1.29.... Ser bra ut, då.
% \makeatletter
% \newsavebox{\@bra}
% \newsavebox{\@brb}
% \DeclarePairedDelimiterX{\massey}[1]{.}{.}{%
%   \delimsize\langle%
%   \hspace*{0.3mm}\hspace*{0.55mm}\savebox{\@bra}{\(\displaystyle\left\langle\vphantom{#1}\right.\)}\hspace*{-1.035\wd\@bra}%
%   \delimsize\langle%
%   #1% 
%   \delimsize\rangle%
%   \hspace*{0.3mm}\hspace*{0.55mm}\savebox{\@brb}{\(\displaystyle\left.\vphantom{#1}\right\rangle\)}\hspace*{-1.035\wd\@brb}%
%   \delimsize\rangle
% }
% \makeatother


\DeclarePairedDelimiterX{\abs}[1]{\vert}{\vert}{#1}
\DeclarePairedDelimiterX{\set}[1]{\{}{\}}{#1}
\DeclarePairedDelimiterX{\tuple}[1]{(}{)}{#1}
\DeclarePairedDelimiterX{\toda}[1]{\langle}{\rangle}{#1}
\DeclarePairedDelimiterX{\class}[1]{[}{]}{#1}
% \DeclarePairedDelimiterX{\massey}[1]{\lmangle}{\rmangle}{#1}
\DeclarePairedDelimiterX{\massey}[1]{\langle \! \langle}{\rangle \! \rangle}{#1}

% Maybe use \bigl og \bigr instead?
% \renewcommand*{\tuple}{\tuple*} % Not working? Tex capacity exceeded?

% New math operators
\DeclareMathOperator{\Id}{Id}
\DeclareMathOperator{\StMod}{StMod}
\DeclareMathOperator{\Stmod}{Stmod}
\DeclareMathOperator{\Obj}{Obj}
\DeclareMathOperator{\Hom}{Hom}
\DeclareMathOperator{\Mod}{Mod}
% \DeclareMathOperator{\mod}{mod}
\DeclareMathOperator{\coker}{coker}
\DeclareMathOperator{\im}{im}
\DeclareMathOperator{\Ab}{Ab}
\DeclareMathOperator{\Fun}{Fun}
\DeclareMathOperator{\dg}{dg}
\DeclareMathOperator{\op}{op}
\DeclareMathOperator{\C}{C}
\DeclareMathOperator{\D}{D}
\DeclareMathOperator{\dgMod}{dgMod}
\DeclareMathOperator{\dgFun}{dgFun}

% Toda bracket subscrips
\DeclareMathOperator{\cc}{cc}
\DeclareMathOperator{\fc}{fc}
\DeclareMathOperator{\ff}{ff}
