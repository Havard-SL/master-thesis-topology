% Teoremstil
\theoremstyle{plain}
\newtheorem{theorem}{Theorem}[subsection]
\newtheorem{proposition}[theorem]{Proposition}
\newtheorem{corollary}[theorem]{Corollary}
\newtheorem{lemma}[theorem]{Lemma}

% Definisjonstil
\theoremstyle{definition}
\newtheorem{definition}[theorem]{Definition}
\newtheorem{example}[theorem]{Example}
\newtheorem{remark}[theorem]{Remark}
\newtheorem{construction}[theorem]{Construction}
\newtheorem{notation}[theorem]{Notation}
\newtheorem{fact}[theorem]{Fact}
\newtheorem{question}[theorem]{Question}

% Fancy box engine?
% \newcommand{\createbox}[1]{
%     \tcolorboxenvironment{#1}{
%         breakable, % Makes boxes breakable.
%         enhanced jigsaw, % Required for making breaks fancy.
%         arc = 0mm, % Corners.
%         % sharp corners, % make the corners sharp
%         oversize, % Makes the box not "squish" the text, but rather extend the box into the margins.
%         colback = white, % Changes the background colour.
%         % parskip, % Behave better with parskip package (not sure what it does)
%         % beforeafter skip balanced = 0pt, % Change spacing before and after boxes.
%         parbox = false, % Make text formatting similar to the outside.
%         before upper=\vspace{-2\parskip}, % Fix for extra parskip that comes in the start of the box.
%     }
% }

% Simple box engine?
\newcommand{\createbox}[1]{
    \tcolorboxenvironment{#1}{
        breakable, % Makes boxes breakable.
        empty, % Empty skin.
        arc = 0mm, % Corners.
        % sharp corners, % make the corners sharp
        oversize, % Makes the box not "squish" the text, but rather extend the box into the margins.
        % colback = white, % Changes the background colour.
        % parskip, % Behave better with parskip package (not sure what it does)
        % beforeafter skip balanced = 0pt, % Change spacing before and after boxes.
        parbox = false, % Make text formatting similar to the outside.
        before upper=\vspace{-2\parskip}, % Fix for extra parskip that comes in the start of the box.
        borderline={1pt}{0pt}{black}
    }
}

\createbox{theorem}
\createbox{proposition}
\createbox{corollary}
\createbox{lemma}
\createbox{definition}
\createbox{example}
\createbox{remark}
\createbox{construction}
\createbox{notation}
\createbox{fact}
\createbox{question}

% TODO: Make proof connect to the theorem box. Maybe it will look better? Then it would be difficult to have text in between.
\createbox{proof}