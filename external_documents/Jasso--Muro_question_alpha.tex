\documentclass[a4paper, 12pt]{article}

% Right-justify text. Creates some overfull hbox-es, and works weirdly with microtype.
% \usepackage[document]{ragged2e}

% Make language-specific tweaks, like changing section/theorem words or adding more hyphonation points.
\usepackage[english]{babel}

% For å kunna skriva æøå i tekstar MERK: Blir automatisk ubrukeleg med lualatex og fontspec
% \usepackage[utf8]{inputenc}

\usepackage[T1]{fontenc}

% Adds hyphenation points and aims to improve paragraph rendering.
% Use [activate=false] to disable some features that causes issues with ragged2e. Not sure if setting this option disables every feature in the package entirely, or if it's still worth keeping then.
\usepackage{microtype}

% Enables memoization
% extract=no means that latexmk need to do the extracting itself. This is controlled by the latexmkrc file.
\usepackage[extract=no]{memoize}

% Uncomment the following to recompile every tikz-diagram.
% \mmzset{
%     recompile,
% }

% Fiksa margin
\usepackage[margin=2cm]{geometry}

% Fiksar datoformatet på tiitelen
\usepackage[ddmmyyyy]{datetime}

\usepackage{amssymb}

% For visse mattesymbol, typ \mathbb
\usepackage{amsmath}

% Bilete
\usepackage{graphicx}

% For kodesnuttar og resultat
% \usepackage{minted}

% Kan endra på korleis listar ser ut
\usepackage{enumitem}

% For autoref
\usepackage[hidelinks,colorlinks=true]{hyperref} 

% For fargar på ting ein referer til i autoref
\hypersetup{allcolors=[rgb]{0,0.31,0.62}}

% For not needing to compile twice with hyperref (?)
\usepackage{bookmark}

% For teorem, definisjon, bevis enviornments.
\usepackage{amsthm}

% For meir avanserte teoremkonstruksjonar.
\usepackage{thmtools}

% For psmallmatrix.
\usepackage{mathtools}

% For boksar rundt tekst.
\usepackage{tcolorbox}
\tcbuselibrary{skins} % For å ha meir fancy boksar, trengs for "enhanced".
\tcbuselibrary{breakable} % For å ha breakable boksar.

% For svgar
% \usepackage{svg}

% Set svg mappo
% \svgpath{svg/}

% Fjernar indents ved nye avsnitt, men gjer linjeavstanden kortare (Kanskje)
\usepackage{parskip} 

% Lualatex font greie
\usepackage{fontspec}

\usepackage[warnings-off={mathtools-colon,mathtools-overbracket}]{unicode-math}

% TikZ!
\usepackage{tikz}

% Set fontar som blir brukt
\setmathfont{Latin Modern Math} % Dette er standardfonten
\setmathfont[range=\setminus]{Asana Math} % Somehow setting this changes tikz-cd arrow style
% \setmainfont{Atkinson Hyperlegible}
% \setmainfont{GFS Neohellenic Math}
% \setmainfont{Fira Sans}
% \setmathfont{Fira Math}
% \setmathfont[range=\setminus]{Asana Math}

\usetikzlibrary{matrix}

\usetikzlibrary{cd}

% In order to remove 1 pixel line at end of equal arrows.
\usetikzlibrary{nfold}

% From quiver:
% `pathmorphing` is necessary to draw squiggly arrows.
\usetikzlibrary{decorations.pathmorphing}

% From quiver:
% `calc` is necessary to draw curved arrows.
\usetikzlibrary{calc}

% Externalize TikZ diagrams to save compilation time.
% NOTE: Doesn't work with tikz-cd.
% \usetikzlibrary{external}
% \tikzexternalize[prefix=tikz/]

% \usepackage{memoize}
% \mmzset{memo dir}

% Marius Thaule TikZ matrix.
\newcommand{\diagram}[3]{\matrix[ampersand replacement = \&] (#1) [matrix of math nodes,row
  sep={#2},column sep={#3},text height=1.5ex,text
  depth=0.25ex]}
% Modified to fix the distance between nodes. Useful for diagrams where there are diagonal arrows that should be paralell.
\newcommand{\diagramorigin}[3]{\matrix[ampersand replacement = \&] (#1) [matrix of math nodes,row
  sep={#2},column sep={#3, between origins},text height=1.5ex,text
  depth=0.25ex]}

% Set style of TikZ pictures to tikz-cd style.
\tikzset{every picture/.append style={commutative diagrams/every diagram}}
% \tikzset{math/.style = {commutative diagrams/every arrow,
%   commutative diagrams/every label,
%   execute at begin node=\(, execute at end node=\)}
% }
\tikzset{math/.style = {
  execute at begin node=\(, execute at end node=\)}
}

% Shortcuts to tikz-cd styles.
\tikzset{hook/.style = {commutative diagrams/hook}}
\tikzset{dashed/.style = {commutative diagrams/dashed}}
\tikzset{two heads/.style = {commutative diagrams/two heads}}
\tikzset{equal/.style = {commutative diagrams/equal, nfold}}
\tikzset{squiggly/.style = {commutative diagrams/squiggly}}
\tikzset{marking/.style = {commutative diagrams/marking}}
\tikzset{maps to/.style = {commutative diagrams/maps to}}
\tikzset{shift right/.style = {commutative diagrams/shift right}}
\tikzset{shift left/.style = {commutative diagrams/shift left}}

% From quiver:
% A TikZ style for curved arrows of a fixed height, due to AndréC.
\tikzset{curve/.style={settings={#1},to path={(\tikztostart)
    .. controls ($(\tikztostart)!\pv{pos}!(\tikztotarget)!\pv{height}!270:(\tikztotarget)$)
    and ($(\tikztostart)!1-\pv{pos}!(\tikztotarget)!\pv{height}!270:(\tikztotarget)$)
    .. (\tikztotarget)\tikztonodes}},
    settings/.code={\tikzset{quiver/.cd,#1}
        \def\pv##1{\pgfkeysvalueof{/tikz/quiver/##1}}},
    quiver/.cd,pos/.initial=0.35,height/.initial=0
}

% Needed for suspension style.
\usetikzlibrary{decorations.markings}

% Marks the arrows with a suspension style.
\tikzset{
  suspension/.style = {postaction = decorate,
      decoration = {
          markings,
          mark = at position 0.3 with {\draw[-] (0,-0.075) -- (0,0.075);}
      },
  },
}
    

% Ny type lista med ganske perfekt spacing
\newlist{plist}{enumerate}{5}
\setlist[plist]{align=left, itemindent = 0cm, labelsep = 0cm, labelindent = 0cm}
\setlist[plist,1]{label=\arabic*, font=\bf\Large}
\setlist[plist,2]{label*=.\arabic*, labelwidth=1.25cm, leftmargin=1.25cm}
\setlist[plist,3]{label*=.\arabic*, labelwidth=1.5cm, leftmargin=1.5cm}

% Teoremstil
\theoremstyle{plain}
\newtheorem{theorem}{Theorem}[subsection]
\newtheorem{proposition}[theorem]{Proposition}
\newtheorem{corollary}[theorem]{Corollary}
\newtheorem{lemma}[theorem]{Lemma}

% Definisjonstil
\theoremstyle{definition}
\newtheorem{definition}[theorem]{Definition}
\newtheorem{example}[theorem]{Example}
\newtheorem{remark}[theorem]{Remark}
\newtheorem{construction}[theorem]{Construction}
\newtheorem{notation}[theorem]{Notation}
\newtheorem{fact}[theorem]{Fact}
\newtheorem{question}[theorem]{Question}

% Fancy box engine?
% \newcommand{\createbox}[1]{
%     \tcolorboxenvironment{#1}{
%         breakable, % Makes boxes breakable.
%         enhanced jigsaw, % Required for making breaks fancy.
%         arc = 0mm, % Corners.
%         % sharp corners, % make the corners sharp
%         oversize, % Makes the box not "squish" the text, but rather extend the box into the margins.
%         colback = white, % Changes the background colour.
%         % parskip, % Behave better with parskip package (not sure what it does)
%         % beforeafter skip balanced = 0pt, % Change spacing before and after boxes.
%         parbox = false, % Make text formatting similar to the outside.
%         before upper=\vspace{-2\parskip}, % Fix for extra parskip that comes in the start of the box.
%     }
% }

% Simple box engine?
\newcommand{\createbox}[1]{
    \tcolorboxenvironment{#1}{
        breakable, % Makes boxes breakable.
        empty, % Empty skin.
        arc = 0mm, % Corners.
        % sharp corners, % make the corners sharp
        oversize, % Makes the box not "squish" the text, but rather extend the box into the margins.
        % colback = white, % Changes the background colour.
        % parskip, % Behave better with parskip package (not sure what it does)
        % beforeafter skip balanced = 0pt, % Change spacing before and after boxes.
        parbox = false, % Make text formatting similar to the outside.
        before upper=\vspace{-2\parskip}, % Fix for extra parskip that comes in the start of the box.
        borderline={1pt}{0pt}{black}
    }
}

\createbox{theorem}
\createbox{proposition}
\createbox{corollary}
\createbox{lemma}
\createbox{definition}
\createbox{example}
\createbox{remark}
\createbox{construction}
\createbox{notation}
\createbox{fact}
\createbox{question}

% TODO: Make proof connect to the theorem box. Maybe it will look better? Then it would be difficult to have text in between.
\createbox{proof}

% Blackboard shortcuts
\newcommand{\Fb}{{\mathbb{F}}}
\newcommand{\Nb}{{\mathbb{N}}}
\newcommand{\Qb}{{\mathbb{Q}}}
\newcommand{\Rb}{{\mathbb{R}}}
\newcommand{\Zb}{{\mathbb{Z}}}

% Caligraphy shortcuts
\newcommand{\Ac}{{\mathcal{A}}}
\newcommand{\Bc}{{\mathcal{B}}}
\newcommand{\Cc}{{\mathcal{C}}}
\newcommand{\Ic}{{\mathcal{I}}}
\newcommand{\Kc}{{\mathcal{K}}}
\newcommand{\Mc}{{\mathcal{M}}}
\newcommand{\Nc}{{\mathcal{N}}}
\newcommand{\Pc}{{\mathcal{P}}}
\newcommand{\Tc}{{\mathcal{T}}}

% Set management shortcuts
\newcommand{\intersect}{\mathop{\cap}\limits}
\newcommand{\union}{\mathop{\cup}\limits}
\newcommand{\directsum}{\mathop{\oplus}\limits}

% Shorthands
\newcommand{\abs}[1]{ \lvert #1 \rvert }
\newcommand{\set}[1]{ \left\{ #1 \right\} }
\newcommand{\tuple}[1]{ \left( #1 \right) }
\newcommand{\toda}[1]{ \langle #1 \rangle }

\newcommand{\Stmod}[1]{\stablemod\tuple{ #1 }}

% New math operators
\DeclareMathOperator{\Id}{Id}
\DeclareMathOperator{\StMod}{StMod}
\DeclareMathOperator{\stablemod}{Stmod}
\DeclareMathOperator{\Obj}{Obj}
\DeclareMathOperator{\Hom}{Hom}
\DeclareMathOperator{\Mod}{Mod}
% \DeclareMathOperator{\mod}{mod}
\DeclareMathOperator{\coker}{coker}
\DeclareMathOperator{\im}{im}
\DeclareMathOperator{\Ab}{Ab}
\DeclareMathOperator{\Fun}{Fun}
\DeclareMathOperator{\dg}{dg}
\DeclareMathOperator{\C}{C}
\DeclareMathOperator{\dgMod}{dgMod}


\DeclareMathOperator{\cone}{cone}

\title{Question regarding implicit change of codomain of morphisms in \( \dgMod_{\dg}(\Ac) \)}
\author{Håvard Skjetne Lilleheie}

\begin{document}

\maketitle

This is a question pertaining to the proof of in \cite[Proposition 4.2.8]{Jasso--Muro_2023_arXiv}, using the notation and definitions from the same article.

To simplify the question, I will be using \( d = 1 \).

Let the following be a diagram in \( H^0(\dgMod_{\dg}(\Ac)) \)
\begin{center}
    \begin{tikzpicture}
        \diagram{m}{1cm}{1cm} {
            M_3 \& M_2 \& M_1 \& M_0 \\
        };

        \draw[math]
            (m-1-1) edge node {f_3} (m-1-2)
            (m-1-2) edge node {f_2} (m-1-3)
            (m-1-3) edge node {f_1} (m-1-4);
    \end{tikzpicture}
\end{center}

Let the following be a defining system for the above diagram
\[
    \set{g_{ij}: M_j \to M_i \mid 0 \leq i < j \leq 3, j - i < 3}.
\]

In the very first part of the proof, one takes the defining system and construct the maps \( \alpha \) and \( \beta \), and here lies the issue I've encountered.

In the proof, one lets
\[
    \alpha :=
    \begin{pmatrix}
        g_{23} \\
        g_{13}
    \end{pmatrix}
\]
and declares that
\[
    \alpha: M_3 \to \cone(g_{12})[-1].
\]

However, by the definition of \( \cone(g_{12}) \) on page 48, one has that the underlying graded \( \Ac \)-module of \( \cone(g_{12})[-1] \) is \( M_2 \oplus M_3[-1] \), which does not fit with the codomain of \( g_{13} \) directly. Therefore I assume that there is some implicit way of turning the morphism
\[
    g_{13}: M_3 \to M_1
\]
into some morphism
\[
    \widetilde{g_{13}}: M_3 \to M_1[-1].
\]
In addition, this implicit change should change the degree from \( |g_{13}| = -1 \) into \( |\widetilde{g_{13}}| = 0 \) in order to later show that \( |\alpha| = 0 \).

My question is then: What is \( \widetilde{g_{13}} \)?

I have the following hypothesis that I can't make work:

Consider the following morphisms in \( \C_{\dg}(k) \): For any \( A \in \C_{\dg}(k) \) let \( \sigma_A \) be the following degree \( 1 \) morphism
\begin{center}
    \newcommand{\height}{2cm}
    \begin{tikzpicture}
        \diagramorigin{m}{\height}{1cm} {
            A \\
            A[-1] \\
        };

        \draw[math]
            (m-1-1) edge node {\sigma_A} (m-2-1);
    \end{tikzpicture}
    %
    \begin{tikzpicture}
        \diagramorigin{m}{\height}{2cm} {
            \cdots \& A_{-1} \& A_0 \& A_1 \& \cdots \\
            \cdots \& A_{-2} \& A_{-1} \& A_0 \& \cdots \\
        };

        \draw[math]
            (m-1-1) edge node {d_A^{-2}} (m-1-2)
                edge node {\Id} (m-2-2)
            (m-1-2) edge node {d_A^{-1}} (m-1-3)
                edge node {\Id} (m-2-3)
            (m-1-3) edge node {d_A^0} (m-1-4)
                edge node {\Id} (m-2-4)
            (m-1-4) edge node {d_A^1} (m-1-5)
                edge node {\Id} (m-2-5)

            (m-2-1) edge node {-d_A^{-3}} (m-2-2)
            (m-2-2) edge node {-d_A^{-2}} (m-2-3)
            (m-2-3) edge node {-d_A^{-1}} (m-2-4)
            (m-2-4) edge node {-d_A^0} (m-2-5);
    \end{tikzpicture}
\end{center}

Since \( g_{13} \in \dgMod_{\dg}(\Ac)(M_3, M_1) \) it is a \( \Zb \)-graded collection of natural transformations where in grade \( i \), the natural transformation maps consist of morphisms in \( \C_{\dg}(k) \) of degree \( i \).

Let \( (g_{13})^i \) be the notation for the \( i \)-th natural transformation of \( g_{13} \).

Furthermore, let \( (g_{13})^i_A \) be the notation for the \( i \)-th natural tranformation's map from \( M_3(A) \) to \( M_1(A) \).

Using the above notation, I postulate that \( \widetilde{g_{13}} \) is the homogeneous of degree \( 0 \)-morphism with the \( 0 \)-th component being the natural transformation where for any \( A \in \Ac \), one has \( (\widetilde{g_{13}})_A^0 := \sigma_{M_1(A)} \circ (g_{13})^{-1}_A \). (This should encapsulate the entire behaviour of \( g_{13} \), since \( g_{13} \) is by definition homogeneous of degree \( -1 \).)

In order for this to be true, one would need that for any \( f \in \Ac^{\op}(A, B) \), the outer rectangle of the following diagram should commute
\begin{center}
    \begin{tikzpicture}
        \diagram{m}{2cm}{2cm} {
            M_3(A) \& M_1(A) \& M_1[-1](A) \\
            M_3(B) \& M_1(B) \& M_1[-1](B) \\
        };

        \draw[math]
            (m-1-1) edge node {(g_{13})_A^{-1}} (m-1-2)
                edge node {M_3(f)} (m-2-1)
            (m-1-2) edge node {\sigma_{M_1(A)}} (m-1-3)
                edge node {M_1(f)} (m-2-2)
            (m-1-3) edge node {M_1[-1](f)} (m-2-3)

            (m-2-1) edge node {(g_{13})_B^{-1}} (m-2-2)
            (m-2-2) edge node {\sigma_{M_1(B)}} (m-2-3);
    \end{tikzpicture}.
\end{center}

Since the left square commutes by definition, a sufficient condition in order for the outer rectangle to commute would be that the rightmost square commutes.

However, I can't get the rightmost square to commute by the following argument:

Consider some \( A \in \Ac \) as well as a homogeneous of degree \( -1 \) endomorphism \( f \in \Ac(A, A) \). Then assuming that the rightmost square commutes, the following square should commute
\begin{center}
    \begin{tikzpicture}
        \diagram{m}{2cm}{2cm} {
            M_1(A) \& M_1[-1](A) \\
            M_1(A) \& M_1[-1](A) \\
        };

        \draw[math]
        (m-1-1) edge node {\sigma_{M_1(A)}} (m-1-2)
            edge node {M_1(f)} (m-2-1)
        (m-1-2) edge node {M_1[-1](f)} (m-2-2)

        (m-2-1) edge node {\sigma_{M_1(A)}} (m-2-2);
    \end{tikzpicture}
\end{center}
Let's look at what the different paths of the diagram look like:
\[
    M_1[-1](f) \circ \sigma_{M_1(A)} :
\]
\begin{center}
    \begin{tikzpicture}
        \diagramorigin{m}{2cm}{3cm} {
            \cdots \& M_1(A)_{-1} \& M_1(A)_0 \& M_1(A)_1 \& \cdots \\
            \cdots \& M_1(A)_{-2} \& M_1(A)_{-1} \& M_1(A)_0 \& \cdots \\
            \cdots \& M_1(A)_{-2} \& M_1(A)_{-1} \& M_1(A)_0 \& \cdots \\
        };

        \draw[math]
            (m-1-1) edge (m-1-2)
                edge node {\Id} (m-2-2)
            (m-1-2) edge (m-1-3)
                edge node {\Id} (m-2-3)
            (m-1-3) edge (m-1-4)
                edge node {\Id} (m-2-4)
            (m-1-4) edge (m-1-5)
                edge node {\Id} (m-2-5)

            (m-2-1) edge (m-2-2)
            (m-2-2) edge (m-2-3)
                edge node {-M_1(f)_{-2}} (m-3-1)
            (m-2-3) edge (m-2-4)
                edge node {-M_1(f)_{-1}} (m-3-2)
            (m-2-4) edge (m-2-5)
                edge node {-M_1(f)_0} (m-3-3)
            (m-2-5) edge node {-M_1(f)_1} (m-3-4)

            (m-3-1) edge (m-3-2)
            (m-3-2) edge (m-3-3)
            (m-3-3) edge (m-3-4)
            (m-3-4) edge (m-3-5);
    \end{tikzpicture}
\end{center}
\[
    \sigma_{M_1(A)} \circ M_1(f) :
\]
\begin{center}
    \begin{tikzpicture}
        \diagramorigin{m}{2cm}{3cm} {
            \cdots \& M_1(A)_{-1} \& M_1(A)_0 \& M_1(A)_1 \& \cdots \\
            \cdots \& M_1(A)_{-1} \& M_1(A)_0 \& M_1(A)_1 \& \cdots \\
            \cdots \& M_1(A)_{-2} \& M_1(A)_{-1} \& M_1(A)_0 \& \cdots \\
        };

        \draw[math]
            (m-1-1) edge (m-1-2)
            (m-1-2) edge (m-1-3)
                edge node {M_1(f)_{-1}} (m-2-1)
            (m-1-3) edge (m-1-4)
                edge node {M_1(f)_0} (m-2-2)
            (m-1-4) edge (m-1-5)
                edge node {M_1(f)_1} (m-2-3)
            (m-1-5) edge node {M_1(f)_2} (m-2-4)

            (m-2-1) edge (m-2-2)
                edge node {\Id} (m-3-2)
            (m-2-2) edge (m-2-3)
                edge node {\Id} (m-3-3)
            (m-2-3) edge (m-2-4)
                edge node {\Id} (m-3-4)
            (m-2-4) edge (m-2-5)
                edge node {\Id} (m-3-5)

            (m-3-1) edge (m-3-2)
            (m-3-2) edge (m-3-3)
            (m-3-3) edge (m-3-4)
            (m-3-4) edge (m-3-5);
    \end{tikzpicture}
\end{center}
one can see that there is a sign difference between the two maps, and the square therefore does not commute if there exist an endomorphism of degree \( -1 \) in \( \Ac \), which would be odd.

Another reason that it seems unlikely that the rightmost square should commute is because it could possibly imply that in \( \dgMod_{\dg}(\Ac) \) that the shift functor is naturally isomorphic to the identity functor (since \( \sigma_A \) is an isomorphism for any \( A \)), which if true, sounds like a result that should have been mentioned at least once somewhere.

\bibliography{Jasso--Muro_question}{}
\bibliographystyle{utcaps}

\end{document}
