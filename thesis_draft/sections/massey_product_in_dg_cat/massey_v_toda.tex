% First need to extend massey-prod definition to H^0

% MS-Question: I Jasso--Muro så er dette berre definert for element av dgMod (Ingen subscript!)
% TODO: This is also the same shift functor that makes H^0(dgmodblabla) triangulated. Probably not neccesary to specify then.
\begin{proposition}
    \label{prop:H^i_dgmod_cong_H^0_with_shift}
    Let \( \Cc \) be a DG-category over \( R \). Let \( \Sigma \) denote the shift functor on \( H^0(\dgMod_{\dg}(\Cc)) \). And let \( A, B \in \dgMod_{\dg}(\Cc) \).

    Then there is an isomorphism
    \[
        \phi: H^i(\dgMod_{\dg}(\Cc)(A, B)) \stackrel{\sim}{\to} H^0(\dgMod_{\dg}(\Cc))(A, \Sigma^i B).
    \]
\end{proposition}
\begin{proof}
    TODO
\end{proof}

% TODO: Show this is well defined!!
\begin{remark}
    \label{rem:H^0_into_H^*_inclusion}
    Let \( \Cc \) be a DG-category.

    There is a dense and faithful functor \( \iota: H^0(\Cc) \rightarrowtail H^*(\Cc) \) given by
    \begin{align*}
        \iota: H^0(\Cc) &\to H^*(\Cc) \\
        A &\mapsto A \\
        \iota_{A, B}: H^0(\Cc)(A, B) &\to H^*(\Cc)(A, B) \\
        f &\mapsto \tuple{\dots, 0, f, 0, \dots} \quad \text{\( f \) is in degree \( 0 \)}
    \end{align*}

    TODO: Show well defined.
\end{remark}

\begin{remark}
    \label{rem:massey_in_H^0(dgMod_dg)}
    Let \( \Cc \) be a DG-category. Let the following be a diagram in \( H^0(\dgMod_{\dg}(\Cc)) \)
    \begin{center}
        \begin{tikzpicture}
            \diagram{m}{1cm}{1cm} {
                X_1 \& X_2 \& X_3 \& X_4 \\
            };

            \draw[math]
                (m-1-1) edge node {f_1} (m-1-2)
                (m-1-2) edge node {f_2} (m-1-3)
                (m-1-3) edge node {f_3} (m-1-4);
        \end{tikzpicture}
    \end{center}
    Using the functor \( \iota \) from \autoref{rem:H^0_into_H^*_inclusion}, one can view the above diagram as a diagram in \( H^*(\dgMod_{\dg}(\Cc)) \) in the following manner
    \begin{center}
        \begin{tikzpicture}
            \diagram{m}{1cm}{1cm} {
                X_1 \& X_2 \& X_3 \& X_4 \\
            };

            \draw[math]
                (m-1-1) edge node {\iota(f_1)} (m-1-2)
                (m-1-2) edge node {\iota(f_2)} (m-1-3)
                (m-1-3) edge node {\iota(f_3)} (m-1-4);
        \end{tikzpicture}
    \end{center}
    where all the maps are homogeneous of degree \( 0 \).

    % MS-Question: Overly complicated and imprecise on domains and stuff.
    By TODO the massey product of these maps, \( \massey{\iota(f_3), \iota(f_2), \iota(f_1)} \), only have non-zero components in \( H^{-1}(\dgMod_{\dg}(\Cc)(X_1, X_4)) \). Then, using the ismorphism \( \phi \) from \autoref{prop:H^i_dgmod_cong_H^0_with_shift} one has that
    \[
        \phi(\massey{\iota(f_3), \iota(f_2), \iota(f_1)}) \subseteq H^0(\dgMod_{\dg})(\Cc)(X_1, \Sigma^{-1} X_4).
    \]
    Since
    \[
        H^0(\dgMod_{\dg})(\Cc)(X_1, \Sigma^{-1} X_4) \cong H^0(\dgMod_{\dg})(\Cc)(\Sigma X_1, X_4)
    \]
    one has the subset
    \[
        \Sigma(\phi(\massey{\iota(f_3), \iota(f_2), \iota(f_1)})) \subseteq H^0(\dgMod_{\dg})(\Cc)(\Sigma X_1, X_4).
    \]
    Denote \( \Sigma(\phi(\massey{\iota(f_3), \iota(f_2), \iota(f_1)})) \), suggestively, as \( \massey{f_3, f_2, f_1} \).
\end{remark}

\begin{definition}[Massey product on \( H^0(\dgMod_{\dg}(\Cc)) \)]
    \label{def:massey_product_H^0(dgMod_dg(C))}
    Let \( \Cc \) be a DG-category. Let the following be a diagram in \( H^0(\dgMod_{\dg}(\Cc)) \)
    \begin{center}
        \begin{tikzpicture}
            \diagram{m}{1cm}{1cm} {
                X_1 \& X_2 \& X_3 \& X_4 \\
            };

            \draw[math]
                (m-1-1) edge node {f_1} (m-1-2)
                (m-1-2) edge node {f_2} (m-1-3)
                (m-1-3) edge node {f_3} (m-1-4);
        \end{tikzpicture}
    \end{center}
    Let \( \massey{f_3, f_2, f_1} \subseteq H^0(\dgMod_{\dg})(\Cc)(X_1, \Sigma^{-1} X_4) \) be as in \autoref{rem:massey_in_H^0(dgMod_dg)}. This subset is called the \emph{massey product of \( f_3, f_2 \) and \( f_1 \) in \( H^0(\dgMod_{\dg}(\Cc)) \)}.
\end{definition}

% TODO: Specify that the functor is exact?
\begin{remark}
    \label{rem:massey_in_alg_tri_cat}
    Let \( \Tc \) be an algebraic triangulated category, and let the following be a diagram in \( \Tc \)
    \begin{center}
        \begin{tikzpicture}
            \diagram{m}{1cm}{1cm} {
                X_1 \& X_2 \& X_3 \& X_4 \\
            };

            \draw[math]
                (m-1-1) edge node {f_1} (m-1-2)
                (m-1-2) edge node {f_2} (m-1-3)
                (m-1-3) edge node {f_3} (m-1-4);
        \end{tikzpicture}.
    \end{center}

    Since \( \Tc \) is algebraic, it is equivalent to \( H^0(\Cc) \) for some pre-triangulated DG-category \( \Cc \). Furthermore, since \( \Cc \) is a pre-triangulated DG-category, one has that \( H^0(\Cc) \) is equivalent to \( \im(H^0(\mathbf{h})) \). And since \( \im(H^0(\mathbf{h})) \) is a full subcategory of \( H^0(\dgMod_{\dg}(\Cc)) \), the diagram above can be looked at as a diagram in \( H^0(\dgMod_{\dg}(\Cc)) \).

    To recap, one has the relation:
    \[
        \Tc \cong H^0(\Cc) \cong \im(H^0(\mathbf{h})) \stackrel{\text{full}}{\subseteq} \D^c(\Cc) \stackrel{\text{full}}{\subseteq} \D(\Cc) \stackrel{\text{full}}{\subseteq} H^0(\dgMod_{\dg}(\Cc))
    \]
    Let \( F \) denote the functor that takes \( F: \Tc \rightarrowtail H^0(\dgMod_{\dg}(\Cc)) \). Then one has that the above diagram in \( \Tc \) can be viewed as a diagram in \( H^0(\dgMod_{\dg}(\Cc)) \) as follows
    \begin{center}
        \begin{tikzpicture}
            \diagram{m}{1cm}{1cm} {
                F(X_1) \& F(X_2) \& F(X_3) \& F(X_4). \\
            };

            \draw[math]
                (m-1-1) edge node {F(f_1)} (m-1-2)
                (m-1-2) edge node {F(f_2)} (m-1-3)
                (m-1-3) edge node {F(f_3)} (m-1-4);
        \end{tikzpicture}
    \end{center}
    
    On the above diagram one can take the massey product (as in \autoref{def:massey_product_H^0(dgMod_dg(C))}). This yields a subset
    \[
        \massey{F(f_3), F(f_2), F(f_1)} \subseteq H^0(\dgMod_{\dg}(\Cc))(F(X_1), \Sigma^{-1} F(X_4)).
    \]
    And since \( \im(H^0(\mathbf{h})) \) is a full subcategory of \( H^0(\dgMod_{\dg}(\Cc)) \), one has that
    \[
        \massey{F(f_3), F(f_2), F(f_1)} \subseteq \im(H^0(\mathbf{h}))(F(X_1), \Sigma^{-1} F(X_4)).
    \]
    And since \( \Tc \cong \im(H^0(\mathbf{h})) \), there is a bijection from \( \massey{F(f_3), F(f_2), F(f_1)} \) to some \( M \subseteq \Tc(X_1, \Sigma^{-1} X_4) \).

    Then since \( \Tc(X_1, \Sigma^{-1} X_4) \cong \Tc(\Sigma X_1, X_4) \), one can look at the subset
    \[
        \Sigma(M) \subseteq \Tc(\Sigma X_1, X_4).
    \]
    This subset, \( \Sigma(M) \), is (suggestively) denoted as \( \massey{f_3, f_2, f_1} \).
\end{remark}

\begin{definition}[Massey product in an algebraic triangulated category]
    Let \( \Tc \) be an algebraic triangulated category, and let the following be a diagram in \( \Tc \)
    \begin{center}
        \begin{tikzpicture}
            \diagram{m}{1cm}{1cm} {
                X_1 \& X_2 \& X_3 \& X_4 \\
            };

            \draw[math]
                (m-1-1) edge node {f_1} (m-1-2)
                (m-1-2) edge node {f_2} (m-1-3)
                (m-1-3) edge node {f_3} (m-1-4);
        \end{tikzpicture}
    \end{center}
    Let \( \massey{f_3, f_2, f_1} \subseteq \Tc(\Sigma X_1, X_4) \) be as in \autoref{rem:massey_in_alg_tri_cat}.

    This subset is called the \emph{massey product of \( f_3, f_2 \) and \( f_1 \) in an algebraic triangulated category}.
\end{definition}

% \begin{lemma}
%     Let \( \Cc \) be a DG-category over \( R \). Let \( f \in \dgMod_{\dg}(\Cc)(F, G) \) be homogeneous of degree \( d \). Let  \( \Sigma \) denote the shift functor in \( \dgMod_{\dg}(\Cc) \).

%     Then the map
%     \begin{align*}
%         \tilde{f}: F &\to \Sigma^d(G) \\
%     \end{align*}

%     WIP
% \end{lemma}

\begin{question}[Possibility of shifting one end of a morphism in \( \dgMod_{\dg}(\Cc) \).]
    Let \( t \in \dgMod_{\dg}(\Cc)(X_1, X_3) \) be a homogeneous morphism of degree \( -1 \).

    Question: How do I create a morphism \( \tilde{t} \in \dgMod_{\dg}(\Cc)(X_1, \Sigma^{-1} X_3) \) such that \( \tilde{t} \) is of degree \( 0 \) (and in ``some'' sense closely resembles \( t \)).

    Since \( t \) is homogeneous of degree \( -1 \), we can capture its entire behaviour from the natural transformation \( t^{-1} \) (NB: the superscript represents the index, not that the map is an inverse). By definition, for any \( A, B \in \Cc^{\op} \) and any \( f \in \Cc^{\op}(A, B) \) the following diagram commutes
    \begin{center}
        \begin{tikzpicture}
            \diagram{m}{1cm}{1cm} {
                X_1(A) \& X_3(A) \\
                X_1(B) \& X_3(B) \\
            };

            \draw[math]
                (m-1-1) edge node {t_A^{-1}} (m-1-2)
                    edge node {X_1(f)} (m-2-1)
                (m-1-2) edge node {X_3(f)} (m-2-2)

                (m-2-1) edge node {t_B^{-1}} (m-2-2);
        \end{tikzpicture}.
    \end{center}

    Then, look at what \( t^{-1}_A \) does:
    \begin{center}
        \begin{tikzpicture}
            \diagramorigin{m}{2cm}{3cm} {
                \cdots \& \tuple{X_1(A)}_{-1} \& \tuple{X_1(A)}_0 \& \tuple{X_1(A)}_1 \& \cdots \\
                \cdots \& \tuple{X_3(A)}_{-1} \& \tuple{X_3(A)}_0 \& \tuple{X_3(A)}_1 \& \cdots \\
            };

            \draw[math]
                (m-1-1) edge node {d^{-2}} (m-1-2)
                (m-1-2) edge node {d^{-1}} (m-1-3)
                    edge node[swap] {\tuple{t^{-1}_A}_{-1}} (m-2-1)
                (m-1-3) edge node {d^0} (m-1-4)
                    edge node[swap] {\tuple{t^{-1}_A}_0} (m-2-2)
                (m-1-4) edge node {d^1} (m-1-5)
                    edge node[swap] {\tuple{t^{-1}_A}_1} (m-2-3)
                (m-1-5) edge node[swap] {\tuple{t^{-1}_A}_2} (m-2-4)

                (m-2-1) edge node {d^{-2}} (m-2-2)
                (m-2-2) edge node {d^{-1}} (m-2-3)
                (m-2-3) edge node {d^0} (m-2-4)
                (m-2-4) edge node {d^1} (m-2-5);
        \end{tikzpicture}.
    \end{center}
    % \[
    %     t^{-1}_A:
    %     \begin{pmatrix}
    %         \vdots \\
    %         a_1 \\
    %         a_0 \\
    %         a_{-1} \\
    %         \vdots
    %     \end{pmatrix}
    %     \mapsto
    %     \begin{pmatrix}
    %         \vdots \\
    %         (t^{-1}_A)_2(a_2) \\
    %         (t^{-1}_A)_1(a_1) \\
    %         (t^{-1}_A)_0(a_0) \\
    %         \vdots
    %     \end{pmatrix}.
    % \]
    A ``natural'' choice of \( \tilde{t}: X_1 \to \Sigma^{-1} X_3 \) would therefore be the homogeneous of degree \( 0 \) morphism with \( \tuple{\tilde{t}_A^{0}}_i := \tuple{t_A^{-1}}_i \):
    \begin{center}
        \begin{tikzpicture}
            \diagramorigin{m}{2cm}{3cm} {
                \cdots \& \tuple{X_1(A)}_{-1} \& \tuple{X_1(A)}_0 \& \tuple{X_1(A)}_1 \& \cdots \\
                \cdots \& \tuple{X_3(A)}_{-2} \& \tuple{X_3(A)}_{-1} \& \tuple{X_3(A)}_0 \& \cdots \\
            };

            \draw[math]
                (m-1-1) edge node {d^{-2}} (m-1-2)
                (m-1-2) edge node {d^{-1}} (m-1-3)
                    edge node[swap] {\tuple{t^{-1}_A}_{-1}} (m-2-2)
                (m-1-3) edge node {d^0} (m-1-4)
                    edge node[swap] {\tuple{t^{-1}_A}_0} (m-2-3)
                (m-1-4) edge node {d^1} (m-1-5)
                    edge node[swap] {\tuple{t^{-1}_A}_1} (m-2-4)

                (m-2-1) edge node {-d^{-3}} (m-2-2)
                (m-2-2) edge node {-d^{-2}} (m-2-3)
                (m-2-3) edge node {-d^{-1}} (m-2-4)
                (m-2-4) edge node {-d^0} (m-2-5);
        \end{tikzpicture}.
    \end{center}
    % \[
    %     \tilde{t}^0_A:
    %     \begin{pmatrix}
    %         \vdots \\
    %         a_1 \\
    %         a_0 \\
    %         a_{-1} \\
    %         \vdots
    %     \end{pmatrix}
    %     \mapsto
    %     \begin{pmatrix}
    %         \vdots \\
    %         (t^{-1}_A)_1(a_1) \\
    %         (t^{-1}_A)_0(a_0) \\
    %         (t^{-1}_A)_{-1}(a_{-1}) \\
    %         \vdots
    %     \end{pmatrix}.
    % \]
    Which implies that \( \tilde{t}_A^0 \) is exactly the same as
    \[
        -\Sigma^{-1}(\sigma_A^{-1}) \circ t_A^{-1}
    \]
    which would look like the negative of the composition
    \begin{center}
        \begin{tikzpicture}
            \diagramorigin{m}{2cm}{3cm} {
                \cdots \& \tuple{X_1(A)}_{-1} \& \tuple{X_1(A)}_0 \& \tuple{X_1(A)}_1 \& \cdots \\
                \cdots \& \tuple{X_3(A)}_{-1} \& \tuple{X_3(A)}_0 \& \tuple{X_3(A)}_1 \& \cdots \\
                \cdots \& \tuple{X_3(A)}_{-2} \& \tuple{X_3(A)}_{-1} \& \tuple{X_3(A)}_0 \& \cdots \\
            };

            \draw[math]
                (m-1-1) edge node {d^{-2}} (m-1-2)
                (m-1-2) edge node {d^{-1}} (m-1-3)
                    edge node[swap] {\tuple{t^{-1}_A}_{-1}} (m-2-1)
                (m-1-3) edge node {d^0} (m-1-4)
                    edge node[swap] {\tuple{t^{-1}_A}_0} (m-2-2)
                (m-1-4) edge node {d^1} (m-1-5)
                    edge node[swap] {\tuple{t^{-1}_A}_1} (m-2-3)
                (m-1-5) edge node[swap] {\tuple{t^{-1}_A}_2} (m-2-4)

                (m-2-1) edge node {d^{-2}} (m-2-2)
                    edge node {-\Id} (m-3-2)
                (m-2-2) edge node {d^{-1}} (m-2-3)
                    edge node {-\Id} (m-3-3)
                (m-2-3) edge node {d^0} (m-2-4)
                    edge node {-\Id} (m-3-4)
                (m-2-4) edge node {d^1} (m-2-5)
                    edge node {-\Id} (m-3-5)

                (m-3-1) edge node {-d^{-3}} (m-3-2)
                (m-3-2) edge node {-d^{-2}} (m-3-3)
                (m-3-3) edge node {-d^{-1}} (m-3-4)
                (m-3-4) edge node {-d^0} (m-3-5);
        \end{tikzpicture}.
    \end{center}
    
    In order to be a valid morphism (aka. a natural transformation), by composition rules as well as definition, the outer square of the following diagram has to commute for any \( A, B \in \Cc^{\op} \) and \( f \in \Cc^{\op}(A, B) \)
    \begin{center}
        \begin{tikzpicture}
            \diagram{m}{2cm}{2cm} {
                X_1(A) \& X_3(A) \& (\Sigma^{-1} X_3)(A) \\
                X_1(B) \& X_3(B) \& (\Sigma^{-1} X_3)(B) \\
            };

            \draw[math]
            (m-1-1) edge node {t_A^{-1}} (m-1-2)
                edge node {X_1(f)} (m-2-1)
            (m-1-2) edge node {\Sigma^{-1}(\sigma_A^{-1})} (m-1-3)
                edge node {X_3(f)} (m-2-2)
            (m-1-3) edge node {(\Sigma^{-1} X_3)(f)} (m-2-3)

            (m-2-1) edge node {t_B^{-1}} (m-2-2)
            (m-2-2) edge node {\Sigma^{-1}(\sigma_B^{-1})} (m-2-3);
        \end{tikzpicture}.
    \end{center}

    Since the leftmost square already commutes by definition, a sufficient result to show that the outer square commutes would be to show that the rightmost square also commutes.

    However, let \( f \in \Cc^{\op}(A, A) \) be a homogeneous map of degree \( -1 \). Then want to see if the following square could commute:
    \begin{center}
        \begin{tikzpicture}
            \diagram{m}{2cm}{2cm} {
                (\Sigma X_3)(A) \& X_3(A) \\
                (\Sigma X_3)(A) \& X_3(A) \\
            };

            \draw[math]
            (m-1-1) edge node {\sigma_A^{-1}} (m-1-2)
                edge node {(\Sigma X_3)(f)} (m-2-1)
            (m-1-2) edge node {X_3(f)} (m-2-2)

            (m-2-1) edge node {\sigma_A^{-1}} (m-2-2);
        \end{tikzpicture}
    \end{center}
    Let's look at what the different paths of the diagram look like:
    \[
        X_3(f) \circ \sigma_A^{-1} :
    \]
    \begin{center}
        \begin{tikzpicture}
            \diagramorigin{m}{2cm}{3cm} {
                \cdots \& X_3(A)_0 \& X_3(A)_1 \& X_3(A)_2 \& \cdots \\
                \cdots \& X_3(A)_{-1} \& X_3(A)_{0} \& X_3(A)_1 \& \cdots \\
                \cdots \& X_3(A)_{-1} \& X_3(A)_{0} \& X_3(A)_1 \& \cdots \\
            };

            \draw[math]
                (m-1-1) edge (m-1-2)
                    edge node {\Id} (m-2-2)
                (m-1-2) edge (m-1-3)
                    edge node {\Id} (m-2-3)
                (m-1-3) edge (m-1-4)
                    edge node {\Id} (m-2-4)
                (m-1-4) edge (m-1-5)
                    edge node {\Id} (m-2-5)

                (m-2-1) edge (m-2-2)
                (m-2-2) edge (m-2-3)
                    edge node {X_3(f)_{-1}} (m-3-1)
                (m-2-3) edge (m-2-4)
                    edge node {X_3(f)_0} (m-3-2)
                (m-2-4) edge (m-2-5)
                    edge node {X_3(f)_1} (m-3-3)
                (m-2-5) edge node {X_3(f)_2} (m-3-4)

                (m-3-1) edge (m-3-2)
                (m-3-2) edge (m-3-3)
                (m-3-3) edge (m-3-4)
                (m-3-4) edge (m-3-5);
        \end{tikzpicture}
    \end{center}
    \[
        \sigma_A^{-1} \circ (\Sigma X_3)(f) :
    \]
    \begin{center}
        \begin{tikzpicture}
            \diagramorigin{m}{2cm}{3cm} {
                \cdots \& X_3(A)_0 \& X_3(A)_1 \& X_3(A)_2 \& \cdots \\
                \cdots \& X_3(A)_0 \& X_3(A)_1 \& X_3(A)_2 \& \cdots \\
                \cdots \& X_3(A)_{-1} \& X_3(A)_{0} \& X_3(A)_1 \& \cdots \\
            };

            \draw[math]
                (m-1-1) edge (m-1-2)
                (m-1-2) edge (m-1-3)
                    edge node {-X_3(f)_0} (m-2-1)
                (m-1-3) edge (m-1-4)
                    edge node {-X_3(f)_1} (m-2-2)
                (m-1-4) edge (m-1-5)
                    edge node {-X_3(f)_2} (m-2-3)
                (m-1-5) edge node {-X_3(f)_3} (m-2-4)

                (m-2-1) edge (m-2-2)
                    edge node {\Id} (m-3-2)
                (m-2-2) edge (m-2-3)
                    edge node {\Id} (m-3-3)
                (m-2-3) edge (m-2-4)
                    edge node {\Id} (m-3-4)
                (m-2-4) edge (m-2-5)
                    edge node {\Id} (m-3-5)

                (m-3-1) edge (m-3-2)
                (m-3-2) edge (m-3-3)
                (m-3-3) edge (m-3-4)
                (m-3-4) edge (m-3-5);
        \end{tikzpicture}
    \end{center}

one can see that there is a sign difference between the two maps. This implies that \( \Sigma X_3 \) is not naturally isomorphic to \( X_3 \) if any endomorphism ring contains a map of degree \( -1 \).

My question is then: How do I construct this \( \tilde{t} \in \dgMod_{\Cc}(X_1, \Sigma^{-1} X_3) \)?
\end{question}

\begin{theorem}
    Let \( \Cc \) be a DG-category. Let the following be a diagram in \( H^0(\dgMod_{\dg}(\Cc)) \).
    \begin{center}
        \begin{tikzpicture}
            \diagram{m}{1cm}{1cm} {
                X_1 \& X_2 \& X_3 \& X_4 \\
            };

            \draw[math]
                (m-1-1) edge node {f_1} (m-1-2)
                (m-1-2) edge node {f_2} (m-1-3)
                (m-1-3) edge node {f_3} (m-1-4);
        \end{tikzpicture}
    \end{center}
    Then \( \toda{f_3, f_2, f_1} = (-1)^{TODO} \massey{f_3, f_2, f_1} \).
\end{theorem}
\begin{proof}
    TODO: This is a sketch of the main idea, fix it up.

    Want to prove this by showing the two inclusions \( \supseteq \) and \( \subseteq \).

    Firstly, want to show \( \supseteq \):

    Let \( f \in \massey{f_3, f_2, f_1} \) and let \( \bar{(-)} \) be as in TODO.
    
    Then by definition there exist representatives
    \[
        g_i \in \dgMod_{\dg}(X_i, X_{i + 1}), s \in \dgMod_{\dg}(X_2, X_4) \text{ and } t \in \dgMod_{\dg}(X_1, X_3)
    \]
    such that
    \[
        f = \phi\tuple{\class{\bar{s} \circ g_1 - \bar{g}_3 \circ t}}.
    \]

    Define
    \[
        \alpha := 
        \begin{pmatrix}
            
        \end{pmatrix}
    \]
    TODO: Write out why \( \alpha \) and \( \beta \) are homogeneous of degree 0, as well as why their codomain and domain is valid respectively.

    

    % TODO:
    % Create alpha and beta in a way that they have degree 0.
    % Show that alpha and beta are cocycles of degree 0
    % Show that alpha and beta fits into a fiber-cofiber TB definition.
\end{proof}

\begin{corollary}
    Let \( \Tc \) be an algebraic triangulated category. Furthermore let the following be a diagram in \( \Tc \).
    \begin{center}
        \begin{tikzpicture}
            \diagram{m}{1cm}{1cm} {
                X_1 \& X_2 \& X_3 \& X_4 \\
            };

            \draw[math]
                (m-1-1) edge node {f_1} (m-1-2)
                (m-1-2) edge node {f_2} (m-1-3)
                (m-1-3) edge node {f_3} (m-1-4);
        \end{tikzpicture}
    \end{center}
    % MS-question: Massey product ser stygt ut.
    Then \( \toda{f_3, f_2, f_1} = (-1)^{TODO} \massey{f_3, f_2, f_1} \).
\end{corollary}
