There are multiple definitions of an algebraic triangulated category. In this thesis, the definition of an algebraic triangulated category will be the existence of a DG-enhancement as defined in \cite[Definition 3.1.3]{Jasso-Muro_2023}. This section goes through this definition in detail.

% TODO Cite: Jasso--Muro
% TODO: Not define based on g_i,j's?
% TODO: Verify definition with borceux definition of composition p. 295
% TODO: Specify unit morphism? Why are they needed???
\begin{definition}[\( \C_{\dg}(\Mod(R)) \)]
    \label{def:c_dg_mod_r}
    Let \( R \) be a commutative ring with identity.

    Then let \emph{\( \C_{\dg}(\Mod(R)) \)} be a DG-category defined as follows
    \begin{enumerate}
        \item {
            \( \Obj(\C_{\dg}(\Mod(R))) := \Obj(\C(\Mod(R))) \).
        }
        \item {
            Let \( A, B \in \C_{\dg}(\Mod(R)) \). Let \( \class*{A, B} \) denote the internal hom of \( \C(\Mod(R)) \) (\autoref{def:internal_hom_of_chain_complexes_over_Mod(R)}) with respect to \( A, B \) as objects from \( \C(\Mod(R)) \).

            Then let \( \C_{\dg}(\Mod(R))(A, B) := \class*{A, B} \).
        }
        \item {
            For \( A, B, C \in \C_{\dg}(\Mod(R)) \), let
            \[
                c_{\C_{\dg}(\Mod(R))}: \C_{\dg}(\Mod(R))(B, C) \otimes \C_{\dg}(\Mod(R))(A, B) \to \C_{\dg}(\Mod(R))(A, C)
            \]
            be defined as the chain morphism where \( c_{\C_{\dg}(\Mod(R)), n} \) is uniquely defined on
            \[
                \tuple*{g_p}_{p \in \Zb} \in [B, C]_i \text{ and } \tuple*{f_q}_{q \in \Zb} \in [A, B]_j
            \]
            as follows
            \begin{align*}
                c_{\C_{\dg}(\Mod(R)), n}: \tuple*{ \C_{\dg}(\Mod(R))(B, C) \otimes \C_{\dg}(\Mod(R))(A, B) }_n &\to \C_{\dg}(\Mod(R))(A, C)_n \\
                \tuple*{g_p}_{p \in \Zb} \otimes \tuple*{f_q}_{q \in \Zb} &\mapsto \tuple*{g_{p + j} \circ f_p}_{p \in \Zb}.
            \end{align*}
        }
    \end{enumerate}
\end{definition}

\begin{remark}
    The composition definition in \autoref{def:c_dg_mod_r} is well defined by the following two arguments:

    \begin{enumerate}
        \item {
            Firstly, the morphisms \( c_{\C_{\dg}(\Mod(R)), n} \) are uniqely defined by \autoref{lem:map_out_of_tensor_unique} where the \( g_{i, j} \)'s are as follows
            \begin{align*}
                g_{i, j}: \prod_{p \in \Zb} \Mod(R)(B_p, C_{p + i}) \times \prod_{p \in \Zb} \Mod(R)(A_p, B_{p + j}) &\to \prod_{p \in \Zb} \Mod(R)(A_p, C_{p + i + j}) \\
                \tuple*{ g_p }_{p \in \Zb} \times \tuple*{ f_q }_{q \in \Zb} &\mapsto \tuple*{ g_{p + j} \circ f_p }_{p \in \Zb}.
            \end{align*}
            And these can be checked to be \( R \)-bilinear.
        }
        \item {
            Secondly, need to check that the \( c_{\C_{\dg}(\Mod(R)), n} \)'s form a chain morphism.

            Look at the following equation (with shortened notation for brevity) 
            \begin{align*}
                &( d_{(A, C), n} \circ c_n - c_{n + 1} \circ d_{(B, C) \otimes (A, B), n} )\tuple*{ \tuple*{ g_p }_{p \in \Zb} \otimes \tuple*{ f_q }_{q \in \Zb} } \\
                &= d_{(A, C), n} \circ c_n \tuple*{ \tuple*{ g_p }_{p \in \Zb} \otimes \tuple*{ f_q }_{q \in \Zb} } - c_{n + 1} \circ d_{(B, C) \otimes (A, B), n}\tuple*{ \tuple*{ g_p }_{p \in \Zb} \otimes \tuple*{ f_q }_{q \in \Zb} } \\
                \intertext{by definition of composition as well as differential of tensor product it follows that}
                &= d_{(A, C), n}\tuple*{
                    \tuple*{ g_{p + j} \circ f_p }_{p \in \Zb}
                } \\
                &\hspace{0.4cm} - c_{n + 1} \tuple*{
                    d_{(B, C), i}\tuple*{ \tuple*{ g_p }_{p \in \Zb} } \otimes \tuple*{ f_q }_{q \in \Zb}
                    + (-1)^i \tuple*{ g_p }_{p \in \Zb} \otimes d_{(A, B), j} \tuple*{ \tuple*{ f_q }_{q \in \Zb} }
                } \\
                \intertext{then by the definition of the differential of the internal hom}
                &= \tuple*{
                    d_{C, p + n} \circ g_{p + j} \circ f_p - (-1)^n g_{p + j + 1} \circ f_{p + 1} \circ d_{A, p}
                }_{p \in \Zb} \\
                &\hspace{0.4cm} - c_{n + 1} (
                    \tuple*{
                        d_{C, p + i} \circ g_p - (-1)^i g_{p + 1} \circ d_{B, p}
                    }_{p \in \Zb} \otimes \tuple*{ f_q }_{q \in \Zb} \\
                    &\hspace{0.8cm}+ (-1)^i \tuple*{ g_p }_{p \in \Zb} \otimes \tuple*{
                        d_{B, q + j} \circ f_q - (-1)^j f_{q + 1} \circ d_{A, q}
                    }_{q \in \Zb}
                ) \\
                \intertext{then by the fact that composition is an \( R \)-homomorphism}
                &= \tuple*{
                    d_{C, p + n} \circ g_{p + j} \circ f_p - (-1)^n \circ g_{p + j + 1} \circ f_{p + 1} \circ d_{A, p}
                }_{p \in \Zb} \\
                &\hspace{0.4cm} - c_{n + 1} \tuple*{
                    \tuple*{
                        d_{C, p + i} \circ g_p - (-1)^i g_{p + 1} \circ d_{B, p}
                    }_{p \in \Zb} \otimes \tuple*{ f_q }_{q \in \Zb}
                } \\
                &\hspace{0.4cm} - (-1)^i c_{n + 1} \tuple*{
                    \tuple*{ g_p }_{p \in \Zb} \otimes \tuple*{
                        d_{B, q + j} \circ f_q - (-1)^j f_{q + 1} \circ d_{A, q}
                    }_{q \in \Zb}
                } \\
                \intertext{then by the definition of composition}
                &= \tuple*{
                    d_{C, p + n} \circ g_{p + j} \circ f_p
                }_{p \in \Zb} - (-1)^n \tuple*{
                    g_{p + j + 1} \circ f_{p + 1} \circ d_{A, p}
                }_{p \in \Zb} \\
                &\hspace{0.4cm} - \tuple*{
                    d_{C, p + i + j} \circ g_{p + j} \circ f_p
                }_{p \in \Zb} + (-1)^i \tuple*{
                    g_{p + j + 1} \circ d_{B, p + j} \circ f_p
                }_{p \in \Zb} \\
                &\hspace{0.4cm} - (-1)^i \tuple*{
                    g_{p + j + 1} \circ d_{B, p + j} \circ f_p
                }_{p \in \Zb} + (-1)^{i + j} \tuple*{
                    g_{p + j + 1} \circ f_{p + 1} \circ d_{A, p}
                }_{p \in \Zb} \\
                &= 0.
            \end{align*}
            By \autoref{lem:map_out_of_tensor_unique}, this shows that the morphism
            \begin{multline*}
                d_{(A, C), n} \circ c_n - c_{n + 1} \circ d_{(B, C) \otimes (A, B), n}: \\
                \tuple*{ \C_{\dg}(\Mod(R))(B, C) \otimes \C_{\dg}(\Mod(R))(A, B) }_n \to \C_{\dg}(\Mod(R))(A, C)_{n + 1}
            \end{multline*}             
            corresponds to the \( g_{i,j} \)'s where \( g_{i, j} = 0 \). However, by uniqueness, this implies that
            \[
                d_{(A, C), n} \circ c_n - c_{n + 1} \circ d_{(B, C) \otimes (A, B), n} = 0.
            \]
        }
    \end{enumerate}
\end{remark}

% TODO: Unit morphism?
\begin{definition}[Opposite DG-category]
    \label{def:opposite_dg_category}
    Let \( \Cc \) be a DG-category.

    Then let \( \Cc^{op} \) be the DG-category defined as follows
    \begin{enumerate}
        \item {
            \( \Obj(\Cc^{op}) := \Obj(\Cc) \)
        }
        \item {
            For \( A, B \in \Cc^{op} \), let \( \Cc^{op}(A, B) := \Cc(B, A) \).
        }
        \item {
            For \( A, B, C \in \Cc^{op} \), let composition be the chain morphism
            \[
                c_{\Cc^{\op}}: \Cc^{\op} (B, C) \otimes \Cc^{\op} (A, B) \to \Cc^{\op} (A, C)
            \]
            where in degree \( n \) there is the following morphism that is uniquely defined by the following assignments for any \( i, j \in \Zb \) with \( i + j = n \) and any \( a \in \Cc^{\op} (B, C)_i, b \in \Cc^{\op} (A, B)_j \)
            \begin{align*}
                c_{\Cc^{\op}, n}: \tuple*{ \Cc^{\op} (B, C) \otimes \Cc^{\op} (A, B) }_n &\to \Cc^{\op} (A, C)_n \\
                a \otimes b &\mapsto (-1)^{ij} c_{\Cc, n} \tuple*{ b \otimes a }
            \end{align*}
        }
    \end{enumerate}
\end{definition}
\begin{remark}
    In order to show that the opposite DG-category defined in \autoref{def:opposite_dg_category} is a well defined DG-category, need to show the following
    \begin{enumerate}
        \item  {
            The composition morphism is uniquely defined by \autoref{lem:map_out_of_tensor_unique}.
        }
        \item {
            The composition morphism is a chain morphism.

            For any \( n \) want to show that the following diagram commutes
            \begin{center}
                \begin{tikzpicture}
                    \diagram{m}{1cm}{1cm} {
                        \tuple*{ \Cc^{\op} (B, C) \otimes \Cc^{\op} (A, B) }_n \& \Cc^{\op} (A, C)_n \\
                        \tuple*{ \Cc^{\op} (B, C) \otimes \Cc^{\op} (A, B) }_{n + 1} \& \Cc^{\op} (A, C)_{n + 1}. \\
                    };

                    \draw[math]
                        (m-1-1) edge node {c_{\Cc^{\op}, n}} (m-1-2)
                            edge node {d_n} (m-2-1)
                        (m-1-2) edge node {d_n} (m-2-2)

                        (m-2-1) edge node {c_{\Cc^{\op}, n + 1}} (m-2-2);
                \end{tikzpicture}
            \end{center}

            For \( i, j \in \Zb \) with \( i + j = n \), let \( a \in \Cc^{\op}(B, C)_i \) and \( b \in \Cc^{\op}(A, B)_j \).

            Consider the following equation
            \begin{align*}
                &d_n \circ c_{\Cc^{\op}, n} (a \otimes b) - c_{\Cc^{\op}, n + 1} \circ d_n (a \otimes b) \\
                &= (-1)^{ij} d_n \circ c_{\Cc, n}(b \otimes a) - c_{\Cc^{\op}, n + 1}\tuple{ d_{\Cc^{\op} (B, C), i}(a) \otimes b + (-1)^i a \otimes d_{\Cc^{\op} (A, B), j}(b) } \\
                &= (-1)^{ij} d_n \circ c_{\Cc, n}(b \otimes a) - (-1)^{(i + 1)j} c_{\Cc, n + 1}\tuple{ b \otimes d_{\Cc^{\op} (B, C), i}(a) } \\
                &\hspace{0.4cm} - (-1)^{i(j + 1)} c_{\Cc, n + 1}\tuple{ d_{\Cc^{\op} (A, B), j}(b) \otimes a } \\
                &= (-1)^{ij} c_{\Cc, n + 1} \circ d_n\tuple{ b \otimes a } - (-1)^{(i + 1)j} c_{\Cc, n + 1}\tuple{ b \otimes d_{\Cc^{\op} (B, C), i}(a) } \\
                &\hspace{0.4cm} - (-1)^{i(j + 1)} c_{\Cc, n + 1}\tuple{ d_{\Cc^{\op} (A, B), j}(b) \otimes a } \\
                &= (-1)^{ij} c_{\Cc, n + 1}\tuple{ d_{\Cc^{\op} (A, B), j}(b) \otimes a + (-1)^j b \otimes d_{\Cc^{\op} (B, C), j}(a) } \\
                &\hspace{0.4cm} - (-1)^{(i + 1)j} c_{\Cc, n + 1}\tuple{ b \otimes d_{\Cc^{\op} (B, C), i}(a) } - (-1)^{i(j + 1)} c_{\Cc, n + 1}\tuple{ d_{\Cc^{\op} (A, B), j}(b) \otimes a } \\
                &= (-1)^{ij} c_{\Cc, n + 1}\tuple{ d_{\Cc^{\op} (A, B), j}(b) \otimes a } + (-1)^{(i + 1)j} c_{\Cc, n + 1}\tuple{ b \otimes d_{\Cc^{\op} (B, C), j}(a) } \\
                &\hspace{0.4cm} - (-1)^{(i + 1)j} c_{\Cc, n + 1}\tuple{ b \otimes d_{\Cc^{\op} (B, C), i}(a) } - (-1)^{i(j + 1)} c_{\Cc, n + 1}\tuple{ d_{\Cc^{\op} (A, B), j}(b) \otimes a } \\
            \end{align*}
        }
    \end{enumerate}
\end{remark}

% TODO Cite: Jasso--Muro, Borceux
\begin{definition}[DG-functor]
    An enriched functor between two DG-categories is called a \emph{DG-functor}.
\end{definition}

\begin{definition}[\( \Sigma_{\C_{\dg}(\Mod(R))} \)]
    \label{def:sigma_c_dg}
    Let the DG-functor \( \Sigma_{\C_{\dg}(\Mod(R))} \) be defined as follows
    \begin{enumerate}
        \item {
            For \( A \in \C_{\dg}(\Mod(R)) \), let
            \[
                \Sigma_{\C_{\dg}(\Mod(R))}(A) := \Sigma_{\C(\Mod(R))}(A)
            \]
        }
        \item {
            For any \( A, B \in \C_{\dg}(\Mod(R)) \), let
            \begin{align*}
                \Sigma_{\C_{\dg}(\Mod(R))}: \C_{\dg}(\Mod(R))(A, B) \to \C_{\dg}(\Mod(R))(\Sigma_{\C_{\dg}(\Mod(R))}(A), \Sigma_{\C_{\dg}(\Mod(R))}(B))
            \end{align*}
            be the chain homomorphism where the \( n \)-th map is
            \begin{align*}
                \Sigma_{\C_{\dg}(\Mod(R)), n}: \C_{\dg}(\Mod(R))(A, B)_n &\to \C_{\dg}(\Mod(R))(\Sigma A, \Sigma B)_n \\
                \tuple*{ f_p }_{p \in \Zb} &\mapsto (-1)^n \tuple*{ f_{p + 1} }_{p \in \Zb}.
            \end{align*}
        }
    \end{enumerate}
\end{definition}
\begin{remark}
    In order to show that \( \Sigma_{\C_{\dg}(\Mod(R))} \) from \autoref{def:sigma_c_dg} is a DG-functor, need to show the following three properties:
    \begin{enumerate}
        \item {
            Firstly, need to show that
            \[
                \Sigma_{\C_{\dg}(\Mod(R))}: \C_{\dg}(\Mod(R))(A, B) \to \C_{\dg}(\Mod(R))(\Sigma_{\C_{\dg}(\Mod(R))}(A), \Sigma_{\C_{\dg}(\Mod(R))}(B))
            \]
            is a chain morphism by checking if it commutes with the differential.

            WIP
        }
        \item {
            Secondly, need to show the ``functoriality'' of \( \Sigma_{\C_{\dg}(\Mod(R))} \).

            WIP
        }
        \item {
            Thirdly, need to show that the unit morphism commutes with \( \Sigma_{\C_{\dg}(\Mod(R))} \).

            WIP
        }
    \end{enumerate}
\end{remark}

% SRC: Berest--Mehrle 2017 LN
% TODO: Must \Ac be small in order to be well defined??
\begin{definition}[\( \Fun_{\dg}(\Ac, \Bc) \)]
    \label{def:dg_functor_category}
    Let \( \Ac \) and \( \Bc \) be DG-categories over a commutative ring with identity, \( R \). In addition, let \( \Ac \) be small.

    Then let \( \Fun_{\dg}(\Ac, \Bc) \) be the following DG-category:
    \begin{enumerate}
        \item{
            Let \( \Obj(\Fun_{\dg}(\Ac, \Bc)) \) be the class of every DG-functor from \( \Ac \) to \( \Bc \).
        }
        \item{
            For \( F, G \in \Fun_{\dg}(\Ac, \Bc) \), let \( \Fun_{\dg}(\Ac, \Bc)(F, G) \) be defined as in \cite[Proposition 6.3.1]{Borceux_1994}.
        }
        \item {
            Let composition be defined as WIP.
        }
    \end{enumerate}
\end{definition}

\begin{remark}
    The term \emph{DG-natural transformation} is often used for ``elements'' in the morphism spaces defined in \autoref{def:dg_functor_category}, but the term is also used for elements in
    \[
        Z^0\tuple{ \Fun_{\dg}(\Ac, \Bc)(F, G) }.
    \]
    As an example, in \cite[Definition 6.2.4, Definition 6.3.1]{Borceux_1994} the same term is used in both contexts.
\end{remark}

% TODO: Cite: Jasso--Muro
% No mention of the ring in the notation? -> Implied since DG-category is over a ring!
\begin{definition}[\( \dgMod_{\dg}(\Cc) \)]
    Let \( \Cc \) be a DG-category over \( R \).

    % TODO: Why "Right"?
    Then define the \emph{DG-category of (right) DG \( \Cc \)-modules} as
    \[
        \dgMod_{\dg}(\Cc) := \Fun_{\dg}(\Cc^{op}, \C_{\dg}(\Mod(R))).
    \]
    Objects in \( \dgMod_{\dg}(\Cc) \) are called \emph{DG-modules over \( \Cc \)}.
\end{definition}

% TODO: Does symmetry isomorphism commute with morphisms? I would guess so....
% TODO-Q: Need to show that any element in the n-th component of tensor product of two chain complexes are a ``sum'' of elementary tensors.
\begin{remark}[Functor category structure from Borceux]
    The goal of this remark is to understand what the morphism structure of \( \dgMod_{\dg}(\Cc) \) is, and what properties it has from the enriched category theory perspective based on \cite{Borceux_1994}.

    In this remark, let \( \C_{\dg} \) be shorthand for \( \C_{\dg}(\Mod(R)) \).
    
    Let \( F, G \in \dgMod_{\dg}(\Cc) \) and let
    \[
        \tuple{ \eta_{i, A} }_{A \in \Cc^{\op}} \in \tuple*{ \prod_{A \in \Cc^{\op}} \C_{\dg}(\Mod(R))(F(A), G(A)) }_i
    \]
    and let
    \[
        f_j \in \Cc^{\op}(A', A'')_j.
    \]
    Construct the following composition of morphisms, where the right hand side is the element-wise action on the \( n \)-th component on an arbitrary elementary tensor with \( i +j = n \) (this is enough information to deduce it's action on any element by \autoref{lem:map_out_of_tensor_unique})
    \begin{equation}
        \label{eq:functor_category_borceux}
        \newcommand{\height}{1cm}
        %
        \mmznext{meaning to context=\height}
        \begin{tikzpicture}
            \diagram{m}{\height}{1cm} {
                \tuple*{ \prod_{A \in \Cc^{\op}} \C_{\dg}(\Mod(R))(F(A), G(A)) } \otimes \Cc^{\op}(A', A'') \\
                \C_{\dg}(F(A'), G(A')) \otimes \C_{\dg}(G(A'), G(A'')) \\
                \C_{\dg}(G(A'), G(A'')) \otimes \C_{\dg}(F(A'), G(A')) \\
                \C_{\dg}(F(A'), G(A'')) \\
            };

            \draw[math]
                (m-1-1) edge node {\pi_{A'} \otimes G_{A', A''}} (m-2-1)

                (m-2-1) edge node {s} (m-3-1)

                (m-3-1) edge node {c_{\C_{\dg}(\Mod(R))}} (m-4-1);
        \end{tikzpicture}
        %
        \mmznext{meaning to context=\height}
        \begin{tikzpicture}
            \diagram{m}{\height}{1cm} {
                \tuple{ \eta_{i, A} }_{A \in \Cc^{\op}} \otimes f_j  \\
                \eta_{i, A'} \otimes G(f_j) \\
                (-1)^{ij} G(f_j) \otimes \eta_{i, A'} \\
                (-1)^{ij} G(f_j) \circ \eta_{i, A'} \\
            };

            \path[math]
                ([yshift=-2.5mm]m-1-1.south) edge[maps to] (m-2-1)

                (m-2-1) edge[maps to] (m-3-1)

                (m-3-1) edge[maps to] (m-4-1);
        \end{tikzpicture}
    \end{equation}
    NOTE: The composition, \( \circ \), in bottom right of the above diagram is the composition as defined for DG-diagrams.

    Name the entire above composition of chain complex morphisms for \( \psi_{A', A''} \).

    % For any \( i, j \in \Zb \) let \( \iota_{i,j} \) denote the inclusion
    % \begin{multline*}
    %     \iota_{i, j}: \tuple*{ \prod_{A \in \Cc^{\op}} \C_{\dg}(\Mod(R))(F(A), G(A)) }_i \otimes \Cc^{\op}(A', A'')_j \\
    %     \hookrightarrow \tuple*{ \tuple*{ \prod_{A \in \Cc^{\op}} \C_{\dg}(\Mod(R))(F(A), G(A)) } \otimes \Cc^{\op}(A', A'') }_{i + j}
    % \end{multline*}

    Take the adjoint of \autoref{eq:functor_category_borceux} morphism gives the chain complex morphism \( \phi_{A', A''} \), where
    \begin{center}
        \newcommand{\height}{1cm}
        %
        \mmznext{meaning to context=\height}
        \begin{tikzpicture}
            \diagram{m}{\height}{1cm} {
                \prod\limits_{A \in \Cc^{\op}} \C_{\dg}(\Mod(R))(F(A), G(A)) \\
                \left[ \Cc^{\op}(A', A''), \C_{\dg}(F(A'), G(A'')) \right] \\
            };

            \draw[math]
                (m-1-1) edge node {\phi_{A', A''}} (m-2-1);
        \end{tikzpicture}
        %
        \mmznext{meaning to context=\height}
        \begin{tikzpicture}
            \diagram{m}{\height}{1cm} {
                \tuple{ \eta_{i, A} }_{A \in \Cc^{\op}} \\
                \tuple{ \psi_{A', A'', k} \tuple{ \tuple{ \eta_{i, A} }_{A \in \Cc^{\op}} \otimes ? } }_{k \in \Zb}. \\
            };

            \draw[math]
                (m-1-1) edge[maps to] (m-2-1);
        \end{tikzpicture}
    \end{center}
    NOTE: The notation causes some ambiguity here and so it is neccesary to specify that in the bottom right of the above diagram, \( ? \in \Cc^{\op}(A', A'')_{k - i} \).

    Finally take the product of this map over all \( A', A'' \in \Cc^{\op} \) to get the morphism
    \begin{center}
        \newcommand{\height}{1cm}
        %
        \mmznext{meaning to context=\height}
        \begin{tikzpicture}
            \diagram{m}{\height}{1cm} {
                \prod\limits_{A \in \Cc^{\op}} \C_{\dg}(\Mod(R))(F(A), G(A)) \\
                \prod\limits_{A', A'' \in \Cc^{\op}} \left[ \Cc^{\op}(A', A''), \C_{\dg}(F(A'), G(A'')) \right] \\
            };

            \draw[math]
                (m-1-1) edge node {\prod\limits_{A', A'' \in \Cc^{\op}} \phi_{A', A''}} (m-2-1);
        \end{tikzpicture}
        %
        \mmznext{meaning to context=\height}
        \begin{tikzpicture}
            \diagram{m}{\height}{1cm} {
                \tuple{ \eta_{i, A} }_{A \in \Cc^{\op}} \\
                \tuple{ \tuple{ \psi_{A', A'', k} \tuple{ \tuple{ \eta_{i, A} }_{A \in \Cc^{\op}} \otimes ? } }_{k \in \Zb} }_{A', A'' \in \Cc^{\op}}. \\
            };

            \draw[math]
                (m-1-1) edge[maps to] (m-2-1);
        \end{tikzpicture}
    \end{center}
    Doing a similar construction as in \autoref{eq:functor_category_borceux} look at the composition
    \begin{center}
        \newcommand{\height}{1cm}
        %
        \mmznext{meaning to context=\height}
        \begin{tikzpicture}
            \diagram{m}{\height}{1cm} {
                \tuple*{ \prod\limits_{A \in \Cc^{\op}} \C_{\dg}(\Mod(R))(F(A), G(A)) } \otimes \Cc^{\op}(A', A'') \\
                \C_{\dg}(\Mod(R))(F(A''), G(A'')) \otimes \C_{\dg}(F(A'), F(A'')) \\
                \C_{\dg}(F(A'), G(A'')) \\
            };

            \draw[math]
                (m-1-1) edge node {\pi_{A''} \otimes F_{A', A''}} (m-2-1)

                (m-2-1) edge node {\circ} (m-3-1);
        \end{tikzpicture}
        %
        \mmznext{meaning to context=\height}
        \begin{tikzpicture}
            \diagram{m}{\height}{1cm} {
                \tuple{ \eta_{i, A} }_{A \in \Cc^{\op}} \otimes f_j  \\
                \eta_{i, A''} \otimes F(f_j) \\
                \eta_{i, A''} \circ F(f_j) \\
            };

            \draw[math]
                (m-1-1) edge[maps to] (m-2-1)

                (m-2-1) edge[maps to] (m-3-1);
        \end{tikzpicture}
    \end{center}
    Using this as \( \widetilde{\psi}_{A', A''} \) continue the construction as before by taking the adjoint and the product to end up with the map \( \widetilde{\phi}_{A', A''} \).

    Then \cite[Proposition 6.3.1]{Borceux_1994} states that \( \dgMod_{\dg}(\Cc)(F, G) \) is the equalizer of the following diagram
    \begin{center}
        \begin{tikzpicture}
            \diagram{m}{1cm}{1cm} {
                \prod\limits_{A \in \Cc^{\op}} \C_{\dg}(\Mod(R))(F(A), G(A)) \\
                \prod\limits_{A', A'' \in \Cc^{\op}} \left[ \Cc^{\op}(A', A''), \C_{\dg}(F(A'), G(A'')) \right] \\
            };

            \path[math] ([xshift=2.5mm]m-1-1.south) edge node {\prod\limits_{A', A'' \in \Cc^{\op}} \phi_{A', A''}} ([xshift=2.5mm]m-2-1.north);
            \draw[math] ($(m-1-1.south) + (-0.25, 0)$) to node[swap] {\prod\limits_{A', A'' \in \Cc^{\op}} \widetilde{\phi}_{A', A''}} ($(m-2-1.north) + (-0.25, 0)$);
        \end{tikzpicture}
    \end{center}
    Then the question remains, what properties would this equalizer have?
    
    Consider a chain subcomplex
    \[
        H \stackrel{\iota}{\hookrightarrow} \prod_{A \in \Cc^{\op}} \C_{\dg}(\Mod(R))(F(A), G(A))
    \]
    where the following holds
    \[
        \prod\limits_{A', A'' \in \Cc^{\op}} \phi_{A', A''} \circ \iota = \prod_{A', A'' \in \Cc^{\op}} \widetilde{\phi}_{A', A''} \circ \iota.
    \]
    Then \( H \) would for any \( i \in \Zb \) contain the elements
    \[
        \tuple{ \eta_{i, A} }_{A \in \Cc^{\op}} \in \tuple*{ \prod_{A \in \Cc^{\op}} \C_{\dg}(\Mod(R))(F(A), G(A)) }_i
    \]
    such that
    \begin{align*}
        \tuple*{ \prod\limits_{A', A'' \in \Cc^{\op}} \phi_{A', A''} }_j \tuple{ \tuple{ \eta_{i, A} }_{A \in \Cc^{\op}} } &= \tuple*{ \prod_{A', A'' \in \Cc^{\op}} \widetilde{\phi}_{A', A''} }_j \tuple{ \tuple{ \eta_{i, A} }_{A \in \Cc^{\op}} } \\
        &\Updownarrow \\
        \tuple{ \tuple{ \psi_{A', A'', k}\tuple{ \tuple{ \eta_{i, A} }_{A \in \Cc^{\op}} \otimes ? } }_{k \in \Zb} }_{A', A'' \in \Cc^{\op}} &= \tuple{ \tuple{ \widetilde{\psi}_{A', A'', k}\tuple{ \tuple{ \eta_{i, A} }_{A \in \Cc^{\op}} \otimes ? } }_{k \in \Zb} }_{A', A'' \in \Cc^{\op}}
        \intertext{Which are equal if for any \( A', A'' \in \Cc^{\op} \) and any \( k \in \Zb \) and any \( f \in \Cc^{\op}\tuple*{A', A''}_{k - i} \) one has the following}
        \psi_{A', A'', k}\tuple{ \tuple{ \eta_{i, A} }_{A \in \Cc^{\op}} \otimes f } &= \widetilde{\psi}_{A', A'', k} \tuple{ \tuple{ \eta_{i, A} }_{A \in \Cc^{\op}} \otimes f } \\
        &\Updownarrow \\
        (-1)^{(k - i)*i}G(f) \circ \eta_{i, A'} &= \eta_{i, A''} \circ F(f) \\
        &\Updownarrow \\
        (-1)^{|f||\tuple{ \eta_{i, A} }_{A \in \Cc^{\op}}|}G(f) \circ \eta_{i, A'} &= \eta_{i, A''} \circ F(f). \\
    \end{align*}

    In plain english, this means that the morphism space \( \dgMod_{\dg}(\Cc)(F, G) \) has the following properties:
    \begin{enumerate}
        \item {
            The structure of \( \dgMod_{\dg}(\Cc)(F, G) \) is as a sub chain complex of
            \[
                \prod_{A \in \Cc^{\op}} \C_{\dg}(\Mod(R))(F(A), G(A)).
            \]
            Which is a chain complex where in the \( i \)-th component, one has
            \[
                \tuple*{ \prod_{A \in \Cc^{\op}} \C_{\dg}(\Mod(R))(F(A), G(A)) }_i = \prod_{A \in \Cc^{\op}} \C_{\dg}(\Mod(R))(F(A), G(A))_i
            \]
            where every element is denoted as \( \tuple{ \eta_{i, A} }_{A \in \Cc^{\op}} \).

            Let \( \widetilde{d_A} \) be the differential of \( \C_{\dg}(\Mod(R))(F(A), G(A)) \), then the differential of
            \[
                \prod_{A \in \Cc^{\op}} \C_{\dg}(\Mod(R))(F(A), G(A))
            \]
            is
            \begin{align*}
                \prod_{A \in \Cc^{\op}} \widetilde{d_A} : \prod_{A \in \Cc^{\op}} \C_{\dg}(\Mod(R))(F(A), G(A))_j &\to \prod_{A \in \Cc^{\op}} \C_{\dg}(\Mod(R))(F(A), G(A))_{j + 1} \\
                \tuple{ \eta_{i, A} }_{A \in \Cc^{\op}} &\mapsto \tuple{ \widetilde{d_A}\tuple{ \eta_{i, A} } }_{A \in \Cc^{\op}} \\
                &= \tuple*{ d_{G(A)} \circ \eta_{j, A} - (-1)^j \eta_{j, A} \circ d_{F(A)} }_{A \in \Cc^{\op}}.
            \end{align*}
        }
        \item {
            In addition, \( \dgMod_{\dg}(\Cc)(F, G) \) has the property that for any
            \[
                \tuple{ \eta_{i, A} }_{A \in \Cc^{\op}} \in \dgMod_{\dg}(\Cc)(F, G)_i,
            \]
            and any \( f \in \Cc^{\op}(A', A'')_j \), one has that
            \[
                G_{A', A'', j}(f) \circ \eta_{i, A'} = (-1)^{ij} \eta_{i, A''} \circ F_{A', A'', j}(f).
            \]
        }
    \end{enumerate}

    However, it still remains to show that \( H \) is the equalizer. WIP
\end{remark}

% TODO: Why is \Cc(-, A) a functor into \C_{\dg}(\Mod(R))?
\begin{definition}[DG Yoneda embedding]
    \label{def:DG_Yoneda_embedding}
    Let \( \Cc \) be a DG-category over \( R \).
    
    Then let \( \mathbf{h} \) be the DG-functor defined as follows
    \begin{align*}
        \mathbf{h}: \Cc &\to \dgMod_{\dg}(\Cc) \\
        A &\mapsto \Cc(-, A)
    \end{align*}

    This functor is called the \emph{DG Yoneda embedding of \( \Cc \)}.
\end{definition}

% TODO: Add Yoneda embedding identifies \Cc with a full subcategory of \dgMod_dg(\Cc)?

% TODO: Could define this for Mod(R)-enriched categories?
\begin{definition}[0th cohomology category of a DG category]
    Let \( \Cc \) be a DG category over \( R \).

    Then let \( H^0(\Cc) \) be the following (enriched over \( \Mod(R) \) TODO) category defined as follows
    \begin{enumerate}
        \item {
            Let \( \Obj(H^0(\Cc)) := \Obj(\Cc) \).
        }
        \item {
            Let \( A, B \in H^0(\Cc) \).

            Then let \( H^0(\Cc)(A, B) := H^0(\Cc(A, B)) \).
        }
        \item {
            Let \( A, B, C \in H^0(\Cc) \) with \( f_1 \in H^0(A, B) \) and \( f_2 \in H^0(B, C) \).

            Then by TODO, there exists \( g_1 \in \Cc(A, B) \), and \( g_2 \in \Cc(B, C) \) such that \( \class*{g_1} = f_1 \) and \( \class*{g_2} = f_2 \).

            % TODO: Slight abuse of notation taking a "class" of an element of a chain complex.
            % TODO: Need to show that this is well defined?
            Then let composition be defined on elementary tensors as follows
            \begin{align*}
                \circ_{H^0(\Cc)}: H^0(\Cc)(B, C) \otimes H^0(\Cc)(A, B) &\to H^0(\Cc)(A, C) \\
                f_2 \otimes f_1 &\mapsto \class*{ \circ_{\Cc}(g_2 \otimes g_1) }
            \end{align*}
        }
    \end{enumerate}
\end{definition}

% TODO: What are the triangles? Is the shift functor correct on maps?
% TODO: Incorrect/abuse of notation, how does the shift work on maps?
\begin{theorem}
    Let \( \Cc \) be a DG-category over \( R \). Let \( \Sigma_{\C_{\dg}(\Mod(R))} \) be the shift functor on \( \C_{\dg}(\Mod(R)) \).

    Then \( H^0(\dgMod_{\dg}(\Cc)) \) is a triangulated category with the shift functor \( \Sigma(-) = \Sigma_{\C_{\dg}(\Mod(R))} \circ - \).
\end{theorem}
\begin{proof}
    TODO
\end{proof}

% MS-Question: Have seen definition of small category that is that the class of iso classes are small, not the class of objects. What is correct? Are they equivalent? -> Essentially small.

\begin{definition}[Acyclic DG-module]
    Let \( \Cc \) be a DG-category over a commutative ring (with identity) \( R \). Furthermore, let \( A \in \dgMod_{\dg}(\Cc) \) be a DG-module over \( \Cc \).

    Then \( A \) is called \emph{acyclic} if for any \( X \in \Cc \), one has that \( A(X) \in \C_{\dg}(\Mod(R)) \) is acyclic, i.e. \( H^*(A(X)) = 0 \).
\end{definition}

% TODO: Is \dgMod_{\dg}(\Cc) abelian?/Have kernels?
\begin{definition}[DG-projective module]
    Let \( P \in \dgMod_{\dg}(\Cc) \) be a DG-module.

    Then \( P \) is called a \emph{DG-projective module over \( \Cc \)} if:
    
    For any DG-module \( A \in \dgMod_{\dg}(\Cc) \) and any epimorphism \( f \in \dgMod_{dg}(\Cc)(A, P) \) where \( \ker(f) \in \dgMod_{\dg}(\Cc) \) is acyclic. Then \( f \) is split.
\end{definition}

% TODO: Various definitions and idiosyncracies. Which is correct?
    % Acyclic kernel? Projective objects? Spanned?
    % Krause 07 -> Compact objects, maybe more.
    % Keller 94 -> Another definition of derived DG category.
% TODO: Following def from Krause 07, but not explicitly written down. Is it correct?
\begin{definition}[Derived DG-category]
    Let \( \Cc \) be a DG-category.

    Then the \emph{derived DG-category} of \( \Cc \), denoted \( \D(\Cc) \), is defined as the full subcategory of \( H^0(\dgMod_{\dg}(\Cc)) \) spanned by the objects of \( \dgMod_{\dg}(\Cc) \) that are DG-projective.
\end{definition}

% MS-Question: What is coproduct for the derived category?
% Cite: Jasso--Muro p.31, only a statement, no proof
\begin{proposition}
    \( \D(\Cc) \) is closed under arbitrary coproduct.
\end{proposition}
\begin{proof}
    TODO
\end{proof}

% MS-Quastion: Why are small categories sometimes mentioned in def of derived category? Something to do with localization being well defined?

% MS-Question: Arbitrary coproducts <=> infinite coproducts? -> Need triangulated property for this to make sense.
% Cite: Krause 07 p. 29
\begin{definition}[Compact objects of a category]
    Let \( \Cc \) be a triangulated category with arbitrary coproduct. Let \( X \in \Cc \). 
    
    Then if \( X \) has the following property:
    
    For any index set \( I \) and any morphism \( f: X \to \coprod_{i \in I} Y_i \), there is a finite index set \( J \subseteq I \) such that \( f \) factors through \( \coprod_{j \in J} Y_j \).
    
    Then define \( X \) as a \emph{compact object in \( \Cc \)}.
\end{definition}

% TODO: Is the derived category triangulated? Need to be in order for previous def to apply.
\begin{definition}[Perfect derived DG-category \( \D^c(\Cc) \)]
    Let \( \D(\Cc) \) be the derived DG-category of \( \Cc \).

    Then define \( \D^c(\Cc) \) to be the full subcategory of \( \D(\Cc) \) consisting of all compact objects in \( \D(\Cc) \). This is called the \emph{perfect derived DG-category of \( \Cc \)}.
\end{definition}

% TODO: Cite: Jasso--Muro says so
\begin{proposition}
    \( \D^c(\Cc) \) is triangulated.
\end{proposition}
\begin{proof}
    TODO
\end{proof}

% TODO: Is this even a functor? Or well defined? Probably need to show well defined and that every morphism in H^0(A) has a representative in A.
\begin{definition}[\( H^0 \)-induced functor]
    \label{def:H^0-induced_functor}
    Let \( \Ac \) and \( \Bc \) be two DG-categories, and let \( F: \Ac \to \Bc \) be a functor between them.

    Then define the functor \( H^0(F) \) as follows:
    \begin{align*}
        H^0(F): H^0(\Ac) &\to H^0(\Bc) \\
        A &\mapsto F(A) \\
        (H^0(f): A \to B) &\mapsto (H^0(F(f)): F(A) \to F(B)) 
    \end{align*}

    This is called the \( H^0 \)-induced functor of \( F \).
\end{definition}

\begin{theorem}
    \autoref{def:H^0-induced_functor} is a well-defined functor.
\end{theorem}
\begin{proof}
    TODO
\end{proof}

% Want to show that H^0(h) has codomain D^c(\Cc)
\begin{remark}
    Let \( \mathbf{h}: \Cc \to \dgMod_{\dg}(\Cc) \) be the DG-Yoneda embedding from \autoref{def:DG_Yoneda_embedding}.

    Then for any \( A \in \Cc \), one has that \( H^0(\mathbf{h})(A) \) is both DG-projective and compact.
    
    TODO: SHOW!!!

    Therefore one has that the functor \( H^0(\mathbf{h}): H^0(\Cc) \to H^0(\dgMod_{\dg}(\Cc)) \) factors through \( \D^c(\Cc) \). Denote this functor with the same notation:
    \[
        H^0(\mathbf{h}): H^0(\Cc) \to \D^c(\Cc)
    \]
\end{remark}

\begin{remark}
    \( H^0(\mathbf{h}): H^0(\Cc) \to \D^c(\Cc) \) is fully faithful.

    TODO: Prove
\end{remark}

% TODO: Why need small? Probably something with derived.
% TODO: Heilt ordrett nesten frå Jasso--Muro 2023 p. 32, burde kanskje omformulera?
\begin{definition}[pre-triangulated DG-category]
    Let \( \Cc \) be a small DG-category.

    Then \( \Cc \) is called a \emph{pre-triangulated DG-category} if the image of the (fully faithful) functor \( H^0(\mathbf{h}): H^0(\Cc) \to \D^c(\Cc) \) is a triangulated subcategory of \( \D^c(Cc) \).
\end{definition}

\begin{definition}[Algebraic triangulated category]
    Let \( \Tc \) be a triangulated category.

    Then \( \Tc \) is called an \emph{algebraic triangulated category} if there exist a pre-triangulated DG-category, \( \Cc \), such that \( H^0(\Cc) \) is equivalent to \( \Tc \).
\end{definition}


% DUMP

% \begin{remark}
%     Consider the following homogeneous of degree \( -1 \) map for \( A \in \C_{\dg}(\Mod(R)) \)
%     \begin{center}
%         \newcommand{\height}{2cm}
%         %
%         \mmznext{meaning to context=\height}
%         \begin{tikzpicture}
%             \diagram{m}{\height}{1cm} {
%                 A: \\
%                 \Sigma(A): \\
%             };
            
%             \draw[math]
%                 (m-1-1) edge node {\sigma_A} (m-2-1);
%         \end{tikzpicture}
%         %
%         \mmznext{meaning to context=\height}
%         \begin{tikzpicture}
%             \diagramorigin{m}{\height}{3cm} {
%                 \cdots \& A_{-1} \& A_0 \& A_1 \& \cdots \\
%                 \cdots \& A_0 \& A_1 \& A_2 \& \cdots \\
%             };

%             \draw[math]
%                 (m-1-1) edge (m-1-2)
%                 (m-1-2) edge (m-1-3)
%                     edge node {\Id} (m-2-1)
%                 (m-1-3) edge (m-1-4)
%                     edge node {\Id} (m-2-2)
%                 (m-1-4) edge (m-1-5)
%                     edge node {\Id} (m-2-3)
%                 (m-1-5) edge node {\Id} (m-2-4)
                
%                 (m-2-1) edge (m-2-2)
%                 (m-2-2) edge (m-2-3)
%                 (m-2-3) edge (m-2-4)
%                 (m-2-4) edge (m-2-5);
%         \end{tikzpicture}
%     \end{center}

%     Consider this (suggestively named) homogeneous of degree \( 1 \) morphism
%     \begin{center}
%         \newcommand{\height}{2cm}
%         %
%         \mmznext{meaning to context=\height}
%         \begin{tikzpicture}
%             \diagram{m}{\height}{1cm} {
%                 \Sigma(A): \\
%                 A: \\
%             };
            
%             \draw[math]
%                 (m-1-1) edge node {\sigma_A^{-1}} (m-2-1);
%         \end{tikzpicture}
%         %
%         \mmznext{meaning to context=\height}
%         \begin{tikzpicture}
%             \diagramorigin{m}{\height}{3cm} {
%                 \cdots \& A_0 \& A_1 \& A_2 \& \cdots \\
%                 \cdots \& A_{-1} \& A_0 \& A_1 \& \cdots \\
%             };

%             \draw[math]
%                 (m-1-1) edge (m-1-2)
%                     edge node {\Id} (m-2-2)
%                 (m-1-2) edge (m-1-3)
%                     edge node {\Id} (m-2-3)
%                 (m-1-3) edge (m-1-4)
%                     edge node {\Id} (m-2-4)
%                 (m-1-4) edge (m-1-5)
%                     edge node {\Id} (m-2-5)
                
%                 (m-2-1) edge (m-2-2)
%                 (m-2-2) edge (m-2-3)
%                 (m-2-3) edge (m-2-4)
%                 (m-2-4) edge (m-2-5);
%         \end{tikzpicture}
%     \end{center}

%     One can see that
%     \[
%         \sigma_A \circ \sigma_A^{-1} = \Id_{\Sigma(A)}
%     \]
%     and
%     \[
%         \sigma_A^{-1} \circ \sigma_A = \Id_A.
%     \]
%     Therefore, \( \sigma_A^{-1} \) is the inverse of \( \sigma_A \) in \( \C_{\dg}(\Mod(R)) \).

%     Furthermore, for \( f \in \C_{\dg}(\Mod(R))(A, B) \), homogeneous of degree \( i \), consider the morphism
%     \[
%         \sigma_B \circ f \circ \sigma_A^{-1}.
%     \]
%     Looking at what the morphism is doing
%     \begin{center}
%         \newcommand{\height}{3cm}
%         %
%         \mmznext{meaning to context=\height}
%         \begin{tikzpicture}
%             \diagram{m}{\height}{1cm} {
%                 \Sigma(A): \\
%                 A: \\
%                 B: \\
%                 \Sigma(B): \\
%             };

%             \draw[math]
%                 (m-1-1) edge node {\sigma_A^{-1}} (m-2-1)

%                 (m-2-1) edge node {f} (m-3-1)

%                 (m-3-1) edge node {\sigma_B} (m-4-1);
%         \end{tikzpicture}
%         %
%         \mmznext{meaning to context=\height}
%         \begin{tikzpicture}
%             \diagramorigin{m}{\height}{3cm} {
%                 \cdots \& A_0 \& A_1 \& A_2 \& \cdots \\
%                 \cdots \& A_{-1} \& A_0 \& A_1 \& \cdots \\
%                 \cdots \& B_{i-1} \& B_i \& B_{i + 1} \& \cdots \\
%                 \cdots \& B_i \& B_{i + 1} \& B_{i + 2} \& \cdots \\
%             };

%             \draw[math]
%                 (m-1-1) edge (m-1-2)
%                     edge node {\Id} (m-2-2)
%                 (m-1-2) edge (m-1-3)
%                     edge node {\Id} (m-2-3)
%                 (m-1-3) edge (m-1-4)
%                     edge node {\Id} (m-2-4)
%                 (m-1-4) edge (m-1-5)
%                     edge node {\Id} (m-2-5)

%                 (m-2-1) edge (m-2-2)
%                 (m-2-2) edge (m-2-3)
%                     edge node {f_{-1}} (m-3-2)
%                 (m-2-3) edge (m-2-4)
%                     edge node {f_0} (m-3-3)
%                 (m-2-4) edge (m-2-5)
%                     edge node {f_1} (m-3-4)

%                 (m-3-1) edge (m-3-2)
%                 (m-3-2) edge (m-3-3)
%                     edge node {\Id} (m-4-1)
%                 (m-3-3) edge (m-3-4)
%                     edge node {\Id} (m-4-2)
%                 (m-3-4) edge (m-3-5)
%                     edge node {\Id} (m-4-3)
%                 (m-3-5) edge node {\Id} (m-4-4)

%                 (m-4-1) edge (m-4-2)
%                 (m-4-2) edge (m-4-3)
%                 (m-4-3) edge (m-4-4)
%                 (m-4-4) edge (m-4-5);
%         \end{tikzpicture}
%     \end{center}
%     % TODO: Impliserar dette at det er naturleg iso?
%     one can see that it is exacly equal to \( (-1)^i\Sigma(f) \).
% \end{remark}

% % MS-question: Remark below. -> Probably OK.
% \begin{remark}
%     % Bondal--Kapranov has as definition
%     Enriched functor between DG-categories implies it preserves differentials and grading? TODO
% \end{remark}

% SRC: Berest--Mehrle 2017 LN
% \begin{definition}[DG-natural transformation]
%     Let \( F, G: \Ac \to \Bc \) be two DG-functors.

%     Then a collection of morphisms
%     \[
%         \alpha = \set*{ \alpha_A \in \Bc(F(A), G(A)) \mid A \in \Ac }
%     \]

%     TODO: Find secondary source, can't see the connection to nlab
% \end{definition}

% TODO: Need to show that it's a category?
% MS-Question: Correct? Berest--Mehrle (LN) has another def. -> Subscript dg betyr enriched.
% \begin{definition}[\( \Fun(\Ac, \Bc) \)]
%     Let \( \Ac, \Bc \) be two DG-categories over the same commutative ring \( R \).
    
%     Then let \( \Fun(\Ac, \Bc) \) denote the category of all DG-functors from \( \Ac \) to \( \Bc \), with morphisms being DG-natural transformations.
% TODO: Can't use this unless I have a definition of "DG-natural transformations".
% \end{definition}