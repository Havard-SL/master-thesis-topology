\begin{notation}
    Let \( \Cc \) be any additive category.
    
    Then let \( \C \tuple{\Cc} \) denote the category of chain complexes of objects in \( \Cc \).

    Furthermore let the differential in these chain complexes have \emph{ascending} order, i.e., for \( M_i, M_{i+1} \in \Cc \) which are adjacent objects in a chain complex \( M \in \C \tuple{\Cc} \), the differential would be
    \[
        d_i : M_i \to M_{i + 1}.
    \]
\end{notation}

% TODO: Obvious?
\begin{notation}
    In this section, there will be a lot of products and coproducts of modules mentioned, and so a small note on the notation of elements, and the reasoning behind the notation could be useful.

    Let \( R \) be a commutative ring with identity. Let \( A_i \in \Mod(R) \) and let
    \[
        \iota_i: A_i \to \coprod_{i \in \Zb} A_i
    \]
    denote the canonical split monomorphism by the universal property of the coproduct in \( \Mod(R) \).

    Then for any \( a_i \in A_i \), the element
    \[
        \iota_i(a_i) \in \coprod_{i \in \Zb} A_i
    \]
    is just denoted as
    \[
        a_i \in \coprod_{i \in \Zb} A_i.
    \]
    
    The reasoning for the above notation is twofold. Firstly, it reduces notation while not being ambigous. Secondly, one never talks about a general element of a coproduct. Almost always when talking about what a morphism does to an element of the coproduct, it is what happens to the \( \iota_i(a_i) \)'s, which is a consequence of the universal property of the coproduct.

    In addition, let 
    \[
        \pi_i: \prod_{i \in \Zb} A_i \to A_i
    \]
    be the universal split epimorphism by the universal property of the product in \( \Mod(R) \).
    
    Then for any element \( a \in \prod_{i \in \Zb} A_i \), denote
    \[
        a = \tuple{a_i}_{i \in \Zb} \in \prod_{i \in \Zb} A_i
    \]
    where \( a_i := \pi_i(a) \in A_i \).
    
    The reasoning behind this notation is because the product in \( \Mod(R) \) is the direct product, and because the universal property of the product. That is because it makes it so that when talking about morphisms into the product, the morphism is fully defined by what it does to each degree in \( \prod_{i \in \Zb} A_i \), which is easily shown using the above notation.  
\end{notation}

% Section wide TODO's: Unit morphism in enriched category, notation for elements in a coproduct.

\subsection{Definition of a DG-category}
In this thesis the definition of a DG-Category is based on enriched category theory, as it is a modern approach, and by the opinion of the author it is also the most elegeant approach.

\begin{definition}[Tensor product of chain complexes over \( \Mod(R) \)]
    \label{def:tensor_product_of_chain_complexes_over_Mod(R)}
    Let \( R \) be a commutative ring with identity. Furthermore let \( A, B \in \C \tuple{\Mod(R)} \).

    Then, for any  \( n \in \Zb \) define the modules
    \[
        (A \otimes B)_n := \coprod_{p + q = n} A_p \otimes B_q
    \]
    which are a part of the chain complex
    \begin{center}
        \begin{tikzpicture}
            \diagram{m}{1cm}{1cm} {
                A \otimes B: \\
            };
        \end{tikzpicture}
        %
        \begin{tikzpicture}
            \diagram{m}{1cm}{1cm} {
                \cdots \& (A \otimes B)_{-1} \& (A \otimes B)_0 \& (A \otimes B)_1 \& \cdots \\
            };

            \draw[math]
                (m-1-1) edge (m-1-2)
                (m-1-2) edge node {d_{-1}} (m-1-3)
                (m-1-3) edge node {d_0} (m-1-4)
                (m-1-4) edge (m-1-5);
        \end{tikzpicture}
    \end{center}

    Where the differentials, \( d_n \), are defined as follows:
    
    Let \( i, j \in \Zb \) with \( i + j = n \), and let
    \[
        \iota_{i, j}: A_i \otimes B_j \hookrightarrow \tuple{A \otimes B}_n
    \]
    be the canonical split-monomorphism. Furthermore let \( a \otimes b \in A_i \otimes B_j \) be an elementary tensor.

    Then the differential is uniquely defined as follows
    \[
        d_n(\iota_{i, j}(a \otimes b)) := \iota_{i + 1, j}\tuple{d_{A, i}(a) \otimes b} + (-1)^{i} \iota_{i, j + 1}\tuple{a \otimes d_{B, j}(b)}.
    \]

    This is called the \emph{tensor product of chain complexes over \( \Mod(R) \)}.
\end{definition}

% TODO: Generalize to a lemma and show the two special cases?
\begin{lemma}
    Let \( A, B \in \C(\Mod(R)) \) and let \( C \in \Mod(R) \). Furthermore let \( i, j \in \Zb \) with \( i + j = n \). And let
    \[
        \iota_{i, j}: A_i \otimes B_j \to (A \otimes B)_n := \coprod_{p + q = n} A_p \otimes B_q
    \]
    be the canocial split monomorphisms.

    Then for any
    \[
        f: (A \otimes B)_n \to C
    \]
    where for any \( a \in A_i, b \in B_j \) one has
    \[
        f_n(\iota_{i, j}(a \otimes b)) = g_{i, j}(a, b)
    \]
    for some \( R \)-bilinear morphisms
    \[
        g_{i, j}: A_i \times B_j \to C.
    \]

    Then \( f_n \) is uniquely defined by the \( g_{i, j} \)-s.
\end{lemma}
\begin{proof}
    Look at the following diagram
    \begin{equation}
        \label{tikz:differential_of_tensor_product_of_chain_complexes_over_Mod(R)}
        \begin{tikzpicture}
            \diagram{m}{2cm}{2cm} {
                A_i \otimes B_j \& \coprod\limits_{p + q = n} A_p \otimes B_q \\
                A_i \times B_j \& C \\
            };

            \draw[math]
                (m-1-1) edge[hook] node {\iota_{i, j}} (m-1-2)
                    edge[dashed] node {\alpha_{i, j}} (m-2-2)
                (m-1-2) edge[dashed] node {\beta} (m-2-2)

                (m-2-1) edge node {g_{i, j}} (m-2-2)
                    edge node {\otimes} (m-1-1);
        \end{tikzpicture}
    \end{equation}

    Where the \( g_{i, j} \)-s are \( R \)-bilienar.
\end{proof}
\begin{remark}
    The definition of the differentials in \autoref{def:tensor_product_of_chain_complexes_over_Mod(R)} is well defined and unique by the following argument.

    Look at the following diagram
    \begin{equation}
        \label{tikz:differential_of_tensor_product_of_chain_complexes_over_Mod(R)}
        \begin{tikzpicture}
            \diagram{m}{2cm}{2cm} {
                A_i \otimes B_j \& \coprod\limits_{p + q = n} A_p \otimes B_q \\
                A_i \times B_j \& \coprod\limits_{p' + q' = n + 1} A_{p'} \otimes B_{q'} \\
            };

            \draw[math]
                (m-1-1) edge[hook] node {\iota_{i, j}} (m-1-2)
                    edge[dashed] node {\alpha_{i, j}} (m-2-2)
                (m-1-2) edge[dashed] node {\beta} (m-2-2)

                (m-2-1) edge node {f} (m-2-2)
                    edge node {\otimes} (m-1-1);
        \end{tikzpicture}
    \end{equation}
    Let \( f \) be the map defined element-wise as follows
    \[
        a \times b \mapsto \iota_{i + 1, j}\tuple{d_{A, i}(a) \otimes b} + (-1)^{i} \iota_{i, j + 1}\tuple{a \otimes d_{B, j}(b)}
    \]
    One can check that this map is \( R \)-balanced, and therefore a morphism. In addition, by the universal property of tensor product in \( \Mod(R) \), \( f \) induces a unique morphism, \( \alpha_{i, j} \), which is induced from the elementary tensors as follows
    \[
        a \otimes b \mapsto \iota_{i + 1, j}\tuple{d_{A, i}(a) \otimes b} + (-1)^{i} \iota_{i, j + 1}\tuple{a \otimes d_{B, j}(b)}.
    \]
    Since this works for any \( i, j \) as long as \( i + j = n \), one can construct \( \alpha_{i, j} \) for every valid \( i, j \) pair.

    Then by using the universal property of the coproduct one gets the unique map \( \beta \) which is by \autoref{tikz:differential_of_tensor_product_of_chain_complexes_over_Mod(R)} uniquely determined by it's actions on elementary tensors in \( A_i \otimes B_j \) in the following way
    \[
        \iota_{i, j}(a \otimes b) \mapsto \iota_{i + 1, j}\tuple{d_{A, i}(a) \otimes b} + (-1)^{i} \iota_{i, j + 1}\tuple{a \otimes d_{B, j}(b)}.
    \]
    Which is exactly equal to \( d_n \).

    By a similar argument as above, it follows that since \( d_{n + 1} \circ d_n \) sends every \( \iota_{i, j}(a \otimes b) \) to \( 0 \), then it has to be the zero map, and \( d_n \) is therefore a differential.
\end{remark}

\begin{definition}[Internal hom of chain complexes over \( \Mod(R) \)]
    \label{def:internal_hom_of_chain_complexes_over_Mod(R)}
    Let \( R \) be a commutative ring with identity. Furthermore let \( A, B \in \C \tuple{\Mod(R)} \).

    Then, for any \( n \in \Ab \) define the modules
    \[
        \class{A, B}_n := \prod_{i \in \Zb} \Mod(R)(A_i, B_{i + n})
    \]
    which are a part of the chain complex
    \begin{center}
        \begin{tikzpicture}
            \diagram{m}{1cm}{1cm} {
                \class{A, B}: \\
            };
        \end{tikzpicture}
        %
        \begin{tikzpicture}
            \diagram{m}{1cm}{1cm} {
                \cdots \& \class{A, B}_{-1} \& \class{A, B}_0 \& \class{A, B}_1 \& \cdots \\
            };

            \draw[math]
                (m-1-1) edge (m-1-2)
                (m-1-2) edge node {d_{-1}} (m-1-3)
                (m-1-3) edge node {d_0} (m-1-4)
                (m-1-4) edge (m-1-5);
        \end{tikzpicture}
    \end{center}

    Where the differentials, \( d_n \), are defined as follows:
    
    For any \( i \in \Zb \), let \( \partial_{n, i} \) be the follwing morphism
    \begin{align*}
        \partial_{n, j} : \prod_{i \in \Zb}\Mod(R)(A_i, B_{i + n}) &\to \Mod(R)(A_j, B_{j + n + 1}) \\
        \bigtimes_{i \in \Zb} f_i &\mapsto d_{B, j + n} \circ f_j - (-1)^n f_{j + 1} \circ d_{A, j}.
    \end{align*}
    Then
    \begin{align*}
        d_n := \prod_{i \in \Zb} \partial_{n, i}: \prod_{i \in \Zb} \Mod(R)(A_i, B_{i + n}) &\to \prod_{i \in \Zb} \Mod(R)(A_i, B_{i + n + 1}) \\
        \bigtimes_{i \in \Zb} f_i &\mapsto \bigtimes_{i \in \Zb} d_{B, i + n} \circ f_i - (-1)^n f_{i + 1} \circ d_{A, i}.
    \end{align*}
    This is called the \emph{internal hom of chain complexes over \( \Mod(R) \)}.
\end{definition}

\begin{remark}
    To verify that the differential defined in \autoref{def:internal_hom_of_chain_complexes_over_Mod(R)} works on the elements as described; first verify that it satisfies the universal property, and then by uniqueness it must be \emph{the} differential as defined dy the product of the \( \partial_{n, j} \)-s.

    To verify that \( d_{n + 1} \circ d_n = 0 \), use the element-wise definition.
\end{remark}

% TODO: Fix the indices of the iotas to coincide with the indices in the definition of tensor product of chain complexes.
% TODO: Prove point 2 and 3, or at least find some sources on them.
\begin{remark}[Tensor product and internal hom adjunction in \( \C(\Mod(R)) \)]
    Let \( A, B, C \) be chain complexes in \( \C(\Mod(R)) \) for some commutative ring \( R \).

    Let \( f \in \C(\Mod(R))\tuple{A \otimes B, C} \). Want to find out what the adjoint of \( f \) is.

    Let \( f = \set{f_n}_{n \in \Zb} \) where \( f_n \in \Mod(R)\tuple{ \tuple{A \otimes B}_n, C_n} \) are the individual level-wise morphisms of the chain morphism \( f \).

    Then, unwrapping the definitions, one has that \( f \) looks like the following diagram
    \begin{center}
        \begin{tikzpicture}
            \diagram{m}{1cm}{1cm} {
                \cdots \& \coprod\limits_{i + j = -1} A_i \otimes B_j \& \coprod\limits_{i + j = 0} A_i \otimes B_j \& \coprod\limits_{i + j = 1} A_i \otimes B_j \& \cdots \\
                \cdots \& C_{-1} \& C_0 \& C_1 \& \cdots \\
            };

            \draw[math]
                (m-1-1) edge (m-1-2)
                    edge (m-2-1)
                (m-1-2) edge node {d_{A \otimes B, -1}} (m-1-3)
                    edge node {f_{-1}} (m-2-2)
                (m-1-3) edge node {d_{A \otimes B, 0}} (m-1-4)
                    edge node {f_0} (m-2-3)
                (m-1-4) edge (m-1-5)
                    edge node {f_1} (m-2-4)
                (m-1-5) edge (m-2-5)

                (m-2-1) edge (m-2-2)
                (m-2-2) edge node {d_{C, -1}} (m-2-3)
                (m-2-3) edge node {d_{C, 0}} (m-2-4)
                (m-2-4) edge (m-2-5);
        \end{tikzpicture}
    \end{center}
    Likewise, the adjoint has to look like the following diagram
    \begin{center}
        \begin{tikzpicture}
            \diagram{m}{1cm}{0.75cm} {
                \cdots \& A_{-1} \& A_0 \& A_1 \& \cdots \\
                \cdots \& \prod\limits_{j \in \Zb} \Mod(R)(B_j, C_{j - 1}) \& \prod\limits_{j \in \Zb} \Mod(R)(B_j, C_j) \& \prod\limits_{j \in \Zb} \Mod(R)(B_j, C_{j + 1}) \& \cdots \\
            };

            \draw[math]
                (m-1-1) edge (m-1-2)
                    edge (m-2-1)
                (m-1-2) edge node {d_{A, -1}} (m-1-3)
                    edge node {?_{-1}} (m-2-2)
                (m-1-3) edge node {d_{A, 0}} (m-1-4)
                    edge node {?_0} (m-2-3)
                (m-1-4) edge (m-1-5)
                    edge node {?_1} (m-2-4)
                (m-1-5) edge (m-2-5)

                (m-2-1) edge (m-2-2)
                (m-2-2) edge node {d_{\class{B, C}, -1}} (m-2-3)
                (m-2-3) edge node {d_{\class{B, C}, 0}} (m-2-4)
                (m-2-4) edge (m-2-5);
        \end{tikzpicture}
    \end{center}

    For any \( n \in \Zb \), let \( i', j' \in \Zb \) with \( i' + j' = n \) let
    \[
        \iota_{i', j'}: A_{i'} \otimes B_{j'} \hookrightarrow \coprod_{i + j = n} A_i \otimes B_j
    \]
    be the canonical \( i' \)-th split monomorphism by the definintion of the coproduct \( \coprod_{i + j = n} A_i \otimes B_j \).

    Then take the hom-tensor adjoint in \( \Mod(R) \) of the morphism
    \[
        f_{i + j} \circ \iota_{i, j}: A_i \otimes B_j \to C_{i + j}.
    \]
    This yields a morphism
    \begin{align*}
        \phi_{f, i, j}: A_i &\to \Mod(R)(B_j, C_{j + i}) \\
        a &\mapsto f_{i + j} \circ \iota_{i, j}(a \otimes ?).
    \end{align*}
    Then by the universal property of the product there is some morphism
    \[
        \phi_{f, i} := \prod_{j \in \Zb} \phi_{f, i, j}: A_i \to \prod_{j \in \Zb} \Mod(R)\tuple{B_j, C_{j + i}}.
    \]
    Collecting these morphisms yields a morphism, which I claim to be the adjoint of \( f \), namely 
    \[
        \phi_f := \set{\phi_{f, i}}_{i \in \Zb}.
    \]
    In order to show that this is the proper adjoint definition, one need to show the following properties
    \begin{enumerate}
        \item {
            That \( \phi_f \) is a chain morphism.
        }
        \item {
            That the assignment \( f \mapsto \phi_f \) is an isomorphism of groups.
        }
        \item {
            That there is a natural transformation
            \[
                \C(\Mod(R))(?_1 \otimes ?_2, ?_3) \cong \C(\Mod(R))(?_1, \left[ ?_2, ?_3 \right])
            \]
            where the natural morphisms are \( f \mapsto \phi_f \).
        }
    \end{enumerate}
    % TODO: Possibly expand this proof? or TODO: SRC
    In this thesis I will only prove the first statement.

    1) Want to show that \( \phi_f \) is a chain morphism.

    Need to check that for any \( i \in \Zb \) that the following diagram commutes
    \begin{center}
        \begin{tikzpicture}
            \diagram{m}{1cm}{1cm} {
                A_i \& A_{i + 1} \\
                \class{B, C}_i \& \class{B, C}_{i + 1} \\
            };

            \draw[math]
                (m-1-1) edge node {d_{A, i}} (m-1-2)
                    edge node {\phi_{f, i}} (m-2-1)
                (m-1-2) edge node {\phi_{f, i + 1}} (m-2-2)

                (m-2-1) edge node {d_{\class{B, C}, i}} (m-2-2);
        \end{tikzpicture}
    \end{center}
    Pick an arbitrary \( a \in A_i \), look at the following equation.
    \begin{align*}
        \phi_{f, i + 1} \circ d_{A, i}(a) - d_{\class{B, C}, i} \circ \phi_{f, i}(a)
        &= \bigtimes_{j \in \Zb} \tuple{ f_{i + j + 1} \circ \iota_{i + 1, j} (d_{A, i}(a) \otimes ?) } \\
        &\hspace{1cm}- d_{\class{B, C}, i} \tuple{ \bigtimes_{j \in \Zb} f_{i + j} \circ \iota_{i, j} (a \otimes ?) }. \\
        \intertext{Then by expanding out the definition of \( d_{\class{B, C}, i} \) it follows that}
        &= \bigtimes_{j \in \Zb} ( f_{i + j + 1} \circ \iota_{i + 1, j} (d_{A, i}(a) \otimes ?) \\
        &\hspace{1cm} - d_{C, i + j} \circ f_{i + j} \circ \iota_{i, j} (a \otimes ?) \\
        &\hspace{1cm} + (-1)^i f_{i + j + 1} \circ \iota_{i, j + 1} (a \otimes d_{B, j}(?)) ). \\
        \intertext{Then by consolodating the two terms that post-compose by \( f_{i + j + 1} \) it follows that}
        &= \bigtimes_{j \in \Zb} ( f_{i + j + 1} \circ ( \iota_{i + 1, j} (d_{A, i}(a) \otimes ?) \\
        &\hspace{1cm} + (-1)^i \iota_{i, j + 1} (a \otimes d_{B, j}(?)) ) \\
        &\hspace{1cm} - d_{C, i + j} \circ f_{i + j} \circ \iota_{i, j} (a \otimes ?) ). \\
        \intertext{Then by the definition of the differential of \( A \otimes B \) it follows that}
        &= \bigtimes_{j \in \Zb} ( f_{i + j + 1} \circ d_{A \otimes B, i + j} \tuple{\iota_{i, j}(a \otimes ?)} \\
        &\hspace{1cm} - d_{C, i + j} \circ f_{i + j} ( \iota_{i, j} (a \otimes ?) ) ). \\
        \intertext{Then by \( f \) being a chain homomorphism from \( A \otimes B \) to \( C \) it follows that}
        &= 0.
    \end{align*}
\end{remark}

% TODO: SRC or show.
% TODO: Connect with the above statements.
% https://ncatlab.org/nlab/show/category+of+chain+complexes
\begin{fact}[\( \C(\Mod(R)) \) is symmetric monoidal]
    Let \( R \) be a commutative ring with identity, and let \( \otimes \) denote the tensor product on \( \C \tuple{\Mod(R)} \). Furthermore let \( I \) be the chain complex in \( \C \tuple{\Mod(R)} \) consisting solely of \( 0 \)-objects exept for the \( R \)-module \( R \) in index \( 0 \).

    Then \( \tuple{\C \tuple{\Mod(R)}, \otimes, I} \) is a symmetric closed monoidal category.
\end{fact}

\begin{definition}[DG-category]
    Let \( R \) be a commutative ring with identity.

    Then \( \Cc \) is a \emph{DG-category over \( R \)} if it is a category enriched over \( \C \tuple{\Mod(R)} \).
\end{definition}
This definition also appears in \cite[p. 29]{Jasso-Muro_2023}, except they define it for a field and not a commutative ring with identity as is done in this thesis.


\subsection{Definition of Massey Product in a DG-category}
\begin{notation}
    Let \( R \) be a commutative ring with identity.

    Then let \( G \Mod(R) \) denote the category of graded \( R \)-modules.
\end{notation}

\begin{notation}
    Let \( R \) be a commutative ring with identity.

    Then let
    \[
        H^*: C \tuple{\Mod(R)} \to G \Mod(R)
    \]
    be the cohomology functor.
\end{notation}

% SRC Künneth: https://ncatlab.org/nlab/show/K%C3%BCnneth+theorem
\begin{fact}
    \label{fact:massey_product_in_dg_cat/massey_product_definition/algebraic_kunneth_isomorphism}
    Considering objects of \( G \Mod(R) \) as chain complexes with a zero-differential, one can define a tensor product using \autoref{def:massey_product_in_dg_cat/what_is_a_dg_cat/tensor_product_of_chain_complexes}.

    Let \( A, B \in C \tuple{\Mod(R)} \).

    By the algebraic Künneth theorem when \( R \) is a field, there is an isomorphism
    \[
        \phi: H^*(A) \otimes H^*(B) \stackrel{\sim}{\to} H^*(A \otimes B)
    \]
    TODO
\end{fact}

% MS-Question: Does the composition definition work? Also TODO below.
% TODO: Find definition of composition that does not rely on kunneth isomorphism, but only the existance of a morphism. Ask MS!
\begin{definition}
    Let \( \Cc \) be a differentially graded category over a field \( R \), and let \( \phi \) be as in \autoref{fact:massey_product_in_dg_cat/massey_product_definition/algebraic_kunneth_isomorphism}.

    Let \( \Kc \) be the following (enriched over \( C\tuple{\Mod(R)} \)) category:
    \begin{enumerate}
        \item Let \( \Obj(\Kc) = \Obj(\Cc) \).
        \item For any \( A, B \in \Obj(\Kc) \), let \( \Kc(A, B) = H^* \tuple{\Cc(A, B)} \).
        \item {
            For any \( A, B, C \in \Obj(\Kc) \), for any \( f \in \Kc(A, B) \) and \( g \in \Kc(B, C) \)

            Let composition in \( \Kc \) be defined as follows
            \[
                \circ_{\Kc}: \Kc(B, C) \otimes \Kc(A, B) \to \Kc(A, C)
            \]
            where \( \circ_{\Kc} \) is the composition
            \begin{center}
                \begin{tikzpicture}
                    \diagram{m}{1cm}{1.3cm} {
                        \Kc(B, C) \otimes \Kc(A, B) & & \Kc(A, C) \\
                        H^*\tuple{\Cc(B, C)} \otimes H^*\tuple{\Cc(A, B)} & H^*\tuple{\Cc(B, C) \otimes \Cc(A, B)} & H^*\tuple{\Cc(A, C)} \\
                    };

                    \draw[math]
                        (m-1-1) edge[dashed] node {\circ_{\Kc}} (m-1-3)
                            edge[equal] (m-2-1)
                        (m-1-3) edge[equal] (m-2-3)
                        
                        (m-2-1) edge node {\phi} node[swap] {\sim} (m-2-2)
                        (m-2-2) edge node {H^*(\circ_{\Cc})} (m-2-3); 
                \end{tikzpicture}
            \end{center}
        }
    \end{enumerate}

    Then the category \( \Kc \) is denoted as \( H^*(\Cc) \).
\end{definition}

% TODO: Why is degrees preserved when choosing representatives?
% SRC: Wikipedia https://en.wikipedia.org/wiki/Massey_product and TODO:guessing
\begin{definition}
    \label{def:massey_product_in_dg_cat/massey_product_definition/massey_product_dg_cat}
    Let \( \Cc \) be a differentially graded category over \( R \).

    Let the following be a diagram in \( H^*(\Cc) \)
    \begin{center}
        \begin{tikzpicture}
            \diagram{m}{1cm}{1cm} {
                X_1 & X_2 & X_3 & X_4 \\
            };

            \draw[math]
                (m-1-1) edge node {g_1} (m-1-2)
                (m-1-2) edge node {g_2} (m-1-3)
                (m-1-3) edge node {g_3} (m-1-4);
        \end{tikzpicture}
    \end{center}

    Where \( g_1, g_2 \) and \( g_3 \) are homogenous elements with degree \( d_1, d_2 \) and \( d_3 \) respectively.

    Then let \( f_1, f_2 \) and \( f_3 \) be homogeneous representatives of the equivalence class of \( g_1, g_2 \) and \( g_3 \) respectively. These will also have degree \( d_1, d_2 \) and \( d_3 \) respectively. I.e: \( H^*(f_1) = g_1, H^*(f_2) = g_2 \) and \( H^*(f_3) = g_3 \).

    Furthermore for a homogeneous element, \( h \) with degree \( d_h \), let \( \bar{h} := (-1)^{d_h + 1}h \).

    Then let:
    \[
        \set{
            H^* \tuple{
                \bar{s} \circ f_1 + \bar{f_3} \circ t
            }
            \mid
            d(s) = \bar{f_3} \circ f_2, \quad
            d(t) = \bar{f_2} \circ f_1
        }
    \]
    This is a subset of \( H^* \tuple{\Cc \tuple{X_1, X_4}} \), called the \emph{Massey product of \( g_3, g_2 \) and \( g_1 \)}, and is denoted as \( \toda{g_3, g_2, g_1 } \).
\end{definition}

\begin{theorem}
    The Massey product definition in \autoref{def:massey_product_in_dg_cat/massey_product_definition/massey_product_dg_cat} is well-defined.
\end{theorem}
\begin{proof}
    Want to show that for different choices of representatives, that the definition returns the same subset.

    Also want to show that homogeneous representatives exist.

    Also want to show that the degree of homogenous representatives are the same.

    TODO
\end{proof}

\begin{remark}
    In particular, considering the degrees of \( g_1, g_2 \) and \( g_3 \), one can see that the only non-zero degree of \( \toda{g_3, g_2, g_1 } \subseteq H^* \tuple{\Cc \tuple{X_1, X_4}} \) is in \( H^{d_3 + d_2 + d_1 - 1} \tuple{\Cc \tuple{X_1, X_4}} \).
\end{remark}

% SRC: Computations done on ReMarkable Massey Product/Definition Massey prod p2
\begin{remark}
    Comparing \autoref{def:massey_product_in_dg_cat/massey_product_definition/massey_product_dg_cat} and the definition for massey product in a DG category given by Jasso--Muro 2023 (TODO: Ref) in ``Definition 4.2.1'' with \( d = 1 \) we can see that both definitions agree.
\end{remark}

% TODO: Add definition for non-homogenous. Is there even a definition for non-homogenous?

\subsection{What is a pre-triangulated category?}
There are multiple definitions of an algebraic triangulated category. In this thesis, the definition of an algebraic triangulated category will be the existence of a DG-enhancement as defined in \cite[Definition 3.1.3]{Jasso-Muro_2023}. This section goes through this definition in detail.

% TODO Cite: Jasso--Muro
% TODO: Not define based on g_i,j's?
% TODO: Verify definition with borceux definition of composition p. 295
% TODO: Specify unit morphism? Why are they needed???
\begin{definition}[\( \C_{\dg}(\Mod(R)) \)]
    \label{def:c_dg_mod_r}
    Let \( R \) be a commutative ring with identity.

    Then let \emph{\( \C_{\dg}(\Mod(R)) \)} be a DG-category defined as follows
    \begin{enumerate}
        \item {
            \( \Obj(\C_{\dg}(\Mod(R))) := \Obj(\C(\Mod(R))) \).
        }
        \item {
            Let \( A, B \in \C_{\dg}(\Mod(R)) \). Let \( \class{A, B} \) denote the internal hom of \( \C(\Mod(R)) \) (\autoref{def:internal_hom_of_chain_complexes_over_Mod(R)}) with respect to \( A, B \) as objects from \( \C(\Mod(R)) \).

            Then let \( \C_{\dg}(\Mod(R))(A, B) := \class{A, B} \).
        }
        \item {
            For \( A, B, C \in \C_{\dg}(\Mod(R)) \), let
            \[
                c_{\C_{\dg}(\Mod(R))}: \C_{\dg}(\Mod(R))(B, C) \otimes \C_{\dg}(\Mod(R))(A, B) \to \C_{\dg}(\Mod(R))(A, C)
            \]
            be defined as the chain morphism where \( c_{\C_{\dg}(\Mod(R)), n} \) is uniquely defined on
            \[
                \bigtimes_{p \in \Zb} g_p \in [B, C]_i = \prod_{p \in \Zb} \Mod(R)(B_p, C_{p + i}) \text{ and } \bigtimes_{q \in \Zb} f_q \in [A, B]_j = \prod_{p \in \Zb} \Mod(R)(A_p, B_{p + j})
            \]
            as follows
            \begin{align*}
                c_{\C_{\dg}(\Mod(R)), n}: \tuple{ \C_{\dg}(\Mod(R))(B, C) \otimes \C_{\dg}(\Mod(R))(A, B) }_n &\to \C_{\dg}(\Mod(R))(A, C)_n \\
                \iota_{i, j} \tuple{\tuple{ \bigtimes_{p \in \Zb} g_p } \otimes \tuple{ \bigtimes_{q \in \Zb} f_q } } &\mapsto \bigtimes_{p \in \Zb} g_{p + j} \circ f_p
            \end{align*}
            where \( \iota_{i, j} \) is the canonical split monomorphism into the coproduct.
        }
    \end{enumerate}
\end{definition}

\begin{remark}
    The composition definition in \autoref{def:c_dg_mod_r} is well defined by the following two arguments:

    \begin{enumerate}
        \item {
            Firstly, the morphisms \( c_{\C_{\dg}(\Mod(R)), n} \) are uniqely defined by \autoref{lem:map_out_of_tensor_unique} where the \( g_{i, j} \)'s are as follows
            \begin{align*}
                g_{i, j}: \prod_{p \in \Zb} \Mod(R)(B_p, C_{p + i}) \times \prod_{p \in \Zb} \Mod(R)(A_p, B_{p + j}) &\to \prod_{p \in \Zb} \Mod(R)(A_p, C_{p + i + j}) \\
                \tuple{ \bigtimes_{p \in \Zb} g_p } \times \tuple{ \bigtimes_{q \in \Zb} f_q } &\mapsto \bigtimes_{p \in \Zb} g_{p + j} \circ f_p.
            \end{align*}
            And these can be checked to be \( R \)-bilinear.
        }
        \item {
            Secondly, need to check that the \( c_{\C_{\dg}(\Mod(R)), n} \)'s form a chain morphism.

            Look at the following equation (with shortened notation for brevity) 
            \begin{align*}
                &( d_{(A, C), n} \circ c_n - c_{n + 1} \circ d_{(B, C) \otimes (A, B), n} )\tuple{ \iota_{i, j} \tuple{\tuple{ \bigtimes_{p \in \Zb} g_p } \otimes \tuple{ \bigtimes_{q \in \Zb} f_q } } } \\
                &= d_{(A, C), n}\tuple{c_n \tuple{ \iota_{i, j} \tuple{\tuple{ \bigtimes_{p \in \Zb} g_p } \otimes \tuple{ \bigtimes_{q \in \Zb} f_q } } }} - c_{n + 1} \tuple{ d_{(B, C) \otimes (A, B), n}\tuple{ \iota_{i, j} \tuple{\tuple{ \bigtimes_{p \in \Zb} g_p } \otimes \tuple{ \bigtimes_{q \in \Zb} f_q } } } } \\
                \intertext{then by definition of composition as well as differential of tensor product}
                &= d_{(A, C), n}\tuple{
                    \bigtimes_{p \in \Zb} g_{p + j} \circ f_p 
                } \\
                &\hspace{0.4cm} - c_{n + 1} \tuple{
                    \iota_{i + 1, j}\tuple{
                        d_{(B, C), i}\tuple{ \bigtimes_{p \in \Zb} g_p } \otimes \tuple{ \bigtimes_{q \in \Zb} f_q }
                    } + (-1)^i \iota_{i, j + 1}\tuple{
                        \tuple{ \bigtimes_{p \in \Zb} g_p } \otimes d_{(A, B), j} \tuple{ \bigtimes_{q \in \Zb} f_q }
                    }
                } \\
                \intertext{then by the definition of the differential of the internal hom}
                &= \bigtimes_{p \in \Zb}\tuple{
                    d_{C, p + n} \circ g_{p + j} \circ f_p - (-1)^n \circ g_{p + j + 1} \circ f_{p + 1} \circ d_{A, p}
                } \\
                &\hspace{0.4cm} - c_{n + 1} (
                    \iota_{i + 1, j}\tuple{              
                        \bigtimes_{p \in \Zb} \tuple{
                            d_{C, p + i} \circ g_p - (-1)^i g_{p + 1} \circ d_{B, p}
                        } \otimes \tuple{ \bigtimes_{q \in \Zb} f_q }
                    } \\
                    &\hspace{0.8cm}+ (-1)^i \iota_{i, j + 1}\tuple{
                        \tuple{ \bigtimes_{p \in \Zb} g_p } \otimes \bigtimes_{q \in \Zb}\tuple{
                            d_{B, q + j} \circ f_q - (-1)^j f_{q + 1} \circ d_{A, q}
                        }
                    }
                ) \\
                \intertext{then by the fact that composition is an \( R \)-homomorphism}
                &= \bigtimes_{p \in \Zb}\tuple{
                    d_{C, p + n} \circ g_{p + j} \circ f_p - (-1)^n \circ g_{p + j + 1} \circ f_{p + 1} \circ d_{A, p}
                } \\
                &\hspace{0.4cm} - c_{n + 1} \tuple{
                    \iota_{i + 1, j}\tuple{              
                        \bigtimes_{p \in \Zb} \tuple{
                            d_{C, p + i} \circ g_p - (-1)^i g_{p + 1} \circ d_{B, p}
                        } \otimes \tuple{ \bigtimes_{q \in \Zb} f_q }
                    }
                } \\
                &\hspace{0.4cm} - (-1)^i c_{n + 1} \tuple{
                    \iota_{i, j + 1}\tuple{
                        \tuple{ \bigtimes_{p \in \Zb} g_p } \otimes \bigtimes_{q \in \Zb}\tuple{
                            d_{B, q + j} \circ f_q - (-1)^j f_{q + 1} \circ d_{A, q}
                        }
                    }
                } \\
                \intertext{then by the definition of composition}
                &= \bigtimes_{p \in \Zb}\tuple{
                    d_{C, p + n} \circ g_{p + j} \circ f_p
                } - (-1)^n \bigtimes_{p \in \Zb}\tuple{
                    \circ g_{p + j + 1} \circ f_{p + 1} \circ d_{A, p}
                } \\
                &\hspace{0.4cm} - \bigtimes_{p \in \Zb}\tuple{
                    d_{C, p + i + j} \circ g_{p + j} \circ f_p
                } + (-1)^i \bigtimes_{p \in \Zb}\tuple{
                    g_{p + j + 1} \circ d_{B, p + j} \circ f_p
                } \\
                &\hspace{0.4cm} - (-1)^i \bigtimes_{p \in \Zb} \tuple{
                    g_{p + j + 1} \circ d_{B, p + j} \circ f_p
                } + (-1)^{i + j} \bigtimes_{p \in \Zb} \tuple{
                    g_{p + j + 1} \circ f_{p + 1} \circ d_{A, p}
                } \\
                &= 0.
            \end{align*}
            By \autoref{lem:map_out_of_tensor_unique}, this shows that the morphism
            \begin{multline*}
                d_{(A, C), n} \circ c_n - c_{n + 1} \circ d_{(B, C) \otimes (A, B), n}: \\
                \tuple{ \C_{\dg}(\Mod(R))(B, C) \otimes \C_{\dg}(\Mod(R))(A, B) }_n \to \C_{\dg}(\Mod(R))(A, C)_{n + 1}
            \end{multline*}             
            corresponds to the \( g_{i,j} \)'s where \( g_{i, j} = 0 \). However, by uniqueness, this implies that
            \[
                d_{(A, C), n} \circ c_n - c_{n + 1} \circ d_{(B, C) \otimes (A, B), n} = 0.
            \]
        }
    \end{enumerate}
\end{remark}

% TODO: Unit morphism?
\begin{definition}[Opposite DG-category]
    \label{def:opposite_dg_category}
    Let \( \Cc \) be a DG-category.

    Then let \( \Cc^{op} \) be the DG-category defined as follows
    \begin{enumerate}
        \item {
            \( \Obj(\Cc^{op}) := \Obj(\Cc) \)
        }
        \item {
            For \( A, B \in \Cc^{op} \), let \( \Cc^{op}(A, B) := \Cc(B, A) \).
        }
        \item {
            For \( A, B, C \in \Cc^{op} \), let composition be the chain morphism
            \[
                c_{\Cc^{\op}}: \Cc^{\op} (B, C) \otimes \Cc^{\op} (A, B) \to \Cc^{\op} (A, C)
            \]
            where in degree \( n \) there is the following morphism that is uniquely defined by the following assignments for any \( i, j \in \Zb \) with \( i + j = n \) and any \( a \in \Cc^{\op} (B, C)_i, b \in \Cc^{\op} (A, B)_j \)
            \begin{align*}
                c_{\Cc^{\op}, n}: \tuple{ \Cc^{\op} (B, C) \otimes \Cc^{\op} (A, B) }_n &\to \Cc^{\op} (A, C)_n \\
                \iota_{i, j}\tuple{ a \otimes b } &\mapsto (-1)^{ij} c_\Cc \tuple{ b \otimes a }
            \end{align*}
        }
    \end{enumerate}
\end{definition}
\begin{remark}
    In order to show that the opposite DG-category defined in \autoref{def:opposite_dg_category} is a well defined DG-category, need to show the following
    \begin{enumerate}
        \item  {
            The composition morphism is uniquely defined as stated.

            WIP
        }
        \item {
            The composition morphism is a chain morphism.

            WIP
        }
    \end{enumerate}
\end{remark}

% TODO Cite: Jasso--Muro, Borceux
\begin{definition}[DG-functor]
    An enriched functor between two DG-categories is called a \emph{DG-functor}.
\end{definition}

\begin{definition}[\( \Sigma_{\C_{\dg}(\Mod(R))} \)]
    \label{def:sigma_c_dg}
    Let the DG-functor \( \Sigma_{\C_{\dg}(\Mod(R))} \) be defined as follows
    \begin{enumerate}
        \item {
            For \( A \in \C_{\dg}(\Mod(R)) \), let
            \[
                \Sigma_{\C_{\dg}(\Mod(R))}(A) := \Sigma_{\C(\Mod(R))}(A)
            \]
        }
        \item {
            For any \( A, B \in \C_{\dg}(\Mod(R)) \), let
            \begin{align*}
                \Sigma_{\C_{\dg}(\Mod(R))}: \C_{\dg}(\Mod(R))(A, B) \to \C_{\dg}(\Mod(R))(\Sigma_{\C_{\dg}(\Mod(R))}(A), \Sigma_{\C_{\dg}(\Mod(R))}(B))
            \end{align*}
            be the chain homomorphism where the \( n \)-th map is
            \begin{align*}
                \Sigma_{\C_{\dg}(\Mod(R)), n}: \C_{\dg}(\Mod(R))(A, B)_n &\to \C_{\dg}(\Mod(R))(\Sigma(A), \Sigma(B))_n \\
                \bigtimes_{p \in \Zb} f_p &\mapsto (-1)^n \bigtimes_{p \in \Zb} f_{p + 1}.
            \end{align*}
        }
    \end{enumerate}
\end{definition}
\begin{remark}
    In order to show that \( \Sigma_{\C_{\dg}(\Mod(R))} \) from \autoref{def:sigma_c_dg} is a DG-functor, need to show the following three properties:
    \begin{enumerate}
        \item {
            Firstly, need to show that
            \[
                \Sigma_{\C_{\dg}(\Mod(R))}: \C_{\dg}(\Mod(R))(A, B) \to \C_{\dg}(\Mod(R))(\Sigma_{\C_{\dg}(\Mod(R))}(A), \Sigma_{\C_{\dg}(\Mod(R))}(B))
            \]
            is a chain morphism by checking if it commutes with the differential.

            WIP
        }
        \item {
            Secondly, need to show the functoriality of \( \Sigma_{\C_{\dg}(\Mod(R))} \).

            WIP
        }
        \item {
            Thirdly, need to show that the unit morphism commutes with \( \Sigma_{\C_{\dg}(\Mod(R))} \).

            WIP
        }
    \end{enumerate}
\end{remark}

% SRC: Berest--Mehrle 2017 LN
% TODO: Must \Ac be small in order to be well defined??
\begin{definition}[\( \Fun_{\dg}(\Ac, \Bc) \)]
    \label{def:dg_functor_category}
    Let \( \Ac \) and \( \Bc \) be DG-categories over a commutative ring with identity, \( R \). In addition, let \( \Ac \) be small.

    Then let \( \Fun_{\dg}(\Ac, \Bc) \) be the following DG-category:
    \begin{enumerate}
        \item{
            Let \( \Obj(\Fun_{\dg}(\Ac, \Bc)) \) be the class of every DG-functor from \( \Ac \) to \( \Bc \).
        }
        \item{
            For \( F, G \in \Fun_{\dg}(\Ac, \Bc) \), let \( \Fun_{\dg}(\Ac, \Bc)(F, G) \) be defined as in \cite[Proposition 6.3.1]{Borceux_1994}.
        }
        \item {
            Let composition be defined as WIP.
        }
    \end{enumerate}
\end{definition}

\begin{remark}
    The term \emph{DG-natural transformation} is often used for ``elements'' in the morphism spaces defined in \autoref{def:dg_functor_category}, but the term is also used for elements in \( Z^0(\Fun_{\dg}(\Ac, \Bc)(F, G)) \). As an example, in \cite[Definition 6.2.4, Definition 6.3.1]{Borceux_1994} the same term is used in both contexts.
\end{remark}

% TODO: Cite: Jasso--Muro
% No mention of the ring in the notation? -> Implied since DG-category is over a ring!
\begin{definition}[\( \dgMod_{\dg}(\Cc) \)]
    Let \( \Cc \) be a DG-category over \( R \).

    % TODO: Why "Right"?
    Then define the \emph{DG-category of (right) DG \( \Cc \)-modules} as
    \[
        \dgMod_{\dg}(\Cc) := \Fun_{\dg}(\Cc^{op}, \C_{\dg}(\Mod(R))).
    \]
    Objects in \( \dgMod_{\dg}(\Cc) \) are called \emph{DG-modules over \( \Cc \)}.
\end{definition}

% TODO: Does symmetry isomorphism commute with morphisms? I would guess so....
% TODO-Q: Need to show that any element in the n-th component of tensor product of two chain complexes are a ``sum'' of elementary tensors.
\begin{remark}[Functor category structure from Borceux]
    Want to see what the homomorphism structure of \( \dgMod_{\dg}(\Cc) \) is, and what properties it has from the enriched category theory perspective based on \cite{Borceux_1994}.
    
    Let \( F, G \in \dgMod_{\dg}(\Cc) \) and let
    \[
        \eta = \set{\eta_n}_{n \in \Zb} \text{ where } \eta_n \in \tuple{ \prod_{A \in \Cc^{\op}} \C_{\dg}(\Mod(R))(F(A), G(A)) }_n
    \]
    and let
    \[
        f = \set{f_n}_{n \in \Zb} \text{ where } f_i \in \Cc^{\op}(A', A'')_i.
    \]
    Construct the following composition of morphisms, where the right hand side is 
    \begin{equation}
        \label{eq:functor_category_borceux}
        \newcommand{\height}{1cm}
        %
        \mmznext{meaning to context=\height}
        \begin{tikzpicture}
            \diagram{m}{\height}{1cm} {
                \tuple{ \prod_{A \in \Cc^{\op}} \C_{\dg}(\Mod(R))(F(A), G(A)) } \otimes \Cc^{\op}(A', A'') \\
                \C_{\dg}(\Mod(R))(F(A'), G(A')) \otimes \C_{\dg}(G(A'), G(A'')) \\
                \C_{\dg}(G(A'), G(A'')) \otimes \C_{\dg}(\Mod(R))(F(A'), G(A')) \\
                \C_{\dg}(F(A'), G(A'')) \\
            };

            \draw[math]
                (m-1-1) edge node {\pi_{A'} \otimes G_{A', A''}} (m-2-1)

                (m-2-1) edge node {s} (m-3-1)

                (m-3-1) edge node {\circ} (m-4-1);
        \end{tikzpicture}
        %
        \mmznext{meaning to context=\height}
        \begin{tikzpicture}
            \diagram{m}{\height}{1cm} {
                \sum_{i + j = n} \eta_i \otimes f_j  \\
                \sum_{i + j = n} \eta_{i, A'} \otimes G(f_j) \\
                \sum_{i + j = n} (-1)^{ij} G(f_j) \otimes \eta_{i, A'} \\
                \sum_{i + j = n} (-1)^{ij} G(f_j) \circ \eta_{i, A'} \\
            };

            \path[math]
                ([yshift=-2.5mm]m-1-1.south) edge[maps to] (m-2-1)

                (m-2-1) edge[maps to] (m-3-1)

                (m-3-1) edge[maps to] (m-4-1);
        \end{tikzpicture}
    \end{equation}
    Name the entire above composition of chain complex morphisms for \( \psi_{A', A''} \).

    For any \( k, j \in \Zb \) let \( \iota_{k,j} \) denote the inclusion
    \begin{multline*}
        \iota_{k, j}: \tuple{ \prod_{A \in \Cc^{\op}} \C_{\dg}(\Mod(R))(F(A), G(A)) }_j \otimes \Cc^{\op}(A', A'')_{k - j} \\
        \hookrightarrow \tuple{ \tuple{ \prod_{A \in \Cc^{\op}} \C_{\dg}(\Mod(R))(F(A), G(A)) } \otimes \Cc^{\op}(A', A'') }_k
    \end{multline*}

    Take the adjoint of \autoref{eq:functor_category_borceux} morphism gives the chain complex morphism
    \[
        \phi_{A', A''} = \set{\phi_{A', A'', j}}_{j \in \Zb}
    \]
    where
    % \begin{align*}
    %     \phi_{A', A'', j}:  \tuple{ \prod_{A \in \Cc^{\op}} \C_{\dg}(\Mod(R))(F(A), G(A)) }_j &\to \left[ \Cc^{\op}(A', A''), \C_{\dg}(F(A'), G(A'')) \right]_j \\
    %     \eta_j &\mapsto \bigtimes_{k \in \Zb} \psi_{A', A'', k} \circ \iota_{k,j}(\eta_j \otimes ? ).
    % \end{align*}
    % Or in the same shape as previously
    \begin{center}
        \newcommand{\height}{1cm}
        %
        \mmznext{meaning to context=\height}
        \begin{tikzpicture}
            \diagram{m}{\height}{1cm} {
                \prod_{A \in \Cc^{\op}} \C_{\dg}(\Mod(R))(F(A), G(A)) \\
                \left[ \Cc^{\op}(A', A''), \C_{\dg}(F(A'), G(A'')) \right] \\
            };

            \draw[math]
                (m-1-1) edge node {\phi_{A', A''}} (m-2-1);
        \end{tikzpicture}
        %
        \mmznext{meaning to context=\height}
        \begin{tikzpicture}
            \diagram{m}{\height}{1cm} {
                \eta_j \\
                \bigtimes_{k \in \Zb} \psi_{A', A'', k} \circ \iota_{k,j}(\eta_j \otimes ? ). \\
            };

            \draw[math]
                (m-1-1) edge[maps to] (m-2-1);
        \end{tikzpicture}
    \end{center}
    And finally take the product of this map over all \( A', A'' \in \Cc^{\op} \) to get the morphism
    \begin{center}
        \newcommand{\height}{1cm}
        %
        \mmznext{meaning to context=\height}
        \begin{tikzpicture}
            \diagram{m}{\height}{1cm} {
                \prod_{A \in \Cc^{\op}} \C_{\dg}(\Mod(R))(F(A), G(A)) \\
                \prod_{A', A'' \in \Cc^{\op}} \left[ \Cc^{\op}(A', A''), \C_{\dg}(F(A'), G(A'')) \right] \\
            };

            \draw[math]
                (m-1-1) edge node {\prod_{A', A'' \in \Cc^{\op}} \phi_{A', A''}} (m-2-1);
        \end{tikzpicture}
        %
        \mmznext{meaning to context=\height}
        \begin{tikzpicture}
            \diagram{m}{\height}{1cm} {
                \eta_j \\
                \bigtimes_{A', A'' \in \Cc^{\op}} \bigtimes_{k \in \Zb} \psi_{A', A'', k} \circ \iota_{k,j}(\eta_j \otimes ? ). \\
            };

            \draw[math]
                (m-1-1) edge[maps to] (m-2-1);
        \end{tikzpicture}
    \end{center}
    Doing a similar construction as in \autoref{eq:functor_category_borceux} look at the composition
    \begin{center}
        \newcommand{\height}{1cm}
        %
        \mmznext{meaning to context=\height}
        \begin{tikzpicture}
            \diagram{m}{\height}{1cm} {
                \tuple{ \prod_{A \in \Cc^{\op}} \C_{\dg}(\Mod(R))(F(A), G(A)) } \otimes \Cc^{\op}(A', A'') \\
                \C_{\dg}(\Mod(R))(F(A''), G(A'')) \otimes \C_{\dg}(F(A'), F(A'')) \\
                \C_{\dg}(F(A'), G(A'')) \\
            };

            \draw[math]
                (m-1-1) edge node {\pi_{A''} \otimes F_{A', A''}} (m-2-1)

                (m-2-1) edge node {\circ} (m-3-1);
        \end{tikzpicture}
        %
        \mmznext{meaning to context=\height}
        \begin{tikzpicture}
            \diagram{m}{\height}{1cm} {
                \sum_{i + j = n} \eta_i \otimes f_j  \\
                \sum_{i + j = n} \eta_{i, A''} \otimes F(f_j) \\
                \sum_{i + j = n} \eta_{i, A''} \circ F(f_j) \\
            };

            \draw[math]
                (m-1-1) edge[maps to] (m-2-1)

                (m-2-1) edge[maps to] (m-3-1);
        \end{tikzpicture}
    \end{center}
    Using this as \( \widetilde{\psi}_{A', A''} \) continue the construction as before by taking the adjoint and the product to end up with the map \( \widetilde{\phi}_{A', A''} \).

    Then Borceux TODO:Cite states that \( \dgMod_{\dg}(\Cc)(F, G) \) is the equalizer of the following diagram
    \begin{center}
        \begin{tikzpicture}
            \diagram{m}{1cm}{1cm} {
                \prod_{A \in \Cc^{\op}} \C_{\dg}(\Mod(R))(F(A), G(A)) \\
                \prod_{A', A'' \in \Cc^{\op}} \left[ \Cc^{\op}(A', A''), \C_{\dg}(F(A'), G(A'')) \right] \\
            };

            \path[math] ([xshift=2.5mm]m-1-1.south) edge node {\prod\limits_{A', A'' \in \Cc^{\op}} \phi_{A', A''}} ([xshift=2.5mm]m-2-1.north);
            \draw[math] ($(m-1-1.south) + (-0.25, 0)$) to node[swap] {\prod_{A', A'' \in \Cc^{\op}} \widetilde{\phi}_{A', A''}} ($(m-2-1.north) + (-0.25, 0)$);
        \end{tikzpicture}
    \end{center}
    Then the question remains, what properties would this equalizer have?

    TODO: What is a sub chain complex? Does it make sense? 
    
    Consider a sub chain complex
    \[
        H \stackrel{\iota}{\hookrightarrow} \prod_{A \in \Cc^{\op}} \C_{\dg}(\Mod(R))(F(A), G(A))
    \]
    where the following holds
    \[
        \prod\limits_{A', A'' \in \Cc^{\op}} \phi_{A', A''} \circ \iota = \prod_{A', A'' \in \Cc^{\op}} \widetilde{\phi}_{A', A''} \circ \iota.
    \]
    Then \( H \) would for any \( j \in \Zb \) contain the elements
    \[
        \eta_j \in \tuple{ \prod_{A \in \Cc^{\op}} \C_{\dg}(\Mod(R))(F(A), G(A)) }_j = \prod_{A \in \Cc^{\op}} \C_{\dg}(\Mod(R))(F(A), G(A))_j
    \]
    such that
    \begin{align*}
        \tuple{ \prod\limits_{A', A'' \in \Cc^{\op}} \phi_{A', A''} }_j (\eta_j) &= \tuple{ \prod_{A', A'' \in \Cc^{\op}} \widetilde{\phi}_{A', A''} }_j (\eta_j) \\
        &\Updownarrow \\
        \bigtimes_{A', A'' \in \Cc^{\op}} \bigtimes_{k \in \Zb} \psi_{A', A'', k} \circ \iota_{k,j}(\eta_j \otimes ? ) &= \bigtimes_{A', A'' \in \Cc^{\op}} \bigtimes_{k \in \Zb} \widetilde{\psi}_{A', A'', k} \circ \iota_{k,j}(\eta_j \otimes ? )
        \intertext{Which are equal if for any \( A', A'' \in \Cc^{\op} \) and any \( k \in \Zb \) and any \( f \in \Cc^{\op}\tuple{A', A''}_{k - j} \) one has the following}
        \psi_{A', A'', k} \circ \iota_{k,j}(\eta_j \otimes f ) &= \widetilde{\psi}_{A', A'', k} \circ \iota_{k,j}(\eta_j \otimes f ) \\
        &\Updownarrow \\
        (-1)^{(k - j)*j}G_{A',A'',k - j}(f) \circ \eta_{j, A'} &= \eta_{j, A''} \circ F_{A', A'', k - j}(f) \\
        % &\Updownarrow \\
        % (-1)^{|f||\eta_j|}G_{A',A'',|f|}(f) \circ \eta_{j, A'} &= \eta_{j, A''} \circ F_{A', A'', |f|}(f). \\
        % EXPLANATION: There is no ``homogeneous'' element in a chain complex.
    \end{align*}

    In plain english, this means that the morphism space \( \dgMod_{\dg}(\Cc)(F, G) \) has the following properties:
    \begin{enumerate}
        \item {
            The structure of \( \dgMod_{\dg}(\Cc)(F, G) \) is as a sub chain complex of
            \[
                \prod_{A \in \Cc^{\op}} \C_{\dg}(\Mod(R))(F(A), G(A)).
            \]
            Which is a chain complex where in the \( j \)-th component, one has
            \[
                \tuple{ \prod_{A \in \Cc^{\op}} \C_{\dg}(\Mod(R))(F(A), G(A)) }_j = \prod_{A \in \Cc^{\op}} \C_{\dg}(\Mod(R))(F(A), G(A))_j
            \]
            where every element, \( \eta_j \), is in the form
            \[
                \eta_j = \bigtimes_{A \in \Cc^{\op}} \eta_{j, A}
            \]
            where
            \[
                \eta_{j, A} \in \C_{\dg}(\Mod(R))(F(A), G(A))_j = \prod_{i \in \Zb} \Mod(R)(F(A)_i, G(A_{i + j})).
            \]
            Let \( \widetilde{d_A} \) be the differential of \( \C_{\dg}(\Mod(R))(F(A), G(A)) \), then the differential of
            \[
                \prod_{A \in \Cc^{\op}} \C_{\dg}(\Mod(R))(F(A), G(A))
            \]
            is
            \begin{align*}
                \prod_{A \in \Cc^{\op}} \widetilde{d_A} : \prod_{A \in \Cc^{\op}} \C_{\dg}(\Mod(R))(F(A), G(A))_j &\to \prod_{A \in \Cc^{\op}} \C_{\dg}(\Mod(R))(F(A), G(A))_{j + 1} \\
                \bigtimes_{A \in \Cc^{\op}} \eta_{j, A} &\mapsto \bigtimes_{A \in \Cc^{\op}} \widetilde{d_A}\tuple{\eta_{j, A}} \\
                &= \bigtimes_{A \in \Cc^{\op}} \tuple{ d_{G(A)} \circ \eta_{j, A} - (-1)^j \eta_{j, A} \circ d_{F(A)} }.
            \end{align*}
        }
        \item {
            In addition, \( \dgMod_{\dg}(\Cc)(F, G) \) has the property that for any \( \eta_j \in \dgMod_{\dg}(\Cc)(F, G)_j \), and any \( f \in \Cc^{\op}(A', A'')_i \), one has that
            \[
                G_{A', A'', i}(f) \circ \eta_{j, A'} = (-1)^{ji} \eta_{j, A''} \circ F_{A', A'', i}(f).
            \]
        }
    \end{enumerate} 
\end{remark}

% TODO: Why is \Cc(-, A) a functor into \C_{\dg}(\Mod(R))?
\begin{definition}[DG Yoneda embedding]
    \label{def:DG_Yoneda_embedding}
    Let \( \Cc \) be a DG-category over \( R \).
    
    Then let \( \mathbf{h} \) be the functor defined as follows
    \begin{align*}
        \mathbf{h}: \Cc &\to \dgMod_{\dg}(\Cc) \\
        A &\mapsto \Cc(-, A)
    \end{align*}

    This functor is called the \emph{DG Yoneda embedding of \( \Cc \)}.
\end{definition}

% TODO: Add Yoneda embedding identifies \Cc with a full subcategory of \dgMod_dg(\Cc)?

% TODO: Could define this for Mod(R)-enriched categories?
\begin{definition}[0th cohomology category of a DG category]
    Let \( \Cc \) be a DG category over \( R \).

    Then let \( H^0(\Cc) \) be the following (enriched over \( \Mod(R) \) TODO) category defined as follows
    \begin{enumerate}
        \item {
            Let \( \Obj(H^0(\Cc)) := \Obj(\Cc) \).
        }
        \item {
            Let \( A, B \in H^0(\Cc) \).

            Then let \( H^0(\Cc)(A, B) := H^0(\Cc(A, B)) \).
        }
        \item {
            Let \( A, B, C \in H^0(\Cc) \) with \( f_1 \in H^0(A, B) \) and \( f_2 \in H^0(B, C) \).

            Then by \autoref{lem:massey_product_in_dg_cat/massey_product_definition/exist_lifting_h_star}, there exists \( g_1 \in \Cc(A, B) \), and \( g_2 \in \Cc(B, C) \) such that \( \class{g_1} = f_1 \) and \( \class{g_2} = f_2 \).

            % TODO: Slight abuse of notation taking a "class" of an element of a chain complex.
            % TODO: Need to show that this is well defined?
            Then let composition be defined on elementary tensors as follows
            \begin{align*}
                \circ_{H^0(\Cc)}: H^0(\Cc)(B, C) \otimes H^0(\Cc)(A, B) &\to H^0(\Cc)(A, C) \\
                f_2 \otimes f_1 &\mapsto \class{ \circ_{\Cc}(g_2 \otimes g_1) }
            \end{align*}
        }
    \end{enumerate}
\end{definition}

% TODO: What are the triangles? Is the shift functor correct on maps?
% TODO: Incorrect/abuse of notation, how does the shift work on maps?
\begin{theorem}
    Let \( \Cc \) be a DG-category over \( R \). Let \( \Sigma_{\C_{\dg}(\Mod(R))} \) be the shift functor on \( \C_{\dg}(\Mod(R)) \).

    Then \( H^0(\dgMod_{\dg}(\Cc)) \) is a triangulated category with the shift functor \( \Sigma(-) = \Sigma_{\C_{\dg}(\Mod(R))} \circ - \).
\end{theorem}
\begin{proof}
    TODO
\end{proof}

% MS-Question: Have seen definition of small category that is that the class of iso classes are small, not the class of objects. What is correct? Are they equivalent? -> Essentially small.

\begin{definition}[Acyclic DG-module]
    Let \( \Cc \) be a DG-category over a commutative ring (with identity) \( R \). Furthermore, let \( A \in \dgMod_{\dg}(\Cc) \) be a DG-module over \( \Cc \).

    Then \( A \) is called \emph{acyclic} if for any \( X \in \Cc \), one has that \( A(X) \in \C_{\dg}(\Mod(R)) \) is acyclic, i.e. \( H^*(A(X)) = 0 \).
\end{definition}

% TODO: Is \dgMod_{\dg}(\Cc) abelian?/Have kernels?
\begin{definition}[DG-projective module]
    Let \( P \in \dgMod_{\dg}(\Cc) \) be a DG-module.

    Then \( P \) is called a \emph{DG-projective module over \( \Cc \)} if:
    
    For any DG-module \( A \in \dgMod_{\dg}(\Cc) \) and any epimorphism \( f \in \dgMod_{dg}(\Cc)(A, P) \) where \( \ker(f) \in \dgMod_{\dg}(\Cc) \) is acyclic. Then \( f \) is split.
\end{definition}

% TODO: Various definitions and idiosyncracies. Which is correct?
    % Acyclic kernel? Projective objects? Spanned?
    % Krause 07 -> Compact objects, maybe more.
    % Keller 94 -> Another definition of derived DG category.
% TODO: Following def from Krause 07, but not explicitly written down. Is it correct?
\begin{definition}[Derived DG-category]
    Let \( \Cc \) be a DG-category.

    Then the \emph{derived DG-category} of \( \Cc \), denoted \( \D(\Cc) \), is defined as the full subcategory of \( H^0(\dgMod_{\dg}(\Cc)) \) spanned by the objects of \( \dgMod_{\dg}(\Cc) \) that are DG-projective.
\end{definition}

% MS-Question: What is coproduct for the derived category?
% Cite: Jasso--Muro p.31, only a statement, no proof
\begin{proposition}
    \( \D(\Cc) \) is closed under arbitrary coproduct.
\end{proposition}
\begin{proof}
    TODO
\end{proof}

% MS-Quastion: Why are small categories sometimes mentioned in def of derived category? Something to do with localization being well defined?

% MS-Question: Arbitrary coproducts <=> infinite coproducts? -> Need triangulated property for this to make sense.
% Cite: Krause 07 p. 29
\begin{definition}[Compact objects of a category]
    Let \( \Cc \) be a triangulated category with arbitrary coproduct. Let \( X \in \Cc \). 
    
    Then if \( X \) has the following property:
    
    For any index set \( I \) and any morphism \( f: X \to \coprod_{i \in I} Y_i \), there is a finite index set \( J \subseteq I \) such that \( f \) factors through \( \coprod_{j \in J} Y_j \).
    
    Then define \( X \) as a \emph{compact object in \( \Cc \)}.
\end{definition}

% TODO: Is the derived category triangulated? Need to be in order for previous def to apply.
\begin{definition}[Perfect derived DG-category \( \D^c(\Cc) \)]
    Let \( \D(\Cc) \) be the derived DG-category of \( \Cc \).

    Then define \( \D^c(\Cc) \) to be the full subcategory of \( \D(\Cc) \) consisting of all compact objects in \( \D(\Cc) \). This is called the \emph{perfect derived DG-category of \( \Cc \)}.
\end{definition}

% TODO: Cite: Jasso--Muro says so
\begin{proposition}
    \( \D^c(\Cc) \) is triangulated.
\end{proposition}
\begin{proof}
    TODO
\end{proof}

% TODO: Is this even a functor? Or well defined? Probably need to show well defined and that every morphism in H^0(A) has a representative in A.
\begin{definition}[\( H^0 \)-induced functor]
    \label{def:H^0-induced_functor}
    Let \( \Ac \) and \( \Bc \) be two DG-categories, and let \( F: \Ac \to \Bc \) be a functor between them.

    Then define the functor \( H^0(F) \) as follows:
    \begin{align*}
        H^0(F): H^0(\Ac) &\to H^0(\Bc) \\
        A &\mapsto F(A) \\
        (H^0(f): A \to B) &\mapsto (H^0(F(f)): F(A) \to F(B)) 
    \end{align*}

    This is called the \( H^0 \)-induced functor of \( F \).
\end{definition}

\begin{theorem}
    \autoref{def:H^0-induced_functor} is a well-defined functor.
\end{theorem}
\begin{proof}
    TODO
\end{proof}

% Want to show that H^0(h) has codomain D^c(\Cc)
\begin{remark}
    Let \( \mathbf{h}: \Cc \to \dgMod_{\dg}(\Cc) \) be the DG-Yoneda embedding from \autoref{def:DG_Yoneda_embedding}.

    Then for any \( A \in \Cc \), one has that \( H^0(\mathbf{h})(A) \) is both DG-projective and compact.
    
    TODO: SHOW!!!

    Therefore one has that the functor \( H^0(\mathbf{h}): H^0(\Cc) \to H^0(\dgMod_{\dg}(\Cc)) \) factors through \( \D^c(\Cc) \). Denote this functor with the same notation:
    \[
        H^0(\mathbf{h}): H^0(\Cc) \to \D^c(\Cc)
    \]
\end{remark}

\begin{remark}
    \( H^0(\mathbf{h}): H^0(\Cc) \to \D^c(\Cc) \) is fully faithful.

    TODO: Prove
\end{remark}

% TODO: Why need small? Probably something with derived.
% TODO: Heilt ordrett nesten frå Jasso--Muro 2023 p. 32, burde kanskje omformulera?
\begin{definition}[pre-triangulated DG-category]
    Let \( \Cc \) be a small DG-category.

    Then \( \Cc \) is called a \emph{pre-triangulated DG-category} if the image of the (fully faithful) functor \( H^0(\mathbf{h}): H^0(\Cc) \to \D^c(\Cc) \) is a triangulated subcategory of \( \D^c(Cc) \).
\end{definition}

\begin{definition}[Algebraic triangulated category]
    Let \( \Tc \) be a triangulated category.

    Then \( \Tc \) is called an \emph{algebraic triangulated category} if there exist a pre-triangulated DG-category, \( \Cc \), such that \( H^0(\Cc) \) is equivalent to \( \Tc \).
\end{definition}


% DUMP

% \begin{remark}
%     Consider the following homogeneous of degree \( -1 \) map for \( A \in \C_{\dg}(\Mod(R)) \)
%     \begin{center}
%         \newcommand{\height}{2cm}
%         %
%         \mmznext{meaning to context=\height}
%         \begin{tikzpicture}
%             \diagram{m}{\height}{1cm} {
%                 A: \\
%                 \Sigma(A): \\
%             };
            
%             \draw[math]
%                 (m-1-1) edge node {\sigma_A} (m-2-1);
%         \end{tikzpicture}
%         %
%         \mmznext{meaning to context=\height}
%         \begin{tikzpicture}
%             \diagramorigin{m}{\height}{3cm} {
%                 \cdots \& A_{-1} \& A_0 \& A_1 \& \cdots \\
%                 \cdots \& A_0 \& A_1 \& A_2 \& \cdots \\
%             };

%             \draw[math]
%                 (m-1-1) edge (m-1-2)
%                 (m-1-2) edge (m-1-3)
%                     edge node {\Id} (m-2-1)
%                 (m-1-3) edge (m-1-4)
%                     edge node {\Id} (m-2-2)
%                 (m-1-4) edge (m-1-5)
%                     edge node {\Id} (m-2-3)
%                 (m-1-5) edge node {\Id} (m-2-4)
                
%                 (m-2-1) edge (m-2-2)
%                 (m-2-2) edge (m-2-3)
%                 (m-2-3) edge (m-2-4)
%                 (m-2-4) edge (m-2-5);
%         \end{tikzpicture}
%     \end{center}

%     Consider this (suggestively named) homogeneous of degree \( 1 \) morphism
%     \begin{center}
%         \newcommand{\height}{2cm}
%         %
%         \mmznext{meaning to context=\height}
%         \begin{tikzpicture}
%             \diagram{m}{\height}{1cm} {
%                 \Sigma(A): \\
%                 A: \\
%             };
            
%             \draw[math]
%                 (m-1-1) edge node {\sigma_A^{-1}} (m-2-1);
%         \end{tikzpicture}
%         %
%         \mmznext{meaning to context=\height}
%         \begin{tikzpicture}
%             \diagramorigin{m}{\height}{3cm} {
%                 \cdots \& A_0 \& A_1 \& A_2 \& \cdots \\
%                 \cdots \& A_{-1} \& A_0 \& A_1 \& \cdots \\
%             };

%             \draw[math]
%                 (m-1-1) edge (m-1-2)
%                     edge node {\Id} (m-2-2)
%                 (m-1-2) edge (m-1-3)
%                     edge node {\Id} (m-2-3)
%                 (m-1-3) edge (m-1-4)
%                     edge node {\Id} (m-2-4)
%                 (m-1-4) edge (m-1-5)
%                     edge node {\Id} (m-2-5)
                
%                 (m-2-1) edge (m-2-2)
%                 (m-2-2) edge (m-2-3)
%                 (m-2-3) edge (m-2-4)
%                 (m-2-4) edge (m-2-5);
%         \end{tikzpicture}
%     \end{center}

%     One can see that
%     \[
%         \sigma_A \circ \sigma_A^{-1} = \Id_{\Sigma(A)}
%     \]
%     and
%     \[
%         \sigma_A^{-1} \circ \sigma_A = \Id_A.
%     \]
%     Therefore, \( \sigma_A^{-1} \) is the inverse of \( \sigma_A \) in \( \C_{\dg}(\Mod(R)) \).

%     Furthermore, for \( f \in \C_{\dg}(\Mod(R))(A, B) \), homogeneous of degree \( i \), consider the morphism
%     \[
%         \sigma_B \circ f \circ \sigma_A^{-1}.
%     \]
%     Looking at what the morphism is doing
%     \begin{center}
%         \newcommand{\height}{3cm}
%         %
%         \mmznext{meaning to context=\height}
%         \begin{tikzpicture}
%             \diagram{m}{\height}{1cm} {
%                 \Sigma(A): \\
%                 A: \\
%                 B: \\
%                 \Sigma(B): \\
%             };

%             \draw[math]
%                 (m-1-1) edge node {\sigma_A^{-1}} (m-2-1)

%                 (m-2-1) edge node {f} (m-3-1)

%                 (m-3-1) edge node {\sigma_B} (m-4-1);
%         \end{tikzpicture}
%         %
%         \mmznext{meaning to context=\height}
%         \begin{tikzpicture}
%             \diagramorigin{m}{\height}{3cm} {
%                 \cdots \& A_0 \& A_1 \& A_2 \& \cdots \\
%                 \cdots \& A_{-1} \& A_0 \& A_1 \& \cdots \\
%                 \cdots \& B_{i-1} \& B_i \& B_{i + 1} \& \cdots \\
%                 \cdots \& B_i \& B_{i + 1} \& B_{i + 2} \& \cdots \\
%             };

%             \draw[math]
%                 (m-1-1) edge (m-1-2)
%                     edge node {\Id} (m-2-2)
%                 (m-1-2) edge (m-1-3)
%                     edge node {\Id} (m-2-3)
%                 (m-1-3) edge (m-1-4)
%                     edge node {\Id} (m-2-4)
%                 (m-1-4) edge (m-1-5)
%                     edge node {\Id} (m-2-5)

%                 (m-2-1) edge (m-2-2)
%                 (m-2-2) edge (m-2-3)
%                     edge node {f_{-1}} (m-3-2)
%                 (m-2-3) edge (m-2-4)
%                     edge node {f_0} (m-3-3)
%                 (m-2-4) edge (m-2-5)
%                     edge node {f_1} (m-3-4)

%                 (m-3-1) edge (m-3-2)
%                 (m-3-2) edge (m-3-3)
%                     edge node {\Id} (m-4-1)
%                 (m-3-3) edge (m-3-4)
%                     edge node {\Id} (m-4-2)
%                 (m-3-4) edge (m-3-5)
%                     edge node {\Id} (m-4-3)
%                 (m-3-5) edge node {\Id} (m-4-4)

%                 (m-4-1) edge (m-4-2)
%                 (m-4-2) edge (m-4-3)
%                 (m-4-3) edge (m-4-4)
%                 (m-4-4) edge (m-4-5);
%         \end{tikzpicture}
%     \end{center}
%     % TODO: Impliserar dette at det er naturleg iso?
%     one can see that it is exacly equal to \( (-1)^i\Sigma(f) \).
% \end{remark}

% % MS-question: Remark below. -> Probably OK.
% \begin{remark}
%     % Bondal--Kapranov has as definition
%     Enriched functor between DG-categories implies it preserves differentials and grading? TODO
% \end{remark}

% SRC: Berest--Mehrle 2017 LN
% \begin{definition}[DG-natural transformation]
%     Let \( F, G: \Ac \to \Bc \) be two DG-functors.

%     Then a collection of morphisms
%     \[
%         \alpha = \set{ \alpha_A \in \Bc(F(A), G(A)) \mid A \in \Ac }
%     \]

%     TODO: Find secondary source, can't see the connection to nlab
% \end{definition}

% TODO: Need to show that it's a category?
% MS-Question: Correct? Berest--Mehrle (LN) has another def. -> Subscript dg betyr enriched.
% \begin{definition}[\( \Fun(\Ac, \Bc) \)]
%     Let \( \Ac, \Bc \) be two DG-categories over the same commutative ring \( R \).
    
%     Then let \( \Fun(\Ac, \Bc) \) denote the category of all DG-functors from \( \Ac \) to \( \Bc \), with morphisms being DG-natural transformations.
% TODO: Can't use this unless I have a definition of "DG-natural transformations".
% \end{definition}

\subsection{Why do Massey product and toda brackets intersect?}
% First need to extend massey-prod definition to H^0

% MS-Question: I Jasso--Muro så er dette berre definert for element av dgMod (Ingen subscript!)
% TODO: This is also the same shift functor that makes H^0(dgmodblabla) triangulated. Probably not neccesary to specify then.
\begin{proposition}
    \label{prop:H^i_dgmod_cong_H^0_with_shift}
    Let \( \Cc \) be a DG-category over \( R \). Let \( \Sigma \) denote the shift functor on \( H^0(\dgMod_{\dg}(\Cc)) \). And let \( A, B \in \dgMod_{\dg}(\Cc) \).

    Then there is an isomorphism
    \[
        \phi: H^i(\dgMod_{\dg}(\Cc)(A, B)) \stackrel{\sim}{\to} H^0(\dgMod_{\dg}(\Cc))(A, \Sigma^i(B)).
    \]
\end{proposition}
\begin{proof}
    TODO
\end{proof}

% TODO: Show this is well defined!!
\begin{remark}
    \label{rem:H^0_into_H^*_inclusion}
    Let \( \Cc \) be a DG-category.

    There is a dense and faithful functor \( \iota: H^0(\Cc) \hookrightarrow H^*(\Cc) \) given by
    \begin{align*}
        \iota: H^0(\Cc) &\to H^*(\Cc) \\
        A &\mapsto A \\
        \iota_{A, B}: H^0(\Cc)(A, B) &\to H^*(\Cc)(A, B) \\
        f &\mapsto \tuple{\dots, 0, f, 0, \dots} \quad \text{\( f \) is in degree \( 0 \)}
    \end{align*}

    TODO: Show well defined.
\end{remark}

\begin{remark}
    \label{rem:massey_in_H^0(dgMod_dg)}
    Let \( \Cc \) be a DG-category. Let the following be a diagram in \( H^0(\dgMod_{\dg}(\Cc)) \)
    \begin{center}
        \begin{tikzpicture}
            \diagram{m}{1cm}{1cm} {
                X_1 & X_2 & X_3 & X_4 \\
            };

            \draw[math]
                (m-1-1) edge node {f_1} (m-1-2)
                (m-1-2) edge node {f_2} (m-1-3)
                (m-1-3) edge node {f_3} (m-1-4);
        \end{tikzpicture}
    \end{center}
    Using the functor \( \iota \) from \autoref{rem:H^0_into_H^*_inclusion}, one can view the above diagram as a diagram in \( H^*(\dgMod_{\dg}(\Cc)) \) in the following manner
    \begin{center}
        \begin{tikzpicture}
            \diagram{m}{1cm}{1cm} {
                X_1 & X_2 & X_3 & X_4 \\
            };

            \draw[math]
                (m-1-1) edge node {\iota(f_1)} (m-1-2)
                (m-1-2) edge node {\iota(f_2)} (m-1-3)
                (m-1-3) edge node {\iota(f_3)} (m-1-4);
        \end{tikzpicture}
    \end{center}
    where all the maps are homogeneous of degree \( 0 \).

    % MS-Question: Overly complicated and imprecise on domains and stuff.
    By \autoref{rem:massey_product_in_dg_cat/massey_product_definition/massey_product_sum_of_degrees} the massey product of these maps, \( \massey{\iota(f_3), \iota(f_2), \iota(f_1)} \), only have non-zero components in \( H^{-1}(\dgMod_{\dg}(\Cc)(X_1, X_4)) \). Then, using the ismorphism \( \phi \) from \autoref{prop:H^i_dgmod_cong_H^0_with_shift} one has that
    \[
        \phi(\massey{\iota(f_3), \iota(f_2), \iota(f_1)}) \subseteq H^0(\dgMod_{\dg})(\Cc)(X_1, \Sigma^{-1}X_4).
    \]
    Since
    \[
        H^0(\dgMod_{\dg})(\Cc)(X_1, \Sigma^{-1}X_4) \cong H^0(\dgMod_{\dg})(\Cc)(\Sigma(X_1), X_4)
    \]
    one has the subset
    \[
        \Sigma(\phi(\massey{\iota(f_3), \iota(f_2), \iota(f_1)})) \subseteq H^0(\dgMod_{\dg})(\Cc)(\Sigma(X_1), X_4).
    \]
    Denote \( \Sigma(\phi(\massey{\iota(f_3), \iota(f_2), \iota(f_1)})) \), suggestively, as \( \massey{f_3, f_2, f_1} \).
\end{remark}

\begin{definition}[Massey product on \( H^0(\dgMod_{\dg}(\Cc)) \)]
    \label{def:massey_product_H^0(dgMod_dg(C))}
    Let \( \Cc \) be a DG-category. Let the following be a diagram in \( H^0(\dgMod_{\dg}(\Cc)) \)
    \begin{center}
        \begin{tikzpicture}
            \diagram{m}{1cm}{1cm} {
                X_1 & X_2 & X_3 & X_4 \\
            };

            \draw[math]
                (m-1-1) edge node {f_1} (m-1-2)
                (m-1-2) edge node {f_2} (m-1-3)
                (m-1-3) edge node {f_3} (m-1-4);
        \end{tikzpicture}
    \end{center}
    Let \( \massey{f_3, f_2, f_1} \subseteq H^0(\dgMod_{\dg})(\Cc)(X_1, \Sigma^{-1}X_4) \) be as in \autoref{rem:massey_in_H^0(dgMod_dg)}. This subset is called the \emph{massey product of \( f_3, f_2 \) and \( f_1 \) in \( H^0(\dgMod_{\dg}(\Cc)) \)}.
\end{definition}

% TODO: Specify that the functor is exact?
\begin{remark}
    \label{rem:massey_in_alg_tri_cat}
    Let \( \Tc \) be an algebraic triangulated category, and let the following be a diagram in \( \Tc \)
    \begin{center}
        \begin{tikzpicture}
            \diagram{m}{1cm}{1cm} {
                X_1 & X_2 & X_3 & X_4 \\
            };

            \draw[math]
                (m-1-1) edge node {f_1} (m-1-2)
                (m-1-2) edge node {f_2} (m-1-3)
                (m-1-3) edge node {f_3} (m-1-4);
        \end{tikzpicture}.
    \end{center}

    Since \( \Tc \) is algebraic, it is equivalent to \( H^0(\Cc) \) for some pre-triangulated DG-category \( \Cc \). Furthermore, since \( \Cc \) is a pre-triangulated DG-category, one has that \( H^0(\Cc) \) is equivalent to \( \im(H^0(\mathbf{h})) \). And since \( \im(H^0(\mathbf{h})) \) is a full subcategory of \( H^0(\dgMod_{\dg}(\Cc)) \), the diagram above can be looked at as a diagram in \( H^0(\dgMod_{\dg}(\Cc)) \).

    To recap, one has the relation:
    \[
        \Tc \cong H^0(\Cc) \cong \im(H^0(\mathbf{h})) \stackrel{\text{full}}{\subseteq} \D^c(\Cc) \stackrel{\text{full}}{\subseteq} \D(\Cc) \stackrel{\text{full}}{\subseteq} H^0(\dgMod_{\dg}(\Cc))
    \]
    Let \( F \) denote the functor that takes \( F: \Tc \hookrightarrow H^0(\dgMod_{\dg}(\Cc)) \). Then one has that the above diagram in \( \Tc \) can be viewed as a diagram in \( H^0(\dgMod_{\dg}(\Cc)) \) as follows
    \begin{center}
        \begin{tikzpicture}
            \diagram{m}{1cm}{1cm} {
                F(X_1) & F(X_2) & F(X_3) & F(X_4). \\
            };

            \draw[math]
                (m-1-1) edge node {F(f_1)} (m-1-2)
                (m-1-2) edge node {F(f_2)} (m-1-3)
                (m-1-3) edge node {F(f_3)} (m-1-4);
        \end{tikzpicture}
    \end{center}
    
    On the above diagram one can take the massey product (as in \autoref{def:massey_product_H^0(dgMod_dg(C))}). This yields a subset
    \[
        \massey{F(f_3), F(f_2), F(f_1)} \subseteq H^0(\dgMod_{\dg}(\Cc))(F(X_1), \Sigma^{-1}(F(X_4))).
    \]
    And since \( \im(H^0(\mathbf{h})) \) is a full subcategory of \( H^0(\dgMod_{\dg}(\Cc)) \), one has that
    \[
        \massey{F(f_3), F(f_2), F(f_1)} \subseteq \im(H^0(\mathbf{h}))(F(X_1), \Sigma^{-1}(F(X_4))).
    \]
    And since \( \Tc \cong \im(H^0(\mathbf{h})) \), there is a bijection from \( \massey{F(f_3), F(f_2), F(f_1)} \) to some \( M \subseteq \Tc(X_1, \Sigma^{-1}(X_4)) \).

    Then since \( \Tc(X_1, \Sigma^{-1}(X_4)) \cong \Tc(\Sigma(X_1), X_4) \), one can look at the subset
    \[
        \Sigma(M) \subseteq \Tc(\Sigma(X_1), X_4).
    \]
    This subset, \( \Sigma(M) \), is (suggestively) denoted as \( \massey{f_3, f_2, f_1} \).
\end{remark}

\begin{definition}[Massey product in an algebraic triangulated category]
    Let \( \Tc \) be an algebraic triangulated category, and let the following be a diagram in \( \Tc \)
    \begin{center}
        \begin{tikzpicture}
            \diagram{m}{1cm}{1cm} {
                X_1 & X_2 & X_3 & X_4 \\
            };

            \draw[math]
                (m-1-1) edge node {f_1} (m-1-2)
                (m-1-2) edge node {f_2} (m-1-3)
                (m-1-3) edge node {f_3} (m-1-4);
        \end{tikzpicture}
    \end{center}
    Let \( \massey{f_3, f_2, f_1} \subseteq \Tc(\Sigma(X_1), X_4) \) be as in \autoref{rem:massey_in_alg_tri_cat}.

    This subset is called the \emph{massey product of \( f_3, f_2 \) and \( f_1 \) in an algebraic triangulated category}.
\end{definition}

% \begin{lemma}
%     Let \( \Cc \) be a DG-category over \( R \). Let \( f \in \dgMod_{\dg}(\Cc)(F, G) \) be homogeneous of degree \( d \). Let  \( \Sigma \) denote the shift functor in \( \dgMod_{\dg}(\Cc) \).

%     Then the map
%     \begin{align*}
%         \tilde{f}: F &\to \Sigma^d(G) \\
%     \end{align*}

%     WIP
% \end{lemma}

\begin{question}[Possibility of shifting one end of a morphism in \( \dgMod_{\dg}(\Cc) \).]
    Let \( t \in \dgMod_{\dg}(\Cc)(X_1, X_3) \) be a homogeneous morphism of degree \( -1 \).

    Question: How do I create a morphism \( \tilde{t} \in \dgMod_{\dg}(\Cc)(X_1, \Sigma^{-1}(X_3)) \) such that \( \tilde{t} \) is of degree \( 0 \) (and in ``some'' sense closely resembles \( t \)).

    Since \( t \) is homogeneous of degree \( -1 \), we can capture its entire behaviour from the natural transformation \( t^{-1} \) (NB: the superscript represents the index, not that the map is an inverse). By definition, for any \( A, B \in \Cc^{\op} \) and any \( f \in \Cc^{\op}(A, B) \) the following diagram commutes
    \begin{center}
        \begin{tikzpicture}
            \diagram{m}{1cm}{1cm} {
                X_1(A) & X_3(A) \\
                X_1(B) & X_3(B) \\
            };

            \draw[math]
                (m-1-1) edge node {t_A^{-1}} (m-1-2)
                    edge node {X_1(f)} (m-2-1)
                (m-1-2) edge node {X_3(f)} (m-2-2)

                (m-2-1) edge node {t_B^{-1}} (m-2-2);
        \end{tikzpicture}.
    \end{center}

    Then, look at what \( t^{-1}_A \) does element-wise
    \[
        t^{-1}_A:
        \begin{pmatrix}
            \vdots \\
            a_1 \\
            a_0 \\
            a_{-1} \\
            \vdots
        \end{pmatrix}
        \mapsto
        \begin{pmatrix}
            \vdots \\
            (t^{-1}_A)_2(a_2) \\
            (t^{-1}_A)_1(a_1) \\
            (t^{-1}_A)_0(a_0) \\
            \vdots
        \end{pmatrix}.
    \]h
    A ``natural'' choice of \( \tilde{t}: X_1 \to \Sigma^{-1}(X_3) \) would therefore be the homogeneous of degree \( 0 \) morphism with only the following part
    \[
        \tilde{t}^0_A:
        \begin{pmatrix}
            \vdots \\
            a_1 \\
            a_0 \\
            a_{-1} \\
            \vdots
        \end{pmatrix}
        \mapsto
        \begin{pmatrix}
            \vdots \\
            (t^{-1}_A)_1(a_1) \\
            (t^{-1}_A)_0(a_0) \\
            (t^{-1}_A)_{-1}(a_{-1}) \\
            \vdots
        \end{pmatrix}.
    \]
    Which implies that \( \tilde{t}_A^0 \) is exactly the same as
    \[
        \Sigma^{-1}(\sigma_A^{-1}) \circ t_A^{-1}
    \]
    
    In order to be a valid morphism (aka. a natural transformation), by composition rules as well as definition, the outer square of the following diagram has to commute for any \( A, B \in \Cc^{\op} \) and \( f \in \Cc^{\op}(A, B) \).
    \begin{center}
        \begin{tikzpicture}
            \diagram{m}{2cm}{2cm} {
                X_1(A) & X_3(A) & \Sigma^{-1}(X_3)(A) \\
                X_1(B) & X_3(B) & \Sigma^{-1}(X_3)(B) \\
            };

            \draw[math]
            (m-1-1) edge node {t_A^{-1}} (m-1-2)
                edge node {X_1(f)} (m-2-1)
            (m-1-2) edge node {\Sigma^{-1}(\sigma_A^{-1})} (m-1-3)
                edge node {X_3(f)} (m-2-2)
            (m-1-3) edge node {\Sigma^{-1}(X_3)(f)} (m-2-3)

            (m-2-1) edge node {t_B^{-1}} (m-2-2)
            (m-2-2) edge node {\Sigma^{-1}(\sigma_B^{-1})} (m-2-3);
        \end{tikzpicture}
    \end{center}

    Since the leftmost square already commutes by definition, a sufficient result to show that the outer square commutes would be to show that the rightmost square also commutes.

    However, let \( f \in \Cc^{\op}(A, A) \) be a homogeneous map of degree \( -1 \). Then want to see if the following square could commute:
    \begin{center}
        \begin{tikzpicture}
            \diagram{m}{2cm}{2cm} {
                \Sigma(X_3)(A) & X_3(A) \\
                \Sigma(X_3)(A) & X_3(A) \\
            };

            \draw[math]
            (m-1-1) edge node {\sigma_A^{-1}} (m-1-2)
                edge node {\Sigma(X_3)(f)} (m-2-1)
            (m-1-2) edge node {X_3(f)} (m-2-2)

            (m-2-1) edge node {\sigma_A^{-1}} (m-2-2);
        \end{tikzpicture}
    \end{center}
    Let's look at the maps element-wise:
    \[
        X_3(f) \circ \sigma_A^{-1} :
        \begin{pmatrix}
            \vdots \\
            a_1 \\
            a_0 \\
            a_{-1} \\
            \vdots
        \end{pmatrix}
        \mapsto
        \begin{pmatrix}
            \vdots \\
            a_0 \\
            a_{-1} \\
            a_{-2} \\
            \vdots
        \end{pmatrix}
        \mapsto
        \begin{pmatrix}
            \vdots \\
            f_2(a_1) \\
            f_1(a_0) \\
            f_0(a_{-1}) \\
            \vdots
        \end{pmatrix}
    \]
    \[
        \sigma_A^{-1} \circ \Sigma(X_3)(f) :
        \begin{pmatrix}
            \vdots \\
            a_1 \\
            a_0 \\
            a_{-1} \\
            \vdots
        \end{pmatrix}
        \mapsto
        -
        \begin{pmatrix}
            \vdots \\
            f_1(a_0) \\
            f_0(a_{-1}) \\
            f_{-1}(a_{-2}) \\
            \vdots
        \end{pmatrix}
        \mapsto
        -
        \begin{pmatrix}
            \vdots \\
            f_2(a_1) \\
            f_1(a_0) \\
            f_0(a_{-1}) \\
            \vdots
        \end{pmatrix}
    \]
one can see that there is a sign difference between the two maps. This implies that \( \Sigma(X_3) \) is not naturally isomorphic to \( X_3 \) if any endomorphism ring contains a map of degree \( -1 \).

My question is then: How do I construct this \( \tilde{t} \in \dgMod_{\Cc}(X_1, \Sigma^{-1}(X_3)) \)?
\end{question}

\begin{theorem}
    Let \( \Cc \) be a DG-category. Let the following be a diagram in \( H^0(\dgMod_{\dg}(\Cc)) \).
    \begin{center}
        \begin{tikzpicture}
            \diagram{m}{1cm}{1cm} {
                X_1 & X_2 & X_3 & X_4 \\
            };

            \draw[math]
                (m-1-1) edge node {f_1} (m-1-2)
                (m-1-2) edge node {f_2} (m-1-3)
                (m-1-3) edge node {f_3} (m-1-4);
        \end{tikzpicture}
    \end{center}
    Then \( \toda{f_3, f_2, f_1} = (-1)^{TODO} \massey{f_3, f_2, f_1} \).
\end{theorem}
\begin{proof}
    TODO: This is a sketch of the main idea, fix it up.

    Want to prove this by showing the two inclusions \( \supseteq \) and \( \subseteq \).

    Firstly, want to show \( \supseteq \):

    Let \( f \in \massey{f_3, f_2, f_1} \) and let \( \bar{(-)} \) be as in \autoref{def:massey_product_in_dg_cat/massey_product_definition/massey_product_dg_cat}.
    
    Then by definition there exist representatives
    \[
        g_i \in \dgMod_{\dg}(X_i, X_{i + 1}), s \in \dgMod_{\dg}(X_2, X_4) \text{ and } t \in \dgMod_{\dg}(X_1, X_3)
    \]
    such that
    \[
        f = \phi\tuple{\class{\bar{s} \circ g_1 - \bar{g}_3 \circ t}}.
    \]

    Define
    \[
        \alpha := 
        \begin{pmatrix}
            
        \end{pmatrix}
    \]

    % TODO:
    % Create alpha and beta in a way that they have degree 0.
    % Show that alpha and beta are cocycles of degree 0
    % Show that alpha and beta fits into a fiber-cofiber TB definition.
\end{proof}

\begin{corollary}
    Let \( \Tc \) be an algebraic triangulated category. Furthermore let the following be a diagram in \( \Tc \).
    \begin{center}
        \begin{tikzpicture}
            \diagram{m}{1cm}{1cm} {
                X_1 & X_2 & X_3 & X_4 \\
            };

            \draw[math]
                (m-1-1) edge node {f_1} (m-1-2)
                (m-1-2) edge node {f_2} (m-1-3)
                (m-1-3) edge node {f_3} (m-1-4);
        \end{tikzpicture}
    \end{center}
    % MS-question: Massey product ser stygt ut.
    Then \( \toda{f_3, f_2, f_1} = (-1)^{TODO} \massey{f_3, f_2, f_1} \).
\end{corollary}

