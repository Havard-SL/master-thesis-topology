To get a better understanding of how to compute Toda brackets, as well as illustrate some properties they have, I have calculated the four simple examples, which will serve as a warm-up for the future more complicated examples.

First, here are some preliminary information relevant to this subsection:

Let \( R = \Fb_2C_2 \), with \( g \in C_2 \) being the generator. Then \( \Stmod(R) \) is a triangulated category.

Then \( J = \tuple{1 + g} \) is the only ideal of \( R \).

First note that \( J \) is not projective, since the short exact sequence
\begin{center}
	\begin{tikzpicture}
		\diagram{m}{1cm}{1cm}{
			J \& R \& J \\
		};
	
		\draw[math]
			(m-1-1) edge[tailed] node {\iota} (m-1-2)
			(m-1-2) edge[two headed] node {\phi} (m-1-3);
	\end{tikzpicture}
\end{center}
where \( \iota \) is the inclusion and \( \phi \) is the morphism
\begin{align*}
    \phi: R &\to J \\
    0, 1 + g &\mapsto 0 \\
    1, g &\mapsto 1 + g
\end{align*}
is not split.

This is because \( \iota \) is the only monomorphism of \( J \) into \( R \), but composes to \( 0 \) with \( \phi \). Therefore \( J \) cannot be projective, since every epimorphism into a projective module splits.

Furthermore, since \( \phi \) is an epimorphism with kernel \( J \), one gets from the third isomorphism theorem that \( \frac{R}{J} \cong J \).

The supension of \( J \) is the cokernel of \( \iota_J \). In \( \Mod(R) \), this is isomorphic to \( \frac{R}{J} \).

A sanity check:

The cone of \( \Id_J \) is the pushout of \( (\Id_J, \iota_J) \), where \( \iota_J \) is a monomorphism into the injective/projective module \( R \). By construction the pushout is \( \frac{R \oplus J}{\sim} \), where \( (0, 1+g) \sim (1+g, 0) \). One can see that this is isomorphic to \( R \). And, since \( R \) is projective, one has that \( R \cong 0 \) in \( \Stmod(R) \). Which is what is expected by a triangulated category.

The first example I calculated was the Toda bracket of the following diagram:

\begin{center}
	\begin{tikzpicture}
		\diagram{m}{1cm}{1cm}{
				J \& J \& J \& J \\
		};

		\draw[math]
			(m-1-1) edge node {\Id} (m-1-2)
			(m-1-2) edge node {0} (m-1-3)
			(m-1-3) edge node {\Id} (m-1-4);
	\end{tikzpicture}
\end{center}

The cone of \( \Id_J \) is the pushout of \( (\Id_J, \iota_J) \), where \( \iota_J \) is a monomorphism into the injective/projective module \( R \).

This is by construction \( \frac{R \oplus J}{\sim} \), where \( (0, 1+g) \sim (1+g, 0) \). This is isomorphic to \( R \).

The supension of \( J \) is the cokernel of \( \iota_J \). In \( \Mod(R) \), this is isomorphic to \( \frac{R}{J} \).

Using the cofiber-cofiber definition, one gets the following diagram:

\begin{center}
	\begin{tikzpicture}
		\diagram{m}{1cm}{1cm} {
			J \& J \& R \& {\frac{R}{J}} \\
			J \& J \& J \& J \\
		};

		\draw[math]
			(m-1-1) edge node {\Id} (m-1-2)
				edge[equal] (m-2-1)
			(m-1-2) edge (m-1-3)
				edge[equal] (m-2-2)
			(m-1-3) edge (m-1-4)
				edge node {\rho} (m-2-3)
			(m-1-4) edge node {\psi} (m-2-4)

			(m-2-1) edge node {\Id} (m-2-2)
			(m-2-2) edge node {0} (m-2-3)
			(m-2-3) edge node {\Id} (m-2-4);
	\end{tikzpicture}
\end{center}

However, since the diagram is in \( \StMod(R) \), one has that \( R \cong 0 \), and therefore \( \rho = 0 \). And from earlier one has that \( \frac{R}{J} \cong J \).

This gives the following diagram in \( \StMod(R) \):

\begin{center}
	\begin{tikzpicture}
		\diagram{m}{1cm}{1cm} {
			J \& J \& 0 \& J \\
			J \& J \& J \& J \\
		};

		\draw[math]
			(m-1-1) edge node {\Id} (m-1-2)
				edge[equal] (m-2-1)
			(m-1-2) edge node {0} (m-1-3)
				edge[equal] (m-2-2)
			(m-1-3) edge node {0} (m-1-4)
				edge node {0} (m-2-3)
			(m-1-4) edge node {\psi} (m-2-4)

			(m-2-1) edge node {\Id} (m-2-2)
			(m-2-2) edge node {0} (m-2-3)
			(m-2-3) edge node {\Id} (m-2-4);
	\end{tikzpicture}
\end{center}

But this shows that any endomorphism on \( J \) makes the rightmost square commute, and is therefore in the Toda bracket (up to pre-composition by an isomorphism \( \frac{R}{J} \to J \)) of \( \toda{\Id_J, 0, \Id_J} \). 

Toda brackets are denoted up to pre/post-composition of isomorphism, one can write that \( \toda{\Id_J, 0, \Id_J} = \StMod(R)(J, J) \).


\begin{definition} \label{def:P_M}
    Let the functor \( F \) be as in \autoref{thm:F_functor}, and let \( \phi \) be as in \autoref{rem:big_iso}. Furthermore use the notation from \autoref{rem:F_properties}.

    Let \( P_M \) be defined as follows:

    For any object \( A, B \in \Mc \), and for any \( f \in \Mc(A, B) \).

    Then \( f \in P_M  \iff \)

    All of the following are true:
    \begin{enumerate}
        \item \( F(f)_{S_a}^{S_b} = 0 \) for all \( a, b \).
        \item \( F(f)_{M_a}^{S_b} = 0 \) for all \( a, b \).
        \item \( F(f)_{M_a}^{M_b} = 0 \) for all \( a, b \).
    \end{enumerate}
\end{definition}

\begin{example}
    Let \( P_M \) be as in \autoref{def:P_M}.

    Let \( \Pc = \set{M^n \mid n \in \Nb_0} \) and let \( \Nc = P_M \).

    Then \( (\Pc, \Nc) \) is a projective class.
\end{example}
\begin{proof}
    Need to show that \( (\Pc, \Nc) \) satisfies the three properties in \autoref{def:projective_class}.

    \begin{enumerate}
        \item {
            \( (\Rightarrow) \) Let \( f \in \Nc \).

            If \( f = 0 \), then the statement is true. Therefore assume that \( f \neq 0 \).

            Then \( f \in \Mc\tuple{A, B} \) for two non-zero modules \( A, B \in \Mc \), satisfying \autoref{def:P_M}.

            Let \( \tilde{P} \in \Pc \).
            
            If \( \tilde{P} = 0 \), then the statement is true. Therefore assume that \( \tilde{P} \neq 0 \). This implies that \( \tilde{P} = M^k \) for some \( k \in \Nb \).

            Then from \autoref{thm:hom_direct_sum_map_nice} one gets the following commutative diagram
            \begin{center}
                \begin{tikzpicture}
                    \diagram{m}{1cm}{1cm} {
                        \Mc\tuple{M^k, A} \& \Mc\tuple{M^k, B} \\
                        \Mc\tuple{M, A}^k \& \Mc\tuple{M, B}^k \\
                    };

                    \draw[math]
                        (m-1-1) edge node {f_*} (m-1-2)
                            edge node[marking, above] {\sim} (m-2-1)
                        (m-1-2) edge node[marking, above] {\sim} (m-2-2)

                        (m-2-1) edge node {(f_*)^k} (m-2-2);
                \end{tikzpicture}
            \end{center}
            This implies that it suffices to check that \( f_*: \Mc\tuple{M, A} \to \Mc\tuple{M, B} \) is zero.

            Using \autoref{rem:F_properties} with \( j = 0 \) and \( k = 1 \) gives the following map
            \[
                \phi_B \circ F(f)_* \circ \phi_A^{-1} =
                \begin{pmatrix}
                    L_{M, S, S} & L_{M, M, S} \\
                    L_{M, S, M} & L_{M, M, M}
                \end{pmatrix}
            \]
            From the definition of \( \Nc \) it follows that \( L_{M, S, S} = 0, L_{M, M, S} = 0 \) and \( L_{M, M, M} = 0 \). Therefore, one can write
            \[
                \phi_B \circ F(f)_* \circ \phi_A^{-1} =
                \begin{pmatrix}
                    0 & 0 \\
                    L_{M, S, M} & 0
                \end{pmatrix}
            \]
            Looking at the defintion of \( L_{M, S, M} \) one can see
            \[
                L_{M, S, M} =
                \begin{pmatrix}
                    \tuple{ F(f)_{S_1}^{M_1} }_* &
                    \tuple{ F(f)_{S_2}^{M_1} }_* &
                    \dots &
                    \tuple{ F(f)_{S_{n_A}}^{M_1} }_* \\
                    \tuple{ F(f)_{S_1}^{M_2} }_* &
                    \tuple{ F(f)_{S_2}^{M_2} }_* &
                    \dots &
                    \tuple{ F(f)_{S_{n_A}}^{M_2} }_* \\
                    \vdots & \vdots & \ddots & \vdots \\
                    \tuple{ F(f)_{S_1}^{M_{m_B}} }_* &
                    \tuple{ F(f)_{S_2}^{M_{m_B}} }_* &
                    \dots &
                    \tuple{ F(f)_{S_{n_A}}^{M_{m_B}} }_* \\
                \end{pmatrix}
            \]
            By \autoref{thm:f_3c_3_nu} it follows that
            \[
                F(f)_{S_i}^{M_j} \in \Mc\tuple{S, M} = \set{0, \pm \nu}.
            \]
            And also by \autoref{thm:f_3c_3_mu} it follows that for any
            \[
                \alpha = \tuple{a_1, a_2, \dots, a_{n_A}, b_1, b_2, \dots, b_{m_A}} \in \Mc\tuple{M, S}^{n_A} \oplus \Mc\tuple{M, M}^{m_A}
            \]
            one has that \( a_1, a_2, \dots, a_{n_A} \in \Mc\tuple{M, S} = \set{ 0, \pm \mu } \).

            Therefore by \autoref{thm:f_3c_3_nu_circ_mu_zero} one has that
            \begin{multline*}
                \phi_B \circ F(f)_* \circ \phi_A^{-1}(\alpha) =
                \phi_B \circ F(f)_* \circ \phi_A^{-1} =
                \begin{pmatrix}
                    0 & 0 \\
                    L_{M, S, M} & 0
                \end{pmatrix}
                \tuple{a_1, a_2, \dots, a_{n_A}, b_1, b_2, \dots, b_{m_A}} \\
                =
                \begin{pmatrix}
                    0 \\
                    \vdots \\
                    0 \\
                    L_{M, S, M}
                    \begin{psmallmatrix}
                        \set{0, \pm \mu} \\
                        \vdots \\
                        \set{0, \pm \mu}
                    \end{psmallmatrix}
                \end{pmatrix}
                = 0
            \end{multline*}

            \( (\Leftarrow) \) it is easier to show this implication using a contrapositive argument.

            Assume \( f \in \Mc\tuple{A, B} \), but with \( f \not\in \Nc \).

            Then first of all \( f \neq 0 \) and so \( A, B \neq 0 \).

            Want to show that there exist some \( \tilde{P} \in \Pc \) such that
            \[
                f_*: \Mc\tuple{\tilde{P}, A} \to \Mc\tuple{\tilde{P}, B}
            \]
            is non-zero.

            Assume that \( \tilde{P} = M \).

            By \autoref{rem:F_properties} it is sufficient to show that \( \phi_B \circ F(f)_* \circ \phi_A^{-1} \) is non-zero.

            In order to prove this, split the cases up by which property in \autoref{def:P_M} that one assume that \( f \) does not fulfill:
            \begin{enumerate}
                \item {
                    Assume that \( F(f)_{S_a}^{S_b} \neq 0 \) for some \( a, b \).

                    That implies that 
                    \( 
                        L_{M, S, S} \neq 0 
                    \)
                    in the 
                    \( 
                        b, a
                    \)
                    -th coordinate.
                    Therefore for
                    \[
                        \alpha = \tuple{ 0, \dots 0, \mu, 0, \dots, 0 }
                    \]
                    a
                    \(
                        (n_a + m_a)
                    \)
                    -tuple that is entirely zero, except for in coordinate
                    \(
                        a
                    \)
                    where it is
                    \(
                        \mu
                    \).

                    Taking
                    \[
                        \phi_B \circ F(f)_* \circ \phi_A^{-1} (\alpha) =
                        \begin{pmatrix}
                            L_{M, S, S} & L_{M, M, S} \\
                            L_{M, S, M} & L_{M, M, M}
                        \end{pmatrix}
                        \tuple{ 0, \dots 0, \mu, 0, \dots, 0 }
                        = \beta
                    \]
                    one gets that the \( b \)-th coordinate of \( \beta \) has the value \( F(f)_{S_a}^{S_b} \circ \mu \).
                    
                    By assumption, \( F(f)_{S_a}^{S_b} \neq 0 \), and therefore by \autoref{lem:S-to-S} \( F(f)_{S_a}^{S_b} = \pm \Id_S \). However, this implies that \( F(f)_{S_a}^{S_b} \circ \mu = \mu \neq 0 \).
                    
                    And so \( \phi_B \circ F(f)_* \circ \phi_A^{-1} \) is non-zero.
                }
                \item {
                    Assume that there is some \( a, b \) such that \( F(f)_{M_a}^{S_b} \neq 0 \).

                    Let
                    \[
                        \alpha = \tuple{0, \dots, 0, \Id_M, 0, \dots, 0}
                    \]
                    be a \( ( n_a + m_A ) \)-tuple that is all zeroes, except for \( \Id_M \) in the \( ( n_A + a ) \)-th coordinate.

                    Then
                    \[
                        \phi_B \circ F(f)_* \circ \phi_A^{-1} (\alpha) =
                        \begin{pmatrix}
                            L_{M, S, S} & L_{M, M, S} \\
                            L_{M, S, M} & L_{M, M, M}
                        \end{pmatrix}
                        \tuple{ 0, \dots 0, \Id_M, 0, \dots, 0 }
                        = \beta
                    \]
                    where in the \( b \)-th coordinate it has the value \( F(f)_{M_a}^{S_b} \circ \Id_M = F(f)_{M_a}^{S_b} \), which is non-zero by assumption, and therefore it follows that \( f \not\in \Nc \).
                }
                \item {
                    Assume that there is some \( a, b \) such that \( F(f)_{M_a}^{M_b} \neq 0 \).

                    Let
                    \[
                        \alpha = \tuple{0, \dots, 0, \Id_M, 0, \dots, 0}
                    \]
                    be a \( ( n_a + m_A ) \)-tuple that is all zeroes, except for \( \Id_M \) in the \( ( n_A + a ) \)-th coordinate.

                    Then
                    \[
                        \phi_B \circ F(f)_* \circ \phi_A^{-1} (\alpha) =
                        \begin{pmatrix}
                            L_{M, S, S} & L_{M, M, S} \\
                            L_{M, S, M} & L_{M, M, M}
                        \end{pmatrix}
                        \tuple{ 0, \dots 0, \Id_M, 0, \dots, 0 }
                        = \beta
                    \]
                    where in the \( ( n_B + b ) \)-th coordinate it has the value \( F(f)_{M_a}^{M_b} \circ \Id_M = F(f)_{M_a}^{M_b} \), which is non-zero by assumption, and therefore it follows that \( f \not\in \Nc \).
                }
            \end{enumerate}
        }
        \item {
            \( ( \Rightarrow ) \) this is implied by point 1 \( (\Rightarrow) \).

            \( ( \Leftarrow ) \) want to show this by a contrapositive argument.

            Assume \( \tilde{P} \not\in \Pc \). Then there are some \( j \in \Nb \) and \( k \in \Nb_0 \) such that \( \tilde{P} = S^j \oplus M^k \).

            Want to show that there there exists some \( A, B, \in \Mc \) and \( f \in \Mc\tuple{A, B} \intersect \Nc \) such that
            \[
                f_*: \Mc\tuple{\tilde{P}, A} \to \Mc\tuple{\tilde{P}, B}
            \]
            is non-zero.

            Let \( A \cong S \) and \( B \cong M \) with \( F(f) = \nu \).

            Then one has that \( n_a = m_b = 1 \) and \( m_a = n_b = 0 \).

            This makes
            \begin{align*}
                \phi_B \circ F(f)_* \circ \phi_A^{-1} &=
                \begin{pmatrix}
                    L_{S, S, S} & L_{S, M, S} \\
                    L_{S, S, M } & L_{S, M, M}
                \end{pmatrix}
                \oplus
                \begin{pmatrix}
                    L_{M, S, S} & L_{M, M, S} \\
                    L_{M, S, M} & L_{M, M, M}
                \end{pmatrix} \\
                &= L_{S, S, M} \oplus L_{M, S, M} \\
                &= \tuple{\nu}_*^j \oplus \tuple{\nu}_*^k
            \end{align*}

            Let \( \alpha = \tuple{\Id_S, 0, \dots, 0} \) be a \( (j + k) \)-tuple where the only non-zero coordinate is the first one, which is \( \Id_S \).

            First of all \( \alpha \) is well-defined, since by assumption \( j > 0 \), and so the first element will always exist.

            Considering
            \[
                \phi_B \circ F(f)_* \circ \phi_A^{-1} \tuple{\alpha} =
                \tuple{\nu}_*^j \oplus \tuple{\nu}_*^k \tuple{\alpha} =
                \beta
            \]

            Where the first coordinate of \( \beta \) is
            \[
                \nu \circ \Id_S = \nu
            \]
            which is non-zero, and therefore \( \phi_B \circ F(f)_* \circ \phi_A^{-1} \neq 0 \).
        }
        \item {
            From \autoref{thm:f_3c_3_decomposition} one has that for any \( X \in \Mc \), there exist \( j, k \in \Nb \) such that \( X \cong S^j \oplus M^k \).

            First, note that the following triangle is distinguished
            \begin{center}
                \begin{tikzpicture}
                    \diagram{m}{1cm}{1cm} {
                        M \& M \& 0 \& \Sigma(M) \\
                    };

                    \draw[math]
                        (m-1-1) edge node {\Id_M} (m-1-2)
                        (m-1-2) edge (m-1-3)
                        (m-1-3) edge (m-1-4);
                \end{tikzpicture}
            \end{center}

            And taking the direct summand of a distinguished triangle is distinguished, therefore the following triangle is also distinguished
            \begin{center}
                \begin{tikzpicture}
                    \diagram{m}{1cm}{1cm} {
                        M^k \& M^k \& 0 \& \Sigma(M)^k \\
                    };

                    \draw[math]
                        (m-1-1) edge node {\tuple{\Id_M}^k} (m-1-2)
                        (m-1-2) edge (m-1-3)
                        (m-1-3) edge (m-1-4);
                \end{tikzpicture}
            \end{center}

            By \autoref{lem:s_m_s_distinguished} one has the following distinguished triangle
            \begin{center}
                \begin{tikzpicture}
                    \diagram{m}{1cm}{1cm} {
                        S \& M \& S \& \Sigma(S) \\
                    };

                    \draw[math]
                        (m-1-1) edge node {\nu} (m-1-2)
                        (m-1-2) edge node {\mu} (m-1-3)
                        (m-1-3) edge node {\nu} (m-1-4);
                \end{tikzpicture}
            \end{center}

            Shifting the above triangle yields the following distinguished triangle
            \begin{center}
                \begin{tikzpicture}
                    \diagram{m}{1cm}{1cm} {
                        \Sigma^{-1}(S) \& S \& M \& S \\
                    };

                    \draw[math]
                        (m-1-1) edge node {-\mu} (m-1-2)
                        (m-1-2) edge node {\nu} (m-1-3)
                        (m-1-3) edge node {\mu} (m-1-4);
                \end{tikzpicture}
            \end{center}

            Taking the direct summand of this distinguished triangle with itself \( j \) times, as well as using \autoref{lem:sigma_switch_s_m} by identifying \( \Sigma^{-1}\tuple{S} \cong M \) and \( S \cong \Sigma\tuple{M} \), yields the following distinguished triangle
            \begin{center}
                \begin{tikzpicture}
                    \diagram{m}{1cm}{1cm} {
                        M^j \& S^j \& M^j \& \Sigma(M)^j \\
                    };

                    \draw[math]
                        (m-1-1) edge node {\tuple{-\mu}^j} (m-1-2)
                        (m-1-2) edge node {\tuple{\nu}^j} (m-1-3)
                        (m-1-3) edge node {\tuple{\mu}^j} (m-1-4);
                \end{tikzpicture}
            \end{center}

            % TODO: Refrence the tikzpictures
            And finally, taking the direct summand of the two biggest distinguished triangles yields the following distinguished triangle
            \begin{center}
                \begin{tikzpicture}
                    \diagram{m}{1cm}{2cm} {
                        M^{j + k} \& S^j \oplus M^k \& M^j \& \Sigma(M)^{j + k} \\
                    };

                    \draw[math]
                        (m-1-1) edge node {\tuple{-\mu}^j \oplus \tuple{\Id_M}^k} (m-1-2)
                        (m-1-2) edge node {
                            \begin{psmallmatrix}
                                \tuple{\nu}^j & 0
                            \end{psmallmatrix}
                            } (m-1-3)
                        (m-1-3) edge node {
                            \begin{psmallmatrix}
                                \tuple{\mu}^j \\
                                0
                            \end{psmallmatrix}
                            } (m-1-4);
                \end{tikzpicture}
            \end{center}
            Need to check if this triangle satisfies the conditions.

            Firstly, \( M^{j + k} \) is in \( \Pc \).

            Secondly, need to check if \( 
                \begin{psmallmatrix}
                    \tuple{\nu}^j & 0
                \end{psmallmatrix}
            \) is in \( \Nc \).

            Uisng \autoref{rem:phi_and_L_connection}, it follows that
            \[
                \phi\tuple{
                    \begin{psmallmatrix}
                        \tuple{\nu}^j & 0
                    \end{psmallmatrix}
                }
                =
                \begin{psmallmatrix}
                    0 \\
                    \vdots \\
                    0 \\
                    \nu \\
                    \vdots \\
                    \nu \\
                    0 \\
                    \vdots \\
                    0 \\
                    0 \\
                    \vdots \\
                    0
                \end{psmallmatrix}
            \]
            which satisfies the criteria that any map in \( \Nc \) need to follow.
        }
    \end{enumerate}
    
    Therefore \( \tuple{\Pc, \Nc} \) is a projective class.
\end{proof}

Let \( \toda{f_3, 0, \Id} \) be a well defined Toda bracket. Then one has the following diagram:

\begin{center}
	\begin{tikzpicture}
		\diagram{m}{1cm}{1cm} {
			{X_1} \& {X_1} \& 0 \& {\Sigma X_1} \\
			{X_1} \& {X_1} \& {X_2} \& {X_3} \\
		};

		\draw[math]
			(m-1-1) edge node {\Id} (m-1-2)
				edge[equal] (m-2-1)
			(m-1-2) edge (m-1-3)
				edge[equal] (m-2-2)
			(m-1-3) edge (m-1-4)
				edge (m-2-3)
			(m-1-4) edge node {\phi} (m-2-4)

			(m-2-1) edge node {\Id} (m-2-2)
			(m-2-2) edge node {0} (m-2-3)
			(m-2-3) edge node {f_3} (m-2-4);
	\end{tikzpicture}
\end{center}

Here one has that any possible \( \psi: \Sigma X_1 \to X_3 \) will make the right square commute. Therefore \( \toda{f_3, 0, \Id} = \Tc(\Sigma X_1, X_3) \).


Want to find the Toda bracket of the following maps using the cofiber-cofiber definition:

\begin{center}
	\begin{tikzpicture}
		\diagram{m}{1cm}{1cm} {
			J \& {J \oplus J} \& {J \oplus J} \& J \\
		};

		\draw[math]
			(m-1-1) edge node {\begin{psmallmatrix} 1 \\ 1 \end{psmallmatrix}} (m-1-2)
			(m-1-2) edge node {\begin{psmallmatrix} 1 & 1 \\ 1 & 1 \end{psmallmatrix}} (m-1-3)
			(m-1-3) edge node {\begin{psmallmatrix} 1 & 1 \end{psmallmatrix}} (m-1-4);
	\end{tikzpicture}
\end{center}

First need to find the standard triangle of \( \begin{psmallmatrix} 1 \\ 1 \end{psmallmatrix}: J \to J \oplus J \):

The cone is defined as the pushout of the following diagram:

\begin{center}
	\begin{tikzpicture}
		\diagram{m}{1cm}{1cm} {
			J \& R \\
			J \oplus J \\
		};

		\draw[math]
			(m-1-1) edge[hook] node {\iota_J} (m-1-2)
				edge[swap] node {\begin{psmallmatrix} 1 \\ 1 \end{psmallmatrix}} (m-2-1);
	\end{tikzpicture}
\end{center}

Where \( \iota_J \) is a monomorphism into a injective module. Here, this map is chosen to be the only monomorphism from \( J \to R \), namely \( \iota \).

In the category of modules, the pushout becomes:

\begin{center}
	\begin{tikzpicture}
		\diagram{m}{1cm}{1cm} {
			J \& R \\
			J \oplus J \& \frac{J \oplus J \oplus R}{\sim} \\
		};

		\draw[math]
			(m-1-1) edge[hook] node {\iota_J} (m-1-2)
				edge[swap] node {\begin{psmallmatrix} 1 \\ 1 \end{psmallmatrix}} (m-2-1)
			(m-1-2) edge node {\gamma} (m-2-2)

			(m-2-1) edge[hook] node {\rho} (m-2-2);
	\end{tikzpicture}
\end{center}

Where \( (1 + g, 1 + g, 0) \sim (0, 0, 1 + g) \).

The map \( \rho \) is given as the composition:
\begin{center}
	\begin{tikzpicture}
		\diagram{m}{1cm}{1cm} {
			J \oplus J \& J \oplus J \oplus R \& \frac{J \oplus J \oplus R}{\sim} \\
		};

		\draw[math]
			(m-1-1) edge[hook] node {i} (m-1-2)
			(m-1-2) edge[two heads] node {\pi} (m-1-3);
	\end{tikzpicture}
\end{center}

Where \( i \) is the embedding, and \( \pi \) is the quotient epimorphism.

One can check that the pushout is isomorphic to \( J \oplus R \) via the map:

\begin{align*}
	\alpha: \frac{J \oplus J \oplus R}{\sim} &\to J \oplus R \\
	(0, 0, r) &\mapsto (0, r) \\
	(1 + g, 0, r) &\mapsto (1 + g, r) \\
	(0, 1 + g, r) &\mapsto (1 + g, r) \\
	(1+ + g, 1 + g, r) &\mapsto (0, r) \\
\end{align*}

Therefore, by checking the map \( \alpha \circ \rho \) one can see that it becomes the map:
\[ 
	(\begin{psmallmatrix}
		1 & 1 \\
	\end{psmallmatrix}, 0):  J \oplus J \to J \oplus R
\]

Doing a similar argument for \( \gamma \), one gets that it is simply the embedding \( \pi \).

Furthermore, the map \( J \oplus R \to \Sigma(J) \cong \frac{R}{J} \) is given as the unique pushout map \( \beta \), satisfying the following commutative diagram:

\begin{center}
	\begin{tikzpicture}
		\diagram{m}{1cm}{2cm} {
			J \& R \& \frac{R}{J} \\
			J \oplus J \& J \oplus R \\
		};

		\draw[math]
			(m-1-1) edge[hook] node {\iota_J} (m-1-2)
				edge[swap] node {\begin{psmallmatrix} 1 \\ 1 \end{psmallmatrix}} (m-2-1)
			(m-1-2) edge[two heads] node {\pi_J} (m-1-3)
				edge node {\pi} (m-2-2)

			(m-2-1) edge[hook] node {(\begin{psmallmatrix} 1 & 1 \\ \end{psmallmatrix}, 0)} (m-2-2)
				edge[curve={height=60pt}] node {0} (m-1-3)
			(m-2-2) edge node {\beta} (m-1-3);
	\end{tikzpicture}
\end{center}

One can check that a candidate for the map \( \beta \) is \( (0, \pi) \). And since \( \beta \) is unique, this is the only map making the diagram commute.

Therefore the standard triangle becomes:

\begin{center}
	\begin{tikzpicture}
		\diagram{m}{1cm}{2cm} {
			J \& J \oplus J \& J \oplus R \& \frac{R}{J} \\
		};

		\draw[math]
			(m-1-1) edge node {\begin{psmallmatrix} 1 \\ 1 \end{psmallmatrix}} (m-1-2)
			(m-1-2) edge node {(\begin{psmallmatrix} 1 & 1 \\ \end{psmallmatrix}, 0)} (m-1-3)
			(m-1-3) edge node {(0, \pi)} (m-1-4);
	\end{tikzpicture}
\end{center}

However, since the objects and morphisms are in \( \StMod(R) \), one has that \( R \cong 0 \), and by using the isomorphism \( \frac{R}{J} \cong J \), it becomes:

\begin{center}
	\begin{tikzpicture}
		\diagram{m}{1cm}{1cm} {
			J \& J \oplus J \& J \& J \\
		};

		\draw[math]
			(m-1-1) edge node {\begin{psmallmatrix} 1 \\ 1 \end{psmallmatrix}} (m-1-2)
			(m-1-2) edge node {\begin{psmallmatrix} 1 & 1 \\ \end{psmallmatrix}} (m-1-3)
			(m-1-3) edge node {0} (m-1-4);
	\end{tikzpicture}
\end{center}

Using the cofiber-cofiber definition, one gets the following commutative diagram:

\begin{center}
	\begin{tikzpicture}
		\diagram{m}{1cm}{1cm} {
			J \& {J \oplus J} \& J \& J \\
			J \& {J \oplus J} \& {J \oplus J} \& J \\
		};

		\draw[math]
			(m-1-1) edge node {\begin{psmallmatrix} 1 \\ 1 \end{psmallmatrix}} (m-1-2)
				edge[equal] (m-2-1)
			(m-1-2) edge node {{\begin{psmallmatrix} 1 & 1 \end{psmallmatrix}}} (m-1-3)
				edge[equal] (m-2-2)
			(m-1-3) edge node {0} (m-1-4)
				edge node {\phi} (m-2-3)
			(m-1-4) edge node {\psi} (m-2-4)

			(m-2-1) edge node {\begin{psmallmatrix} 1 \\ 1 \end{psmallmatrix}} (m-2-2)
			(m-2-2) edge node {\begin{psmallmatrix} 1 & 1 \\ 1 & 1 \end{psmallmatrix}} (m-2-3)
			(m-2-3) edge node {\begin{psmallmatrix} 1 & 1 \end{psmallmatrix}} (m-2-4);
	\end{tikzpicture}
\end{center}

Where the top row is distinguished.

Firstly, using the fact that \( \begin{psmallmatrix}
		1 & 1 \\
	\end{psmallmatrix}: J \oplus J \to J \) is an epimorphism, one gets that \( \phi \circ \begin{psmallmatrix}
		1 & 1 \\
	\end{psmallmatrix} = \begin{psmallmatrix}
		1 \\ 1 \\
	\end{psmallmatrix} \begin{psmallmatrix}
		1 & 1 \\
	\end{psmallmatrix} \implies \phi = \begin{psmallmatrix}
		1 \\ 1 \\
	\end{psmallmatrix} \) by the epimorphism property.

And secondly one has that \( \begin{psmallmatrix}
		1 & 1 \\
	\end{psmallmatrix} \circ \phi = \begin{psmallmatrix}
		1 & 1 \\
	\end{psmallmatrix} \circ \begin{psmallmatrix}
		1 \\ 1 \\
	\end{psmallmatrix} = 0 \), so the Toda-bracket is non-empty.

Finally one can see that for any \( \psi: J \to J \), the right square will commute, and so the toda bracket becomes \( \StMod(R)(J, J) \).

