Want to find the Toda bracket of the following maps using the cofiber-cofiber definition:

\begin{center}
	\begin{tikzpicture}
		\diagram{m}{1cm}{1cm} {
			J \& {J \oplus J} \& {J \oplus J} \& J \\
		};

		\draw[math]
			(m-1-1) edge node {\begin{psmallmatrix} 1 \\ 1 \end{psmallmatrix}} (m-1-2)
			(m-1-2) edge node {\begin{psmallmatrix} 1 & 1 \\ 1 & 1 \end{psmallmatrix}} (m-1-3)
			(m-1-3) edge node {\begin{psmallmatrix} 1 & 1 \end{psmallmatrix}} (m-1-4);
	\end{tikzpicture}
\end{center}

First need to find the standard triangle of \( \begin{psmallmatrix} 1 \\ 1 \end{psmallmatrix}: J \to J \oplus J \):

The cone is defined as the pushout of the following diagram:

\begin{center}
	\begin{tikzpicture}
		\diagram{m}{1cm}{1cm} {
			J \& R \\
			J \oplus J \\
		};

		\draw[math]
			(m-1-1) edge[tailed] node {\iota_J} (m-1-2)
				edge[swap] node {\begin{psmallmatrix} 1 \\ 1 \end{psmallmatrix}} (m-2-1);
	\end{tikzpicture}
\end{center}

Where \( \iota_J \) is a monomorphism into a injective module. Here, this map is chosen to be the only monomorphism from \( J \to R \), namely \( \iota \).

In the category of modules, the pushout becomes:

\begin{center}
	\begin{tikzpicture}
		\diagram{m}{1cm}{1cm} {
			J \& R \\
			J \oplus J \& \frac{J \oplus J \oplus R}{\sim} \\
		};

		\draw[math]
			(m-1-1) edge[tailed] node {\iota_J} (m-1-2)
				edge[swap] node {\begin{psmallmatrix} 1 \\ 1 \end{psmallmatrix}} (m-2-1)
			(m-1-2) edge node {\gamma} (m-2-2)

			(m-2-1) edge[tailed] node {\rho} (m-2-2);
	\end{tikzpicture}
\end{center}

Where \( (1 + g, 1 + g, 0) \sim (0, 0, 1 + g) \).

The map \( \rho \) is given as the composition:
\begin{center}
	\begin{tikzpicture}
		\diagram{m}{1cm}{1cm} {
			J \oplus J \& J \oplus J \oplus R \& \frac{J \oplus J \oplus R}{\sim} \\
		};

		\draw[math]
			(m-1-1) edge[tailed] node {i} (m-1-2)
			(m-1-2) edge[two headed] node {\pi} (m-1-3);
	\end{tikzpicture}
\end{center}

Where \( i \) is the embedding, and \( \pi \) is the quotient epimorphism.

One can check that the pushout is isomorphic to \( J \oplus R \) via the map:

\begin{align*}
	\alpha: \frac{J \oplus J \oplus R}{\sim} &\to J \oplus R \\
	(0, 0, r) &\mapsto (0, r) \\
	(1 + g, 0, r) &\mapsto (1 + g, r) \\
	(0, 1 + g, r) &\mapsto (1 + g, r) \\
	(1+ + g, 1 + g, r) &\mapsto (0, r) \\
\end{align*}

Therefore, by checking the map \( \alpha \circ \rho \) one can see that it becomes the map:
\[ 
	(\begin{psmallmatrix}
		1 & 1 \\
	\end{psmallmatrix}, 0):  J \oplus J \to J \oplus R
\]

Doing a similar argument for \( \gamma \), one gets that it is simply the embedding \( \pi \).

Furthermore, the map \( J \oplus R \to \Sigma J \cong \frac{R}{J} \) is given as the unique pushout map \( \beta \), satisfying the following commutative diagram:

\begin{center}
	\begin{tikzpicture}
		\diagram{m}{1cm}{2cm} {
			J \& R \& \frac{R}{J} \\
			J \oplus J \& J \oplus R \\
		};

		\draw[math]
			(m-1-1) edge[tailed] node {\iota_J} (m-1-2)
				edge[swap] node {\begin{psmallmatrix} 1 \\ 1 \end{psmallmatrix}} (m-2-1)
			(m-1-2) edge[two headed] node {\pi_J} (m-1-3)
				edge node {\pi} (m-2-2)

			(m-2-1) edge[tailed] node {(\begin{psmallmatrix} 1 & 1 \\ \end{psmallmatrix}, 0)} (m-2-2)
				edge[curve={height=60pt}] node {0} (m-1-3)
			(m-2-2) edge node {\beta} (m-1-3);
	\end{tikzpicture}
\end{center}

One can check that a candidate for the map \( \beta \) is \( (0, \pi) \). And since \( \beta \) is unique, this is the only map making the diagram commute.

Therefore the standard triangle becomes:

\begin{center}
	\begin{tikzpicture}
		\diagram{m}{1cm}{2cm} {
			J \& J \oplus J \& J \oplus R \& \frac{R}{J} \\
		};

		\draw[math]
			(m-1-1) edge node {\begin{psmallmatrix} 1 \\ 1 \end{psmallmatrix}} (m-1-2)
			(m-1-2) edge node {(\begin{psmallmatrix} 1 & 1 \\ \end{psmallmatrix}, 0)} (m-1-3)
			(m-1-3) edge node {(0, \pi)} (m-1-4);
	\end{tikzpicture}
\end{center}

However, since the objects and morphisms are in \( \StMod(R) \), one has that \( R \cong 0 \), and by using the isomorphism \( \frac{R}{J} \cong J \), it becomes:

\begin{center}
	\begin{tikzpicture}
		\diagram{m}{1cm}{1cm} {
			J \& J \oplus J \& J \& J \\
		};

		\draw[math]
			(m-1-1) edge node {\begin{psmallmatrix} 1 \\ 1 \end{psmallmatrix}} (m-1-2)
			(m-1-2) edge node {\begin{psmallmatrix} 1 & 1 \\ \end{psmallmatrix}} (m-1-3)
			(m-1-3) edge node {0} (m-1-4);
	\end{tikzpicture}
\end{center}

Using the cofiber-cofiber definition, one gets the following commutative diagram:

\begin{center}
	\begin{tikzpicture}
		\diagram{m}{1cm}{1cm} {
			J \& {J \oplus J} \& J \& J \\
			J \& {J \oplus J} \& {J \oplus J} \& J \\
		};

		\draw[math]
			(m-1-1) edge node {\begin{psmallmatrix} 1 \\ 1 \end{psmallmatrix}} (m-1-2)
				edge[equality] (m-2-1)
			(m-1-2) edge node {{\begin{psmallmatrix} 1 & 1 \end{psmallmatrix}}} (m-1-3)
				edge[equality] (m-2-2)
			(m-1-3) edge node {0} (m-1-4)
				edge node {\phi} (m-2-3)
			(m-1-4) edge node {\psi} (m-2-4)

			(m-2-1) edge node {\begin{psmallmatrix} 1 \\ 1 \end{psmallmatrix}} (m-2-2)
			(m-2-2) edge node {\begin{psmallmatrix} 1 & 1 \\ 1 & 1 \end{psmallmatrix}} (m-2-3)
			(m-2-3) edge node {\begin{psmallmatrix} 1 & 1 \end{psmallmatrix}} (m-2-4);
	\end{tikzpicture}
\end{center}

Where the top row is distinguished.

Firstly, using the fact that \( \begin{psmallmatrix}
		1 & 1 \\
	\end{psmallmatrix}: J \oplus J \to J \) is an epimorphism, one gets that \( \phi \circ \begin{psmallmatrix}
		1 & 1 \\
	\end{psmallmatrix} = \begin{psmallmatrix}
		1 \\ 1 \\
	\end{psmallmatrix} \begin{psmallmatrix}
		1 & 1 \\
	\end{psmallmatrix} \implies \phi = \begin{psmallmatrix}
		1 \\ 1 \\
	\end{psmallmatrix} \) by the epimorphism property.

And secondly one has that \( \begin{psmallmatrix}
		1 & 1 \\
	\end{psmallmatrix} \circ \phi = \begin{psmallmatrix}
		1 & 1 \\
	\end{psmallmatrix} \circ \begin{psmallmatrix}
		1 \\ 1 \\
	\end{psmallmatrix} = 0 \), so the Toda-bracket is non-empty.

Finally one can see that for any \( \psi: J \to J \), the right square will commute, and so the toda bracket becomes \( \StMod(R)(J, J) \).
