\begin{definition}[Projective class, projective object] \label{def:projective_class}
    Let \( \Tc \) be a triangulated category.

    Let \( (\Pc, \Nc) \) be a tuple where \( \Pc \) is a class of objects, and \( \Nc \) is a class of morphisms, satisfying the following properties:

    \begin{enumerate}
        \item {Let \( f \in \Tc(X, Y) \).
        
        Then \( f \in \Nc \) if and only if for all \( P \in \Pc \) one has that \( f_*: \Tc(P, X) \to \Tc(P, Y) \) is the zero map.}

        \item {Let \( P \in \Tc \).
        
        Then \( P \in \Pc \) if and only if for all \( f \in \Tc(X, Y) \intersect \Nc \), one has that \( f_*: \Tc(P, X) \to \Tc(P, Y) \) is the zero map.}

        \item {For every \( X \in \Tc \) there exists objects \( Y \in \Tc \) and \( P \in \Pc \) along with a morphism \( f \in \Tc(X, Y) \intersect \Nc \) such that there exists a distinguished triangle on the form \( P \to X \stackrel{f}{\to} Y \to \Sigma P \).}
    \end{enumerate}

    Then \( (\Pc, \Nc) \) is called a \emph{projective class in \( \Tc \)}, and an object \( P \in \Pc \) is called \emph{projective}. % Projective over a projective class? TODO
\end{definition}

\begin{example}
    Let \( \Tc \) be any triangulated category. Let \( \Pc = \set{0} \), and \( \Nc = \) every morphism in \( \Tc \).

    Then \( (\Pc, \Nc) \) is a projective class in \( \Tc \).
\end{example}
\begin{proof}
    Check every property:
    \begin{enumerate}
        \item {
            For any morphism \( f \in \Tc(X, Y) \) one has that \( \Tc(0, Y) \) is the just the zero map, and so the statement holds.
        }
        \item {
            \( (\Leftarrow) \) For any object \( P \in \Tc \) one has that \( f = \Id_P \in \Tc(P, P) \), but \( (\Id_P)_*: \Tc(P, P) \to \Tc(P, P) \) is the zero map if and only if \( P = 0 \).

            \( (\Rightarrow) \) If \( P = 0 \), then for any \( f \in \Tc(X, Y) \), one has that \( \Tc(0, Y) \) is the zero map, so the statement holds. 
        }
        \item {
            Let \( X \in \Tc \), then the (right-shifted) trivial triangle
            \begin{center}
                \begin{tikzpicture}
                    \diagram{m}{1cm}{1cm} {
                        0 \& X \& X \& 0 \\
                    };

                    \draw[math]
                        (m-1-1) edge (m-1-2)
                        (m-1-2) edge node {\Id_X} (m-1-3)
                        (m-1-3) edge (m-1-4);
                \end{tikzpicture}
            \end{center}
            satisfies this property.
        }
    \end{enumerate}
\end{proof}

\begin{remark}
    For any projective class \( ( \Pc, \Nc ) \) in \( \Tc \), one must have \( 0 \in \Pc \), since it always satisfies property 2 in \autoref{def:projective_class}, and \( 0 \in \Nc \), since it always satisfies property 1.
\end{remark}

\begin{definition}[Stable projective class]
    Let \( (\Pc, \Nc) \) be a projective class in \( \Tc \).

    If for any \( P \in \Pc \) and \( n \in \Zb \) one has that \( \Sigma^n P \in \Pc \).
    
    Then \( (\Pc, \Nc) \) is called a \emph{stable projective class}.
\end{definition}

% Projective class stable under coproduct and retracts? TODO

\begin{definition}[\( \Pc \)-epic, \( \Pc \)-monic]
    Let \( f \in \Tc(X, Y) \) and let \( (\Pc, \Nc) \) be a projective class in \( \Tc \).

    Then one has the following definitions:

    % surjective or epimorphism? Injective or monomorphism?
    \begin{enumerate}
        \item {If for all \( P \in \Pc \), one has that \( f_*: \Tc(P, X) \to \Tc(P, Y) \) is surjective.
        
        Then \( f \) is called \emph{\( \Pc \)-epic}.}
        
        \item {If for all \( P \in \Pc \), one has that \( f_*: \Tc(P, X) \to \Tc(P, Y) \) is injective.
        
        Then \( f \) is called \emph{\( \Pc \)-monic}.}
    \end{enumerate}
\end{definition}

% Equivalent to cofiber map being P-null, or fiber map being P-null. TODO

\begin{definition}[Injective class, injective object]
    Let \( \Tc \) be a triangulated category.

    If the tuple \( (\Ic, \Nc) \) is a projective class in \( \Tc^{op} \).
    
    Then \( (\Ic, \Nc) \) is called an \emph{injective class in \( \Tc \)}, and an object \( I \in \Ic \) is called \emph{injective}.
\end{definition}

% Explicit definition. TODO

% Stable under products and retracts? TODO.

\begin{definition}[Stable injective class]
    Let \( (\Ic, \Nc) \) be an injective class in \( \Tc \).

    If for any \( I \in \Ic \) and \( n \in \Zb \) one has that \( \Sigma^n I \in \Ic \).
    
    Then \( (\Ic, \Nc) \) is called a \emph{stable injective class}.
\end{definition}

\begin{definition}[\( \Ic \)-monic, \( \Ic \)-epic]
    Let \( f \in \Tc(X, Y) \) and let \( (\Ic, \Nc) \) be an injective class in \( \Tc \).

    Then one has the following definitions:

    \begin{enumerate} % surjective or epimorphism? Injective or monomorphism?
        \item {If for all \( I \in \Ic \), one has that \( f_*: \Tc(I, X) \to \Tc(I, Y) \) is surjective.
        
        Then \( f \) is called \emph{\( \Ic \)-monic}.}
        
        \item {If for all \( I \in \Ic \), one has that \( f_*: \Tc(I, X) \to \Tc(I, Y) \) is injective.
        
        Then \( f \) is called \emph{\( \Ic \)-epic}.}
    \end{enumerate}
\end{definition}

% Equivalent to fiber map being I-null, or cofiber map being I-null. TODO

% Remark -> Connection to lifitng and extension property.

% Convention neccesary?

\begin{definition}[Adams resolution w.r.t. a projective class] \label{def:adams_resolution_projective_class} % Need stable? TODO
    Let \( (\Pc, \Nc) \) be a projective class in \( \Tc \). Let \( X_0 \in \Tc \).

    Then given objects and morphisms in \( \Tc \) fitting in the following diagram:

    \begin{center}
        \begin{tikzpicture}
            \diagram{m}{1cm}{1cm} {
                X_0 \&\& X_1 \&\& X_2 \&\& X_3 \& \dots \\
                \& P_0 \&\& P_1 \&\& P_2 \\
            };

            \draw[math]
                (m-1-1) edge node {i_0} (m-1-3)
                (m-1-3) edge node {i_1} (m-1-5)
                    edge[suspension] node {\delta_0} (m-2-2)
                (m-1-5) edge node {i_2} (m-1-7)
                    edge[suspension] node {\delta_1} (m-2-4)
                (m-1-7) edge (m-1-8)
                    edge[suspension] node {\delta_2} (m-2-6)

                (m-2-2) edge node {p_0} (m-1-1)
                (m-2-4) edge node {p_1} (m-1-3)
                (m-2-6) edge node {p_2} (m-1-5);
        \end{tikzpicture}
    \end{center}

    Where every \( i_n \in \Nc \), and \( P_n \in \Pc \), and marked arrows denote degree shifting maps, with \( \delta_n \in \Tc(X_{n + 1}, \Sigma P_n ) \). And where every triangle on the form:

    \begin{center}
        \begin{tikzpicture}
            \diagram{m}{1cm}{1cm} {
                P_n \& X_n \& X_{n + 1} \& \Sigma P_n \\
            };

            \draw[math]
                (m-1-1) edge node {p_n} (m-1-2)
                (m-1-2) edge node {i_n} (m-1-3)
                (m-1-3) edge node {\delta_n} (m-1-4);
        \end{tikzpicture}
    \end{center}

    is distinguished.

    This is called an \emph{Adams resolution of \( X_0 \) with respect to a projective class \( (\Pc, \Nc) \)}.
\end{definition}

\begin{definition}[Adams resolution w.r.t. an injective class] \label{def:adams_resolution_injective_class} % Need stable? TODO
    Let \( (\Ic, \Nc) \) be an injective class in \( \Tc \). Let \( Y_0 \in \Tc \).

    Then given objects and morphisms in \( \Tc \) fitting in the following diagram:

    \begin{center}
        \begin{tikzpicture}
            \diagram{m}{1cm}{1cm} {
                Y_0 \&\& Y_1 \&\& Y_2 \&\& Y_3 \& \dots \\
                \& I_0 \&\& I_1 \&\& I_2 \\
            };

            \draw[math]
                (m-1-1) edge[swap] node {p_0} (m-2-2)
                (m-1-3) edge[swap] node {i_0} (m-1-1)
                    edge[swap] node {p_1} (m-2-4)
                (m-1-5) edge[swap] node {i_1} (m-1-3)
                    edge[swap] node {p_2} (m-2-6)
                (m-1-7) edge[swap] node {i_2} (m-1-5)
                (m-1-8) edge (m-1-7)

                (m-2-2) edge[suspension] node[swap] {\delta_0} (m-1-3)
                (m-2-4) edge[suspension] node[swap] {\delta_1} (m-1-5)
                (m-2-6) edge[suspension] node[swap] {\delta_2} (m-1-7);
        \end{tikzpicture}
    \end{center}

    Where every \( i_n \in \Nc \), and \( I_n \in \Ic \), and marked arrows denote degree shifting maps, with \( \delta_n \in \Tc(I_n, \Sigma Y_{n + 1}) \). And where every triangle on the form:

    \begin{center}
        \begin{tikzpicture}
            \diagram{m}{1cm}{2cm} {
                \Sigma^{-1} I_n \& Y_{n + 1} \& Y_n \& I_n \\
            };

            \draw[math]
                (m-1-1) edge node {\Sigma^{-1}(\delta_s)} (m-1-2)
                (m-1-2) edge node {i_n} (m-1-3)
                (m-1-3) edge node {p_n} (m-1-4);
        \end{tikzpicture}
    \end{center}

    is distinguished.

    This is called an \emph{Adams resolution of \( Y_0 \) with respect to an injective class \( (\Ic, \Nc) \)}.
\end{definition}

\begin{theorem} % Stable? TODO
    Let \( (\Pc, \Nc) \) be a projective class in \( \Tc \).

    Then for any object \( X \in \Tc \), there exists an Adams Resolution of \( X \) with respect to the projective class \( (\Pc, \Nc) \).
\end{theorem}
\begin{proof}
    Let \( X_0 = X \).

    Then by \autoref{def:projective_class} property 3, one has that there exists an object \( X_1 \in \Tc \) and object \( P_0 \in \Pc \), and three morphisms \( i_0, p_0, \delta_0 \), such that the following is a distinguished triangle
    \[
        P_0 \stackrel{p_0}{\longrightarrow} X_0 \stackrel{i_0}{\longrightarrow} X_1 \stackrel{\delta_0}{\longrightarrow} \Sigma P_0
    \]
    and \( i_0 \in \Nc \).

    Repeat this process with \( X_1, X_2, \dots, X_n \), until one gets every map \( i_n, p_n, \delta_n \) for any \( n \in \Nb \).

    These maps fit into the diagram seen in \autoref{def:adams_resolution_projective_class}.
\end{proof}

\begin{theorem} \label{thm:exists_adams_resolution_injective_class} % Stable? TODO
    Let \( (\Ic, \Nc) \) be an injective class in \( \Tc \).

    Then for any object \( Y \in \Tc \), there exists an Adams resolution of \( Y \) with respect to the injective class \( (\Ic, \Nc) \).
\end{theorem}
\begin{proof}
    TODO
\end{proof}

\begin{construction} \label{construction:adams_spectral_sequence} 
    Let \( (\Ic, \Nc) \) be an injective class in \( \Tc \).

    For any \( Y \in \Tc \), one can construct an Adams resolution of \( Y \) with respect to the injective class \( (\Ic, \Nc) \) by \autoref{thm:exists_adams_resolution_injective_class}, using the notation from \autoref{def:adams_resolution_injective_class}.

    For any \( X \in \Tc, t \in \Zb, s \in \Nb_0 \), let:
    \[ 
        \phi: \Tc(\Sigma^{t - s} X, \Sigma Y_{s + 1}) \stackrel{\cong}{\longrightarrow} \Tc(\Sigma^{t - s - 1} X, Y_{s + 1}) 
    \]
    Denote the natural isomorphism from \( \Sigma \)'s automorphism property.

    Furthermore, let
    \begin{align*}
        (i_s)_*: \Tc(\Sigma^{t - s} X, Y_s) &\to \Tc(\Sigma^{t - s} X, Y_{s - 1}) \\
        (p_s)_*: \Tc(\Sigma^{t - s} X, Y_s) &\to \Tc(\Sigma^{t - s} X, I_s) \\
        (\phi \circ \delta_s)_*: \Tc(\Sigma^{t - s} X, I_s) &\to \Tc(\Sigma^{t - s - 1} X, Y_{s + 1})
    \end{align*}
    be maps given by post-composition by \( i_s, p_s, \phi \circ \delta_s \) respectively.

    Let \( i, p, \delta \) be morphisms that fit in the following diagram:

    \begin{center}
        \begin{tikzpicture}
            \diagram{m}{2cm}{1cm} {
                \directsum_{s \in \Nb_0, t \in \Zb}\Tc(\Sigma^{t - s} X, Y_s) \&\& \directsum_{s \in \Nb_0, t \in \Zb}\Tc(\Sigma^{t - s} X, Y_s) \\
                \&  \directsum_{s \in \Nb_0, t \in \Zb}\Tc(\Sigma^{t - s} X, I_s) \\
            };

            % Very sus 0-length arrow created when using anchors.
            % tips=proper or use "to" instead of "edge" (only on final arrow, or else there is no head).
            \draw[math, tips=proper]
                (m-1-1) edge node {i} (m-1-3)
                (m-1-3) edge node {p} (m-2-2.north east)

                (m-2-2.north west) edge node {\delta} (m-1-1);
        \end{tikzpicture}
    \end{center}

    Where \( i \) is a direct sum of either \( 0 \) on the coordinates with \( s = 0 \), or \( (i_s)_* \) for some \( s \neq 0 \), depending on what fits. Likewise, \( p \) is a direct sum of \( (p_s)_* \), and \( \delta \) is a direct sum of \( (\phi \circ \delta_s)_* \), for \( s \) such that it fits.
\end{construction}

\begin{theorem}
    The diagram in \autoref{construction:adams_spectral_sequence} is an exact couple.
\end{theorem}
\begin{proof} % Simplify proof, not dependant on Adams resolution structure at all? TODO
    Want to prove that the diagram is exact in the three corners.

    Firstly, note that since \( i, p, \delta \) are all direct summands of maps, it is sufficient to check that it is exact for every ``level'' of the direct sum.

    Therefore, fix any \( t \in \Zb \) and \( s \in \Nb_0 \).

    Focusing on the top right corner one needs to show that the maps \( i \) and \( p \) are exact given these \( t \) and \( s \).

    Firstly note that
    \begin{center}
        \begin{tikzpicture}
            \diagram{m}{1cm}{1cm} {
                \Sigma^{-1} I_s \& Y_{s+1} \& Y_s \& I_s \\
            };

            \draw[math]
                (m-1-1) edge node {\Sigma^{-1}(\delta_s)} (m-1-2)
                (m-1-2) edge node {i_s} (m-1-3)
                (m-1-3) edge node {p_s} (m-1-4);
        \end{tikzpicture}
    \end{center}
    is a distinguished triangle by \autoref{def:adams_resolution_injective_class}. Then one can construct the following long exact sequence:
    \begin{center}
        \begin{tikzpicture}
            \diagram{m}{1cm}{1cm} {
                \dots \& \Tc(\Sigma^{t-s} X, Y_{s+1}) \& \Tc(\Sigma^{t-s} X, Y_s) \& \Tc(\Sigma^{t-s} X, I_s) \& \dots \\
            };

            \draw[math]
                (m-1-1) edge (m-1-2)
                (m-1-2) edge node {(i_s)_*} (m-1-3)
                (m-1-3) edge node {(p_s)_*} (m-1-4)
                (m-1-4) edge (m-1-5);
        \end{tikzpicture}
    \end{center}

    Which is exactly how the maps \( i \) and \( p \) would interact on the coordinate with \( s \) and \( t \) fixed, and is therefore exact.

    Focusing on the bottom corner, it is very similar:

    The following triangle is distinguished
    \begin{center}
        \begin{tikzpicture}
            \diagram{m}{1cm}{1cm} {
                Y_{s+1} \& Y_s \& I_s \& \Sigma Y_{s+1}) \\
            };

            \draw[math]
                (m-1-1) edge node {i_s} (m-1-2)
                (m-1-2) edge node {p_s} (m-1-3)
                (m-1-3) edge node {\delta_s} (m-1-4);
        \end{tikzpicture}
    \end{center}
    since it is a left-rotated version of the distinguished triangle above.

    Again, this gives rise to the following long exact sequence
    \begin{center}
        \begin{tikzpicture}
            \diagram{m}{1cm}{1cm} {
                \dots \& \Tc(\Sigma^{t-s} X, Y_s) \& \Tc(\Sigma^{t-s} X, I_s) \& \Tc(\Sigma^{t-s} X, \Sigma Y_{s + 1}) \& \dots \\
            };

            \draw[math]
                (m-1-1) edge (m-1-2)
                (m-1-2) edge node {(p_s)_*} (m-1-3)
                (m-1-3) edge node {(\delta_s)_*} (m-1-4)
                (m-1-4) edge (m-1-5);
        \end{tikzpicture}
    \end{center}
    Changing the right map to \( (\phi \circ \delta_s)_* \), one gets the following sequence
    \begin{center}
        \begin{tikzpicture}
            \diagram{m}{1cm}{1cm} {
                \dots \& \Tc(\Sigma^{t-s} X, Y_s) \& \Tc(\Sigma^{t-s} X, I_s) \& \Tc(\Sigma^{t-s-1} X, Y_{s + 1}) \& \dots \\
            };

            \draw[math]
                (m-1-1) edge (m-1-2)
                (m-1-2) edge node {(p_s)_*} (m-1-3)
                (m-1-3) edge node {\phi \circ (\delta_s)_*} (m-1-4)
                (m-1-4) edge (m-1-5);
        \end{tikzpicture}
    \end{center}
    Which is still exact in the middle, because \( \ker((\delta_s)_*) = \ker((\phi \circ \delta_s)_*) \), since \( \phi \) is an isomorphism.

    And since the middle part is exactly how the maps \( p \) and \( \delta \) would meet for given \( s \) and \( t \), the diagram is also exact at the bottom.

    Lastly at the top left corner one has to divide the problem into two cases:

    If \( s \neq 0 \):

    Then the following triangle is distinguished and exists since \( s \geq 1 \)
    \begin{center}
        \begin{tikzpicture}
            \diagram{m}{1cm}{1cm} {
                I_{s - 1} \& \Sigma Y_s \& \Sigma Y_{s - 1} \& \Sigma I_{s - 1} \\
            };

            \draw[math]
                (m-1-1) edge node[above=5pt] {\delta_{s - 1}} (m-1-2)
                (m-1-2) edge node[above=5pt] {\Sigma(i_{s - 1})} (m-1-3)
                (m-1-3) edge node[above=5pt] {\Sigma(p_{s - 1})} (m-1-4);
        \end{tikzpicture}
    \end{center}
    because it is the same as the triangles above, but with the indexes reduced by one.

    Using this, one can construct the following long exact sequence
    \begin{center}
        \begin{tikzpicture}
            \diagram{m}{1cm}{1cm} {
                \dots \& \Tc(\Sigma^{t-s} X, I_{s - 1}) \& \Tc(\Sigma^{t-s} X, \Sigma Y_s)) \& \Tc(\Sigma^{t - s} X, \Sigma Y_{s - 1}) \& \dots \\
            };

            \draw[math]
                (m-1-1) edge (m-1-2)
                (m-1-2) edge node[above=5pt] {(\delta_{s - 1})_*} (m-1-3)
                (m-1-3) edge node[above=5pt] {(\Sigma(i_{s - 1}))_*} (m-1-4)
                (m-1-4) edge (m-1-5);
        \end{tikzpicture}
    \end{center}

    Let \( \psi \) be the natural isomorphism
    \[
        \psi: \Tc(\Sigma^{t-s} X, \Sigma Y_{s - 1}) \stackrel{\cong}{\longrightarrow} \Tc(\Sigma^{t - s - 1} X, Y_{s - 1}).
    \]

    Then by changing the maps, one gets the following sequence
    \begin{center}
        \begin{tikzpicture}
            \diagram{m}{1cm}{1cm} {
                \dots \& \Tc(\Sigma^{t-s} X, I_{s - 1}) \& \Tc(\Sigma^{t - s - 1} X, Y_s) \& \Tc(\Sigma^{t - s - 1} X, Y_{s - 1}) \& \dots \\
            };

            \draw[math]
                (m-1-1) edge (m-1-2)
                (m-1-2) edge node[above=5pt] {\phi \circ (\delta_{s - 1})_*} (m-1-3)
                (m-1-3) edge node[above=5pt] {\psi \circ (\Sigma(i_{s - 1}))_* \circ \phi^{-1}} (m-1-4)
                (m-1-4) edge (m-1-5);
        \end{tikzpicture}
    \end{center}
    Where the middle part is still exact, because
    % Need lemma for last part?
    \begin{align*}
        \im((\delta_{s-1})_*) &= \ker((\Sigma(i_{s-1}))_*) \\
        &\Updownarrow \\
        \phi(\im((\delta_{s-1})_*)) &= \phi(\ker((\Sigma(i_{s-1}))_*)) \\
        \im(\phi \circ (\delta_{s-1})_*) &= \phi(\ker((\Sigma(i_{s-1}))_*)) \\
        &= \ker((\Sigma(i_{s-1}))_* \circ \phi^{-1})
    \end{align*}
    and since \( \ker(\psi \circ (\Sigma(i_{s-1}))_* \circ \phi^{-1}) = \ker((\Sigma(i_{s-1}))_* \circ \phi^{-1}) \), since \( \psi \) is an isomorphism.

    Furthermore, by the definition of \( \phi \) and \( \psi \), one has the following commutative diagram
    \begin{center}
        \begin{tikzpicture}
            \diagram{m}{1cm}{1cm} {
                \Sigma^{t - s} X \& \Sigma Y_s \& \Sigma Y_{s - 1} \\
                \Sigma^{t - s - 1} X \& Y_s \& Y_{s - 1} \\
            };

            \draw[math]
                (m-1-1) edge node {\Sigma( f )} (m-1-2)
                    edge node {\eta_1} (m-2-1)
                (m-1-2) edge node {\Sigma( i_s )} (m-1-3)
                    edge node {\eta_2} (m-2-2)
                (m-1-3) edge node {\eta_3} (m-2-3)

                (m-2-1) edge node {f} (m-2-2)
                (m-2-2) edge node {i_s} (m-2-3);
        \end{tikzpicture}
    \end{center}
    Where the \( \eta_j \)'s are the natural isomorphisms that define both \( \phi \) and \( \psi \).
    
    % TODO: Manglande objekt?
    I.e. for any \( f \in \Tc(\Sigma^{t - s - 1} X, Y_s) \) and \( g \in \Tc(\Sigma^{t - s - 1}, Y_{s - 1}) \), one has
    \[
        \phi: \Sigma( f ) \mapsto f = \eta_2 \circ \Sigma( f ) \circ \eta_1^{-1}
    \]
    and
    \[
        \psi: \Sigma( g ) \mapsto g = \eta_3 \circ \Sigma( g ) \circ \eta_1^{-1}.
    \]

    Then look at \( \psi \circ (\Sigma(i_{s-1}))_* \circ \phi^{-1} \) pointwise
    \begin{align*}
        \psi \circ (\Sigma(i_{s-1}))_* \circ \phi^{-1} ( f ) &= \psi \circ (\Sigma(i_{s-1}))_* ( \eta_2^{-1} \circ f \circ \eta_1 ) \\
        &= \psi ( \Sigma(i_{s - 1}) \circ \eta_2^{-1} \circ f \circ \eta_1 ) \\
        &= \eta_3 \circ \Sigma(i_{s - 1}) \circ \eta_2^{-1} \circ f \circ \eta_1 \circ \eta_1^{-1} \\
        &= \eta_3 \circ \Sigma(i_{s - 1}) \circ \eta_2^{-1} \circ f \\
        \intertext{By commutativity of the above diagram, this becomes}
        &= i_{s - 1} \circ f
    \end{align*}

    Which means that \( \psi \circ (\Sigma(i_{s-1}))_* \circ \phi^{-1} = ( i_{s - 1} )_* \). One thereforore gets the following sequence that is identical to the one above
    \begin{center}
        \begin{tikzpicture}
            \diagram{m}{1cm}{1cm} {
                \dots \& \Tc(\Sigma^{t-s} X, I_{s - 1}) \& \Tc(\Sigma^{t - s - 1} X, Y_s) \& \Tc(\Sigma^{t - s - 1} X, Y_{s - 1}) \& \dots \\
            };

            \draw[math]
                (m-1-1) edge (m-1-2)
                (m-1-2) edge node[above=5pt] {\phi \circ (\delta_{s - 1})_*} (m-1-3)
                (m-1-3) edge node[above=5pt] {i_{s - 1}} (m-1-4)
                (m-1-4) edge (m-1-5);
        \end{tikzpicture}
    \end{center}
    Which is exactly how the maps \( \delta \) and \( i \) would meet for given \( s \) and \( t \), and is exact in the middle, implying the top left corner of the above diagram would be exact.

    And finally, if \( s = 0 \):

    The image of \( (\phi \circ \delta_n)_* \) for any \( n \) would never have codomain \( Y_0 \), except for the zero map by the definition, and therefore there is no image onto \( \Tc(\Sigma^t X, Y_0) \). There is also no map \( i_n \) with domain \( \Tc(\Sigma^t X , Y_0) \), except for the zero map \( i_0 = 0 \). And so, it is exact by default.

    Therefore one has that the top left corner is also exact for the given \( t \) and \( s \).

    By conclusion, since the choice of \( s \) and \( t \) was arbitrary, every corner is exact for any \( s \in \Nb_0 \) and \( t \in \Zb \), and by the argument at the start of the proof, the theorem follows.
\end{proof}

\begin{definition}[Adams spectral sequence]
    Let \( (\Ic, \Nc) \) be a stable injective class in \( \Tc \).

    Then

    TODO: Finish
\end{definition}
