\begin{definition}[Triangles]
    Let \( \Tc \) be an additive category, and let \( \Sigma: \Tc \to \Tc \) be an additive autoequivalence.

    A \emph{triangle in \( \Tc \)}, is a diagram in \( \Tc \) on the form
    \begin{center}
        \begin{tikzpicture}
            \diagram{m}{1cm}{1cm} {
                A \& B \& C \& \Sigma A \\
            };

            \draw[math]
                (m-1-1) edge node {f} (m-1-2)
                (m-1-2) edge node {g} (m-1-3)
                (m-1-3) edge node {h} (m-1-4);
        \end{tikzpicture}
    \end{center}

    A \emph{triangle morphism} from
    \begin{center}
        \begin{tikzpicture}
            \diagram{m}{1cm}{1cm} {
                A \& B \& C \& \Sigma A \\
            };

            \draw[math]
                (m-1-1) edge node {f} (m-1-2)
                (m-1-2) edge node {g} (m-1-3)
                (m-1-3) edge node {h} (m-1-4);
        \end{tikzpicture}
        to
        \begin{tikzpicture}
            \diagram{m}{1cm}{1cm} {
                A' \& B' \& C' \& \Sigma A', \\
            };

            \draw[math]
                (m-1-1) edge node {f'} (m-1-2)
                (m-1-2) edge node {g'} (m-1-3)
                (m-1-3) edge node {h'} (m-1-4);
        \end{tikzpicture}
    \end{center}
    are three morphisms in \( \Tc \);
    \[
        a: A \to A', \quad b: B \to B', \text{ and } c: C \to C'
    \]
    such that the following diagram commutes
    \begin{center}
        \begin{tikzpicture}
            \diagram{m}{1cm}{1cm} {
                A \& B \& C \& \Sigma A \\
                A' \& B' \& C' \& \Sigma A'. \\
            };

            \draw[math]
                (m-1-1) edge node {f} (m-1-2)
                    edge node {a} (m-2-1)
                (m-1-2) edge node {g} (m-1-3)
                    edge node {b} (m-2-2)
                (m-1-3) edge node {h} (m-1-4)
                    edge node {c} (m-2-3)
                (m-1-4) edge node {\Sigma(a)} (m-2-4)

                (m-2-1) edge node {f'} (m-2-2)
                (m-2-2) edge node {g'} (m-2-3)
                (m-2-3) edge node {h'} (m-2-4);
        \end{tikzpicture}
    \end{center}
    Furthermore if \( a, b, \) and \( c \) are isomorphisms, then it is called an \emph{isomorphism of triangles} and the two triangles are said to be \emph{isomorphic}.
\end{definition}

% TODO: nemn at det er fleire forskjellige definisjonar.

\begin{definition}[Triangulated category]
    Let \( \Tc \) be an additive category, and let \( \Sigma: \Tc \to \Tc \) be an additive autoequivalence. In addition, let \( \Delta \) be a class of triangles in \( \Tc \).

    Furthermore let \( \Tc \), \( \Sigma \) and \( \Delta \) satisfy the following four axioms:
    \begin{enumerate}[label={(\bfseries T\arabic*)}]
        \item {
            The following three points must be satisfied:
            \begin{itemize}
                \item {
                    For any \( f: X \to Y \) in \( \Tc \), then there is some triangle
                    \begin{center}
                        \begin{tikzpicture}
                            \diagram{m}{1cm}{1cm} {
                                X \& Y \& Z \& \Sigma X \\
                            };

                            \draw[math]
                                (m-1-1) edge node {f} (m-1-2)
                                (m-1-2) edge (m-1-3)
                                (m-1-3) edge (m-1-4);
                        \end{tikzpicture}
                    \end{center}
                    in \( \Delta \).
                }
                \item {
                    For any \( X \in \Tc \), the triangle
                    \begin{center}
                        \begin{tikzpicture}
                            \diagram{m}{1cm}{1cm} {
                                X \& X \& 0 \& \Sigma X \\
                            };

                            \draw[math]
                                (m-1-1) edge node {\Id_X} (m-1-2)
                                (m-1-2) edge (m-1-3)
                                (m-1-3) edge (m-1-4);
                        \end{tikzpicture}
                    \end{center}
                    is in \( \Delta \).
                }
                \item {
                    \( \Delta \) is closed under isomorphisms of triangles.
                }
            \end{itemize}
        }
        \item {
            For any triangle
            \begin{center}
                \begin{tikzpicture}
                    \diagram{m}{1cm}{1cm} {
                        A \& B \& C \& \Sigma A \\
                    };
        
                    \draw[math]
                        (m-1-1) edge node {f} (m-1-2)
                        (m-1-2) edge node {g} (m-1-3)
                        (m-1-3) edge node {h} (m-1-4);
                \end{tikzpicture}
            \end{center}
            in \( \Delta \), then both
            \begin{center}
                \begin{tikzpicture}
                    \diagram{m}{1cm}{1cm} {
                        B \& C \& \Sigma A \& \Sigma B \\
                    };
        
                    \draw[math]
                        (m-1-1) edge node {g} (m-1-2)
                        (m-1-2) edge node {h} (m-1-3)
                        (m-1-3) edge node {-\Sigma(f)} (m-1-4);
                \end{tikzpicture}
                and
                \begin{tikzpicture}
                    \diagram{m}{1cm}{1cm} {
                        \Sigma^{-1}(C) \& A \& B \& C \\
                    };
        
                    \draw[math]
                        (m-1-1) edge node {- \Sigma^{-1}(h)} (m-1-2)
                        (m-1-2) edge node {f} (m-1-3)
                        (m-1-3) edge node {g} (m-1-4);
                \end{tikzpicture}
            \end{center}
            are also in \( \Delta \).
        }
        \item {
            Let the solid part of the following diagram
            \begin{center}
                \begin{tikzpicture}
                    \diagram{m}{1cm}{1cm} {
                        A \& B \& C \& \Sigma A \\
                        A' \& B' \& C' \& \Sigma A'. \\
                    };
        
                    \draw[math]
                        (m-1-1) edge node {f} (m-1-2)
                            edge node {a} (m-2-1)
                        (m-1-2) edge node {g} (m-1-3)
                            edge node {b} (m-2-2)
                        (m-1-3) edge node {h} (m-1-4)
                            edge[dashed] node {c} (m-2-3)
                        (m-1-4) edge node {\Sigma(a)} (m-2-4)
        
                        (m-2-1) edge node {f'} (m-2-2)
                        (m-2-2) edge node {g'} (m-2-3)
                        (m-2-3) edge node {h'} (m-2-4);
                \end{tikzpicture}
            \end{center}
            commute, and let both of the rows be triangles in \( \Delta \).
            
            Then there exists a morphism \( c: C \to C' \) such that the above diagram including \( c \) commutes, i.e. \( a, b, c \) becomes a morphism of triangles.
        }
        \item {
            Let the solid part of the following diagram
            \begin{center}
                \begin{tikzpicture}
                    \diagram{m}{1cm}{1cm} {
                        A \& B \& C \& \Sigma A \\
                        A \& D \& E \& \Sigma A \\
                        \& F \& F \& \Sigma B \\
                        \& \Sigma B \& \Sigma C \\
                    };
        
                    \draw[math]
                        (m-1-1) edge node {f} (m-1-2)
                            edge[equal] (m-2-1)
                        (m-1-2) edge node {g} (m-1-3)
                            edge (m-2-2)
                        (m-1-3) edge (m-1-4)
                            edge[dashed] (m-2-3)
                        (m-1-4) edge[equal] (m-2-4)
        
                        (m-2-1) edge (m-2-2)
                        (m-2-2) edge (m-2-3)
                            edge (m-3-2)
                        (m-2-3) edge (m-2-4)
                            edge[dashed] (m-3-3)
                        (m-2-4) edge node {\Sigma(f)} (m-3-4)

                        (m-3-2) edge[equal] (m-3-3)
                            edge node[swap] {h} (m-4-2)
                        (m-3-3) edge node {h} (m-3-4)
                            edge node {\Sigma(g) \circ h} (m-4-3)

                        (m-4-2) edge node {\Sigma(g)} (m-4-3);
                \end{tikzpicture}
            \end{center}
            commute, and let the top two rows and the leftmost column that is four tall be three triangles in \( \Delta \).

            Then there exist some morphisms where there are dashed arrows such that the rightmost four tall column is a triangle in \( \Delta \) and the entire diagram, including the dashed arrows, commute.
        }
    \end{enumerate}

    Then \( \tuple{ \Tc, \Sigma, \Delta } \), or shortened to just \( \Tc \), is called an \emph{triangulated category}, \( \Sigma \) is called the \emph{shift functor of \( \Tc \)}, and every triangle in \( \Delta \) is called a \emph{distinguished triangle in \( \Tc \)}.
\end{definition}