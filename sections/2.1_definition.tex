A triangulated category is closely linked to some objects and morphisms making ``triangles'' and showing that there exist some class of triangles that satisfy certain properties. We start by defining what a triangle in an additive category is, as well as what a morphism between triangles is.

\begin{definition}[Triangles]
    \label{def:triangles}
    Let \( \Tc \) be an additive category, and let \( \Sigma: \Tc \to \Tc \) be an additive auto-equivalence.

    A \emph{triangle in \( \Tc \)}, is a diagram in \( \Tc \) on the form
    \begin{center}
        \begin{tikzpicture}
            \diagram{m}{1cm}{1cm} {
                A \& B \& C \& \Sigma A \\
            };

            \draw[math]
                (m-1-1) edge node {f} (m-1-2)
                (m-1-2) edge node {g} (m-1-3)
                (m-1-3) edge node {h} (m-1-4);
        \end{tikzpicture}
    \end{center}

    A \emph{triangle morphism} from
    \begin{center}
        \begin{tikzpicture}
            \diagram{m}{1cm}{1cm} {
                A \& B \& C \& \Sigma A \&[-0.8cm] \text{to} \&[-0.8cm]  A' \& B' \& C' \& \Sigma A', \\
            };

            \draw[math]
                (m-1-1) edge node {f} (m-1-2)
                (m-1-2) edge node {g} (m-1-3)
                (m-1-3) edge node {h} (m-1-4)
                (m-1-6) edge node {f'} (m-1-7)
                (m-1-7) edge node {g'} (m-1-8)
                (m-1-8) edge node {h'} (m-1-9);
        \end{tikzpicture}
    \end{center}
    is a touple of three morphisms \( (a, b, c) \) in \( \Tc \) such that the following diagram commutes
    \begin{center}
        \begin{tikzpicture}
            \diagram{m}{1cm}{1cm} {
                A \& B \& C \& \Sigma A \\
                A' \& B' \& C' \& \Sigma A'. \\
            };

            \draw[math]
                (m-1-1) edge node {f} (m-1-2)
                    edge node {a} (m-2-1)
                (m-1-2) edge node {g} (m-1-3)
                    edge node {b} (m-2-2)
                (m-1-3) edge node {h} (m-1-4)
                    edge node {c} (m-2-3)
                (m-1-4) edge node {\Sigma a} (m-2-4)

                (m-2-1) edge node {f'} (m-2-2)
                (m-2-2) edge node {g'} (m-2-3)
                (m-2-3) edge node {h'} (m-2-4);
        \end{tikzpicture}
    \end{center}
    Furthermore if \( a, b, \) and \( c \) are isomorphisms, then \( (a, b, c) \) is called an \emph{isomorphism of triangles} and the two triangles are said to be \emph{isomorphic}.
\end{definition}

This thesis will use the following definition of a triangulated category.

\begin{definition}[Triangulated category]
    \label{def:triangulated_category}
    Let \( \Tc \) be an additive category, let \( \Sigma: \Tc \to \Tc \) be an additive auto-equivalence, and let \( \Delta \) be a class of triangles (\autoref{def:triangles}) in \( \Tc \).

    Furthermore, let \( \Tc \), \( \Sigma \) and \( \Delta \) satisfy the following four axioms:
    \begin{enumerate}[label={(\bfseries TR\arabic*)}]
        \item {
            The following three points must be satisfied:
            \begin{enumerate}
                \item {
                    For any \( f: X \to Y \) in \( \Tc \), there exist some object \( C_f \in \Tc \), called a \emph{cone of \( f \)} along with morphisms \( \iota_f \) and \( \pi_f \), such that there is a triangle,
                    \begin{center}
                        \begin{tikzpicture}
                            \diagram{m}{1cm}{1cm} {
                                X \& Y \& C_f \& \Sigma X \\
                            };

                            \draw[math]
                                (m-1-1) edge node {f} (m-1-2)
                                (m-1-2) edge node {\iota_f} (m-1-3)
                                (m-1-3) edge node {\pi_f} (m-1-4);
                        \end{tikzpicture}
                    \end{center}
                    in \( \Delta \).

                    Any triangle of the above form is called a \emph{standard triangle of \( f \)}.
                }
                \item {
                    For any \( X \in \Tc \), the \emph{trivial triangle of \( X \)}, which is the following triangle
                    \begin{center}
                        \begin{tikzpicture}
                            \diagram{m}{1cm}{1cm} {
                                X \& X \& 0 \& \Sigma X \\
                            };

                            \draw[math]
                                (m-1-1) edge node {\Id_X} (m-1-2)
                                (m-1-2) edge (m-1-3)
                                (m-1-3) edge (m-1-4);
                        \end{tikzpicture}
                    \end{center}
                    is in \( \Delta \).
                }
                \item {
                    \( \Delta \) is closed under isomorphisms of triangles.
                }
            \end{enumerate}
        }
        \item {
            Any triangle
            \begin{diagramlabel}[\label{eq:tri_cat_def_right_rotation}]
                \begin{tikzpicture}
                    \diagram{m}{1cm}{1cm} {
                        A \& B \& C \& \Sigma A \\
                    };
        
                    \draw[math]
                        (m-1-1) edge node {f} (m-1-2)
                        (m-1-2) edge node {g} (m-1-3)
                        (m-1-3) edge node {h} (m-1-4);
                \end{tikzpicture}
            \end{diagramlabel}
            is in \( \Delta \) if and only if
            \begin{diagramlabel}[\label{eq:tri_cat_def_left_rotation}]
                \begin{tikzpicture}
                    \diagram{m}{1cm}{1cm} {
                        B \& C \& \Sigma A \& \Sigma B \\
                    };
        
                    \draw[math]
                        (m-1-1) edge node {g} (m-1-2)
                        (m-1-2) edge node {h} (m-1-3)
                        (m-1-3) edge node {-\Sigma f} (m-1-4);
                \end{tikzpicture}
            \end{diagramlabel}
            is in \( \Delta \). The triangle \autoref{eq:tri_cat_def_left_rotation} is called the \emph{left rotation} of \autoref{eq:tri_cat_def_right_rotation}, and \autoref{eq:tri_cat_def_right_rotation} is called the \emph{right rotation} of \autoref{eq:tri_cat_def_left_rotation}.
        }
        \item {
            Let the solid part of the following diagram
            \begin{center}
                \begin{tikzpicture}
                    \diagram{m}{1cm}{1cm} {
                        A \& B \& C \& \Sigma A \\
                        A' \& B' \& C' \& \Sigma A'. \\
                    };
        
                    \draw[math]
                        (m-1-1) edge node {f} (m-1-2)
                            edge node {a} (m-2-1)
                        (m-1-2) edge node {g} (m-1-3)
                            edge node {b} (m-2-2)
                        (m-1-3) edge node {h} (m-1-4)
                            edge[dashed] node {c} (m-2-3)
                        (m-1-4) edge node {\Sigma a} (m-2-4)
        
                        (m-2-1) edge node {f'} (m-2-2)
                        (m-2-2) edge node {g'} (m-2-3)
                        (m-2-3) edge node {h'} (m-2-4);
                \end{tikzpicture}
            \end{center}
            commute, and let both of the rows be triangles in \( \Delta \).
            
            Then there exists a morphism \( c: C \to C' \) such that \( (a, b, c) \) becomes a morphism of triangles.
        }
        \item {
            Let the solid part of the following diagram commute,
            \begin{center}
                \begin{tikzpicture}
                    \diagram{m}{1cm}{1cm} {
                        A \& B \& C \& \Sigma A \\
                        A \& D \& E \& \Sigma A \\
                        \& F \& F \& \Sigma B \\
                        \& \Sigma B \& \Sigma C, \\
                    };
        
                    \draw[math]
                        (m-1-1) edge node {f} (m-1-2)
                            edge[equality] (m-2-1)
                        (m-1-2) edge node {g} (m-1-3)
                            edge node {n} (m-2-2)
                        (m-1-3) edge node {h} (m-1-4)
                            edge[dashed] node {\phi} (m-2-3)
                        (m-1-4) edge[equality] (m-2-4)
        
                        (m-2-1) edge node {n \circ f} (m-2-2)
                        (m-2-2) edge node {j} (m-2-3)
                            edge node {m} (m-3-2)
                        (m-2-3) edge node {l} (m-2-4)
                            edge[dashed] node {\psi} (m-3-3)
                        (m-2-4) edge node {\Sigma f} (m-3-4)

                        (m-3-2) edge[equality] (m-3-3)
                            edge node[swap] {k} (m-4-2)
                        (m-3-3) edge node {k} (m-3-4)
                            edge node {\Sigma(g) \circ k} (m-4-3)

                        (m-4-2) edge node {\Sigma g} (m-4-3);
                \end{tikzpicture}
            \end{center}
            and let \( \tuple{f, g, h} \), \( \tuple{n, m, k} \), and \( \tuple{n \circ f, j, l} \) be three triangles in \( \Delta \).

            Then there exist some morphisms, \( \phi \) and \( \psi \), such that \( \tuple{\phi, \psi, (\Sigma g) \circ k} \) is a triangle in \( \Delta \) and the entire diagram, including the dashed arrows, commute.
        }
    \end{enumerate}

    Then \( \tuple{ \Tc, \Sigma, \Delta } \), or shortened to just \( \Tc \), is called a \emph{triangulated category}, or a \emph{triangulation} of \( \Tc \). The functor \( \Sigma \) is called the \emph{shift} (or \emph{suspension}) \emph{functor} of \( \Tc \), and every triangle in \( \Delta \) is called a \emph{distinguished} (or \emph{exact}) \emph{triangle} in \( \Tc \). 
\end{definition}

Although the definition is axiomatic and may seem arbitrary, there are surprisingly many different categories that admit the structure of a triangulated category.

There are multiple different, but equivalent, definitions of a triangulated category with different axioms. An example of this is \cite[Definition 2.1]{May_2001}, where instead of what this thesis calls {\bf (TR4)}, he instead uses another axiom, closer to Verdier's original definition.

Some of the axioms of a triangulated category are superflous in the fact that they are implied by the other axioms. An example is that {\bf (TR3)} is wholly contained in the definition of {\bf (TR4)} (TODO: May sin TR4 er ansleis frå den eg brukar.). Another example of this is that given {\bf (TR1)}, {\bf (TR3)} and the \( (\Rightarrow) \) implication of {\bf (TR2)}, then it implies the \( (\Leftarrow) \) implication as well. The details are contained in the following lemma.

\begin{lemma}
    \label{lem:triangulated_category-TR2-only_one_rotation}
    Let \( \tuple{ \Tc, \Sigma, \Delta } \) satisfy the axioms {\bf (TR1)} and {\bf (TR3)} from \autoref{def:triangulated_category} as well as the \( (\Rightarrow) \) implication of {\bf (TR2)}.

    Then the \( (\Leftarrow) \) implication of {\bf (TR2)} is also satisfied.
\end{lemma}
For a proof of this see \cite[Lemma 2.4]{May_2001}.

