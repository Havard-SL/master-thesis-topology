Before defining what a triangulated category is, it can be helpful to first define what a triangle, and morphism between them are.

\begin{definition}[Triangles]
    \label{def:triangles}
    Let \( \Tc \) be an additive category, and let \( \Sigma: \Tc \to \Tc \) be an additive autoequivalence.

    A \emph{triangle in \( \Tc \)}, is a diagram in \( \Tc \) on the form
    \begin{center}
        \begin{tikzpicture}
            \diagram{m}{1cm}{1cm} {
                A \& B \& C \& \Sigma A \\
            };

            \draw[math]
                (m-1-1) edge node {f} (m-1-2)
                (m-1-2) edge node {g} (m-1-3)
                (m-1-3) edge node {h} (m-1-4);
        \end{tikzpicture}
    \end{center}

    A \emph{triangle morphism} from
    \begin{center}
    %     \begin{tikzpicture}
    %         \diagram{m}{1cm}{1cm} {
    %             A \& B \& C \& \Sigma A \\
    %         };

    %         \draw[math]
    %             (m-1-1) edge node {f} (m-1-2)
    %             (m-1-2) edge node {g} (m-1-3)
    %             (m-1-3) edge node {h} (m-1-4);
    %     \end{tikzpicture}
    %     to
    %     \begin{tikzpicture}
    %         \diagram{m}{1cm}{1cm} {
    %             A' \& B' \& C' \& \Sigma A', \\
    %         };

    %         \draw[math]
    %             (m-1-1) edge node {f'} (m-1-2)
    %             (m-1-2) edge node {g'} (m-1-3)
    %             (m-1-3) edge node {h'} (m-1-4);
    %     \end{tikzpicture}
        \begin{tikzpicture}
            \diagram{m}{1cm}{1cm} {
                A \& B \& C \& \Sigma A \&[-0.8cm] \text{to} \&[-0.8cm]  A' \& B' \& C' \& \Sigma A', \\
            };

            \draw[math]
                (m-1-1) edge node {f} (m-1-2)
                (m-1-2) edge node {g} (m-1-3)
                (m-1-3) edge node {h} (m-1-4)
                (m-1-6) edge node {f'} (m-1-7)
                (m-1-7) edge node {g'} (m-1-8)
                (m-1-8) edge node {h'} (m-1-9);
        \end{tikzpicture}
    \end{center}
    is a touple of three morphisms \( (a, b, c) \) in \( \Tc \) such that the following diagram commutes
    \begin{center}
        \begin{tikzpicture}
            \diagram{m}{1cm}{1cm} {
                A \& B \& C \& \Sigma A \\
                A' \& B' \& C' \& \Sigma A'. \\
            };

            \draw[math]
                (m-1-1) edge node {f} (m-1-2)
                    edge node {a} (m-2-1)
                (m-1-2) edge node {g} (m-1-3)
                    edge node {b} (m-2-2)
                (m-1-3) edge node {h} (m-1-4)
                    edge node {c} (m-2-3)
                (m-1-4) edge node {\Sigma a} (m-2-4)

                (m-2-1) edge node {f'} (m-2-2)
                (m-2-2) edge node {g'} (m-2-3)
                (m-2-3) edge node {h'} (m-2-4);
        \end{tikzpicture}
    \end{center}
    Furthermore if \( a, b, \) and \( c \) are isomorphisms, then \( (a, b, c) \) is called an \emph{isomorphism of triangles} and the two triangles are said to be \emph{isomorphic}.
\end{definition}

There are multiple different, but equivalent, definitions of a triangulated category with different axioms TODO: UTFYLL, CITE. This thesis will use the following definition.

\begin{definition}[Triangulated category]
    \label{def:triangulated_category}
    Let \( \Tc \) be an additive category, let \( \Sigma: \Tc \to \Tc \) be an additive autoequivalence, and let \( \Delta \) be a class of triangles (\autoref{def:triangles}) in \( \Tc \).

    Furthermore, let \( \Tc \), \( \Sigma \) and \( \Delta \) satisfy the following four axioms:
    \begin{enumerate}[label={(\bfseries TR\arabic*)}]
        \item {
            The following three points must be satisfied:
            \begin{enumerate}
                \item {
                    For any \( f: X \to Y \) in \( \Tc \), there is some object \( C_f \in \Tc \), called a \emph{cone of \( f \)} along with morphisms \( \iota_f \) and \( \pi_f \), such that there is a triangle, called the \emph{standard triangle of \( f \)},
                    \begin{center}
                        \begin{tikzpicture}
                            \diagram{m}{1cm}{1cm} {
                                X \& Y \& C_f \& \Sigma X \\
                            };

                            \draw[math]
                                (m-1-1) edge node {f} (m-1-2)
                                (m-1-2) edge node {\iota_f} (m-1-3)
                                (m-1-3) edge node {\pi_f} (m-1-4);
                        \end{tikzpicture}
                    \end{center}
                    in \( \Delta \).
                }
                \item {
                    For any \( X \in \Tc \), the \emph{trivial triangle of \( X \)}, which is the following triangle
                    \begin{center}
                        \begin{tikzpicture}
                            \diagram{m}{1cm}{1cm} {
                                X \& X \& 0 \& \Sigma X \\
                            };

                            \draw[math]
                                (m-1-1) edge node {\Id_X} (m-1-2)
                                (m-1-2) edge (m-1-3)
                                (m-1-3) edge (m-1-4);
                        \end{tikzpicture}
                    \end{center}
                    is in \( \Delta \).
                }
                \item {
                    \( \Delta \) is closed under isomorphisms of triangles.
                }
            \end{enumerate}
        }
        \item {
            For any object \( A \in \Tc \), let \( \phi_A \) be the natural isomorphism from \( \Sigma^{-1} \Sigma A \) to \( A \).

            Then for any triangle
            \begin{center}
                \begin{tikzpicture}
                    \diagram{m}{1cm}{1cm} {
                        A \& B \& C \& \Sigma A \\
                    };
        
                    \draw[math]
                        (m-1-1) edge node {f} (m-1-2)
                        (m-1-2) edge node {g} (m-1-3)
                        (m-1-3) edge node {h} (m-1-4);
                \end{tikzpicture}
            \end{center}
            in \( \Delta \), both the \emph{left/right shifted triangles}
            \begin{center}
                % \begin{tikzpicture}
                %     \diagram{m}{1cm}{1cm} {
                %         B \& C \& \Sigma A \& \Sigma B, \\
                %     };
        
                %     \draw[math]
                %         (m-1-1) edge node {g} (m-1-2)
                %         (m-1-2) edge node {h} (m-1-3)
                %         (m-1-3) edge node {-\Sigma f} (m-1-4);
                % \end{tikzpicture}
                % and
                % \begin{tikzpicture}
                %     \diagram{m}{1cm}{1cm} {
                %         \Sigma^{-1}(C) \& A \& B \& C, \\
                %     };
        
                %     \draw[math]
                %         (m-1-1) edge node {- \Sigma^{-1}(h)} (m-1-2)
                %         (m-1-2) edge node {f} (m-1-3)
                %         (m-1-3) edge node {g} (m-1-4);
                % \end{tikzpicture}
                \begin{tikzpicture}
                    \diagram{m}{1cm}{1cm} {
                        B \& C \& \Sigma A \& \Sigma B \&[-8mm] \text{and} \&[-8mm] \Sigma^{-1}(C) \& A \& B \& C \\
                    };
        
                    \draw[math]
                        (m-1-1) edge node {g} (m-1-2)
                        (m-1-2) edge node {h} (m-1-3)
                        (m-1-3) edge node {-\Sigma f} (m-1-4)
                        (m-1-6) edge node {-\phi_A \circ \Sigma^{-1} h} (m-1-7)
                        (m-1-7) edge node {f} (m-1-8)
                        (m-1-8) edge node {g} (m-1-9);
                \end{tikzpicture}
            \end{center}
            respectively, are also in \( \Delta \).
        }
        \item {
            Let the solid part of the following diagram
            \begin{center}
                \begin{tikzpicture}
                    \diagram{m}{1cm}{1cm} {
                        A \& B \& C \& \Sigma A \\
                        A' \& B' \& C' \& \Sigma A'. \\
                    };
        
                    \draw[math]
                        (m-1-1) edge node {f} (m-1-2)
                            edge node {a} (m-2-1)
                        (m-1-2) edge node {g} (m-1-3)
                            edge node {b} (m-2-2)
                        (m-1-3) edge node {h} (m-1-4)
                            edge[dashed] node {c} (m-2-3)
                        (m-1-4) edge node {\Sigma a} (m-2-4)
        
                        (m-2-1) edge node {f'} (m-2-2)
                        (m-2-2) edge node {g'} (m-2-3)
                        (m-2-3) edge node {h'} (m-2-4);
                \end{tikzpicture}
            \end{center}
            commute, and let both of the rows be triangles in \( \Delta \).
            
            Then there exists a morphism \( c: C \to C' \) such that the above diagram including \( c \) commutes, i.e. \( (a, b, c) \) becomes a morphism of triangles.
        }
        \item {
            Let the solid part of the following diagram
            \begin{center}
                \begin{tikzpicture}
                    \diagram{m}{1cm}{1cm} {
                        A \& B \& C \& \Sigma A \\
                        A \& D \& E \& \Sigma A \\
                        \& F \& F \& \Sigma B \\
                        \& \Sigma B \& \Sigma C \\
                    };
        
                    \draw[math]
                        (m-1-1) edge node {f} (m-1-2)
                            edge[equal] (m-2-1)
                        (m-1-2) edge node {g} (m-1-3)
                            edge (m-2-2)
                        (m-1-3) edge (m-1-4)
                            edge[dashed] (m-2-3)
                        (m-1-4) edge[equal] (m-2-4)
        
                        (m-2-1) edge (m-2-2)
                        (m-2-2) edge (m-2-3)
                            edge (m-3-2)
                        (m-2-3) edge (m-2-4)
                            edge[dashed] (m-3-3)
                        (m-2-4) edge node {\Sigma f} (m-3-4)

                        (m-3-2) edge[equal] (m-3-3)
                            edge node[swap] {h} (m-4-2)
                        (m-3-3) edge node {h} (m-3-4)
                            edge node {\Sigma(g) \circ h} (m-4-3)

                        (m-4-2) edge node {\Sigma g} (m-4-3);
                \end{tikzpicture}
            \end{center}
            commute, and let the top two rows and the leftmost column that is four tall be three triangles in \( \Delta \).

            Then there exist some morphisms where there are dashed arrows such that the rightmost four tall column is a triangle in \( \Delta \) and the entire diagram, including the dashed arrows, commute.
        }
    \end{enumerate}

    Then \( \tuple{ \Tc, \Sigma, \Delta } \), or shortened to just \( \Tc \), is called an \emph{triangulated category}, \( \Sigma \) is called the \emph{shift functor of \( \Tc \)} (or sometimes in other litterature, the \emph{suspension functor of \( \Tc \)}), and every triangle in \( \Delta \) is called a \emph{distinguished triangle in \( \Tc \)} (or sometimes in other litterature, \emph{exact triangle in \( \Tc \)}).
\end{definition}

Although the definition is axiomatic and may seem arbitrary, there are surprisingly many different categories that admit the structure of a triangulated category.

Some of the axioms of a triangulated category are superflous in the fact that they are implied by the other axioms. An example of this is that given {\bf TR1}, {\bf TR3} and the left shift part of {\bf TR2}, then it implies the right shift part as well. The details are contained in the following lemma.

\begin{lemma}
    Let \( \tuple{ \Tc, \Sigma, \Delta } \) satisfy the axioms {\bf TR1} and {\bf TR3} from \autoref{def:triangulated_category}, as well as the left shift part of {\bf TR2}.

    Then the right shift part of {\bf TR2} is also satisfied.
\end{lemma}
\begin{proof}
    As a reminder, \( \phi_A \) denotes the natural isomorphism from \( \Sigma^{-1} \Sigma A \) to \( A \).

    Given a distinguished triangle
    \begin{center}
        \begin{tikzpicture}
            \diagram{m}{1cm}{1cm} {
                X \& Y \& Z \& \Sigma X, \\
            };

            \draw[math]
                (m-1-1) edge node {f} (m-1-2)
                (m-1-2) edge node {g} (m-1-3)
                (m-1-3) edge node {h} (m-1-4);
        \end{tikzpicture}
    \end{center}
    and the natural isomorphisms
    \[
        \psi_A: \Sigma \Sigma^{-1} A \to A,
    \]
    one gets the following isomorphism of triangles
    \begin{center}
        \begin{tikzpicture}
            \diagram{m}{1cm}{1cm} {
                X \& Y \& Z \& \Sigma X \\
                \Sigma^{-1} \Sigma X \& Y \& \Sigma \Sigma^{-1} Z \& \Sigma \Sigma^{-1} \Sigma X \\
            };

            \draw[math]
                (m-1-1) edge node {f} (m-1-2)
                    edge node {\phi_X^{-1}} (m-2-1)
                (m-1-2) edge node {g} (m-1-3)
                    edge[equal] (m-2-2)
                (m-1-3) edge node {h} (m-1-4)
                    edge node {\psi_Z^{-1}} (m-2-3)
                (m-1-4) edge node {\psi_{\Sigma X}^{-1}} (m-2-4)

                (m-2-1) edge node {f \circ \phi_X} (m-2-2)
                (m-2-2) edge node {\psi_Z^{-1} \circ g} (m-2-3)
                (m-2-3) edge node {\Sigma \Sigma^{-1} h} (m-2-4);
        \end{tikzpicture}
    \end{center}
    which implies the bottom row is a distinguished triangle.

    Then by \cite[Chapter 1, Subsection 1.3, Lemma]{Happel_1988}, it follows that the top row in the following diagram is a distinguished triangle
    % \begin{center}
    %     \begin{tikzpicture}
    %         \diagram{m}{1cm}{1cm} {
    %             X \& Y \& \Sigma \Sigma^{-1} Z \& \Sigma X. \\
    %         };

    %         \draw[math]
    %             (m-1-1) edge node {f} (m-1-2)
    %             (m-1-2) edge node {\psi^{-1} \circ g} (m-1-3)
    %             (m-1-3) edge node {h \circ \psi} (m-1-4);
    %     \end{tikzpicture}
    % \end{center}
    \begin{center}
        \begin{tikzpicture}
            \diagram{m}{1cm}{1cm} {
                \Sigma^{-1} Z \& \Sigma^{-1} \Sigma X \& Y \& \Sigma \Sigma^{-1} Z \\
                \Sigma^{-1} Z \& X \& Y \& Z. \\
            };

            \draw[math]
                (m-1-1) edge node {-\Sigma^{-1} h} (m-1-2)
                    edge[equal] (m-2-1)
                (m-1-2) edge node {f \circ \phi_X} (m-1-3)
                    edge node {\phi_X} (m-2-2)
                (m-1-3) edge node {\psi_Z^{-1} \circ g} (m-1-4)
                    edge[equal] (m-2-3)
                (m-1-4) edge node {\psi_Z} (m-2-4)

                (m-2-1) edge node {-\phi_X \circ \Sigma^{-1} h} (m-2-2)
                (m-2-2) edge node {f} (m-2-3)
                (m-2-3) edge node {g} (m-2-4);
        \end{tikzpicture}
    \end{center}
    The above diagram commutes, and every vertical arrow is an isomorphism. \emph{However, the rightmost arrow is not the supspension of the leftmost arrow, and so this is not a triangle isomorphism.} WIP
\end{proof}
