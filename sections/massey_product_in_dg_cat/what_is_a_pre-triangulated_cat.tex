There are multiple definitions of an algebraic triangulated categoery. In this thesis, the definition used will be from TODO:Cite Jasso--Muro.

% TODO Cite: Jasso--Muro
\begin{definition}[\( \C_{\dg}(\Mod(R)) \)]
    Let \( R \) be a commutative ring with identity.

    Then let \emph{\( \C_{\dg}(\Mod(R)) \)} be a DG-category defined as follows
    \begin{enumerate}
        \item {
            \( \Obj(\C_{\dg}(\Mod(R))) := \Obj(\C(\Mod(R))) \).
        }
        \item {
            For \( A, B \in \C_{\dg}(\Mod(R)) \) let
            \[ 
                \tuple{\C_{\dg}(\Mod(R))(A, B)}_i := \bigoplus_{j \in \Zb} \Mod(R)(A_j, B_{j + i})
            \]
            with
            \[
                \C_{\dg}(\Mod(R))(A, B) = \bigoplus_{i \in \Zb} \tuple{\C_{\dg}(\Mod(R))(A, B)}_i.
            \]

            Let \( d_A \) and \( d_B \) be the differential of \( A \) and \( B \) as objects of \( \C(\Mod(R)) \), respectively.

            % TODO: This is slight abuse of notation (d_B and d_A are defined on A, B, not A_j, B_{j + i}). Should I fix?
            Let the differential of \( \C_{\dg}(\Mod(R))(A, B) \) be defined as follows
            \begin{align*}
                d_i: \tuple{\C_{\dg}(\Mod(R))(A, B)}_i &\to \tuple{\C_{\dg}(\Mod(R))(A, B)}_{i + 1} \\
                f &\mapsto d_B \circ f - (-1)^i f \circ d_A
            \end{align*}
        }
        \item {
            For \( A, B, C \in \C_{\dg}(\Mod(R)) \), let
            \[
                \circ_{\C_{\dg}(\Mod(R))}: \C_{\dg}(\Mod(R))(B, C) \otimes \C_{\dg}(\Mod(R))(A, B) \to \C_{\dg}(\Mod(R))(A, C)
            \]
            % TODO: Explain more?
            be defined as expected.
        }
    \end{enumerate}
\end{definition}

% SRC: JAsso--Muro 2023, p. 29
\begin{definition}[\( \Sigma_{\C_{\dg}(Mod(R))} \)]
    Let the functor \( \Sigma_{\C_{\dg}(Mod(R))} \) be defined as follows
    \begin{enumerate}
        \item {
            For \( A \in \C_{\dg}(Mod(R)) \), let
            \[
                \Sigma_{\C_{\dg}(Mod(R))}(A) := \Sigma_{\C(\Mod(R))}(A)
            \]
        }
        \item {
            For \( f \in \C_{\dg}(Mod(R))(A, B) \) homogeneous of degree \( i \), let
            \begin{center}
                \begin{tikzpicture}
                    \diagram{m}{2cm}{1cm} {
                        \Sigma(A): \& \cdots \& A_0 \& A_1 \& A_2 \& \cdots \\
                        \Sigma(B): \& \cdots \& B_i \& B_{i + 1} \& B_{i + 2} \& \cdots \\
                    };

                    \draw[math]
                        (m-1-1) edge node {\Sigma_{\C_{\dg}(Mod(R))}(f)} (m-2-1)
                        (m-1-2) edge (m-1-3)
                        (m-1-3) edge (m-1-4)
                            edge node {(-1)^i f_0} (m-2-3)
                        (m-1-4) edge (m-1-5)
                            edge node {(-1)^i f_1} (m-2-4)
                        (m-1-5) edge (m-1-6)
                            edge node {(-1)^i f_2} (m-2-5)

                        (m-2-2) edge (m-2-3)
                        (m-2-3) edge (m-2-4)
                        (m-2-4) edge (m-2-5)
                        (m-2-5) edge (m-2-6);
                \end{tikzpicture}
            \end{center}
            % TODO: Keep this?
            Or equivalently: for \( j \in \Zb \), let \( \tuple{\Sigma_{\C_{\dg}(Mod(R))}(f)}^j = (-1)^i f^{j + 1} \).
        }
    \end{enumerate}
\end{definition}

\begin{remark}
    Consider the following homogeneous of degree \( -1 \) map for \( A \in \C_{\dg}(\Mod(R)) \)
    \begin{center}
        \newcommand{\height}{2cm}
        %
        \mmznext{meaning to context=\height}
        \begin{tikzpicture}
            \diagram{m}{\height}{1cm} {
                A: \\
                \Sigma(A): \\
            };
            
            \draw[math]
                (m-1-1) edge node {\sigma_A} (m-2-1);
        \end{tikzpicture}
        %
        \mmznext{meaning to context=\height}
        \begin{tikzpicture}
            \diagramorigin{m}{\height}{3cm} {
                \cdots \& A_{-1} \& A_0 \& A_1 \& \cdots \\
                \cdots \& A_0 \& A_1 \& A_2 \& \cdots \\
            };

            \draw[math]
                (m-1-1) edge (m-1-2)
                (m-1-2) edge (m-1-3)
                    edge node {\Id} (m-2-1)
                (m-1-3) edge (m-1-4)
                    edge node {\Id} (m-2-2)
                (m-1-4) edge (m-1-5)
                    edge node {\Id} (m-2-3)
                (m-1-5) edge node {\Id} (m-2-4)
                
                (m-2-1) edge (m-2-2)
                (m-2-2) edge (m-2-3)
                (m-2-3) edge (m-2-4)
                (m-2-4) edge (m-2-5);
        \end{tikzpicture}
    \end{center}

    Consider this (suggestively named) homogeneous of degree \( 1 \) morphism
    \begin{center}
        \newcommand{\height}{2cm}
        %
        \mmznext{meaning to context=\height}
        \begin{tikzpicture}
            \diagram{m}{\height}{1cm} {
                \Sigma(A): \\
                A: \\
            };
            
            \draw[math]
                (m-1-1) edge node {\sigma_A^{-1}} (m-2-1);
        \end{tikzpicture}
        %
        \mmznext{meaning to context=\height}
        \begin{tikzpicture}
            \diagramorigin{m}{\height}{3cm} {
                \cdots \& A_0 \& A_1 \& A_2 \& \cdots \\
                \cdots \& A_{-1} \& A_0 \& A_1 \& \cdots \\
            };

            \draw[math]
                (m-1-1) edge (m-1-2)
                    edge node {\Id} (m-2-2)
                (m-1-2) edge (m-1-3)
                    edge node {\Id} (m-2-3)
                (m-1-3) edge (m-1-4)
                    edge node {\Id} (m-2-4)
                (m-1-4) edge (m-1-5)
                    edge node {\Id} (m-2-5)
                
                (m-2-1) edge (m-2-2)
                (m-2-2) edge (m-2-3)
                (m-2-3) edge (m-2-4)
                (m-2-4) edge (m-2-5);
        \end{tikzpicture}
    \end{center}

    One can see that
    \[
        \sigma_A \circ \sigma_A^{-1} = \Id_{\Sigma(A)}
    \]
    and
    \[
        \sigma_A^{-1} \circ \sigma_A = \Id_A.
    \]
    Therefore, \( \sigma_A^{-1} \) is the inverse of \( \sigma_A \) in \( \C_{\dg}(\Mod(R)) \).

    Furthermore, for \( f \in \C_{\dg}(\Mod(R))(A, B) \), homogeneous of degree \( i \), consider the morphism
    \[
        \sigma_B \circ f \circ \sigma_A^{-1}.
    \]
    Looking at what the morphism is doing
    \begin{center}
        \newcommand{\height}{3cm}
        %
        \mmznext{meaning to context=\height}
        \begin{tikzpicture}
            \diagram{m}{\height}{1cm} {
                \Sigma(A): \\
                A: \\
                B: \\
                \Sigma(B): \\
            };

            \draw[math]
                (m-1-1) edge node {\sigma_A^{-1}} (m-2-1)

                (m-2-1) edge node {f} (m-3-1)

                (m-3-1) edge node {\sigma_B} (m-4-1);
        \end{tikzpicture}
        %
        \mmznext{meaning to context=\height}
        \begin{tikzpicture}
            \diagramorigin{m}{\height}{3cm} {
                \cdots \& A_0 \& A_1 \& A_2 \& \cdots \\
                \cdots \& A_{-1} \& A_0 \& A_1 \& \cdots \\
                \cdots \& B_{i-1} \& B_i \& B_{i + 1} \& \cdots \\
                \cdots \& B_i \& B_{i + 1} \& B_{i + 2} \& \cdots \\
            };

            \draw[math]
                (m-1-1) edge (m-1-2)
                    edge node {\Id} (m-2-2)
                (m-1-2) edge (m-1-3)
                    edge node {\Id} (m-2-3)
                (m-1-3) edge (m-1-4)
                    edge node {\Id} (m-2-4)
                (m-1-4) edge (m-1-5)
                    edge node {\Id} (m-2-5)

                (m-2-1) edge (m-2-2)
                (m-2-2) edge (m-2-3)
                    edge node {f_{-1}} (m-3-2)
                (m-2-3) edge (m-2-4)
                    edge node {f_0} (m-3-3)
                (m-2-4) edge (m-2-5)
                    edge node {f_1} (m-3-4)

                (m-3-1) edge (m-3-2)
                (m-3-2) edge (m-3-3)
                    edge node {\Id} (m-4-1)
                (m-3-3) edge (m-3-4)
                    edge node {\Id} (m-4-2)
                (m-3-4) edge (m-3-5)
                    edge node {\Id} (m-4-3)
                (m-3-5) edge node {\Id} (m-4-4)

                (m-4-1) edge (m-4-2)
                (m-4-2) edge (m-4-3)
                (m-4-3) edge (m-4-4)
                (m-4-4) edge (m-4-5);
        \end{tikzpicture}
    \end{center}
    % TODO: Impliserar dette at det er naturleg iso?
    one can see that it is exacly equal to \( (-1)^i\Sigma(f) \).
\end{remark}

% TODO Cite: Jasso--Muro
% TODO: Elaborate on enriched functor?
\begin{definition}[DG-functor]
    An enriched functor between two DG-categories is called a \emph{DG-functor}.
\end{definition}

% MS-question: Remark below. -> Probably OK.
\begin{remark}
    % Bondal--Kapranov has as definition
    Enriched functor between DG-categories implies it preserves differentials and grading? TODO
\end{remark}

% SRC: Berest--Mehrle 2017 LN
% \begin{definition}[DG-natural transformation]
%     Let \( F, G: \Ac \to \Bc \) be two DG-functors.

%     Then a collection of morphisms
%     \[
%         \alpha = \set{ \alpha_A \in \Bc(F(A), G(A)) \mid A \in \Ac }
%     \]

%     TODO: Find secondary source, can't see the connection to nlab
% \end{definition}

% TODO: Need to show that it's a category?
% MS-Question: Correct? Berest--Mehrle (LN) has another def. -> Subscript dg betyr enriched.
% \begin{definition}[\( \Fun(\Ac, \Bc) \)]
%     Let \( \Ac, \Bc \) be two DG-categories over the same commutative ring \( R \).
    
%     Then let \( \Fun(\Ac, \Bc) \) denote the category of all DG-functors from \( \Ac \) to \( \Bc \), with morphisms being DG-natural transformations.
% TODO: Can't use this unless I have a definition of "DG-natural transformations".
% \end{definition}

% TODO-Q: Does symmetry isomorphism commute with morphisms? I would guess so....
% TODO-Q: The fact that \sum_{i \in \Zb} \sum_{j \in \Zb} G(f)_j \otimes \eta_{A, i} = G(f) \otimes \eta_A may only be true if chain complexes are viewed as coproducts.
\begin{remark}[Functor category structure from Borceux]
    \begin{center}
        \newcommand{\height}{1cm}
        %
        \mmznext{meaning to context=\height}
        \begin{tikzpicture}
            \diagram{m}{\height}{1cm} {
                \left( \prod_{A \in \Cc} \C_{\dg}(\Mod(R))(F(A), G(A)) \right) \otimes \Cc(A', A'') \\
                \C_{\dg}(\Mod(R))(F(A'), G(A')) \otimes \C_{\dg}(G(A'), G(A'')) \\
                \C_{\dg}(G(A'), G(A'')) \otimes \C_{\dg}(\Mod(R))(F(A'), G(A')) \\
                \C_{\dg}(F(A'), G(A'')) \\
            };

            \draw[math]
                (m-1-1) edge node {\pi_{A'} \otimes G_{A', A''}} (m-2-1)

                (m-2-1) edge node {s} (m-3-1)

                (m-3-1) edge node {\circ} (m-4-1);
        \end{tikzpicture}
        %
        \mmznext{meaning to context=\height}
        \begin{tikzpicture}
            \diagram{m}{\height}{1cm} {
                \eta \otimes f \\
                \eta_A \otimes G(f) \\
                \sum_{i \in \Zb} \sum_{j \in \Zb} (-1)^{ij} G(f)_j \otimes \eta_{A, i} \\
                \sum_{i \in \Zb} \sum_{j \in \Zb} (-1)^{ij} G(f)_j \circ \eta_{A, i} \\
            };

            \draw[math]
                (m-1-1) edge[maps to] (m-2-1)

                (m-2-1) edge[maps to] (m-3-1)

                (m-3-1) edge[maps to] (m-4-1);
        \end{tikzpicture}
    \end{center}
    Name the entire above composition of chain complex morphisms for \( \psi \).

    For any \( k \in \Zb \) let \( \iota_{n,k} \) denote the inclusion
    \[
        \iota_n: \tuple{ \prod_{A \in \Cc} \C_{\dg}(\Mod(R))(F(A), G(A)) }_n \otimes \Cc(A', A'')_{n - k} \hookrightarrow \tuple{ \tuple{ \prod_{A \in \Cc} \C_{\dg}(\Mod(R))(F(A), G(A)) } \otimes \Cc(A', A'') }_k
    \]

    Take the adjoint of (TODO:refrence above morphism) morphism gives the chain complex morphism
    \[
        \set{\phi_n}_{n \in \Zb}, \text{ where } \phi_n = \prod_{k \in \Zb} \tuple{ \eta_n \mapsto \psi_k \circ \iota_{n,k}(\eta_n \otimes ? ) }
    \]
    \begin{center}
        \newcommand{\height}{1cm}
        %
        \mmznext{meaning to context=\height}
        \begin{tikzpicture}
            \diagram{m}{\height}{1cm} {
                \prod_{A \in \Cc} \C_{\dg}(\Mod(R))(F(A), G(A)) \\
                \left[ \Cc(A', A''), \C_{\dg}(F(A'), G(A'')) \right] \\
            };

            \draw[math]
                (m-1-1) edge node {\phi} (m-2-1);
        \end{tikzpicture}
        %
        \mmznext{meaning to context=\height}
        \begin{tikzpicture}
            \diagram{m}{\height}{1cm} {
                \eta \\
                TODO \\
            };
        \end{tikzpicture}
    \end{center}
\end{remark}

% SRC: Berest--Mehrle 2017 LN
% TODO: Must \Ac be small in order to be well defined??
\begin{definition}[\( \Fun_{\dg}(\Ac, \Bc) \)]
    \label{def:dg_functor_category}
    Let \( \Ac \) and \( \Bc \) be DG-categories over \( R \).

    Then let \( \Fun_{\dg}(\Ac, \Bc) \) be the following DG-category:
    \begin{enumerate}
        \item{
            Let \( \Obj(\Fun_{\dg}(\Ac, \Bc)) \) be the class of every DG-functor from \( \Ac \) to \( \Bc \)
        }
        \item{
            % TODO: Maybe fix the wording a bit here?
            For \( F, G \in \Fun_{\dg}(\Ac, \Bc) \), let
            \[
                \eta^i \in \prod_{A \in \Ac} \tuple{\Bc\tuple{F(A), G(A)}}_i
            \]
            be written as a collection of elements \( \eta_A^i \in \tuple{\Bc\tuple{F(A), G(A)}}_i \) for each \( A \in \Ac \).
            
            Then let
            \[
                \tuple{\Fun_{\dg}(\Ac, \Bc)(F, G)}_i
                \subseteq \prod_{A \in \Ac} \tuple{\Bc\tuple{F(A), G(A)}}_i
            \]
            be the subset such that for any \( A, B \in \Ac \) and any \( f \in \Ac\tuple{A, B} \) the following diagram in \( \Bc \), where \( \eta_A^i \) and \( \eta_B^i \) are considered as homogeneous elements of degree \( i \), commutes
            \begin{center}
                \begin{tikzpicture}
                    \diagram{m}{1cm}{1cm} {
                        F(A) \& G(A) \\
                        F(B) \& G(B) \\
                    };

                    \draw[math]
                        (m-1-1) edge node {\eta_A^i} (m-1-2)
                            edge node {F(f)} (m-2-1)
                        (m-1-2) edge node {G(f)} (m-2-2)

                        (m-2-1) edge node {\eta_B^i} (m-2-2);
                \end{tikzpicture}
            \end{center}
            % TODO: Show this is a R-module

            Then let
            \[
                \Fun_{\dg}(\Ac, \Bc)(F, G) := \bigoplus_{i \in \Zb} \tuple{\Fun_{\dg}(\Ac, \Bc)(F, G)}_i
            \]
            with the differential acting component wise for each \( i \in \Zb \) and \( A \in \Ac \) as follows
            \[
                d: \eta_A^i \mapsto d_{\Bc(F(A), G(A))}(\eta_A^i)
            \]
            % TODO: Show that this differential has correct codomain.
        }
        \item {
            Let composition be defined as expected.
            % TODO: Show that composition is wll defined.
        }
        TODO -- Need resources to define this
        WIP: In notes, have tried showing that the differential is well defined.
        % Current ideas: MS-Question
        % End
        % Berest--Mehrle 2017 def?
    \end{enumerate}
\end{definition}

\begin{definition}[Opposite DG-category]
    Let \( \Cc \) be a DG-category.

    Then let \( \Cc^{op} \) be the DG-category defined as follows
    \begin{enumerate}
        \item {
            \( \Obj(\Cc^{op}) := \Obj(\Cc) \)
        }
        \item {
            For \( A, B \in \Cc^{op} \), let \( \Cc^{op}(A, B) := \Cc(B, A) \).
        }
        \item {
            For \( A, B, C \in \Cc^{op} \), with \( f \in \Cc^{op}(B, C) \) and \( g \in \Cc^{op}(A, B) \) homogeneous elements of degree \( d_f \) and \( d_g \) respectively.

            Let composition be defined as
            \begin{align*}
                \circ_{\Cc^{op}}: \Cc^{op}(B, C) \otimes \Cc^{op}(A, B) &\to \Cc^{op}(A, C) \\
                f \otimes g &\mapsto (-1)^{d_f d_g} \circ_{\Cc} (g \otimes f)
            \end{align*}
            % TODO: Why can I say this? Need some statement saying all elementary tensors are a sum of homogeneous elementary tensors? As well as saying that the set of elementary tensors generate all elements in the tensor product?
            and extended to all other elementary tensors.
        }
    \end{enumerate}
\end{definition}

% TODO: Cite: Jasso--Muro
% No mention of the ring in the notation? -> Implied since DG-category is over a ring!
\begin{definition}[\( \dgMod_{\dg}(\Cc) \)]
    Let \( \Cc \) be a DG-category over \( R \).

    % TODO: Why "Right"?
    Then define the \emph{DG-category of (right) DG \( \Cc \)-modules} as
    \[
        \dgMod_{\dg}(\Cc) := \Fun_{\dg}(\Cc^{op}, \C_{\dg}(\Mod(R))).
    \]
    Objects in \( \dgMod_{\dg}(\Cc) \) are called \emph{DG-modules over \( \Cc \)}.

    
\end{definition}

% TODO: Why is \Cc(-, A) a functor into \C_{\dg}(\Mod(R))?
\begin{definition}[DG Yoneda embedding]
    \label{def:DG_Yoneda_embedding}
    Let \( \Cc \) be a DG-category over \( R \).
    
    Then let \( \mathbf{h} \) be the functor defined as follows
    \begin{align*}
        \mathbf{h}: \Cc &\to \dgMod_{\dg}(\Cc) \\
        A &\mapsto \Cc(-, A)
    \end{align*}

    This functor is called the \emph{DG Yoneda embedding of \( \Cc \)}.
\end{definition}

% TODO: Add Yoneda embedding identifies \Cc with a full subcategory of \dgMod_dg(\Cc)?

% TODO: Could define this for Mod(R)-enriched categories?
\begin{definition}[0th cohomology category of a DG category]
    Let \( \Cc \) be a DG category over \( R \).

    Then let \( H^0(\Cc) \) be the following (enriched over \( \Mod(R) \) TODO) category defined as follows
    \begin{enumerate}
        \item {
            Let \( \Obj(H^0(\Cc)) := \Obj(\Cc) \).
        }
        \item {
            Let \( A, B \in H^0(\Cc) \).

            Then let \( H^0(\Cc)(A, B) := H^0(\Cc(A, B)) \).
        }
        \item {
            Let \( A, B, C \in H^0(\Cc) \) with \( f_1 \in H^0(A, B) \) and \( f_2 \in H^0(B, C) \).

            Then by \autoref{lem:massey_product_in_dg_cat/massey_product_definition/exist_lifting_h_star}, there exists \( g_1 \in \Cc(A, B) \), and \( g_2 \in \Cc(B, C) \) such that \( \class{g_1} = f_1 \) and \( \class{g_2} = f_2 \).

            % TODO: Slight abuse of notation taking a "class" of an element of a chain complex.
            % TODO: Need to show that this is well defined?
            Then let composition be defined on elementary tensors as follows
            \begin{align*}
                \circ_{H^0(\Cc)}: H^0(\Cc)(B, C) \otimes H^0(\Cc)(A, B) &\to H^0(\Cc)(A, C) \\
                f_2 \otimes f_1 &\mapsto \class{ \circ_{\Cc}(g_2 \otimes g_1) }
            \end{align*}
        }
    \end{enumerate}
\end{definition}

% TODO: What are the triangles? Is the shift functor correct on maps?
% TODO: Incorrect/abuse of notation, how does the shift work on maps?
\begin{theorem}
    Let \( \Cc \) be a DG-category over \( R \). Let \( \Sigma_{\C_{\dg}(\Mod(R))} \) be the shift functor on \( \C_{\dg}(\Mod(R)) \).

    Then \( H^0(\dgMod_{\dg}(\Cc)) \) is a triangulated category with the shift functor \( \Sigma(-) = \Sigma_{\C_{\dg}(\Mod(R))} \circ - \).
\end{theorem}
\begin{proof}
    TODO
\end{proof}

% MS-Question: Have seen definition of small category that is that the class of iso classes are small, not the class of objects. What is correct? Are they equivalent? -> Essentially small.

\begin{definition}[Acyclic DG-module]
    Let \( \Cc \) be a DG-category over a commutative ring (with identity) \( R \). Furthermore, let \( A \in \dgMod_{\dg}(\Cc) \) be a DG-module over \( \Cc \).

    Then \( A \) is called \emph{acyclic} if for any \( X \in \Cc \), one has that \( A(X) \in \C_{\dg}(\Mod(R)) \) is acyclic, i.e. \( H^*(A(X)) = 0 \).
\end{definition}

% TODO: Is \dgMod_{\dg}(\Cc) abelian?/Have kernels?
\begin{definition}[DG-projective module]
    Let \( P \in \dgMod_{\dg}(\Cc) \) be a DG-module.

    Then \( P \) is called a \emph{DG-projective module over \( \Cc \)} if:
    
    For any DG-module \( A \in \dgMod_{\dg}(\Cc) \) and any epimorphism \( f \in \dgMod_{dg}(\Cc)(A, P) \) where \( \ker(f) \in \dgMod_{\dg}(\Cc) \) is acyclic. Then \( f \) is split.
\end{definition}

% TODO: Various definitions and idiosyncracies. Which is correct?
    % Acyclic kernel? Projective objects? Spanned?
    % Krause 07 -> Compact objects, maybe more.
    % Keller 94 -> Another definition of derived DG category.
% TODO: Following def from Krause 07, but not explicitly written down. Is it correct?
\begin{definition}[Derived DG-category]
    Let \( \Cc \) be a DG-category.

    Then the \emph{derived DG-category} of \( \Cc \), denoted \( \D(\Cc) \), is defined as the full subcategory of \( H^0(\dgMod_{\dg}(\Cc)) \) spanned by the objects of \( \dgMod_{\dg}(\Cc) \) that are DG-projective.
\end{definition}

% MS-Question: What is coproduct for the derived category?
% Cite: Jasso--Muro p.31, only a statement, no proof
\begin{proposition}
    \( \D(\Cc) \) is closed under arbitrary coproduct.
\end{proposition}
\begin{proof}
    TODO
\end{proof}

% MS-Quastion: Why are small categories sometimes mentioned in def of derived category? Something to do with localization being well defined?

% MS-Question: Arbitrary coproducts <=> infinite coproducts? -> Need triangulated property for this to make sense.
% Cite: Krause 07 p. 29
\begin{definition}[Compact objects of a category]
    Let \( \Cc \) be a triangulated category with arbitrary coproduct. Let \( X \in \Cc \). 
    
    Then if \( X \) has the following property:
    
    For any index set \( I \) and any morphism \( f: X \to \coprod_{i \in I} Y_i \), there is a finite index set \( J \subseteq I \) such that \( f \) factors through \( \coprod_{j \in J} Y_j \).
    
    Then define \( X \) as a \emph{compact object in \( \Cc \)}.
\end{definition}

% TODO: Is the derived category triangulated? Need to be in order for previous def to apply.
\begin{definition}[Perfect derived DG-category \( \D^c(\Cc) \)]
    Let \( \D(\Cc) \) be the derived DG-category of \( \Cc \).

    Then define \( \D^c(\Cc) \) to be the full subcategory of \( \D(\Cc) \) consisting of all compact objects in \( \D(\Cc) \). This is called the \emph{perfect derived DG-category of \( \Cc \)}.
\end{definition}

% TODO: Cite: Jasso--Muro says so
\begin{proposition}
    \( \D^c(\Cc) \) is triangulated.
\end{proposition}
\begin{proof}
    TODO
\end{proof}

% TODO: Is this even a functor? Or well defined? Probably need to show well defined and that every morphism in H^0(A) has a representative in A.
\begin{definition}[\( H^0 \)-induced functor]
    \label{def:H^0-induced_functor}
    Let \( \Ac \) and \( \Bc \) be two DG-categories, and let \( F: \Ac \to \Bc \) be a functor between them.

    Then define the functor \( H^0(F) \) as follows:
    \begin{align*}
        H^0(F): H^0(\Ac) &\to H^0(\Bc) \\
        A &\mapsto F(A) \\
        (H^0(f): A \to B) &\mapsto (H^0(F(f)): F(A) \to F(B)) 
    \end{align*}

    This is called the \( H^0 \)-induced functor of \( F \).
\end{definition}

\begin{theorem}
    \autoref{def:H^0-induced_functor} is a well-defined functor.
\end{theorem}
\begin{proof}
    TODO
\end{proof}

% Want to show that H^0(h) has codomain D^c(\Cc)
\begin{remark}
    Let \( \mathbf{h}: \Cc \to \dgMod_{\dg}(\Cc) \) be the DG-Yoneda embedding from \autoref{def:DG_Yoneda_embedding}.

    Then for any \( A \in \Cc \), one has that \( H^0(\mathbf{h})(A) \) is both DG-projective and compact.
    
    TODO: SHOW!!!

    Therefore one has that the functor \( H^0(\mathbf{h}): H^0(\Cc) \to H^0(\dgMod_{\dg}(\Cc)) \) factors through \( \D^c(\Cc) \). Denote this functor with the same notation:
    \[
        H^0(\mathbf{h}): H^0(\Cc) \to \D^c(\Cc)
    \]
\end{remark}

\begin{remark}
    \( H^0(\mathbf{h}): H^0(\Cc) \to \D^c(\Cc) \) is fully faithful.

    TODO: Prove
\end{remark}

% TODO: Why need small? Probably something with derived.
% TODO: Heilt ordrett nesten frå Jasso--Muro 2023 p. 32, burde kanskje omformulera?
\begin{definition}[pre-triangulated DG-category]
    Let \( \Cc \) be a small DG-category.

    Then \( \Cc \) is called a \emph{pre-triangulated DG-category} if the image of the (fully faithful) functor \( H^0(\mathbf{h}): H^0(\Cc) \to \D^c(\Cc) \) is a triangulated subcategory of \( \D^c(Cc) \).
\end{definition}

\begin{definition}[Algebraic triangulated category]
    Let \( \Tc \) be a triangulated category.

    Then \( \Tc \) is called an \emph{algebraic triangulated category} if there exist a pre-triangulated DG-category, \( \Cc \), such that \( H^0(\Cc) \) is equivalent to \( \Tc \).
\end{definition}