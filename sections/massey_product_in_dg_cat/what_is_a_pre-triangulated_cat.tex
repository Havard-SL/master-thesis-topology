There are multiple definitions of an algebraic triangulated categoery. In this thesis, the definition used will be from TODO:Cite Jasso--Muro.

% TODO Cite: Jasso--Muro
\begin{definition}[\( \C_{\dg}(\Mod(R)) \)]
    Let \( R \) be a commutative ring with identity.

    Then let \emph{\( \C_{\dg}(\Mod(R)) \)} be a DG-category defined as follows
    \begin{enumerate}
        \item {
            \( \Obj(\C_{\dg}(\Mod(R))) := \Obj(\C(\Mod(R))) \).
        }
        \item {
            For \( A, B \in \Obj(\C_{\dg}(\Mod(R))) \) let
            \[ 
                \tuple{\C_{\dg}(\Mod(R))(A, B)}_i := \bigoplus_{j \in \Zb} \Mod(R)(A_j, B_{j + i})
            \]
            with
            \[
                \C_{\dg}(\Mod(R))(A, B) = \bigoplus_{i \in \Zb} \tuple{\C_{\dg}(\Mod(R))(A, B)}_i.
            \]

            Let \( d_A \) and \( d_B \) be the differential of \( A \) and \( B \) as objects of \( \C(\Mod(R)) \), respectively.

            % TODO: This is slight abuse of notation (d_B and d_A are defined on A, B, not A_j, B_{j + i}). Should I fix?
            Let the differential of \( \C_{\dg}(\Mod(R))(A, B) \) be defined as follows
            \begin{align*}
                d_i: \tuple{\C_{\dg}(\Mod(R))(A, B)}_i &\to \tuple{\C_{\dg}(\Mod(R))(A, B)}_{i + 1} \\
                f &\mapsto d_B \circ f - (-1)^i f \circ d_A
            \end{align*}
        }
        \item {
            For \( A, B, C \in \Obj(\C_{\dg}(\Mod(R))) \), let
            \[
                \circ_{\C_{\dg}(\Mod(R))}: \C_{\dg}(\Mod(R))(B, C) \otimes \C_{\dg}(\Mod(R))(A, B) \to \C_{\dg}(\Mod(R))(A, C)
            \]
            % TODO: Explain more?
            be defined as expected.
        }
    \end{enumerate}
\end{definition}

% TODO Cite: Jasso--Muro
% TODO: Elaborate on enriched functor?
\begin{definition}[DG-functor]
    An enriched functor between two DG-categories is called a \emph{DG-functor}.
\end{definition}

% MS-question: Remark below.
\begin{remark}
    % Bondal--Kapranov has as definition
    Enriched functor between DG-categories implies it preserves differentials and grading? TODO
\end{remark}

% TODO: DG-categories over the same ring?
% TODO: Need to show that it's a category?
% MS-Question: Correct? Berest--Mehrle (LN) has another def.
\begin{notation}
    For two DG-categories \( \Ac, \Bc \), let \( \Fun_{\dg}(\Ac, \Bc) \) denote the category of all DG-functors from \( \Ac \) to \( \Bc \).
\end{notation}

\begin{definition}[Opposite DG-category]
    Let \( \Cc \) be a DG-category.

    Then let \( \Cc^{op} \) be the DG-category defined as follows
    \begin{enumerate}
        \item {
            \( \Obj(\Cc^{op}) := \Obj(\Cc) \)
        }
        \item {
            For \( A, B \in \Cc^{op} \), let \( \Cc^{op}(A, B) := \Cc(B, A) \).
        }
        \item {
            For \( A, B, C \in \Cc^{op} \), with \( f \in \Cc^{op}(B, C) \) and \( g \in \Cc^{op}(A, B) \) homogeneous elements of degree \( d_f \) and \( d_g \) respectively.

            Let composition be defined as
            \begin{align*}
                \circ_{\Cc^{op}}: \Cc^{op}(B, C) \otimes \Cc^{op}(A, B) &\to \Cc^{op}(A, C) \\
                f \otimes g &\mapsto (-1)^{d_f d_g} \circ_{\Cc} (g \otimes f)
            \end{align*}
            % TODO: Why can I say this? Need some statement saying all elementary tensors are a sum of homogeneous elementary tensors? As well as saying that the set of elementary tensors generate all elements in the tensor product?
            and extended to all other elementary tensors.
        }
    \end{enumerate}
\end{definition}

% TODO: Cite: Jasso--Muro
% No mention of the ring in the notation? -> Implied since DG-category is over a ring!
\begin{definition}[\( \dgMod_{\dg}(\Cc) \)]
    Let \( \Cc \) be a DG-category over \( R \).

    % TODO: Why "Right"?
    Then define the \emph{DG-category of (right) DG \( \Cc \)-modules} as
    \[
        \dgMod_{\dg}(\Cc) := \Fun_{\dg}(\Cc^{op}, \C_{\dg}(\Mod(R))).
    \]
\end{definition}

% TODO: Should be true according to Jasso--Muro
% TODO: DG-category over R?
\begin{proposition}
    \( \dgMod_{\dg}(\Cc) \) is a DG-category.
\end{proposition}
\begin{proof}
    TODO
\end{proof}

\begin{definition}[DG Yoneda embedding]
    Let \( \Cc \) be a DG-category over \( R \).
    
    Then let \( \mathbf{h} \) be the functor defined as follows
    \begin{align*}
        \mathbf{h}: \Cc &\to \dgMod_{\dg}(\Cc) \\
        A &\mapsto \Cc(-, A)
    \end{align*}

    This functor is called the \emph{DG Yoneda embedding of \( \Cc \)}.
\end{definition}

% TODO: Add Yoneda embedding identifies \Cc with a full subcategory of \dgMod_dg(\Cc)?

\begin{definition}[0th cohomology category of a DG category]
    Let \( \Cc \) be a DG category over \( R \).

    Then let \( H^0(\Cc) \) be the following (enriched over \( \Mod(R) \) TODO) category defined as follows
    \begin{enumerate}
        \item {
            Let \( \Obj(H^0(\Cc)) := \Obj(\Cc) \).
        }
        \item {
            Let \( A, B \in H^0(\Cc) \).

            Then let \( H^0(\Cc)(A, B) := H^0(\Cc(A, B)) \).
        }
        \item {
            Let \( A, B, C \in H^0(\Cc) \) with \( f \in H^0(B, C) \) and \( g \in H^0(A, B) \).

            Then by \autoref{lem:massey_product_in_dg_cat/massey_product_definition/exist_lifting_h_star}, there exists \( \tilde{f} \in \Cc(B, C), \tilde{g} \in \Cc(A, B) \) such that \( H^0(\tilde{f}) = f \) and \( H^0(\tilde{g}) = g \).

            % TODO: Need to show that this is well defined?
            Then let composition be defined on elementary tensors as follows
            \begin{align*}
                \circ_{H^0(\Cc)}: H^0(\Cc)(B, C) \otimes H^0(\Cc)(A, B) &\to H^0(\Cc)(A, C) \\
                f \otimes g &\mapsto H^0(\circ_{\Cc}(\tilde{f} \otimes \tilde{g}))
            \end{align*}
        }
    \end{enumerate}
\end{definition}

\begin{theorem}
    \( H^0(\dgMod_{\dg}(\Cc)) \) is a triangulated category.
\end{theorem}
\begin{proof}
    TODO
\end{proof}

% MS-Question: Have seen definition of small category that is that the class of iso classes are small, not the class of objects. What is correct? Are they equivalent?

% TODO: Various definitions and idiosyncracies. Which is correct?
    % Acyclic kernel? Projective objects? Spanned?
    % Krause 07 -> Compact objects, maybe more.
    % Keller 94 -> Another definition of derived DG category.
% TODO: Following def from Krause 07, but not explicitly written down. Is it correct?
\begin{definition}[Derived DG category]
    Let \( \Cc \) be a DG category.

    TODO
    % Localize WRT  quasi-iso.

\end{definition}

% MS-Quastion: Why are small categories sometimes mentioned in def of derived category? Something to do with localization being well defined?