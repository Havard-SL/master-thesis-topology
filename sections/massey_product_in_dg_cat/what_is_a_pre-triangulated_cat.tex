There are multiple definitions of an algebraic triangulated categoery. In this thesis, the definition used will be from TODO:Cite Jasso--Muro.

% TODO Cite: Jasso--Muro
\begin{definition}[\( \C_{\dg}(\Mod(R)) \)]
    Let \( R \) be a commutative ring with identity.

    Then let \emph{\( \C_{\dg}(\Mod(R)) \)} be a DG-category defined as follows
    \begin{enumerate}
        \item {
            \( \Obj(\C_{\dg}(\Mod(R))) := \Obj(\C(\Mod(R))) \).
        }
        \item {
            For \( A, B \in \Obj(\C_{\dg}(\Mod(R))) \) let
            \[ 
                \tuple{\C_{\dg}(\Mod(R))(A, B)}_i := \bigoplus_{j \in \Zb} \Mod(R)(A_j, B_{j + i})
            \]
            with
            \[
                \C_{\dg}(\Mod(R))(A, B) = \bigoplus_{i \in \Zb} \tuple{\C_{\dg}(\Mod(R))(A, B)}_i.
            \]

            Let \( d_A \) and \( d_B \) be the differential of \( A \) and \( B \) as objects of \( \C(\Mod(R)) \), respectively.

            % TODO: This is slight abuse of notation (d_B and d_A are defined on A, B, not A_j, B_{j + i}). Should I fix?
            Let the differential of \( \C_{\dg}(\Mod(R))(A, B) \) be defined as follows
            \begin{align*}
                d_i: \tuple{\C_{\dg}(\Mod(R))(A, B)}_i &\to \tuple{\C_{\dg}(\Mod(R))(A, B)}_{i + 1} \\
                f &\mapsto d_B \circ f - (-1)^i f \circ d_A
            \end{align*}
        }
        \item {
            For \( A, B, C \in \Obj(\C_{\dg}(\Mod(R))) \), let
            \[
                \circ_{\C_{\dg}(\Mod(R))}: \C_{\dg}(\Mod(R))(B, C) \otimes \C_{\dg}(\Mod(R))(A, B) \to \C_{\dg}(\Mod(R))(A, C)
            \]
            % TODO: Explain more?
            be defined as expected.
        }
    \end{enumerate}
\end{definition}

% TODO Cite: Jasso--Muro
% TODO: Elaborate on enriched functor?
\begin{definition}[DG-functor]
    An enriched functor between two DG-categories is called a \emph{DG-functor}.
\end{definition}

% MS-question: Remark below.
\begin{remark}
    % Bondal--Kapranov has as definition
    Enriched functor between DG-categories implies it preserves differentials and grading? TODO
\end{remark}

\begin{definition}[Opposite DG-category]
    Let \( \Cc \) be a DG-category.

    Then let \( \Cc^{op} \) be the DG-category defined as follows
    \begin{enumerate}
        \item {
            \( \Obj(\Cc^{op}) := \Obj(\Cc) \)
        }
        \item {
            For \( A, B \in \Cc^{op} \), let \( \Cc^{op}(A, B) := \Cc(B, A) \).
        }
        \item {
            For \( A, B, C \in \Cc^{op} \), with \( f \in \Cc^{op}(B, C) \) and \( g \in \Cc^{op}(A, B) \) homogeneous elements of degree \( d_f \) and \( d_g \) respectively.

            Let composition be defined as
            \begin{align*}
                \circ_{\Cc^{op}}: \Cc^{op}(B, C) \otimes \Cc^{op}(A, B) &\to \Cc^{op}(A, C) \\
                f \otimes g &\mapsto (-1)^{d_f d_g} \circ_{\Cc} (g \otimes f)
            \end{align*}
            % TODO: Why can I say this? Need some statement saying all elementary tensors are a sum of homogeneous elementary tensors?
            and extended to all other elementary tensors.
        }
    \end{enumerate}
\end{definition}

\begin{notation}
    For two DG-categories \( \Ac, \Bc \), let \( \Fun_{\dg}(\Ac, \Bc) \) denote the category of all DG-functors from \( \Ac \) to \( \Bc \).
\end{notation}

% TODO: Cite: Jasso--Muro
\begin{definition}[\( \dgMod_{\dg}(\Cc) \)]
    Let \( \Cc \) be a DG-category, and let \( R \) be a commutative ring (with unit).

    % TODO: Why "Right"?
    Then define the \emph{DG-category of (right) DG \( \Cc \)-modules} as
    \[
        \dgMod_{\dg}(\Cc) := \Fun_{\dg}(\Cc^{op}, \C_{\dg}(\Mod(R))).
    \]
\end{definition}

% TODO: Should be true according to Jasso--Muro
\begin{proposition}
    \( \dgMod_{\dg}(\Cc) \) is a DG-category.
\end{proposition}
\begin{proof}
    TODO
\end{proof}