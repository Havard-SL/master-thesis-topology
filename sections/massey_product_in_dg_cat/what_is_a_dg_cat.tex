\begin{notation}
    Let \( \Cc \) be any additive category.
    
    Then let \( C \tuple{\Cc} \) denote the category of chain complexes of objects in \( \Cc \).

    Furthermore let the differential in these chain complexes have \emph{ascending} order. I.e. for \( M_i, M_{i+1} \in \Cc \) which are adjacent objects in a chain complex \( M \in C \tuple{\Cc} \), the differential would be
    \[
        d_i : M_i \to M_{i + 1}.
    \]
\end{notation}

\begin{notation}
    Let \( R \) be a commutative ring with identity. Let \( A \in C \tuple{\Mod(R)} \).

    Then the degree of a \emph{homogenous} element \( a \in A \) is denoted \emph{\( \abs{a} \)}.
\end{notation}

\begin{definition}
    Let \( R \) be a commutative ring with identity. Furthermore let \( A, B \in C \tuple{\Mod(R)} \).

    Then define the modules
    \[
        (A \otimes B)_n := \bigoplus_{p + q = n} A_p \otimes B_q
    \]
    which are a part of the chain complex
    \[
        A \otimes B := \bigoplus_n \tuple{A \otimes B}_n
    \]
    with the differential (for \( a \) a homogenous element of \( A \))
    \[
        d(a \otimes b) := d(a) \otimes b + (-1)^{|a|} a \otimes d(b).
    \]

    This is called the \emph{tensor product of chain complexes over \( \Mod(R) \)}.
\end{definition}

I have found three different definitions of a DG-category:

\begin{definition}[DG-category, Bondal--Kapranov 1991]
    \label{def:massey_product_in_dg_cat/what_is_a_dg_cat/dg_cat_bondal--kapranov_1991}
    A \emph{DG-category} \( \Cc \) is a category satisfying the following criteria:
    \begin{enumerate}
        \item \( \Cc \) is pre-additive.
        \item The hom-sets of \( \Cc \) are objects in \( C \tuple{Ab} \).
        \item{
            Composition is done with the tensor product, and should be a \( C \tuple{\Ab} \)-morphism.

            I.e:
            \[
                \circ_{A,B,C}: \Cc \tuple{B, C} \otimes \Cc \tuple{A, B} \to \tuple{A, C}
            \]
            should be a \( C \tuple{\Ab} \)-morphism.
        }
        \item For all \( A \in \Cc \), \( d(\Id_A) = 0 \).
    \end{enumerate}
\end{definition}

\begin{definition}[DG-category, Keller 1994]
    \label{def:massey_product_in_dg_cat/what_is_a_dg_cat/dg_cat_keller_1994}
    Let \( R \) be a commutative ring.

    A \emph{DG-category} \( \Cc \) is a category satisfying the following criteria:
    \begin{enumerate}
        \item The hom-sets of \( \Cc \) are objects in \( C \tuple{\Mod \tuple{R} } \).
        \item{
            Composition is done with the tensor product, and should be a \( C \tuple{\Mod \tuple{R} } \)-morphism.

            I.e:
            \[
                \circ_{A, B, C}: \Cc \tuple{B, C} \otimes \Cc \tuple{A, B} \to \tuple{A, C}
            \]
            should be a \( C \tuple{\Mod \tuple{R} } \)-morphism.
        }
        \item {
            For \( f \in \Cc \tuple{A, B} \) any element, and \( g \in \Cc \tuple{B, C} \) a homogenous element, then

            \[
                d(g \circ f) = d(g) \circ f + (-1)^{|g|} g \circ d(f).
            \]
        }
    \end{enumerate}
\end{definition}

\begin{definition}[DG-category, Berest--Mehrle 2017 (Lecture notes)]
    \label{def:massey_product_in_dg_cat/what_is_a_dg_cat/dg_cat_berest--mehrle_2017}
    Let \( R \) be a commutative ring.

    A \emph{DG-category} \( \Cc \) is a small category enriched in the category \( C \tuple{\Mod(R)} \).

    This is excplicitly that \( \Cc \) consists of the following data:
    \begin{enumerate}
        \item A set of objects \( Ob(\Cc) \).
        \item Every pair of objects \( A, B \in \Cc \) correspond to an object in \( C \tuple{\Mod(R)} \).
        \item{
            For every triple \( A, B, C \in \Cc \) there is a \( C \tuple{\Mod(R)} \) homomorphism \( \circ_{A,B,C} \):
            \[
                \circ_{A, B, C}: \Cc(B, C) \otimes \Cc(A, B) \to \Cc(A, C)
            \]
        }
        \item For all \( A \in \Cc \) there is an identity morphism \( \Id_A \in \Cc(A, A) \) with the expected properties.
    \end{enumerate}
\end{definition}

\begin{remark}
    \label{rem:massey_product_in_dg_cat/what_is_a_dg_cat/d_of_id_is_zero}
    If one has that \( \Id_A \) is the identity element in the chain complex \( \Cc\tuple{A, A} \). Then \( \abs{\Id_A} = 0 \) and it follows that
    \[
        d(\Id_A) = d(\Id_A \circ \Id_A) = d(\Id_A) \circ \Id_A + (-1)^0 \Id_A \circ d(\Id_A) = 2d(\Id_A)
    \]
    which implies that \( d(\Id_A) = 0 \).
\end{remark}

% TODO: Remove or fix this remark.
\begin{remark}
    Every definition of a DG-category given above seem to be very similar in many aspects.

    In particular, \autoref{def:massey_product_in_dg_cat/what_is_a_dg_cat/dg_cat_bondal--kapranov_1991} is a special case of \autoref{def:massey_product_in_dg_cat/what_is_a_dg_cat/dg_cat_keller_1994}, where \( R = \Zb \).

    Also, \emph{I think} many of the points of the definitions are redundant. The supsecting redudnat points are as follows:
    \begin{itemize}
        \item {
            In \autoref{def:massey_product_in_dg_cat/what_is_a_dg_cat/dg_cat_bondal--kapranov_1991} both point 1 and point 4 are redundant. Since objects in \( C \tuple{\Ab} \) are abelian groups, the category is already pre-additive.
            
            And by sign conventions (TODO: Ref/explain/find out/understand/help!), point 2 and 3 imply a leibniz-like rule as in \autoref{def:massey_product_in_dg_cat/what_is_a_dg_cat/dg_cat_keller_1994} point 3, which by \autoref{rem:massey_product_in_dg_cat/what_is_a_dg_cat/d_of_id_is_zero} implies point 4.
        }
        \item {
            In \autoref{def:massey_product_in_dg_cat/what_is_a_dg_cat/dg_cat_keller_1994} point 3 seems to follow from sign conventions (TODO: Ref/explain/find out/understand/help!).
        }
    \end{itemize}

    In view of the above redundant points, \emph{I believe} that \autoref{def:massey_product_in_dg_cat/what_is_a_dg_cat/dg_cat_berest--mehrle_2017} implies the other definitions.
\end{remark}

% MS-Question: Do I lack data in this definition?
% https://ncatlab.org/nlab/show/category+of+chain+complexes
\begin{fact}[nlab]
    Let \( R \) be a commutative ring with identity, and let \( \otimes \) denote the tensor product on \( C \tuple{\Mod(R)} \). Furthermore let \( I \) be the chain complex in \( C \tuple{\Mod(R)} \) consisting solely of \( 0 \)-objects in non-zero degrees, and the \( R \)-module \( R \) in degree 0. 

    Then \( \tuple{C \tuple{\Mod(R)}, \otimes, I} \) is a symmetric monoidal category.
\end{fact}

This thesis will be using this definition, which is similar to the one given by Berest--Mehrle, but not restricted to small categories.
\begin{definition}[DG-category, will be used in this thesis]
    Let \( R \) be a commutative ring with identity.

    Then \( \Cc \) is a \emph{DG-category over \( R \)} if it is a category enriched over \( C \tuple{\Mod(R)} \).
\end{definition}

% https://mathoverflow.net/questions/17951/what-tensor-product-of-chain-complexes-satisfies-the-usual-universal-property
\begin{remark}
    TODO: Connection between cartesian product and tensor product.

    Let
    \[
        \phi: A \oplus B \to C.
    \]
    For any element \( a \in A \), one can look at the morphism resulting from fixing one of the factors of \( \phi \). Denote this by
    \[
        \phi_a: B \to C
    \]
    where \( \phi_a(b) = \phi(a, b) \).
    
    This gives a morphism
    \[
        \psi: A \to \Hom(B, C).
    \]

    However, by tensor left-adjunction, this \( \phi \) corresponds to a morphism
    \[
        \tilde{\psi}: A \otimes B \to C.
    \]
\end{remark}

\begin{definition}
    Let \( A \) be an associative algebra with identity over a commutative ring with identity \( R \). Furthermore, let \( A \) have a graded ring structure. Let \( R \cdot 1_A \subseteq A_0 \).

    Then \( A \) is called a \emph{graded algebra over \( R \)}.
\end{definition}

\begin{definition}
    Let \( A \) be a graded algebra over \( R \).

    If there is a 'graded algebra over \( R \)' -epimorphism of degree \( 1 \), denoted \( d_A \), (i.e. for any \( i \in \Zb \), \( d_A |_{A_i}: A_i \to A_{i + 1} \))
    with the following properties:
    \begin{enumerate}
        \item One has that \( d \circ d = 0 \).
        \item {
            For \( a, b \in A \) with \( a \) homogenous, one has that
            \[
                d\tuple{a \cdot b}
                =
                d\tuple{a} \cdot b + \tuple{-1}^{\abs{a}}a \cdot d\tuple{b}.
            \]
            }
    \end{enumerate}

    Then \( A \) is called a \emph{differentially graded algebra over \( R \)}.
\end{definition}

\begin{theorem}
    Let \( \Cc \) be a DG-category over \( R \).

    Then for any \( A \in \Cc \) one has that \( \Cc \tuple{A, A} \) is a differentially graded algebra over \( R \).
\end{theorem}

\begin{proof}
    TODO
\end{proof}
