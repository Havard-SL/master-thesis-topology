In this thesis the definition of a DG-Category is based on enriched category theory, as it is a modern approach, and by the opinion of the author it is also the most elegeant approach.

\begin{notation}
    Let \( \Cc \) be any additive category.
    
    Then let \( \C \tuple{\Cc} \) denote the category of chain complexes of objects in \( \Cc \).

    Furthermore let the differential in these chain complexes have \emph{ascending} order. I.e. for \( M_i, M_{i+1} \in \Cc \) which are adjacent objects in a chain complex \( M \in \C \tuple{\Cc} \), the differential would be
    \[
        d_i : M_i \to M_{i + 1}.
    \]
\end{notation}

\begin{definition}[Tensor product of chain complexes over \( \Mod(R) \)]
    \label{def:tensor_product_of_chain_complexes_over_Mod(R)}
    Let \( R \) be a commutative ring with identity. Furthermore let \( A, B \in \C \tuple{\Mod(R)} \).

    Then define the modules
    \[
        (A \otimes B)_n := \coprod_{p + q = n} A_p \otimes B_q
    \]
    which are a part of the chain complex
    \begin{center}
        \begin{tikzpicture}
            \diagram{m}{1cm}{1cm} {
                A \otimes B: \\
            };
        \end{tikzpicture}
        %
        \begin{tikzpicture}
            \diagram{m}{1cm}{1cm} {
                \cdots \& (A \otimes B)_{-1} \& (A \otimes B)_0 \& (A \otimes B)_1 \& \cdots \\
            };

            \draw[math]
                (m-1-1) edge (m-1-2)
                (m-1-2) edge node {d_{-1}} (m-1-3)
                (m-1-3) edge node {d_0} (m-1-4)
                (m-1-4) edge (m-1-5);
        \end{tikzpicture}
    \end{center}

    Where the differentials, \( d_n \), are defined as follows:
    
    Let \( i, j \in \Zb \) with \( i + j = n \), and let
    \[
        \iota_{i, j}: A_i \otimes B_j \hookrightarrow \tuple{A \otimes B}_n
    \]
    be the canonical split-monomorphism. Furthermore let \( a \otimes b \in A_i \otimes B_j \) be an elementary tensor.

    Then the differential is uniquely defined as follows
    \[
        d_n(\iota_{i, j}(a \otimes b)) := \iota_{i + 1, j}\tuple{d_{A, i}(a) \otimes b} + (-1)^{i} \iota_{i, j + 1}\tuple{a \otimes d_{B, j}(b)}.
    \]

    This is called the \emph{tensor product of chain complexes over \( \Mod(R) \)}.
\end{definition}

\begin{remark}
    The definition of the differentials in \autoref{def:tensor_product_of_chain_complexes_over_Mod(R)} is well defined and unique by the following argument.

    Look at the following diagram
    \begin{equation}
        \label{tikz:differential_of_tensor_product_of_chain_complexes_over_Mod(R)}
        \begin{tikzpicture}
            \diagram{m}{2cm}{2cm} {
                A_i \otimes B_j \& \coprod\limits_{p + q = n} A_p \otimes B_q \\
                A_i \times B_j \& \coprod\limits_{p' + q' = n + 1} A_{p'} \otimes B_{q'} \\
            };

            \draw[math]
                (m-1-1) edge[hook] node {\iota_{i, j}} (m-1-2)
                    edge[dashed] node {\alpha_{i, j}} (m-2-2)
                (m-1-2) edge[dashed] node {\beta} (m-2-2)

                (m-2-1) edge node {f} (m-2-2)
                    edge node {\otimes} (m-1-1);
        \end{tikzpicture}
    \end{equation}
    Let \( f \) be the map defined element-wise as follows
    \[
        a \times b \mapsto \iota_{i + 1, j}\tuple{d_{A, i}(a) \otimes b} + (-1)^{i} \iota_{i, j + 1}\tuple{a \otimes d_{B, j}(b)}
    \]
    One can check that this map is \( R \)-balanced, and therefore a morphism. In addition, by the universal property of tensor product in \( \Mod(R) \), \( f \) induces a unique morphism, \( \alpha_{i, j} \), which is induced from the elementary tensors as follows
    \[
        a \otimes b \mapsto \iota_{i + 1, j}\tuple{d_{A, i}(a) \otimes b} + (-1)^{i} \iota_{i, j + 1}\tuple{a \otimes d_{B, j}(b)}.
    \]
    Since this works for any \( i, j \) as long as \( i + j = n \), one can construct \( \alpha_{i, j} \) for every valid \( i, j \) pair.

    Then by using the universal property of the coproduct one gets the unique map \( \beta \) which is by \autoref{tikz:differential_of_tensor_product_of_chain_complexes_over_Mod(R)} uniquely determined by it's actions on elementary tensors in \( A_i \otimes B_j \) in the following way
    \[
        \iota_{i, j}(a \otimes b) \mapsto \iota_{i + 1, j}\tuple{d_{A, i}(a) \otimes b} + (-1)^{i} \iota_{i, j + 1}\tuple{a \otimes d_{B, j}(b)}.
    \]
    Which is exactly equal to \( d_n \).

    By a similar argument as above, it follows that since \( d_{n + 1} \circ d_n \) sends every \( \iota_{i, j}(a \otimes b) \) to \( 0 \), then it has to be the zero map, and \( d_n \) is therefore a differential.
\end{remark}

\begin{definition}[Internal hom of chain complexes over \( \Mod(R) \)]
    \label{def:internal_hom_of_chain_complexes_over_Mod(R)}
    Let \( R \) be a commutative ring with identity. Furthermore let \( A, B \in \C \tuple{\Mod(R)} \).

    Then define the modules
    \[
        \class{A, B}_n := \prod_{j \in \Zb} \Mod(R)(A_j, B_{j + n})
    \]
    which are a part of the chain complex
    \begin{center}
        \begin{tikzpicture}
            \diagram{m}{1cm}{1cm} {
                \class{A, B}: \\
            };
        \end{tikzpicture}
        %
        \begin{tikzpicture}
            \diagram{m}{1cm}{1cm} {
                \cdots \& \class{A, B}_{-1} \& \class{A, B}_0 \& \class{A, B}_1 \& \cdots \\
            };

            \draw[math]
                (m-1-1) edge (m-1-2)
                (m-1-2) edge node {d_{-1}} (m-1-3)
                (m-1-3) edge node {d_0} (m-1-4)
                (m-1-4) edge (m-1-5);
        \end{tikzpicture}
    \end{center}

    Where the differentials, \( d_n \), are defined as follows:
    WIP
    \begin{align*}
        d_i: \prod_{j \in \Zb} \Mod(R)(A_j, B_{j + i}) &\to \prod_{j \in \Zb} \Mod(R)(A_j, B_{j + i + 1}) \\
        f &\mapsto d_B \circ f - (-1)^i f \circ d_A
    \end{align*}
    This is called the \emph{internal hom of chain complexes over \( \Mod(R) \)}.
\end{definition}

% MS-Question: Do I lack data in this definition?
% https://ncatlab.org/nlab/show/category+of+chain+complexes
\begin{fact}[nlab]
    Let \( R \) be a commutative ring with identity, and let \( \otimes \) denote the tensor product on \( \C \tuple{\Mod(R)} \). Furthermore let \( I \) be the chain complex in \( \C \tuple{\Mod(R)} \) consisting solely of \( 0 \)-objects in non-zero degrees, and the \( R \)-module \( R \) in degree 0. 

    Then \( \tuple{\C \tuple{\Mod(R)}, \otimes, I} \) is a symmetric closed monoidal category.
\end{fact}

% This thesis will be using the following definition, which is similar to the one given by Berest--Mehrle, but not restricted to small categories.
% TODO: Cite
\begin{definition}% [DG-category]
    Let \( R \) be a commutative ring with identity.

    Then \( \Cc \) is a \emph{DG-category over \( R \)} if it is a category enriched over \( \C \tuple{\Mod(R)} \).
\end{definition}
This definition also appear in Jasso--Muro p. 29. (TODO: Ref), except they define it for a field and not a commutative ring with identity.

