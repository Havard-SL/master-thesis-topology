In this thesis the definition of a DG-Category is based on enriched category theory, as it is a modern approach, and by the opinion of the author it is also the most elegeant approach.

\begin{definition}[Tensor product of chain complexes over \( \Mod(R) \)]
    \label{def:tensor_product_of_chain_complexes_over_Mod(R)}
    Let \( R \) be a commutative ring with identity. Furthermore let \( A, B \in \C \tuple{\Mod(R)} \).

    Then, for any  \( n \in \Zb \) define the modules
    \[
        (A \otimes B)_n := \coprod_{p + q = n} A_p \otimes B_q
    \]
    which are a part of the chain complex
    \begin{center}
        \begin{tikzpicture}
            \diagram{m}{1cm}{1cm} {
                A \otimes B: \\
            };
        \end{tikzpicture}
        %
        \begin{tikzpicture}
            \diagram{m}{1cm}{1cm} {
                \cdots \& (A \otimes B)_{-1} \& (A \otimes B)_0 \& (A \otimes B)_1 \& \cdots \\
            };

            \draw[math]
                (m-1-1) edge (m-1-2)
                (m-1-2) edge node {d_{-1}} (m-1-3)
                (m-1-3) edge node {d_0} (m-1-4)
                (m-1-4) edge (m-1-5);
        \end{tikzpicture}
    \end{center}

    Where the differentials, \( d_n \), are defined as follows:
    
    Let \( i, j \in \Zb \) with \( i + j = n \), and let
    \[
        \iota_{i, j}: A_i \otimes B_j \hookrightarrow \tuple{A \otimes B}_n
    \]
    be the canonical split-monomorphism. Furthermore let \( a \otimes b \in A_i \otimes B_j \) be an elementary tensor.

    Then the differential is (uniquely) defined by the following actions
    \[
        d_n(\iota_{i, j}(a \otimes b)) := \iota_{i + 1, j}\tuple{d_{A, i}(a) \otimes b} + (-1)^{i} \iota_{i, j + 1}\tuple{a \otimes d_{B, j}(b)}.
    \]

    This is called the \emph{tensor product of chain complexes over \( \Mod(R) \)}.
\end{definition}

% TODO: Generalize to a lemma and show the two special cases?
\begin{lemma}
    \label{lem:map_out_of_tensor_unique}
    Let \( A, B \in \C(\Mod(R)) \) and let \( C \in \Mod(R) \). Furthermore let \( i, j \in \Zb \) with \( i + j = n \). And let
    \[
        \iota_{i, j}: A_i \otimes B_j \to (A \otimes B)_n := \coprod_{p + q = n} A_p \otimes B_q
    \]
    be the canocial split monomorphisms.

    Then for any
    \[
        f: (A \otimes B)_n \to C
    \]
    where for any \( a \in A_i, b \in B_j \) one has
    \[
        f(\iota_{i, j}(a \otimes b)) = g_{i, j}(a, b)
    \]
    for some \( R \)-bilinear morphisms
    \[
        g_{i, j}: A_i \times B_j \to C.
    \]

    Then \( f \) is uniquely defined by the \( g_{i, j} \)'s.
\end{lemma}
\begin{proof}
    Look at the following diagram
    \begin{equation}
        \label{tikz:differential_of_tensor_product_of_chain_complexes_over_Mod(R)}
        \begin{tikzpicture}
            \diagram{m}{2cm}{2cm} {
                A_i \otimes B_j \& \coprod\limits_{p + q = n} A_p \otimes B_q \\
                A_i \times B_j \& C \\
            };

            \draw[math]
                (m-1-1) edge[hook] node {\iota_{i, j}} (m-1-2)
                    edge[dashed] node {\alpha_{i, j}} (m-2-2)
                (m-1-2) edge[dashed] node {\beta} (m-2-2)

                (m-2-1) edge node {g_{i, j}} (m-2-2)
                    edge node {\otimes} (m-1-1);
        \end{tikzpicture}
    \end{equation}

    Where the \( g_{i, j} \)'s are \( R \)-bilienar.

    Then by the universal property of tensor product in \( \Mod(R) \), \( g_{i, j} \) induces a unique morphism, \( \alpha_{i, j} \), which is induced from the elementary tensors as follows
    \[
        a \otimes b \mapsto g_{i, j}(a, b).
    \]

    Since this works for any \( i, j \) as long as \( i + j = n \), one can construct \( \alpha_{i, j} \) for every valid \( i, j \) pair.

    Then by using the universal property of the coproduct one gets the unique map \( \beta \) which is by \autoref{tikz:differential_of_tensor_product_of_chain_complexes_over_Mod(R)} uniquely determined by it's actions on elementary tensors in \( A_i \otimes B_j \) in the following way
    \[
        \beta: \iota_{i, j}(a \otimes b) \mapsto g_{i, j}.
    \]
    Which is exactly equal to \( f \), and \( f \) is therefore uniquely deterined by the \( g_{i, j} \)'s.
\end{proof}
\begin{remark}
    The definition of the differentials in \autoref{def:tensor_product_of_chain_complexes_over_Mod(R)} is well defined and unique by the following argument.

    One can check that for \( i + j = n \) that
    \begin{align*}
        g_{i, j}: A_i \times B_j &\to (A \otimes B)_{n + 1} \\
        (a, b) &\mapsto \iota_{i + 1, j}\tuple{d_{A, i}(a) \otimes b} + (-1)^{i} \iota_{i, j + 1}\tuple{a \otimes d_{B, j}(b)}
    \end{align*}
    is \( R \)-bilinear.

    Then by \autoref{lem:map_out_of_tensor_unique} it follows that \( d_n \) is uniquely defined.

    Similarly, by seeing that
    \[
        d_{n + 1} \circ d_n: \iota_{i, j}(a \otimes b) \mapsto 0
    \]
    and the \( 0 \) map is \( R \)-bilinear, so \( d_{n + 1} \circ d_n \) is uniquely defined. Since the zero map would also send \( \iota_{i, j}(a \otimes b) \) to \( 0 \), then by uniquenes \( d_{n + 1} \circ d_n = 0 \).
\end{remark}

\begin{definition}[Internal hom of chain complexes over \( \Mod(R) \)]
    \label{def:internal_hom_of_chain_complexes_over_Mod(R)}
    Let \( R \) be a commutative ring with identity. Furthermore let \( A, B \in \C \tuple{\Mod(R)} \).

    Then, for any \( n \in \Ab \) define the modules
    \[
        \class{A, B}_n := \prod_{i \in \Zb} \Mod(R)(A_i, B_{i + n})
    \]
    which are a part of the chain complex
    \begin{center}
        \begin{tikzpicture}
            \diagram{m}{1cm}{1cm} {
                \class{A, B}: \\
            };
        \end{tikzpicture}
        %
        \begin{tikzpicture}
            \diagram{m}{1cm}{1cm} {
                \cdots \& \class{A, B}_{-1} \& \class{A, B}_0 \& \class{A, B}_1 \& \cdots \\
            };

            \draw[math]
                (m-1-1) edge (m-1-2)
                (m-1-2) edge node {d_{-1}} (m-1-3)
                (m-1-3) edge node {d_0} (m-1-4)
                (m-1-4) edge (m-1-5);
        \end{tikzpicture}
    \end{center}

    Where the differentials, \( d_n \), are defined as follows:
    
    For any \( i \in \Zb \), let \( \partial_{n, i} \) be the follwing morphism
    \begin{align*}
        \partial_{n, j} : \prod_{i \in \Zb}\Mod(R)(A_i, B_{i + n}) &\to \Mod(R)(A_j, B_{j + n + 1}) \\
        \bigtimes_{i \in \Zb} f_i &\mapsto d_{B, j + n} \circ f_j - (-1)^n f_{j + 1} \circ d_{A, j}.
    \end{align*}
    Then
    \begin{align*}
        d_n := \prod_{i \in \Zb} \partial_{n, i}: \prod_{i \in \Zb} \Mod(R)(A_i, B_{i + n}) &\to \prod_{i \in \Zb} \Mod(R)(A_i, B_{i + n + 1}) \\
        \bigtimes_{i \in \Zb} f_i &\mapsto \bigtimes_{i \in \Zb} d_{B, i + n} \circ f_i - (-1)^n f_{i + 1} \circ d_{A, i}.
    \end{align*}
    This is called the \emph{internal hom of chain complexes over \( \Mod(R) \)}.
\end{definition}

\begin{remark}    
    Let \( j \in \Zb \), and let \( \pi_j \) be the canonical split epimiormphism
    \[
        \pi_j: \prod_{i \in \Zb} \Mod(R)(A_i, B_{i + n + 1}) \to \Mod(R)(A_j, B_{j + n + 1})
    \]

    Then one can verify that the actions \( d_n \) takes on elements defined in \autoref{def:internal_hom_of_chain_complexes_over_Mod(R)} is correct by seeing that \( \pi_j \circ d_n = \partial_{n, j} \) and so by uniqueness of the universal property of product, this must correspond to \( \prod_{i \in \Zb} \partial_{n, i} \).

    To verify that \( d_{n + 1} \circ d_n = 0 \), use the element-wise definition and check that every element gets sent to \( 0 \).
\end{remark}

% TODO: Fix the indices of the iotas to coincide with the indices in the definition of tensor product of chain complexes.
% TODO: Prove point 2 and 3, or at least find some sources on them.
\begin{remark}[Tensor product and internal hom adjunction in \( \C(\Mod(R)) \)]
    Let \( A, B, C \) be chain complexes in \( \C(\Mod(R)) \) for some commutative ring \( R \).

    Let \( f \in \C(\Mod(R))\tuple{A \otimes B, C} \). Want to find out what the adjoint of \( f \) is.

    Let \( f = \set{f_n}_{n \in \Zb} \) where \( f_n \in \Mod(R)\tuple{ \tuple{A \otimes B}_n, C_n} \) are the individual level-wise morphisms of the chain morphism \( f \).

    Then, unwrapping the definitions, one has that \( f \) looks like the following diagram
    \begin{center}
        \begin{tikzpicture}
            \diagram{m}{1cm}{1cm} {
                \cdots \& \coprod\limits_{i + j = -1} A_i \otimes B_j \& \coprod\limits_{i + j = 0} A_i \otimes B_j \& \coprod\limits_{i + j = 1} A_i \otimes B_j \& \cdots \\
                \cdots \& C_{-1} \& C_0 \& C_1 \& \cdots \\
            };

            \draw[math]
                (m-1-1) edge (m-1-2)
                    edge (m-2-1)
                (m-1-2) edge node {d_{A \otimes B, -1}} (m-1-3)
                    edge node {f_{-1}} (m-2-2)
                (m-1-3) edge node {d_{A \otimes B, 0}} (m-1-4)
                    edge node {f_0} (m-2-3)
                (m-1-4) edge (m-1-5)
                    edge node {f_1} (m-2-4)
                (m-1-5) edge (m-2-5)

                (m-2-1) edge (m-2-2)
                (m-2-2) edge node {d_{C, -1}} (m-2-3)
                (m-2-3) edge node {d_{C, 0}} (m-2-4)
                (m-2-4) edge (m-2-5);
        \end{tikzpicture}
    \end{center}
    Likewise, the adjoint has to look like the following diagram
    \begin{center}
        \begin{tikzpicture}
            \diagram{m}{1cm}{0.75cm} {
                \cdots \& A_{-1} \& A_0 \& A_1 \& \cdots \\
                \cdots \& \prod\limits_{j \in \Zb} \Mod(R)(B_j, C_{j - 1}) \& \prod\limits_{j \in \Zb} \Mod(R)(B_j, C_j) \& \prod\limits_{j \in \Zb} \Mod(R)(B_j, C_{j + 1}) \& \cdots \\
            };

            \draw[math]
                (m-1-1) edge (m-1-2)
                    edge (m-2-1)
                (m-1-2) edge node {d_{A, -1}} (m-1-3)
                    edge node {?_{-1}} (m-2-2)
                (m-1-3) edge node {d_{A, 0}} (m-1-4)
                    edge node {?_0} (m-2-3)
                (m-1-4) edge (m-1-5)
                    edge node {?_1} (m-2-4)
                (m-1-5) edge (m-2-5)

                (m-2-1) edge (m-2-2)
                (m-2-2) edge node {d_{\class{B, C}, -1}} (m-2-3)
                (m-2-3) edge node {d_{\class{B, C}, 0}} (m-2-4)
                (m-2-4) edge (m-2-5);
        \end{tikzpicture}
    \end{center}

    For any \( n \in \Zb \), let \( i', j' \in \Zb \) with \( i' + j' = n \) let
    \[
        \iota_{i', j'}: A_{i'} \otimes B_{j'} \hookrightarrow \coprod_{i + j = n} A_i \otimes B_j
    \]
    be the canonical \( i' \)-th split monomorphism by the definintion of the coproduct \( \coprod_{i + j = n} A_i \otimes B_j \).

    Then take the hom-tensor adjoint in \( \Mod(R) \) of the morphism
    \[
        f_{i + j} \circ \iota_{i, j}: A_i \otimes B_j \to C_{i + j}.
    \]
    This yields a morphism
    \begin{align*}
        \phi_{f, i, j}: A_i &\to \Mod(R)(B_j, C_{j + i}) \\
        a &\mapsto f_{i + j} \circ \iota_{i, j}(a \otimes ?).
    \end{align*}
    Then by the universal property of the product there is some morphism
    \[
        \phi_{f, i} := \prod_{j \in \Zb} \phi_{f, i, j}: A_i \to \prod_{j \in \Zb} \Mod(R)\tuple{B_j, C_{j + i}}.
    \]
    Collecting these morphisms yields a morphism, which I claim to be the adjoint of \( f \), namely 
    \[
        \phi_f := \set{\phi_{f, i}}_{i \in \Zb}.
    \]
    In order to show that this is the proper adjoint definition, one need to show the following properties
    \begin{enumerate}
        \item {
            That \( \phi_f \) is a chain morphism.
        }
        \item {
            That the assignment \( f \mapsto \phi_f \) is an isomorphism of groups.
        }
        \item {
            That there is a natural transformation
            \[
                \C(\Mod(R))(?_1 \otimes ?_2, ?_3) \cong \C(\Mod(R))(?_1, \left[ ?_2, ?_3 \right])
            \]
            where the natural morphisms are \( f \mapsto \phi_f \).
        }
    \end{enumerate}
    % TODO: Possibly expand this proof? or TODO: SRC
    In this thesis I will only prove the first statement.

    1) Want to show that \( \phi_f \) is a chain morphism.

    Need to check that for any \( i \in \Zb \) that the following diagram commutes
    \begin{center}
        \begin{tikzpicture}
            \diagram{m}{1cm}{1cm} {
                A_i \& A_{i + 1} \\
                \class{B, C}_i \& \class{B, C}_{i + 1} \\
            };

            \draw[math]
                (m-1-1) edge node {d_{A, i}} (m-1-2)
                    edge node {\phi_{f, i}} (m-2-1)
                (m-1-2) edge node {\phi_{f, i + 1}} (m-2-2)

                (m-2-1) edge node {d_{\class{B, C}, i}} (m-2-2);
        \end{tikzpicture}
    \end{center}
    Pick an arbitrary \( a \in A_i \), look at the following equation.
    \begin{align*}
        \phi_{f, i + 1} \circ d_{A, i}(a) - d_{\class{B, C}, i} \circ \phi_{f, i}(a)
        &= \bigtimes_{j \in \Zb} \tuple{ f_{i + j + 1} \circ \iota_{i + 1, j} (d_{A, i}(a) \otimes ?) } \\
        &\hspace{1cm}- d_{\class{B, C}, i} \tuple{ \bigtimes_{j \in \Zb} f_{i + j} \circ \iota_{i, j} (a \otimes ?) }. \\
        \intertext{Then by expanding out the definition of \( d_{\class{B, C}, i} \) it follows that}
        &= \bigtimes_{j \in \Zb} ( f_{i + j + 1} \circ \iota_{i + 1, j} (d_{A, i}(a) \otimes ?) \\
        &\hspace{1cm} - d_{C, i + j} \circ f_{i + j} \circ \iota_{i, j} (a \otimes ?) \\
        &\hspace{1cm} + (-1)^i f_{i + j + 1} \circ \iota_{i, j + 1} (a \otimes d_{B, j}(?)) ). \\
        \intertext{Then by consolodating the two terms that post-compose by \( f_{i + j + 1} \) it follows that}
        &= \bigtimes_{j \in \Zb} ( f_{i + j + 1} \circ ( \iota_{i + 1, j} (d_{A, i}(a) \otimes ?) \\
        &\hspace{1cm} + (-1)^i \iota_{i, j + 1} (a \otimes d_{B, j}(?)) ) \\
        &\hspace{1cm} - d_{C, i + j} \circ f_{i + j} \circ \iota_{i, j} (a \otimes ?) ). \\
        \intertext{Then by the definition of the differential of \( A \otimes B \) it follows that}
        &= \bigtimes_{j \in \Zb} ( f_{i + j + 1} \circ d_{A \otimes B, i + j} \tuple{\iota_{i, j}(a \otimes ?)} \\
        &\hspace{1cm} - d_{C, i + j} \circ f_{i + j} ( \iota_{i, j} (a \otimes ?) ) ). \\
        \intertext{Then by \( f \) being a chain homomorphism from \( A \otimes B \) to \( C \) it follows that}
        &= 0.
    \end{align*}
\end{remark}

% TODO: SRC or show.
% TODO: Connect with the above statements.
% https://ncatlab.org/nlab/show/category+of+chain+complexes
\begin{fact}[\( \C(\Mod(R)) \) is symmetric monoidal]
    Let \( R \) be a commutative ring with identity, and let \( \otimes \) denote the tensor product on \( \C \tuple{\Mod(R)} \). Furthermore let \( I \) be the chain complex in \( \C \tuple{\Mod(R)} \) consisting solely of \( 0 \)-objects exept for the \( R \)-module \( R \) in index \( 0 \).

    Then \( \tuple{\C \tuple{\Mod(R)}, \otimes, I} \) is a symmetric closed monoidal category.
\end{fact}

\begin{definition}[DG-category]
    Let \( R \) be a commutative ring with identity.

    Then \( \Cc \) is a \emph{DG-category over \( R \)} if it is a category enriched over \( \C \tuple{\Mod(R)} \).
\end{definition}
This definition also appears in \cite[p. 29]{Jasso-Muro_2023}, except they define it for a field and not a commutative ring with identity as is done in this thesis.
