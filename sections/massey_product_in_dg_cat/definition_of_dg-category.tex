In this thesis the definition of a DG-Category is based on enriched category theory, as it is a modern approach, and by the opinion of the author it is also the most elegeant approach.

\begin{definition}[Tensor product of chain complexes over \( \Mod(R) \)]
    \label{def:tensor_product_of_chain_complexes_over_Mod(R)}
    Let \( R \) be a commutative ring with identity. Furthermore let \( A, B \in \C \tuple{\Mod(R)} \).

    Then, for any  \( n \in \Zb \) define the modules
    \[
        (A \otimes B)_n := \coprod_{p + q = n} A_p \otimes B_q
    \]
    which are a part of the chain complex
    \begin{center}
        \begin{tikzpicture}
            \diagram{m}{1cm}{1cm} {
                A \otimes B: \\
            };
        \end{tikzpicture}
        %
        \begin{tikzpicture}
            \diagram{m}{1cm}{1cm} {
                \cdots \& (A \otimes B)_{-1} \& (A \otimes B)_0 \& (A \otimes B)_1 \& \cdots \\
            };

            \draw[math]
                (m-1-1) edge (m-1-2)
                (m-1-2) edge node {d_{-1}} (m-1-3)
                (m-1-3) edge node {d_0} (m-1-4)
                (m-1-4) edge (m-1-5);
        \end{tikzpicture}
    \end{center}

    Where the differentials, \( d_n \), are defined as follows:
    
    Let \( i, j \in \Zb \) with \( i + j = n \), and let
    \[
        \iota_{i, j}: A_i \otimes B_j \hookrightarrow \tuple{A \otimes B}_n
    \]
    be the canonical split-monomorphism. Furthermore let \( a \otimes b \in A_i \otimes B_j \) be an elementary tensor.

    Then the differential is uniquely defined as follows
    \[
        d_n(\iota_{i, j}(a \otimes b)) := \iota_{i + 1, j}\tuple{d_{A, i}(a) \otimes b} + (-1)^{i} \iota_{i, j + 1}\tuple{a \otimes d_{B, j}(b)}.
    \]

    This is called the \emph{tensor product of chain complexes over \( \Mod(R) \)}.
\end{definition}

\begin{remark}
    The definition of the differentials in \autoref{def:tensor_product_of_chain_complexes_over_Mod(R)} is well defined and unique by the following argument.

    Look at the following diagram
    \begin{equation}
        \label{tikz:differential_of_tensor_product_of_chain_complexes_over_Mod(R)}
        \begin{tikzpicture}
            \diagram{m}{2cm}{2cm} {
                A_i \otimes B_j \& \coprod\limits_{p + q = n} A_p \otimes B_q \\
                A_i \times B_j \& \coprod\limits_{p' + q' = n + 1} A_{p'} \otimes B_{q'} \\
            };

            \draw[math]
                (m-1-1) edge[hook] node {\iota_{i, j}} (m-1-2)
                    edge[dashed] node {\alpha_{i, j}} (m-2-2)
                (m-1-2) edge[dashed] node {\beta} (m-2-2)

                (m-2-1) edge node {f} (m-2-2)
                    edge node {\otimes} (m-1-1);
        \end{tikzpicture}
    \end{equation}
    Let \( f \) be the map defined element-wise as follows
    \[
        a \times b \mapsto \iota_{i + 1, j}\tuple{d_{A, i}(a) \otimes b} + (-1)^{i} \iota_{i, j + 1}\tuple{a \otimes d_{B, j}(b)}
    \]
    One can check that this map is \( R \)-balanced, and therefore a morphism. In addition, by the universal property of tensor product in \( \Mod(R) \), \( f \) induces a unique morphism, \( \alpha_{i, j} \), which is induced from the elementary tensors as follows
    \[
        a \otimes b \mapsto \iota_{i + 1, j}\tuple{d_{A, i}(a) \otimes b} + (-1)^{i} \iota_{i, j + 1}\tuple{a \otimes d_{B, j}(b)}.
    \]
    Since this works for any \( i, j \) as long as \( i + j = n \), one can construct \( \alpha_{i, j} \) for every valid \( i, j \) pair.

    Then by using the universal property of the coproduct one gets the unique map \( \beta \) which is by \autoref{tikz:differential_of_tensor_product_of_chain_complexes_over_Mod(R)} uniquely determined by it's actions on elementary tensors in \( A_i \otimes B_j \) in the following way
    \[
        \iota_{i, j}(a \otimes b) \mapsto \iota_{i + 1, j}\tuple{d_{A, i}(a) \otimes b} + (-1)^{i} \iota_{i, j + 1}\tuple{a \otimes d_{B, j}(b)}.
    \]
    Which is exactly equal to \( d_n \).

    By a similar argument as above, it follows that since \( d_{n + 1} \circ d_n \) sends every \( \iota_{i, j}(a \otimes b) \) to \( 0 \), then it has to be the zero map, and \( d_n \) is therefore a differential.
\end{remark}

\begin{definition}[Internal hom of chain complexes over \( \Mod(R) \)]
    \label{def:internal_hom_of_chain_complexes_over_Mod(R)}
    Let \( R \) be a commutative ring with identity. Furthermore let \( A, B \in \C \tuple{\Mod(R)} \).

    Then, for any \( n \in \Ab \) define the modules
    \[
        \class{A, B}_n := \prod_{i \in \Zb} \Mod(R)(A_i, B_{i + n})
    \]
    which are a part of the chain complex
    \begin{center}
        \begin{tikzpicture}
            \diagram{m}{1cm}{1cm} {
                \class{A, B}: \\
            };
        \end{tikzpicture}
        %
        \begin{tikzpicture}
            \diagram{m}{1cm}{1cm} {
                \cdots \& \class{A, B}_{-1} \& \class{A, B}_0 \& \class{A, B}_1 \& \cdots \\
            };

            \draw[math]
                (m-1-1) edge (m-1-2)
                (m-1-2) edge node {d_{-1}} (m-1-3)
                (m-1-3) edge node {d_0} (m-1-4)
                (m-1-4) edge (m-1-5);
        \end{tikzpicture}
    \end{center}

    Where the differentials, \( d_n \), are defined as follows:
    
    For any \( i \in \Zb \), let \( \partial_{n, i} \) be the follwing morphishm
    \begin{align*}
        \partial_{n, i} : \Mod(R)(A_i, B_{i + n}) &\to \Mod(R)(A_i, B_{i + n + 1}) \\
        f &\mapsto d_B \circ f - (-1)^n f \circ d_A.
    \end{align*}
    Then
    \begin{align*}
        d_n := \prod_{i \in \Zb} \partial_{n, i}: \prod_{i \in \Zb} \Mod(R)(A_i, B_{i + n}) &\to \prod_{i \in \Zb} \Mod(R)(A_i, B_{i + n + 1}) \\
        \bigtimes_{i \in \Zb} f_i &\mapsto \bigtimes_{i \in \Zb} d_B \circ f_i - (-1)^n f_i \circ d_A.
    \end{align*}
    This is called the \emph{internal hom of chain complexes over \( \Mod(R) \)}.
\end{definition}

% SRC: Remarkable
\begin{remark}[Tensor product and internal hom adjunction in \( \C(\Mod(R)) \)]
    Let \( A, B, C \) be chain complexes in \( \C(\Mod(R)) \) for some commutative ring \( R \).

    Let \( f \in \C(\Mod(R))\tuple{A \otimes B, C} \). Want to find out what the adjoint of \( f \) is.

    Let \( f = \set{f_n}_{n \in \Zb} \) where \( f_n \in \Mod(R)\tuple{ \tuple{A \otimes B}_n, C_n} \) are the individual level-wise morphisms of the chain morphism \( f \).

    Then, unwrapping the definitions, one has that \( f \) looks like the following diagram
    \begin{center}
        \begin{tikzpicture}
            \diagram{m}{1cm}{1cm} {
                \cdots \& \coprod\limits_{i \in \Zb} A_i \otimes B_{-1 - i} \& \coprod\limits_{i \in \Zb} A_i \otimes B_{-i} \& \coprod\limits_{i \in \Zb} A_i \otimes B_{1 - i} \& \cdots \\
                \cdots \& C_{-1} \& C_0 \& C_1 \& \cdots \\
            };

            \draw[math]
                (m-1-1) edge (m-1-2)
                    edge (m-2-1)
                (m-1-2) edge node {d_{-1}} (m-1-3)
                    edge node {f_{-1}} (m-2-2)
                (m-1-3) edge node {d_0} (m-1-4)
                    edge node {f_0} (m-2-3)
                (m-1-4) edge (m-1-5)
                    edge node {f_1} (m-2-4)
                (m-1-5) edge (m-2-5)

                (m-2-1) edge (m-2-2)
                (m-2-2) edge node {d_{-1}} (m-2-3)
                (m-2-3) edge node {d_0} (m-2-4)
                (m-2-4) edge (m-2-5);
        \end{tikzpicture}
    \end{center}
    Likewise, the adjoint has to look like the following diagram
    \begin{center}
        \begin{tikzpicture}
            \diagram{m}{1cm}{0.75cm} {
                \cdots \& A_{-1} \& A_0 \& A_1 \& \cdots \\
                \cdots \& \prod\limits_{i \in \Zb} \Mod(R)(B_i, C_{i - 1}) \& \prod\limits_{i \in \Zb} \Mod(R)(B_i, C_i) \& \prod\limits_{i \in \Zb} \Mod(R)(B_i, C_{i + 1}) \& \cdots \\
            };

            \draw[math]
                (m-1-1) edge (m-1-2)
                    edge (m-2-1)
                (m-1-2) edge node {d_{-1}} (m-1-3)
                    edge node {?_{-1}} (m-2-2)
                (m-1-3) edge node {d_0} (m-1-4)
                    edge node {?_0} (m-2-3)
                (m-1-4) edge (m-1-5)
                    edge node {?_1} (m-2-4)
                (m-1-5) edge (m-2-5)

                (m-2-1) edge (m-2-2)
                (m-2-2) edge node {d_{-1}} (m-2-3)
                (m-2-3) edge node {d_0} (m-2-4)
                (m-2-4) edge (m-2-5);
        \end{tikzpicture}
    \end{center}

    For any \( k, j \in \Zb \) let
    \[
        \iota_{k, j}: A_j \otimes B_{k - j} \hookrightarrow \coprod_{i \in \Zb} A_i \otimes B_{k - i}
    \]
    be the canonical \( j \)-th split monomorphism by the definintion of the coproduct \( \coprod_{i \in \Zb} A_i \otimes B_{k - i} \).

    Then take the hom-tensor adjoint in \( \Mod(R) \) of the morphism
    \[
        f_k \circ \iota_{k, j}: A_j \otimes B_{k - j} \to C_k.
    \]
    This yields a morphism
    \begin{align*}
        \phi_{f, j, k}: A_j &\to \Mod(R)(B_{k - j}, C_k) \\
        a &\mapsto f_k \circ \iota_{k, j}(a \otimes ?).
    \end{align*}
    Then, since the choic of \( k \) is independent of the domain, by the universal property of the product there is some morphism
    % TODO: What to do about the change in degrees on rhs. There must be some isomorphism there. How to show this isomorphism for myself?
    \[
        \phi_{f, j} := \prod_{k \in \Zb} \phi_{f, j, k}: A_j \to \prod_{k \in \Zb} \Mod(R)\tuple{B_k, C_{k + j}}.
    \]
    Then the claim is that the adjoint is exactly
    \[
        \phi_f := \set{\phi_{f, j}}_{j \in \Zb}.
    \]
    In order to show that this is the proper adjoint definition, one need to show the following properties
    \begin{enumerate}
        \item {
            That \( \phi_f \) is a chain morphism.
        }
        \item {
            That the assignment \( f \mapsto \phi_f \) is an isomorphism of groups.
        }
        \item {
            That there is a natural transformation
            \[
                \C(\Mod(R))(?_1 \otimes ?_2, ?_3) \cong \C(\Mod(R))(?_1, \left[ ?_2, ?_3 \right])
            \]
            where the natural isomorphisms are \( f \mapsto \phi_f \).
        }
    \end{enumerate}
    % TODO: Possibly expand this proof? or TODO: SRC
    In this thesis I will only prove the first statement.

    1) Want to show that \( \phi_f \) is a chain morphism.

    Need to check that for any \( n \in \Zb \) that the following diagram commutes
    \begin{center}
        \begin{tikzpicture}
            \diagram{m}{1cm}{1cm} {
                A_n \& A_{n + 1} \\
                \class{B, C}_n \& \class{B, C}_{n + 1} \\
            };

            \draw[math]
                (m-1-1) edge node {d_n} (m-1-2)
                    edge node {\phi_{f, n}} (m-2-1)
                (m-1-2) edge node {\phi_{f, n + 1}} (m-2-2)

                (m-2-1) edge node {d_n} (m-2-2);
        \end{tikzpicture}
    \end{center}
    Pick an arbitrary \( a \in A_n \), look at the following equation. WIP
    \begin{align*}
        \phi_{f, n + 1} \circ d_n(a) - d_n \circ \phi_{f, n}(a) &= \phi_{f, n + 1} \circ d_n(a) - 
    \end{align*}
\end{remark}

% TODO: SRC
% https://ncatlab.org/nlab/show/category+of+chain+complexes
\begin{fact}[\( \C(\Mod(R)) \) is symmetric monoidal]
    Let \( R \) be a commutative ring with identity, and let \( \otimes \) denote the tensor product on \( \C \tuple{\Mod(R)} \). Furthermore let \( I \) be the chain complex in \( \C \tuple{\Mod(R)} \) consisting solely of \( 0 \)-objects exept for the \( R \)-module \( R \) in index \( 0 \).

    Then \( \tuple{\C \tuple{\Mod(R)}, \otimes, I} \) is a symmetric closed monoidal category.
\end{fact}

\begin{definition}[DG-category]
    Let \( R \) be a commutative ring with identity.

    Then \( \Cc \) is a \emph{DG-category over \( R \)} if it is a category enriched over \( \C \tuple{\Mod(R)} \).
\end{definition}
This definition also appear in Jasso--Muro p. 29. (TODO: Ref), except they define it for a field and not a commutative ring with identity.

