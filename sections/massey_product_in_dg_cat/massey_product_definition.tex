% MS-Question: Is H^* a full functor on a DG-category? -> It should be full.
% TODO: Decide on notation, class-notation or H^*-notation?
\begin{definition}
    Let \( \Cc \) be a differentially graded category over \( R \).

    Let the following be a diagram in \( \Cc \)
    \begin{center}
        \begin{tikzpicture}
            \diagram{m}{1cm}{1cm} {
                X_1 & X_2 & X_3 & X_4 \\
            };

            \draw[math]
                (m-1-1) edge node {f_1} (m-1-2)
                (m-1-2) edge node {f_2} (m-1-3)
                (m-1-3) edge node {f_3} (m-1-4);
        \end{tikzpicture}
    \end{center}

    Furthermore, let
    \[
        [f_1] = H^* \tuple{f_1} \in H^* \tuple{\Cc \tuple{X_1, X_2}},
    \]
    \[
        [f_2] = H^* \tuple{f_2} \in H^* \tuple{\Cc \tuple{X_2, X_3}},
    \]
    and
    \[
        [f_3] = H^* \tuple{f_3} \in H^* \tuple{\Cc \tuple{X_3, X_4}}.
    \]
    Then let:
    \[
        \set{
            [(-1)^{\abs{f_3} + \abs{f_2}}s f_1 + (-1)^{\abs{f_3} + 1} f_3 t]
            \mid
            d(s) = (-1)^{f_3 + 1} f_3 f_2, \quad
            d(t) = (-1)^{f_2 + 1} f_2 f_1
        }
    \]
    This is a subset of \( H^* \tuple{\Cc \tuple{X_1, X_4}} \), called the \emph{Massey product of \( f_3, f_2 \) and \( f_1 \)}, and is denoted as \( \toda{[f_3], [f_2], [f_1]} \).
\end{definition}

\begin{remark}
    If \( \Cc \) is a differentially graded category over \( R \) with the following diagram
    \begin{center}
        \begin{tikzpicture}
            \diagram{m}{1cm}{1cm} {
                X_1 & X_2 & X_3 & X_4 \\
            };

            \draw[math]
                (m-1-1) edge node {f_1} (m-1-2)
                (m-1-2) edge node {f_2} (m-1-3)
                (m-1-3) edge node {f_3} (m-1-4);
        \end{tikzpicture}
    \end{center}
    where \( \abs{f_1} = d_1, \abs{f_2} = d_2 \), and \( \abs{f_3} = d_3 \).

    Then one gets that
    
\end{remark}