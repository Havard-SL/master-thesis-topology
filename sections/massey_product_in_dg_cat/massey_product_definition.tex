\begin{notation}
    Let \( R \) be a commutative ring with identity.

    Then let
    \[
        H^\bullet: \C\tuple{\Mod(R)} \to \C(\Mod(R))
    \]
    be the cohomology functor, with the differentials on the right hand side all being \( 0 \).
\end{notation}

\begin{definition}
    \label{def:H_bullet_dg_category}
    Let \( \Cc \) be a differentially graded category over  \( R \).

    Let \( H^\bullet(\Cc) \) be the following (enriched over \( C\tuple{\Mod(R)} \)) category:
    \begin{enumerate}
        \item Let \( \Obj(H^\bullet(\Cc)) := \Obj(\Cc) \).
        \item For any \( A, B \in \Obj(H^\bullet(\Cc)) \), let \( H^\bullet(\Cc)(A, B) := H^\bullet \tuple{\Cc(A, B)} \).
        \item {
            For any \( A, B, C \in H^\bullet(\Cc) \), define the composition morphism
            \begin{align*}
                c_{H^\bullet(\Cc)}: H^\bullet(\Cc)(B, C) \otimes H^\bullet(\Cc)(A, B) &\to H^\bullet(\Cc)(A, C)
            \end{align*}
            to be the chain morphism with the \( n \)-th component being
            \begin{align*}
                c_n: (H^\bullet(\Cc)(B, C) \otimes H^\bullet(\Cc)(A, B))_n &\to H^n(A, C) \\
                \iota_{i, j}([g_i] \otimes [f_j]) &\mapsto \class{c_{\Cc,n}\tuple{\iota_{i, j}(g_i \otimes f_j)}}
            \end{align*}
            for any \( i, j \in \Zb \) with \( i + j = n \) and \( [g_i] \in H^i(\Cc)(B, C) \) and \( [f_j] \in H^j(\Cc)(A, B) \).
        }
    \end{enumerate}
\end{definition}
\begin{remark}
    \label{rem:H_bullet_composition_alpha}
    The composition definition in \autoref{def:H_bullet_dg_category} is actually the composition of two different morphisms.

    Look at the maps
    \begin{align*}
        \alpha_{i, j}: H^i(\Cc(B, C)) \times H^j(\Cc(A, B)) &\to H^n(\Cc(B, C) \otimes \Cc(A, B)) \\
        ([g_i], [f_j]) &\mapsto \class{\iota_{i, j}(g_i \otimes f_j)}.
    \end{align*}
    Assuming these are well defined \( R \)-bilinear morphisms, denote the unique morphism they define by \autoref{lem:map_out_of_tensor_unique} by \( \alpha_n \), and the entire chain morphism by \( \alpha := \set{\alpha_i}_{i \in \Zb} \). \footnote{
        The map \( \alpha \) is know at the cross product morphism, as explained in \cite[p. 273]{Hatcher}. In addition for \( R \) a field (as is assumed in \cite{Jasso-Muro_2023}) it is known that by the Algebraic Künneth Theorem that \( \alpha \) is an isomorphism \cite[Theorem 3B.5]{Hatcher}.
    }
    
    Then one can see that
    \[
        c_{H^\bullet(\Cc)} := H^\bullet(c_{\Cc}) \circ \alpha.
    \] 
\end{remark}
\begin{remark}
    The composition definition in \autoref{def:H_bullet_dg_category} is well defined and unique by the following argument.

    By \autoref{rem:H_bullet_composition_alpha} it is sufficient to only have to verify that the \( \alpha_{i, j} \)-s are well defined and \( R \)-balanced in order to show that \( c_{H^\bullet(\Cc)} \) is well defined and unique.

    One can check that the maps are \( R \)-balanced, but one still needs to check if the maps are well defined, which is a bit more difficult. There are two points that need to be shown:
    \begin{enumerate}
        \item {
            Firstly, is
            \[
                \iota_{i, j}(g_i \otimes f_j) \in H^n(\Cc(B, C) \otimes \Cc(A, B))?
            \]
            To show this, one has to verify that
            \[
                \iota_{i, j}(g_i \otimes f_j) \in \ker(d_{\Cc(B, C) \otimes \Cc(A, B), n}).
            \]
            This is true because by assumption, both \( g_i \) and \( f_j \) are cycles, and by definition of the differential of the tensor product
            \[
                d_{\Cc(B, C) \otimes \Cc(A, B), n}(\iota_{i, j}(g_i \otimes f_j)) = 0.
            \]
        }
        \item {
            Secondly, are the values of the \( \alpha_{i, j} \)-s independent of the choice of representative?

            Let \( b_g \) be a boundary in \( \Cc(B, C)_i \), and let \( b_f \) be a boundary in \( \Cc(A, B)_j \).
            \begin{align*}
                \alpha_{i, j}([g_i + b_g], [f_j + b_f]) &= [\iota_{i, j}((g_i + b_g) \otimes (f_j + b_f))] \\
                &= [\iota_{i, j}(g_i \otimes f_j + g_i \otimes b_f + b_g \otimes f_j + b_g \otimes b_f)] \\
                &= [\iota_{i, j}(g_i \otimes f_j) + \iota_{i, j}(g_i \otimes b_f) + \iota_{i, j}(b_g \otimes f_j) + \iota_ {i, j}(b_g \otimes b_f)]
            \end{align*}
            By the definition of the differential of the tensor product, one has that \( \iota_{i, j} \) of the tensor product betweeen a boundary and a cycle, a cycle and a boundary, and a boundary and a boundary are all boundaries in the tensor product.

            As an example, let's take the case above of \( \iota_{i, j}(g_i \otimes b_f) \). Since \( b_f \) is a boundary in \( \Cc(A, B)_j \), there is some \( b_f' \in \Cc(A, B)_{j - 1} \) such that \( d_{\Cc(A, B), j - 1}(b_f') = b_f \).

            Then look at the following equation
            \begin{align*}
                d_{\Cc(B, C) \otimes \Cc(A, B)}(&(-1)^i \iota_{i, j - 1}(g_i, b_f')) \\
                &= (-1)^i \iota_{i + 1, j - 1}(d_{\Cc(B, C), i}(g_i) \otimes b_f') + (-1)^i(-1)^i\iota_{i, j}(g_i \otimes d_{\Cc(A, B), j - 1}(b_f')) \\
                &= (-1)^i \iota_{i + 1, j}(0 \otimes b_f') + \iota_{i, j}(g_i \otimes b_f) \\
                &= \iota_{i, j}(g_i \otimes b_f).
            \end{align*}
            A similar argument can be made for the other cases.

            Therefore is follows that
            \[
                \alpha_{i, j}([g_i + b_g], [f_j + b_f]) = [\iota_{i, j}(g_i \otimes f_j) + \iota_{i, j}(g_i \otimes b_f) + \iota_{i, j}(b_g \otimes f_j) + \iota_ {i, j}(b_g \otimes b_f)] = [\iota_{i, j}(g_i \otimes f_j)]
            \]
            And \( \alpha_{i, j} \) is therefore well-defined.
        }
    \end{enumerate}
\end{remark}

% TODO: Why is degrees preserved when choosing representatives?
% SRC: Wikipedia https://en.wikipedia.org/wiki/Massey_product and TODO:guessing
% TODO: Show that definition is well defined as in massey product makes cocycles and can therefore take representatives.
\begin{definition}
    \label{def:massey_product_in_dg_cat/massey_product_definition/massey_product_dg_cat}
    Let \( \Cc \) be a differentially graded category over \( R \).

    Let the following be a diagram in \( H^\bullet(\Cc) \)
    \begin{center}
        \begin{tikzpicture}
            \diagram{m}{1cm}{1cm} {
                X_1 \& X_2 \& X_3 \& X_4 \\
            };

            \draw[math]
                (m-1-1) edge node {f_1} (m-1-2)
                (m-1-2) edge node {f_2} (m-1-3)
                (m-1-3) edge node {f_3} (m-1-4);
        \end{tikzpicture}
    \end{center}

    Where \( f_1, f_2 \) and \( f_3 \) are homogenous elements with degree \( d_1, d_2 \) and \( d_3 \) respectively.

    Then let \( g_1, g_2 \) and \( g_3 \) be the homogeneous cocycle representatives of \( f_1, f_2 \) and \( f_3 \) respectively from \autoref{lem:massey_product_in_dg_cat/massey_product_definition/exist_lifting_h_star}. Be the same lemma, these will also have degree \( d_1, d_2 \) and \( d_3 \) respectively.

    Furthermore for a homogeneous element, \( h \in \C(\Mod(R)) \) with degree \( d_h \), let \( \bar{h} := (-1)^{d_h + 1}h \).

    Then let:
    \[
        \set{
            \class{
                \bar{s} \circ g_1 + \bar{g_3} \circ t
            }
            \mid
            d(s) = \bar{g_3} \circ g_2, \quad
            d(t) = \bar{g_2} \circ g_1
        }
    \]
    This is a subset of \( H^\bullet \tuple{\Cc \tuple{X_1, X_4}} \) and therefore also a subset of \( H^\bullet(\Cc)(X_1, X_4) \), called the \emph{Massey product of \( f_3, f_2 \) and \( f_1 \)}, and is denoted as \( \massey{f_3, f_2, f_1 } \).
\end{definition}

\begin{theorem}
    The Massey product definition in \autoref{def:massey_product_in_dg_cat/massey_product_definition/massey_product_dg_cat} is well-defined.
\end{theorem}
\begin{proof}
    Want to show that for different choices of representatives, that the definition returns the same subset. As well as that the massey product creates cocycles.

    TODO
\end{proof}

% TODO: Add comment about composition in DG-category adding the degrees of homogeneous elements.
\begin{remark}
    \label{rem:massey_product_in_dg_cat/massey_product_definition/massey_product_sum_of_degrees}
    In particular, considering the degrees of \( f_1, f_2 \) and \( f_3 \), one can see that the only non-zero degree of \( \massey{f_3, f_2, f_1 } \subseteq H^\bullet \tuple{\Cc \tuple{X_1, X_4}} \) is in \( H^{d_3 + d_2 + d_1 - 1} \tuple{\Cc \tuple{X_1, X_4}} \).
\end{remark}

% SRC: Computations done on ReMarkable Massey Product/Definition Massey prod p2
\begin{remark}
    Comparing \autoref{def:massey_product_in_dg_cat/massey_product_definition/massey_product_dg_cat} and the definition for massey product in a DG category given by Jasso--Muro 2023 (TODO: Ref) in ``Definition 4.2.1'' with \( d = 1 \) we can see that both definitions agree.
\end{remark}

% TODO: Add definition for non-homogenous. Is there even a definition for non-homogenous?
