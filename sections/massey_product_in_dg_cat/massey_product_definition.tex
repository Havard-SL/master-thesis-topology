\begin{notation}
    Let \( R \) be a commutative ring with identity.

    Then let \( \Mod_{\bullet}(R) \) denote the category of graded \( R \)-modules.
\end{notation}

\begin{notation}
    Let \( R \) be a commutative ring with identity.

    Then let
    \[
        H^\bullet: \C\tuple{\Mod(R)} \to \Mod_{\bullet}(R)
    \]
    be the cohomology functor.
\end{notation}

% SRC Künneth: https://ncatlab.org/nlab/show/K%C3%BCnneth+theorem
\begin{fact}
    \label{fact:massey_product_in_dg_cat/massey_product_definition/algebraic_kunneth_isomorphism}
    Considering objects of \( \Mod_{\bullet}(R) \) as chain complexes with a zero-differential, one can define a tensor product using \autoref{def:massey_product_in_dg_cat/what_is_a_dg_cat/tensor_product_of_chain_complexes}.

    Let \( A, B \in \C\tuple{\Mod(R)} \).

    By the algebraic Künneth theorem when \( R \) is a field, there is an isomorphism
    \[
        \phi: H^\bullet(A) \otimes H^\bullet(B) \stackrel{\sim}{\to} H^\bullet(A \otimes B)
    \]
    TODO
\end{fact}

\begin{lemma}
    \label{lem:massey_product_in_dg_cat/massey_product_definition/exist_lifting_h_star}
    Let \( \Cc \) be a differentially graded category over \( R \), and let \( A, B \in H^\bullet(\Cc) \). Let \( f \in H^\bullet(\Cc)(A, B) \).

    Then there is a cocycle representative \( g \in \Cc(A, B) \) such that \( [g] = f \).

    Furthermore, if \( f \) is homogeneous of degree \( d \), then there exist an \( g \) that is also homogeneous of degree \( d \).
\end{lemma}
\begin{proof}
    TODO: Prove
\end{proof}

% MS-Question: Does the composition definition work? Also TODO below.
% TODO: Find definition of composition that does not rely on kunneth isomorphism, but only the existance of a morphism. Ask MS!
\begin{definition}
    Let \( \Cc \) be a differentially graded category over  \( R \).

    Let \( H^\bullet(\Cc) \) be the following (enriched over \( C\tuple{\Mod(R)} \)) category:
    \begin{enumerate}
        \item Let \( \Obj(H^\bullet(\Cc)) := \Obj(\Cc) \).
        \item For any \( A, B \in \Obj(H^\bullet(\Cc)) \), let \( H^\bullet(\Cc)(A, B) := H^\bullet \tuple{\Cc(A, B)} \).
        \item {
            % TODO: Need proof of the following statement
            % TODO: Change wording
            For any \( A, B, C \in \Obj(H^\bullet(\Cc)) \) and for any \( f_1 \in H^\bullet(\Cc)(A, B) \) and \( f_2 \in H^\bullet(\Cc)(B, C) \), let \( g_1 \in \Cc(A, B) \) and \( g_2 \in \Cc(B, C) \) be representatives of \( f_1 \) and \( f_2 \), respectively from \autoref{lem:massey_product_in_dg_cat/massey_product_definition/exist_lifting_h_star}.

            % TODO: Slight abuse of notation taking a "class" of an element of a chain complex.
            % TODO: Need to check if this is well defined, as in not dependant on choice of representative as well as that \circ_{\Cc}(g_1 \otimes g_2) is a cocycle.
            Let composition in \( H^\bullet(\Cc) \) be defined on elementary tensors as follows
            \begin{align*}
                \circ_{H^\bullet(\Cc)}: H^\bullet(\Cc)(B, C) \otimes H^\bullet(\Cc)(A, B) &\to H^\bullet(\Cc)(A, C) \\
                f_1 \otimes f_2 &\mapsto \class{ \circ_{\Cc}(g_1 \otimes g_2) }
            \end{align*}
            % Connection to [f] \otimes [g] \mapsto [f \otimes g] \mapsto [\circ(f \otimes g)]?? TODO
            % Is this even well defined/allowed? Need to check if every element in tensor product is a (finite?) sum of elementary tensors. TODO
        }
    \end{enumerate}
\end{definition}

\begin{remark}
    Connection to Kunneth? TODO

    % Let \( \phi \) be as in \autoref{fact:massey_product_in_dg_cat/massey_product_definition/algebraic_kunneth_isomorphism}.

    % For any \( A, B, C \in \Obj(\Kc) \), for any \( f \in \Kc(A, B) \) and \( g \in \Kc(B, C) \)

    %         Let composition in \( \Kc \) be defined as follows
    %         \[
    %             \circ_{\Kc}: \Kc(B, C) \otimes \Kc(A, B) \to \Kc(A, C)
    %         \]
    %         where \( \circ_{\Kc} \) is the composition
    %         \begin{center}
    %             \begin{tikzpicture}
    %                 \diagram{m}{1cm}{1.3cm} {
    %                     \Kc(B, C) \otimes \Kc(A, B) \& \& \Kc(A, C) \\
    %                     H^\bullet\tuple{\Cc(B, C)} \otimes H^\bullet\tuple{\Cc(A, B)} \& H^\bullet\tuple{\Cc(B, C) \otimes \Cc(A, B)} \& H^\bullet\tuple{\Cc(A, C)} \\
    %                 };

    %                 \draw[math]
    %                     (m-1-1) edge[dashed] node {\circ_{\Kc}} (m-1-3)
    %                         edge[equal] (m-2-1)
    %                     (m-1-3) edge[equal] (m-2-3)
                        
    %                     (m-2-1) edge node {\phi} node[swap] {\sim} (m-2-2)
    %                     (m-2-2) edge node {H^\bullet(\circ_{\Cc})} (m-2-3); 
    %             \end{tikzpicture}
    %         \end{center}
\end{remark}

% TODO: Why is degrees preserved when choosing representatives?
% SRC: Wikipedia https://en.wikipedia.org/wiki/Massey_product and TODO:guessing
% TODO: Show that definition is well defined as in massey product makes cocycles and can therefore take representatives.
\begin{definition}
    \label{def:massey_product_in_dg_cat/massey_product_definition/massey_product_dg_cat}
    Let \( \Cc \) be a differentially graded category over \( R \).

    Let the following be a diagram in \( H^\bullet(\Cc) \)
    \begin{center}
        \begin{tikzpicture}
            \diagram{m}{1cm}{1cm} {
                X_1 \& X_2 \& X_3 \& X_4 \\
            };

            \draw[math]
                (m-1-1) edge node {f_1} (m-1-2)
                (m-1-2) edge node {f_2} (m-1-3)
                (m-1-3) edge node {f_3} (m-1-4);
        \end{tikzpicture}
    \end{center}

    Where \( f_1, f_2 \) and \( f_3 \) are homogenous elements with degree \( d_1, d_2 \) and \( d_3 \) respectively.

    Then let \( g_1, g_2 \) and \( g_3 \) be the homogeneous cocycle representatives of \( f_1, f_2 \) and \( f_3 \) respectively from \autoref{lem:massey_product_in_dg_cat/massey_product_definition/exist_lifting_h_star}. Be the same lemma, these will also have degree \( d_1, d_2 \) and \( d_3 \) respectively.

    Furthermore for a homogeneous element, \( h \in \C(\Mod(R)) \) with degree \( d_h \), let \( \bar{h} := (-1)^{d_h + 1}h \).

    Then let:
    \[
        \set{
            \class{
                \bar{s} \circ g_1 + \bar{g_3} \circ t
            }
            \mid
            d(s) = \bar{g_3} \circ g_2, \quad
            d(t) = \bar{g_2} \circ g_1
        }
    \]
    This is a subset of \( H^\bullet \tuple{\Cc \tuple{X_1, X_4}} \) and therefore also a subset of \( H^\bullet(\Cc)(X_1, X_4) \), called the \emph{Massey product of \( f_3, f_2 \) and \( f_1 \)}, and is denoted as \( \massey{f_3, f_2, f_1 } \).
\end{definition}

\begin{theorem}
    The Massey product definition in \autoref{def:massey_product_in_dg_cat/massey_product_definition/massey_product_dg_cat} is well-defined.
\end{theorem}
\begin{proof}
    Want to show that for different choices of representatives, that the definition returns the same subset. As well as that the massey product creates cocycles.

    TODO
\end{proof}

% TODO: Add comment about composition in DG-category adding the degrees of homogeneous elements.
\begin{remark}
    \label{rem:massey_product_in_dg_cat/massey_product_definition/massey_product_sum_of_degrees}
    In particular, considering the degrees of \( f_1, f_2 \) and \( f_3 \), one can see that the only non-zero degree of \( \massey{f_3, f_2, f_1 } \subseteq H^\bullet \tuple{\Cc \tuple{X_1, X_4}} \) is in \( H^{d_3 + d_2 + d_1 - 1} \tuple{\Cc \tuple{X_1, X_4}} \).
\end{remark}

% SRC: Computations done on ReMarkable Massey Product/Definition Massey prod p2
\begin{remark}
    Comparing \autoref{def:massey_product_in_dg_cat/massey_product_definition/massey_product_dg_cat} and the definition for massey product in a DG category given by Jasso--Muro 2023 (TODO: Ref) in ``Definition 4.2.1'' with \( d = 1 \) we can see that both definitions agree.
\end{remark}

% TODO: Add definition for non-homogenous. Is there even a definition for non-homogenous?
