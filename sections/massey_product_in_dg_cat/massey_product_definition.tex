\begin{notation}
    Let \( R \) be a commutative ring with identity.

    Then let \( G \Mod(R) \) denote the category of graded \( R \)-modules.
\end{notation}

\begin{notation}
    Let \( R \) be a commutative ring with identity.

    Then let
    \[
        H^*: C \tuple{\Mod(R)} \to G \Mod(R)
    \]
    be the cohomology functor.
\end{notation}

% SRC Künneth: https://ncatlab.org/nlab/show/K%C3%BCnneth+theorem
% MS-Question: Isn't cohomology contravariant?
\begin{fact}
    \label{fact:massey_product_in_dg_cat/massey_product_definition/algebraic_kunneth_isomorphism}
    Considering objects of \( G \Mod(R) \) as chain complexes with a zero-differential, one can define a tensor product using \autoref{def:massey_product_in_dg_cat/what_is_a_dg_cat/tensor_product_of_chain_complexes}.

    Let \( A, B \in C \tuple{\Mod(R)} \).

    By the algebraic Künneth theorem when \( R \) is a field, there is an isomorphism
    \[
        \phi: H^*(A) \otimes H^*(B) \stackrel{\sim}{\to} H^*(A \otimes B)
    \]
    TODO
\end{fact}

% MS-Question: Does the composition definition work?
\begin{definition}
    Let \( \Cc \) be a differentially graded category over a field \( R \), and let \( \phi \) be as in \autoref{fact:massey_product_in_dg_cat/massey_product_definition/algebraic_kunneth_isomorphism}.

    Let \( \Kc \) be the following (enriched) category:
    \begin{enumerate}
        \item Let \( \Obj(\Kc) = \Obj(\Cc) \).
        \item For any \( A, B \in \Obj(\Kc) \), let \( \Kc(A, B) = H^* \tuple{\Cc(A, B)} \).
        \item {
            For any \( A, B, C \in \Obj(\Kc) \), for any \( f \in \Kc(A, B) \) and \( g \in \Kc(B, C) \)

            Let composition in \( \Kc \) be defined as follows
            \[
                \circ_{\Kc}(g \otimes f): \Kc(B, C) \otimes \Kc(A, B) \to \Kc(A, C)
            \]
            where \( \circ_{\Kc} \) is the composition
            \begin{center}
                \begin{tikzpicture}
                    \diagram{m}{1cm}{1.3cm} {
                        \Kc(B, C) \otimes \Kc(A, B) & & \Kc(A, C) \\
                        H^*\tuple{\Cc(B, C)} \otimes H^*\tuple{\Cc(A, B)} & H^*\tuple{\Cc(B, C) \otimes \Cc(A, B)} & H^*\tuple{\Cc(A, C)} \\
                    };

                    \draw[math]
                        (m-1-1) edge[dashed] node {\circ_{\Kc}} (m-1-3)
                            edge[equal] (m-2-1)
                        (m-1-3) edge[equal] (m-2-3)
                        
                        (m-2-1) edge node {\phi} node[swap] {\sim} (m-2-2)
                        (m-2-2) edge node {H^*(\circ_{\Cc})} (m-2-3); 
                \end{tikzpicture}
            \end{center}
        }
    \end{enumerate}

    Then the category \( \Kc \) is denoted as \( H^*(\Cc) \).
\end{definition}

% TODO: Fix definition, should be defined on H^*(\Cc)!!
\begin{definition}
    \label{def:massey_product_in_dg_cat/massey_product_definition/massey_product_dg_cat}
    Let \( \Cc \) be a differentially graded category over \( R \).

    Let the following be a diagram in \( \Cc \)
    \begin{center}
        \begin{tikzpicture}
            \diagram{m}{1cm}{1cm} {
                X_1 & X_2 & X_3 & X_4 \\
            };

            \draw[math]
                (m-1-1) edge node {f_1} (m-1-2)
                (m-1-2) edge node {f_2} (m-1-3)
                (m-1-3) edge node {f_3} (m-1-4);
        \end{tikzpicture}
    \end{center}

    Where \( f_1, f_2 \) and \( f_3 \) are homogenous elements with degree \( d_1, d_2 \) and \( d_3 \) respectively.

    Then let:
    \[
        \set{
            H^* \tuple{
                (-1)^{\abs{f_2}} \tuple{ (-1)^{\abs{f_3}} f_3 \circ t - s \circ f_1 }
            }
            \mid
            d(s) = f_3 \circ f_2, \quad
            d(t) = f_2 \circ f_1
        }
    \]
    This is a subset of \( H^* \tuple{\Cc \tuple{X_1, X_4}} \), called the \emph{Massey product of \( f_3, f_2 \) and \( f_1 \)}, and is denoted as \( \toda{H^* \tuple{f_3}, H^* \tuple{f_2}, H^* \tuple{f_1} } \).
\end{definition}

\begin{remark}
    In particular, considering the degrees of \( f_1, f_2 \) and \( f_3 \), one can see that the only non-zero degree of \( \toda{H^* \tuple{f_3}, H^* \tuple{f_2}, H^* \tuple{f_1} } \subset H^* \tuple{\Cc \tuple{X_1, X_4}} \) is in \( H^{d_3 + d_2 + d_1 - 1} \tuple{\Cc \tuple{X_1, X_4}} \).
\end{remark}

% Computations done on ReMarkable Massey Product/Definition Massey prod p2
\begin{remark}
    Comparing \autoref{def:massey_product_in_dg_cat/massey_product_definition/massey_product_dg_cat} and the definition for massey product in a DG category given by Jasso--Muro 2023 (TODO: Ref) in ``Definition 4.2.1'' with \( d = 1 \) we can see that both definitions agree.
\end{remark}

% TODO: Add definition for non-homogenous. Is there even a definition for non-homogenous?