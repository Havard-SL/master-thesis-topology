\begin{notation}
    Let \( R \) be a commutative ring with identity.

    Then let
    \[
        H^\bullet: \C\tuple*{\Mod(R)} \to \C(\Mod(R))
    \]
    be the cohomology functor, with the differentials on the right hand side all being \( 0 \).
\end{notation}

% TODO: Unit morphisms?
\begin{definition}[\( H^\bullet(\Cc) \)]
    \label{def:H_bullet_dg_category}
    Let \( \Cc \) be a differentially graded category over  \( R \).

    Let \( H^\bullet(\Cc) \) be the following (enriched over \( C\tuple*{\Mod(R)} \)) category:
    \begin{enumerate}
        \item Let \( \Obj(H^\bullet(\Cc)) := \Obj(\Cc) \).
        \item For any \( A, B \in \Obj(H^\bullet(\Cc)) \), let \( H^\bullet(\Cc)(A, B) := H^\bullet \tuple*{\Cc(A, B)} \).
        \item {
            For any \( A, B, C \in H^\bullet(\Cc) \), define the composition morphism
            \begin{align*}
                c_{H^\bullet(\Cc)}: H^\bullet(\Cc)(B, C) \otimes H^\bullet(\Cc)(A, B) &\to H^\bullet(\Cc)(A, C)
            \end{align*}
            to be the chain morphism with the \( n \)-th component being
            \begin{align*}
                c_n: (H^\bullet(\Cc)(B, C) \otimes H^\bullet(\Cc)(A, B))_n &\to H^n(A, C) \\
                [g_i] \otimes [f_j] &\mapsto \class*{c_{\Cc,n}\tuple*{g_i \otimes f_j}}
            \end{align*}
            for any \( i, j \in \Zb \) with \( i + j = n \) and \( [g_i] \in H^i(\Cc)(B, C) \) and \( [f_j] \in H^j(\Cc)(A, B) \).
        }
    \end{enumerate}
\end{definition}
\begin{remark}
    \label{rem:H_bullet_composition_alpha}
    The composition definition in \autoref{def:H_bullet_dg_category} is actually the composition of two different morphisms.

    Consider the maps
    \begin{align*}
        \alpha_{i, j}: H^i(\Cc(B, C)) \times H^j(\Cc(A, B)) &\to H^n(\Cc(B, C) \otimes \Cc(A, B)) \\
        ([g_i], [f_j]) &\mapsto \class*{g_i \otimes f_j}.
    \end{align*}
    Assuming these are well-defined \( R \)-bilinear morphisms, denote the unique morphism they define by \autoref{lem:map_out_of_tensor_unique} by \( \alpha_n \), and the entire chain morphism by \( \alpha := \set*{\alpha_i}_{i \in \Zb} \). \footnote{
        The map \( \alpha \) is know at the cross product morphism, as explained in \cite[p. 273]{Hatcher_2002}. In addition for \( R \) a field (as is assumed in \cite{Jasso-Muro_2023}) it is known that by the Algebraic Künneth Theorem that \( \alpha \) is an isomorphism \cite[Theorem 3B.5]{Hatcher_2002}.
    }
    
    Then one can see that
    \[
        c_{H^\bullet(\Cc)} := H^\bullet(c_{\Cc}) \circ \alpha.
    \] 
\end{remark}
\begin{remark}
    \label{rem:composition_in_H_bullet_is_well_defined}
    The composition definition in \autoref{def:H_bullet_dg_category} is well-defined and unique by the following argument.

    By \autoref{rem:H_bullet_composition_alpha} it is sufficient to only have to verify that the \( \alpha_{i, j} \)'s are well-defined and \( R \)-balanced in order to show that \( c_{H^\bullet(\Cc)} \) is well-defined and unique.

    One can check that the maps are \( R \)-balanced, but one still needs to check if the maps are well-defined, which is a bit more difficult. There are two points that need to be shown:
    \begin{enumerate}
        \item {
            Firstly, is
            \[
                g_i \otimes f_j \in H^n(\Cc(B, C) \otimes \Cc(A, B))?
            \]
            To show this, one has to verify that
            \[
                g_i \otimes f_j \in \ker(d_{\Cc(B, C) \otimes \Cc(A, B), n}).
            \]
            This is true because by assumption, both \( g_i \) and \( f_j \) are cycles, and by definition of the differential of the tensor product
            \[
                d_{\Cc(B, C) \otimes \Cc(A, B), n}(g_i \otimes f_j) = 0.
            \]
        }
        \item {
            Secondly, are the values of the \( \alpha_{i, j} \)'s independent of the choice of representative?

            Let \( b_g \) be a boundary in \( \Cc(B, C)_i \), and let \( b_f \) be a boundary in \( \Cc(A, B)_j \).
            \begin{align*}
                \alpha_{i, j}([g_i + b_g], [f_j + b_f]) &= [(g_i + b_g) \otimes (f_j + b_f)] \\
                &= [g_i \otimes f_j + g_i \otimes b_f + b_g \otimes f_j + b_g \otimes b_f] \\
                &= [g_i \otimes f_j + g_i \otimes b_f + b_g \otimes f_j + b_g \otimes b_f]
            \end{align*}
            By the definition of the differential of the tensor product, one has that the tensor product betweeen a boundary and a cycle, a cycle and a boundary, and a boundary and a boundary are all boundaries in the tensor product.

            As an example, let's take the case above of \( g_i \otimes b_f \). Since \( b_f \) is a boundary in \( \Cc(A, B)_j \), there is some \( b_f' \in \Cc(A, B)_{j - 1} \) such that \( d_{\Cc(A, B), j - 1}(b_f') = b_f \). Consider the following equation
            \begin{align*}
                d_{\Cc(B, C) \otimes \Cc(A, B)}(&(-1)^i g_i \otimes b_f') \\
                &= (-1)^i d_{\Cc(B, C), i}(g_i) \otimes b_f' + (-1)^i (-1)^i g_i \otimes d_{\Cc(A, B), j - 1}(b_f') \\
                &= (-1)^i 0 \otimes b_f' + g_i \otimes b_f \\
                &= g_i \otimes b_f.
            \end{align*}
            A similar argument can be made for the other cases.

            Therefore is follows that
            \[
                \alpha_{i, j}([g_i + b_g], [f_j + b_f]) = [g_i \otimes f_j + g_i \otimes b_f + b_g \otimes f_j + b_g \otimes b_f] = [g_i \otimes f_j]
            \]
            And \( \alpha_{i, j} \) is therefore well-defined.
        }
    \end{enumerate}
    There is no need to verify if composition is a chain morphism, since the differential is zero by definition.
\end{remark}
\begin{remark}
    \label{rem:diagram_dg-category}
    Now there is an issue with usual category theory notation of a ``diagram'' for DG-categories. Because a DG-category doesn't have the notion of a morphism because there is no element of a chain complex, there can't be a diagram since the arrows would correspond to morphisms. However, a notion of a diagram in a DG-category make certain future results easier to both state and understand, at the cost of some difficulty in the proofs. And that is why the following notation convention will be used:
\end{remark}
% TODO: Maybe move this notation above the definition of H-bullet of a DG category as it is used in the definition.
% TODO: Remove all mentions of iota_i, j's in every massety section. They are implied.
% TODO: Maybe a separate notation for composition of morphisms, since morphisms are not always a apart of a dg-diagram.
\begin{notation}
    By the discussion from \autoref{rem:diagram_dg-category}, it is prudent to properly define what a ``diagram'' in a DG-category is.

    Let \( \Cc \) be a DG-category. A \emph{diagram in a DG-category} (or a \emph{DG-diagram}) is a quiver \( \Gamma = \tuple*{V, E, s, t} \) where every vertex in \( V \) corresponds to an object in \( \Cc \), and every edge in \( e \in E \) corresponds to an element in \( \Cc\tuple*{s(e), t(e)}_n \) for some \( n \in \Zb \). These edges are called \emph{morphisms} and \( n \) is called the \emph{degree} of \( e \), and is also denoted as \( |e| \).

    For two morphisms \( f, g \in E \) with \( s(g) = t(f) \), they are denoted as \emph{composable}. And their composition, denoted \( g \circ f \), is the morphism
    \[
        g \circ f := c_{\Cc, |g| + |f|}(g \otimes f) \in \Cc\tuple*{s(f), t(g)}_{|g| + |f|}
    \]
    where
    \[
        c_{\Cc} := \set*{c_{\Cc, i}}_{i \in \Zb} \text{ with } c_{\Cc, i}: \tuple*{ \Cc\tuple*{s(g), t(g)} \otimes \Cc\tuple*{s(f), t(f)} }_i \to \Cc\tuple*{s(f), t(g)}_i
    \]
    is the composition chain morphism for \( \Cc \).
\end{notation}

% TODO: Obvious?
\begin{remark}
    Composition in a DG-diagram is essentially usual composition in the DG-category, but restricted to just one component of the coproduct.
\end{remark}

\begin{lemma}[Associativity of composition in a DG-diagram]
    \label{lem:dg-composition_associative}
    Let the following be a DG-diagram in a DG-category \( \Cc \)
    \begin{center}
        \begin{tikzpicture}
            \diagram{m}{1cm}{1cm} {
                A \& B \& C \& D. \\
            };

            \draw[math]
                (m-1-1) edge node {f} (m-1-2)
                (m-1-2) edge node {g} (m-1-3)
                (m-1-3) edge node {h} (m-1-4);
        \end{tikzpicture}
    \end{center}

    Then \( h \circ (g \circ f) = (h \circ g) \circ f \).
\end{lemma}
\begin{proof}
    Expanding the definitions, it is necessary to show that the following equation holds (the category is omitted for readability)
    \begin{equation}
        \label{eq:dg-composition_associative}
        c_{|h| + |g| + |f|}\tuple*{h \otimes \tuple*{c_{|g| + |f|}(g \otimes f)}} = c_{|h| + |g| + |f|}\tuple*{\tuple*{c_{|h| + |g|}(h \otimes g)} \otimes f}.
    \end{equation}
    % TODO: This needs a proof. In RM: "Associativity of dg composition" p. 2, there is a diagram of how the diagram could exist. Would need to show that the stated composition of (natural?) isomorphisms are in fact exactly the associativity map.
    I state without proof that the following diagram in \( \Mod(R) \) commutes
    \begin{center}
        \begin{tikzpicture}
            \diagram{m}{1cm}{0.89cm} {
                \Cc(C, D)_{|h|} \otimes \tuple*{ \Cc(B, C)_{|g|} \otimes \Cc(A, B)_{|f|} } \& \tuple*{ \Cc(C, D)_{|h|} \otimes \Cc(B, C)_{|g|} } \otimes \Cc(A, B)_{|f|} \\
                \Cc(C, D)_{|h|} \otimes \tuple*{ \Cc(B, C) \otimes \Cc(A, B) }_{|g| + |f|} \& \tuple*{ \Cc(C, D) \otimes \Cc(B, C) }_{|h| + |g|} \otimes \Cc(A, B)_{|f|} \\
                \tuple*{ \Cc(C, D) \otimes ( \Cc(B, C) \otimes \Cc(A, B) ) }_{|h| + |g| + |f|} \& \tuple*{ ( \Cc(C, D) \otimes \Cc(B, C) ) \otimes \Cc(A, B) }_{|h| + |g| + |f|}. \\
            };

            \draw[math]
                (m-1-1) edge node {\sim} (m-1-2)
                    edge node {\Id \otimes \iota_{|g|, |f|}} (m-2-1)
                (m-1-2) edge node {\iota_{|h|, |g|} \otimes \Id} (m-2-2)

                (m-2-1) edge node {\iota_{|h|, |g| + |f|}} (m-3-1)
                (m-2-2) edge node {\iota_{|h| + |g|, |f|}} (m-3-2)

                (m-3-1) edge node {\sim} node[swap] {a} (m-3-2);
        \end{tikzpicture}
    \end{center}
    Where the top horizontal morphism is the usual associativity morphism of the tensor product of \( R \)-modules, and \( a \) is the associativity morphism for the tensor product of chain complexes (in degree \( |h| + |g| + |f| \) ).

    And by \cite[Definition 6.2.1]{Borceux_1994} the following diagram of chain morphisms commute
    \begin{center}
        \begin{tikzpicture}
            \diagram{m}{1cm}{1cm} {
                \Cc(C, D) \otimes \tuple*{ \Cc(B, C) \otimes \Cc(A, B) } \& \& \tuple*{ \Cc(C, D) \otimes \Cc(B, C) } \otimes \Cc(A, B) \\
                \Cc(C, D) \otimes \Cc(A, C) \& \& \Cc(B, D) \otimes \Cc(A, B) \\
                \& \Cc(A, D). \\
            };
            
            \draw[math]
                (m-1-1) edge node {\sim} node[swap] {a} (m-1-3)
                    edge node {\Id \otimes c} (m-2-1)
                (m-1-3) edge node {c \otimes \Id} (m-2-3)

                (m-2-1) edge node {c} (m-3-2)
                (m-2-3) edge node[swap] {c} (m-3-2);
        \end{tikzpicture}
    \end{center}
    Gluing together the top diagram with the bottom diagram restricted to degree \( |h| + |g| + |f| \), and looking at where the element \( h \otimes (g \otimes f) \) is sent, yields exactly \autoref{eq:dg-composition_associative}.
\end{proof}

\begin{lemma}
    Composition in a dg-diagram is a morphism in \( \Mod(R) \), in particular, it is \( R \)-linear in both components.
\end{lemma}
\begin{proof}
    By definition, composition in a dg-diagram is the composition of two \( \Mod(R) \) morphisms, \( \iota_{i, j} \) and \( c_{\Cc, i + j} \), and so it is a \( \Mod(R) \) morphism.
\end{proof}

% SRC: Jasso-Muro
% TODO: What is the relation between different reprersentatives and different choices of s & t? Does it matter, and how does it matter?
% TODO: Write definition clearer? A lot of implied domains and codomains.
\begin{definition}
    \label{def:massey_product_dg_cat}
    Let \( \Cc \) be a differentially graded category over a commutative ring with identity \( R \).

    Let the following be a DG-diagram in \( H^\bullet(\Cc) \)
    \begin{center}
        \begin{tikzpicture}
            \diagram{m}{1cm}{1cm} {
                X_1 \& X_2 \& X_3 \& X_4 \\
            };

            \draw[math]
                (m-1-1) edge node {f_1} (m-1-2)
                (m-1-2) edge node {f_2} (m-1-3)
                (m-1-3) edge node {f_3} (m-1-4);
        \end{tikzpicture}
    \end{center}

    Furthermore for an element, \( h \in \C(\Mod(R))_{|h|} \) (like a morphism in a DG-diagram), let \( \bar{h} := (-1)^{|h| + 1}h \).

    % Then for any choice of cocycle representatives of \( f_1, f_2 \) and \( f_3 \), denoted \( g_1, g_2 \) and \( g_3 \) which would correspond to the following DG-diagram in \( \Cc \)
    % \begin{center}
    %     \begin{tikzpicture}
    %         \diagram{m}{1cm}{1cm} {
    %             X_1 \& X_2 \& X_3 \& X_4 \\
    %         };

    %         \draw[math]
    %             (m-1-1) edge node {g_1} (m-1-2)
    %             (m-1-2) edge node {g_2} (m-1-3)
    %             (m-1-3) edge node {g_3} (m-1-4);
    %     \end{tikzpicture}
    % \end{center}
    % where \( |g_i| = |f_i| \) by definition.
    Then let
    \begin{multline*}
        \massey{f_3, f_2, f_1 } :=
        \{
            \class*{
                \bar{s} \circ g_1 + \bar{g_3} \circ t
            }
            \mid \\
            d_{\Cc, |f_3| + |f_2| - 1}(s) = \bar{g_3} \circ g_2, \quad
            d_{\Cc, |f_2| + |f_1| - 1}(t) = \bar{g_2} \circ g_1
            \mid
            [g_i] = f_i, i = 1,2,3
        \}.
    \end{multline*}

    This is a subset of
    \[
        H^{|f_1| + |f_2| + |f_3| - 1}\tuple*{\Cc}\tuple*{X_1, X_4}
    \]
    and is called the \emph{Massey product of \( f_3, f_2 \) and \( f_1 \)}.
\end{definition}

\begin{remark}
    The definition of Massey product in \autoref{def:massey_product_dg_cat} is well-defined by the following argument:

    Want to show that \( \bar{s} \circ g_1 + \bar{g_3} \circ t \) is a cocyle in \( \Cc(X_1, X_4)_{|g_1| + |g_2| + |g_3| - 1} \).
    
    Look at the following equation (with the degrees of the differentials omitted)
    \begin{align*}
        d_{\Cc(X_1, X_4)}(\bar{s} \circ g_1 &+ \bar{g_3} \circ t) = d_{\Cc(X_1, X_4)}(\bar{s} \circ g_1) + d_{\Cc(X_1, X_4)}(\bar{g_3} \circ t) \\
        \intertext{then by the definition of composition of morphisms in a dg-diagram}
        &= d_{\Cc(X_1, X_4)}(c_{\Cc, |g_1| + |g_2| + |g_3| - 1}(\iota_{|g_2| + |g_3| - 1, |g_1|}(\bar{s} \otimes g_1))) \\
        &\hspace{0.4cm} + d_{\Cc(X_1, X_4)}(c_{\Cc, |g_1| + |g_2| + |g_3| - 1}(\iota_{|g_3|, |g_1| + |g_2| - 1}(\bar{g_3} \otimes t))) \\
        \intertext{then by the fact that composition is a chain morphism}
        &= c_{\Cc, |g_1| + |g_2| + |g_3|}(d_{\Cc(X_2, X_4) \otimes \Cc(X_1, X_2)}(\iota_{|g_2| + |g_3| - 1, |g_1|}(\bar{s} \otimes g_1))) \\
        &\hspace{0.4cm} + c_{\Cc, |g_1| + |g_2| + |g_3|}(d_{\Cc(X_3, X_4) \otimes \Cc(X_1, X_3)}(\iota_{|g_3|, |g_1| + |g_2| - 1}(\bar{g}_3 \otimes t)))
        \intertext{then by the definition of the differential of the tensor products}
        &= c_{\Cc, |g_1| + |g_2| + |g_3|}(\iota_{|g_2| + |g_3|, |g_1|}((-1)^{|g_2| + |g_3|}(\bar{g}_3 \circ g_2) \otimes g_1)) \\
        &\hspace{0.4cm} + c_{\Cc, |g_1| + |g_2| + |g_3|}(\iota_{|g_3|, |g_1| + |g_2|}((-1)^{|g_3|}\bar{g}_3 \otimes (\bar{g}_2 \circ g_1))) \\
        \intertext{then by pulling out every sign}
        &= (-1)^{|g_2| + 2|g_3| - 1}c_{\Cc, |g_1| + |g_2| + |g_3|}(\iota_{|g_2| + |g_3|, |g_1|}((g_3 \circ g_2) \otimes g_1)) \\
        &\hspace{0.4cm} + (-1)^{|g_2| + 2|g_3| - 2}c_{\Cc, |g_1| + |g_2| + |g_3|}(\iota_{|g_3|, |g_1| + |g_2|}(g_3 \otimes (g_2 \circ g_1))) \\
        \intertext{then by the definition of composition of morphisms in a dg-diagram}
        &= (-1)^{|g_2| + 2|g_3| - 1}(g_3 \circ g_2) \circ g_1 + (-1)^{|g_2| + 2|g_3| - 2}g_3 \circ (g_2 \circ g_1) \\
        \intertext{then by simplifying signs}
        &= (-1)^{|g_2|}\tuple*{ g_3 \circ (g_2 \circ g_1) - (g_3 \circ g_2) \circ g_1 } \\
        \intertext{then by \autoref{lem:dg-composition_associative} it follows that}
        &= 0.
    \end{align*}
    % Secondly, what to show that the massey product is independent of the choice of cocyle-representatives.

    % Let \( b_i \in \Cc(X_i, X_{i + 1}) \) for \( i = 1, 2 \) and \( 3 \) be boundaries.

    % Then look at
    % \[
    %     \set*{ \class*{ \overline{s} \circ (g_1 + b_1) + (\overline{g_3 + b_3}) \circ t } \mid d(s) = (\overline{g_3 + b_3}) \circ (g_2 + b_2), \quad d(t) = (\overline{g_2 + b_2}) \circ (g_1 + b_1) }.
    % \]
    
    % By linearity, the first part can be written as 
    % \[
    %     [\overline{s} \circ g_1 + \overline{s} \circ b_1 + g_3 \circ t + b_3 \circ t].
    % \]
    % But since \( b_1 \) is a boundary, by a similar argument as in \autoref{rem:composition_in_H_bullet_is_well_defined} item 2, one has that
    % \[
    %     \overline{s} \circ b_1 = c_{|g_1| + |g_2| + |g_3| - 1}( \iota_{|g_2| + |g_3| - 1, |g_1|} ( \overline{s} \otimes b_1 ) )
    % \]
    % is in fact
    % \[
    %     \overline{s} \circ b_1 = c_{|g_1| + |g_2| + |g_3| - 1}( d_{\Cc(X_2, X_4) \otimes \Cc(X_1, X_2), |g_1| + |g_2| + |g_3| - 2} (\tilde{g}) )
    % \]
    % for some \( \tilde{g} \in (\Cc(X_2, X_4) \otimes \Cc(X_1, X_2))_{|g_1| + |g_2| + |g_3| - 2} \).

    % And since \( c \) is a chain morphism, it follows that
    % \[
    %     \overline{s} \circ b_1 = d_{\Cc(X_1, X_4), |g_1| + |g_2| + |g_3| - 2}(c_{|g_1| + |g_2| + |g_3| - 2}(\tilde{g}))
    % \]
    % and so \( \overline{s} \circ b_1 \) is a boundary, and a similar argument can be made of \( b_3 \circ t \).

    % It remains to show that
    % \[
    %     \set*{ \class*{ \overline{s} \circ g_1 + \overline{g_3} \circ t } \mid d(s) = (\overline{g_3 + b_3}) \circ (g_2 + b_2), \quad d(t) = (\overline{g_2 + b_2}) \circ (g_1 + b_1) }.
    % \]
    % is independent of the \( b_i \)'s.
\end{remark}

% SRC: Computations done on ReMarkable Massey Product/Definition Massey prod p2
\begin{remark}
    Comparing \autoref{def:massey_product_dg_cat} and the definition for massey product in a DG category given by \cite[Definition 4.2.1]{Jasso-Muro_2023} with \( d = 1 \) we can see that both definitions agree.
\end{remark}
