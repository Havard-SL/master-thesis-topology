\begin{definition}
    Let \( \Cc \) be a differentially graded category over \( R \).

    Let the following be a diagram in \( \Cc \)
    \begin{center}
        \begin{tikzpicture}
            \diagram{m}{1cm}{1cm} {
                X_1 & X_2 & X_3 & X_4 \\
            };

            \draw[math]
                (m-1-1) edge node {f_1} (m-1-2)
                (m-1-2) edge node {f_2} (m-1-3)
                (m-1-3) edge node {f_3} (m-1-4);
        \end{tikzpicture}
    \end{center}

    Where \( f_1, f_2 \) and \( f_3 \) are homogenous elements with degree \( d_1, d_2 \) and \( d_3 \) respectively.

    Furthermore, let
    \[
        H^*(-): DGAlg_R \to GRing
    \]
    be the cohomology functor.

    Then let:
    \[
        \set{
            H^* \tuple{
                (-1)^{\abs{f_3} + \abs{f_2}}s \circ f_1 + (-1)^{\abs{f_3} + 1} f_3 \circ t
            }
            \mid
            d(s) = (-1)^{f_3 + 1} f_3 \circ f_2, \quad
            d(t) = (-1)^{f_2 + 1} f_2 \circ f_1
        }
    \]
    This is a subset of \( H^* \tuple{\Cc \tuple{X_1, X_4}} \), called the \emph{Massey product of \( f_3, f_2 \) and \( f_1 \)}, and is denoted as \( \toda{H^* \tuple{f_3}, H^* \tuple{f_2}, H^* \tuple{f_1} } \).
\end{definition}

Can rewrite last part to:
\[
    \set{
        H^* \tuple{
             (-1)^{\abs{f_2}} \tuple{ (-1)^{\abs{f_3}} f_3 \circ t - s \circ f_1 }
        }
        \mid
        d(s) = f_3 \circ f_2, \quad
        d(t) = f_2 \circ f_1
    }
\]

\begin{remark}
    In particular, considering the degrees of \( f_1, f_2 \) and \( f_3 \), one can see that the only non-zero degree of \( \toda{H^* \tuple{f_3}, H^* \tuple{f_2}, H^* \tuple{f_1} } \subset H^* \tuple{\Cc \tuple{X_1, X_4}} \) is in \( H^{d_3 + d_2 + d_1 - 1} \tuple{\Cc \tuple{X_1, X_4}} \).
\end{remark}

% TODO: Add definition for non-homogenous