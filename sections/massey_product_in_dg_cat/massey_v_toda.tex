% First need to extend massey-prod definition to H^0

% MS-Question: I Jasso--Muro så er dette berre definert for element av dgMod (Ingen subscript!)
% TODO: This is also the same shift functor that makes H^0(dgmodblabla) triangulated. Probably not neccesary to specify then.
\begin{proposition}
    Let \( \Cc \) be a DG-category over \( R \). Let \( \Sigma \) be the shift functor on \( \C_{\dg}(\Mod(R)) \). Let \( A, B \in \dgMod_{\dg} \).

    Then
    \[
        H^i(\dgMod_{\dg}(\Cc)(A, B)) \cong H^0(\dgMod_{\dg}(\Cc))(A, \Sigma^i \circ B).
    \]
\end{proposition}
\begin{proof}
    TODO
\end{proof}

% TODO: H^0-category is a "subset" of H^*
\begin{remark}
    \label{rem:induced_massey_product_on_H-0}
    Given a DG-category \( \Cc \) and a diagram in \( H^0(\Cc) \) as follows
    \begin{center}
        \begin{tikzpicture}
            \diagram{m}{1cm}{1cm} {
                X_1 & X_2 & X_3 & X_4 \\
            };

            \draw[math]
                (m-1-1) edge node {f_1} (m-1-2)
                (m-1-2) edge node {f_2} (m-1-3)
                (m-1-3) edge node {f_3} (m-1-4);
        \end{tikzpicture}
    \end{center}
    Then this can be considered a diagram in \( H^*(\Cc) \) consisting of homogeneous of degree \( 0 \) morphisms.

    Then by \autoref{rem:massey_product_in_dg_cat/massey_product_definition/massey_product_sum_of_degrees} one has that this ``induced'' massey product of \( \massey{f_3, f_2, f_1} \) is a subset of \( H^{-1} \tuple{\Cc \tuple{X_1, X_4}} \).
\end{remark}

\begin{remark}
    Building on the idea in \autoref{rem:induced_massey_product_on_H-0}, let \( \Tc \) be an algebraic triangulated category, and let the following be a diagram in \( \Tc \)
    \begin{center}
        \begin{tikzpicture}
            \diagram{m}{1cm}{1cm} {
                X_1 & X_2 & X_3 & X_4 \\
            };

            \draw[math]
                (m-1-1) edge node {f_1} (m-1-2)
                (m-1-2) edge node {f_2} (m-1-3)
                (m-1-3) edge node {f_3} (m-1-4);
        \end{tikzpicture}
    \end{center}

    TODO: Write out more details in order to connect massey and toda in algebraic triangulated categories.
\end{remark}

\begin{theorem}
    Let \( \Tc \) be an algebraic triangulated category. Furthermore let the following be a diagram in \( \Tc \)
    \begin{center}
        \begin{tikzpicture}
            \diagram{m}{1cm}{1cm} {
                X_1 & X_2 & X_3 & X_4 \\
            };

            \draw[math]
                (m-1-1) edge node {f_1} (m-1-2)
                (m-1-2) edge node {f_2} (m-1-3)
                (m-1-3) edge node {f_3} (m-1-4);
        \end{tikzpicture}
    \end{center}
    % MS-question: Massey product ser stygt ut.
    Then \( \toda{f_3, f_2, f_1} = (-1)^{TODO} \massey{f_3, f_2, f_1} \).
\end{theorem}