% First need to extend massey-prod definition to H^0

% MS-Question: I Jasso--Muro så er dette berre definert for element av dgMod (Ingen subscript!)
% TODO: This is also the same shift functor that makes H^0(dgmodblabla) triangulated. Probably not neccesary to specify then.
\begin{proposition}
    \label{prop:H^i_dgmod_cong_H^0_with_shift}
    Let \( \Cc \) be a DG-category over \( R \). Let \( \Sigma \) be the shift functor on \( H^0(\dgMod_{\dg}(\Cc)) \). Let \( A, B \in \dgMod_{\dg}(\Cc) \).

    Then there is an isomorphism
    \[
        \phi: H^i(\dgMod_{\dg}(\Cc)(A, B)) \stackrel{\sim}{\to} H^0(\dgMod_{\dg}(\Cc))(A, \Sigma^i(B)).
    \]
\end{proposition}
\begin{proof}
    TODO
\end{proof}

% TODO: Show this is well defined!!
\begin{remark}
    \label{rem:H^0_into_H^*_inclusion}
    Let \( \Cc \) be a DG-category.

    There is a dense and faithful functor \( \iota: H^0(\Cc) \hookrightarrow H^*(\Cc) \) given by
    \begin{align*}
        \iota: H^0(\Cc) &\to H^*(\Cc) \\
        A &\mapsto A \\
        \iota_{A, B}: H^0(\Cc)(A, B) &\to H^*(\Cc)(A, B) \\
        f &\mapsto \tuple{\dots, 0, f, 0, \dots} \quad \text{\( f \) is in degree \( 0 \)}
    \end{align*}

    TODO: Show well defined.
\end{remark}

\begin{definition}[Massey product on \( H^0(\dgMod_{\dg}(\Cc)) \)]
    \label{def:massey_product_H^0(dgMod_dg(C))}
    Let \( \Cc \) be a DG-category. Let the following be a diagram in \( H^0(\dgMod_{\dg}(\Cc)) \)
    \begin{center}
        \begin{tikzpicture}
            \diagram{m}{1cm}{1cm} {
                X_1 & X_2 & X_3 & X_4 \\
            };

            \draw[math]
                (m-1-1) edge node {f_1} (m-1-2)
                (m-1-2) edge node {f_2} (m-1-3)
                (m-1-3) edge node {f_3} (m-1-4);
        \end{tikzpicture}
    \end{center}
    Using the functor \( \iota \) in \autoref{rem:H^0_into_H^*_inclusion} one can view the above diagram as a diagram in \( H^*(\dgMod_{\dg}(\Cc)) \) as follows
    \begin{center}
        \begin{tikzpicture}
            \diagram{m}{1cm}{1cm} {
                X_1 & X_2 & X_3 & X_4 \\
            };

            \draw[math]
                (m-1-1) edge node {\iota(f_1)} (m-1-2)
                (m-1-2) edge node {\iota(f_2)} (m-1-3)
                (m-1-3) edge node {\iota(f_3)} (m-1-4);
        \end{tikzpicture}
    \end{center}
    where all the maps are of degree \( 0 \).

    % MS-Question: Overly complicated and imprecise on domains and stuff.
    By \autoref{rem:massey_product_in_dg_cat/massey_product_definition/massey_product_sum_of_degrees} the massey product of these maps \( \massey{\iota(f_3), \iota(f_2), \iota(f_1)} \) only have non-zero components in \( H^{-1}(\dgMod_{\dg}(\Cc)(X_1, X_4)) \). Then, using the ismorphism \( \phi \) in \autoref{prop:H^i_dgmod_cong_H^0_with_shift} one has that \( \phi(\massey{\iota(f_3), \iota(f_2), \iota(f_1)}) \subseteq H^0(\dgMod_{\dg}(\Cc)(X_1, \Sigma^{-1}X_4)) \). This in turn means that \( \phi(\massey{\iota(f_3), \iota(f_2), \iota(f_1)}) \subseteq \im(\iota_{X_1, \Sigma^{-1}(X_4)}) \). But since \( \iota \) is a faithful and dense funcor, it follows that there is a unique subset \( M \subseteq H^0(\dgMod_{\dg}(\Cc))(X_1, \Sigma^{-1}(X_4)) \) such that \( \iota(M) = \phi(\massey{\iota(f_3), \iota(f_2), \iota(f_1)}) \).

    Then define this \( M \) as the \emph{massey product on \( H^0(\dgMod_{\dg}(\Cc)) \)}.
\end{definition}

% TODO: Specify that the functor is exact?
\begin{definition}[Massey product in an algebraic triangulated category]
    Let \( \Tc \) be an algebraic triangulated category, and let the following be a diagram in \( \Tc \)
    \begin{center}
        \begin{tikzpicture}
            \diagram{m}{1cm}{1cm} {
                X_1 & X_2 & X_3 & X_4 \\
            };

            \draw[math]
                (m-1-1) edge node {f_1} (m-1-2)
                (m-1-2) edge node {f_2} (m-1-3)
                (m-1-3) edge node {f_3} (m-1-4);
        \end{tikzpicture}
    \end{center}

    Since \( \Tc \) is algebraic, it is equivalent to \( H^0(\Cc) \) for some pre-triangulated DG-category \( \Cc \). Furthermore, since \( \Cc \) is a pre-triangulated DG-category, one has that \( H^0(\Cc) \) is equivalent to \( \im(H^0(\mathbf{h})) \). And since \( \im(H^0(\mathbf{h})) \) is a full subcategory of \( H^0(\dgMod_{\dg}(\Cc)) \), the diagram above can be looked at as a diagram in \( H^0(\dgMod_{\dg}(\Cc)) \).

    To recap, one has the relation:
    \[
        \Tc \cong H^0(\Cc) \cong \im(H^0(\mathbf{h})) \stackrel{full}{\subseteq} \D^c(\Cc) \stackrel{full}{\subseteq} \D(\Cc) \stackrel{full}{\subseteq} H^0(\dgMod_{\dg}(\Cc))
    \]

    Let \( F \) denote the functor that takes \( F: \Tc \hookrightarrow H^0(\dgMod_{\dg}(\Cc)) \) by the maps above. Then one has that the above diagram in \( \Tc \) can be viewed as a diagram in \( H^0(\dgMod_{\dg}(\Cc)) \) as follows
    \begin{center}
        \begin{tikzpicture}
            \diagram{m}{1cm}{1cm} {
                F(X_1) & F(X_2) & F(X_3) & F(X_4). \\
            };

            \draw[math]
                (m-1-1) edge node {F(f_1)} (m-1-2)
                (m-1-2) edge node {F(f_2)} (m-1-3)
                (m-1-3) edge node {F(f_3)} (m-1-4);
        \end{tikzpicture}
    \end{center}
    
    On the above diagram one can take the massey product (as in \autoref{def:massey_product_H^0(dgMod_dg(C))}). This yields a subset \( \massey{F(f_3), F(f_2), F(f_1)} \subseteq H^0(\dgMod_{\dg}(\Cc))(F(X_1), \Sigma^{-1}(F(X_4))) \), and since \( \im(H^0(\mathbf{h})) \) is a full subcategory of \( H^0(\dgMod_{\dg}(\Cc)) \), one has that \( \massey{F(f_3), F(f_2), F(f_1)} \subseteq \im(H^0(\mathbf{h}))(F(X_1), \Sigma^{-1}(F(X_4))) \) and therefore isomorphic as a set to a subset of \( \Tc(X_1, \Sigma^{-1}(X_4)) \cong \Tc(\Sigma(X_1), X_4) \). This subset is denoted as \( \massey{f_3, f_2, f_1} \) and is called the \emph{massey product of \( \Tc \)}.
\end{definition}

\begin{lemma}
    Let \( \Cc \) be a DG-category over \( R \). Let \( f \in \dgMod_{\dg}(\Cc)(A, B) \) be homogeneous of degree \( d \). Let \( d_B \) denote the differential on \( B \) and let \( \Sigma \) denote the shift functor in \( \dgMod_{\dg}(\Cc) \).

    Then the map
    \begin{align*}
        \tilde{f}: A &\to \Sigma^d(B) \\
    \end{align*}

    TODO
\end{lemma}

\begin{theorem}
    Let \( \Cc \) be a DG-category. Let the following be a diagram in \( H^0(\dgMod_{\dg}(\Cc)) \).
    \begin{center}
        \begin{tikzpicture}
            \diagram{m}{1cm}{1cm} {
                X_1 & X_2 & X_3 & X_4 \\
            };

            \draw[math]
                (m-1-1) edge node {f_1} (m-1-2)
                (m-1-2) edge node {f_2} (m-1-3)
                (m-1-3) edge node {f_3} (m-1-4);
        \end{tikzpicture}
    \end{center}
    Then \( \toda{f_3, f_2, f_1} = (-1)^{TODO} \massey{f_3, f_2, f_1} \).
\end{theorem}
\begin{proof}
    TODO: This is a sketch of the main idea, fix it up.

    Want to prove this by showing the two inclusions \( \supseteq \) and \( \subseteq \).

    Firstly, want to show \( \supseteq \):

    Let \( f \in \massey{f_3, f_2, f_1} \) and let \( \bar{(-)} \) be as in \autoref{def:massey_product_in_dg_cat/massey_product_definition/massey_product_dg_cat}.
    
    Then by definition there exist representatives \( g_i \in \dgMod_{\dg}(X_i, X_{i + 1}) \) and some \( s, t \) such that
    \[
        f = \class{\bar{s} \circ g_1 - \bar{g}_3 \circ t}.
    \]
\end{proof}

\begin{corollary}
    Let \( \Tc \) be an algebraic triangulated category. Furthermore let the following be a diagram in \( \Tc \).
    \begin{center}
        \begin{tikzpicture}
            \diagram{m}{1cm}{1cm} {
                X_1 & X_2 & X_3 & X_4 \\
            };

            \draw[math]
                (m-1-1) edge node {f_1} (m-1-2)
                (m-1-2) edge node {f_2} (m-1-3)
                (m-1-3) edge node {f_3} (m-1-4);
        \end{tikzpicture}
    \end{center}
    % MS-question: Massey product ser stygt ut.
    Then \( \toda{f_3, f_2, f_1} = (-1)^{TODO} \massey{f_3, f_2, f_1} \).
\end{corollary}
