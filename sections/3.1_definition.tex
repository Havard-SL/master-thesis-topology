Let \( \Tc \) be a triangulated category. Given the following diagram in \( \Tc \),
\begin{center}
    \begin{tikzpicture}
        \diagram{m}{1cm}{1cm} {
            X_1 \& X_2 \& X_3 \& X_4, \\
        };

        \draw[math]
            (m-1-1) edge node {f_1} (m-1-2)
            (m-1-2) edge node {f_2} (m-1-3)
            (m-1-3) edge node {f_3} (m-1-4);
    \end{tikzpicture}
\end{center}
we can define the \emph{three-fold Toda bracket of \( f_1, f_2, \) and \( f_3 \)} in three different (but actually identical, see \autoref{prop:toda-bracket-definitions-coincide}) ways:

% MS-Question: Problem med at venstre rotasjon ikkje blir skriven på den måten?
\begin{definition}[Toda bracket]
    \label{def:toda_bracket}
    \phantom{hei}

    \begin{enumerate}
        \item {
            Let
            \begin{center}
                \begin{tikzpicture}
                    \diagram{m}{1cm}{1cm} {
                        X_1 \& X_2 \& Y \& \Sigma X_1 \\
                    };

                    \draw[math]
                        (m-1-1) edge node {f_1} (m-1-2)
                        (m-1-2) edge (m-1-3)
                        (m-1-3) edge (m-1-4);
                \end{tikzpicture}
            \end{center}
            be any distinguished triangle.

            The set of every possible \( \psi \in \Tc(\Sigma X_0, X_3) \) that makes the following diagram commute,
            \begin{center}
                \begin{tikzpicture}
                    \diagram{m}{1cm}{1cm} {
                        X_1 \& X_2 \& Y \& \Sigma X_0 \\
                        X_1 \& X_2 \& X_3 \& X_4, \\
                    };

                    \draw[math]
                        (m-1-1) edge node {f_1} (m-1-2)
                            edge[equality] (m-2-1)
                        (m-1-2) edge (m-1-3)
                            edge[equality] (m-2-2)
                        (m-1-3) edge (m-1-4)
                            edge (m-2-3)
                        (m-1-4) edge node {\psi} (m-2-4)

                        (m-2-1) edge node {f_1} (m-2-2)
                        (m-2-2) edge node {f_2} (m-2-3)
                        (m-2-3) edge node {f_3} (m-2-4);
                \end{tikzpicture}
            \end{center}
            is denoted as \( \toda{f_3, f_2, f_1}_{\cc} \). This is called the \emph{three-fold iterated cofiber Toda Bracket of \( f_1, f_2, \) and \( f_3 \)}.
        }
        \item {
            Let
            \begin{center}
                \begin{tikzpicture}
                    \diagram{m}{1cm}{1cm} {
                        \Sigma^{-1} Y \& X_1 \& X_2 \& \Sigma \Sigma^{-1} Y \\
                    };

                    \draw[math]
                        (m-1-1) edge (m-1-2)
                        (m-1-2) edge node {f_2} (m-1-3)
                        (m-1-3) edge (m-1-4);
                \end{tikzpicture}
            \end{center}
            be any distinguished triangle.

            The set of every composite \( \beta \circ (\Sigma \alpha) \in \Tc(\Sigma X_0, X_3) \) that makes the following diagram commute,
            \begin{center}
                \begin{tikzpicture}
                    \diagram{m}{1cm}{1cm} {
                        {X_0} \& {X_1} \\
                        {\Sigma^{-1} Y} \& {X_1} \& {X_2} \& \Sigma \Sigma^{-1} Y \\
                        \&\& {X_2} \& {X_3}, \\
                    };

                    \draw[math]
                        (m-1-1) edge node {f_1} (m-1-2)
                            edge node {\alpha} (m-2-1)
                        (m-1-2) edge[equality] (m-2-2)

                        (m-2-1) edge (m-2-2)
                        (m-2-2) edge node {f_2} (m-2-3)
                        (m-2-3) edge (m-2-4)
                            edge[equality] (m-3-3)
                        (m-2-4) edge node {\beta} (m-3-4)

                        (m-3-3) edge node {f_3} (m-3-4);
                \end{tikzpicture}
            \end{center}
            is denoted as \( \toda{f_3, f_2, f_1}_{\fc} \). This is called the \emph{three-fold fiber-cofiber Toda Bracket of \( f_1, f_2, \) and \( f_3 \)}.
        }
        \item {
            Let \( \phi = \set{\phi_A}_{A \in \Tc} \) denote the natural transformation from \( \Sigma \Sigma^{-1} \) to \( \Id_{\Tc} \), and let
            \begin{center}
                \begin{tikzpicture}
                    \diagram{m}{1cm}{1cm} {
                        \Sigma^{-1} X_4 \& Y \& X_3 \& \Sigma \Sigma^{-1} X_4 \\
                    };

                    \draw[math]
                        (m-1-1) edge (m-1-2)
                        (m-1-2) edge (m-1-3)
                        (m-1-3) edge node {\phi_{X_3}^{-1} \circ f_3} (m-1-4);
                \end{tikzpicture}
            \end{center}
            be any distinguished triangle.
            
            The set of every morphism \( \phi_{X_3} \circ (\Sigma \delta) \in \Tc(\Sigma X_0, X_3) \), where \( \delta \) makes the following diagram commute
            \begin{center}
                \begin{tikzpicture}
                    \diagram{m}{1cm}{1cm} {
                        X_1 \& X_2 \& X_3 \& X_4 \\
                        \Sigma^{-1} X_4 \& Y \& X_3 \& \Sigma \Sigma^{-1} X_4 \\
                    };

                    \draw[math]
                        (m-1-1) edge node {f_1} (m-1-2)
                            edge node {\delta} (m-2-1)
                        (m-1-2) edge node {f_2} (m-1-3)
                            edge (m-2-2)
                        (m-1-3) edge node {f_3} (m-1-4)
                            edge[equality] (m-2-3)
                        (m-1-4) edge node {\phi_{X_3}^{-1}} (m-2-4)

                        (m-2-1) edge (m-2-2)
                        (m-2-2) edge (m-2-3)
                        (m-2-3) edge node {\phi_{X_3}^{-1} \circ f_3} (m-2-4);
                \end{tikzpicture}
            \end{center}
            is denoted as \( \toda{f_3, f_2, f_1}_{\ff} \). This is called the \emph{three-fold iterated fiber Toda Bracket of \( f_1, f_2, \) and \( f_3 \)}.
        }
    \end{enumerate}
\end{definition}

These are in fact equivalent ways to describe the same subset of \( \Tc(\Sigma X_0, X_3) \) which is clear by the following proposition.

\begin{proposition}
    \label{prop:toda-bracket-definitions-coincide}
    All three of the Toda brackets definitions in \autoref{def:toda_bracket} are equal.
\end{proposition}

For a proof of \autoref{prop:toda-bracket-definitions-coincide}, see \cite[Proposition 3.3]{Christensen-Frankland_2017}. Note that they specify the natural transformations between \( \Sigma \Sigma^{-1} \) and \( \Id_{\Tc} \), however, they use them implicit when it fits as mentioned in \cite[Section 2, p. 2690]{Christensen-Frankland_2017}.

As a consequence of \autoref{prop:toda-bracket-definitions-coincide} we can uniquely define the \emph{Toda bracket of \( f_1, f_2, \) and \( f_3 \)} as either \( \toda{f_3, f_2, f_1}_{\cc} \), \( \toda{f_3, f_2, f_1}_{\fc} \), or \( \toda{f_3, f_2, f_1}_{\ff} \), and it is denoted simply as \emph{\( \toda{f_3, f_2, f_1} \)}.

A helpful property of Toda brackets is that they are well-defined in the sense that different choices of distinguished triangles yields the same subset. We will be using a (rotated) standard triangle in our calculations.

% TODO: Bevis? Skjønne ikkje korleis Daria sitt bevis er fullstendig.
% TODO: Trur det må vere non-empty.
Another property helpful in calculations, is the ``indeterminacy'' of a Toda bracket. It is the result of the following lemma, which we will omit the proof of.
\begin{lemma}
    \label{lem:indeterminacy}
    Let the following be a diagram in a triangulated category \( \Tc \),
    \begin{center}
        \begin{tikzpicture}
            \diagram{m}{1cm}{1cm} {
                X_1 \& X_2 \& X_3 \& X_4. \\
            };
    
            \draw[math]
                (m-1-1) edge node {f_1} (m-1-2)
                (m-1-2) edge node {f_2} (m-1-3)
                (m-1-3) edge node {f_3} (m-1-4);
        \end{tikzpicture}
    \end{center}

    If \( \toda{f_3, f_2, f_1} \) is non-empty, then the Toda bracket \( \toda{f_3, f_2, f_1} \) is a coset of the subgroup,
    \[
        f_3 \circ \Tc(\Sigma X_1, X_3)  + \Tc(\Sigma X_2, X_4) \circ (\Sigma f_1) \subseteq \Tc(\Sigma X_1, X_4),
    \]
    which is called the \emph{indeterminacy of \( \toda{f_3, f_2, f_1} \)}.
\end{lemma}
