The first example of a triangulated category is the Spanier-Whitehead category. It is one of the triangulated categories that shows up from the topological side.

First, some notation.
\begin{notation}
    For \( X, Y \) two topological spaces.

    Let \( \class{ X, Y } \) denote the homotopy class of continous functions from \( X \) to \( Y \).
\end{notation}

Now one can define the \emph{underlying} Spanier-Whitehead category. The triangulated structure will be descirbed afterwards.

The Spanier-Whitehead category is motivated by the Freudental suspension theorem and historically lead to the definition of the stable homotopy category.

\begin{definition}[Spanier-Whitehead category]
    \label{def:sw-cat}
    Let \( SW \) be the category with the following properties:
    \begin{enumerate}
        \item {
            Objects in \( SW \) are touples of pointed CW-complexes and an integer, \( X, n \).
        }
        \item {
            Let \( \Sigma \) here denote the pointed suspension of a topological space.

            % MS-Question: Er dette riktig? Må r >= max(|n_X| + 2, |n_Y|)?
            Morphisms in \( SW \) are the following colimits of abelian groups
            \[
                \Hom\tuple{ (X, n_X), (Y, n_Y) } := \colim_{q \to \infty} \class{ \Sigma^{n_X + q}(X), \Sigma^{n_Y + q}(Y) }
            \]
        }
        \item {
            % TODO lim-colim nat. iso? How to explain this?
            The definition of composition is omitted for brevity in this thesis.
        }
    \end{enumerate}
\end{definition}

It can be shown that the Spanier-Whietehead category is an additive category.

To see the triangulated structure of the Spanier-Whitehead category it is necessary to define the shift functor and the class of distinguished triangles.

\begin{definition}[Shift functor in \( SW \)]
    \label{def:sw-shift}
    Let \( \Sigma \) be the following assignment of objects and morphisms in \( SW \).

    Let
    \[
        \Sigma(X, n) := (X, n + 1),
    \] 
    and for \( f: (X, n_X) \to (Y, n_Y) \), let
    \[
        \Sigma(f) := f.
    \]

    This can be shown to be a functor, and in fact is an automorphism.
\end{definition}

As the notation would imply, in \( SW \), one has that \( \Sigma(X, n) = (X, n + 1) \cong ( \Sigma X, n ) \).

The following definition of the distinguished triangles to make \( SW \) triangulated omits a lot of details as it is not necessary later in this thesis. For those who are interested, more details can be found in TODO.

\begin{definition}[Distinguished triangles in \( SW \)]
    \label{def:sw-dist_triangles}
    Let \( \Delta \) be the collection of triangles in \( SW \) satisfying the following property.

    A triangle
    \[
        (X, n_X) \to (Y, n_Y) \to (Z, n_Z) \to (X, n_X + 1)
    \]
    is in \( \Delta \) if and only if there is some even number \( k \), such that the following triangle in the homotopy category of pointed CW-complexes
    \[
        \Sigma^{n_X + k} X \to \Sigma^{n_Y + k} Y \to \Sigma^{n_Z + k} Z \to \Sigma^{n_X + k + 1} X
    \]
    is isomorphic as a triangle in the homotopy category of pointed CW-complexes to a triangle of the form
    \[
        A \stackrel{f}{\to} B \to C(f) \to \Sigma A
    \]
    where \( C(f) \) is the mapping cone of \( f \).
\end{definition}

Finally one can define \( SW \) as a triangulated category.

\begin{example}
    Let \( SW \) be as in \autoref{def:sw-cat}, let \( \Sigma: SW \to SW \) be as in \autoref{def:sw-shift} and let \( \Delta \) be as in \autoref{def:sw-dist_triangles}.

    Then \( \tuple{SW, \Sigma, \Delta} \) is a triangulated category.
\end{example}

For a proof, see TODO.

From this example of a triangulated category one can see where a lot of the notation in the definition of a triangulated category comes from. There is no coincidence that it is common to use the same symbol for both pointed suspension and for the shift functor ( \( \Sigma \) ), and the same symbol for the mapping cone as well as the cone in a triangulated category ( \( C(f) \) ).
