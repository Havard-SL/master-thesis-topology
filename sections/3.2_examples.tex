To get a better understanding of how to compute Toda brackets, as well as illustrate some properties they have, this section contains four simple examples in the category \( \Mc := \StMod(R) \) where \( R := \Fb_2 C_2 \), which is a known Frobenius ring. These examples will serve as a warm-up for more complicated examples.

The following remark contains some preliminary information that is helpful in the following computations:

\begin{remark}
	\label{rem:toda_bracket_examples_properties}
    Let \( g \) denote the generator of \( C_2 \).

    The ideal \( J := \tuple{1 + g} \) is the only non-trivial ideal in \( R \) (up to isomorphism).

    The ideal \( J \) is not projective, since the short exact sequence
    \begin{center}
        \begin{tikzpicture}
            \diagram{m}{1cm}{1cm}{
                J \& R \& J, \\
            };
        
            \draw[math]
                (m-1-1) edge[tailed] node {\kappa_J} (m-1-2)
                (m-1-2) edge[two headed] node {\rho_J} (m-1-3);
        \end{tikzpicture}
    \end{center}
    where \( \kappa_J \) is the inclusion and \( \rho_J \) is the morphism
    \begin{align*}
        \rho_J: R &\to J \\
        0, 1 + g &\mapsto 0 \\
        1, g &\mapsto 1 + g,
    \end{align*}
    is not split.

    This is because \( \kappa_J \) is the only monomorphism of \( J \) into \( R \), but composes to \( 0 \) with \( \rho_J \). Therefore, \( J \) cannot be projective, since every epimorphism into a projective module splits.

    The suspension \( \Sigma J \) is assumed to be the cokernel of \( \kappa_J \), which by the short exact sequence above, can be assumed to be \( J \).

	% Furthermore, since \( \rho_J \) is an epimorphism with kernel \( J \), we get from the third isomorphism theorem that \( \frac{R}{J} \cong J \). Let the isomorphism from \( \frac{R}{J} \) to \( J \) also be denoted as \( \rho_J \).
\end{remark}

The following example gives the idea of how to calculate Toda brackets in the simplest cases.

\begin{example}
	Let the following be a diagram in \( \Mc \)
	\begin{center}
		\begin{tikzpicture}
			\diagram{m}{1cm}{1cm}{
					J \& J \& J \& J. \\
			};

			\draw[math]
				(m-1-1) edge node {[\Id_J]} (m-1-2)
				(m-1-2) edge node {[0]} (m-1-3)
				(m-1-3) edge node {[\Id_J]} (m-1-4);
		\end{tikzpicture}
	\end{center}
	
	The goal is to calculate the Toda bracket \( \toda{\Id_J, 0, \Id_J} \).

	Using the iterated cofiber definition of Toda brackets, we need to choose a distinguished triangle. We will use a standard triangle of \( \Id_J \), which is the trivial triangle,
	\begin{center}
		\begin{tikzpicture}
			\diagram{m}{1cm}{1cm} {
				J \& J \& 0 \& J. \\
			};

			\draw[math]
				(m-1-1) edge node {[\Id_J]} (m-1-2)
				(m-1-2) edge node {[0]} (m-1-3)
				(m-1-3) edge node {[0]} (m-1-4);
		\end{tikzpicture}
	\end{center}

	By the definition of \( \toda{f_3, f_2, f_1}_{cc} \), we get the following diagram
	\begin{center}
		\begin{tikzpicture}
			\diagram{m}{1cm}{1cm} {
				J \& J \& 0 \& J \\
				J \& J \& J \& J. \\
			};

			\draw[math]
				(m-1-1) edge node {[\Id_J]} (m-1-2)
					edge[equality] (m-2-1)
				(m-1-2) edge node {[0]} (m-1-3)
					edge[equality] (m-2-2)
				(m-1-3) edge node {[0]} (m-1-4)
					edge node {[\rho]} (m-2-3)
				(m-1-4) edge node {[\psi]} (m-2-4)

				(m-2-1) edge node {[\Id_J]} (m-2-2)
				(m-2-2) edge node {[0]} (m-2-3)
				(m-2-3) edge node {[\Id_J]} (m-2-4);
		\end{tikzpicture}
	\end{center}

	Since \( [\rho] = 0 \), any endomorphism on \( J \) makes the rightmost square commute. In addition, since \( \Mod(R)(J, J) \) is simply \( \set{0, \Id_J} \), this implies that \( \Mc(J, J) = \set{[0], [\Id_J]} \).

	Therefore, \( \toda{[\Id_J], [0], [\Id_J]} = \set{[0], [\Id_J]} \).

	Another way we could have proven this, is by calculating the indeterminacy (\autoref{lem:indeterminacy}), which would be
	\[
		\Id_J \circ \Mc(J, J) + \Mc(J, J) \circ \Id_J = \Mc(J, J),
	\]
	which only has one coset, namely \( \Mc(J, J) \).
\end{example}

The previous example demonstrates that the Toda bracket is dependent on the choice of \( I \) in \autoref{def:stmod_sigma}. In the previous example, we could have assumed \( \Sigma J \) to be \( R/J \), which would have yielded the Toda bracket \( \toda{[\Id_J], [0], [\Id_J]} = \set{[0], [\mu]} \), where \( \mu \) is the unique isomorphism \( R/J \cong J \). Therefore, for applications of Toda brackets on stable module categories, it is important to keep in mind that the choice of \( I \) in the definition of \( \Sigma \) is consistent.

\begin{example}
	\label{ex:toda_bracket-1}
	We want to calculate \( \toda{\Id_J, \Id_J, \Id_J} \).

	We have the following diagram by the iterated cofiber definition of Toda brackets using the assumptions made in \autoref{rem:toda_bracket_examples_properties}
	\begin{center}
		\begin{tikzpicture}
			\diagram{m}{1cm}{1cm} {
				J \& J \& 0 \& J \\
				J \& J \& J \& J. \\
			};

			\draw[math]
				(m-1-1) edge node {[\Id_J]} (m-1-2)
					edge[equality] (m-2-1)
				(m-1-2) edge node {[0]} (m-1-3)
					edge[equality] (m-2-2)
				(m-1-3) edge node {[0]} (m-1-4)
					edge[squiggly] node {[\rho]} (m-2-3)
				(m-1-4) edge node {[\psi]} (m-2-4)

				(m-2-1) edge node {[\Id_J]} (m-2-2)
				(m-2-2) edge node {[\Id_J]} (m-2-3)
				(m-2-3) edge node {[\Id_J]} (m-2-4);
		\end{tikzpicture}
	\end{center}

	Here there is an issue. There exists no \( [\rho]: 0 \to J \) such that \( [\rho] \circ [0] = [0]: J \to J \) is equal to \( [\Id_J] \). It could only be true if \( J \cong 0 \), which it is not, since \( J \) is not projective by the argument in \autoref{rem:toda_bracket_examples_properties}.

	Therefore, the Toda bracket is \emph{empty}, i.e., \( \toda{\Id_J, \Id_J, \Id_J} = \emptyset \).
\end{example}

The previous example shows another property of Toda brackets, namely that it can be empty. The following proposition and proof shows this property for a general case.

\begin{proposition}
	Let \( f_1, f_2, \) and \( f_3 \) be three composable morphisms in any triangulated category, \( \Tc \),
	\begin{center}
		\begin{tikzpicture}
			\diagram{m}{1cm}{1cm} {
				X_1 \& X_2 \& X_3 \& X_4, \\
			};

			\draw[math]
				(m-1-1) edge node {f_1} (m-1-2)
				(m-1-2) edge node {f_2} (m-1-3)
				(m-1-3) edge node {f_3} (m-1-4);
		\end{tikzpicture}
	\end{center}
	such that \( f_2 \circ f_1 \neq 0 \) or \( f_3 \circ f_2 \neq 0 \).

	Then \( \toda{f_3, f_2, f_1} = \emptyset \).
\end{proposition}
\begin{proof}
	Assume that \( \toda{f_3, f_2, f_1} \neq \emptyset \). Then from the iterated cofiber definition of Toda brackets, there exists morphisms \( \alpha, \beta, \phi, \) and \( \psi \) such that the following diagram commutes, and the top row is distinguished
	\begin{center}
		\begin{tikzpicture}
			\diagram{m}{1cm}{1cm} {
				X_1 \& X_2 \& C_{f_1} \& \Sigma X_1 \\
				X_1 \& X_2 \& X_3 \& X_4. \\
			};

			\draw[math]
				(m-1-1) edge node {f_1} (m-1-2)
					edge[equality]	(m-2-1)
				(m-1-2) edge node {\alpha} (m-1-3)
					edge[equality] (m-2-2)
				(m-1-3) edge node {\beta} (m-1-4)
					edge node {\phi} (m-2-3)
				(m-1-4) edge node {\psi} (m-2-4)

				(m-2-1) edge node {f_1} (m-2-2)
				(m-2-2) edge node {f_2} (m-2-3)
				(m-2-3) edge node {f_3} (m-2-4);
		\end{tikzpicture}
	\end{center}

	We split the proof into two different contradictions:

	\begin{itemize}
		\item{
			Case 1:

			Assume that \( f_2 \circ f_1 \neq 0 \). Since \( \alpha, f_1 \) are two composable morphisms from the same distinguished triangle, we have
			\[
				\phi \circ \alpha \circ f_1 = \phi \circ 0 = 0.
			\]

			But from commutativity of the diagram we also have
			\[
				\phi \circ \alpha \circ f_1 = f_2 \circ f_1 \neq 0,
			\]
			which is a contradiction.
		}
		\item{
			Case 2:

			Assume that \( f_3 \circ f_2 \neq 0 \). Then we have that
			\[
				0 = \psi \circ 0 = \psi \circ \beta \circ \alpha = f_3 \circ \phi \circ \alpha = f_3 \circ f_2 \neq 0,
			\]
			which is also a contradiction.
		}
	\end{itemize}

	Therefore, both \( f_2 \circ f_1 = 0 \) and \( f_3 \circ f_2 = 0 \) if \( \toda{f_3, f_2, f_1} \neq \emptyset \), which is contrapositive to the statement in the theorem.
\end{proof}

The following example is a generalization of \autoref{ex:toda_bracket-1} to any triangulated category \( \Tc \).

\begin{example}
	We want to compute \( \toda{f, 0, \Id} \) for any triangulated category \( \Tc \) and for any \( f \in \Tc(X_2, X_3) \).

	We use the iterated cofiber definition of Toda brackets.

	Using the trivial triangle as the distinguished triangle, get the following diagram
	\begin{center}
		\begin{tikzpicture}
			\diagram{m}{1cm}{1cm} {
				X_1 \& X_1 \& 0 \& \Sigma X_1 \\
				X_1 \& X_1 \& X_2 \& X_3. \\
			};

			\draw[math]
				(m-1-1) edge node {\Id} (m-1-2)
					edge[equality] (m-2-1)
				(m-1-2) edge (m-1-3)
					edge[equality] (m-2-2)
				(m-1-3) edge (m-1-4)
					edge (m-2-3)
				(m-1-4) edge node {\psi} (m-2-4)

				(m-2-1) edge node {\Id} (m-2-2)
				(m-2-2) edge node {0} (m-2-3)
				(m-2-3) edge node {f} (m-2-4);
		\end{tikzpicture}
	\end{center}

	Here we have that any possible \( \psi: \Sigma X_1 \to X_3 \) will make the right square commute. Therefore, \( \toda{f_3, 0, \Id} = \Tc(\Sigma X_1, X_3) \).

	Calculating the indeterminacy,
	\[
		f \circ \Tc(\Sigma X_1, X_2) + \Tc(\Sigma X_1, X_4) \circ \Id_{X_1} = \Tc(\Sigma X_1, X_4),
	\]
	yields the same result.
\end{example}

Finally, the last example will be in \( \Mc \), but where the computations are a bit more complicated.

\begin{example}
	We want to compute the Toda bracket of the following morphisms using the iterated cofiber definition first, and then using the indeterminacy afterwards,
	\begin{center}
		\begin{tikzpicture}
			\diagram{m}{1cm}{1cm} {
				J \& {J \oplus J} \& {J \oplus J} \& J. \\
			};

			\draw[math]
				(m-1-1) edge node {\class{\begin{psmallmatrix} 1 \\ 1 \end{psmallmatrix}}} (m-1-2)
				(m-1-2) edge node {\class{\begin{psmallmatrix} 1 & 1 \\ 1 & 1 \end{psmallmatrix}}} (m-1-3)
				(m-1-3) edge node {\class{\begin{psmallmatrix} 1 & 1 \end{psmallmatrix}}} (m-1-4);
		\end{tikzpicture}
	\end{center}

	First, we find a standard triangle of \( \class{\begin{psmallmatrix} 1 \\ 1 \end{psmallmatrix}}: J \to J \oplus J \).

	The cone is defined as the pushout of the following diagram,
	\begin{center}
		\begin{tikzpicture}
			\diagram{m}{1cm}{1cm} {
				J \& J \oplus J \\
				R, \\
			};

			\draw[math]
				(m-1-1) edge node {\begin{psmallmatrix} 1 \\ 1 \end{psmallmatrix}} (m-1-2)
					edge[tailed] node[swap] {\kappa_J} (m-2-1);
		\end{tikzpicture}
	\end{center}
	where \( \kappa_J \) is assumed to be as in \autoref{rem:toda_bracket_examples_properties}.

	In the category of \( \Mod(R) \), the pushout becomes
	\begin{center}
		\begin{tikzpicture}
			\diagram{m}{1cm}{1cm} {
				J \& J \oplus J \\
				R \& (J \oplus J \oplus R)/\sim, \\
			};

			\draw[math]
				(m-1-1) edge node {\begin{psmallmatrix} 1 \\ 1 \end{psmallmatrix}} (m-1-2)
					edge[tailed] node[swap] {\kappa_J} (m-2-1)
				(m-1-2) edge[tailed] node {\rho} (m-2-2)

				(m-2-1) edge node {\gamma} (m-2-2);
		\end{tikzpicture}
	\end{center}
	where \( (1 + g, 1 + g, 0) \sim (0, 0, 1 + g) \).

	The morphism \( \rho \) is given as the composition
	\begin{center}
		\begin{tikzpicture}
			\diagram{m}{1cm}{1cm} {
				J \oplus J \& J \oplus J \oplus R \& (J \oplus J \oplus R)/\sim \\
			};

			\draw[math]
				(m-1-1) edge[tailed] node {i} (m-1-2)
				(m-1-2) edge[two headed] node {\pi} (m-1-3);
		\end{tikzpicture}
	\end{center}
	where \( i \) is the embedding, and \( \pi \) is the quotient epimorphism.

	We can check that the cone is isomorphic to \( J \oplus R \) via the morphism
	\begin{align*}
		\alpha: (J \oplus J \oplus R)/\sim &\to J \oplus R \\
		(0, 0, r) &\mapsto (0, r) \\
		(1 + g, 0, r) &\mapsto (1 + g, r + 1 + g) \\
		(0, 1 + g, r) &\mapsto (1 + g, r) \\
		(1 + g, 1 + g, r) &\mapsto (0, r + 1 + g). \\
	\end{align*}

	Therefore, by checking the morphism \( \alpha \circ \rho \) we can see that it becomes
	\[
		\begin{pmatrix}
			1 & 1 \\
			\kappa_J & 0 \\
		\end{pmatrix}
		: J \oplus J \to J \oplus R.
	\]

	By a similar argument for \( \gamma \), we get that it is simply the embedding into \( J \oplus R \), here denoted as \( \begin{psmallmatrix} 0 \\ 1 \\ \end{psmallmatrix} \).

	Thus, we can rewrite the pushout to the form
	\begin{center}
		\begin{tikzpicture}
			\diagram{m}{1cm}{1cm} {
				J \& J \oplus J \\
				R \& J \oplus R \\
			};

			\draw[math]
				(m-1-1) edge node {\begin{psmallmatrix} 1 \\ 1 \end{psmallmatrix}} (m-1-2)
					edge[tailed] node[swap] {\kappa_J} (m-2-1)
				(m-1-2) edge[tailed] node {\begin{psmallmatrix} 1 & 1 \\ \kappa_J & 0 \\ \end{psmallmatrix}} (m-2-2)

				(m-2-1) edge node {\begin{psmallmatrix} 0 \\ 1 \\ \end{psmallmatrix}} (m-2-2);
		\end{tikzpicture}
	\end{center}
	
	Furthermore, the morphism \( J \oplus R \to \Sigma J = J \) is given as the unique pushout morphism \( \beta \), satisfying the following commutative diagram
	\begin{center}
		\begin{tikzpicture}
			\diagram{m}{1cm}{1cm} {
				J \& J \oplus J \\
				R \& J \oplus R \\
				\& J. \\
			};

			\draw[math]
				(m-1-1) edge node {\begin{psmallmatrix} 1 \\ 1 \end{psmallmatrix}} (m-1-2)
					edge[tailed] node[swap] {\kappa_J} (m-2-1)
				(m-1-2) edge node[swap] {\begin{psmallmatrix} 1 & 1 \\ \kappa_J & 0 \\ \end{psmallmatrix}} (m-2-2)
					edge[curve={height=-25pt}] node {0} (m-3-2) 

				(m-2-1) edge[tailed] node {\begin{psmallmatrix} 0 \\ 1 \\ \end{psmallmatrix}} (m-2-2)
					edge node {\rho_J} (m-3-2)
				(m-2-2) edge node {\beta} (m-3-2);
		\end{tikzpicture}
	\end{center}

	We can check that a candidate for the morphism \( \beta \) is \( \begin{psmallmatrix} 0 & \rho_J \\ \end{psmallmatrix} \). By uniqueness of the pushout property, this morphism is \( \beta \).

	Therefore, the standard triangle becomes
	\begin{center}
		\begin{tikzpicture}
			\diagram{m}{1cm}{2cm} {
				J \& J \oplus J \& J \oplus R \& J. \\
			};

			\draw[math]
				(m-1-1) edge node {\class{\begin{psmallmatrix} 1 \\ 1 \end{psmallmatrix}}} (m-1-2)
				(m-1-2) edge node {\class{\begin{psmallmatrix} 1 & 1 \\ \kappa_J & 0 \\ \end{psmallmatrix}}} (m-1-3)
				(m-1-3) edge node {\class{\begin{psmallmatrix} 0 & \rho_J \\ \end{psmallmatrix}}} (m-1-4);
		\end{tikzpicture}
	\end{center}

	In \( \Mc \), \( R \cong 0 \), and the morphism \( \class{\begin{psmallmatrix} 1 & 0 \\ \end{psmallmatrix}}: J \oplus R \to J \) becomes an isomorphism which yields the following distinguished triangle
	\begin{center}
		\begin{tikzpicture}
			\diagram{m}{1cm}{2cm} {
				J \& J \oplus J \& J \& J. \\
			};

			\draw[math]
				(m-1-1) edge node {\class{\begin{psmallmatrix} 1 \\ 1 \end{psmallmatrix}}} (m-1-2)
				(m-1-2) edge node {\class{\begin{psmallmatrix} 1 & 1 \\ \end{psmallmatrix}}} (m-1-3)
				(m-1-3) edge node {\class{0}} (m-1-4);
		\end{tikzpicture}
	\end{center}

	Using the iterated cofiber definition of Toda brackets, we get the following commutative diagram
	\begin{center}
		\begin{tikzpicture}
			\diagram{m}{1cm}{1cm} {
				J \& {J \oplus J} \& J \& J \\
				J \& {J \oplus J} \& {J \oplus J} \& J, \\
			};

			\draw[math]
				(m-1-1) edge node {\class{\begin{psmallmatrix} 1 \\ 1 \end{psmallmatrix}}} (m-1-2)
					edge[equality] (m-2-1)
				(m-1-2) edge node {\class{\begin{psmallmatrix} 1 & 1 \\ \end{psmallmatrix}}} (m-1-3)
					edge[equality] (m-2-2)
				(m-1-3) edge node {[0]} (m-1-4)
					edge node {\phi} (m-2-3)
				(m-1-4) edge node {\psi} (m-2-4)

				(m-2-1) edge node {\class{\begin{psmallmatrix} 1 \\ 1 \end{psmallmatrix}}} (m-2-2)
				(m-2-2) edge node {\class{\begin{psmallmatrix} 1 & 1 \\ 1 & 1 \end{psmallmatrix}}} (m-2-3)
				(m-2-3) edge node {\class{\begin{psmallmatrix} 1 & 1 \end{psmallmatrix}}} (m-2-4);
		\end{tikzpicture}
	\end{center}
	where the top row is distinguished.

	Now, we have to find every possible \( \psi \) such that the above diagram commutes for some \( \phi \).

	Start by assuming
	\[ 
		\phi = \class{
			\begin{psmallmatrix}
				1 \\
				1
			\end{psmallmatrix}
		},
	\]
	this makes the square to the left of \( \phi \) commute, in addition, it makes the square to the right of \( \phi \) also commute, because
	\[
		\class{\begin{psmallmatrix} 1 & 1 \end{psmallmatrix}} \circ
		\begin{psmallmatrix}
			1 \\
			1
		\end{psmallmatrix}
		=
		[2] = [0].
	\]

	Thus, with the assumed \( \phi \), any \( \psi \) will make the diagram commute. Therefore,
	\[ 
		\toda{\class{\begin{psmallmatrix} 1 & 1 \end{psmallmatrix}}, \class{\begin{psmallmatrix} 1 & 1 \\ 1 & 1 \end{psmallmatrix}}, \class{\begin{psmallmatrix} 1 \\ 1 \end{psmallmatrix}}} = \Mc(J, J).
	\]
	
	Then we want to try the same calculations using indeterminacy:
	
	Consider the subgroup
	\[
		\class{\begin{psmallmatrix} 1 & 1 \end{psmallmatrix}} \circ \Mc(J, J \oplus J) + \Mc(J \oplus J, J) \circ \tuple{\Sigma \class{\begin{psmallmatrix} 1 \\ 1 \end{psmallmatrix}}}.
	\]
	For any \( [f] \in \Mc(J, J) \),
	\[
		\class{\begin{psmallmatrix} 1 & 1 \end{psmallmatrix}} \circ \class{\begin{psmallmatrix} f \\ 0 \end{psmallmatrix}} = [f],
	\]
	we get that
	\[
		\class{\begin{psmallmatrix} 1 & 1 \end{psmallmatrix}} \circ \Mc(J, J \oplus J) = \Mc(J, J),
	\]
	and so
	\[
		\toda{\class{\begin{psmallmatrix} 1 & 1 \end{psmallmatrix}}, \class{\begin{psmallmatrix} 1 & 1 \\ 1 & 1 \end{psmallmatrix}}, \class{\begin{psmallmatrix} 1 \\ 1 \end{psmallmatrix}}} = \Mc(J, J).
	\]
\end{example}

If, only using the Toda bracket definition, at the end of the final example we weren't so lucky and couldn't have chosen a \( \phi \), such that every \( \psi \) was expressed, the calculations would have become much more tedious. Then we would have to find every possible \( \phi \), and from every possible \( \phi \) we would then find every possible \( \psi \). In addition, working with morphisms in \( \Mc \) is not simple, and we could end up with a very complicated and annoying calculation. Combining this with the fact that simply calculating an appropriate distinguished triangle was tedious, and the calculations become even worse. Calculating the Toda bracket using the indeterminacy, at least in the above cases turned out to be much simpler.

This illustrates the reason for wanting to combine Massey products and Toda brackets as mentioned in the introduction, as additional tools, such as indeterminacy, might significantly simplify the calculations in certain cases.
