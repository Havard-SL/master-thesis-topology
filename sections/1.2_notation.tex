This thesis will assume that the reader is already familiar with basic homological algebra as well as some module theory.

Some assumptions will only be mentioned in the definitions not be explicitly stated in the theorems that use said definitions.

The following is the chain complex convention.
\begin{notation}
    \label{not:chain_complex}
    Every chain complex will have \emph{ascending} order. This is sometimes called a \emph{cochain complex} in other literature.

    Differentials are indexed according to the order of their domain, i.e., for a chain complex \( A = \tuple*{A_i}_{i \in \Zb} \), the \( n \)th differential has the same index as its domain, \( d_n^A: A_n \to A_{n + 1} \).
    
    We try to keep to the same convention of indexing with respect to the domain when considering morphisms between graded objects.
\end{notation}

We will denote the module categories as follows.
\begin{notation}
    \( \Mod(R) \) will refer to the category of \( R \)-modules, and \( \mod(R) \) will refer to the category of \emph{finitely generated} \( R \)-modules.
\end{notation}

The following arrow conventions will be used.
\begin{notation}
    Tailed arrows \( (\rightarrowtail) \) represent monomorphisms, and double-headed arrows \( (\twoheadrightarrow) \) represent epimorphisms.
\end{notation}

Certain canonical isomorphisms will be omitted in the notation.
\begin{notation}
    \label{not:suppress_canonical_isomorphisms}
    We will mention when there is a canonical isomorphism, but will suppress them in the notation for brevity.

    For example, given a category \( \Cc \), \( \Cc(A, B) \) is \emph{canonically isomorphic} to \( \Cc^{\op}(B, A) \), since by definition \( \Cc^{\op}(B, A) = \Cc(A, B) \).

    However, in calculations we will write, for example, \( f \in \Cc(A, B) \iff f \in \Cc^{\op}(B, A) \).
\end{notation}

We distinguish between ``morphisms'' and ``maps.''
\begin{notation}
    A ``morphism'' is a morphism in some category, while a ``map'' is a function on the underlying sets.
\end{notation}

We will use both superscript and subscript for indexing.
\begin{notation}
    If a morphism has multiple subscripts, then we will denote the first subscript, as long as it is an object or (enriched) category, as a superscript instead.

    For example, for some chain complex \( A \), we will denote the \( n \)th differential as \( d_n^A \), instead of \( d_{A, n} \).

    We will also skip some subscripts if it is clear from the context, e.g., just write \( d_n \) for the \( n \)th differential of \( A \) instead of \( d_n^A \).
\end{notation}