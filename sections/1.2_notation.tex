This thesis will assume that the reader is already familiar with basic homological algebra as well as some module theory.

Some assumptions in results will be implied by definitions and not be explicitly stated in theorems. However, there are some conventions that are used throughout this thesis.

First, the following is the chain complex convention.
\begin{notation}
    \label{not:chain_complex}
    Every chain complex will have \emph{ascending} order. This is sometimes called a \emph{cochain complex} in other literature.

    Differentials are indexed according to the order of their domain, i.e., for a chain complex \( A = \tuple*{A_i}_{i \in \Zb} \), the \( n \)th differential has the same index as its domain, \( d_n^A: A_n \to A_{n + 1} \).
    
    We try to keep to the same convention of indexing with respect to the domain when considering morphisms between graded objects.
\end{notation}

Second, the module notation is as follows.
\begin{notation}
    \( \Mod(R) \) will refer to \( R \)-modules, and \( \mod(R) \) will refer to \emph{finitely generated} \( R \)-modules.
\end{notation}

Third, the following arrow conventions will be used.
\begin{notation}
    Tailed arrows \( (\rightarrowtail) \) represent monomorphisms, and double-headed arrows \( (\twoheadrightarrow) \) represent epimorphisms.
\end{notation}

Fourth, certain canonical isomorphisms will be omitted in the notation.
\begin{notation}
    \label{not:suppress_canonical_isomorphisms}
    We will mention when there is a canonical isomorphism, but will suppress them in calculations for clarity.

    For example, given a category \( \Cc \), \( \Cc(A, B) \) is \emph{canonically isomorphic} to \( \Cc^{\op}(B, A) \), since by definition \( \Cc^{\op}(B, A) = \Cc(A, B) \).

    However, in calculations we will write, for example, \( f \in \Cc(A, B) \iff f \in \Cc^{\op}(B, A) \).
\end{notation}

Fifth, we differ between ``morphisms'' and ``maps.''
\begin{notation}
    A ``morphism'' is a morphism in some category, while a ``map'' is a function on the underlying sets.
\end{notation}