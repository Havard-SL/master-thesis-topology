In this thesis, the focus is solely on the definition of Massey products on a triangulated category with the final goal of connecting Massey products and toda brackets in the next section.

Before starting this section, some notation needs to be established first. First, this section is going to contain a lot of chain complexes, the notation of which is explained in \autoref{not:chain_complex}.

Second, this section is going to contain a lot of products and coproducts, which would benefit a lot from simplified notation, and the reasoning behind the notation.

\begin{notation}
    Let \( R \) be a commutative ring with identity. Let \( A_i \in \Mod(R) \) and let
    \[
        \iota_i: A_i \to \coprod_{i \in \Zb} A_i
    \]
    denote the canonical split monomorphism by the universal property of the coproduct in \( \Mod(R) \).

    Then for any \( a_i \in A_i \), the element
    \[
        \iota_i(a_i) \in \coprod_{i \in \Zb} A_i
    \]
    is just denoted as
    \[
        a_i \in \coprod_{i \in \Zb} A_i.
    \]
    
    The reasoning for the above notation is twofold. Firstly, it reduces notation while not being ambigous. Secondly, one never talks about a general element of a coproduct. Almost always when talking about what a morphism does to an element of the coproduct, it is what happens to the \( \iota_i(a_i) \)'s, which is a consequence of the universal property of the coproduct.

    In addition, let 
    \[
        \pi_i: \prod_{i \in \Zb} A_i \to A_i
    \]
    be the universal split epimorphism by the universal property of the product in \( \Mod(R) \).
    
    Then for any element \( a \in \prod_{i \in \Zb} A_i \), denote
    \[
        a = \tuple{a_i}_{i \in \Zb} \in \prod_{i \in \Zb} A_i
    \]
    where \( a_i := \pi_i(a) \in A_i \).
    
    The reasoning behind this notation is because the product in \( \Mod(R) \) is the direct product, and because the universal property of the product. That is because it makes it so that when talking about morphisms into the product, the morphism is fully defined by what it does to each degree in \( \prod_{i \in \Zb} A_i \), which is easily shown using the above notation.
\end{notation}

Before defining Massey products, one first has to define what a DG-category is.

\subsection{DG-categories}
In this thesis the definition of a DG-category is based on enriched category theory, as it is a modern approach, and by the opinion of the author it is also the most elegeant approach. The enriched category theory that this thesis is based on, is \cite[Section 6.2]{Borceux_1994}.

The first definition necessary to understanding DG-categories, is the tensor product of chain complexes over modules.

\begin{definition}[Tensor product of \( \C \)]
    \label{def:tensor_product_of_chain_complexes_over_Mod(R)}
    Let \( R \) be a commutative ring with identity. Furthermore, let \( A, B \in \C \).

    Then, for any  \( n \in \Zb \) define the modules
    \[
        (A \otimes B)_n := \coprod_{p + q = n} A_p \otimes B_q
    \]
    which are a part of the chain complex
    \begin{center}
        \begin{tikzpicture}
            \diagram{m}{1cm}{1cm} {
                A \otimes B: \\
            };
        \end{tikzpicture}
        %
        \begin{tikzpicture}
            \diagram{m}{1cm}{1cm} {
                \cdots \& \tuple*{A \otimes B}_{-1} \& \tuple*{A \otimes B}_0 \& \tuple*{A \otimes B}_1 \& \cdots \\
            };

            \draw[math]
                (m-1-1) edge (m-1-2)
                (m-1-2) edge node {d_{-1}} (m-1-3)
                (m-1-3) edge node {d_0} (m-1-4)
                (m-1-4) edge (m-1-5);
        \end{tikzpicture}
    \end{center}

    Where the differentials, \( d_n \), are defined as follows:
    
    Let \( i + j = n \) and \( a \otimes b \in A_i \otimes B_j \) be an elementary tensor.

    Then the differential is (uniquely) defined by the following assignments
    \[
        d_n(a \otimes b) := d_{A, i}(a) \otimes b + (-1)^{i} a \otimes d_{B, j}(b).
    \]

    This is called the \emph{tensor product of \( \C \)}.
\end{definition}

In order to see that the above definition of the differential on the chain complex is well defined, we need the following lemma, which will be useful later as well.

\begin{lemma}
    \label{lem:map_out_of_tensor_unique}
    Let \( A, B \in \C \) and let \( C \in \Mod(R) \). Furthermore, let \( i, j \in \Zb \) with \( i + j = n \).

    Then for any
    \[
        f: (A \otimes B)_n \to C
    \]
    where for any \( a \in A_i, b \in B_j \) we have
    \[
        f(a \otimes b) = g_{i, j}(a, b)
    \]
    for some \( R \)-bilinear morphisms
    \[
        g_{i, j}: A_i \times B_j \to C.
    \]

    Then \( f \) is uniquely defined by the \( g_{i, j} \)'s.
\end{lemma}
\begin{proof}
    Consider at the following diagram where \( \iota_{i, j} \) is the canocial split monomorphism by the universal property of the coproduct, and where the \( g_{i, j} \)'s are \( R \)-bilienar
    \begin{diagramlabel}[\label{tikz:differential_of_tensor_product_of_chain_complexes_over_Mod(R)}]
        \begin{tikzpicture}
            \diagram{m}{2cm}{2cm} {
                A_i \otimes B_j \& \coprod\limits_{p + q = n} A_p \otimes B_q \\
                A_i \times B_j \& C. \\
            };

            \draw[math]
                (m-1-1) edge[tailed] node {\iota_{i, j}} (m-1-2)
                    edge[dashed] node {\alpha_{i, j}} (m-2-2)
                (m-1-2) edge[dashed] node {\beta} (m-2-2)

                (m-2-1) edge node {g_{i, j}} (m-2-2)
                    edge node {\otimes} (m-1-1);
        \end{tikzpicture}
    \end{diagramlabel}

    Then by the universal property of tensor product in \( \Mod(R) \), \( g_{i, j} \) induces a unique morphism, \( \alpha_{i, j} \), which is induced from the elementary tensors as follows
    \[
        a \otimes b \mapsto g_{i, j}(a, b).
    \]

    Since this works for any \( i, j \) as long as \( i + j = n \), we can construct \( \alpha_{i, j} \) for every valid \( i, j \) pair.

    Then by using the universal property of the coproduct we get the unique map \( \beta \) which is by \autoref{tikz:differential_of_tensor_product_of_chain_complexes_over_Mod(R)} uniquely determined by it's actions on elementary tensors \( a \otimes b \in A_i \otimes B_j \) in the following way
    \[
        \beta: a \otimes b \mapsto g_{i, j}(a, b)
    \]
    which is exactly equal to \( f \), and \( f \) is therefore uniquely deterined by the \( g_{i, j} \)'s.
\end{proof}

The following remark shows that \autoref{def:tensor_product_of_chain_complexes_over_Mod(R)} is well-defined.

\begin{remark}
    The definition of the differentials in \autoref{def:tensor_product_of_chain_complexes_over_Mod(R)} is well-defined and unique by the following argument.

    We can check that for \( i + j = n \) that
    \begin{align*}
        g_{i, j}: A_i \times B_j &\to (A \otimes B)_{n + 1} \\
        (a, b) &\mapsto d_{A, i}(a) \otimes b + (-1)^i a \otimes d_{B, j}(b)
    \end{align*}
    is \( R \)-bilinear.

    Then by \autoref{lem:map_out_of_tensor_unique} it follows that \( d_n \) is uniquely defined.

    Similarly, by seeing that
    \[
        d_{n + 1} \circ d_n: a \otimes b \mapsto 0
    \]
    and the \( 0 \) map is \( R \)-bilinear, so \( d_{n + 1} \circ d_n \) is uniquely defined. Since the zero map would also send \( a \otimes b \) to \( 0 \), then by uniquenes \( d_{n + 1} \circ d_n = 0 \).
\end{remark}

An important property of the tensor product of chain complexes is symmetry. This is shown in the following remark.

\begin{remark}[Symmetry of tensor product in \( \C \)]
    \label{rem:symmetry_tensor_product_of_chain_complex}
    For two \( A, B \in \C \), there exist an isomorphism
    \[
        s = \tuple{s_n}_{n \in \Zb}: A \otimes B \to B \otimes A
    \]
    that for any \( n \) and \( i + j = n \) with \( a \in A_i \) and \( b \in B_j \) is defined on the \( n \)-th component as follows,
    \[
        s_n: a \otimes b \mapsto (-1)^{ij} b \otimes a.
    \]

    By \autoref{lem:map_out_of_tensor_unique} every \( s_n \) is well-defined. It is also clear that every \( s_n \) is an isomorphism by checking injectivity and surjectivity. It only remains to verify that \( s = \tuple{s_n}_{n \in \Zb} \) is a chain morphism.

    Consider the following difference
    \begin{align*}
        &s_{n + 1} \circ d_n(a \otimes b) - d_n \circ s_n (a \otimes b) = s_{n + 1} \tuple{d_{A, i}(a) \otimes b + (-1)^i a \otimes d_{B, j}(b)} - d_n \tuple{ (-1)^{ij} b \otimes a } \\
        &= (-1)^{(i + 1)j} b \otimes d_{A, i}(a) + (-1)^{i + i(j + 1)} d_{B, j}(b) \otimes a - (-1)^{ij} d_{B, j}(b) \otimes a - (-1)^{j + ij} b \otimes d_{A, i}(a) \\
        &= (-1)^{(i + 1)j} b \otimes d_{A, i}(a) - (-1)^{(i + 1)j} b \otimes d_{A, i}(a) + (-1)^{ij} d_{B, j}(b) \otimes a - (-1)^{ij} d_{B, j}(b) \otimes a \\
        &= 0,
    \end{align*}
    which implies that the morphism \( s_{n + 1} \circ d_n - d_n \circ s_n \) sends every elementary tensor to \( 0 \), which again implies that the morphism is equal to \( 0 \).
\end{remark}

Another definition closely linked to the tensor product in \( \C \) is the internal hom.

\begin{definition}[Internal hom of \( \C \)]
    \label{def:internal_hom_of_chain_complexes_over_Mod(R)}
    Let \( R \) be a commutative ring with identity. Furthermore, let \( A, B \in \C \).

    Then, for any \( n \in \Zb \) define the modules
    \[
        \class*{A, B}_n := \prod_{i \in \Zb} \Mod(R)(A_i, B_{i + n})
    \]
    which are a part of the chain complex
    \begin{center}
        \begin{tikzpicture}
            \diagram{m}{1cm}{1cm} {
                \class*{A, B}: \&[-0.5cm] \cdots \& \class*{A, B}_{-1} \& \class*{A, B}_0 \& \class*{A, B}_1 \& \cdots \\
            };

            \draw[math]
                (m-1-2) edge (m-1-3)
                (m-1-3) edge node {d_{-1}} (m-1-4)
                (m-1-4) edge node {d_0} (m-1-5)
                (m-1-5) edge (m-1-6);
        \end{tikzpicture}
    \end{center}

    where the differentials, \( d_n \), are defined as follows,
    \begin{align*}
        d_n : \class*{A, B}_n &\to \class*{A, B}_{n + 1} \\
        \tuple*{f_i}_{i \in \Zb} &\mapsto \tuple*{d_{B, i + n} \circ f_i - (-1)^n f_{i + 1} \circ d_{A, i}}_{i \in \Zb}.
    \end{align*}
    This is called the \emph{internal hom of chain complexes over \( \Mod(R) \)}.
\end{definition}

The following remark explains why the differentials in the previous definition is well-defined.

\begin{remark}
    The definition of the differential in \autoref{def:internal_hom_of_chain_complexes_over_Mod(R)} is well-defined because the only thing that needs to be checked is if the \( d_n \)'s are differentials, which can be checked straight forwardly.
\end{remark}

As the names would imply, the tensor product and internal hom of chain complexes over modules are adjoint in the usual sense. This is shown in the following remark.

\begin{remark}[Tensor product and internal hom adjunction in \( \C \)]
    \label{rem:tensor_prod_internal_hom_adjoint}
    Let \( A, B, C \) be chain complexes in \( \C \) for some commutative ring \( R \), and let
    \[
        f \in \C\tuple*{A \otimes B, C}.
    \]
    
    The goal of this remark is to understand what the adjoint of \( f \) is.

    Let \( f = \tuple*{f_n}_{n \in \Zb} \) where \( f_n \in \Mod(R)\tuple*{ \tuple*{A \otimes B}_n, C_n} \) are the individual level-wise morphisms of the chain morphism \( f \).

    Then, unpacking the definitions, we have that \( f \) looks like the following diagram
    \begin{center}
        \begin{tikzpicture}
            \diagram{m}{1cm}{1cm} {
                \cdots \& \coprod\limits_{i + j = -1} A_i \otimes B_j \& \coprod\limits_{i + j = 0} A_i \otimes B_j \& \coprod\limits_{i + j = 1} A_i \otimes B_j \& \cdots \\
                \cdots \& C_{-1} \& C_0 \& C_1 \& \cdots \\
            };

            \draw[math]
                (m-1-1) edge (m-1-2)
                (m-1-2) edge node {d_{A \otimes B, -1}} (m-1-3)
                    edge node {f_{-1}} (m-2-2)
                (m-1-3) edge node {d_{A \otimes B, 0}} (m-1-4)
                    edge node {f_0} (m-2-3)
                (m-1-4) edge (m-1-5)
                    edge node {f_1} (m-2-4)

                (m-2-1) edge (m-2-2)
                (m-2-2) edge node {d_{C, -1}} (m-2-3)
                (m-2-3) edge node {d_{C, 0}} (m-2-4)
                (m-2-4) edge (m-2-5);
        \end{tikzpicture}
    \end{center}
    Likewise, the adjoint has to look like the following diagram
    \begin{center}
        \begin{tikzpicture}
            \diagram{m}{1cm}{0.70cm} {
                \cdots \&[-0.5cm] A_{-1} \& A_0 \& A_1 \&[-0.5cm] \cdots \\
                \cdots \& \prod\limits_{j \in \Zb} \Mod(R)(B_j, C_{j - 1}) \& \prod\limits_{j \in \Zb} \Mod(R)(B_j, C_j) \& \prod\limits_{j \in \Zb} \Mod(R)(B_j, C_{j + 1}) \& \cdots \\
            };

            \draw[math]
                (m-1-1) edge (m-1-2)
                (m-1-2) edge node {d_{A, -1}} (m-1-3)
                    edge node {?_{-1}} (m-2-2)
                (m-1-3) edge node {d_{A, 0}} (m-1-4)
                    edge node {?_0} (m-2-3)
                (m-1-4) edge (m-1-5)
                    edge node {?_1} (m-2-4)

                (m-2-1) edge (m-2-2)
                (m-2-2) edge node {d_{\class*{B, C}, -1}} (m-2-3)
                (m-2-3) edge node {d_{\class*{B, C}, 0}} (m-2-4)
                (m-2-4) edge (m-2-5);
        \end{tikzpicture}
    \end{center}

    % For any \( n \in \Zb \), let \( i', j' \in \Zb \) with \( i' + j' = n \) let
    % \[
    %     \iota_{i', j'}: A_{i'} \otimes B_{j'} \rightarrowtail \coprod_{i + j = n} A_i \otimes B_j
    % \]
    % be the canonical split monomorphism by the definintion of the coproduct \( \coprod_{i + j = n} A_i \otimes B_j \).
    Let \( \iota_{i, j} \) be as defined in \autoref{not:coprod_prod_forvirring}.

    Then take the hom-tensor adjoint in \( \Mod(R) \) of the morphism
    \[
        f_{i + j} \circ \iota_{i, j}: A_i \otimes B_j \to C_{i + j}.
    \]
    This yields a morphism
    \begin{align*}
        \phi_{f, i, j}: A_i &\to \Mod(R)(B_j, C_{j + i}) \\
        a &\mapsto f_{i + j}\tuple*{ a \otimes ? }.
    \end{align*}
    Then by the universal property of the product there is some morphism
    \[
        \phi_{f, i} := \prod_{j \in \Zb} \phi_{f, i, j}: A_i \to \prod_{j \in \Zb} \Mod(R)\tuple*{B_j, C_{j + i}}.
    \]
    Collecting these morphisms yields a morphism, which is a candidate for the adjoint of \( f \), namely
    \[
        \phi_f := \tuple*{\phi_{f, i}}_{i \in \Zb}.
    \]
    In order to show that this is the proper adjoint definition, we need to show the following properties:
    \begin{enumerate}
        \item {
            First, \( \phi_f \) is a chain morphism.
        }
        \item {
            Second, the assignment \( f \mapsto \phi_f \) is an isomorphism of groups.
        }
        \item {
            Third, there is a natural transformation
            \[
                \C(?_1 \otimes ?_2, ?_3) \cong \C(?_1, \left[ ?_2, ?_3 \right])
            \]
            where the natural morphisms are \( f \mapsto \phi_f \).
        }
    \end{enumerate}
    % TODO: Possibly expand this proof? or TODO: SRC (Marius kommenterte dette og)
    In this thesis only the first statement will be proven.

    1) Want to show that \( \phi_f \) is a chain morphism.

    Need to check that for any \( i \in \Zb \) that the following diagram commutes
    \begin{center}
        \begin{tikzpicture}
            \diagram{m}{1cm}{1cm} {
                A_i \& A_{i + 1} \\
                \class*{B, C}_i \& \class*{B, C}_{i + 1} \\
            };

            \draw[math]
                (m-1-1) edge node {d_{A, i}} (m-1-2)
                    edge node {\phi_{f, i}} (m-2-1)
                (m-1-2) edge node {\phi_{f, i + 1}} (m-2-2)

                (m-2-1) edge node {d_{\class*{B, C}, i}} (m-2-2);
        \end{tikzpicture}
    \end{center}
    Pick an arbitrary \( a \in A_i \) and consider the following equation
    \begin{align*}
        \phi_{f, i + 1} \circ d_{A, i}(a) &- d_{\class*{B, C}, i} \circ \phi_{f, i}(a)
        = \tuple*{ f_{i + j + 1}\tuple*{d_{A, i}(a) \otimes ?} }_{j \in \Zb}
        - d_{\class*{B, C}, i} \tuple*{ \tuple*{ f_{i + j}\tuple*{a \otimes ?} }_{j \in \Zb} } \\
        \intertext{by expanding out the definition of \( d_{\class*{B, C}, i} \) it follows that}
        &= \tuple*{ f_{i + j + 1}\tuple*{d_{A, i}(a) \otimes ?}
        - d_{C, i + j} \circ f_{i + j}\tuple*{a \otimes ?}
        + (-1)^i f_{i + j + 1}\tuple*{a \otimes d_{B, j}(?)} }_{j \in \Zb} \\
        \intertext{by consolodating the two terms that post-compose by \( f_{i + j + 1} \) it follows that}
        &= \tuple*{ f_{i + j + 1}\bigl( \tuple*{d_{A, i}(a) \otimes ?}
        + (-1)^i\tuple*{ a \otimes d_{B, j}(?) } \bigr)
        - d_{C, i + j} \circ f_{i + j}\tuple*{ a \otimes ? } }_{j \in \Zb} \\
        \intertext{by the definition of the differential of \( A \otimes B \) it follows that}
        &= \tuple*{ f_{i + j + 1} \circ d_{A \otimes B, i + j} \tuple*{ a \otimes ? }
        - d_{C, i + j} \circ f_{i + j} ( a \otimes ? ) }_{j \in \Zb} \\
        \intertext{by \( f \) being a chain homomorphism from \( A \otimes B \) to \( C \) it follows that}
        &= 0.
    \end{align*}
\end{remark}

Together the above definitions and remarks gives the structure of a symmetric monoidal category, which will not be shown in this thesis. For a definition of symmetric monoidal categories, see TODO: CITE: Appendix.

\begin{fact}[\( \C \) is symmetric monoidal]
    Let \( R \) be a commutative ring with identity, and let \( \otimes \) denote the tensor product on \( \C \). Furthermore let \( I \) be the chain complex in \( \C \) consisting solely of \( 0 \)-objects exept for the \( R \)-module \( R \) in index \( 0 \).

    Then \( \tuple*{\C, \otimes, I} \) is a symmetric closed monoidal category.
\end{fact}

Finally we can define what a DG-category is.

\begin{definition}[DG-category]
    \label{def:dg_cat}
    Let \( R \) be a commutative ring with identity.

    Then \( \Cc \) is a \emph{DG-category over \( R \)} if it is a category enriched over \( \C \).
\end{definition}

This definition also appears in \cite[p. 29]{Jasso-Muro_2023}, except they define it for a field and not a commutative ring with identity as is done in this thesis.


\subsection{Definitions}
In this subsection, the goal is to define what a Massey product is.

Start off with defining the notation that will be used for the cohomology functor.

\begin{notation}
    Let \( R \) be a commutative ring with identity.

    Then let
    \[
        H^\bullet: \C \to \C
    \]
    be the cohomology functor, with the differentials on the right hand side all being \( 0 \).
\end{notation}

Now there is an issue with usual category theory notation of a ``diagram'' for DG-categories. This is because a DG-category doesn't have the notion of a morphism since there is no element of a chain complex. Therefore there can't be a diagram since the arrows would correspond to an elements of a chain complex. However, having a notion of a diagram in a DG-category would make certain future results easier to both state and understand, at the cost of some difficulty in the proofs. And that is why the following notation convention will be used.

% TODO: Maybe move this notation above the definition of H-bullet of a DG category as it is used in the definition.
% TODO: Remove all mentions of iota_i, j's in every massety section. They are implied.
% TODO: Maybe a separate notation for composition of morphisms, since morphisms are not always a apart of a dg-diagram.
\begin{notation}
    It is prudent to properly define what a ``diagram'' in a DG-category is by the discussion above.

    Let \( \Cc \) be a DG-category. A \emph{diagram in a DG-category} (or a \emph{DG-diagram}) is a quiver \( \Gamma = \tuple*{V, E, s, t} \) where every vertex in \( V \) corresponds to an object in \( \Cc \), and every edge \( e \in E \) corresponds to an element in \( \Cc\tuple*{s(e), t(e)}_n \) for some \( n \in \Zb \). These edges are called \emph{DG-morphisms} and \( n \) is called the \emph{degree} of \( e \), and is also denoted as \( |e| \).

    For two morphisms \( f, g \in E \) with \( s(g) = t(f) \), they are denoted as \emph{composable}. And their composition, denoted \( g \circ f \), is the morphism
    \[
        g \circ f := c_{\Cc, |g| + |f|}(g \otimes f) \in \Cc\tuple*{s(f), t(g)}_{|g| + |f|},
    \]
    where
    \[
        c_{\Cc} := \set*{c_{\Cc, i}}_{i \in \Zb} \text{ with } c_{\Cc, i}: \tuple*{ \Cc\tuple*{s(g), t(g)} \otimes \Cc\tuple*{s(f), t(f)} }_i \to \Cc\tuple*{s(f), t(g)}_i
    \]
    is the composition chain morphism for \( \Cc \).
\end{notation}

Composition in a DG-diagram is essentially usual composition in the DG-category, but restricted to just one component of the coproduct.

In order to justify using this notation, there are som properties that are helpful to know. First, that composition as defined above is associative.

\begin{lemma}[Associativity of composition in a DG-diagram]
    \label{lem:dg-composition_associative}
    Let the following be a DG-diagram in a DG-category \( \Cc \)
    \begin{center}
        \begin{tikzpicture}
            \diagram{m}{1cm}{1cm} {
                A \& B \& C \& D. \\
            };

            \draw[math]
                (m-1-1) edge node {f} (m-1-2)
                (m-1-2) edge node {g} (m-1-3)
                (m-1-3) edge node {h} (m-1-4);
        \end{tikzpicture}
    \end{center}

    Then \( h \circ (g \circ f) = (h \circ g) \circ f \).
\end{lemma}
\begin{proof}
    Expanding the definitions, it is necessary to show that the following equation holds (the category is omitted for readability)
    \begin{equation}
        \label{eq:dg-composition_associative}
        c_{|h| + |g| + |f|}\tuple*{h \otimes \tuple*{c_{|g| + |f|}(g \otimes f)}} = c_{|h| + |g| + |f|}\tuple*{\tuple*{c_{|h| + |g|}(h \otimes g)} \otimes f}.
    \end{equation}
    % TODO: This needs a proof. In RM: "Associativity of dg composition" p. 2, there is a diagram of how the diagram could exist. Would need to show that the stated composition of (natural?) isomorphisms are in fact exactly the associativity map.
    Assume (without proof) that the following diagram in \( \Mod(R) \), where the top horizontal morphism is the usual associativity morphism of the tensor product of \( R \)-modules, and \( a \) is the associativity morphism for the tensor product of chain complexes (in degree \( |h| + |g| + |f| \) ),
    \begin{center}
        \begin{tikzpicture}
            \diagram{m}{1cm}{0.45cm} {
                \Cc(C, D)_{|h|} \otimes \tuple*{ \Cc(B, C)_{|g|} \otimes \Cc(A, B)_{|f|} } \& \tuple*{ \Cc(C, D)_{|h|} \otimes \Cc(B, C)_{|g|} } \otimes \Cc(A, B)_{|f|} \\
                \Cc(C, D)_{|h|} \otimes \tuple*{ \Cc(B, C) \otimes \Cc(A, B) }_{|g| + |f|} \& \tuple*{ \Cc(C, D) \otimes \Cc(B, C) }_{|h| + |g|} \otimes \Cc(A, B)_{|f|} \\
                \tuple*{ \Cc(C, D) \otimes ( \Cc(B, C) \otimes \Cc(A, B) ) }_{|h| + |g| + |f|} \& \tuple*{ ( \Cc(C, D) \otimes \Cc(B, C) ) \otimes \Cc(A, B) }_{|h| + |g| + |f|}, \\
            };

            \draw[math]
                (m-1-1) edge node {\sim} (m-1-2)
                    edge node {\Id \otimes \iota_{|g|, |f|}} (m-2-1)
                (m-1-2) edge node {\iota_{|h|, |g|} \otimes \Id} (m-2-2)

                (m-2-1) edge node {\iota_{|h|, |g| + |f|}} (m-3-1)
                (m-2-2) edge node {\iota_{|h| + |g|, |f|}} (m-3-2)

                (m-3-1) edge node {\sim} node[swap] {a} (m-3-2);
        \end{tikzpicture}
    \end{center}
    commutes.

    And by \cite[Definition 6.2.1]{Borceux_1994} the following diagram of chain morphisms commute
    \begin{center}
        \begin{tikzpicture}
            \diagram{m}{1cm}{1cm} {
                \Cc(C, D) \otimes \tuple*{ \Cc(B, C) \otimes \Cc(A, B) } \& \& \tuple*{ \Cc(C, D) \otimes \Cc(B, C) } \otimes \Cc(A, B) \\
                \Cc(C, D) \otimes \Cc(A, C) \& \& \Cc(B, D) \otimes \Cc(A, B) \\
                \& \Cc(A, D). \\
            };
            
            \draw[math]
                (m-1-1) edge node {\sim} node[swap] {a} (m-1-3)
                    edge node {\Id \otimes c} (m-2-1)
                (m-1-3) edge node {c \otimes \Id} (m-2-3)

                (m-2-1) edge node {c} (m-3-2)
                (m-2-3) edge node[swap] {c} (m-3-2);
        \end{tikzpicture}
    \end{center}
    Gluing together the top diagram with the bottom diagram restricted to degree \( |h| + |g| + |f| \), and looking at where the element \( h \otimes (g \otimes f) \) is sent, yields exactly \autoref{eq:dg-composition_associative}.
\end{proof}

In addition to being associative, it would also be nice to prove that composition is \( R \)-bilinear, which is proven in the following lemma.

\begin{lemma}
    Composition in a DG-diagram is a morphism in \( \Mod(R) \), in particular, it is \( R \)-linear in both components.
\end{lemma}
\begin{proof}
    By definition, composition in a DG-diagram is the composition of two \( \Mod(R) \) morphisms, \( \iota_{i, j} \) and \( c_{\Cc, i + j} \), and so it is a \( \Mod(R) \) morphism.
\end{proof}

Now the notation should be sufficient for working with DG-categories.

Before defining the Massey product, we first have to define the category the product is taken in. That is the following category, the cohomology category of any DG-category.

\begin{definition}[Cohomology category, \( H^\bullet(\Cc) \)]
    \label{def:H_bullet_dg_category}
    Let \( \Cc \) be a differentially graded category over  \( R \).

    Let \( H^\bullet(\Cc) \) be the following (enriched over \( C\tuple*{\Mod(R)} \)) category:
    \begin{enumerate}
        \item Let \( \Obj(H^\bullet(\Cc)) := \Obj(\Cc) \).
        \item For any \( A, B \in H^\bullet(\Cc) \), let \( H^\bullet(\Cc)(A, B) := H^\bullet \tuple*{\Cc(A, B)} \).
        \item {
            For any \( A, B, C \in H^\bullet(\Cc) \), define the composition morphism
            \begin{align*}
                c_{H^\bullet(\Cc)}: H^\bullet(\Cc)(B, C) \otimes H^\bullet(\Cc)(A, B) &\to H^\bullet(\Cc)(A, C)
            \end{align*}
            to be the chain morphism with the \( n \)-th component being
            \begin{align*}
                c_n: (H^\bullet(\Cc)(B, C) \otimes H^\bullet(\Cc)(A, B))_n &\to H^n(\Cc(A, C)) \\
                [g_i] \otimes [f_j] &\mapsto \class*{g_i \circ f_j}
            \end{align*}
            for any \( i, j \in \Zb \) with \( i + j = n \) and \( [g_i] \in H^i(\Cc)(B, C) \) and \( [f_j] \in H^j(\Cc)(A, B) \).
        }
        \item {
            Let \( u: I_A \to \Cc(A, A) \) be the unit morphism for \( \Cc \).
            
            For \( A \in H^\bullet(\Cc) \), the unit morphism of \( H^\bullet(\Cc) \) is defined as the chain morphism
            \[
                H^\bullet(u): I \to H^\bullet(\Cc)(A, A).
            \]
        }
    \end{enumerate}

    Then \( H^\bullet(\Cc) \) is called the \emph{cohomology category of \( \Cc \)}.
\end{definition}

In order to show that composition is well-defined the following remark can make the proof a bit simpler.

\begin{remark}
    \label{rem:H_bullet_composition_alpha}
    The composition definition in \autoref{def:H_bullet_dg_category} is actually the composition of two different morphisms.

    Consider the maps
    \begin{align*}
        \alpha_{i, j}: H^i(\Cc(B, C)) \times H^j(\Cc(A, B)) &\to H^n(\Cc(B, C) \otimes \Cc(A, B)) \\
        ([g_i], [f_j]) &\mapsto \class*{g_i \otimes f_j}.
    \end{align*}
    Assuming these are well-defined \( R \)-bilinear morphisms, denote the unique morphism they define by \autoref{lem:map_out_of_tensor_unique} by \( \alpha_n \), and the entire chain morphism by \( \alpha := \set*{\alpha_i}_{i \in \Zb} \). \footnote{
        The map \( \alpha \) is know at the cross product morphism, as explained in \cite[p. 273]{Hatcher_2002}. In addition for \( R \) a field (as is assumed in \cite{Jasso-Muro_2023}) it is known that by the Algebraic Künneth Theorem that \( \alpha \) is an isomorphism \cite[Theorem 3B.5]{Hatcher_2002}.
    }
    
    Then we can see that
    \[
        c_{H^\bullet(\Cc)} := H^\bullet(c_{\Cc}) \circ \alpha.
    \] 
\end{remark}

Using the above remark, since \( H^{\bullet}(c_{\Cc}) \) is already a well-defined morphism, it is only neccesary to show that \( \alpha \) exists, is well-defined and is unique.

\begin{remark}
    \label{rem:composition_in_H_bullet_is_well_defined}
    The composition definition in \autoref{def:H_bullet_dg_category} is well-defined and unique by the following argument.

    By \autoref{rem:H_bullet_composition_alpha} it is sufficient to only have to verify that the \( \alpha_{i, j} \)'s are well-defined and \( R \)-balanced in order to show that \( c_{H^\bullet(\Cc)} \) is well-defined and unique.

    We can check that the maps are \( R \)-balanced, but we still need to check if the maps are well-defined, which is a bit more difficult. There are two points that need to be shown:
    \begin{enumerate}
        \item {
            Firstly, is
            \[
                g_i \otimes f_j \in H^n(\Cc(B, C) \otimes \Cc(A, B))?
            \]
            To show this, we have to verify that
            \[
                g_i \otimes f_j \in \ker(d_{\Cc(B, C) \otimes \Cc(A, B), n}).
            \]
            This is true because by assumption, both \( g_i \) and \( f_j \) are cycles, and by definition of the differential of the tensor product
            \[
                d_{\Cc(B, C) \otimes \Cc(A, B), n}(g_i \otimes f_j) = 0.
            \]
        }
        \item {
            Secondly, are the values of the \( \alpha_{i, j} \)'s independent of the choice of representative?

            Let \( b_g \) be a boundary in \( \Cc(B, C)_i \), and let \( b_f \) be a boundary in \( \Cc(A, B)_j \).
            \begin{align*}
                \alpha_{i, j}([g_i + b_g], [f_j + b_f]) &= [(g_i + b_g) \otimes (f_j + b_f)] \\
                &= [g_i \otimes f_j + g_i \otimes b_f + b_g \otimes f_j + b_g \otimes b_f] \\
                &= [g_i \otimes f_j + g_i \otimes b_f + b_g \otimes f_j + b_g \otimes b_f]
            \end{align*}
            By the definition of the differential of the tensor product, we have that the tensor product betweeen a boundary and a cycle, a cycle and a boundary, and a boundary and a boundary are all boundaries in the tensor product.

            As an example, let's take the case above of \( g_i \otimes b_f \). Since \( b_f \) is a boundary in \( \Cc(A, B)_j \), there is some \( b_f' \in \Cc(A, B)_{j - 1} \) such that \( d_{\Cc(A, B), j - 1}(b_f') = b_f \). Consider the following equation
            \begin{align*}
                d_{\Cc(B, C) \otimes \Cc(A, B)}(&(-1)^i g_i \otimes b_f') \\
                &= (-1)^i d_{\Cc(B, C), i}(g_i) \otimes b_f' + (-1)^i (-1)^i g_i \otimes d_{\Cc(A, B), j - 1}(b_f') \\
                &= (-1)^i 0 \otimes b_f' + g_i \otimes b_f \\
                &= g_i \otimes b_f.
            \end{align*}
            A similar argument can be made for the other cases.

            Therefore is follows that
            \[
                \alpha_{i, j}([g_i + b_g], [f_j + b_f]) = [g_i \otimes f_j + g_i \otimes b_f + b_g \otimes f_j + b_g \otimes b_f] = [g_i \otimes f_j]
            \]
            And \( \alpha_{i, j} \) is therefore well-defined.
        }
    \end{enumerate}
    There is no need to verify if the composition is a chain morphism, since the differential of \( H^\bullet(\Cc)(A, C) \) is zero by definition.
\end{remark}

This thesis will refrain from proving that the above definition of \( H^\bullet(\Cc) \) and any future alleged DG-category actually satisfy the associativity axiom and the unit axiom as defined by \cite[Definition 6.2.1]{Borceux_1994}. This is because doing so would require defining the associativity morphism (properly), as well as the left and right tensor unit, only to do proofs that are straight forward to prove using the lemmas already defined. Defining the morphisms and doing the proofs would require a lot of boilerplate for something that, in the end, is not necessary for the goal of this thesis.

Therefore in this thesis, both the associativity axiom and the unit axiom is assumed to be true for every defined DG-category.

The following is the definition of the Massey product on the cohomology category of any DG-category. The definition is the 3-fold variant of \cite[Definition 4.2.1]{Jasso-Muro_2023}.

% SRC: Jasso-Muro
% TODO: What is the relation between different reprersentatives and different choices of s & t? Does it matter, and how does it matter?
% TODO: Write definition clearer? A lot of implied domains and codomains.
\begin{definition}
    \label{def:massey_product_dg_cat}
    Let \( \Cc \) be a differentially graded category over a commutative ring with identity \( R \).

    Let the following be a DG-diagram in \( H^\bullet(\Cc) \)
    \begin{center}
        \begin{tikzpicture}
            \diagram{m}{1cm}{1cm} {
                X_1 \& X_2 \& X_3 \& X_4 \\
            };

            \draw[math]
                (m-1-1) edge node {[f_1]} (m-1-2)
                (m-1-2) edge node {[f_2]} (m-1-3)
                (m-1-3) edge node {[f_3]} (m-1-4);
        \end{tikzpicture}
    \end{center}

    Furthermore for an element, \( h \in \C_{|h|} \) (like a morphism in a DG-diagram), let \( \bar{h} := (-1)^{|h| + 1}h \).

    Then let
    \begin{multline*}
        \massey{[f_3], [f_2], [f_1]} :=
        \{
            \class*{
                \bar{s} \circ g_1 + \bar{g_3} \circ t
            }
            \mid g_i \sim f_i, i = 1, 2, 3 \quad \\
            d_{\Cc, |f_3| + |f_2| - 1}(s) = \bar{g_3} \circ g_2, \,
            d_{\Cc, |f_2| + |f_1| - 1}(t) = \bar{g_2} \circ g_1
        \}.
    \end{multline*}

    This is a subset of
    \[
        H^{|f_1| + |f_2| + |f_3| - 1}\tuple*{\Cc\tuple*{X_1, X_4}}
    \]
    and is called the \emph{Massey product of \( [f_3], [f_2] \) and \( [f_1] \)}.
\end{definition}

The following remark shows that \autoref{def:massey_product_dg_cat} is well-defined.

\begin{remark}
    The definition of Massey product in \autoref{def:massey_product_dg_cat} is well-defined by the following argument:

    Want to show that \( \bar{s} \circ g_1 + \bar{g_3} \circ t \) is a cocyle in \( \Cc(X_1, X_4)_{|g_1| + |g_2| + |g_3| - 1} \).

    Let \( n := |g_1| + |g_2| + |g_3| \), and omit the degrees of the differentials for readability. Consider the following equation
    \begin{align*}
        d_{\Cc(X_1, X_4)}(\bar{s} \circ g_1 &+ \bar{g_3} \circ t) = d_{\Cc(X_1, X_4)}(\bar{s} \circ g_1) + d_{\Cc(X_1, X_4)}(\bar{g_3} \circ t) \\
        \intertext{by the definition of composition of morphisms in a DG-diagram, it follows that}
        &= d_{\Cc(X_1, X_4)}(c_{\Cc, n - 1}(\bar{s} \otimes g_1)) + d_{\Cc(X_1, X_4)}(c_{\Cc, n - 1}(\bar{g_3} \otimes t)) \\
        \intertext{by the fact that composition is a chain morphism, it follows that}
        &= c_{\Cc, n}(d_{\Cc(X_2, X_4) \otimes \Cc(X_1, X_2)}(\bar{s} \otimes g_1)) + c_{\Cc, n}(d_{\Cc(X_3, X_4) \otimes \Cc(X_1, X_3)}(\bar{g}_3 \otimes t))
        \intertext{by the definition of the differential of the tensor products, it follows that}
        &= c_{\Cc, n}((-1)^{|g_2| + |g_3|}(\bar{g}_3 \circ g_2) \otimes g_1) + c_{\Cc, n}((-1)^{|g_3|}\bar{g}_3 \otimes (\bar{g}_2 \circ g_1)) \\
        \intertext{by pulling out every sign, it follows that}
        &= (-1)^{|g_2| + 2|g_3| - 1}c_{\Cc, n}((g_3 \circ g_2) \otimes g_1) + (-1)^{|g_2| + 2|g_3| - 2}c_{\Cc, n}(g_3 \otimes (g_2 \circ g_1)) \\
        \intertext{by the definition of composition of morphisms in a DG-diagram, it follows that}
        &= (-1)^{|g_2| + 2|g_3| - 1}(g_3 \circ g_2) \circ g_1 + (-1)^{|g_2| + 2|g_3| - 2}g_3 \circ (g_2 \circ g_1) \\
        \intertext{by simplifying signs, it follows that}
        &= (-1)^{|g_2|}\tuple*{ g_3 \circ (g_2 \circ g_1) - (g_3 \circ g_2) \circ g_1 } \\
        \intertext{by \autoref{lem:dg-composition_associative} it follows that}
        &= 0.
    \end{align*}
    % Secondly, what to show that the massey product is independent of the choice of cocyle-representatives.

    % Let \( b_i \in \Cc(X_i, X_{i + 1}) \) for \( i = 1, 2 \) and \( 3 \) be boundaries.

    % Then look at
    % \[
    %     \set*{ \class*{ \overline{s} \circ (g_1 + b_1) + (\overline{g_3 + b_3}) \circ t } \mid d(s) = (\overline{g_3 + b_3}) \circ (g_2 + b_2), \quad d(t) = (\overline{g_2 + b_2}) \circ (g_1 + b_1) }.
    % \]
    
    % By linearity, the first part can be written as 
    % \[
    %     [\overline{s} \circ g_1 + \overline{s} \circ b_1 + g_3 \circ t + b_3 \circ t].
    % \]
    % But since \( b_1 \) is a boundary, by a similar argument as in \autoref{rem:composition_in_H_bullet_is_well_defined} item 2, we have that
    % \[
    %     \overline{s} \circ b_1 = c_{|g_1| + |g_2| + |g_3| - 1}( \iota_{|g_2| + |g_3| - 1, |g_1|} ( \overline{s} \otimes b_1 ) )
    % \]
    % is in fact
    % \[
    %     \overline{s} \circ b_1 = c_{|g_1| + |g_2| + |g_3| - 1}( d_{\Cc(X_2, X_4) \otimes \Cc(X_1, X_2), |g_1| + |g_2| + |g_3| - 2} (\tilde{g}) )
    % \]
    % for some \( \tilde{g} \in (\Cc(X_2, X_4) \otimes \Cc(X_1, X_2))_{|g_1| + |g_2| + |g_3| - 2} \).

    % And since \( c \) is a chain morphism, it follows that
    % \[
    %     \overline{s} \circ b_1 = d_{\Cc(X_1, X_4), |g_1| + |g_2| + |g_3| - 2}(c_{|g_1| + |g_2| + |g_3| - 2}(\tilde{g}))
    % \]
    % and so \( \overline{s} \circ b_1 \) is a boundary, and a similar argument can be made of \( b_3 \circ t \).

    % It remains to show that
    % \[
    %     \set*{ \class*{ \overline{s} \circ g_1 + \overline{g_3} \circ t } \mid d(s) = (\overline{g_3 + b_3}) \circ (g_2 + b_2), \quad d(t) = (\overline{g_2 + b_2}) \circ (g_1 + b_1) }.
    % \]
    % is independent of the \( b_i \)'s.
\end{remark}

Astute readers may notice that nowhere in the definition of a Massey product does it state or imply that the cohomology category of \( \Cc \), \( H^\bullet(\Cc) \), is triangulated. Which is necessary in order to compare Massey products to Toda brackets. And even if it were triangulated, the scope of categories that could be equivalent to a category consisiting of chain complexes with zero differential is very small, which would make applicability an issue. However, by moving over to the category of ``DG-modules'' which will be defined in the next section, we can define the Massey product on a big class of triangulated categories, namely the ``algebraic triangulated categories''.


\subsection{Examples}
TODO: MS-Question: Skal laga eksempel for generelle Massey produkt eller venta til Massey porodukt på alg. tri. kat?