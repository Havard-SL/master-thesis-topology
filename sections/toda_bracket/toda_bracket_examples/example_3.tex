Let \( \toda{f_3, 0, \Id} \) be a well defined Toda bracket. Then one has the following diagram:

\begin{center}
	\begin{tikzpicture}
		\diagram{m}{1cm}{1cm} {
			{X_1} \& {X_1} \& 0 \& {\Sigma(X_1)} \\
			{X_1} \& {X_1} \& {X_2} \& {X_3} \\
		};

		\draw[math]
			(m-1-1) edge node {\Id} (m-1-2)
				edge[equal] (m-2-1)
			(m-1-2) edge (m-1-3)
				edge[equal] (m-2-2)
			(m-1-3) edge (m-1-4)
				edge (m-2-3)
			(m-1-4) edge node {\phi} (m-2-4)

			(m-2-1) edge node {\Id} (m-2-2)
			(m-2-2) edge node {0} (m-2-3)
			(m-2-3) edge node {f_3} (m-2-4);
	\end{tikzpicture}
\end{center}

Here one has that any possible \( \psi: \Sigma(X_1) \to X_3 \) will make the right square commute. Therefore \( \toda{f_3, 0, \Id} = \Tc(\Sigma(X_1), X_3) \).
