\begin{example}
	Let the following be a diagram in \( \Stmod(R) \)
	\begin{center}
		\begin{tikzpicture}
			\diagram{m}{1cm}{1cm}{
					J \& J \& J \& J. \\
			};

			\draw[math]
				(m-1-1) edge node {\Id} (m-1-2)
				(m-1-2) edge node {0} (m-1-3)
				(m-1-3) edge node {\Id} (m-1-4);
		\end{tikzpicture}
	\end{center}

	Using the iterated cofiber definition of Toda bracket (\autoref{def:iterated-cofiber-toda-bracket}), one gets the following diagram
	\begin{center}
		\begin{tikzpicture}
			\diagram{m}{1cm}{1cm} {
				J \& J \& R \& {\frac{R}{J}} \\
				J \& J \& J \& J \\
			};

			\draw[math]
				(m-1-1) edge node {\Id} (m-1-2)
					edge[equal] (m-2-1)
				(m-1-2) edge (m-1-3)
					edge[equal] (m-2-2)
				(m-1-3) edge (m-1-4)
					edge node {\rho} (m-2-3)
				(m-1-4) edge node {\psi} (m-2-4)

				(m-2-1) edge node {\Id} (m-2-2)
				(m-2-2) edge node {0} (m-2-3)
				(m-2-3) edge node {\Id} (m-2-4);
		\end{tikzpicture}
	\end{center}

	However, since the diagram is in \( \StMod(R) \), one has that \( R \cong 0 \), and therefore \( \rho = 0 \). And from earlier one has that \( \frac{R}{J} \cong J \).

	This gives the following diagram in \( \StMod(R) \):

	\begin{center}
		\begin{tikzpicture}
			\diagram{m}{1cm}{1cm} {
				J \& J \& 0 \& J \\
				J \& J \& J \& J \\
			};

			\draw[math]
				(m-1-1) edge node {\Id} (m-1-2)
					edge[equal] (m-2-1)
				(m-1-2) edge node {0} (m-1-3)
					edge[equal] (m-2-2)
				(m-1-3) edge node {0} (m-1-4)
					edge node {0} (m-2-3)
				(m-1-4) edge node {\psi} (m-2-4)

				(m-2-1) edge node {\Id} (m-2-2)
				(m-2-2) edge node {0} (m-2-3)
				(m-2-3) edge node {\Id} (m-2-4);
		\end{tikzpicture}
	\end{center}

	But this shows that any endomorphism on \( J \) makes the rightmost square commute, and is therefore in the Toda bracket (up to pre-composition by an isomorphism \( \frac{R}{J} \to J \)) of \( \toda{\Id_J, 0, \Id_J} \). 

	Toda brackets are denoted up to pre/post-composition of isomorphism, one can write that \( \toda{\Id_J, 0, \Id_J} = \StMod(R)(J, J) \).
\end{example}

