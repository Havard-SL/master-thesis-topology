Want to show that \( \toda{\Id_J, \Id_J, \Id_J} = \emptyset \) since there is no distinguished triangle one can put in the cofiber-cofiber definition such that the squares commute.

The cone of the identity map is \( 0 \), and \( \Sigma J \cong J \), and so one has the following diagram by the iterated cofiber definition of toda bracket
\begin{center}
	\begin{tikzpicture}
		\diagram{m}{1cm}{1cm} {
			J \& J \& 0 \& J \\
			J \& J \& J \& J. \\
		};

		\draw[math]
			(m-1-1) edge node {\Id} (m-1-2)
				edge[equal] (m-2-1)
			(m-1-2) edge node {0} (m-1-3)
				edge[equal] (m-2-2)
			(m-1-3) edge node {0} (m-1-4)
				edge[squiggly] node {0} (m-2-3)
			(m-1-4) edge node {\psi} (m-2-4)

			(m-2-1) edge node {\Id} (m-2-2)
			(m-2-2) edge node {\Id} (m-2-3)
			(m-2-3) edge node {\Id} (m-2-4);
	\end{tikzpicture}
\end{center}

There is no squiggly map in the diagram above that can make the middle square commute, unless \( J \cong 0 \).

Therefore \( \toda{\Id_J, \Id_J, \Id_J} = \emptyset \).

Generally, toda brackets have the following property
\begin{theorem}
	Let \( f_1, f_2, f_3 \) be three composable maps in any triangulated category, \( \Tc \):

	\begin{center}
		\begin{tikzpicture}
			\diagram{m}{1cm}{1cm} {
				X_1 \& X_2 \& X_3 \& X_4 \\
			};

			\draw[math]
				(m-1-1) edge node {f_1} (m-1-2)
				(m-1-2) edge node {f_2} (m-1-3)
				(m-1-3) edge node {f_3} (m-1-4);
		\end{tikzpicture}
	\end{center}

	such that \( f_2 \circ f_1 \neq 0 \) or \( f_3 \circ f_2 \neq 0 \).

	Then \( \toda{f_3, f_2, f_1} = \emptyset \).
\end{theorem}
\begin{proof}
	Assume that \( \toda{f_3, f_2, f_1} \neq \emptyset \). Then from the definition of Toda bracket there exists maps \(\alpha, \beta, \phi, \psi \) such that the following diagram commutes, and the top row is distinguished
	\begin{center}
		\begin{tikzpicture}
			\diagram{m}{1cm}{1cm} {
				X_1 \& X_2 \& C_{f_1} \& \Sigma X_1 \\
				X_1 \& X_2 \& X_3 \& X_4. \\
			};

			\draw[math]
				(m-1-1) edge node {f_1} (m-1-2)
					edge[equal]	(m-2-1)
				(m-1-2) edge node {\alpha} (m-1-3)
					edge[equal] (m-2-2)
				(m-1-3) edge node {\beta} (m-1-4)
					edge node {\phi} (m-2-3)
				(m-1-4) edge node {\psi} (m-2-4)

				(m-2-1) edge node {f_1} (m-2-2)
				(m-2-2) edge node {f_2} (m-2-3)
				(m-2-3) edge node {f_3} (m-2-4);
		\end{tikzpicture}
	\end{center}

	Split the proof into two different contradictions:

	\begin{itemize}
		\item{
			Case 1:

			Assume that \( f_2 \circ f_1 \neq 0 \). Then since \( \alpha, f_1 \) are two composable maps from the same distinguished triangle one has
			\[
				\phi \circ \alpha \circ f_1 = \phi \circ 0 = 0.
			\]

			But from commutativity of the diagram one also has
			\[
				\phi \circ \alpha \circ f_1 = f_2 \circ f_1 \neq 0.
			\]

			Which is a contradiction, so \( f_2 \circ f_1 = 0 \).
		}
		\item{
			Case 2:

			Assume that \( f_3 \circ f_2 \neq 0 \). Then one has that
			\[
				0 = \psi \circ 0 = \psi \circ \beta \circ \alpha = f_3 \circ \phi \circ \alpha = f_3 \circ f_2 \neq 0.
			\]
			Which is also a contradiction.
		}
	\end{itemize}

	Therefore both \( f_2 \circ f_1 = 0 \) and \( f_3 \circ f_2 = 0 \) if \( \toda{f_3, f_2, f_1} \neq \emptyset \), which is contrapositive to the statement in the theorem.

\end{proof}
