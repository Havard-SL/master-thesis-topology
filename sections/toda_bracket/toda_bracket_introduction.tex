The following three definitions are all based on \cite[Definition 3.1]{Christensen-Frankland_2017}.

Let \( \Tc \) be a triangulated category, given the following diagram in \( \Tc \):
\begin{center}
    \begin{tikzpicture}
        \diagram{m}{1cm}{1cm} {
            {X_0} \& {X_1} \& {X_2} \& {X_3} \\
        };

        \draw[math]
            (m-1-1) edge node {f_1} (m-1-2)
            (m-1-2) edge node {f_2} (m-1-3)
            (m-1-3) edge node {f_3} (m-1-4);
    \end{tikzpicture}
\end{center}

one can define the \emph{three-fold Toda bracket of \( f_3, f_2, f_1 \)} in three different (but actually identical, see \autoref{prop:toda-bracket-definitions-coincide}) ways:

\begin{definition}[Iterated cofiber Toda bracket]
    \label{def:iterated-cofiber-toda-bracket}
    Let the set of every possible \( \psi \in \Tc(\Sigma(X_0), X_3) \) that makes the following diagram commute
    \begin{center}
        \begin{tikzpicture}
            \diagram{m}{1cm}{1cm} {
                {X_0} \& {X_1} \& {Y} \& {\Sigma(X_0)} \\
                {X_0} \& {X_1} \& {X_2} \& {X_3} \\
            };

            \draw[math]
                (m-1-1) edge node {f_1} (m-1-2)
                    edge[equal] (m-2-1)
                (m-1-2) edge (m-1-3)
                    edge[equal] (m-2-2)
                (m-1-3) edge (m-1-4)
                    edge (m-2-3)
                (m-1-4) edge node {\psi} (m-2-4)

                (m-2-1) edge node {f_1} (m-2-2)
                (m-2-2) edge node {f_2} (m-2-3)
                (m-2-3) edge node {f_3} (m-2-4);
        \end{tikzpicture}
    \end{center}
    where the top row is distinguished, be denoted as \( \toda{f_3, f_2, f_1}_{\cc} \). This is called the \emph{three-fold iterated cofiber Toda Bracket of \( f_3, f_2, f_1 \)}.
\end{definition}

\begin{definition}[Fiber-cofiber Toda bracket]
    \label{def:fiber-cofiber-toda-bracket}
    Let the set of every composite \( \beta \circ \Sigma(\alpha) \in \Tc(\Sigma(X_0), X_3) \) that makes the following diaram commute
    \begin{center}
        \begin{tikzpicture}
            \diagram{m}{1cm}{1cm} {
                {X_0} \& {X_1} \\
                {\Sigma^{-1}(Y)} \& {X_1} \& {X_2} \& {Y} \\
                \&\& {X_2} \& {X_3} \\
            };

            \draw[math]
                (m-1-1) edge node {f_1} (m-1-2)
                    edge node {\alpha} (m-2-1)
                (m-1-2) edge[equal] (m-2-2)

                (m-2-1) edge (m-2-2)
                (m-2-2) edge node {f_2} (m-2-3)
                (m-2-3) edge (m-2-4)
                    edge[equal] (m-3-3)
                (m-2-4) edge node {\beta} (m-3-4)

                (m-3-3) edge node {f_3} (m-3-4);
        \end{tikzpicture}
    \end{center}
    where the middle row is distinguished, be denoted as \( \toda{f_3, f_2, f_1}_{\fc} \). This is called the \emph{three-fold fiber-cofiber Toda Bracket of \( f_3, f_2, f_1 \)}.
\end{definition}

\begin{definition}[Iterated fiber Toda bracket]
    \label{def:iterated-fiber-toda-bracket}
    Let the set of every morphism \( \Sigma(\delta) \in \Tc(\Sigma(X_0), X_3) \), where \( \delta \) makes the following diagram commute
    \begin{center}
        \begin{tikzpicture}
            \diagram{m}{1cm}{1cm} {
                {X_0} \& {X_1} \& {X_2} \& {X_3} \\
                {\Sigma^{-1}(X_3)} \& {Y} \& {X_2} \& {X_3} \\
            };

            \draw[math]
                (m-1-1) edge node {f_1} (m-1-2)
                    edge node {\delta} (m-2-1)
                (m-1-2) edge node {f_2} (m-1-3)
                    edge (m-2-2)
                (m-1-3) edge node {f_3} (m-1-4)
                    edge[equal] (m-2-3)
                (m-1-4) edge[equal] (m-2-4)

                (m-2-1) edge (m-2-2)
                (m-2-2) edge (m-2-3)
                (m-2-3) edge node {f_3} (m-2-4);
        \end{tikzpicture}
    \end{center}
    where the bottom row is distinguished, be denoted as \( \toda{f_3, f_2, f_1}_{\ff} \). This is called the \emph{three-fold iterated fiber Toda Bracket of \( f_3, f_2, f_1 \)}.
\end{definition}

As is alluded by the name, as well as mentioned above, these are in fact equivalent ways to describe the same subset of \( \Tc(\Sigma(X_0), X_3) \) which is clear by the following proposition.

\begin{proposition}
    \label{prop:toda-bracket-definitions-coincide}
    All three of the toda brackets definitions above (\autoref{def:iterated-cofiber-toda-bracket}, \autoref{def:fiber-cofiber-toda-bracket}, \autoref{def:iterated-fiber-toda-bracket}) are equal.

    I.e:
    \[
        \toda{f_3, f_2, f_1}_{\cc} = \toda{f_3, f_2, f_1}_{\fc} = \toda{f_3, f_2, f_1}_{\ff}.
    \]
\end{proposition}

For a proof of \autoref{prop:toda-bracket-definitions-coincide}, see \cite[Proposition 3.3]{Christensen-Frankland_2017}. Note that in the beforementioned proof, that the final centered equation shold say ``\( f_3 - \beta q_2 = \theta q \)'' instead of `` \( f_3 - \beta q_2 = \theta \iota \) ''.
% TODO: Check "Dual" statement. MS-Question: Is it even a dual statement?

As a consequence of \autoref{prop:toda-bracket-definitions-coincide}, one can define just a \emph{Toda bracket} in the following manner.

\begin{definition}
    As a consequence of \autoref{prop:toda-bracket-definitions-coincide} one can uniquely define the \emph{Toda bracket of \( f_3, f_2, f_1 \)} as either \( \toda{f_3, f_2, f_1}_{\cc} \), \( \toda{f_3, f_2, f_1}_{\fc} \), or \( \toda{f_3, f_2, f_1}_{\ff} \), and it is denoted simply as \emph{\( \toda{f_3, f_2, f_1} \)}.
\end{definition}

% TODO: Indeterminancy?

% TODO: Cut?
% \begin{remark}
%     Note that every \( Y \) in the definitions \autoref{def:iterated-cofiber-toda-bracket}, \autoref{def:fiber-cofiber-toda-bracket} and \autoref{def:iterated-fiber-toda-bracket}, are isomorphic to a cone of the respective map.
% \end{remark}
