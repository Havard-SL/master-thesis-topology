English:

The main goal of this thesis is to prove that Toda brackets and Massey products are equal in algebraic triangulated categories. We start by defining triangulated categories, and prove why the stable module category is triangulated. Then we define Toda brackets on triangulated categories, and calculate some examples in the stable module category. Afterwards, we sketch a proof of why the category of chain complexes of modules over a commutative ring with identity is symmetric monoidal closed. We define the (enriched) category of DG-modules as an enriched functor category, and algebraic triangulated categories by the way of (small) DG-enhancements. Massey products are defined in three ways: on the cohomology category of a DG-category, on the 0th cohomology category of the (enriched) category of DG-modules, as well as on any algebraic triangulated category. Then we calculate some examples of Massey products in the stable module category and see that they coincide with the previous examples of Toda brackets. To conclude with, we mention a potential application where the equality between Toda brackets and Massey products could be useful.

Norwegian:

Føremålet med denne oppgåva er å bevisa at Todaklammer og Masseyprodukt er like i algebraiske triangulerte kategoriar. Me byrjar med å definera triangulerte kategoriar, og viser kvifor den stabile modulkategorien er triangulert. Deretter definerer me Todaklammer på triangulerte kategoriar, og reiknar nokre eksempel i den stabile modulkategorien. Etterpå skisserer me eit bevis for kvifor kjedekomplekskategorien av modular over ein kommutativ ring med identitet er symmetrisk monoidal lukka. Me definerer den (berika) kategorien av DG-modular som ein berika funktorkategori, og algebraiske triangulerte kategoriar ved å bruka (små) DG-forbetringar. Masseyprodukt vert definert på tre ulike måtar: på kohomologikategorien av DG-kategoriar, på den 0. kohomologi kategorien til den (berika) kategorien av DG-modular, og på vilkårlege algebraiske triangulerte kategoriar. Me reiknar ut nokre eksempel på Masseyprodukt i den stabile modulkategorien, og ser at dei samsvarar med dei førre utreikningane av Todaklammar. Avslutningsvis nemner me ei potensiell anvending der likskapen mellom Todaklammer og Masseyprodukt kan vera hjelpsam.