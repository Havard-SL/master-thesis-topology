We start by proving some lemmas and giving some definitions that are required for the proof.

% TODO: Burde klarifisera likhet vs isomorfi. Kvifor er Sigma^i for morfiar i degree partal likheit?

\begin{corollary}
    \label{cor:dgm_i_eq_dgm_0_shifted_codomain}
    \[
        \dgM(F, G)_i = \dgM(F, \Sigma^i G)_0
    \]
\end{corollary}
\begin{proof}
    By \autoref{lem:dgmod_shift_eq_plus},
    \[
        \dgM(F, G)_i = \dgM(\Sigma^{-i} F, G)_0.
    \]
    Then, by applying \( \Sigma^i \), we get the result.
\end{proof}

\begin{remark}
    \label{rem:dgm_different_dg_morphisms_same_space_give_degree-wise_same_morphisms}
    An interesting consequence of the definition of \( \dgM \) is that since \( (\Sigma F) \oplus G \) and \( C_{\eta} \) has the property that for every \( A \in \Cc \), and every \( n \in \Zb \),
    \[
        ((\Sigma F) \oplus G)(A)_n = (\Sigma F A)_n \oplus (G A)_n = C_{\eta}(A)_n,
    \]
    then for any \( H \in \dgM \) and any \( i \in \Zb \), \( \dgM(H, (\Sigma F) \oplus G)_i = \dgM(H, C_{\eta})_i \) as well as \( \dgM((\Sigma F) \oplus G, H)_i = \dgM(C_{\eta}, H)_i \).

    In other words, the differentials of the \( ((\Sigma F) \oplus G) A \)s and \( C_{\eta} A \)s only affect the differentials of the morphism spaces in \( \dgM \), but not the modules in each degree.
\end{remark}

The following lemma connects Toda brackets and Massey products in \( H^0(\dgM) \) and is the reason we can compare Toda brackets and Massey products in algebraic triangulated categories.

\begin{lemma}
    Let \( \Cc \) be a small DG-category, and let the following be a diagram in \( H^0(\dgM) \),
    \begin{center}
        \begin{tikzpicture}
            \diagram{m}{1cm}{1cm} {
                X_1 \& X_2 \& X_3 \& X_4. \\
            };

            \draw[math]
                (m-1-1) edge node {f_1} (m-1-2)
                (m-1-2) edge node {f_2} (m-1-3)
                (m-1-3) edge node {f_3} (m-1-4);
        \end{tikzpicture}
    \end{center}
    Then
    \[
        \toda{f_3, f_2, f_1} = TODO\massey{f_3, f_2, f_1}.
    \]
\end{lemma}
\begin{proof}
    We will prove this by showing the two inclusions \( \supseteq \) and \( \subseteq \).

    Start by showing \( \supseteq \).

    Let \( f \in \massey{f_3, f_2, f_1} \). Then by definition of massey product there exists
    \[
        g_i \in \dgM(X_i, X_{i + 1})
    \]
    with \( [g_i] = f_i \) for \( i = 1, 2, 3 \), as well as
    \[
        s \in \dgM(X_2, X_4)_{-1}
    \]
    and
    \[
        t \in \dgM(X_1, X_3)_{-1}
    \]
    with \( d(s) = \bar{g_3} \circ g_2 \) and \( d(t) = \bar{g_2} \circ g_1 \) such that
    \[
        f = \class{\bar{s} \circ g_1 + \bar{g_3} \circ t}.
    \]

    % WIP: Kvifor kan eg laga matrise av dgmod morfiar og få ein dgmod morfi?
    %   Kvifor kan alpha gå inn i cone?
    By \autoref{cor:dgm_i_eq_dgm_0_shifted_codomain}, consider \( t \) as an element of \( \dgM(X_1, \Sigma^{-1} X_3)_0 \). Then we can construct the following morphism.
    \[
        \alpha =
        \begin{pmatrix}
            g_1 \\
            t
        \end{pmatrix}
        \in \dgM(X_1, X_2 \oplus \Sigma^{-1} X_3)_0.
    \]
    By \autoref{rem:dgm_different_dg_morphisms_same_space_give_degree-wise_same_morphisms}, we can consider \( \alpha \) as an element of \( \dgM(X_1, \Sigma^{-1} C_{\eta}) \).

    Similarly, consider \( s \) as an element of \( \dgM(X_2, \Sigma^{-1} X_4)_0 \), and \( g_3 \) as a morphism in \( \dgM(\Sigma^{-1} X_3, \Sigma^{-1} X_4)_0 \). Then define
    \[
        \beta =
        \begin{pmatrix}
            s & -g_3
        \end{pmatrix}
        : X_2 \oplus \Sigma^{-1} X_3 \to \Sigma^{-1} X_4,
    \]
    which also by \autoref{rem:dgm_different_dg_morphisms_same_space_give_degree-wise_same_morphisms} can be considered as a morphism in \( \dgM(\Sigma^{-1} C_{\eta}, \Sigma^{-1} X_4)_0 \).

    WIP
\end{proof}

The following is the proof of \autoref{theorem:massey_equals_toda}.

\begin{proof}[Proof of Massey products = Toda brackets]
    
\end{proof}