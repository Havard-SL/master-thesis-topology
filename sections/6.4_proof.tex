We start by proving some lemmas and giving some definitions that are required for the proof.

\begin{definition}
    
\end{definition}

The following lemma connects Toda brackets and Massey products in \( H^0(\dgM) \) and is the reason we can compare Toda brackets and Massey products in algebraic triangulated categories.

\begin{lemma}
    Let \( \Cc \) be a small DG-category, and let the following be a diagram in \( H^0(\dgM) \),
    \begin{center}
        \begin{tikzpicture}
            \diagram{m}{1cm}{1cm} {
                X_1 \& X_2 \& X_3 \& X_4. \\
            };

            \draw[math]
                (m-1-1) edge node {f_1} (m-1-2)
                (m-1-2) edge node {f_2} (m-1-3)
                (m-1-3) edge node {f_3} (m-1-4);
        \end{tikzpicture}
    \end{center}
    Then
    \[
        \toda{f_3, f_2, f_1} = TODO\massey{f_3, f_2, f_1}.
    \]
\end{lemma}
\begin{proof}
    We will prove this by showing the two inclusions \( \supseteq \) and \( \subseteq \).

    Start by showing \( \supseteq \).

    Let \( f \in \massey{f_3, f_2, f_1} \). Then by definition of massey product there exists
    \[
        g_i \in \dgM(X_i, X_{i + 1})
    \]
    with \( [g_i] = f_i \) for \( i = 1, 2, 3 \), as well as
    \[
        s \in \dgM(X_2, X_4)_{-1}
    \]
    and
    \[
        t \in \dgM(X_1, X_3)_{-1}
    \]
    with \( d(s) = \bar{g_3} \circ g_2 \) and \( d(t) = \bar{g_2} \circ g_1 \) such that
    \[
        f = \class{\bar{s} \circ g_1 + \bar{g_3} \circ t}.
    \]

    Let
    \[
        \alpha =
        \begin{pmatrix}
            g_1 \\
            t
        \end{pmatrix}
    \]
\end{proof}

The following is the proof of \autoref{theorem:massey_equals_toda}.

\begin{proof}[Proof of Massey products = Toda brackets]
    
\end{proof}