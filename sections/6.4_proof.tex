We start by proving that Toda brackets and Massey products coincide in \( H^0(\dgM) \), which we will use to prove they coincide on any algebraic triangulated category.
\begin{theorem}
    \label{thm:dgm_massey_equal_toda}
    Let \( \Cc \) be a small DG-category, and let the following be a diagram in \( H^0(\dgM) \),
    \begin{center}
        \begin{tikzpicture}
            \diagram{m}{1cm}{1cm} {
                X_1 \& X_2 \& X_3 \& X_4. \\
            };

            \draw[math]
                (m-1-1) edge node {f_1} (m-1-2)
                (m-1-2) edge node {f_2} (m-1-3)
                (m-1-3) edge node {f_3} (m-1-4);
        \end{tikzpicture}
    \end{center}
    Then
    \[
        \toda{f_3, f_2, f_1} = \massey{f_3, f_2, f_1}.
    \]
\end{theorem}
% TODO: Korleis fungerar komposisjon og "endring av grad"? F.eks s \circ g_1 i siste del av beviset er rart, kanskje ikkje veldefinert.
\begin{proof}
    We will prove this by showing the two inclusions \( \supseteq \) and \( \subseteq \).

    Start by showing \( \supseteq \).

    Let \( f \in \massey{f_3, f_2, f_1} \). Then by definition of Massey product there exists
    \[
        g_i \in \dgM(X_i, X_{i + 1})_0
    \]
    with \( [g_i] = f_i \) for \( i = 1, 2, 3 \), as well as
    \[
        s \in \dgM(X_2, X_4)_{-1}
    \]
    and
    \[
        t \in \dgM(X_1, X_3)_{-1}
    \]
    with \( d(s) = - g_3 \circ g_2 \) and \( d(t) = - g_2 \circ g_1 \) such that
    \[
        f = \class{s \circ g_1 - g_3 \circ t}.
    \]

    % WIP: Kvifor er alpha i dgM?
    By \autoref{lem:dgmod_shift_eq_plus}, consider \( t \) as an element of \( \dgM(\Sigma X_1, X_3)_0 \). Consider \( \Sigma^{-1} t = t \). Then we can construct the following morphism.
    \[
        \alpha =
        \begin{pmatrix}
            - g_1 \\
            - t
        \end{pmatrix}
        \in \dgM(X_1, X_2 \oplus \Sigma^{-1} X_3)_0.
    \]
    By \autoref{rem:dgm_different_dg_morphisms_same_space_give_degree-wise_same_morphisms}, we can consider \( \alpha \) as an element of \( \dgM(X_1, \Sigma^{-1} C_{g_2})_0 \).

    Similarly, consider \( s \) as an element of \( \dgM(\Sigma X_2, X_4)_0 \).

    Then define
    \[
        \beta =
        \begin{pmatrix}
            - s & g_3
        \end{pmatrix}
        : (\Sigma X_2) \oplus X_3 \to X_4,
    \]
    which also by \autoref{rem:dgm_different_dg_morphisms_same_space_give_degree-wise_same_morphisms} can be considered as a morphism in \( \dgM(C_{g_2}, X_4)_0 \).

    Then we want to show that \( \alpha \) and \( \beta \) are cocycles:

    Consider the following two equations
    \begin{align*}
        d_{\dgM(X_1, \Sigma^{-1} C_{g_2})}(\alpha)
        &= d_{\Sigma^{-1} C_{g_2}} \circ \alpha - \alpha \circ d_{X_1} \\
        &= -
        \begin{pmatrix}
            - d_{X_2} & 0 \\
            g_2 & d_{X_3}
        \end{pmatrix}
        \circ
        \begin{pmatrix}
            - g_1 \\
            - t
        \end{pmatrix}
        -
        \begin{pmatrix}
            - g_1 \\
            - t
        \end{pmatrix}
        \circ
        d_{X_1} \\
        &=
        \begin{pmatrix}
            - d_{X_2} \circ g_1 + g_1 \circ d_{X_1} \\
            g_2 \circ g_1 + d_{X_3} \circ t + t \circ d_{X_1}
        \end{pmatrix} \\
        &=
        \begin{pmatrix}
            - d_{\dgM(X_1, X_2)}(g_1) \\
            - \bar{g_2} \circ g_1 + d_{\dgM(X_1, X_3)}(t)
        \end{pmatrix}
        =
        0,
    \end{align*}
    and,
    \begin{align*}
        d_{\dgM(C_{g_2}, X_4)}(\beta)
        &= d_{X_4} \circ \beta - \beta \circ d_{C_{g_2}} \\
        &= d_{X_4} \circ
        \begin{pmatrix}
            - s & g_3
        \end{pmatrix}
        -
        \begin{pmatrix}
            - s & g_3
        \end{pmatrix}
        \circ
        \begin{pmatrix}
            - d_{X_2} & 0 \\
            g_2 & d_{X_3}
        \end{pmatrix} \\
        &=
        \begin{pmatrix}
            - d_{X_4} \circ s - s \circ d_{X_2} - g_3 \circ g_2 & d_{X_4} \circ g_3 - g_3 \circ d_{X_3}
        \end{pmatrix} \\
        &=
        \begin{pmatrix}
            - d_{\dgM(X_2, X_4)}(s) + \bar{g_3} \circ g_2 & d_{\dgM(X_3, X_4)}(g_3)
        \end{pmatrix}
        = 0.
    \end{align*}
    
    Then we want to show that \( \alpha \) and \( \beta \) fit into the fiber-cofiber definition of Toda brackets.

    Since \( \Sigma_{H^0(\dgM)} \) and is an automorphism by \autoref{rem:dgm_sigma_automorphism}, we can use a simplified version of the fiber-cofiber definition of Toda brackets without assuming a natural isomorphism from \( \Sigma \Sigma^{-1} \) to \( \Id \).

    We want the following diagram to commute, where the middle row is the right-rotated standard triangle of \( g_2 \),
    \begin{center}
        \begin{tikzpicture}
            \diagram{m}{1cm}{1cm} {
                X_1 \& X_2 \\
                \Sigma^{-1} C_{g_2} \& X_2 \& X_3 \& C_{g_2} \\
                \& \& X_3 \& X_4. \\
            };

            \draw[math]
                (m-1-1) edge node {[f_1]} (m-1-2)
                    edge node {[\alpha]} (m-2-1)
                (m-1-2) edge[equality] (m-2-2)

                (m-2-1) edge node {[- \Sigma^{-1} \pi_{\Sigma X_2}]} (m-2-2)
                (m-2-2) edge node {[f_2]} (m-2-3)
                (m-2-3) edge node {[\iota]} (m-2-4)
                    edge[equality] (m-3-3)
                (m-2-4) edge node {[\beta]} (m-3-4)

                (m-3-3) edge node {[f_3]} (m-3-4);
        \end{tikzpicture}
    \end{center}

    We show that the left square commutes,
    \begin{align*}
        \class*{- (\Sigma^{-1} \pi) \circ \alpha} &= \class{- \pi \circ \alpha} \\
        &=
        \class*{
            -
            \begin{pmatrix}
                1 & 0
            \end{pmatrix}
            \circ
            \begin{pmatrix}
                - g_1 \\
                - t
            \end{pmatrix} 
        } \\
        &= \class*{g_1} = \class*{f_1}
    \end{align*}
    and that the right square commutes,
    \begin{align*}
        \class*{\beta \circ \iota} &=
        \class*{
            \begin{pmatrix}
                - s & g_3
            \end{pmatrix}
            \circ
            \begin{pmatrix}
                0 \\
                1
            \end{pmatrix}
         } \\
        &= \class*{g_3} = \class*{f_3}.
    \end{align*}

    That means that:
    \[
        \class*{\beta \circ (\Sigma \alpha)} = \class*{\beta \circ \alpha} = f
    \]
    is in \( \toda{f_3, f_2, f_1} \).

    Then we want to show \( \subseteq \):

    % TODO: Kvifor kan me anta at Y = C_g_2?
    Let \( f \in \toda{f_3, f_2, f_1} \). Using the fiber-cofiber definition of Toda brackets, and assuming \( Y = C_{f_2} \), we get that there exist some \( \alpha \in \dgM(X_1, \Sigma^{-1} C_{f_2})_0 \) and \( \beta \in \dgM(C_{f_2}, X_4)_0 \), such that
    \[
        f = \beta \circ (\Sigma \alpha).
    \]

    We want to find \( g_1 \in \dgM(X_1, X_2)_0 \), \( g_2 \in \dgM(X_2, X_3)_0 \), \( g_3 \in \dgM(X_3, X_4)_0 \), \( s \in \dgM(X_2, X_4)_{-1} \), and \( t \in \dgM(X_1, X_3)_{-1} \) with \( d(t) = - g_2 \circ g_1 \) and \( d(s) = -g_3 \circ g_2 \), such that \( f = \class{s \circ g_1 - g_3 \circ t} \).

    Let \( g_1 := (-\Sigma^{-1} \pi_{\Sigma X_2}) \circ \alpha \), \( g_2 = f_2 \), and let \( g_3 := \beta \circ \iota_{X_3} \). 

    Furthermore, consider \( \iota_{\Sigma X_2} \in \dgM(\Sigma X_2, C_{f_2})_0 \) as an element of \( \dgM(X_2, C_{f_2})_{-1} \), and let \( s := - \beta \circ \iota_{\Sigma X_2} \). In addition, consider \( \pi_{X_3} \in \dgM(C_{f_2}, X_3)_0 \) as an element of \( \dgM(\Sigma^{-1} C_{f_2}, X_3)_{-1} \), and let \( t := - \pi_{X_3} \circ \alpha \).

    Then the above properties all hold by the following three equations:

    Remember the equations from \autoref{rem:dgm_differentials_of_inclusions_and_projections_of_cone}.

    First, \( d(s) = -g_3 \circ g_2 \):
    \begin{align*}
        d_{\dgM(X_2, X_4)}(s) &= d_{\dgM(X_2, X_4)}(- \beta \circ \iota_{\Sigma X_2}) \\
        &= - d_{\dgM(C_{f_2}, X_4)}(\beta) \circ \iota_{\Sigma X_2} - \beta \circ d_{\dgM(X_2, C_{f_2})}(\iota_{\Sigma X_2}) \\
        \intertext{by assumption, \( \beta \) is a cocycle}
        &= 0 \circ \iota_{\Sigma X_2} - \beta \circ \iota_{X_3} \circ f_2 \\
        &= - \beta \circ \iota_{X_3} \circ f_2 \\
        &= - g_3 \circ g_2.
    \end{align*}
    Second, \( d(t) = - g_2 \circ g_1 \):
    \begin{align*}
         d_{\dgM(X_1, X_3)}(t) &= d_{\dgM(X_1, X_3)}(- \pi_{X_3} \circ \alpha) \\
         &= -d_{\dgM(\Sigma^{-1} C_{f_2}, X_3)}(\pi_{X_3}) \circ \alpha \\
         &= f_2 \circ \pi_{\Sigma X_2} \circ \alpha \\
         &= - f_2 \circ (- \Sigma^{-1} \pi_{\Sigma X_2}) \circ \alpha \\
         &= - g_2 \circ g_1.
    \end{align*}
    Third, \( f = \class{s \circ g_1 - g_3 \circ t} \):
    \begin{align*}
        s \circ g_1 - g_3 \circ t &= - \beta \circ \iota_{\Sigma X_2} \circ (-\Sigma^{-1} \pi_{\Sigma X_2}) \circ \alpha - \beta \circ \iota_{X_3} \circ (- \pi_{X_3}) \circ \alpha \\
        &= \beta \circ \tuple*{\iota_{\Sigma X_2} \circ \pi_{\Sigma X_2} + \iota_{X_3} \circ \pi_{X_3}} \circ \alpha \\
        &= \beta \circ \Id_{C_{f_2}} \circ \alpha \\
        &= \beta \circ \alpha = f,
    \end{align*}
    which implies \( f \in \massey{f_3, f_2, f_1} \).
\end{proof}

The following is the proof of \autoref{theorem:massey_equals_toda}.

\begin{proof}[Proof of Massey products = Toda brackets in an algebraic triangulated category]
    \phantom{hei}

    Let \( F := H^0(\mathbf{h}) \Phi \) and \( F^{-1} := \Phi^{-1} (H^0(\mathbf{h}))^{-1} \) in this proof.

    Let \( \phi: F^{-1} F \to \Id_{\Tc} \) be the natural isomorphism from the definition of Massey products on an algebraic triangulated category and let \( \eta: F \Sigma \to \Sigma F \) be the natural isomorphism from the fact that \( F \) is a triangulated equivalence.

    % TODO: Kvifor er dette naturleg? Ganske sikker, treng resultat i forkant?
    Assume that on \( \im(F) \), the natural isomorphism \( \mu \) is defined as follows: Every object \( A' \in \im(F) \) can be written on the form \( F A \) for some \( A \in \Tc \). Likewise, any morphism \( f': F A \to F B \) can be written on the form \( f' = F f \) for some \( f \in \Tc(A, B) \), since \( F \) is full. Therefore, the following composition defines a valid natural isomorphism on \( \im(F) \),
    \begin{align*}
        \mu: &F^{-1} \Sigma \to \Sigma F^{-1} \\
        &= \set{ (\Sigma \phi^{-1}_A) \circ \phi_{\Sigma A} \circ (F^{-1} \eta^{-1}_A)}_{F A \in \im(F)}.
    \end{align*}

    In order to prove the theorem, we want to prove \( \subseteq \) and \( \supseteq \).

    We start by proving \( \subseteq \):
    
    Let \( f \in \toda{f_3, f_2, f_1} \).

    Then by the fiber-cofiber definition of Toda brackets, there exist some \( \alpha \) and \( \beta \), along with an object \( Y \), such that \( f = \beta \circ (\Sigma \alpha) \), and the following diagram in \( \Tc \) commutes,
    \begin{center}
        \begin{tikzpicture}
            \diagram{m}{1cm}{1.2cm} {
                X_1 \& X_2 \\
                \Sigma^{-1} Y \& X_2 \& X_3 \& \Sigma \Sigma^{-1} Y \\
                \& \& X_3 \& X_4. \\
            };

            \draw[math]
                (m-1-1) edge node {f_1} (m-1-2)
                    edge node {\alpha} (m-2-1)
                (m-1-2) edge[equality] (m-2-2)

                (m-2-1) edge node {-\Sigma^{-1} \pi} (m-2-2)
                (m-2-2) edge node {f_2} (m-2-3)
                (m-2-3) edge node {\iota} (m-2-4)
                    edge[equality] (m-3-3)
                (m-2-4) edge node {\beta} (m-3-4)

                (m-3-3) edge node {f_3} (m-3-4);
        \end{tikzpicture}
    \end{center}
    Consider the morphisms \( F \alpha \) and \( F \beta \).

    They fit into the following commutative diagram,
    \begin{center}
        \begin{tikzpicture}
            \diagram{m}{1cm}{1.8cm} {
                F X_1 \& F X_2 \\
                F \Sigma^{-1} Y \& F X_2 \& F X_3 \& F \Sigma \Sigma^{-1} Y \\
                \& \& F X_3 \& F X_4, \\
            };

            \draw[math]
                (m-1-1) edge node {F f_1} (m-1-2)
                    edge node {F \alpha} (m-2-1)
                (m-1-2) edge[equality] (m-2-2)

                (m-2-1) edge node {F (-\Sigma^{-1} \pi)} (m-2-2)
                (m-2-2) edge node {F f_2} (m-2-3)
                (m-2-3) edge node {F \iota} (m-2-4)
                    edge[equality] (m-3-3)
                (m-2-4) edge node[swap] {F \beta} (m-3-4)

                (m-3-3) edge node {F f_3} (m-3-4);
        \end{tikzpicture}
    \end{center}
    which we can rewrite into the following commutative diagram,
    \begin{center}
        \begin{tikzpicture}
            \diagram{m}{1cm}{1.8cm} {
                F X_1 \& F X_2 \\
                F \Sigma^{-1} Y \& F X_2 \& F X_3 \& \Sigma F \Sigma^{-1} Y \\
                \& \& F X_3 \& F X_4, \\
            };

            \draw[math]
                (m-1-1) edge node {F f_1} (m-1-2)
                    edge node {F \alpha} (m-2-1)
                (m-1-2) edge[equality] (m-2-2)

                (m-2-1) edge node {F (-\Sigma^{-1} \pi)} (m-2-2)
                (m-2-2) edge node {F f_2} (m-2-3)
                (m-2-3) edge node {\eta_{\Sigma^{-1} Y} \circ (F \iota)} (m-2-4)
                    edge[equality] (m-3-3)
                (m-2-4) edge node[swap] {(F \beta) \circ \eta^{-1}_{\Sigma^{-1} Y}} (m-3-4)

                (m-3-3) edge node {F f_3} (m-3-4);
        \end{tikzpicture}
    \end{center}

    where the middle row is distinguished since \( F \) is a triangulated functor.

    This yields an element of the Toda bracket,
    \begin{align*}
        (F \beta) \circ \eta^{-1}_{\Sigma^{-1} Y} \circ (\Sigma F \alpha) &= (F \beta) \circ \eta^{-1}_{\Sigma^{-1} Y} \circ \eta_{\Sigma^{-1} Y} \circ (F \Sigma \alpha) \circ \eta^{-1}_{X_1} \\
        &= (F (\beta \circ (\Sigma \alpha))) \circ \eta^{-1}_{X_1} \\
        &= (F f ) \circ \eta^{-1}_{X_1}.
    \end{align*}
    Thus, by \autoref{thm:dgm_massey_equal_toda}, we get
    \[
        (F f ) \circ \eta^{-1}_{X_1} \in \toda{F f_3, F f_2, F f_1} = \massey{F f_3, F f_2, F f_1}.
    \]

    Pre-composing with \( \eta_{X_1} \) and applying \( F^{-1} \) as well as the natural isomorphism \( \phi \) yields
    \begin{align*}
        \phi_{X_4} \circ (F^{-1} ((F f ) \circ \eta^{-1}_{X_1} \circ \eta_{X_1})) \circ \phi^{-1}_{\Sigma X_1} &= \phi_{X_4} \circ \phi^{-1}_{X_4} \circ f \circ \phi_{\Sigma X_1} \circ \phi^{-1}_{\Sigma X_1} = f,
    \end{align*}
    and therefore \( f \in \massey{f_3, f_2, f_1} \).

    Finally, we prove \( \supseteq \):

    Assume \( f \in \massey{f_3, f_2, f_1} \).

    Then, by \autoref{thm:dgm_massey_equal_toda}, there exists some
    \[
        \tilde{f} \in \massey{F f_3, F f_2, F f_1} = \toda{F f_3, F f_2, F f_1}
    \]
    such that
    \[
        f = \phi_{X_4} \circ (F^{-1} (\tilde{f} \circ \eta_{X_1})) \circ \phi^{-1}_{\Sigma X_1}.
    \]

    Then \( \tilde{f} = \beta \circ (\Sigma \alpha) \) where \( \alpha \) and \( \beta \) are defined as morphisms that make the following diagram, where the middle row is a distinguished triangle,
    \begin{center}
        \begin{tikzpicture}
            \diagram{m}{1cm}{1.7cm} {
                F X_1 \& F X_2 \\
                F \Sigma^{-1} Y \& F X_2 \& F X_3 \& \Sigma F \Sigma^{-1} Y \\
                \& \& F X_3 \& F X_4, \\
            };

            \draw[math]
                (m-1-1) edge node {F f_1} (m-1-2)
                    edge node {\alpha} (m-2-1)
                (m-1-2) edge[equality] (m-2-2)

                (m-2-1) edge node {F (-\Sigma^{-1} \pi)} (m-2-2)
                (m-2-2) edge node {F f_2} (m-2-3)
                (m-2-3) edge node {\eta_{\Sigma^{-1} Y} \circ (F \iota)} (m-2-4)
                    edge[equality] (m-3-3)
                (m-2-4) edge node[swap] {\beta} (m-3-4)

                (m-3-3) edge node {F f_3} (m-3-4);
        \end{tikzpicture}
    \end{center}
    commute.

    Consider
    \[
        \tilde{\alpha} = \phi_{\Sigma^{-1} Y} \circ (F^{-1} \alpha) \circ \phi^{-1}_{X_1}
    \]
    and
    \[
        \tilde{\beta} = \phi_{X_4} \circ (F^{-1} (\beta \circ \eta_{\Sigma^{-1} Y})) \circ \phi^{-1}_{\Sigma \Sigma^{-1} Y}.
    \]

    Consider the following diagram
    \begin{center}
        \begin{tikzpicture}
            \diagram{m}{1cm}{1.2cm} {
                X_1 \& X_2 \\
                \Sigma^{-1} Y \& X_2 \& X_3 \& \Sigma \Sigma^{-1} Y \\
                \& \& X_3 \& X_4. \\
            };

            \draw[math]
                (m-1-1) edge node {f_1} (m-1-2)
                    edge node {\tilde{\alpha}} (m-2-1)
                (m-1-2) edge[equality] (m-2-2)

                (m-2-1) edge node {-\Sigma^{-1} \pi} (m-2-2)
                (m-2-2) edge node {f_2} (m-2-3)
                (m-2-3) edge node {\iota} (m-2-4)
                    edge[equality] (m-3-3)
                (m-2-4) edge node {\tilde{\beta}} (m-3-4)

                (m-3-3) edge node {f_3} (m-3-4);
        \end{tikzpicture}
    \end{center}
    This commutes because
    \begin{align*}
        (- \Sigma^{-1} \pi) \circ \tilde{\alpha} &= \phi_{X_2} \circ (F^{-1} F (- \Sigma^{-1} \pi)) \circ \phi^{-1}_{\Sigma^{-1} Y} \circ \phi_{\Sigma^{-1} Y} \circ (F^{-1} \alpha) \circ \phi^{-1}_{X_1}\\
        &= \phi_{X_2} \circ (
            F^{-1} (
                (F (- \Sigma^{-1} \pi)) \circ \alpha
                )
            ) \circ \phi^{-1}_{X_1} \\
        &= \phi_{X_2} \circ (F^{-1} (F f_1)) \circ \phi^{-1}_{X_1} \\
        &= f_1,
    \end{align*}
    and
    \begin{align*}
        \tilde{\beta} \circ \iota &= \phi_{X_4} \circ (F^{-1} (\beta \circ \eta_{\Sigma^{-1} Y})) \circ \phi^{-1}_{\Sigma \Sigma^{-1} Y} \circ \phi_{\Sigma \Sigma^{-1} Y} \circ (\Phi^{-1} H^0(\mathbf{h})^{-1} F \iota) \circ \phi^{-1}_{X_3} \\
        &= \phi_{X_4} \circ (F^{-1}(\beta \circ \eta_{\Sigma^{-1} Y} \circ (F \iota))) \circ \phi^{-1}_{X_3} \\
        &= \phi_{X_4} \circ (F^{-1} F f_3) \circ \phi^{-1}_{X_3} \\
        &= f_3.
    \end{align*}

    Since
    \begin{align*}
        \tilde{\beta} \circ (\Sigma \tilde{\alpha}) &= \phi_{X_4} \circ (F^{-1} (\beta \circ \eta_{\Sigma^{-1} Y})) \circ \phi^{-1}_{\Sigma \Sigma^{-1} Y} \circ
        (\Sigma (\phi_{\Sigma^{-1} Y} \circ (F^{-1} \alpha) \circ \phi^{-1}_{X_1})) \\
        &= \phi_{X_4} \circ (F^{-1} (\beta \circ \eta_{\Sigma^{-1} Y})) \circ \phi^{-1}_{\Sigma \Sigma^{-1} Y} \circ
        (\Sigma \phi_{\Sigma^{-1} Y}) \circ (\Sigma F^{-1} \alpha) \circ (\Sigma \phi^{-1}_{X_1}) \\
        &= \phi_{X_4} \circ (F^{-1} (\beta \circ \eta_{\Sigma^{-1} Y})) \circ \phi^{-1}_{\Sigma \Sigma^{-1} Y} \circ
        (\Sigma \phi_{\Sigma^{-1} Y}) \circ \mu_{F \Sigma^{-1} Y} \\
        &\hspace{1cm} \circ (F^{-1} \Sigma \alpha) \circ \mu^{-1}_{F X_1} \circ (\Sigma \phi^{-1}_{X_1}) \\
        &= \phi_{X_4} \circ (F^{-1} (\beta \circ \eta_{\Sigma^{-1} Y})) \circ \phi^{-1}_{\Sigma \Sigma^{-1} Y} \circ
        (\Sigma \phi_{\Sigma^{-1} Y}) \circ (\Sigma \phi^{-1}_{\Sigma^{-1} Y}) \circ \phi_{\Sigma \Sigma^{-1} Y} \circ (F^{-1} \eta^{-1}_{\Sigma^{-1} Y}) \\
        &\hspace{1cm} \circ (F^{-1} \Sigma \alpha) \circ (F^{-1} \eta_{X_1}) \circ \phi^{-1}_{\Sigma X_1} \circ (\Sigma \phi_{X_1}) \circ (\Sigma \phi^{-1}_{X_1}) \\
        &= \phi_{X_4} \circ (F^{-1} (\beta \circ \eta_{\Sigma^{-1} Y} \circ \eta^{-1}_{\Sigma^{-1} Y} \circ (\Sigma \alpha) \circ \eta_{X_1})) \circ \phi^{-1}_{\Sigma X_1} \\
        &= \phi_{X_4} \circ (F^{-1} (\tilde{f} \circ \eta_{X_1})) \circ \phi^{-1}_{\Sigma X_1} \\
        &= f,
    \end{align*}
    this implies that \( f \in \toda{f_3, f_2, f_1} \).
\end{proof}
