\section{What is Toda Brackets?}

Besed on Christensen \& Frankland article (TODO).

\begin{definition}
    Given a triangulated category \( \Tc \), given the following diagram in \( \Tc \):

    % https://q.uiver.app/#q=WzAsNCxbMCwwLCJYXzAiXSxbMSwwLCJYXzEiXSxbMiwwLCJYXzIiXSxbMywwLCJYXzMiXSxbMCwxLCJmXzEiXSxbMSwyLCJmXzIiXSxbMiwzLCJmXzMiXV0=
    \[\begin{tikzcd}
        {X_0} & {X_1} & {X_2} & {X_3}
        \arrow["{f_1}", from=1-1, to=1-2]
        \arrow["{f_2}", from=1-2, to=1-3]
        \arrow["{f_3}", from=1-3, to=1-4]
    \end{tikzcd}\]

    one can define the \emph{three-fold Toda bracket of \( f_3, f_2, f_1 \)}, denoted \( \langle f_3, f_2, f_1 \rangle \), a subset of \( \Hom_{\Tc}(\Sigma(X_0), X_3) \) in three different ways:

    \begin{enumerate}
        \item {
            Iterated cofiber Toda bracket:

            Every possible \( \psi \) that makes the following diagram commute:

            % https://q.uiver.app/#q=WzAsOCxbMCwwLCJYXzAiXSxbMSwwLCJYXzEiXSxbMywwLCJcXFNpZ21hKFhfMCkiXSxbMiwwLCJDX2YiXSxbMCwxLCJYXzAiXSxbMSwxLCJYXzEiXSxbMiwxLCJYXzIiXSxbMywxLCJYXzMiXSxbMCwxLCJmXzEiXSxbMSwzXSxbMywyXSxbMCw0LCIiLDIseyJsZXZlbCI6Miwic3R5bGUiOnsiaGVhZCI6eyJuYW1lIjoibm9uZSJ9fX1dLFsxLDUsIiIsMix7ImxldmVsIjoyLCJzdHlsZSI6eyJoZWFkIjp7Im5hbWUiOiJub25lIn19fV0sWzMsNl0sWzIsNywiXFxwc2kiXSxbNCw1LCJmXzEiXSxbNSw2LCJmXzIiXSxbNiw3LCJmXzMiXV0=
            \[\begin{tikzcd}
                {X_0} & {X_1} & {Y} & {\Sigma(X_0)} \\
                {X_0} & {X_1} & {X_2} & {X_3}
                \arrow["{f_1}", from=1-1, to=1-2]
                \arrow[from=1-2, to=1-3]
                \arrow[from=1-3, to=1-4]
                \arrow[Rightarrow, no head, from=1-1, to=2-1]
                \arrow[Rightarrow, no head, from=1-2, to=2-2]
                \arrow[from=1-3, to=2-3]
                \arrow["\psi", from=1-4, to=2-4]
                \arrow["{f_1}", from=2-1, to=2-2]
                \arrow["{f_2}", from=2-2, to=2-3]
                \arrow["{f_3}", from=2-3, to=2-4]
            \end{tikzcd}\]

            Where the top row is distinguished.
        }
    \item {
            Fiber cofiber Toda bracket:

            Every composite \( \beta \circ \Sigma(\alpha) \)that makes the following diaram commute:

            % https://q.uiver.app/#q=WzAsOCxbMCwwLCJYXzAiXSxbMSwwLCJYXzEiXSxbMSwxLCJYXzEiXSxbMiwxLCJYXzIiXSxbMiwyLCJYXzIiXSxbMywyLCJYXzMiXSxbMywxLCJDX3tmXzJ9Il0sWzAsMSwiXFxTaWdtYV57LTF9KENfe2ZfMn0pIl0sWzAsMSwiZl8xIl0sWzEsMiwiIiwwLHsibGV2ZWwiOjIsInN0eWxlIjp7ImhlYWQiOnsibmFtZSI6Im5vbmUifX19XSxbMiwzLCJmXzIiXSxbMyw0LCIiLDAseyJsZXZlbCI6Miwic3R5bGUiOnsiaGVhZCI6eyJuYW1lIjoibm9uZSJ9fX1dLFs0LDUsImZfMyJdLFszLDZdLFs3LDJdLFswLDcsIlxcYWxwaGEiLDJdLFs2LDUsIlxcYmV0YSIsMl1d
            \[\begin{tikzcd}
                {X_0} & {X_1} \\
                {\Sigma^{-1}(Y)} & {X_1} & {X_2} & {Y} \\
                && {X_2} & {X_3}
                \arrow["{f_1}", from=1-1, to=1-2]
                \arrow[Rightarrow, no head, from=1-2, to=2-2]
                \arrow["{f_2}", from=2-2, to=2-3]
                \arrow[Rightarrow, no head, from=2-3, to=3-3]
                \arrow["{f_3}", from=3-3, to=3-4]
                \arrow[from=2-3, to=2-4]
                \arrow[from=2-1, to=2-2]
                \arrow["\alpha"', from=1-1, to=2-1]
                \arrow["\beta"', from=2-4, to=3-4]
            \end{tikzcd}\]

            Where the middle row is distinguished.
        }
    \item {
        Iterated fiber Toda bracket:

        Every map \( \Sigma(\delta) \), where \( \delta \) makes the following diagram commute:

        % https://q.uiver.app/#q=WzAsOCxbMCwwLCJYXzAiXSxbMSwwLCJYXzEiXSxbMiwwLCJYXzIiXSxbMywwLCJYXzMiXSxbMiwxLCJYXzIiXSxbMywxLCJYXzMiXSxbMCwxLCJcXFNpZ21hXnstMX0oWF8zKSJdLFsxLDEsIkNfe2ZfM30iXSxbMCwxLCJmXzEiXSxbMSwyLCJmXzIiXSxbMiwzLCJmXzMiXSxbNCw1LCJmXzMiXSxbNyw0XSxbNiw3XSxbMCw2LCJcXGRlbHRhIiwyXSxbMSw3XSxbMiw0LCIiLDEseyJsZXZlbCI6Miwic3R5bGUiOnsiaGVhZCI6eyJuYW1lIjoibm9uZSJ9fX1dLFszLDUsIiIsMSx7ImxldmVsIjoyLCJzdHlsZSI6eyJoZWFkIjp7Im5hbWUiOiJub25lIn19fV1d
        \[\begin{tikzcd}
            {X_0} & {X_1} & {X_2} & {X_3} \\
            {\Sigma^{-1}(X_3)} & {Y} & {X_2} & {X_3}
            \arrow["{f_1}", from=1-1, to=1-2]
            \arrow["{f_2}", from=1-2, to=1-3]
            \arrow["{f_3}", from=1-3, to=1-4]
            \arrow["{f_3}", from=2-3, to=2-4]
            \arrow[from=2-2, to=2-3]
            \arrow[from=2-1, to=2-2]
            \arrow["\delta"', from=1-1, to=2-1]
            \arrow[from=1-2, to=2-2]
            \arrow[Rightarrow, no head, from=1-3, to=2-3]
            \arrow[Rightarrow, no head, from=1-4, to=2-4]
        \end{tikzcd}\]

        Where the bottom row is distinguished.
    }
    \end{enumerate}

    Which all gives rise to the same subset of \( \Hom_{\Tc}(\Sigma(X_0), X_3) \).
\end{definition}

\begin{remark}
    Note that every \( Y \) in the above definition is isomorphic to a cone of the respective map.
\end{remark}