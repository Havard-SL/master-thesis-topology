\section{What is Toda Brackets?}

Besed on Christensen \& Frankland article (TODO).

\begin{definition}
    Given a triangulated category \( \Tc \), given the following diagram in \( \Tc \):

    \begin{center}
        \begin{tikzpicture}
            \diagram{m}{1cm}{1cm} {
                {X_0} & {X_1} & {X_2} & {X_3} \\
            };

            \draw[math]
                (m-1-1) edge node {f_1} (m-1-2)
                (m-1-2) edge node {f_2} (m-1-3)
                (m-1-3) edge node {f_3} (m-1-4);
        \end{tikzpicture}
    \end{center}

    one can define the \emph{three-fold Toda bracket of \( f_3, f_2, f_1 \)}, denoted \( \langle f_3, f_2, f_1 \rangle \), a subset of \( \Tc(\Sigma(X_0), X_3) \) in three different ways: \sloppy

    \begin{enumerate}
        \item {
            Iterated cofiber Toda bracket:

            Every possible \( \psi \) that makes the following diagram commute:

            \begin{center}
                \begin{tikzpicture}
                    \diagram{m}{1cm}{1cm} {
                        {X_0} & {X_1} & {Y} & {\Sigma(X_0)} \\
                        {X_0} & {X_1} & {X_2} & {X_3} \\
                    };

                    \draw[math]
                        (m-1-1) edge node {f_1} (m-1-2)
                            edge[equal] (m-2-1)
                        (m-1-2) edge (m-1-3)
                            edge[equal] (m-2-2)
                        (m-1-3) edge (m-1-4)
                            edge (m-2-3)
                        (m-1-4) edge node {\psi} (m-2-4)

                        (m-2-1) edge node {f_1} (m-2-2)
                        (m-2-2) edge node {f_2} (m-2-3)
                        (m-2-3) edge node {f_3} (m-2-4);
                \end{tikzpicture}
            \end{center}

            Where the top row is distinguished.
        }
    \item {
            Fiber cofiber Toda bracket:

            Every composite \( \beta \circ \Sigma(\alpha) \)that makes the following diaram commute:

            \begin{center}
                \begin{tikzpicture}
                    \diagram{m}{1cm}{1cm} {
                        {X_0} & {X_1} \\
                        {\Sigma^{-1}(Y)} & {X_1} & {X_2} & {Y} \\
                        && {X_2} & {X_3} \\
                    };

                    \draw[math]
                        (m-1-1) edge node {f_1} (m-1-2)
                            edge node {\alpha} (m-2-1)
                        (m-1-2) edge[equal] (m-2-2)

                        (m-2-1) edge (m-2-2)
                        (m-2-2) edge node {f_2} (m-2-3)
                        (m-2-3) edge (m-2-4)
                            edge[equal] (m-3-3)
                        (m-2-4) edge node {\beta} (m-3-4)

                        (m-3-3) edge node {f_3} (m-3-4);
                \end{tikzpicture}
            \end{center}

            Where the middle row is distinguished.
        }
    \item {
        Iterated fiber Toda bracket:

        Every map \( \Sigma(\delta) \), where \( \delta \) makes the following diagram commute:

        \begin{center}
            \begin{tikzpicture}
                \diagram{m}{1cm}{1cm} {
                    {X_0} & {X_1} & {X_2} & {X_3} \\
                    {\Sigma^{-1}(X_3)} & {Y} & {X_2} & {X_3} \\
                };

                \draw[math]
                    (m-1-1) edge node {f_1} (m-1-2)
                        edge node {\delta} (m-2-1)
                    (m-1-2) edge node {f_2} (m-1-3)
                        edge (m-2-2)
                    (m-1-3) edge node {f_3} (m-1-4)
                        edge[equal] (m-2-3)
                    (m-1-4) edge[equal] (m-2-4)

                    (m-2-1) edge (m-2-2)
                    (m-2-2) edge (m-2-3)
                    (m-2-3) edge node {f_3} (m-2-4);
            \end{tikzpicture}
        \end{center}

        Where the bottom row is distinguished.
    }
    \end{enumerate}

    Which all gives rise to the same subset of \( \Tc(\Sigma(X_0), X_3) \).
\end{definition}

\begin{remark}
    Note that every \( Y \) in the above definition is isomorphic to a cone of the respective map.
\end{remark}
