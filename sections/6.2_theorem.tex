Finally we can state the theorem that shows the equality between Massey products and Toda brackets on an algebraic triangulated category.

\begin{theorem}[Massey products = Toda brackets]
    \label{theorem:massey_equals_toda}
    Let \( \Tc \) be an algebraic triangulated category, and let the following be a diagram in \( \Tc \)
    \begin{center}
        \begin{tikzpicture}
            \diagram{m}{1cm}{1cm} {
                X_1 \& X_2 \& X_3 \& X_4. \\
            };

            \draw[math]
                (m-1-1) edge node {f_1} (m-1-2)
                (m-1-2) edge node {f_2} (m-1-3)
                (m-1-3) edge node {f_3} (m-1-4);
        \end{tikzpicture}
    \end{center}

    Then
    \[
        \toda{f_3, f_2, f_1} = \massey{f_3, f_2, f_1}.
    \]
\end{theorem}

Some knowledgeable readers might notice that we are mising a sign that other authors have.

\begin{remark}
    In \cite[Theorem 4.2.6]{Jasso-Muro_2023} they get that
    \[
        \toda{f_3, f_2, f_1} = -\massey{f_3, f_2, f_1}.
    \]

    A theory for why we don't get the same result may be found in the way that \cite{Jasso-Muro_2023} compares Toda brackets and Massey products, and how it differs from what is done in this thesis.

    First, while our definition of Massey product on \( H^{\bullet}(\Cc) \) (\autoref{def:massey_product_dg_cat}) is equal to \cite[Definition 4.2.1]{Jasso-Muro_2023} for \( d = 1 \), they use an isomorphism
    \[
        H^{-1}(\dgM(X_1, X_4)) \cong H^0(\dgM)(X_1, \Sigma^{-1} X_4)
    \]
    and then apply
    \[
        \Sigma
    \]
    to end up with a subset of
    \[
        \cong H^0(\dgM)(\Sigma X_1, X_4).
    \]
    With our approach, we don't have to use \( \Sigma \) at all by just using the isomorphism \autoref{cor:H^i_dgmod_cong_H^0_with_shift}, which is essentially the identity morphism, to compare Massey products with Toda brackets.

    In trying to understand the isomorphism,
    \[
        H^{-1}(\dgM(X_1, X_4)) \cong H^0(\dgM)(X_1, \Sigma^{-1} X_4)
    \]
    we get an alternate version of \autoref{lem:shift_one_component_inner_product_chain_complex}, with an isomorphism
    \[
        \phi: [A, B]_{\bullet - 1} \to [A, \Sigma^{-1} B]
    \]
    where \( \phi_i \) alternates between multiplying with \( 1 \) and \( -1 \).

    If we assume \( \phi_0 = -1 \), we end up with \( H^0(\phi) \) being multiplication with \( -1 \). If we assume \( \phi_0 = 1 \), we end up with \( H^0(\phi) \) being multiplication with \( 1 \). Applying \( \Sigma \) is essentially identity, because \( |\phi_0| = 0 \), and so the isomorphism \cite{Jasso-Muro_2023} use to compare Massey products and Toda brackets could be either multiplication with \( 1 \) or \( -1 \). If it is the latter, it could explain the sign difference.

    However, this seems unlikely as no mathematician in their right mind would choose the latter option when faced with such a choice, and it would have been mentioned.

    WIP sjekk beviset till Jasso-Muro
\end{remark}