We can now state the theorem that shows the equality between Massey products and Toda brackets on an algebraic triangulated category.

\begin{theorem}[Massey products = Toda brackets]
    \label{theorem:massey_equals_toda}
    Let \( \Tc \) be an algebraic triangulated category, and let the following be a diagram in \( \Tc \)
    \begin{center}
        \begin{tikzpicture}
            \diagram{m}{1cm}{1cm} {
                X_1 \& X_2 \& X_3 \& X_4. \\
            };

            \draw[math]
                (m-1-1) edge node {f_1} (m-1-2)
                (m-1-2) edge node {f_2} (m-1-3)
                (m-1-3) edge node {f_3} (m-1-4);
        \end{tikzpicture}
    \end{center}

    Then
    \[
        \toda{f_3, f_2, f_1} = \massey{f_3, f_2, f_1}.
    \]
\end{theorem}

Some knowledgeable readers might notice that we are missing a sign that other authors have.

\begin{remark}
    In \cite[Theorem 4.2.6]{Jasso-Muro_2023} they get that
    \[
        \toda{f_3, f_2, f_1} = -\massey{f_3, f_2, f_1}.
    \]

    There are two theories for why we don't get the same result.
    
    The first theory relates to how \cite{Jasso-Muro_2023} compares Toda brackets and Massey products, and how it differs from what is done in this thesis. Our definition of Massey products on \( H^{\bullet}(\Cc) \) (\autoref{def:massey_product_dg_cat}) is equal to \cite[Definition 4.2.1]{Jasso-Muro_2023} for \( d = 1 \). Similarly, their definition of Toda brackets (\cite[Definition 4.1.6]{Jasso-Muro_2023}) is equal to our fiber-cofiber definition of Toda brackets (\autoref{def:toda_bracket}).
    
    However, when comparing the Toda brackets and Massey products in \( H^0(\dgM) \), they use an isomorphism
    \[
        H^{-1}(\dgM(X_1, X_4)) \cong H^0(\dgM)(X_1, \Sigma^{-1} X_4)
    \]
    and then apply \( \Sigma \) to end up with a subset of
    \[
        H^0(\dgM)(\Sigma X_1, X_4).
    \] 

    With our approach, we don't have to use \( \Sigma \) at all by just using the isomorphism \autoref{cor:H^i_dgmod_cong_H^0_with_shift}, which is essentially the identity morphism, to compare Massey products with Toda brackets.

    In trying to understand the isomorphism,
    \[
        H^{-1}(\dgM(X_1, X_4)) \cong H^0(\dgM)(X_1, \Sigma^{-1} X_4)
    \]
    we see that it boils down to an alternate version of \autoref{lem:shift_one_component_inner_product_chain_complex}, with some isomorphism
    \[
        \phi: [A, B]_{\bullet - 1} \to [A, \Sigma^{-1} B].
    \]
    By calculating, we get that there is an obvious choice of \( \phi \) where \( \phi_i \) alternates between multiplying with \( 1 \) and \( -1 \).

    If we assume \( \phi_0 = -1 \), we end up with \( H^0(\phi) \) being multiplication with \( -1 \). If we assume \( \phi_0 = 1 \), we end up with \( H^0(\phi) \) being multiplication with \( 1 \). Applying \( \Sigma \) is essentially identity, because \( |\phi_0| = 0 \), and so the isomorphism \cite{Jasso-Muro_2023} use to compare Massey products and Toda brackets could be either multiplication with \( 1 \) or \( -1 \). If it is the latter, it could explain the sign difference.

    However, this seems unlikely as no mathematician in their right mind would choose the latter option when faced with such a choice, and it would have been mentioned.

    The second, and more likely theory, is an error in our upcoming proof of \autoref{thm:dgm_massey_equal_toda}, or in \cite[Proposition 4.2.8]{Jasso-Muro_2023}. The reason for this is that, assuming they use the \( H^0(\phi) \) mentioned earlier that is multiplication with \( 1 \), the proof itself, along with every other prerequisite seem correct in the three-fold case.

    In \cite[Proposition 4.2.8]{Jasso-Muro_2023}, restricted to the three-fold case and converted to our definitions, they essentially want to show that
    \[
        \massey{f_3, f_2, f_1} = \union_{\beta \circ (\Sigma \alpha) \in \toda{f_3, f_2, f_1}_{fc}}  \class*{- \beta \circ (\Sigma \alpha)}.
    \]
    However, we can see that this does not agree with our result, since
    \[
        \union_{\beta \circ (\Sigma \alpha) \in \toda{f_3, f_2, f_1}_{fc}}  \class*{- \beta \circ (\Sigma \alpha)} = -\toda{f_3, f_2, f_1}.
    \]

    Their argument starts with some \( g_1, g_2, g_3, s, \) and \( t \) from the definition of the Massey product for some element \( f \in \massey{f_3, f_2, f_1} \). Then they construct \( -\alpha \) and \( -\beta \) in the same manner as we do, and then they verify that \( (-\beta) \circ (\Sigma -\alpha) = \beta \circ (\Sigma \alpha) \in \toda{f_3, f_2, f_1} \), just as we do.

    However, then they make what we can assume to be the error in their proof. They don't verify that \( f = \class*{- \beta \circ (\Sigma \alpha)} \), which would contradict the result we get in our proof.

    Then they finish up the proof by stating that by starting with some \( \beta \) and \( \Sigma \alpha \), they can construct \( g_1, g_2, g_3, s, \) and \( t \), from the definition of the Massey product for some element \( f \in \massey{f_3, f_2, f_1} \). Due to the lack of specificity in this part of their proof, we can not be sure they construct \( g_1, g_2, g_3, s, \) and \( t \) in the same manner we do, but most likely they do. However, again they do not show why \( \class*{- \beta \circ (\Sigma \alpha)} = f \), which might have contradicted our result.
\end{remark}