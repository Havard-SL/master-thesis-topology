The chain homotopy category is one of the simplest triangulated categories to define, and is a prototypical example of an algebraic triangulated category. The definition of an algebraic triangulated category will be given later in the thesis in \autoref{def:alg_tri_cat}. It will become clear from the definition of the chain homotopy category that it is an algebra inspired triangulated category, and so it fits (morally), as an algebraic triangulated category.

First, we define the category of chain complexes.

\begin{definition}[Chain complex, \( \C(\Cc) \)]
    \label{def:chain_complex}
    Let \( \Cc \) be any additive category.
    
    Then let \( \C \tuple{\Cc} \) denote the category of chain complexes of objects in \( \Cc \).

    As mentioned in \autoref{not:chain_complex}, the chain complexes have ascending order, and differentials are indexed according to the order of their domain.
\end{definition}

We start off by defining the category, and will later define the triangulation.

% MS-Question: Ikkje vist at dette er ein kategori...
\begin{definition}[Chain homotopy category, \( \K(\Ac) \)]
    \label{def:chain_homotopy_cat}
    Let \( \Ac \) be an abelian category.

    Then let \( \K(\Ac) \) be the following category:
    \begin{enumerate}
        \item {
            Objects in \( \K(\Ac) \) are chain complexes.
        }
        \item {
            Morphisms in \( \K(\Ac) \) are equivalence classes of chain morphisms up to chain homotopy.
        }
        \item {
            Composition in \( \K(\Ac) \) is the equivalence class of the composition of the representatives of the morphisms, i.e.,
            \[
                [g] \circ [f] := [g \circ f ].
            \]
        }
    \end{enumerate}

    Then \( \K(\Ac) \) is called the \emph{chain homotopy category} over \( \Ac \).
\end{definition}

\( \K(\Ac) \) can be shown to be an additive category.

On \( \C(\Ac) \), we can define a simple automorphism of categories called the shift functor.

\begin{definition}[Shift functor on \( \C(\Ac) \)]
    \label{def:chain_complex_shift}
    Let \( \Sigma \) be the following assignment of objects and morphisms in \( \C(\Ac) \).

    Let \( A = \tuple{A_i}_{i \in \Zb} \in \C(\Ac) \) be a chain complex, and let \( d_{A, n}: A_n \to A_{n + 1} \) denote the \( n \)th differential of \( A \).

    Then let \( \Sigma A := \tuple{A_{i + 1}}_{i \in \Zb} \), with \( d_{\Sigma A, n} := -d_{A, n + 1} \).

    In addition, let \( f = \tuple{f_i}_{i \in \Zb} \in \C(\Ac)(A, B) \) be a chain morphism.
    
    Then
    \[
        \Sigma(f) := \tuple{f_{i + 1}}_{i \in \Zb} \in \C(\Ac)(\Sigma A, \Sigma B).
    \]
\end{definition}

\( \Sigma \) can be shown to be an additive automorphism on \( \C(\Ac) \), and can be induced to an additive automorphism with the same notation and name on \( \K(\Ac) \).

Finally, we need to define what the cone of a morphism in \( \K(\Ac) \) is. This turns out to also be inherited from \( \C(\Ac) \).

\begin{definition}[Cone in \( \C(\Ac) \)]
    Let \( f \in \C(\Ac)(A, B) \).

    Then \( C_f \) is the following chain complex
    \begin{center}
        \mmznext{disable}
        \begin{tikzpicture}
            \diagram{m}{1cm}{1.7cm} {
                \cdots \&[-1cm] A_0 \oplus B_{-1} \& A_1 \oplus B_0 \& A_2 \oplus B_1 \&[-1cm] \cdots \\
            };

            \draw[math]
                (m-1-1) edge (m-1-2)
                (m-1-2) edge node {
                    \begin{psmallmatrix}
                        -d_{A, 0} & 0 \\
                        f_0 & d_{B, -1}
                    \end{psmallmatrix}
                } (m-1-3)
                (m-1-3) edge node {
                    \begin{psmallmatrix}
                        -d_{A, 1} & 0 \\
                        f_1 & d_{B, 0}
                    \end{psmallmatrix}
                } (m-1-4)
                (m-1-4) edge (m-1-5);
        \end{tikzpicture}
    \end{center}
    called the \emph{cone of \( f \)} in \( \C(\Ac) \).
\end{definition}

It is easy to verify that \( C_f \) is a chain complex, and we can also verify that if we didn't have the sign in front of \( d_{A, i + 1} \), or  \( d_{B, i} \), then it would not be a chain complex.

We can think of the cone as having the underlying objects of \( \Sigma A \oplus B \), but with \( f \) ``gluing'' the two components together such that the chain complex can't be written as a direct sum.

\begin{remark}
    Let \( \Ac \) be an abelian category.

    Then the cone fits into the following level-wise short exact sequence in \( \C(\Ac) \),
    \begin{center}
        \begin{tikzpicture}
            \diagram{m}{1cm}{1cm} {
                B \& C_f \& \Sigma A. \\
            };

            \draw[math]
                (m-1-1) edge node {\iota} (m-1-2)
                (m-1-2) edge node {\pi} (m-1-3);
        \end{tikzpicture}
    \end{center}
    where
    \[
        \iota_i := \iota_{B, i}: B_i \to (\Sigma A)_i \oplus B_i,
    \]
    and
    \[
        \pi_i := \pi_{\Sigma A, i}: (\Sigma A)_i \oplus B_i \to (\Sigma A)_i.
    \]

    These are chain morphisms because
    \[
        d_i \circ \iota_i =
        \begin{pmatrix}
            -d_{A, i + 1} & 0 \\
            f_{i + 1} & d_{B, i}
        \end{pmatrix}
        \begin{pmatrix}
            0 \\
            1
        \end{pmatrix}
        =
        \begin{pmatrix}
            0 \\
            d_{B, i}
        \end{pmatrix}
        =
        \iota_{C_f, i + 1} \circ d_{B, i}
    \]
    and
    \[
        \pi_i \circ d_i =
        \begin{pmatrix}
            1 & 0
        \end{pmatrix}
        \begin{pmatrix}
            -d_{A, i + 1} & 0 \\
            f_{i + 1} & d_{B, i}
        \end{pmatrix}
        =
        \begin{pmatrix}
            -d_{A, i + 1} & 0
        \end{pmatrix}
        = d_{\Sigma A, i} \circ \pi_i.
    \]
\end{remark}

The above remark is the reason why we have a sign in front of the differential in \( \Sigma A \) compared to \( A \). If we didn't have the sign, then \( \pi \) would not be a chain morphism.

The distinguished triangles are the triangles which seem to satisfy just the first triangulated axiom, {\bf (TR1)}, but ends up satisfying every axiom.

\begin{definition}
    \label{def:chain_homotopy_dist}
    Let \( \Delta \) be the collection of triangles in \( \K(\Ac) \) consisting of triangles that are isomorphic to any triangle of the form

    \begin{center}
        \begin{tikzpicture}
            \diagram{m}{1cm}{1cm} {
                A \& B \& C_f \& \Sigma A. \\
            };

            \draw[math]
                (m-1-1) edge node {f} (m-1-2)
                (m-1-2) edge node {\iota} (m-1-3)
                (m-1-3) edge node {\pi} (m-1-4);
        \end{tikzpicture}
    \end{center}
\end{definition}

Combining, \( \K(\Ac), \Sigma, \) and \( \Delta \) we get our second example of a triangulated category.

\begin{example}
    Let \( \K(\Ac) \) be as in \autoref{def:chain_homotopy_cat}, let \( \Sigma \) be as in \autoref{def:chain_complex_shift}, and let \( \Delta \) be as in \autoref{def:chain_homotopy_dist}.

    Then \( \tuple{\K(\Ac), \Sigma, \Delta} \) is a triangulated category.
\end{example}
For a proof of \( \K(\Ac) \) being triangulated, see \cite[Proposition 3.5.25]{Zimmermann_2014}.

By \cite[Subsection 7.5]{Krause_2007}, \( \K(\Ac) \) is an ``algebraic triangulated category,'' which we will define in \autoref{section:alg_tri_cats}.

Similar to the Spanier--Whitehead category, the triangulated category terminology of a \emph{shift functor} is derived from the chain homotopy category. We can see from \autoref{def:chain_complex_shift} that \( \Sigma \) takes the chain complex and shifts it to the left (given that differentials are pointing to the right). The terminology used for triangulated categories can be seen as a mixture of terms from topology and algebra, which fits well since it sits right in the intersection between the two fields.
