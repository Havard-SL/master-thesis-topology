We can now state the main theorem of the thesis describing the equality between Massey products and Toda brackets for algebraic triangulated categories.

\begin{theorem}[Massey products = Toda brackets]
    \label{theorem:massey_equals_toda}
    Let \( \Tc \) be an algebraic triangulated category, and let the following be a diagram in \( \Tc \)
    \begin{center}
        \begin{tikzpicture}
            \diagram{m}{1cm}{1cm} {
                X_1 \& X_2 \& X_3 \& X_4. \\
            };

            \draw[math]
                (m-1-1) edge node {f_1} (m-1-2)
                (m-1-2) edge node {f_2} (m-1-3)
                (m-1-3) edge node {f_3} (m-1-4);
        \end{tikzpicture}
    \end{center}

    Then
    \[
        \toda{f_3, f_2, f_1} = \massey{f_3, f_2, f_1}.
    \]
\end{theorem}

Some knowledgeable readers might notice that we are missing a sign that other authors have.

\begin{remark}
    In \cite[Theorem 4.2.6]{Jasso-Muro_2023} they get that
    \[
        \toda{f_3, f_2, f_1} = -\massey{f_3, f_2, f_1}.
    \]

    Here are two explanations for why we do not get the same result.
    
    The first explanation relates to how \cite{Jasso-Muro_2023} compares Toda brackets and Massey products, and how it differs from what is done in this thesis. Our definition of Massey products on \( H^{\bullet}(\Cc) \) (\autoref{def:massey_product_dg_cat}) is equal to \cite[Definition 4.2.1]{Jasso-Muro_2023} for \( d = 1 \). Similarly, their definition of Toda brackets (\cite[Definition 4.1.6]{Jasso-Muro_2023}) is equal to our fiber-cofiber definition of Toda brackets (\autoref{def:toda_bracket}).
    
    However, when comparing the Toda brackets and Massey products in \( H^0(\dgM) \), they use an isomorphism
    \[
        H^{-1}(\dgM(X_1, X_4)) \cong H^0(\dgM)(X_1, \Sigma^{-1} X_4)
    \]
    and then apply \( \Sigma \) to end up with a subset of
    \[
        H^0(\dgM)(\Sigma X_1, X_4).
    \] 

    With our approach, we do not have to use \( \Sigma \) at all by just using the isomorphism \autoref{cor:H^i_dgmod_cong_H^0_with_shift}, which is essentially the identity morphism, to compare Massey products with Toda brackets.

    The isomorphism,
    \[
        H^{-1}(\dgM(X_1, X_4)) \cong H^0(\dgM)(X_1, \Sigma^{-1} X_4)
    \]
    boils down to an alternate version of \autoref{lem:shift_one_component_inner_product_chain_complex}, with some isomorphism
    \[
        \phi: [A, B]_{\bullet - 1} \to [A, \Sigma^{-1} B].
    \]
    There is an obvious choice of \( \phi \) where \( \phi_i \) alternates between multiplying with \( 1 \) and \( -1 \).

    If we assume \( \phi_0 = -1 \), we end up with \( H^0(\phi) \) being multiplication with \( -1 \). If we assume \( \phi_0 = 1 \), we end up with \( H^0(\phi) \) being multiplication with \( 1 \). Applying \( \Sigma \) is essentially the identity, because \( |\phi_0| = 0 \), and so the isomorphism \cite{Jasso-Muro_2023} use to compare Massey products and Toda brackets could be either multiplication with \( 1 \) or \( -1 \). If it is the latter, it could explain the sign difference.

    The second explanation, is an error in our upcoming proof of \autoref{thm:dgm_massey_equal_toda}, or a misunderstanding of \cite[Proposition 4.2.8]{Jasso-Muro_2023}. The reason for this is that, assuming they use the \( H^0(\phi) \) mentioned earlier that is multiplication with \( 1 \), the proof itself, along with every other prerequisite seem correct in the three-fold case.

    In \cite[Proposition 4.2.8]{Jasso-Muro_2023}, restricted to the three-fold case and converted to our definitions, they seem to show that
    \[
        \massey{f_3, f_2, f_1} = \bigcup_{\beta \circ (\Sigma \alpha) \in \toda{f_3, f_2, f_1}_{fc}}  \class*{- \beta \circ (\Sigma \alpha)}.
    \]
    However, we can see that this does not agree with our result, since
    \[
        \bigcup_{\beta \circ (\Sigma \alpha) \in \toda{f_3, f_2, f_1}_{fc}}  \class*{- \beta \circ (\Sigma \alpha)} = -\toda{f_3, f_2, f_1}.
    \]

    Their argument starts with some \( g_1, g_2, g_3, s, \) and \( t \) from the definition of the Massey product for some element \( f \in \massey{f_3, f_2, f_1} \). Then they construct \( -\alpha \) and \( -\beta \) in the same manner as we do, and then they verify that \( (-\beta) \circ (\Sigma -\alpha) = \beta \circ (\Sigma \alpha) \in \toda{f_3, f_2, f_1} \), just as we do.

    However, then they seem to differ from our proof. It seems like they do not directly verify that \( f = \class*{- \beta \circ (\Sigma \alpha)} \), which would contradict the result we get in our proof.

    Then they finish up the proof by stating that starting with some \( -\beta \) and \( -\Sigma \alpha \), for some element \( f \in \toda{f_3, f_2, f_1} \), they can construct \( g_1, g_2, g_3, s, \) and \( t \), from the definition of the Massey product. It is not clear exactly how they would construct \( g_1, g_2, g_3, s, \) and \( t \), so this could also be a source of misunderstanding which, together with the argument above, could yield a sign difference in the final result.
\end{remark}
