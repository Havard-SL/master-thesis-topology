Toda brackets and Massey products are closely linked to triangulated categories. In reality, a triangulated category is a purely algebraic construction, but in essence it is closely linked with many topological structures that appear in the wild. (TODO: Eksempel på topologiske konstruksjonar?)

A triangulated category is an additive category with some additional data. Firstly, it has an auto-equivalence onto itself, called the shift functor, usually denoted as \( \Sigma \). Secondly it has a collection of diagrams
\begin{center}
    \begin{tikzpicture}
        \diagram{m}{1cm}{1cm} {
            X_1 \& X_2 \& X_3 \& \Sigma(X_1) \\
        };

        \draw[math]
            (m-1-1) edge node {f_1} (m-1-2)
            (m-1-2) edge node {f_2} (m-1-3)
            (m-1-3) edge node {f_3} (m-1-4);
    \end{tikzpicture}
\end{center}
called distinguished triangles. And thirdly, the category along with the shift functor as well as the distinguished triangles satisfy four axioms, called T1, T2, T3 and T4. WIP

In addition to triangulated categories, the notion of enriched categories is also important in this thesis, especially around the definition of a massey product and a DG-category. But before one can understand what an enriched category is, it is important to understand what a symmetric closed monoidal category is. In essence, it's a category that formalizes the notion and properties of ``tensor products'' which appear repeatedly in homological algebra. An example of a symmetric closed monoidal category is the category of modules over a commutative ring with identity. To be more specific, a symmetric closed monoidal category satisfies three properties:

Firstly, it's a ``monoidal category'', which means that the category contains the additional data of a ``tensor product'' which is a bifunctor from the category squared onto itself. In addition, the category also has a ``tensor unit'', which is an object in the category that, as the name implies, creates a natural isomorphism from the tensor product of the unit (on either the left or right hand side) onto the category itself. These natural isomorphisms are denoted the left and right ntural isomorphisms, and is also additional data that makes the category ``monoidal''. Finally to make the category monoidal, there is the data of a natural isomorphism called the associator, which makes the tensor product satisfy the usual associativity property. The data also has to satisfy two commutative diagrams to ensure that associativity works as expected with the natural left and right unit isomorphisms, as well as to ensure that associativity works well with itself. These commutative diagrams are shown in full in \cite[Diagram 6.1, Diagram 6.2]{Borceux_1994}.

Secondly, a symmetric closed monoidal category is a ``symmetric monoidal category'', which as the name suggests, is a monoidal category, but where the tensor product is commutative as in there is a ``symmetry natural isomorphism'' that satsify two properties. Firstly, applying the symmetry isomorphism twice yields the original value, and secondly, the symmetry isomorphisms and the associativity natural isomorphisms commute in the expected way. Again, the specific commutative diagrams is shown in \cite[Diagram 6.3, Diagram 6.4, Diagram 6.5]{Borceux_1994}.

Thirdly, a symmetric closed monoidal category has the additional data of an ``internal hom'' bifunctor from the category squared onto the category itself, and where it is contravariant in the first component, but covariant in the second. This internal hom functor is the right adjoint of the tensor product in a similar way as how the tensor product is left adjoint to the hom functor in the category of modules over a commutative ring with identity.

This thesis leans heavily on \cite[Chapter 6]{Borceux_1994} for the defiition and results regarding enriched category theory, including the definition of a symmetric closed monoidal categories.

An enriched category is not a ``category'' in the usual sense, but a similar and in some cases more general notion of another ``type'' of category. Instead of saying that morphisms in a category is a class, one can instead consider morphism spaces as objects in a symmetric monoidal closed category instead. This leads to a richer theory that can more explicitly connect category theory properties with the properties of it's morphism spaces. In many cases, the theory of enriched categories already explain existing categories which are important in e.g. homological algebra. Some examples of categories that are already ``enriched'' are: locally small categories (where morphism spaces are sets), preadditive categories (where the morphism spaces are abelian groups) and more generally, modules over a commutative ring with identity (where the morphism spaces are modules over the same ring). In fact, the latter is both a symmetric closed monoidal category as well as a category enriched over itself! This is partly because the internal hom is equal to the usual hom functor in the category of modules over a commutatuive ring with identity.

