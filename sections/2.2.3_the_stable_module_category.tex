The stable module category is the category that is going to be used the most in this thesis, so it will be given with more details than the previous two examples.

Before showing the triangulated structure, it is as usual necessary to define the underlying category first.

\begin{definition}
    \label{def:stable_module_category}
    Let \( R \) be a \emph{frobenius algebra}.

    Then the \emph{stable module category over \( R \)}, denoted \( \StMod(R) \) is defined in the following way:
    \begin{enumerate}
        \item {
            The objects are infinitely generated modules over \( R \).
        }
        \item {
            For two modules \( A, B \) in \( \Mod(R) \), let \( G \) be the subset of \( \Mod(R)(A, B) \) consisting of module morphisms that factor through a projective object. By \autoref{lem:morphisms_factoring_through_projectives_r-submodule} this is a submodule of \( \Mod(R)(A, B) \).
            
            Then
            \[
                \StMod(A, B) := \Mod(R)/G
            \]
        }
        \item {
            For two morphisms \( [f] \in \StMod(B, C) \), and \( [g] \in \StMod(A, B) \), let composition be defined as follows
            \[
                [f] \circ [g] := [f \circ g].
            \]
            This is well defined by \autoref{lem:stmod_composition_well-defined}.
        }
    \end{enumerate}

    This category is called the \emph{infinitely generated stable module category over \( R \)}.
\end{definition}

In the usual definition of the infinitely generated stable module category, it is not usually required that the ring be a frobenius algebra. However, as will become apparent later in this subsection is that it is a requirement to having a triangulated structure defined on it. Since this thesis will only be considering the stable module category as a triangulated category, this assumption is made from the beginning.

Here are the two lemmas used in the preceding definition. First is the result that makes morphisms in \( \StMod(R) \) well-defined.

\begin{lemma}
    \label{lem:morphisms_factoring_through_projectives_r-submodule}
    Let \( G \) be the subset of \( \Mod(R)(A, B) \) consisting of module morphisms that factor through a projective object.

    Then \( G \) is an \( R \)-submodule of \( \Mod(R)(A, B) \).
\end{lemma}
\begin{proof}
    Let \( f \) and \( g \) be two maps that factor through the projectives \( P \) and \( Q \) respectively. Then we have the following diagrams.

    \begin{center}
        \begin{tikzpicture}
            \diagram{m}{1cm}{1cm} {
                A \& P \& B \\
            };

            \draw[math]
                (m-1-1) edge node {f_1} (m-1-2)
                (m-1-2) edge node {f_2} (m-1-3);
        \end{tikzpicture}
    \end{center}
    where \( f_2 \circ f_1 = f \), and
    \begin{center}
        \begin{tikzpicture}
            \diagram{m}{1cm}{1cm} {
                A \& Q \& B, \\
            };

            \draw[math]
                (m-1-1) edge node {g_1} (m-1-2)
                (m-1-2) edge node {g_2} (m-1-3);
        \end{tikzpicture}
    \end{center}
    where \( g_2 \circ g_1 = g \).

    One can then construct the morphism
    \begin{center}
        \begin{tikzpicture}
            \diagram{m}{1cm}{1cm} {
                A \& {P \oplus Q} \& B. \\
            };

            \draw[math]
                (m-1-1) edge node {
                    \begin{psmallmatrix}
                        f_1 \\
                        g_1
                    \end{psmallmatrix}
                } (m-1-2)
                (m-1-2) edge node {(f_2, g_2)} (m-1-3);
        \end{tikzpicture}
    \end{center}

    Composing these two maps, one gets the map \( f_2 \circ f_1 + g_2 \circ g_1 = f + g \). This maps factors thorugh \( P \oplus Q \), which is projective since it's a direct sum of projective modules.

    Therefore, the set of homomorphisms that factor through a projective is closed under addition. And multiplying with a ring element still factors through the same projective, since every morphism is an \( R \) homomorphism. Therefore the set of maps that factor through a projective is an \( R \) submodule.
\end{proof}

Second is the result that makes composition in \( \StMod(R) \) well defined.

\begin{lemma}
    \label{lem:stmod_composition_well-defined}
    Composition in \( \StMod(R) \) is well-defined.
\end{lemma}
\begin{proof}
    Need to check that two different choices of representatives of \( [f] \) and \( [g] \) yields the same composition.

    Let \( [f + \widetilde{f}] \) and \( [g + \widetilde{g}] \) be two different representatives of \( [f] \) and \( [g] \), with \( \widetilde{f} \) and \( \widetilde{g} \) factoring through some projectives.

    Then it follows that
    \[
        [f + \widetilde{f}] \circ [g + \widetilde{g}] = [f \circ g] + [\widetilde{f} \circ g] + [f \circ \widetilde{g}] + [\widetilde{f} \circ \widetilde{g}].
    \]
    
    However, every term other than \( [f \circ g] \), factors through a projective and is therefore equal to \( 0 \) in \( \StMod(R)(A, C) \).
\end{proof}

In order to be an underlying category of a triangulated category, a categoy needs to be additive.

\begin{lemma}
    \( \StMod(R) \) is an additive category.
\end{lemma}
\begin{proof}
    There are two parts to this proof. First want to show that \( \StMod(R) \) is pre-additive and then second, want to show that \( \StMod(R) \) has finite products which in addition to being pre-additive would imply that \( \StMod(R) \) is additive by TODO:Cite.
    % https://en.wikipedia.org/wiki/Additive_category

    To show that \( \StMod(R) \) is pre-additive there are two properties that need to be shown
    \begin{enumerate}
        \item {
            First, need to show that for any \( A, B \in \StMod(R) \) that \( \StMod(R)(A, B) \) is an abelian group.

            This follows immediately, since \( \StMod(R)(A, B) \) is the quotient of an \( R \)-module, and is therefore an \( R \)-module and hence an abelian group.
        }
        \item {
            Second, need to show that composition is bilinear.

            Consider \( [f], [f'] \in \StMod(R)(B, C) \) and \( [g], [g'] \in \StMod(R)(A, B) \). Look at the following equation
            \begin{align*}
                ([f] + [f']) \circ ([g] + [g']) &= [f + f'] \circ [g + g'] \\
                &= \class{(f + f') \circ (g + g')} \\
                &= \class{f \circ g + f \circ g' + f' \circ g + f' \circ g'} \\
                &= [f] \circ [g] + [f] \circ [g'] + [f'] \circ [g] + [f'] \circ [g'],
            \end{align*}
            which is exactly bilinearity.
        }
    \end{enumerate}

    The claim is that the usual biproduct in \( \Mod(R) \) is the product in \( \StMod(R) \). Let the following be the commutative diagram for the universal property of \( A \oplus B \) as a product in \( \Mod(R) \), but in \( \StMod(R) \) with respect to some object \( X \).
    \begin{center}
        \begin{tikzpicture}
            \diagram{m}{1.5cm}{1.5cm} {
                \& X \\
                A \& A \oplus B \& B \\
            };

            \draw[math]
                (m-1-2) edge node[swap] {[f_A]} (m-2-1)
                    edge[dashed] node {[f]} (m-2-2)
                    edge node {[f_B]} (m-2-3)

                (m-2-2) edge node[swap] {[\pi_B]} (m-2-3)
                    edge node {[\pi_A]} (m-2-1);
        \end{tikzpicture}
    \end{center}
    Let \( [g]: X \to A \times B \) be another morphism that satisfies the universal property.

    Then the following diagram commutes.
    \begin{center}
        \begin{tikzpicture}
            \diagram{m}{1.5cm}{1.5cm} {
                \& X \\
                A \& A \oplus B \& B \\
            };

            \draw[math]
                (m-1-2) edge node[swap] {[0]} (m-2-1)
                    edge[dashed] node {[f - g]} (m-2-2)
                    edge node {[0]} (m-2-3)

                (m-2-2) edge node[swap] {[\pi_B]} (m-2-3)
                    edge node {[\pi_A]} (m-2-1);
        \end{tikzpicture}
    \end{center}

    Let \( \iota_A, \iota_B \) denote the canonical split monomorphisms from the universal property of the coproduct. Since \( A \oplus B \) is a biproduct in \( \Mod(R) \) one has the known fact that
    \[
        \Id_{A \oplus B} = \iota_A \circ \pi_A + \iota_B \circ \pi_B.
    \]
    Consider the following equation,
    \begin{align*}
        [f - g] &= [\Id_{A \oplus B} \circ (f - g)] \\
        &= [\iota_A \circ \pi_A \circ (f - g) + \iota_B \circ \pi_B \circ (f - g)] \\
        &= [\iota_A] \circ [\pi_A \circ (f - g)] + [\iota_B] \circ [\pi_B \circ (f - g)] \\
        &= [0].
    \end{align*}
    Which implies that \( [f] = [g] \), and that \( A \oplus B \) is a product in \( \StMod(R) \), and by the statement at the start, also a biproduct.
\end{proof}

This thesis starts by defining the functor \( \Omega \), which will be shown later to be the inverse of the shift functor.

\begin{definition}
    \label{def:stmod_omega}
    Since \( \Mod(R) \) has enough projectives, one can choose a projecive module \( P_A \) for every \( A \) with epimorphisms \( \pi_A \) from \( P_A \) to \( A \) for every \( A \). Fix a choice of \( P_A \) and \( \pi_A \) for every \( A \).

    Then define \( \Omega \) as the assignment of objects and morphisms in \( \StMod(R) \) as follows:
    \begin{itemize}
        \item For any object \( A \) in \( \StMod(R) \), let \( \Omega(A) := \ker(\pi_A) \).
        \item For any \( [f] \in \StMod(A, B) \), let \( \Omega([f]) \) be as explained in \autoref{rem:stmod_omega_f}.
    \end{itemize}

    By \autoref{lem:stmod_omega_endofunctor}, \( \Omega \) is an endofunctor on \( \StMod(R) \).
\end{definition}

Before showing the result that shows that \( \Omega \) is a functor, it is necessary to show the slightly convoluted definition of \( \Omega \) applied to a morphism.

\begin{remark}
    \label{rem:stmod_omega_f}
    \( \Omega([f]) \) is constructed as follows:

    Look at the following diagram in \( \Mod(R) \)
    \begin{center}
        \begin{tikzpicture}
            \diagram{m}{1cm}{1cm} {
                {\Omega(A)} \& {P_A} \& A \\
                {\Omega(B)} \& {P_B} \& B. \\
            };

            \draw[math]
                (m-1-1) edge[hook] node {\iota_A} (m-1-2)
                    edge node {\Omega'(f)} (m-2-1)
                (m-1-2) edge[two heads] node {\pi_A} (m-1-3)
                    edge node {p_f} (m-2-2)
                (m-1-3) edge node {f} (m-2-3)

                (m-2-1) edge[hook] node {\iota_B} (m-2-2)
                (m-2-2) edge[two heads] node {\pi_B} (m-2-3);
        \end{tikzpicture}
    \end{center}

    One has that for a morphism \( f: A \to B \), there exists a morphism \( p_f \) from the lifitng property of projective modules such that the right square commutes. Please note that this morphism is \emph{not neccesarily} unique.

    Furthermore, since
    \[
        \pi_B \circ p_f \circ \iota_A = f \circ \pi_A \circ \iota_A = f \circ 0 = 0,
    \]
    one has from the universal kernel property that there exists a unique morphism, dependant on the choice of \( p_f \), denoted as \( \Omega'(f) \) from \( \Omega(A) \) to \( \Omega(B) \) such that the left square in the above diagram commutes. Taking the equivalence class of this morphism with respect to morphisms factoring through projectctives yields the morphism defined by the functor, i.e. \( \Omega([f]) := \class{\Omega'(f)} \).

    It is shown in \autoref{lem:stmod_omega_f_is_well_defined} that \( \Omega([f]) \) is in fact unique and independent of the choice of representative, and \( \Omega \) is therefore a well-defined assignment of morphisms.
\end{remark}

The following is the lemma showing that \( \Omega \) is a well-defined assignment of morphisms.

\begin{lemma}
    \label{lem:stmod_omega_f_is_well_defined}
    \( \Omega \) is a well-defined assignment of morphisms.
\end{lemma}
\begin{proof}
    There are two things that need to be proven. Firstly, in the construction of \( \Omega'(f) \), one has to choose a morphism \( p_f \) from the projective property. Need to show that if one choses another projective morphism, that \( \Omega \) still yields the same morphism in \( \StMod(R) \). Second, one need to show that if \( f \sim g \), then \( \Omega'(f) \sim \Omega'(g) \).

    To prove the first part, let \( p_f \) and \( \widetilde{p_f} \) be two different projective morphisms that give the morphisms \( \Omega'(f) \) and \( \widetilde{\Omega'(f)} \) respectively. Then one has the following commutative diagram ewcluding the dotted arrow
    \begin{center}
        \begin{tikzpicture}
            \diagram{m}{1cm}{2cm} {
                {\Omega(A)} \& {P_A} \& A \\
                {\Omega(B)} \& {P_B} \& B. \\
            };

            \draw[math]
                (m-1-1) edge[hook] node {\iota_A} (m-1-2)
                    edge[swap] node {\Omega'(f) - \widetilde{\Omega'(f)}} (m-2-1)
                (m-1-2) edge[two heads] node {\pi_A} (m-1-3)
                    edge[dashed, swap] node {\phi} (m-2-1)
                    edge node {p_f - \widetilde{p_f}} (m-2-2)
                (m-1-3) edge node {f -f = 0} (m-2-3)

                (m-2-1) edge[hook] node {\iota_B} (m-2-2)
                (m-2-2) edge[two heads] node {\pi_B} (m-2-3);
        \end{tikzpicture}
    \end{center}

    Since
    \[
        \pi_B \circ \tuple{p_f - \widetilde{p_f}} = \tuple{f - f} \circ \pi_A = 0
    \]
    there exists a map \( \phi \) induced by the kernel property of \( \Omega(B) \), such that the lower triangle in the diagram commutes. However, since \( \iota_B \) is a monomorphism, one also gets that the upper triangle commutes. This implies that the morphism \( \Omega'(f) - \widetilde{\Omega'(f)}' \) factors through \( P_A \), a projective module. Therefore
    \[
        0 = \class{\Omega'(f) - \widetilde{\Omega'(f)}} = \class{\Omega'(f)} - \class{\widetilde{\Omega'(f)}}
    \]
    which implies \( \Omega([f]) \) is independent of choice of \( p_f \).

    Second, need to show that if \( f \sim g \), then \( \Omega(f) \sim \Omega(g) \). Look at the following diagram

    \begin{center}
        \begin{tikzpicture}
            \diagram{m}{1cm}{3cm} {
                \Omega(A) \& P_A \& A \\
                \&\& P \\
                \Omega(B) \& P_B \& B. \\
            };

            \draw[math]
                (m-1-1) edge[hook] node {\iota_A} (m-1-2)
                    edge node {\Omega'(f) - \Omega'(g)} (m-3-1)
                (m-1-2) edge[two heads] node {\pi_A} (m-1-3)
                    edge node {p_f - p_g} (m-3-2)
                (m-1-3) edge[swap] node {(f - g)_1} (m-2-3)
                    edge[curve={height=-25pt}] node {f - g} (m-3-3)

                (m-2-3) edge[swap, dashed, color={rgb,255:red,214;green,92;blue,92}] node {\theta} (m-3-2)
                    edge[swap] node {(f - g)_2} (m-3-3)

                (m-3-1) edge[hook] node {\iota_B} (m-3-2)
                (m-3-2) edge[two heads] node[swap] {\pi_B} (m-3-3);
        \end{tikzpicture}
    \end{center}

    Let \( P \) be the projective that \( f - g \) factors through. Then from the projective property, there exists a morphism \( \theta: P \to P_B \), which causes the lower triangle to commute.

    Let \( p'_{f - g} := \theta \circ (f - g)_1 \circ \pi_A \). By construction, one has that both \( p_f - p_g \) and \( p'_{f - g} \) are morphisms that would make the right hand square commute.
    
    But since
    \[
        p'_{f - g} \circ \iota_A = \theta \circ (f-g)_1 \circ \pi_A \circ \iota_A = \theta \circ (f-g)_1 \circ 0 = 0,
    \]
    it implies that if the leftmost morphism in the diagram was \( 0 \) and the middle morphism was \( p'_{f - g} \), it would commute. By the previous part of this proof, since the two possible middle morphisms, \( p_f - p_g \) and \( p'_{f - g} \), produce two different leftmost morphisms \( \Omega'(f) - \Omega'(g) \) and \( 0 \), one has that \( \Omega([f]) = \Omega([g]) \).
\end{proof}

Now that \( \Omega \) is a well defined asssignment of objects and morphisms, it only remains to prove functoriality as well as showing that the identity is mapped to the identity to show that it's a functor.

\begin{lemma}
    \label{lem:stmod_omega_endofunctor}
    \( \Omega \) is an endofunctor on \( \StMod(R) \).
\end{lemma}
\begin{proof}
    To show that \( \Omega \) is a functor it remains to prove functoriality as well as showing that the identity is mapped to the identity.

    First want to show that \( \Omega \) is functorial. Let \( A, B, C \in \StMod(R) \). Then by the definition of \( \Omega \), one has the following commutative diagram
    \begin{center}
        \begin{tikzpicture}
            \diagram{m}{1cm}{2cm} {
                {\Omega(A)} \& {P_A} \& A \\
                {\Omega(B)} \& {P_B} \& B \\
                {\Omega(C)} \& {P_C} \& C \\
            };

            \draw[math]
                (m-1-1) edge[hook] (m-1-2)
                    edge[curve={height=30pt}, swap, color={rgb,255:red,214;green,92;blue,92}] node {\Omega(f \circ g)} (m-3-1)
                    edge node {\Omega(g)} (m-2-1)
                (m-1-2) edge[two heads] (m-1-3)
                    edge[curve={height=30pt}, swap, color={rgb,255:red,214;green,92;blue,92}] node[pos=0.3] {p_{f \circ g}} (m-3-2)
                    edge node {p_g} (m-2-2)
                (m-1-3) edge node {g} (m-2-3)
                    edge[curve={height=-30pt}, color={rgb,255:red,214;green,92;blue,92}] node {f \circ g} (m-3-3)

                (m-2-1) edge[hook] (m-2-2)
                    edge node {\Omega(f)} (m-3-1)
                (m-2-2) edge[two heads] (m-2-3)
                    edge node {p_f} (m-3-2)
                (m-2-3) edge node {f} (m-3-3)

                (m-3-1) edge[hook] (m-3-2)
                (m-3-2) edge[two heads] (m-3-3);
        \end{tikzpicture}
    \end{center}

    Looking at the composition of the vertical morphisms, one neds up with the following commutative diagram
    \begin{center}
        \begin{tikzpicture}
            \diagram{m}{1cm}{2cm} {
                {\Omega(A)} \& {P_A} \& A \\
                {\Omega(C)} \& {P_C} \& C. \\
            };

            \draw[math]
                (m-1-1) edge[hook] node {\iota_A} (m-1-2)
                    edge[swap] node {\Omega'(f) \circ \Omega'(g)} (m-2-1)
                (m-1-2) edge[two heads] node {\pi_A} (m-1-3)
                    edge node {p_f \circ p_g} (m-2-2)
                (m-1-3) edge node {f \circ g} (m-2-3)

                (m-2-1) edge[hook] node {\iota_C} (m-2-2)
                (m-2-2) edge[two heads] node {\pi_C} (m-2-3);
        \end{tikzpicture}
    \end{center}
    Since \( \Omega \) is a well-defined assignment of morphisms, this implies
    \[
        \Omega([f]) \circ \Omega([g]) = \class{\Omega'(f) \circ \Omega'(g)} = \Omega(\class{f \circ g}).
    \]

    Second, need to show that \( \Omega(\Id_A) = \Id_{\Omega(A)} \) in \( \StMod(R) \).

    By a similar argument to above, one can see that every square and triangle in the following diagram commutes
    \begin{center}
        \begin{tikzpicture}
            \diagram{m}{1cm}{2cm} {
                {\Omega(A)} \& {P_A} \& A \\
                {\Omega(A)} \& {P_A} \& A. \\
            };

            \draw[math]
                (m-1-1) edge[hook] (m-1-2)
                    edge[swap] node {\Id_{\Omega(A)}} (m-2-1)
                (m-1-2) edge[two heads] (m-1-3)
                    edge node {\Id_{P_A}} (m-2-2)
                (m-1-3) edge node {\Id_A} (m-2-3)

                (m-2-1) edge[hook] (m-2-2)
                (m-2-2) edge[two heads] (m-2-3);
        \end{tikzpicture}
    \end{center}
    Therefore \( \Omega(\class{\Id_A}) = \class{\Id_{\Omega(A)}} \).
\end{proof}

An important detail in \autoref{def:stmod_omega} of \( \Omega \), is that the definition is dependent on a choice of ``projective covers'' (in quotes, since the epimorphisms doesn't have to be essential epimorphisms). This is important to note since otherwise \( \Omega \) would not be well defined.

An important detail still missing to show that \( \Omega^{-1} \) could be a shift functor is showing that it \( \Omega \) is additive.

\begin{lemma}
    \label{lem:stmod_omega_additive_functor}
    \( \Omega \) is an additive functor.
\end{lemma}
\begin{proof}
    Want to show that \( \Omega(f + g) \sim \Omega(f) + \Omega(g) \).
    
    There are two posssible values of \( \Omega'(f + g) \), namely the ones from either of the two commutative diagrams below
    \begin{center}
        \begin{tikzpicture}
            \diagram{m}{1cm}{2cm} {
                {\Omega(A)} \& {P_A} \& A \\
                {\Omega(B)} \& {P_B} \& B \\
            };

            \draw[math]
                (m-1-1) edge[hook] node {\iota_A} (m-1-2)
                    edge[swap] node {\Omega'(f+g)} (m-2-1)
                (m-1-2) edge[two heads] node {\pi_A} (m-1-3)
                    edge node[swap] {p_{f + g}} (m-2-2)
                (m-1-3) edge node[swap] {f + g} (m-2-3)

                (m-2-1) edge[hook] node {\iota_B} (m-2-2)
                (m-2-2) edge[two heads] node {\pi_B} (m-2-3);
        \end{tikzpicture}
        \begin{tikzpicture}
            \diagram{m}{1cm}{2cm} {
                {\Omega(A)} \& {P_A} \& A \\
                {\Omega(B)} \& {P_B} \& B. \\
            };

            \draw[math]
                (m-1-1) edge[hook] node {\iota_A} (m-1-2)
                    edge[swap] node {\Omega'(f)+\Omega'(g)} (m-2-1)
                (m-1-2) edge[two heads] node {\pi_A} (m-1-3)
                    edge node[swap] {p_f + p_g} (m-2-2)
                (m-1-3) edge node[swap] {f + g} (m-2-3)

                (m-2-1) edge[hook] node {\iota_B} (m-2-2)
                (m-2-2) edge[two heads] node {\pi_B} (m-2-3);
        \end{tikzpicture}
    \end{center}

    Since \( \Omega([f + g]) \) is well defined, this implies \( \Omega([f + g]) = \Omega([f]) + \Omega([g]) \).
\end{proof}

Now that \( \Omega \) has been shown to be an additive functor, one can define \( \Sigma \) which will turn out to be the inverse of \( \Omega \) later.

\begin{definition}
    \label{def:stmod_sigma}
    Since \( \Mod(R) \) has enough injectives, one can choose a injective module \( I_A \) for every \( A \) with monomorphisms \( \iota_A \) from \( A \) to \( I_A \) for every \( A \). Fix a choice of \( I_A \) and \( \iota_A \) for every \( A \).

    Then define \( \Sigma \) as the assignment of objects and morphisms in \( \StMod(R) \) as follows:
    \begin{itemize}
        \item For any object \( A \) in \( \StMod(R) \), let \( \Sigma(A) := \coker(\iota_A) \).
        \item For any \( [f] \in \StMod(A, B) \), let \( \Sigma([f]) \) be as explained in \autoref{rem:stmod_sigma_f}.
    \end{itemize}

    By \autoref{lem:stmod_sigma_well-defined_additive_endofunctor}, \( \Omega \) is an additive endofunctor on \( \StMod(R) \).
\end{definition}

As before, it is important to first show how \( \Sigma \) works on a morphism.

\begin{remark}
    \label{rem:stmod_sigma_f}
    \( \Sigma([f]) \) is constructed as follows:

    Look at the following diagram in \( \Mod(R) \)
    \begin{center}
        \begin{tikzpicture}
            \diagram{m}{1cm}{1cm} {
                A \& I_A \& \Sigma(A) \\
                B \& I_B \& \Sigma(B). \\
            };

            \draw[math]
                (m-1-1) edge[hook] node {\iota_A} (m-1-2)
                    edge node {f} (m-2-1)
                (m-1-2) edge[two heads] node {\pi_A} (m-1-3)
                    edge node {i_f} (m-2-2)
                (m-1-3) edge node {\Sigma'(f)} (m-2-3)

                (m-2-1) edge[hook] node {\iota_B} (m-2-2)
                (m-2-2) edge[two heads] node {\pi_B} (m-2-3);
        \end{tikzpicture}
    \end{center}

    One has that for a morphism \( f: A \to B \), there exists a morphism \( i_f \) from the universal property of injective objects such that the left square commutes. Please note that this morphism is \emph{not neccesarily} unique.

    Furthermore, since
    \[
        \pi_B \circ i_f \circ \iota_A = \pi_B \circ \iota_B \circ f = 0,
    \]
    one has from the universal cokernel property that there exists a unique morphism, dependant on the choice of \( i_f \), denoted as \( \Sigma'(f) \) from \( \Sigma(A) \) to \( \Sigma(B) \) such that the right square in the above diagram commutes. Taking the equivalence class of this morphism with respect to morphisms factoring through projectives yields the morphism defined by the functor, i.e. \( \Sigma([f]) := \class{\Sigma'(f)} \).

    It is shown in \autoref{lem:stmod_sigma_well-defined_additive_endofunctor} that \( \Sigma([f]) \) is in fact unique and independent of the choice of representative, and \( \Sigma \) is therefore a well-defined assignment of morphisms.
\end{remark}

Now it remains to show every property of \( \Omega \), like being a well-defined, an endofunctor, and additive. However, that is a lot of work, that can luckily be skipped because of the previous legwork on proving the respective properties for \( \Omega \).

\begin{lemma}
    \label{lem:stmod_sigma_well-defined_additive_endofunctor}
    \( \Sigma \) is a well-defined and additive endofunctor on \( \StMod(R) \).
\end{lemma}
\begin{proof}
    The proofs of the various statements are entirely dual to the proofs of \autoref{lem:stmod_omega_f_is_well_defined}, \autoref{lem:stmod_omega_endofunctor} and \autoref{lem:stmod_omega_additive_functor}.
\end{proof}

Now one can finally show that \( \Sigma = \Omega^{-1} \).

\begin{theorem}
    \( \Omega \) is an auto equivalence with inverse \( \Sigma \).
\end{theorem}
\begin{proof}
    This thesis will only show that \( \Id_{\StMod(R)} \) is naturally isomorphic to \( \Sigma\Omega \). The omitted proof that \( \Id_{\StMod(R)} \) is naturally isomorphic to \( \Omega\Sigma \) is very similar, but uses many dual properties.

    WIP

    Let \( A \in \Obj(\Tc) \).

    First show that there exist a (not neccesarily unique, but for any object, just choose one) isomorphism from \( A \to \Sigma\Omega(A) \). Consider the following diagram:

    \begin{center}
        \begin{tikzpicture}
            \diagram{m}{1cm}{2cm} {
                \Omega(A) \& P_A \& A \\
                \Omega(A) \& I_{\Omega(A)} \& \Sigma \circ \Omega(A) \\
                \Omega(A) \& P_A \& A \\
            };

            \draw[math]
                (m-1-1) edge[hook] node {\iota_A} (m-1-2)
                    edge[equal] (m-2-1)
                (m-1-2) edge[two heads] node {\pi_A} (m-1-3)
                    edge node {i_{\phi_1}} (m-2-2)
                (m-1-3) edge node {\phi_1} (m-2-3)

                (m-2-1) edge[hook] node {\kappa_{\Omega(A)}} (m-2-2)
                    edge[equal] (m-3-1)
                (m-2-2) edge[two heads] node {\rho_{\Omega(A)}} (m-2-3)
                    edge node {i_{\phi_2}} (m-3-2)
                (m-2-3) edge node {\phi_2} (m-3-3)

                (m-3-1) edge[hook] node {\iota_A} (m-3-2)
                (m-3-2) edge[two heads] node {\pi_A} (m-3-3);
        \end{tikzpicture}
    \end{center}

    Where \( i_{\phi_1} \) is the injective property map induced from \( \iota_A \). Then, since \( \rho_{\Omega(A)} \circ i_{\phi_1} \circ \iota_A = \rho_{\Omega(A)} \circ \kappa_{\Omega(A)} = 0 \), and since \( \Mod(R) \) is an abelian category, one has that the cokernel of a kernel of a epimorphism is isomorphic to the codomain of the epimorphism. Therefore one has that \( A \) is the cokernel of \( \iota_A \). Therefore, from the cokernel property, there is a uniquely induced map \( \phi_1: A \to \Sigma\Omega(A) \). Then doing the same for the lower rectangle of the diagram, using the fact that every projective is also injective, then one gets the map \( \phi_2 \).

    Then to show that \( \phi_1 \) and \( \phi_2 \) are isomorphisms, look at the following diagram:

    \begin{center}
        \begin{tikzpicture}
            \diagram{m}{1cm}{2cm} {
                \Omega(A) \& P_A \& A \\
                \Omega(A) \& P_A \& A \\
            };

            \draw[math]
                (m-1-1) edge[hook] node {\iota_A} (m-1-2)
                    edge[swap] node {\Id_{\Omega(A)} \circ \Id_{\Omega(A)} - \Id_{\Omega(A)} = 0} (m-2-1)
                (m-1-2) edge[two heads] node {\pi_A} (m-1-3)
                    edge[swap] node {p_{\phi_2} \circ p_{\phi_1} - \Id_P} (m-2-2)
                (m-1-3) edge[swap, color={rgb,255:red,214;green,92;blue,92}] node {\theta} (m-2-2)
                    edge node {\phi_2 \circ \phi_1 - \Id_A} (m-2-3)

                (m-2-1) edge[hook] node {\iota_A} (m-2-2)
                (m-2-2) edge[two heads] node {\pi_A} (m-2-3);
        \end{tikzpicture}
    \end{center}

    Using the previous commutative diagram, one gets that every square commutes. But then \( (p_{\phi_2} \circ p_{\phi_1} - \Id_{P_A}) \circ \iota_A = \iota_A \circ 0 = 0 \). Then from the cokernel property there exist a map \( \theta: A \to P_A \) such that \( \theta \circ \pi_A = p_{\theta_2} \circ p_{\theta_1} - \Id_P \). But then one has that \( (\phi_2 \circ \phi_1 - \Id_A) \circ \pi_A = \pi_A \circ (p_{\theta_2} \circ p_{\theta_1} - \Id_P) = \pi_A \circ \theta \circ \pi_A \). But since \( \pi_A \) is an epimorphism, one gets that \( \phi_2 \circ \phi_1 - \Id_A = \pi_A \circ \theta \), which means that \( \phi_2 \circ \phi_1 \sim \Id_A \).
    
    Similarly (TODO) one can show that \( \phi_1 \circ \phi_2 \sim Id_A \), which means that \( \phi_1 \) and \( \phi_2 \) are isomorphisms from \( A \) to \( \Sigma\Omega(A) \).

    For \( A, B \in \Tc \), let \( f \in \Mod(R)(A, B) \).

    To show that these isomorphisms are natural, look at the following two diagrams:

    \begin{center}
        \begin{tikzpicture}
            \diagram{m}{1cm}{2cm} {
                \Omega(A) \& P \& A \\
                \Omega(A) \& I \& \Sigma \circ \Omega(A) \\
                \Omega(B) \& I \& \Sigma \circ \Omega(B) \\
            };

            \draw[math]
                (m-1-1) edge[hook] node {\iota_A} (m-1-2)
                    edge[equal] (m-2-1)
                (m-1-2) edge[two heads] node {\pi_A} (m-1-3)
                    edge node {p_{\phi^A_1}} (m-2-2)
                (m-1-3) edge node {\phi^A_1} (m-2-3)

                (m-2-1) edge[hook] node {\kappa_{\Omega(A)}} (m-2-2)
                    edge node {\Omega(f)} (m-3-1)
                (m-2-2) edge[two heads] node {\rho_{\Omega(A)}} (m-2-3)
                    edge node {i_{\Omega(f)}} (m-3-2)
                (m-2-3) edge node {\Sigma \circ \Omega(f)} (m-3-3)

                (m-3-1) edge[hook] node {\kappa_{\Omega(B)}} (m-3-2)
                (m-3-2) edge[two heads] node {\rho_{\Omega(B)}} (m-3-3);
        \end{tikzpicture}
    \end{center}

    And:

    \begin{center}
        \begin{tikzpicture}
            \diagram{m}{1cm}{2cm} {
                \Omega(A) \& P \& A \\
                \Omega(B) \& P \& B \\
                \Omega(B) \& I \& \Sigma \circ \Omega(B) \\
            };

            \draw[math]
                (m-1-1) edge[hook] node {\iota_A} (m-1-2)
                    edge node {\Omega(f)} (m-2-1)
                (m-1-2) edge[two heads] node {\pi_A} (m-1-3)
                    edge node {p_f} (m-2-2)
                (m-1-3) edge node {f} (m-2-3)

                (m-2-1) edge[hook] node {\iota_B} (m-2-2)
                    edge[equal] (m-3-1)
                (m-2-2) edge[two heads] node {\pi_B} (m-2-3)
                    edge node {i_{\phi_1^B}} (m-3-2)
                (m-2-3) edge node {\phi_1^B} (m-3-3)

                (m-3-1) edge[hook] node {\kappa_{\Omega(B)}} (m-3-2)
                (m-3-2) edge[two heads] node {\rho_{\Omega(B)}} (m-3-3);
        \end{tikzpicture}
    \end{center}

    Where every small square, and therefore rectangle, commutes.

    This gives rise to the following commutative diagram:

    \begin{center}
        \begin{tikzpicture}
            \diagram{m}{1cm}{3cm} {
                \Omega(A) \& P \& A \\
                \Omega(B) \& I \& \Sigma \circ \Omega(B) \\
            };

            \draw[math]
                (m-1-1) edge[hook] (m-1-2)
                    edge[swap] node {\Id_{\Omega(B)} \circ \Omega(f) - \Omega(f) \circ \Id_{\Omega(A)} = 0} (m-2-1)
                (m-1-2) edge[two heads] (m-1-3)
                    edge[swap] node {i_{\phi_1^B} \circ p_f - i_{\Omega(f)} \circ p_{\phi_1^A}} (m-2-2)
                (m-1-3) edge[swap, color={rgb,255:red,214;green,92;blue,92}] node {\theta} (m-2-2)
                    edge node {\phi_1^B \circ f - \Sigma\Omega(f) \circ \phi_1^A} (m-2-3)

                (m-2-1) edge[hook] (m-2-2)
                (m-2-2) edge[two heads] (m-2-3);
        \end{tikzpicture}
    \end{center}

    But from the cokernel property of \( A \), one gets an induced map \( \theta \), and from the epimorphism property, it makes the lower triangle commute, which implies \( \phi_1^B \circ f \sim \Sigma\Omega(f) \circ \phi_1^A \), which means that it is natural. And since the choice of \( \phi_1 \) is arbitrary, it is independent of the choice of \( \phi_1 \).
\end{proof}