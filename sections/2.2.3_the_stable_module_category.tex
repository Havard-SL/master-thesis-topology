% TODO: Kan generaliserast til endeleg gen modular om R er noetherske. Og det verkar som Frobenius ringar er noetherske. https://en.wikipedia.org/wiki/Quasi-Frobenius_ring Zimmermann: Prop. 5.1.4

A triangulated category that will be central in this thesis is the stable module category. Therefore, the definition will be given in more details than the previous two examples of triangulated categories.

% TODO: Marius vil definera projektive objekt. Overkill?
Before defining the stable module category, we prove a lemma that will be used to define the morphisms.
\begin{lemma}
    \label{lem:morphisms_factoring_through_projectives_r-submodule}
    Let \( R \) be a commutative ring with identity.

    Let \( G \) be the subset of \( \Mod(R)(A, B) \) consisting of module morphisms that factor through a projective module.

    Then \( G \) is an \( R \)-submodule of \( \Mod(R)(A, B) \).
\end{lemma}
\begin{proof}
    Let \( f \) and \( g \) be two morphisms that factor through the projective objects \( P \) and \( Q \), respectively. Then we have the following commutative diagrams
    \[
        \begin{aligned}
            \begin{tikzpicture}
                \diagram{m}{1cm}{1cm} {
                    A \& P \& B \\
                };
    
                \draw[math]
                    (m-1-1) edge node {f_1} (m-1-2)
                        edge[curve={height=25pt}] node[swap] {f} (m-1-3)
                    (m-1-2) edge node {f_2} (m-1-3);
            \end{tikzpicture}
        \end{aligned}
        \text{ and }
        \begin{aligned}
            \begin{tikzpicture}
                \diagram{m}{1cm}{1cm} {
                    A \& Q \& B. \\
                };
    
                \draw[math]
                    (m-1-1) edge node {g_1} (m-1-2)
                        edge[curve={height=25pt}] node[swap] {g} (m-1-3)
                    (m-1-2) edge node {g_2} (m-1-3);
            \end{tikzpicture}
        \end{aligned}
    \]

    We can then construct the morphism
    \begin{center}
        \begin{tikzpicture}
            \diagram{m}{1cm}{1cm} {
                A \& {P \oplus Q} \& B. \\
            };

            \draw[math]
                (m-1-1) edge node {
                    \begin{psmallmatrix}
                        f_1 \\
                        g_1
                    \end{psmallmatrix}
                } (m-1-2)
                (m-1-2) edge node {(f_2, g_2)} (m-1-3);
        \end{tikzpicture}
    \end{center}

    Composing these two morphisms, we get the morphism \( f_2 \circ f_1 + g_2 \circ g_1 = f + g \). This morphism factors through \( P \oplus Q \), which is projective since it is a direct sum of projective modules. This implies that \( G \) is closed under addition.
    
    Let \( r \in R \). Then the following diagram commutes,
    \begin{center}
        \begin{tikzpicture}
            \diagram{m}{1cm}{1cm} {
                A \& P \& B, \\
            };

            \draw[math]
                (m-1-1) edge node {rf_1} (m-1-2)
                    edge[curve={height=25pt}] node[swap] {rf} (m-1-3)
                (m-1-2) edge node {f_2} (m-1-3);
        \end{tikzpicture}
    \end{center}
    which implies that \( G \) is also closed under multiplication in \( R \).
\end{proof}

In order to get a triangulated category, we have to use a specific type of ring, as defined below.
\begin{definition}
    A commutative ring with identity is called a \emph{Frobenius ring}, if every injective module is projective, and vice versa.
\end{definition}

The following is the definition of the stable module category.
\begin{definition}
    \label{def:stable_module_category}
    Let \( R \) be a \emph{Frobenius ring}.

    Then the \emph{stable module category over \( R \)}, denoted \( \Mc \), is defined in the following way:
    \begin{enumerate}
        \item {
            The objects are modules over \( R \).
        }
        \item {
            For \( A, B \in \Mc \), let \( G \) be the \( R \)-submodule from \autoref{lem:morphisms_factoring_through_projectives_r-submodule}.
        
            Then let
            \[
                \Mc(A, B) := \Mod(R)(A, B)/G.
            \]
        }
        \item {
            For \( [g] \in \Mc(B, C) \), and \( [f] \in \Mc(A, B) \), let composition be defined as follows
            \[
                [g] \circ [f] := [g \circ f].
            \]
        }
    \end{enumerate}
    This category is called the \emph{stable module category over \( R \)}.
\end{definition}

Other authors typically use the notation \( \StMod(R) \) or \( \underline{\Mod}(R) \) for the stable module category. However, for the sake of brevity this is reduced to simply \( \Mc \) in this thesis.

By the following remark, composition in \autoref{def:stable_module_category} is well-defined.
\begin{remark}
    \label{rem:stmod_composition_well-defined}
    Composition in \autoref{def:stable_module_category} is well-defined.
\end{remark}
\begin{proof}
    We need to check that two different choices of representatives of \( [f] \) and \( [g] \) yield the same value.

    Let \( f + \widetilde{f} \) and \( g + \widetilde{g} \) be two different representatives of \( [f] \) and \( [g] \), with \( \widetilde{f} \) and \( \widetilde{g} \) factoring through some projective modules.

    Then it follows that
    \[
         [g + \widetilde{g}] \circ [f + \widetilde{f}] = [g \circ f] + [\widetilde{g} \circ f] + [g \circ \widetilde{f}] + [\widetilde{g} \circ \widetilde{f}].
    \]
    
    Every term other than \( [g \circ f] \) factors through a projective and is therefore equal to \( 0 \) in \( \Mc(A, C) \).
\end{proof}

In the general definition of the stable module category, it is not required that the ring is a Frobenius ring. However, as will become apparent later on, it is a requirement to have a triangulation. Since we will only be considering stable module categories which are triangulated, this assumption is made from the beginning.
% TODO: Faktasjekk, er det verkeleg sant at Frobenius er naudsynt? Las eg ikkje ein stad at nok injektiv og projektive og dei samanfaller er godt nok?

In order to admit a triangulation, we need to prove that the stable module category is additive.

% TODO: Forenkla beviset, biprodukt argumentet spesielt. Ikkje 100% sikker på at Mac Lane seier det eg ynskjer, berre 80%ish.
\begin{lemma}
    \( \Mc \) is an additive category.
\end{lemma}
\begin{proof}
    There are two parts to this proof. First, we show that \( \Mc \) is pre-additive and second, we show that \( \Mc \) has finite products which in addition to being pre-additive would imply that \( \Mc \) is additive by \cite[p. 251]{Mac_Lane_1995}.

    To show that \( \Mc \) is pre-additive there are two properties that need to be shown
    \begin{enumerate}
        \item {
            First, we show that for any \( A, B \in \Mc \) that \( \Mc(A, B) \) is an abelian group.

            This follows immediately from the definition, since \( \Mc(A, B) \) is a quotient module, and is therefore an \( R \)-module and, hence, an abelian group.
        }
        \item {
            Second, we show that composition is bilinear.

            Consider \( [g], [g'] \in \Mc(B, C) \) and \( [f], [f'] \in \Mc(A, B) \). Consider the following equation
            \begin{align*}
                ([g] + [g']) \circ ([f] + [f']) &= [g + g'] \circ [f + f'] \\
                &= \class{(g + g') \circ (f + f')} \\
                &= \class{g \circ f + g \circ f' + g' \circ f + g' \circ f'} \\
                &= [g] \circ [f] + [g] \circ [f'] + [g'] \circ [f] + [g'] \circ [f'],
            \end{align*}
            which is precisely bilinearity.
        }
    \end{enumerate}

    The claim is that the usual biproduct in \( \Mod(R) \) induces the product in \( \Mc \).
    
    Consider the commutative diagram for the universal property of \( A \oplus B \) as a product in \( \Mod(R) \). Taking residue classes of the morphisms yields the following commutative diagram in \( \Mc \),
    \begin{center}
        \begin{tikzpicture}
            \diagram{m}{1.5cm}{1.5cm} {
                \& X \\
                A \& A \oplus B \& B. \\
            };

            \draw[math]
                (m-1-2) edge node[swap] {[f_A]} (m-2-1)
                    edge[dashed] node {[f]} (m-2-2)
                    edge node {[f_B]} (m-2-3)

                (m-2-2) edge node[swap] {[\pi_B]} (m-2-3)
                    edge node {[\pi_A]} (m-2-1);
        \end{tikzpicture}
    \end{center}
    Let \( [g]: X \to A \oplus B \) be another morphism that satisfies the universal property.

    Then the following diagram
    \begin{center}
        \begin{tikzpicture}
            \diagram{m}{1.5cm}{1.5cm} {
                \& X \\
                A \& A \oplus B \& B \\
            };

            \draw[math]
                (m-1-2) edge node[swap] {[0]} (m-2-1)
                    edge[dashed] node {[f - g]} (m-2-2)
                    edge node {[0]} (m-2-3)

                (m-2-2) edge node[swap] {[\pi_B]} (m-2-3)
                    edge node {[\pi_A]} (m-2-1);
        \end{tikzpicture}
    \end{center}
    commutes.

    Let \( \iota_A \), and \( \iota_B \) denote the canonical split monomorphisms from the universal property of the coproduct. Since \( A \oplus B \) is a biproduct in \( \Mod(R) \), we have
    \[
        \Id_{A \oplus B} = \iota_A \circ \pi_A + \iota_B \circ \pi_B.
    \]
    Thus,
    \begin{align*}
        [f - g] &= [\Id_{A \oplus B} \circ (f - g)] \\
        &= [\iota_A \circ \pi_A \circ (f - g) + \iota_B \circ \pi_B \circ (f - g)] \\
        &= [\iota_A] \circ [\pi_A \circ (f - g)] + [\iota_B] \circ [\pi_B \circ (f - g)] \\
        &= [0],
    \end{align*}
    which implies that \( [f] = [g] \), and that \( A \oplus B \) is a product in \( \Mc \), and by the statement at the start, also a biproduct.
\end{proof}

The next step in creating a triangulation for \( \Mc \) is to define the shift functor. In the stable module category, the shift functor is dual to the ``syzygy functor''. We start by defining what the syzygy functor is, and afterwards define the shift functor.

% MS-Question: Marius tilbakemelding i dette avsnittet er uklart.
The syzygy functor as well as the shift functor exists in the category of modules over a commutative ring with identity (e.g., a Frobenius ring) because it has \emph{enough projectives} and \emph{enough injectives}. Having enough projectives means that for any module \( A \), there exists some projective module \( P_A \), as well as an epimorphism \( \pi_A \) from \( P_A \) to \( A \). The choice of \( P_A \) and \( \pi_A \) is not necessarily unique up to isomorphism, and two different choices of \( P_A \) could be non-isomorphic. Similarly, having enough injectives means that for any module \( A \) there exists some injective module \( I_A \) along with a monomorphism \( \iota_A \) from \( A \) to \( I_A \). Equal to the projective case, \( I_A \) is not necessarily unique up to isomorphism and \( \iota_A \) is not necessarily unique.

The definition of the syzygy functor is closely tied to a choice of \( P_A \)'s and \( \pi_A \)'s for every object \( A \).

% TODO: Kopling med syzygy?
% TODO: Skriv betre P (og seinare I) forklaringa.
\begin{definition}[The functor \( \Omega \)]
    \label{def:stmod_omega}
    Let \( R \) be a Frobenius ring.

    Let \( P = \set{\tuple{P_A, \pi_A, \Omega A}}_{A \in \Mod(R)} \) be a collection of tuples for every object \( A \), where \( P_A \) is a projective object, \( \pi_A: P_A \twoheadrightarrow A \) an epimorphism, and \( \Omega A \) is a kernel of \( \pi_A \). This exists because \( \Mod(R) \) has enough projectives.

    WIP: Marius tilbakemeldingar

    Then define \( \Omega \) as the assignment of objects and morphisms in \( \Mc \) as follows:
    \begin{itemize}
        \item {
            For any object \( A \) in \( \Mc \), let \( \Omega A \) be as defined in \( P \) above.
        }
        \item {
            For any \( [f] \in \Mc(A, B) \), let \( \Omega [f] \) be constructed as follows:

            Consider the following diagram in \( \Mod(R) \), with the relevant objects and non-dashed morphisms from \( P \),
            \begin{center}
                \begin{tikzpicture}
                    \diagram{m}{1cm}{1cm} {
                        \Omega A \& P_A \& A \\
                        \Omega B \& P_B \& B. \\
                    };

                    \draw[math]
                        (m-1-1) edge[tailed] node {\iota_A} (m-1-2)
                            edge[dashed] node[swap] {\Omega f} (m-2-1)
                        (m-1-2) edge[two headed] node {\pi_A} (m-1-3)
                            edge[dashed] node {p_f} (m-2-2)
                        (m-1-3) edge node {f} (m-2-3)

                        (m-2-1) edge[tailed] node {\iota_B} (m-2-2)
                        (m-2-2) edge[two headed] node {\pi_B} (m-2-3);
                \end{tikzpicture}
            \end{center}

            We have that for a morphism \( f: A \to B \), there exists a morphism \( p_f \) from the lifting property of projective modules such that the right square commutes. Note that the choice of \( p_f \) is not neccesarily unique.

            Furthermore, since
            \[
                \pi_B \circ p_f \circ \iota_A = f \circ \pi_A \circ \iota_A = f \circ 0 = 0,
            \]
            we have from the universal kernel property that there exists a unique morphism, dependent on the choice of \( p_f \), denoted as \( \Omega f \) from \( \Omega A \) to \( \Omega B \) such that the left square in the above diagram commutes. Taking the equivalence class of this morphism with respect to morphisms factoring through projectives yields the morphism defined by the assignment, i.e., \( \Omega [f] := \class{\Omega f} \).
        }
    \end{itemize}
\end{definition}

Since we are going to work a lot with the definition of \( \Omega \) we want to clarify the notation used.
\begin{notation}
    For \( f \in \Mod(R)(A, B) \), let \( \Omega f \) refer to a morphism which fits the diagram in \autoref{def:stmod_omega}. Note that this is not a well-defined assignment, as multiple morphisms in \( \Mod(R)(\Omega A, \Omega B) \) could be labeled as \( \Omega f \).
\end{notation}

The goal is to show that \( \Omega: \Mc \to \Mc \) is a well-defined endofunctor, and later showing that it is in fact an additive auto-equivalence of categories.

\begin{lemma}
    \label{lem:stmod_omega_f_is_well_defined}
    \( \Omega \) is a well-defined assignment of morphisms.
\end{lemma}
\begin{proof}
    There are two things that need to be proven. First, in the construction of \( \Omega f \), we have to choose a morphism \( p_f \) from the projective property. We need to show that if we choose another projective morphism, that \( \Omega \) still yields the same morphism in \( \Mc \). Second, we need to show that if \( [f] = [g] \), then \( \Omega [f] = \Omega [g] \).

    To prove the first part, let \( p_f \) and \( \widetilde{p_f} \) be two different projective morphisms that give the morphisms \( \Omega f \) and \( \widetilde{\Omega f} \), respectively. Then we have the following commutative diagram excluding the dashed arrow
    \begin{center}
        \begin{tikzpicture}
            \diagram{m}{1cm}{2cm} {
                {\Omega A} \& {P_A} \& A \\
                {\Omega B} \& {P_B} \& B. \\
            };

            \draw[math]
                (m-1-1) edge[tailed] node {\iota_A} (m-1-2)
                    edge[swap] node {\Omega f - \widetilde{\Omega f}} (m-2-1)
                (m-1-2) edge[two headed] node {\pi_A} (m-1-3)
                    edge[dashed, swap] node {\phi} (m-2-1)
                    edge node {p_f - \widetilde{p_f}} (m-2-2)
                (m-1-3) edge node {f -f = 0} (m-2-3)

                (m-2-1) edge[tailed] node {\iota_B} (m-2-2)
                (m-2-2) edge[two headed] node {\pi_B} (m-2-3);
        \end{tikzpicture}
    \end{center}

    Since
    \[
        \pi_B \circ \tuple{p_f - \widetilde{p_f}} = \tuple{f - f} \circ \pi_A = 0
    \]
    there exists a morphism \( \phi \) induced by the kernel property of \( \Omega B \), such that the lower triangle in the diagram commutes. Then because of the monomorphism property of \( \iota_B \), we also get that the upper triangle commutes. This implies that the morphism \( \Omega f - \widetilde{\Omega f} \) factors through \( P_A \), a projective module. Therefore
    \[
        0 = \class{\Omega f - \widetilde{\Omega f}} = \class{\Omega f} - \class{\widetilde{\Omega f}}
    \]
    which implies that \( \Omega [f] \) is independent of the choice of \( p_f \).

    Second, we need to show that if \( [f] = [g] \), then \( \Omega [f] = \Omega [g] \). Consider the following commutative diagram excluding the dashed arrow,
    \begin{center}
        \begin{tikzpicture}
            \diagram{m}{1cm}{3cm} {
                \Omega A \& P_A \& A \\
                \&\& P \\
                \Omega B \& P_B \& B. \\
            };

            \draw[math]
                (m-1-1) edge[tailed] node {\iota_A} (m-1-2)
                    edge node {\Omega f - \Omega g} (m-3-1)
                (m-1-2) edge[two headed] node {\pi_A} (m-1-3)
                    edge node {p_f - p_g} (m-3-2)
                (m-1-3) edge[swap] node {(f - g)_1} (m-2-3)
                    edge[curve={height=-25pt}] node {f - g} (m-3-3)

                (m-2-3) edge[swap, dashed] node {\theta} (m-3-2)
                    edge[swap] node {(f - g)_2} (m-3-3)

                (m-3-1) edge[tailed] node {\iota_B} (m-3-2)
                (m-3-2) edge[two headed] node[swap] {\pi_B} (m-3-3);
        \end{tikzpicture}
    \end{center}

    Let \( P \) be the projective that \( f - g \) factors through. Then from the projective property, there exists a morphism \( \theta: P \to P_B \), which causes the lower triangle to commute.

    Let \( p'_{f - g} := \theta \circ (f - g)_1 \circ \pi_A \). By construction, we have that both \( p_f - p_g \) and \( p'_{f - g} \) are morphisms that would make the right hand square commute.
    
    But since
    \[
        p'_{f - g} \circ \iota_A = \theta \circ (f-g)_1 \circ \pi_A \circ \iota_A = \theta \circ (f-g)_1 \circ 0 = 0,
    \]
    the following diagram commutes
    \begin{center}
        \begin{tikzpicture}
            \diagram{m}{1cm}{1cm} {
                \Omega A \& P_A \& A \\
                \Omega B \& P_B \& B. \\
            };

            \draw[math]
                (m-1-1) edge[tailed] node {\iota_A} (m-1-2)
                    edge node {0} (m-2-1)
                (m-1-2) edge[two headed] node {\pi_A} (m-1-3)
                    edge node {p'_{f - g}} (m-2-2)
                (m-1-3) edge node {f - g} (m-2-3)

                (m-2-1) edge[tailed] node {\iota_B} (m-2-2)
                (m-2-2) edge[two headed] node {\pi_B} (m-2-3);
        \end{tikzpicture}
    \end{center}
    However, by the first part of this proof, this implies that \( [\Omega f - \Omega g] = [0] \), and therefore \( \Omega [f] = \Omega [g] \).
\end{proof}

Now that \( \Omega \) is a well-defined asssignment of objects and morphisms, it only remains to prove functoriality.

\begin{lemma}
    \label{lem:stmod_omega_endofunctor}
    \( \Omega \) is an endofunctor on \( \Mc \).
\end{lemma}
\begin{proof}
    To show that \( \Omega \) is a functor it remains to prove functoriality.

    First, we show that \( \Omega \) is preserves composition. Let \( A, B, C \in \Mc \). Then by the definition of \( \Omega \), we have the following commutative diagram
    \begin{center}
        \begin{tikzpicture}
            \diagram{m}{1cm}{2cm} {
                {\Omega A} \& {P_A} \& A \\
                {\Omega B} \& {P_B} \& B \\
                {\Omega C} \& {P_C} \& C \\
            };

            \draw[math]
                (m-1-1) edge[tailed] (m-1-2)
                    edge[curve={height=30pt}, swap] node {\Omega (g \circ f)} (m-3-1)
                    edge node {\Omega f} (m-2-1)
                (m-1-2) edge[two headed] (m-1-3)
                    edge node {p_f} (m-2-2)
                (m-1-3) edge node {f} (m-2-3)

                (m-2-1) edge[tailed] (m-2-2)
                    edge node {\Omega g} (m-3-1)
                (m-2-2) edge[two headed] (m-2-3)
                    edge node {p_g} (m-3-2)
                (m-2-3) edge node {g} (m-3-3)

                (m-3-1) edge[tailed] (m-3-2)
                (m-3-2) edge[two headed] (m-3-3);
        \end{tikzpicture}
    \end{center}

    Considering the composition of the vertical morphisms, we end up with the following commutative diagram
    \begin{center}
        \begin{tikzpicture}
            \diagram{m}{1cm}{2cm} {
                {\Omega A} \& {P_A} \& A \\
                {\Omega C} \& {P_C} \& C. \\
            };

            \draw[math]
                (m-1-1) edge[tailed] node {\iota_A} (m-1-2)
                    edge[swap] node {\Omega g \circ \Omega f} (m-2-1)
                (m-1-2) edge[two headed] node {\pi_A} (m-1-3)
                    edge node {p_g \circ p_f} (m-2-2)
                (m-1-3) edge node {g \circ f} (m-2-3)

                (m-2-1) edge[tailed] node {\iota_C} (m-2-2)
                (m-2-2) edge[two headed] node {\pi_C} (m-2-3);
        \end{tikzpicture}
    \end{center}
    Since \( \Omega \) is a well-defined assignment of morphisms, this implies
    \[
        (\Omega [g]) \circ (\Omega [f]) = \class{(\Omega g) \circ (\Omega f)} = \Omega \class{g \circ f}.
    \]

    Second, need to show that \( \Omega \Id_A = \Id_{\Omega A} \) in \( \Mc \).

    By a similar argument to above, we can see that every square and triangle in the following diagram commutes
    \begin{center}
        \begin{tikzpicture}
            \diagram{m}{1cm}{2cm} {
                {\Omega A} \& {P_A} \& A \\
                {\Omega A} \& {P_A} \& A. \\
            };

            \draw[math]
                (m-1-1) edge[tailed] (m-1-2)
                    edge[swap] node {\Id_{\Omega A}} (m-2-1)
                (m-1-2) edge[two headed] (m-1-3)
                    edge node {\Id_{P_A}} (m-2-2)
                (m-1-3) edge node {\Id_A} (m-2-3)

                (m-2-1) edge[tailed] (m-2-2)
                (m-2-2) edge[two headed] (m-2-3);
        \end{tikzpicture}
    \end{center}
    Therefore \( \Omega \class{\Id_A} = \class{\Id_{\Omega A}} \).
\end{proof}

An important detail still missing to show that \( \Omega^{-1} \) could be a shift functor is showing that \( \Omega \) is additive.

\begin{lemma}
    \label{lem:stmod_omega_additive_functor}
    \( \Omega \) is an additive functor.
\end{lemma}
\begin{proof}
    Want to show that \( \Omega [f + g] = \Omega [f] + \Omega [g] \).
    
    There are two posssible values of \( \Omega (f + g) \), namely the ones from either of the two commutative diagrams below
    \begin{center}
        \begin{tikzpicture}
            \diagram{m}{1cm}{2cm} {
                {\Omega A} \& {P_A} \& A \\
                {\Omega B} \& {P_B} \& B \\
            };

            \draw[math]
                (m-1-1) edge[tailed] node {\iota_A} (m-1-2)
                    edge node {\Omega (f + g)} (m-2-1)
                (m-1-2) edge[two headed] node {\pi_A} (m-1-3)
                    edge node {p_{f + g}} (m-2-2)
                (m-1-3) edge node[swap] {f + g} (m-2-3)

                (m-2-1) edge[tailed] node {\iota_B} (m-2-2)
                (m-2-2) edge[two headed] node {\pi_B} (m-2-3);
        \end{tikzpicture}
        \begin{tikzpicture}
            \diagram{m}{1cm}{2cm} {
                {\Omega A} \& {P_A} \& A \\
                {\Omega B} \& {P_B} \& B. \\
            };

            \draw[math]
                (m-1-1) edge[tailed] node {\iota_A} (m-1-2)
                    edge node {\Omega f + \Omega g} (m-2-1)
                (m-1-2) edge[two headed] node {\pi_A} (m-1-3)
                    edge node {p_f + p_g} (m-2-2)
                (m-1-3) edge node[swap] {f + g} (m-2-3)

                (m-2-1) edge[tailed] node {\iota_B} (m-2-2)
                (m-2-2) edge[two headed] node {\pi_B} (m-2-3);
        \end{tikzpicture}
    \end{center}

    Since \( \Omega [f + g] \) is well-defined, this implies \( \Omega [f + g] = \Omega [f] + \Omega [g] \).
\end{proof}

We can define another functor, \( \Sigma \), which will turn out to be the inverse of \( \Omega \) and the future shift functor in making \( \Mc \) a triangulated category.

% TODO: Kopling med cosyzygy?
\begin{definition}[The functor \( \Sigma \)]
    \label{def:stmod_sigma}
    Since \( \Mod(R) \) has enough projectives, let \( I \) be a collection of tuples for every object \( A \), on the form \( I = \set{\tuple{I_A, \kappa_A, \Omega A}}_{A \in \Mod(R)} \), consisting of an injective module \( I_A \), a morphism \( \kappa_A \), and a module \( \Sigma A \), where \( \kappa_A \) is a monomorphism from \( A \) to \( P_A \), and \( \Sigma A \) is a cokernel of \( \kappa_A \).

    Then define \( \Sigma \) as the assignment of objects and morphisms in \( \Mc \) as follows:
    \begin{itemize}
        \item {
            For any object \( A \) in \( \Mc \), let \( \Sigma A \) be as defined in \( I \) above.
        }
        \item {
            For \( [f] \in \Mc(A, B) \), let \( \Sigma [f] \) be constructed as follows.

            Consider the following diagram in \( \Mod(R) \), with the relevant objects and non-dashed morphisms from \( I \),
            \begin{center}
                \begin{tikzpicture}
                    \diagram{m}{1cm}{1cm} {
                        A \& I_A \& \Sigma A \\
                        B \& I_B \& \Sigma B. \\
                    };

                    \draw[math]
                        (m-1-1) edge[tailed] node {\kappa_A} (m-1-2)
                            edge node {f} (m-2-1)
                        (m-1-2) edge[two headed] node {\rho_A} (m-1-3)
                            edge[dashed] node {i_f} (m-2-2)
                        (m-1-3) edge[dashed] node {\Sigma f} (m-2-3)

                        (m-2-1) edge[tailed] node {\kappa_B} (m-2-2)
                        (m-2-2) edge[two headed] node {\rho_B} (m-2-3);
                \end{tikzpicture}
            \end{center}

            We have that for a morphism \( f: A \to B \), there exists a morphism \( i_f \) from the universal property of injective objects such that the left square commutes. Note that this morphism is not neccesarily unique.

            Furthermore, since
            \[
                \kappa_B \circ i_f \circ \rho_A = \kappa_B \circ \rho_B \circ f = 0,
            \]
            we have from the universal cokernel property that there exists a unique morphism, dependant on the choice of \( i_f \), denoted as \( \Sigma f \) from \( \Sigma A \) to \( \Sigma B \) such that the right square in the above diagram commutes. Taking the equivalence class of this morphism with respect to morphisms factoring through projectives yields the morphism defined by the functor, i.e., \( \Sigma [f] := \class{\Sigma f} \).
        }
    \end{itemize}
\end{definition}

Similar to \( \Omega \), we have the following notation.
\begin{notation}
    \label{not:stmod_sigma_f}
    For \( f \in \Mod(R)(A, B) \), let \( \Sigma f \) refer to a morphism which fits the diagram in \autoref{def:stmod_sigma}. Note that this is not a well-defined assignment, as multiple morphisms in \( \Mod(R)(\Sigma A, \Sigma B) \) could be labeled as \( \Sigma f \).
\end{notation}

Now it remains to show every desired property of \( \Sigma \), like being well-defined, an endofunctor, and additive.
\begin{lemma}
    \label{lem:stmod_sigma_well-defined_additive_endofunctor}
    \( \Sigma \) is a well-defined and additive endofunctor on \( \Mc \).
\end{lemma}
\begin{proof}
    The proofs of the various statements are entirely dual to the proofs of \autoref{lem:stmod_omega_f_is_well_defined}, \autoref{lem:stmod_omega_endofunctor} and \autoref{lem:stmod_omega_additive_functor}.
\end{proof}

Now we can finally show that \( \Sigma = \Omega^{-1} \).

\begin{theorem}
    \( \Omega \) is an auto-equivalence with inverse \( \Sigma \).
\end{theorem}
\begin{proof}
    We will only show that \( \Id_{\Mc} \) is naturally isomorphic to \( \Sigma \Omega \). The omitted part that \( \Id_{\Mc} \) is naturally isomorphic to \( \Omega \Sigma \) is very similar, and uses many dual properties.

    First show that for any \( A \in \Mc \), there exists an isomorphism \( A \to \Sigma \Omega A \). Consider the following commutative diagram excluding the dashed arrows,
    \begin{center}
        \begin{tikzpicture}
            \diagram{m}{1cm}{2cm} {
                \Omega A \& P_A \& A \\
                \Omega A \& I_{\Omega A} \& \Sigma \Omega A \\
                \Omega A \& P_A \& A, \\
            };

            \draw[math]
                (m-1-1) edge[tailed] node {\iota_A} (m-1-2)
                    edge[equality] (m-2-1)
                (m-1-2) edge[two headed] node {\pi_A} (m-1-3)
                    edge[dashed] node {i_{\phi_1}} (m-2-2)
                (m-1-3) edge[dashed] node {\phi_1} (m-2-3)

                (m-2-1) edge[tailed] node {\kappa_{\Omega A}} (m-2-2)
                    edge[equality] (m-3-1)
                (m-2-2) edge[two headed] node {\rho_{\Omega A}} (m-2-3)
                    edge[dashed] node {i_{\phi_2}} (m-3-2)
                (m-2-3) edge[dashed] node {\phi_2} (m-3-3)

                (m-3-1) edge[tailed] node {\iota_A} (m-3-2)
                (m-3-2) edge[two headed] node {\pi_A} (m-3-3);
        \end{tikzpicture}
    \end{center}
    then there exists \( i_{\phi_1} \) from the injective property of \( I_{\Omega A} \), induced by \( \iota_A \). In addition, since \( \Mod(R) \) is an abelian category, \( A \) is a cokernel of \( \iota_A \), and by
    \[
        \rho_{\Omega A} \circ i_{\phi_1} \circ \iota_A = \rho_{\Omega A} \circ \kappa_{\Omega A} = 0,
    \]
    we get from the cokernel property that there is a uniquely induced morphism \( \phi_1: A \to \Sigma\Omega A \). Then doing the same for the lower rectangle of the diagram, using the fact that every projective is also injective, then we get the morphisms \( i_{\phi_2} \) and \( \phi_2 \) by similar arguments, such that the diagram above including the dashed arrows commute.

    Then to show that \( \phi_1 \) and \( \phi_2 \) are isomorphisms, consider the following diagram
    \begin{center}
        \begin{tikzpicture}
            \diagram{m}{1cm}{2cm} {
                \Omega A \& P_A \& A \\
                \Omega A \& P_A \& A. \\
            };

            \draw[math]
                (m-1-1) edge[tailed] node {\iota_A} (m-1-2)
                    edge[swap] node {\Id_{\Omega A} \circ \Id_{\Omega A} - \Id_{\Omega A} = 0} (m-2-1)
                (m-1-2) edge[two headed] node {\pi_A} (m-1-3)
                    edge[swap] node {i_{\phi_2} \circ i_{\phi_1} - \Id_{P_A}} (m-2-2)
                (m-1-3) edge[swap, dashed] node {\theta} (m-2-2)
                    edge node {\phi_2 \circ \phi_1 - \Id_A} (m-2-3)

                (m-2-1) edge[tailed] node {\iota_A} (m-2-2)
                (m-2-2) edge[two headed] node {\pi_A} (m-2-3);
        \end{tikzpicture}
    \end{center}

    Using the previous commutative diagram, we get that every square commutes. But since
    \[
        (i_{\phi_2} \circ i_{\phi_1} - \Id_{P_A}) \circ \iota_A = \iota_A \circ 0 = 0,
    \]
    then from the cokernel property there exist a morphism \( \theta: A \to P_A \) such that the upper triangle commutes. But then we have that
    \[
        (\phi_2 \circ \phi_1 - \Id_A) \circ \pi_A = \pi_A \circ (i_{\theta_2} \circ i_{\theta_1} - \Id_{P_A}) = \pi_A \circ \theta \circ \pi_A,
    \]
    and since \( \pi_A \) is an epimorphism, we get that the lower triangle commutes. This implies that \( [\phi_2 \circ \phi_1 - \Id_A] = [0] \), which implies \( [\phi_2 \circ \phi_1] = [\Id_A] \).
    
    By a similar agument that is omitted for brevity, we can show that \( [\phi_1 \circ \phi_2] = [\Id_{\Sigma\Omega A}] \), which means that \( [\phi_1] \) and \( [\phi_2] \) are isomorphisms from \( A \) to \( \Sigma\Omega A \).

    To show that these isomorphisms are natural, let \( [f] \in \Mc(A, B) \) and consider the following two diagrams
    \begin{center}
        \begin{tikzpicture}
            \diagram{m}{1cm}{2cm} {
                \Omega A \& P \& A \\
                \Omega A \& I \& \Sigma \Omega A \\
                \Omega B \& I \& \Sigma \Omega B, \\
            };

            \draw[math]
                (m-1-1) edge[tailed] node {\iota_A} (m-1-2)
                    edge[equality] (m-2-1)
                (m-1-2) edge[two headed] node {\pi_A} (m-1-3)
                    edge node {i_{\phi^A_1}} (m-2-2)
                (m-1-3) edge node {\phi^A_1} (m-2-3)

                (m-2-1) edge[tailed] node {\kappa_{\Omega A}} (m-2-2)
                    edge node {\Omega f} (m-3-1)
                (m-2-2) edge[two headed] node {\rho_{\Omega A}} (m-2-3)
                    edge node {i_{\Omega f}} (m-3-2)
                (m-2-3) edge node {\Sigma \Omega f} (m-3-3)

                (m-3-1) edge[tailed] node {\kappa_{\Omega B}} (m-3-2)
                (m-3-2) edge[two headed] node {\rho_{\Omega B}} (m-3-3);
        \end{tikzpicture}
    \end{center}
    and
    \begin{center}
        \begin{tikzpicture}
            \diagram{m}{1cm}{2cm} {
                \Omega A \& P \& A \\
                \Omega B \& P \& B \\
                \Omega B \& I \& \Sigma \Omega B, \\
            };

            \draw[math]
                (m-1-1) edge[tailed] node {\iota_A} (m-1-2)
                    edge node {\Omega f} (m-2-1)
                (m-1-2) edge[two headed] node {\pi_A} (m-1-3)
                    edge node {p_f} (m-2-2)
                (m-1-3) edge node {f} (m-2-3)

                (m-2-1) edge[tailed] node {\iota_B} (m-2-2)
                    edge[equality] (m-3-1)
                (m-2-2) edge[two headed] node {\pi_B} (m-2-3)
                    edge node {i_{\phi_1^B}} (m-3-2)
                (m-2-3) edge node {\phi_1^B} (m-3-3)

                (m-3-1) edge[tailed] node {\kappa_{\Omega B}} (m-3-2)
                (m-3-2) edge[two headed] node {\rho_{\Omega B}} (m-3-3);
        \end{tikzpicture}
    \end{center}
    where every small square, and therefore rectangle, commutes.

    These diagrams gives rise to the following commutative diagram, excluding the dashed arrow,
    \begin{center}
        \begin{tikzpicture}
            \diagram{m}{1cm}{2.7cm} {
                \Omega A \& P \& A \\
                \Omega B \& I \& \Sigma \Omega B, \\
            };

            \draw[math]
                (m-1-1) edge[tailed] node {\iota_A} (m-1-2)
                    edge[swap] node {\Id_{\Omega B} \circ (\Omega f) - (\Omega f) \circ \Id_{\Omega A} = 0} (m-2-1)
                (m-1-2) edge[two headed] node {\pi_A} (m-1-3)
                    edge[swap] node {i_{\phi_1^B} \circ p_f - i_{\Omega f} \circ i_{\phi_1^A}} (m-2-2)
                (m-1-3) edge[swap, dashed] node {\theta} (m-2-2)
                    edge node {\phi_1^B \circ f - (\Sigma \Omega f) \circ \phi_1^A} (m-2-3)

                (m-2-1) edge[tailed] node[swap] {\kappa_{\Omega B}} (m-2-2)
                (m-2-2) edge[two headed] node[swap] {\rho_{\Omega B}} (m-2-3);
        \end{tikzpicture}
    \end{center}
    % TODO: Sebastian synast eg skal fordjupa her, men det er eit veldig likt argument som det eg har gjort ovanfor fleire gangar. Usikker på kva eg skal gjere her.
    where from the cokernel property of \( A \), we get an induced morphism \( \theta \). Furthermore, from the epimorphism property of \( \pi_A \), the lower triangle commutes. And since \( R \) is assumed to be a Frobenius ring, \( I \) is also projective. This implies
    \[
        [\phi_1^B \circ f] = [(\Sigma \Omega f) \circ \phi_1^A],
    \]
    which means that \( [\phi_1] \) is a natural isomorphism from \( \Id_{\Mc} \) to \( \Sigma \Omega \), with inverse \( [\phi_2] \).

    As mentioned at the start, the proof that \( \Id_{\Mc} \) is naturally isomorphic to \( \Omega \Sigma \) is very similar to the above proof, but using dual properties.
\end{proof}

An interesting consequence of how \( \Omega \) and \( \Sigma \) are defined is that if we were to chose a different \( P \) or \( I \) in their definitions, then it would yield different, but naturally isomorphic functors. A proof of this statement for \( \Sigma \) can be found in \cite[Remark on p. 13]{Happel_1988}, with the proof for \( \Omega \) being dual.

Before we can show the triangulation of \( \Mc \), we first needs to define the cone of a morphism. The definition of a cone, as well as the definition of the distinguished triangles leans heavily upon \cite[Chapter 1, Subsection 2.5]{Happel_1988}.

% TODO: Slå i saman denne definisjonen og remarket nedanfor?
\begin{definition}
    \label{def:stmod_cone}
    Let \( f \in \Mod(R)(A, B) \).
    
    Then define a \emph{cone of \( f \)} to be a pushout object of the following diagram
    \begin{center}
        \begin{tikzpicture}
            \diagram{m}{1cm}{1cm} {
                A \& B \\
                I_A, \\
            };

            \draw[math]
                (m-1-1) edge node {f} (m-1-2)
                    edge node[swap] {\kappa_A} (m-2-1);
        \end{tikzpicture}
    \end{center}
    and is labeled as \( C_f \).

    This pushout also defines two morphisms \( g \) and \( \gamma_f \) which fit into the following pushout square,
    \begin{center}
        \begin{tikzpicture}
            \diagram{m}{1cm}{1cm} {
                A \& B \\
                I_A \& C_f. \\
            };

            \draw[math]
                (m-1-1) edge node {f} (m-1-2)
                    edge node[swap] {\kappa_A} (m-2-1)
                (m-1-2) edge node {g} (m-2-2)

                (m-2-1) edge node {\gamma_f} (m-2-2);
        \end{tikzpicture}
    \end{center}
\end{definition}

Following the definition of a cone, we can define the standard triangles of \( \Mc \) as induced from some triangles in \( \Mod(R) \). The following remark walks through the construction, which will be very useful in proofs.

\begin{remark}
    \label{rem:stmod_cone}
    Let \( f \in \Mod(R)(A, B) \), and let \( C_f, g \) and \( \gamma_f \) be as in \autoref{def:stmod_cone}.
    
    Consider the following commutative diagram excluding the dashed arrows,
    \begin{center}
        \begin{tikzpicture}
            \diagram{m}{1cm}{1cm} {
                A \& B \\
                I_A \& C_f \\
                \& \Sigma A. \\
            };

            \draw[math]
                (m-1-1) edge node {f} (m-1-2)
                    edge[swap] node {\kappa_A} (m-2-1)
                (m-1-2) edge node {g} (m-2-2)
                    edge[curve={height=-25pt}] node {0} (m-3-2)

                (m-2-1) edge node {\gamma_f} (m-2-2)
                    edge node {\rho_A} (m-3-2)
                (m-2-2) edge[dashed] node {h} (m-3-2);
        \end{tikzpicture}
    \end{center}
    Then there exist some morphism \( h: C_f \to \Sigma A \) from the pushout property of \( C_f \).

    This diagram will form the backbone of the distinguished triangles in \( \Mc \).
\end{remark}

Now, we can finally define the triangulation on \( \Mc \).

\begin{definition}
    \label{def:stmod_delta}
    Let \( \Delta \) be the collection of triangles in \( \Mc \) isomorphic to any triangle of the form
    \begin{center}
        \begin{tikzpicture}
            \diagram{m}{1cm}{1cm} {
                A \& B \& C_f \& \Sigma A \\
            };

            \draw[math]
                (m-1-1) edge node {[f]} (m-1-2)
                (m-1-2) edge node {[g]} (m-1-3)
                (m-1-3) edge node {[h]} (m-1-4);
        \end{tikzpicture}
    \end{center}
    for any \( f \in \Mod(R)(A, B) \), and where \( C_f \), \( g \), and \( h \) are defined as in \autoref{rem:stmod_cone}.
\end{definition}

There are some important details from the definition of standard triangles that will be important in proving that \( \Mc \) has a triangulation.

% TODO: Sebastian vil at eg skal sløyfa "Consider the objects and ..." setningen. Skal eg?
\begin{remark}
    \label{rem:stmod_cone_pushout_properties}
    Consider the objects and morphisms of \autoref{rem:stmod_cone}.

    Since \( \Mod(R) \) is an abelian category, then by known pushout properties, it follows that since \( \kappa_A \) is a monomorphism, then \( g \) is also a monomorphism since \( \ker(g) = \ker(\kappa_A) = 0 \). Likewise it follows that \( \coker(g) \cong \coker(\kappa_A) \cong \Sigma A \), and therefore \( h \) is a cokernel morphism of \( g \).
\end{remark}

Since morphisms in \( \Mc \) are residue classes and can therefore have multiple representatives, the the cone of a morphism in \( \Mc \) is not neccesarily unique. This is not a problem for us, since by the formulation of \( \Delta \), any standard triangle arising from any representation of \( [f] \) is included in \( \Delta \), and so there are multiple ``standard triangles'' for every morphism in \( \Mc \).

The following lemma is neccesary in proving {\bf (TR4)}, however, it also hints to the fact that we could have chosen \( \Delta \) differently, as is done in \cite[Definition 4.16]{Johan_Bachelor}.
\begin{lemma}
    \label{lem:stmod_pushout_different_injectives_isomorphic}
    Let
    \[
        \begin{aligned}
            \begin{tikzpicture}
                \diagram{m}{1cm}{1cm} {
                    A \& B \\
                    I \& C \\
                };
    
                \draw[math]
                    (m-1-1) edge node {f} (m-1-2)
                        edge node {g} (m-2-1)
                    (m-1-2) edge node {g'} (m-2-2)
    
                    (m-2-1) edge node {\gamma} (m-2-2);
            \end{tikzpicture}
        \end{aligned}
        \text{ and }
        \begin{aligned}
            \begin{tikzpicture}
                \diagram{m}{1cm}{1cm} {
                    A \& B \\
                    I' \& C' \\
                };
    
                \draw[math]
                    (m-1-1) edge node {f} (m-1-2)
                        edge node {h} (m-2-1)
                    (m-1-2) edge node {h'} (m-2-2)
    
                    (m-2-1) edge node {\gamma'} (m-2-2);
            \end{tikzpicture}
        \end{aligned}
    \]
    be two pushout squares in \( \Mod(R) \), with \( g \) and \( h \) monomorphisms into injective objects \( I \) and \( I' \).

    Then there exists an isomorphism \( [\alpha]: C \to C' \) in \( \Mc \), with inverse \( [\alpha^{-1}] \).
    
    In addition, \( \alpha \) and \( \alpha^{-1} \) have the following properties:
    \begin{itemize}
        \item \( \alpha \circ g' = h' \),
        \item \( \alpha \circ \gamma = \gamma' \circ i \),
        \item \( g' = \alpha^{-1} \circ h' \), and
        \item \( \gamma = \alpha^{-1} \circ \gamma' \circ i \),
    \end{itemize}
    where \( i \) is defined in the beginning of the proof.
\end{lemma}
\begin{proof}
    Since both \( I \) and \( I' \) are injective objects and \( g \) and \( h \) monomorphisms, then there exists maps \( i \) and \( i' \) such that the following two diagrams commute
    \[
        \begin{aligned}
            \begin{tikzpicture}
                \diagram{m}{1cm}{0.5cm} {
                    \& A \\
                    I \& \& I' \\
                };

                \draw[math]
                    (m-1-2) edge node {g} (m-2-1)
                        edge node {h} (m-2-3)

                    (m-2-1) edge node {i} (m-2-3);
            \end{tikzpicture}
        \end{aligned}
        \text{ and }
        \begin{aligned}
            \begin{tikzpicture}
                \diagram{m}{1cm}{0.5cm} {
                    \& A \\
                    I \& \& I'. \\
                };

                \draw[math]
                    (m-1-2) edge node {g} (m-2-1)
                        edge node {h} (m-2-3)

                    (m-2-3) edge node {i'} (m-2-1);
            \end{tikzpicture}
        \end{aligned}
    \]
    Using those morphisms, we can create the following commutative diagram, where \( \alpha \) and \( \beta \) are the induced morphisms from the pushout property of \( C \) and \( C' \) respectively, 
    \begin{center}
        \begin{tikzpicture}
            \diagramorigin{m}{1cm}{1.5cm} {
                A \& B \\
                I \& C \\
                I' \& \& C' \\
                I \& \& \& C. \\
            };

            \draw[math]
                (m-1-1) edge node {f} (m-1-2)
                    edge[curve={height=50pt}] node {g} (m-4-1)
                    edge[curve={height=25pt}] node {h} (m-3-1)
                    edge node {g} (m-2-1)
                (m-1-2) edge node {g'} (m-2-2)
                    edge[curve={height=-25pt}] node {h'} (m-3-3)
                    edge[curve={height=-50pt}] node {g'} (m-4-4)

                (m-2-1) edge node {\gamma} (m-2-2)
                    edge node {i} (m-3-1)
                (m-2-2) edge[dashed] node {\alpha} (m-3-3)

                (m-3-1) edge node {\gamma'} (m-3-3)
                    edge node {i'} (m-4-1)
                (m-3-3) edge[dashed] node {\beta} (m-4-4)

                (m-4-1) edge node {\gamma} (m-4-4);
        \end{tikzpicture}
    \end{center}
    The goal is to show that \( [\beta \circ \alpha] = [\Id_C] \).

    By definition of a pushout, the following squence is exact
    \begin{center}
        \begin{tikzpicture}
            \diagram{m}{1cm}{1cm} {
                A \& B \oplus I \& C. \\
            };

            \draw[math]
                (m-1-1) edge node {
                    \begin{psmallmatrix}
                        f \\
                        g
                    \end{psmallmatrix}
                } (m-1-2)
                (m-1-2) edge node {
                    \begin{psmallmatrix}
                        g' & \gamma
                    \end{psmallmatrix}
                } (m-1-3);
        \end{tikzpicture}
    \end{center}
    Consider the following commutative diagram excluding the dashed arrow
    \begin{diagramlabel}[\label{diag:C-iso}]
        \begin{tikzpicture}
            \diagram{m}{1cm}{1cm} {
                A \& B \oplus I \& C \\
                \& I \& C, \\
            };

            \draw[math]
                (m-1-1) edge node {
                    \begin{psmallmatrix}
                        f \\
                        g
                    \end{psmallmatrix}
                } (m-1-2)
                (m-1-2) edge node {
                    \begin{psmallmatrix}
                        g' & \gamma
                    \end{psmallmatrix}
                } (m-1-3)
                (m-1-2) edge node[swap] {
                    \begin{psmallmatrix}
                        0 & i' \circ i - \Id_I
                    \end{psmallmatrix}
                } (m-2-2)
                (m-1-3) edge[dashed] node[swap] {\delta} (m-2-2)
                    edge node {\beta \circ \alpha - \Id_C} (m-2-3)

                (m-2-2) edge node {\gamma} (m-2-3);
        \end{tikzpicture}
    \end{diagramlabel}
    since
    \[
        \begin{psmallmatrix}
            0 & i' \circ i - \Id_I
        \end{psmallmatrix}
        \circ
        \begin{psmallmatrix}
            f \\
            g
        \end{psmallmatrix}
        = 0
    \]
    then by the cokernel property of \( C \) there exist a morphism \( \delta \) such that the top triangle in \autoref{diag:C-iso} commutes. However, since \( 
        \begin{psmallmatrix}
            g' & \gamma
        \end{psmallmatrix}
    \) is an epimorphism by definition, it follows that the bottom triangle also commutes.

    We can prove that \( [\alpha \circ \beta] = [\Id_{C'}] \) in a similar way.

    Let \( \alpha^{-1} := \beta \) to get the statement of the lemma.
\end{proof}

Then finally comes the full triangulation result, the proof of which is heavily inspired by \cite[First theorem in Chapter 1, Subsection 2.6]{Happel_1988}, with additional inspiration for proving {\bf (TR4)} from \cite[Theorem 4.18]{Johan_Bachelor}.

% TODO: Kommenter at den nøyaktige triangulerte strukturen ikkje blir brukt seinare i oppgåva. Etterpå vert berre eksistensen av ein triangulering brukt.

% TODO: Sebastian kommenterte på "... it is a known fact that the pushout of an iso is an iso ..." Med "appeal to common belief". Kva skal eg gjere med det? Burde eg ha med nokre pushout eigenskapar i forkant?
\begin{example}
    \label{example:stable_module_category_triangulated}
    The tuple \( \tuple{\Mc, \Sigma, \Delta} \) is a triangulated category.
\end{example}
\begin{proof} % TODO: Skriv om bevisa, standariser notasjon, kanskje inkluder lemmaet over inn i TR4? Kanskje formuler pushout-unikheit eigenskapen på ein kortare måte, kanskje eit lemma?
    We need to prove {\bf (TR1)} to {\bf (TR4)} from \autoref{def:triangulated_category}.

    \begin{enumerate}[label={(\bfseries TR\arabic*)}]
        \item {
            \begin{enumerate}
                \item {
                    Let \( [f] \in \Mc(A, B) \).
                    
                    Then by the definition of the distinguished triangles in \( \Mc \), the following triangle is distinguished.
                    \begin{center}
                        \begin{tikzpicture}
                            \diagram{m}{1cm}{1cm} {
                                A \& B \& C_f \& \Sigma A. \\
                            };

                            \draw[math]
                                (m-1-1) edge node {[f]} (m-1-2)
                                (m-1-2) edge node {[g]} (m-1-3)
                                (m-1-3) edge node {[h]} (m-1-4);
                        \end{tikzpicture}
                    \end{center}
                }
                \item {
                    Let \( A \in \Mc \).
                    
                    Calculating \( C_{\Id_A} \) we get the pushout
                    \begin{center}
                        \begin{tikzpicture}
                            \diagram{m}{1cm}{1cm} {
                                A \& A \\
                                I_A \& C_{\Id}. \\
                            };

                            \draw[math]
                                (m-1-1) edge node {\Id} (m-1-2)
                                    edge node {\kappa_A} (m-2-1)
                                (m-1-2) edge (m-2-2)

                                (m-2-1) edge node {\gamma_{\Id}} (m-2-2);
                        \end{tikzpicture}
                    \end{center}
                    Since the pushout of an isomorphism is an isomorphism, \( \gamma_{\Id} \) is an isomorphism, which implies \( C_{\Id} \cong 0 \) in \( \Mc \) because all injective modules are projective in \( \Mc \).
                }
                \item {
                    \( \Delta \) is closed under isomorphisms of triangles directly by the definition of \( \Delta \).
                }
            \end{enumerate}
        }
        \item {
            % TODO: Motiver dette beviset mykje betre.
            We need to show \( (\Leftarrow) \) and \( (\Rightarrow) \). By \autoref{lem:triangulated_category-TR2-only_one_rotation}, assuming the upcoming proof of {\bf (TR3)} as well as the previous proof of {\bf (TR1)}, then \( (\Leftarrow) \) is implied by \( (\Rightarrow) \).
            
            Therefore it is sufficient to only prove \( (\Rightarrow) \).

            First note that a left shifted distinguished triangle will be isomorphic to a left shifted standard triangle. Therefore it suffices to check that every left shifted standard triangle is distinguished.

            Consider the following standard triangle
            \begin{center}
                \begin{tikzpicture}
                    \diagram{m}{1cm}{1cm} {
                        A \& B \& C_f \& \Sigma A \\
                    };

                    \draw[math]
                        (m-1-1) edge node {[f]} (m-1-2)
                        (m-1-2) edge node {[g]} (m-1-3)
                        (m-1-3) edge node {[h]} (m-1-4);
                \end{tikzpicture}
            \end{center}

            Keep in mind the following commutative diagrams given by the definition of \( \Sigma [f] \) (\autoref{def:stmod_sigma}) and the construction of the above standard triangle (\autoref{rem:stmod_cone}),
            \[
                \begin{aligned}
                    \begin{tikzpicture}
                        \diagram{m}{1cm}{1cm} {
                            A \& I_A \& \Sigma A \\
                            B \& I_B \& \Sigma B, \\
                        };

                        \draw[math]
                            (m-1-1) edge[tailed] node {\kappa_A} (m-1-2)
                                edge node {f} (m-2-1)
                            (m-1-2) edge[two headed] node {\rho_A} (m-1-3)
                                edge node {i_f} (m-2-2)
                            (m-1-3) edge node {\Sigma f} (m-2-3)

                            (m-2-1) edge[tailed] node {\kappa_B} (m-2-2)
                            (m-2-2) edge[two headed] node {\rho_B} (m-2-3);
                    \end{tikzpicture}
                \end{aligned}
                \quad \text{and} \quad
                \begin{aligned}
                    \begin{tikzpicture}
                        \diagram{m}{1cm}{1cm} {
                            A \& B \\
                            I_A \& C_f \\
                            \& \Sigma A. \\
                        };

                        \draw[math]
                            (m-1-1) edge node {f} (m-1-2)
                                edge[tailed] node {\kappa_A} (m-2-1)
                            (m-1-2) edge node{g} (m-2-2)

                            (m-2-1) edge node {\gamma_f} (m-2-2)
                                edge node {\rho_A} (m-3-2)
                            (m-2-2) edge node {h} (m-3-2);
                    \end{tikzpicture}
                \end{aligned}
            \]
            With the above diagrams in mind, consider the following commutative diagram excluding the dahed arrow,
            \begin{center}
                \begin{tikzpicture}
                    \diagram{m}{1cm}{1cm} {
                        A \& B \\
                        I_A \& C_f \& I_B \& \Sigma B \\
                    };

                    \draw[math]
                        (m-1-1) edge node {f} (m-1-2)
                            edge[tailed] node {\kappa_A} (m-2-1)
                        (m-1-2) edge node{g} (m-2-2)
                            edge[tailed] node {\kappa_B} (m-2-3)

                        (m-2-1) edge node {\gamma_f} (m-2-2)
                            edge[curve={height=1cm}] node {i_f} (m-2-3)
                        (m-2-2) edge[dashed] node {\phi} (m-2-3)

                        (m-2-3) edge node{\rho_B} (m-2-4);
                \end{tikzpicture}
            \end{center}
            where \( \phi \) is given by the pushout universal property.

            Consider the following equalities that follow from the above commutative diagrams
            \[
                \rho_B \circ \phi \circ g = \rho_B \circ \kappa_B = 0 = \Sigma f \circ h \circ g,
            \]
            and
            \[
                \rho_B \circ \phi \circ \gamma_f = \rho_B \circ i_f = \Sigma f \circ \rho_A = \Sigma f \circ h \circ \gamma_f.
            \]
            This implies that considering the following pushout diagram
            \begin{center}
                \begin{tikzpicture}
                    \diagram{m}{1cm}{1cm} {
                        A \& B \\
                        I_A \& C_f \& \Sigma B \\
                    };

                    \draw[math]
                        (m-1-1) edge node {f} (m-1-2)
                            edge[tailed] node {\kappa_A} (m-2-1)
                        (m-1-2) edge node{g} (m-2-2)
                            edge[tailed] node {0} (m-2-3)

                        (m-2-1) edge node {\gamma_f} (m-2-2)
                            edge[curve={height=1cm}] node {\rho_B \circ i_f} (m-2-3)
                        (m-2-2) edge[dashed] (m-2-3);
                \end{tikzpicture}
            \end{center}
            would have two different morphisms, \( \rho_B \circ \phi \) and \( \Sigma f \circ h \), that could satisfy the pushout universal property. By uniqueness, they therefore have to be the same morphism, i.e.,
            \[
                \rho_B \circ \phi = \Sigma f \circ h.
            \]
            Finally, consider the following commutative diagram
            \begin{center}
                \begin{tikzpicture}
                    \diagram{m}{1cm}{1cm} {
                        0 \& B \& C_f \& \Sigma A \& 0 \\
                        0 \& I_B \& I_B \oplus \Sigma A \& \Sigma A \& 0 \\
                        \& \& \Sigma B \\
                    };

                    \draw[math]
                        (m-1-1) edge (m-1-2)
                            edge[equality] (m-2-1)
                        (m-1-2) edge node {g} (m-1-3)
                            edge node {\kappa_B} (m-2-2)
                        (m-1-3) edge node {h} (m-1-4)
                            edge node {
                                \begin{psmallmatrix}
                                    \phi \\
                                    h
                                \end{psmallmatrix}
                            } (m-2-3)
                        (m-1-4) edge (m-1-5)
                                edge[equality] (m-2-4)

                        (m-2-1) edge (m-2-2)
                        (m-2-2) edge node {
                            \begin{psmallmatrix}
                                1 \\
                                0
                            \end{psmallmatrix}
                        } (m-2-3)
                            edge node {\rho_B} (m-3-3)
                        (m-2-3) edge node {
                            \begin{psmallmatrix}
                                0 & 1
                            \end{psmallmatrix}
                        } (m-2-4)
                            edge node {
                                \begin{psmallmatrix}
                                    \rho_B & -\Sigma f
                                \end{psmallmatrix}
                            } (m-3-3)
                        (m-2-4) edge (m-2-5);
                \end{tikzpicture}
            \end{center}
            where the top row is exact by \autoref{rem:stmod_cone_pushout_properties}, and the second row is split-exact. It follows that the middle square is in fact a pushout, which implies \( I_B \oplus \Sigma A  \) is uniquely isomorphic to \( C_g \). % TODO: Cite Notat, klassifisering av pushout og pullback i abelsk kategori.
            
            And since
            \[
                \begin{psmallmatrix}
                    \rho_B & -\Sigma f
                \end{psmallmatrix}
                \begin{psmallmatrix}
                    \phi \\
                    h
                \end{psmallmatrix}
                =
                \rho_B \circ \phi - \Sigma f \circ h = 0,
            \]
            the triangle
            \begin{center}
                \begin{tikzpicture}
                    \diagram{m}{1cm}{1cm} {
                        B \& C_f \& I_B \oplus \Sigma A \& \Sigma B \\
                    };

                    \draw[math]
                        (m-1-1) edge node {[g]} (m-1-2)
                        (m-1-2) edge node {\class{
                            \begin{psmallmatrix}
                                \phi \\
                                h
                            \end{psmallmatrix}
                        }} (m-1-3)
                        (m-1-3) edge node {\class{
                            \begin{psmallmatrix}
                                \rho_B & -\Sigma f
                            \end{psmallmatrix}
                        }} (m-1-4);
                \end{tikzpicture}
            \end{center} 
            is a standard triangle of \( g \), and therefore distinguished.

            Finally it remains to check if the triangle is isomorphic to the expected triangle. Consider the following diagram in \( \Mc \)
            \begin{center}
                \begin{tikzpicture}
                    \diagram{m}{1cm}{1cm} {
                        B \& C_f \& I_B \oplus \Sigma A \& \Sigma B \\
                        B \& C_f \& \Sigma A \& \Sigma B. \\
                    };

                    \draw[math]
                        (m-1-1) edge node {[g]} (m-1-2)
                            edge[equality] (m-2-1)
                        (m-1-2) edge node {\class{
                            \begin{psmallmatrix}
                                \phi \\
                                h
                            \end{psmallmatrix}
                        }} (m-1-3)
                            edge[equality] (m-2-2)
                        (m-1-3) edge node {\class{
                            \begin{psmallmatrix}
                                \rho_B & -\Sigma f
                            \end{psmallmatrix}
                        }} (m-1-4)
                            edge node {\class{
                                \begin{psmallmatrix}
                                    0 & 1
                                \end{psmallmatrix}
                            }} (m-2-3)
                        (m-1-4) edge[equality] (m-2-4)

                        (m-2-1) edge node {[g]} (m-2-2)
                        (m-2-2) edge node {[h]} (m-2-3)
                        (m-2-3) edge node {[-\Sigma f]} (m-2-4);
                \end{tikzpicture}
            \end{center}
            The left and the middle square commute directly, but it remains to check if the right square commutes. In addition we need to check that \( \class{
                \begin{psmallmatrix}
                    0 & 1
                \end{psmallmatrix}
            } \) is an isomorphism.

            First, check the commutativity. Consider the difference
            \[
                \begin{psmallmatrix}
                    \rho_B & -\Sigma f
                \end{psmallmatrix}
                -
                -(\Sigma f) \circ
                \begin{psmallmatrix}
                    0 & 1
                \end{psmallmatrix}
                =
                \begin{psmallmatrix}
                    \rho_B & -\Sigma f
                \end{psmallmatrix}
                +
                \begin{psmallmatrix}
                    0 & \Sigma f
                \end{psmallmatrix}
                =
                \begin{psmallmatrix}
                    \rho_B & 0
                \end{psmallmatrix}.
            \]
            We can see that the diagram
            \begin{center}
                \begin{tikzpicture}
                    \diagram{m}{1cm}{1cm} {
                        I_B \oplus \Sigma A \& \& \Sigma B \\
                        \& I_B \\
                    };

                    \draw[math]
                        (m-1-1) edge node {
                            \begin{psmallmatrix}
                                \rho_B & 0
                            \end{psmallmatrix}
                        } (m-1-3)
                            edge node {
                                \begin{psmallmatrix}
                                    1 & 0
                                \end{psmallmatrix}
                            } (m-2-2)

                        (m-2-2) edge node {\rho_B} (m-1-3);
                \end{tikzpicture}
            \end{center}
            commutes, and since every injective module is also projective, \(\class{
                    \begin{psmallmatrix}
                        \rho_B & -\Sigma f
                    \end{psmallmatrix}
                }
                =
                \class{
                    -(\Sigma f) \circ
                    \begin{psmallmatrix}
                        0 & 1
                    \end{psmallmatrix}
                }
            \)

            Second, we check that \( \class{
                \begin{psmallmatrix}
                    0 & 1
                \end{psmallmatrix}
            } \) is an isomorphism.

            Consider the morphism \( \class{
                \begin{psmallmatrix}
                    0 \\
                    1
                \end{psmallmatrix}
            } \).
            It is already known that \( \class{
                \begin{psmallmatrix}
                    0 & 1
                \end{psmallmatrix}
            } \circ \class{
                \begin{psmallmatrix}
                    0 \\
                    1
                \end{psmallmatrix}
            } = [\Id_A] \).
            Then it remains to check if
            \[
                \Id_{I_B \oplus \Sigma A} -
                \begin{psmallmatrix}
                    0 \\
                    1
                \end{psmallmatrix}
                \begin{psmallmatrix}
                    0 & 1
                \end{psmallmatrix}
                =
                \begin{psmallmatrix}
                    1 & 0 \\
                    0 & 0
                \end{psmallmatrix}
            \]
            factors through a projective.

            Consider the following diagram
            \begin{center}
                \begin{tikzpicture}
                    \diagram{m}{1cm}{1cm} {
                        I_B \oplus \Sigma A \& \& I_B \oplus \Sigma A \\
                        \& I_B. \\
                    };

                    \draw[math]
                        (m-1-1) edge node {
                            \begin{psmallmatrix}
                                1 & 0 \\
                                0 & 0
                            \end{psmallmatrix}
                        } (m-1-3)
                            edge node {
                                \begin{psmallmatrix}
                                    1 & 0
                                \end{psmallmatrix}
                            } (m-2-2)

                        (m-2-2) edge node {
                            \begin{psmallmatrix}
                                1 \\
                                0
                            \end{psmallmatrix}
                        } (m-1-3);
                \end{tikzpicture}
            \end{center}
            It commutes.
        }
        \item {
            By considering two arbitrary distinguished triangles, any morphism between their components will induce a unique morphism between the component of their standard triangles they are isomorphic to, and vice versa. Therefore it suffices to only check {\bf (TR3)} for standard triangles.

            In addition, by the argument made in {\bf (TR2)}, we can not assume {\bf (TR2)} (\( \Leftarrow \)) in this proof, as that would yield a circular argument.

            We need to show that given the following diagram excluding the dashed arrow where the top and botton row are standard triangles,
            \begin{diagramlabel}[\label{eq:stablemod}]
                \begin{tikzpicture}
                    \diagram{m}{1cm}{1cm} {
                        A \& B  \& C_f \& \Sigma A \\
                        D \& E \& C_l \& \Sigma D, \\
                    };

                    \draw[math]
                        (m-1-1) edge node {[f]} (m-1-2)
                            edge node {[\alpha]} (m-2-1)
                        (m-1-2) edge node {[g]} (m-1-3)
                            edge node {[\beta]} (m-2-2)
                        (m-1-3) edge node {[h]} (m-1-4)
                            edge[dashed] node {[\phi]} (m-2-3)
                        (m-1-4) edge node {\Sigma [\alpha]} (m-2-4)

                        (m-2-1) edge node {[l]} (m-2-2)
                        (m-2-2) edge node {[m]} (m-2-3)
                        (m-2-3) edge node {[n]} (m-2-4);
                \end{tikzpicture}
            \end{diagramlabel}
            that there exists some \( \phi: C_f \to C_{l} \) such that the entire diagram including \( \phi \), commutes.

            Consider the following commutative diagrams in \( \Mod(R) \) from the definition of the two standard triangles
            \begin{center}
                \begin{tikzpicture}
                    \diagram{m}{1cm}{1cm} {
                        A \& B \\
                        I_A \& C_f \\
                        \& \Sigma A, \\
                    };

                    \draw[math]
                        (m-1-1) edge node {f} (m-1-2)
                            edge[tailed] node {\kappa_A} (m-2-1)
                        (m-1-2) edge node{g} (m-2-2)

                        (m-2-1) edge node {\gamma_f} (m-2-2)
                            edge node {\rho_A} (m-3-2)
                        (m-2-2) edge node {h} (m-3-2);
                \end{tikzpicture}
                %
                \begin{tikzpicture}
                    \diagram{m}{1cm}{1cm} {
                        D \& E \\
                        I_{D} \& C_{l} \\
                        \& \Sigma D, \\
                    };

                    \draw[math]
                        (m-1-1) edge node {l} (m-1-2)
                            edge[tailed] node {\kappa_{D}} (m-2-1)
                        (m-1-2) edge node {m} (m-2-2)

                        (m-2-1) edge node {\gamma_{l}} (m-2-2)
                            edge node {\rho_{D}} (m-3-2)
                        (m-2-2) edge node {n} (m-3-2);
                \end{tikzpicture}
            \end{center}
            then since \( [l] \circ [\alpha] = [\beta] \circ [f] \) in \( \Mc \), we have that \( l \circ \alpha - \beta \circ f \) factors through a projective. However, since projectives are injectives, then by the universal property of injective objects applied to \( \kappa_A \), it follows that \( l \circ \alpha - \beta \circ f \) factors through \( \kappa_A \) and that there is some morphism \( \xi: I_A \to E \) such that
            \[
                l \circ \alpha - \beta \circ f = \xi \circ \kappa_A.
            \]

            % TODO: Formuler dette betre. Eg trur det framleis er alt for muntleg.
            Furthermore, consider the following diagram by definition of \( \Sigma [\alpha] \)
            \begin{center}
                \begin{tikzpicture}
                    \diagram{m}{1cm}{1cm} {
                        A \& I_A \& \Sigma A \\
                        D \& I_{D} \& \Sigma D, \\
                    };

                    \draw[math]
                        (m-1-1) edge[tailed] node {\kappa_A} (m-1-2)
                            edge node {\alpha} (m-2-1)
                        (m-1-2) edge[two headed] node {\rho_A} (m-1-3)
                            edge node {i_{\alpha}} (m-2-2)
                        (m-1-3) edge node {\Sigma \alpha} (m-2-3)

                        (m-2-1) edge[tailed] node {\kappa_{D}} (m-2-2)
                        (m-2-2) edge[two headed] node {\rho_{D}} (m-2-3);
                \end{tikzpicture}
            \end{center}
            then we can construct the morphism which will be used below
            \[
                \gamma_{l} \circ i_{\alpha} - m \circ \xi: I_A \to C_{l}.
            \]

            Since
            \begin{align*}
                m \circ \beta \circ f &= m \circ l \circ \alpha - m \circ \xi \circ \kappa_A \\
                &= \gamma_{l} \circ \kappa_{D} \circ \alpha - m \circ \xi \circ \kappa_A \\
                &= \gamma_{l} \circ i_{\alpha} \circ \kappa_A - m \circ \xi \circ \kappa_A \\
                &= (\gamma_{l} \circ i_{\alpha} - m \circ \xi) \circ \kappa_A
            \end{align*}
            then by the pushout property of \( C_f \) it follows that there exists some unique \( \phi \) such that the following diagram commutes
            \begin{center}
                \begin{tikzpicture}
                    \diagram{m}{1cm}{1cm} {
                        A \& B \\
                        I_A \& C_f \\
                        \& \& C_{l}. \\
                    };

                    \draw[math]
                        (m-1-1) edge node {f} (m-1-2)
                            edge node {\kappa_A} (m-2-1)
                        (m-1-2) edge node {g} (m-2-2)
                            edge[curve={height=-25pt}] node {m \circ \beta} (m-3-3)

                        (m-2-1) edge node {\gamma_f} (m-2-2)
                            edge[curve={height=25pt}] node[swap] {\gamma_{l} \circ i_{\alpha} - m \circ \xi} (m-3-3)
                        (m-2-2) edge[dashed] node {\phi} (m-3-3);
                \end{tikzpicture}
            \end{center}
            In particular \( \phi \circ \gamma_f = \gamma_{l} \circ i_{\alpha} - m \circ \xi \).

            Finally, check that this \( \phi \) makes \autoref{eq:stablemod} commute.

            By the commutativity of the pushout diagram, the middle square of \autoref{eq:stablemod} commutes. Then it remains to check if \( [n \circ \phi] = [(\Sigma \alpha) \circ h] \). Consider the following pushout diagram that commutes because \( (\Sigma \alpha) \circ h \circ \gamma_f \circ \kappa_A = (\Sigma \alpha) \circ \rho_A \circ \kappa_A = 0 \),
            \begin{center}
                \begin{tikzpicture}
                    \diagram{m}{1cm}{1cm} {
                        A \& B \\
                        I_A \& C_f \\
                        \& \& \Sigma D, \\
                    };

                    \draw[math]
                        (m-1-1) edge node {f} (m-1-2)
                            edge node {\kappa_A} (m-2-1)
                        (m-1-2) edge node {g} (m-2-2)
                            edge[curve={height=-25pt}] node {0} (m-3-3)

                        (m-2-1) edge node {\gamma_f} (m-2-2)
                            edge[curve={height=25pt}] node[swap] {(\Sigma \alpha) \circ h \circ \gamma_f} (m-3-3)
                        (m-2-2) edge[dashed] (m-3-3);
                \end{tikzpicture}
            \end{center}
            and the following two equations
            \[
                n \circ \phi \circ g = n \circ m \circ \beta = 0 = (\Sigma \alpha) \circ h \circ g,
            \]
            and
            \[
                (\Sigma \alpha) \circ h \circ \gamma_f = (\Sigma \alpha) \circ \rho_A = \rho_{D} \circ i_{\alpha} = n \circ \gamma_{l} \circ i_{\alpha} = n \circ (\phi \circ \gamma_f + m \circ \xi) = n \circ \phi \circ \gamma_f.
            \]
            These implies that there are two choices of dashed line in the above diagram that would make it commute. However, by uniqueness, this implies they are equal and therefore
            \[
                (\Sigma \alpha) \circ h = n \circ \phi.
            \]
        }
        \item {
            % TODO: Swap m og n?
            By a similar argument as in {\bf (TR3)} it is sufficient to only check {\bf (TR4)} for standard triangles.

            Consider three standard triangles
            \begin{center}
                \begin{tikzpicture}
                    \diagram{m}{1cm}{1cm} {
                        A \& B \& C_f \& \Sigma A, \\
                        B \& D \& C_n \& \Sigma B, \& \text{and} \\
                        A \& D \& C_{n \circ f} \& \Sigma A, \\
                    };

                    \draw[math]
                        (m-1-1) edge node {[f]} (m-1-2)
                        (m-1-2) edge node {[g]} (m-1-3)
                        (m-1-3) edge node {[h]} (m-1-4)

                        (m-2-1) edge node {[n]} (m-2-2)
                        (m-2-2) edge node {[m]} (m-2-3)
                        (m-2-3) edge node {[k]} (m-2-4)

                        (m-3-1) edge node {[n \circ f]} (m-3-2)
                        (m-3-2) edge node {[j]} (m-3-3)
                        (m-3-3) edge node {[l]} (m-3-4);
                \end{tikzpicture}
            \end{center}
            which fit into the following commutative diagram excluding the dashed arrows
            \begin{center}
                \begin{tikzpicture}
                    \diagram{m}{1cm}{1cm} {
                        A \& B \& C_f \& \Sigma A \\
                        A \& D \& C_{n \circ f} \& \Sigma A \\
                        \& C_n \& C_n \& \Sigma B \\
                        \& \Sigma B \& \Sigma C_f. \\
                    };
        
                    \draw[math]
                        (m-1-1) edge node {[f]} (m-1-2)
                            edge[equality] (m-2-1)
                        (m-1-2) edge node {[g]} (m-1-3)
                            edge node {[n]} (m-2-2)
                        (m-1-3) edge node {[h]} (m-1-4)
                            edge[dashed] node {[\phi]} (m-2-3)
                        (m-1-4) edge[equality] (m-2-4)
        
                        (m-2-1) edge node {[n \circ f]} (m-2-2)
                        (m-2-2) edge node {[j]} (m-2-3)
                            edge node {[m]} (m-3-2)
                        (m-2-3) edge node {[l]} (m-2-4)
                            edge[dashed] node {[\psi]} (m-3-3)
                        (m-2-4) edge node {\Sigma [f]} (m-3-4)

                        (m-3-2) edge[equality] (m-3-3)
                            edge node[swap] {[k]} (m-4-2)
                        (m-3-3) edge node {[k]} (m-3-4)
                            edge node {[(\Sigma g) \circ k]} (m-4-3)

                        (m-4-2) edge node {\Sigma [g]} (m-4-3);
                \end{tikzpicture}
            \end{center}
            We need to show that there exists morphisms \( [\phi] \) and \( [\psi] \) that fit into the above diagram and commutes, such that
            \begin{diagramlabel}[\label{tri:stmod_tr4}]
                \begin{tikzpicture}
                    \diagram{m}{1cm}{1cm} {
                        C_f \& C_{n \circ f} \& C_n \& \Sigma C_f \\
                    };

                    \draw[math]
                        (m-1-1) edge node {[\phi]} (m-1-2)
                        (m-1-2) edge node {[\psi]} (m-1-3)
                        (m-1-3) edge node {[(\Sigma g) \circ k]} (m-1-4);
                \end{tikzpicture}
            \end{diagramlabel}
            is a distinguished triangle.

            In order to construct the proper \( [\phi] \) and \( [\psi] \), we will work with the following commutative diagram, excluding the dahsed arrows, in \( \Mod(R) \)
            \begin{diagramlabel}[\label{diag:stmod_tr4}]
                \begin{tikzpicture}
                    \diagram{m}{1cm}{1cm} {
                        A \& B \& C_f \& \Sigma A \\
                        A \& D \& C_{n \circ f} \& \Sigma A \\
                        \& C_n \& C_n \& \Sigma B \\
                        \& \Sigma B \& \Sigma C_f. \\
                    };
        
                    \draw[math]
                        (m-1-1) edge node {f} (m-1-2)
                            edge[equality] (m-2-1)
                        (m-1-2) edge node {g} (m-1-3)
                            edge node {n} (m-2-2)
                        (m-1-3) edge node {h} (m-1-4)
                            edge[dashed] node {\phi} (m-2-3)
                        (m-1-4) edge[equality] (m-2-4)
        
                        (m-2-1) edge node {n \circ f} (m-2-2)
                        (m-2-2) edge node {j} (m-2-3)
                            edge node {m} (m-3-2)
                        (m-2-3) edge node {l} (m-2-4)
                            edge[dashed] node {\psi} (m-3-3)
                        (m-2-4) edge node {\Sigma f} (m-3-4)

                        (m-3-2) edge[equality] (m-3-3)
                            edge node[swap] {k} (m-4-2)
                        (m-3-3) edge node {k} (m-3-4)
                            edge node {(\Sigma g) \circ k} (m-4-3)

                        (m-4-2) edge node {\Sigma g} (m-4-3);
                \end{tikzpicture}
            \end{diagramlabel}
            If we can find some \( \phi \) and \( \psi \) along with appropriate \( \Sigma f \) such that the above diagram commutes, then all we would have left to prove is that \autoref{tri:stmod_tr4} is a distinguished triangle.

            Consider the following commutative diagrams from the constructions of the three standard triangles mentioned above (\autoref{rem:stmod_cone}),
            \begin{center}
                \begin{tikzpicture}
                    \diagram{m}{1cm}{1cm} {
                        A \& B \&[-0.5cm] B \& D \&[-0.5cm] A \& D \\
                        I_A \& C_f \& I_B \& C_n \& I_A \& C_{n \circ f} \\
                        \& \Sigma A, \& \& \Sigma B, \& \& \Sigma A, \\
                    };

                    \draw[math]
                        (m-1-1) edge node {f} (m-1-2)
                            edge node {\kappa_A} (m-2-1)
                        (m-1-2) edge node {g} (m-2-2)
                        (m-1-3) edge node {n} (m-1-4)
                            edge node {\kappa_B} (m-2-3)
                        (m-1-4) edge node {m} (m-2-4)
                        (m-1-5) edge node {n \circ f} (m-1-6)
                            edge node {\kappa_A} (m-2-5)
                        (m-1-6) edge node {j} (m-2-6)

                        (m-2-1) edge node {\gamma_f} (m-2-2)
                            edge node {\rho_A} (m-3-2)
                        (m-2-2) edge node {h} (m-3-2)
                        (m-2-3) edge node {\gamma_n} (m-2-4)
                            edge node {\rho_B} (m-3-4)
                        (m-2-4) edge node {k} (m-3-4)
                        (m-2-5) edge node {\gamma_{n \circ f}} (m-2-6)
                            edge node {\rho_A} (m-3-6)
                        (m-2-6) edge node {l} (m-3-6);
                \end{tikzpicture}
            \end{center}
            as well as the following short exact sequence from the definition of \( \Sigma C_f \),
            \begin{center}
                \begin{tikzpicture}
                    \diagram{m}{1cm}{1cm} {
                        C_f \& I_{C_f} \& \Sigma C_f. \\
                    };

                    \draw[math]
                        (m-1-1) edge[tailed] node {\kappa_{C_f}} (m-1-2)
                        (m-1-2) edge[two headed] node {\rho_{C_f}} (m-1-3); 
                \end{tikzpicture}
            \end{center}

            Similar, but not equal, to what was done in the proof of {\bf (TR3)}, let \( \phi \) be the (dashed) pushout morphism arising from the following commutative diagram excluding the dashed arrow
            \begin{center}
                \begin{tikzpicture}
                    \diagram{m}{1cm}{1cm} {
                        A \& B \\
                        I_A \& C_f \\
                        \& \& C_{n \circ f}. \\
                    };

                    \draw[math]
                        (m-1-1) edge node {f} (m-1-2)
                            edge node {\kappa_A} (m-2-1)
                        (m-1-2) edge node {g} (m-2-2)
                            edge[curve={height=-25pt}] node {j \circ n} (m-3-3)
                        
                        (m-2-1) edge node {\gamma_f} (m-2-2)
                            edge[curve={height=25pt}] node[swap] {\gamma_{n \circ f}} (m-3-3)
                        (m-2-2) edge[dashed] node {\phi} (m-3-3);
                \end{tikzpicture}
            \end{center}
            By definition, the square left of \( \phi \) in \autoref{diag:stmod_tr4} commutes. In order to check if the square to the right commutes, consider the following commutative diagram excluding the dashed arrow
            \begin{center}
                \begin{tikzpicture}
                    \diagram{m}{1cm}{1cm} {
                        A \& B \\
                        I_A \& C_f \\
                        \& \& \Sigma A. \\
                    };

                    \draw[math]
                        (m-1-1) edge node {f} (m-1-2)
                            edge node {\kappa_A} (m-2-1)
                        (m-1-2) edge node {g} (m-2-2)
                            edge[curve={height=-25pt}] node {l \circ j \circ n} (m-3-3)
                        
                        (m-2-1) edge node {\gamma_f} (m-2-2)
                            edge[curve={height=25pt}] node[swap] {l \circ \gamma_{n \circ f}} (m-3-3)
                        (m-2-2) edge[dashed] (m-3-3);
                \end{tikzpicture}
            \end{center}
            Consider the following equations
            \[
                l \circ \phi \circ g = l \circ j \circ n = n \circ 0 = 0 = h \circ g,
            \]
            and
            \[
                l \circ \phi \circ \gamma_f = l \circ \gamma_{n \circ f} = \rho_A = h \circ \gamma_f,
            \]
            which, by the pushout morphism uniqueness, implies that
            \[
                l \circ \phi = h.
            \]
            And therefore the square to the right of \( \phi \) in \autoref{diag:stmod_tr4_C} also commutes.

            In order to construct \( \psi \) we have to first define two morphisms which will be used in its definition.

            By \autoref{rem:stmod_cone_pushout_properties} it follows that \( g \) is a monomorphism. And since \( \kappa_{C_f} \) is also a monomorphism, it follows that
            \[
                \kappa_{C_f} \circ g: B \to I_{C_f}
            \]
            is also a monomorphism. Then by the injective object property of \( I_B \), there exist some morphism \( i: I_{C_f} \to I_B \) which makes the following diagram commute,
            \begin{center}
                \begin{tikzpicture}
                    \diagram{m}{1cm}{0.5cm} {
                        \& B \\
                        I_{C_f} \& \& I_B. \\
                    };

                    \draw[math]
                        (m-1-2) edge node[swap] {\kappa_{C_f} \circ g} (m-2-1)
                            edge node {\kappa_B} (m-2-3)

                        (m-2-1) edge node {i} (m-2-3);
                \end{tikzpicture}
            \end{center}
            By the injective property of \( I_{C_f} \), there also exists some \( \tilde{i} \) such that the following diagram commutes,
            \begin{center}
                \begin{tikzpicture}
                    \diagram{m}{1cm}{0.5cm} {
                        \& B \\
                        I_{C_f} \& \& I_B. \\
                    };

                    \draw[math]
                        (m-1-2) edge node[swap] {\kappa_{C_f} \circ g} (m-2-1)
                            edge node {\kappa_B} (m-2-3)

                        (m-2-3) edge node {\tilde{i}} (m-2-1);
                \end{tikzpicture}
            \end{center}
            
            Now we can move on to defining \( \psi \).

            Consider the following diagram excluding the dashed arrow,
            \begin{center}
                \begin{tikzpicture}
                    \diagram{m}{1cm}{1cm} {
                        A \& D \\
                        I_A \& C_{n \circ f} \\
                        \& \& C_n, \\
                    };

                    \draw[math]
                        (m-1-1) edge node {n \circ f} (m-1-2)
                            edge node {\kappa_A} (m-2-1)
                        (m-1-2) edge node {j} (m-2-2)
                            edge[curve={height=-25pt}] node {m} (m-3-3)

                        (m-2-1) edge node {\gamma_{n \circ f}} (m-2-2)
                            edge[curve={height=25pt}] node[swap] {\gamma_n \circ i \circ \kappa_{C_f} \circ \gamma_f} (m-3-3)
                        (m-2-2) edge[dashed] node {\psi} (m-3-3);
                \end{tikzpicture}
            \end{center}
            which commutes because
            \begin{equation}
                \label{diag:stmod_tr4_m_n_f_equal_something_else}
                \begin{aligned}
                    m \circ n \circ f &= \gamma_n \circ \kappa_B \circ f \\
                    &= \gamma_n \circ i \circ \kappa_{C_f} \circ g \circ f \\
                    &= \gamma_n \circ i \circ \kappa_{C_f} \circ \gamma_f \circ \kappa_A.
                \end{aligned}
            \end{equation}
            Thus, by the pushout property, there exists a morphism \( \psi: C_{n \circ f} \to C_n \) such that the above diagram including the dashed arrow commutes.

            To show that \( \psi \) makes \autoref{diag:stmod_tr4} commute in \( \Mc \), we can see that the square to the left of \( \psi \) commutes by definition, and so it remains to choose some \( \Sigma f \) such that the square to the right also commutes.

            Consider the following diagram excluding the dashed arrow,
            \begin{center}
                \begin{tikzpicture}
                    \diagram{m}{1cm}{1cm} {
                        A \& I_A \& \Sigma A \\
                        \& C_f \\
                        \& I_{C_f} \\
                        B \& I_B \& \Sigma B, \\
                    };

                    \draw[math]
                        (m-1-1) edge node {\kappa_A} (m-1-2)
                            edge node[swap] {f} (m-4-1)
                        (m-1-2) edge node {\rho_A} (m-1-3)
                            edge node {\gamma_f} (m-2-2)
                        (m-1-3) edge[dashed] node {\Sigma f} (m-4-3)

                        (m-2-2) edge node {\kappa_{C_f}} (m-3-2)

                        (m-3-2) edge node {i} (m-4-2)

                        (m-4-1) edge node {\kappa_B} (m-4-2)
                            edge node {g} (m-2-2)
                        (m-4-2) edge node {\rho_B} (m-4-3);
                \end{tikzpicture}
            \end{center}
            which commutes because the top left and bottom left ``triangles'' of the diagram commute. Then since
            \[
                \rho_B \circ i \circ \kappa_{C_f} \circ \gamma_f \circ \kappa_A = \rho_B \circ \kappa_B = 0,
            \]
            it follows by the cokernel property of \( \Sigma A \) that there exist some morphism, \( \Sigma f \), which is denoted as such because it fits \autoref{not:stmod_sigma_f}, and by \autoref{lem:stmod_sigma_well-defined_additive_endofunctor} is therefore a representative of \( \Sigma [f] \).

            Then consider the following commutative diagram excluding the dashed arrow,
            \begin{center}
                \begin{tikzpicture}
                    \diagram{m}{1cm}{1cm} {
                        A \& D \\
                        I_A \& C_{n \circ f} \\
                        \& \& \Sigma B, \\
                    };

                    \draw[math]
                        (m-1-1) edge node {n \circ f} (m-1-2)
                            edge node {\kappa_A} (m-2-1)
                        (m-1-2) edge node {j} (m-2-2)
                            edge[curve={height=-25pt}] node {0} (m-3-3)

                        (m-2-1) edge node {\gamma_{n \circ f}} (m-2-2)
                            edge[curve={height=25pt}] node[swap] {k \circ \gamma_n \circ i \circ \kappa_{C_f} \circ \gamma_f} (m-3-3)
                        (m-2-2) edge[dashed] (m-3-3);
                \end{tikzpicture}
            \end{center}
            where
            \[
                \Sigma f \circ l \circ j = \Sigma f \circ 0 = 0 = k \circ m = k \circ \psi \circ j
            \]
            and
            \begin{align*}
                k \circ \psi \circ \gamma_{n \circ f} &= k \circ \gamma_n \circ i \circ \kappa_{C_f} \circ \gamma_f \\
                &= \rho_B \circ i \circ \kappa_{C_f} \circ \gamma_f \\
                &= \Sigma f \circ \rho_A \\
                &= \Sigma f \circ l \circ \gamma_{n \circ f}.
            \end{align*}
            Then by uniqueness of the pushout property it follows that \( k \circ \psi = \Sigma f \circ l \).
                
            Therefore the square to the right of \( \psi \) in \autoref{diag:stmod_tr4} commutes.

            Finally, it remains to check that
            \begin{center}
                \begin{tikzpicture}
                    \diagram{m}{1cm}{1cm} {
                        C_f \& C_{n \circ f} \& C_n \& \Sigma C_f \\
                    };

                    \draw[math]
                        (m-1-1) edge node {[\phi]} (m-1-2)
                        (m-1-2) edge node {[\psi]} (m-1-3)
                        (m-1-3) edge node {\Sigma([g]) \circ [k]} (m-1-4);
                \end{tikzpicture}
            \end{center}
            is a distinguished triangle.

            Consider the following commutative diagram
            \begin{center}
                \begin{tikzpicture}
                    \diagram{m}{1cm}{1cm} {
                        A \& B \& D \\
                        I_A \& C_f \& C_{n \circ f}. \\
                    };

                    \draw[math]
                        (m-1-1) edge node {f} (m-1-2)
                            edge node {\kappa_A} (m-2-1)
                        (m-1-2) edge node {n} (m-1-3)
                            edge node {g} (m-2-2)
                        (m-1-3) edge node {j} (m-2-3)

                        (m-2-1) edge node {\gamma_f} (m-2-2)
                            edge[curve={height=25pt}] node[swap] {\gamma_{n \circ f}} (m-2-3)
                        (m-2-2) edge node {\phi} (m-2-3);
                \end{tikzpicture}
            \end{center}
            Since the left and the outer square are pushouts this implies that the right square is also a pushout.

            Consider the following pushout diagram of \( (B, n, \kappa_{C_f} \circ g) \),
            \begin{diagramlabel}[\label{diag:stmod_tr4_C}]
                \begin{tikzpicture}
                    \diagram{m}{1cm}{1cm} {
                        B \& D \\
                        I_{C_f} \& C, \\
                    };

                    \draw[math]
                        (m-1-1) edge node {n} (m-1-2)
                            edge node[swap] {\kappa_{C_f} \circ g} (m-2-1)
                        (m-1-2) edge node {\delta} (m-2-2)

                        (m-2-1) edge node {\gamma} (m-2-2);
                \end{tikzpicture}
            \end{diagramlabel}
            as well as this previous pushout diagram from the definition of \( C_n \),
            \begin{center}
                \begin{tikzpicture}
                    \diagram{m}{1cm}{1cm} {
                        B \& D \\
                        I_B \& C_n. \\
                    };

                    \draw[math]
                        (m-1-1) edge node {n} (m-1-2)
                            edge node[swap] {\kappa_B} (m-2-1)
                        (m-1-2) edge node {m} (m-2-2)

                        (m-2-1) edge node {\gamma_n} (m-2-2);
                \end{tikzpicture}
            \end{center}
            By \autoref{lem:stmod_pushout_different_injectives_isomorphic} it follows that there exists some \( \alpha: C \to C_n \) such that the following equalities hold:
            \begin{itemize}
                \item \( \alpha \circ \delta = m \),
                \item \( \alpha \circ \gamma = \gamma_n \circ i \),
                \item \( \delta = \alpha^{-1} \circ m \), and
                \item \( \gamma = \alpha^{-1} \circ \gamma_{n} \circ i \).
            \end{itemize}
            
            Consider the following diagram
            \begin{diagramlabel}[\label{diag:stmod_tr4_show_psi_phi_equal_something_else}]
                \begin{tikzpicture}
                    \diagram{m}{1cm}{1cm} {
                        B \& D \\
                        C_f \& C_{n \circ f} \\
                        I_{C_f} \& C \\
                    };
                    
                    \draw[math]
                        (m-1-1) edge node {n} (m-1-2)
                            edge node[swap] {g} (m-2-1)
                        (m-1-2) edge node[swap] {j} (m-2-2)
                            edge[curve={height=-25pt}] node {\delta} (m-3-2)

                        (m-2-1) edge node {\phi} (m-2-2)
                            edge node[swap] {\kappa_{C_f}} (m-3-1)
                        (m-2-2) edge node[swap] {\alpha^{-1} \circ \psi} (m-3-2)

                        (m-3-1) edge node {\gamma} (m-3-2);
                \end{tikzpicture}
            \end{diagramlabel}

            the top square already commutes, and the ``bump'' on the right commutes because \( \psi \circ j = m \). It remains to show that the bottom square commutes as well.

            In order to do that, consider the following diagram
            \begin{center}
                \begin{tikzpicture}
                    \diagram{m}{1cm}{1cm} {
                        A \& B \\
                        I_A \& C_f \\
                        \& \& C_n, \\
                    };

                    \draw[math]
                        (m-1-1) edge node {f} (m-1-2)
                            edge node[swap] {\kappa_A} (m-2-1)
                        (m-1-2) edge node {g} (m-2-2)
                            edge[curve={height=-25pt}] node {m \circ n} (m-3-3)

                        (m-2-1) edge node {\gamma_f} (m-2-2)
                            edge[curve={height=25pt}] node[swap] {\gamma_n \circ i \circ \kappa_{C_f} \circ \gamma_f} (m-3-3)
                        (m-2-2) edge[dashed] (m-3-3);
                \end{tikzpicture}
            \end{center}
            where the inner square is a pushout and the outer ``square'' commutes by \autoref{diag:stmod_tr4_m_n_f_equal_something_else}.

            Note the following equalities,
            \begin{align*}
                \psi \circ \phi \circ g &= \psi \circ j \circ n \\
                &= m \circ n \\
                &= \gamma_n \circ \kappa_B \\
                &= \gamma_n \circ i \circ \kappa_{C_f} \circ g,
            \end{align*}
            and
            \begin{align*}
                \psi \circ \phi \circ \gamma_f &= \psi \circ \gamma_{n \circ f} \\
                &= \gamma_n \circ i \circ \kappa_{C_f} \circ \gamma_f,
            \end{align*}
            which implies by the uniqueness of the pushout morphism that
            \[
                \psi \circ \phi = \gamma_n \circ i \circ \kappa_{C_f}.
            \]
            By post-composing with \( \alpha^{-1} \), the above equation implies
            \[
                \alpha^{-1} \circ \psi \circ \phi = \alpha^{-1} \circ \gamma_n \circ i \circ \kappa_{C_f} = \gamma \circ \kappa_{C_f}
            \]
            which makes \autoref{diag:stmod_tr4_show_psi_phi_equal_something_else} commute.
            
            Since the outer rectangle and upper squares of \autoref{diag:stmod_tr4_m_n_f_equal_something_else} are pushouts, then the lower square is also a pushout.
            
            This implies that there exists a morphism \( p \) by the pushout property, such that the following diagram
            \begin{center}
                \begin{tikzpicture}
                    \diagram{m}{1cm}{1cm} {
                        C_f \& C_{n \circ f} \\
                        I_{C_f} \& C \\
                        \& \Sigma C_f, \\
                    };

                    \draw[math]
                        (m-1-1) edge node {\phi} (m-1-2)
                            edge node[swap] {\kappa_{C_f}} (m-2-1)
                        (m-1-2) edge node[swap] {\alpha^{-1} \circ \psi} (m-2-2)
                            edge[curve={height=-25pt}] node {0} (m-3-2)

                        (m-2-1) edge node {\gamma} (m-2-2)
                            edge node[swap] {\rho_{C_f}} (m-3-2)
                        (m-2-2) edge node[swap] {p} (m-3-2);
                \end{tikzpicture}
            \end{center}
            commutes and yields a standard triangle
            \begin{center}
                \begin{tikzpicture}
                    \diagram{m}{1cm}{1cm} {
                        C_f \& C_{n \circ f} \& C \& \Sigma C_f. \\
                    };

                    \draw[math]
                        (m-1-1) edge node {[\phi]} (m-1-2)
                        (m-1-2) edge node {[\alpha^{-1} \circ \psi]} (m-1-3)
                        (m-1-3) edge node {[p]} (m-1-4);
                \end{tikzpicture}
            \end{center}

            The final step is then to show that the following diagram is an isomorphism of triangles
            \begin{center}
                \begin{tikzpicture}
                    \diagram{m}{1cm}{1cm} {
                        C_f \& C_{n \circ f} \& C \& \Sigma C_f \\
                        C_f \& C_{n \circ f} \& C_n \& \Sigma C_f. \\
                    };

                    \draw[math]
                        (m-1-1) edge node {[\phi]} (m-1-2)
                            edge[equality] (m-2-1)
                        (m-1-2) edge node {[\alpha^{-1} \circ \psi]} (m-1-3)
                            edge[equality] (m-2-2)
                        (m-1-3) edge node {[p]} (m-1-4)
                            edge node[swap] {[\alpha]} (m-2-3)
                        (m-1-4) edge[equality] (m-2-4)

                        (m-2-1) edge node {[\phi]} (m-2-2)
                        (m-2-2) edge node {[\psi]} (m-2-3)
                        (m-2-3) edge node {[(\Sigma g) \circ k]} (m-2-4);
                \end{tikzpicture}
            \end{center}

            The leftmost square commutes by difficult mathemathics, and the middle square commutes since \( [\alpha] \) is an isomorphism with inverse \( [\alpha^{-1}] \) by \autoref{lem:stmod_pushout_different_injectives_isomorphic}. It remains to show that the rightmost square commutes, which requires some additional tools.

            Let \( E := \coker(\delta)\). Since \autoref{diag:stmod_tr4_C} is a pushout, we also have that \( E = \coker(\kappa_{C_f} \circ g) \). Let \( c: I_{C_f} \to E \) be the cokernel morphism of \( \kappa_{C_f} \circ g \).

            Let \( e \) and \( e^{-1} \) be the (dashed) morphisms induced by the cokernel properties which fit in the following commutative diagram,
            \begin{center}
                \begin{tikzpicture}
                    \diagram{m}{1cm}{1cm} {
                        B \& I_B \& \Sigma B \\
                        B \& I_{C_f} \& E \\
                        B \& I_B \& \Sigma B. \\
                    };

                    \draw[math]
                        (m-1-1) edge node {\kappa_B} (m-1-2)
                            edge[equality] (m-2-1)
                        (m-1-2) edge node {\rho_B} (m-1-3)
                            edge node {\tilde{i}} (m-2-2)
                        (m-1-3) edge[dashed] node {e^{-1}} (m-2-3)

                        (m-2-1) edge node {\kappa_{C_f} \circ g} (m-2-2)
                            edge[equality] (m-3-1)
                        (m-2-2) edge node {c} (m-2-3)
                            edge node {i} (m-3-2)
                        (m-2-3) edge[dashed] node {e} (m-3-3)

                        (m-3-1) edge node {\kappa_B} (m-3-2)
                        (m-3-2) edge node {\rho_B} (m-3-3);
                \end{tikzpicture}
            \end{center}
            
            We can verify that \( [e \circ e^{-1}] = [\Id_B] \) by \autoref{lem:stmod_sigma_well-defined_additive_endofunctor}. Similarly we can verify that \( [e^{-1} \circ e] = [\Id_E] \).

            The above morphisms fit into the following commutative diagram, where \( d \) is the cokernel morphism of \( \delta \),
            \begin{center}
                \begin{tikzpicture}
                    \diagram{m}{1cm}{1cm} {
                        B \& D \\
                        I_{C_f} \& C \\
                        E \& E \\
                        \& \Sigma B. \\
                    };

                    \draw[math]
                        (m-1-1) edge node {n} (m-1-2)
                            edge node {\kappa_{C_f} \circ g} (m-2-1)
                        (m-1-2) edge node {\delta} (m-2-2)
                        
                        (m-2-1) edge node {\gamma} (m-2-2)
                            edge node {c} (m-3-1)
                        (m-2-2) edge node {d} (m-3-2)

                        (m-3-1) edge[equality] (m-3-2)
                        (m-3-2) edge node {e} (m-4-2);
                \end{tikzpicture}
            \end{center}
            
            With the above morphisms in mind, let \( \mu \) be the (dashed) morphism induced by the cokernel property in the following commutative diagram, and let \( \Sigma g := \mu \circ e^{-1} \),
            \begin{center}
                \begin{tikzpicture}
                    \diagram{m}{1cm}{1cm} {
                        B \& I_B \& \Sigma B \\
                        B \& I_{C_f} \& E \\
                        C_f \& I_{C_f} \& \Sigma C_f. \\
                    };

                    \draw[math]
                        (m-1-1) edge node {\kappa_B} (m-1-2)
                            edge[equality] (m-2-1)
                        (m-1-2) edge node {\rho_B} (m-1-3)
                            edge node {\tilde{i}} (m-2-2)
                        (m-1-3) edge node[swap] {e^{-1}} (m-2-3)
                            edge[dashed, curve={height=-25pt}] node {\Sigma g} (m-3-3)

                        (m-2-1) edge node {\kappa_{C_f} \circ g} (m-2-2)
                            edge node {g} (m-3-1)
                        (m-2-2) edge node {c} (m-2-3)
                            edge[equality] (m-3-2)
                        (m-2-3) edge[dashed] node[swap] {\mu} (m-3-3)

                        (m-3-1) edge node {\kappa_{C_f}} (m-3-2)
                        (m-3-2) edge node {\rho_{C_f}} (m-3-3);
                \end{tikzpicture}
            \end{center}

            Before proving \( [(\Sigma g) \circ k \circ \alpha] = [p] \), we first have to prove some relations between the new morphisms and the old ones.

            First note the following commutative diagram excluding the dashed arrow
            \begin{center}
                \begin{tikzpicture}
                    \diagram{m}{1cm}{1cm} {
                        B \& D \\
                        I_{C_f} \& C \\
                        \& \& \Sigma B \\
                    };

                    \draw[math]
                        (m-1-1) edge node {n} (m-1-2)
                            edge node[swap] {\kappa_{C_f} \circ g} (m-2-1)
                        (m-1-2) edge node {\delta} (m-2-2)
                            edge[curve={height=-25pt}] node {0} (m-3-3)

                        (m-2-1) edge node {\gamma} (m-2-2)
                            edge[curve={height=25pt}] node {\rho_B \circ i} (m-3-3)
                        (m-2-2) edge[dashed] (m-3-3);
                \end{tikzpicture}
            \end{center}
            where the outer ``square'' commutes because
            \[
                \rho_B \circ i \circ \kappa_{C_f} \circ g = \rho_B \circ \kappa_B = 0.
            \]
            Both \( e \circ d \) and \( k \circ \alpha \) could fit as the dashed arrow, because
            \[
                k \circ \alpha \circ \delta = k \circ m = 0 = 0 \circ \delta = e \circ d \circ \delta,
            \]
            and
            \[
                k \circ \alpha \circ \gamma = k \circ \gamma_n \circ i = \rho_B \circ i = e \circ c = e \circ d \circ \gamma.
            \]
            and therefore by uniqueness of the universal property of the pushout, \( e \circ d = k \circ \alpha \).

            Similarly to above, consider the following commutative diagram excluding the dashed arrow,
            \begin{center}
                \begin{tikzpicture}
                    \diagram{m}{1cm}{1cm} {
                        B \& D \\
                        I_{C_f} \& C \\
                        \& \& \Sigma C_f. \\
                    };

                    \draw[math]
                        (m-1-1) edge node {n} (m-1-2)
                            edge node[swap] {\kappa_{C_f} \circ g} (m-2-1)
                        (m-1-2) edge node {\delta} (m-2-2)
                            edge[curve={height=-25pt}] node {0} (m-3-3)

                        (m-2-1) edge node {\gamma} (m-2-2)
                            edge[curve={height=25pt}] node {\rho_{C_f}} (m-3-3)
                        (m-2-2) edge[dashed] (m-3-3);
                \end{tikzpicture}
            \end{center}

            By definition, \( p \) already fits as a dashed arrow. However, \( \mu \circ d \) also fits, by the following equations:
            \[
                \mu \circ d \circ \delta = \mu \circ 0 = 0
            \]
            and
            \[
               \mu \circ d \circ \gamma = \mu \circ c = \rho_{C_f}.
            \]
            Therefore by the uniqueness of the universal property of the pushout, \( p = \mu \circ d \).

            Combining everything, we get the following equation
            \[
                [(\Sigma g) \circ k \circ \alpha] = [(\Sigma g) \circ e \circ d] =[\mu \circ e^{-1} \circ e \circ d] = [\mu \circ d] = [p],
            \]
            which finishes the proof, since \( [\alpha] \) is an isomorphism. \qedhere
        }
    \end{enumerate}
\end{proof}

By \cite[Lemma, Subsection 7.5]{Krause_2007} it follows that \( \Mc \) is in fact what we call an \emph{algebraic triangulated category}. This will be touched on later in the thesis. % TODO: Specify where it will be touched on and the connection between Krause and Jasso-Muro's definition.