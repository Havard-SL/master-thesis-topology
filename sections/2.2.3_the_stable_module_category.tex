% TODO: Kan generaliserast til endeleg gen modular om R er noetherske. Og det verkar som frobenius ringar er noetherske. https://en.wikipedia.org/wiki/Quasi-Frobenius_ring Zimmermann: Prop. 5.1.4

A triangulated ctageory that will be central in this thesis is the stable module category. Therefore the definition will be given in more details than the previous two examples of triangulated categories.

Before showing the triangulated structure, it is as usual necessary to define the underlying category first. However, the underlying category require some lemmas to be well defined.

First, a small lemma that will be used to define the morphisms in the stable module category.

\begin{lemma}
    \label{lem:morphisms_factoring_through_projectives_r-submodule}
    Let \( R \) be a commutative ring with identity.

    Let \( G \) be the subset of \( \Mod(R)(A, B) \) consisting of module morphisms that factor through a projective object.

    Then \( G \) is an \( R \)-submodule of \( \Mod(R)(A, B) \).
\end{lemma}
\begin{proof}
    Let \( f \) and \( g \) be two morphisms that factor through the projectives \( P \) and \( Q \) respectively. Then we have the following diagrams.

    \begin{center}
        \begin{tikzpicture}
            \diagram{m}{1cm}{1cm} {
                A \& P \& B \\
            };

            \draw[math]
                (m-1-1) edge node {f_1} (m-1-2)
                (m-1-2) edge node {f_2} (m-1-3);
        \end{tikzpicture}
    \end{center}
    where \( f_2 \circ f_1 = f \), and
    \begin{center}
        \begin{tikzpicture}
            \diagram{m}{1cm}{1cm} {
                A \& Q \& B, \\
            };

            \draw[math]
                (m-1-1) edge node {g_1} (m-1-2)
                (m-1-2) edge node {g_2} (m-1-3);
        \end{tikzpicture}
    \end{center}
    where \( g_2 \circ g_1 = g \).

    One can then construct the morphism
    \begin{center}
        \begin{tikzpicture}
            \diagram{m}{1cm}{1cm} {
                A \& {P \oplus Q} \& B. \\
            };

            \draw[math]
                (m-1-1) edge node {
                    \begin{psmallmatrix}
                        f_1 \\
                        g_1
                    \end{psmallmatrix}
                } (m-1-2)
                (m-1-2) edge node {(f_2, g_2)} (m-1-3);
        \end{tikzpicture}
    \end{center}

    Composing these two morphisms, one gets the morphism \( f_2 \circ f_1 + g_2 \circ g_1 = f + g \). This morphisms factors thorugh \( P \oplus Q \), which is projective since it's a direct sum of projective modules.

    Therefore, the set of homomorphisms that factor through a projective is closed under addition. And multiplying with a ring element still factors through the same projective, since every morphism is an \( R \) homomorphism. Therefore the set of morphisms that factor through a projective is an \( R \) submodule.
\end{proof}

From now on, in this subsection let \( \Mod(R)(A, B)/G \), with \( R \) a commutative ring with identity, simply be denoted as \( \Mc(A, B) \).

Second is the result that makes composition in the stable module category well defined.

\begin{lemma}
    \label{lem:stmod_composition_well-defined}
    Define the binary operation \( c \) as follows
    \begin{align*}
        c: \Mc(B, C) \times \widetilde{\Mc}(A, B) &\to \Mc(A, C) \\
        [g] \times [f] &\mapsto [g \circ f]
    \end{align*}

    Then \( c \) is well defined.
\end{lemma}
\begin{proof}
    Need to check that two different choices of representatives of \( [f] \) and \( [g] \) yields the same value.

    Let \( [f + \widetilde{f}] \) and \( [g + \widetilde{g}] \) be two different representatives of \( [f] \) and \( [g] \), with \( \widetilde{f} \) and \( \widetilde{g} \) factoring through some projectives.

    Then it follows that
    \[
        [f + \widetilde{f}] \circ [g + \widetilde{g}] = [f \circ g] + [\widetilde{f} \circ g] + [f \circ \widetilde{g}] + [\widetilde{f} \circ \widetilde{g}].
    \]
    
    Every term other than \( [f \circ g] \) factors through a projective and is therefore equal to \( 0 \) in \( \Mc(A, C) \).
\end{proof}

Finally one can define the underlying stable module category.

% TODO: Frobenius algebra for sterk antaking?
\begin{definition}
    \label{def:stable_module_category}
    Let \( R \) be a \emph{frobenius ring}.

    Then the \emph{stable module category over \( R \)}, denoted \( \Mc \) in this thesis, is defined in the following way:
    \begin{enumerate}
        \item {
            The objects are infinitely generated modules over \( R \).
        }
        \item {
            For any two objects \( A, B \in \Mc \), let \( \Mc(A, B) \) be defined as previously.
        }
        \item {
            For two morphisms \( [g] \in \Mc(B, C) \), and \( [f] \in \Mc(A, B) \), let composition be defined as follows
            \[
                [g] \circ [f] := [g \circ f].
            \]
        }
    \end{enumerate}

    This category is called the \emph{infinitely generated stable module category over \( R \)}.
\end{definition}

Other authors typically use the notation \( \StMod(R) \) or \( \underline{\Mod}(R) \) for the infinitely generated stable module category. However, for the sake of clarity this is reduced to simply \( \Mc \) in this thesis.

In the general definition of the infinitely generated stable module category, it is not required that the ring is a frobenius ring. However, as will become apparent later in this subsection is that it is a requirement to having a triangulated structure defined on it. Since this thesis will only be considering the stable module category as a triangulated category, this assumption is made from the beginning.
% TODO: Faktasjekk, er det verkeleg sant at Frobenius er naudsynt? Las eg ikkje ein stad at nok injektiv og projektive og dei samanfaller er godt nok?

In order to be an underlying category of a triangulated category, a category needs to be additive.

\begin{lemma}
    \( \Mc \) is an additive category.
\end{lemma}
\begin{proof}
    There are two parts to this proof. First want to show that \( \Mc \) is pre-additive and then second, want to show that \( \Mc \) has finite products which in addition to being pre-additive would imply that \( \Mc \) is additive by TODO:Cite.
    % https://en.wikipedia.org/wiki/Additive_category

    To show that \( \Mc \) is pre-additive there are two properties that need to be shown
    \begin{enumerate}
        \item {
            First, need to show that for any \( A, B \in \Mc \) that \( \Mc(A, B) \) is an abelian group.

            This follows immediately, since \( \Mc(A, B) \) is the quotient of an \( R \)-module, and is therefore an \( R \)-module and hence an abelian group.
        }
        \item {
            Second, need to show that composition is bilinear.

            Consider \( [f], [f'] \in \Mc(B, C) \) and \( [g], [g'] \in \Mc(A, B) \). Consider the following equation
            \begin{align*}
                ([f] + [f']) \circ ([g] + [g']) &= [f + f'] \circ [g + g'] \\
                &= \class{(f + f') \circ (g + g')} \\
                &= \class{f \circ g + f \circ g' + f' \circ g + f' \circ g'} \\
                &= [f] \circ [g] + [f] \circ [g'] + [f'] \circ [g] + [f'] \circ [g'],
            \end{align*}
            which is exactly bilinearity.
        }
    \end{enumerate}

    The claim is that the usual biproduct in \( \Mod(R) \) is the product in \( \Mc \). Let the following be the commutative diagram for the universal property of \( A \oplus B \) as a product in \( \Mod(R) \), but in \( \Mc \) with respect to some object \( X \).
    \begin{center}
        \begin{tikzpicture}
            \diagram{m}{1.5cm}{1.5cm} {
                \& X \\
                A \& A \oplus B \& B \\
            };

            \draw[math]
                (m-1-2) edge node[swap] {[f_A]} (m-2-1)
                    edge[dashed] node {[f]} (m-2-2)
                    edge node {[f_B]} (m-2-3)

                (m-2-2) edge node[swap] {[\pi_B]} (m-2-3)
                    edge node {[\pi_A]} (m-2-1);
        \end{tikzpicture}
    \end{center}
    Let \( [g]: X \to A \times B \) be another morphism that satisfies the universal property.

    Then the following diagram commutes.
    \begin{center}
        \begin{tikzpicture}
            \diagram{m}{1.5cm}{1.5cm} {
                \& X \\
                A \& A \oplus B \& B \\
            };

            \draw[math]
                (m-1-2) edge node[swap] {[0]} (m-2-1)
                    edge[dashed] node {[f - g]} (m-2-2)
                    edge node {[0]} (m-2-3)

                (m-2-2) edge node[swap] {[\pi_B]} (m-2-3)
                    edge node {[\pi_A]} (m-2-1);
        \end{tikzpicture}
    \end{center}

    Let \( \iota_A, \iota_B \) denote the canonical split monomorphisms from the universal property of the coproduct. Since \( A \oplus B \) is a biproduct in \( \Mod(R) \) one has the known fact that
    \[
        \Id_{A \oplus B} = \iota_A \circ \pi_A + \iota_B \circ \pi_B.
    \]
    Consider the following equation,
    \begin{align*}
        [f - g] &= [\Id_{A \oplus B} \circ (f - g)] \\
        &= [\iota_A \circ \pi_A \circ (f - g) + \iota_B \circ \pi_B \circ (f - g)] \\
        &= [\iota_A] \circ [\pi_A \circ (f - g)] + [\iota_B] \circ [\pi_B \circ (f - g)] \\
        &= [0].
    \end{align*}
    Which implies that \( [f] = [g] \), and that \( A \oplus B \) is a product in \( \Mc \), and by the statement at the start, also a biproduct.
\end{proof}

This thesis starts by defining the functor \( \Omega \), which will be shown later to be the inverse of the shift functor.

But in order to properly define \( \Omega \), a couple of prerequisites are needed first.

An important property of the category of modules over a commutative ring with identity is that it has \emph{enough projectives} and \emph{enough injectives}. Having enough projectives means that for any module \( A \), there exists some projective module \( P_A \) dependent on \( A \), as well as an epimimorphism, \( \pi_A \) from \( P_A \) to \( A \). The choice of \( P_A \) and \( \pi_A \) is \emph{not necessarily unique} and two different choices of \( P_A \) \emph{could be non-isomorphic}. Similarly, having enough injectives means that for any module \( A \) there exists some injective module \( I_A \) along with a monomorphism \( \iota_A \) from \( A \) to \( I_A \). Similar to the projective case, \( I_A \) and \( \iota_A \) are \emph{not necessarily unique potentially non-isomorphic}.

However, in defining \( \Omega \), the definition is closely tied to a choice of \( P_A \) and \( \pi_A \)-s. It is important that the definition is well defined, so a choice of projective objects and epimorphisms are made in advance. Also, the definition of \( \Omega \) is tied to the kernel of the choice of \( \pi_A \), and it therefore makes sense to fix a choice of kernel as well, since it is also not unique (only up to isomorphism).

% Similarly, let \( I \) be a collection of tuples of an injective module, a monomorphisms, and a module, where for any module \( B \in \Mod(R) \), there is some tuple \( \tuple{I_A, \iota_A, \Sigma A} \) such that \( \iota_A \) is a monomorphism from \( A \) to \( \iota_A \), and \( \Sigma A \) is a choice of cokernel of \( \iota_A \).

\begin{definition}[The functor \( \Omega \)]
    \label{def:stmod_omega}
    Since \( \Mod(R) \) has enough projectives, let \( P \) be a collection of tuples for every object \( A \) consisting of a projective module \( P_A \), an epimorphism \( \pi_A \), and a module \( \Omega A \), where \( \pi_A \) is an epimorphism from \( P_A \) to \( A \), and \( \Omega A \) is a kernel of \( \pi_A \).

    Then define \( \Omega \) as the assignment of objects and morphisms in \( \Mc \) as follows:
    \begin{itemize}
        \item {
            For any object \( A \) in \( \Mc \), let \( \Omega A \) be as defined in \( P \) above.
        }
        \item {
            For any \( [f] \in \Mc(A, B) \), let \( \Omega([f]) \) be constructed as follows:

            Consider the following diagram in \( \Mod(R) \), with the relevant objects and morphisms from \( P \) defined above
            \begin{center}
                \begin{tikzpicture}
                    \diagram{m}{1cm}{1cm} {
                        {\Omega A} \& {P_A} \& A \\
                        {\Omega B} \& {P_B} \& B. \\
                    };

                    \draw[math]
                        (m-1-1) edge[hook] node {\iota_A} (m-1-2)
                            edge node {\Omega'(f)} (m-2-1)
                        (m-1-2) edge[two heads] node {\pi_A} (m-1-3)
                            edge node {p_f} (m-2-2)
                        (m-1-3) edge node {f} (m-2-3)

                        (m-2-1) edge[hook] node {\iota_B} (m-2-2)
                        (m-2-2) edge[two heads] node {\pi_B} (m-2-3);
                \end{tikzpicture}
            \end{center}

            One has that for a morphism \( f: A \to B \), there exists a morphism \( p_f \) from the lifitng property of projective modules such that the right square commutes. Please note that the choice of \( p_f \) is \emph{not neccesarily unique}.

            Furthermore, since
            \[
                \pi_B \circ p_f \circ \iota_A = f \circ \pi_A \circ \iota_A = f \circ 0 = 0,
            \]
            one has from the universal kernel property that there exists a unique morphism, dependent on the choice of \( p_f \), denoted as \( \Omega'(f) \) from \( \Omega A \) to \( \Omega B \) such that the left square in the above diagram commutes. Taking the equivalence class of this morphism with respect to morphisms factoring through projectctives yields the morphism defined by the assignment, i.e. \( \Omega [f] := \class{\Omega'(f)} \).
        }
    \end{itemize}
\end{definition}

The goal is to show that the above assignment is a well defined endofunctor, and later showing that it is in fact an additive autoequivalence of categories.

The following is the lemma showing that \( \Omega \) is a well-defined assignment of morphisms.

\begin{lemma}
    \label{lem:stmod_omega_f_is_well_defined}
    \( \Omega \) is a well-defined assignment of morphisms.
\end{lemma}
\begin{proof}
    There are two things that need to be proven. First, in the construction of \( \Omega'(f) \), one has to choose a morphism \( p_f \) from the projective property. Needs to show that if one choses another projective morphism, that \( \Omega \) still yields the same morphism in \( \Mc \). Second, one need to show that if \( f \sim g \), then \( \Omega'(f) \sim \Omega'(g) \).

    To prove the first part, let \( p_f \) and \( \widetilde{p_f} \) be two different projective morphisms that give the morphisms \( \Omega'(f) \) and \( \widetilde{\Omega'(f)} \) respectively. Then one has the following commutative diagram excluding the dotted arrow
    \begin{center}
        \begin{tikzpicture}
            \diagram{m}{1cm}{2cm} {
                {\Omega A} \& {P_A} \& A \\
                {\Omega B} \& {P_B} \& B. \\
            };

            \draw[math]
                (m-1-1) edge[hook] node {\iota_A} (m-1-2)
                    edge[swap] node {\Omega'(f) - \widetilde{\Omega'(f)}} (m-2-1)
                (m-1-2) edge[two heads] node {\pi_A} (m-1-3)
                    edge[dashed, swap] node {\phi} (m-2-1)
                    edge node {p_f - \widetilde{p_f}} (m-2-2)
                (m-1-3) edge node {f -f = 0} (m-2-3)

                (m-2-1) edge[hook] node {\iota_B} (m-2-2)
                (m-2-2) edge[two heads] node {\pi_B} (m-2-3);
        \end{tikzpicture}
    \end{center}

    Since
    \[
        \pi_B \circ \tuple{p_f - \widetilde{p_f}} = \tuple{f - f} \circ \pi_A = 0
    \]
    there exists a morphism \( \phi \) induced by the kernel property of \( \Omega B \), such that the lower triangle in the diagram commutes. However, since \( \iota_B \) is a monomorphism, one also gets that the upper triangle commutes. This implies that the morphism \( \Omega'(f) - \widetilde{\Omega'(f)} \) factors through \( P_A \), a projective module. Therefore
    \[
        0 = \class{\Omega'(f) - \widetilde{\Omega'(f)}} = \class{\Omega'(f)} - \class{\widetilde{\Omega'(f)}}
    \]
    which implies \( \Omega [f] \) is independent of choice of \( p_f \).

    Second, needs to show that if \( f \sim g \), then \( \Omega f \sim \Omega g \). Consider the following diagram

    \begin{center}
        \begin{tikzpicture}
            \diagram{m}{1cm}{3cm} {
                \Omega A \& P_A \& A \\
                \&\& P \\
                \Omega B \& P_B \& B. \\
            };

            \draw[math]
                (m-1-1) edge[hook] node {\iota_A} (m-1-2)
                    edge node {\Omega'(f) - \Omega'(g)} (m-3-1)
                (m-1-2) edge[two heads] node {\pi_A} (m-1-3)
                    edge node {p_f - p_g} (m-3-2)
                (m-1-3) edge[swap] node {(f - g)_1} (m-2-3)
                    edge[curve={height=-25pt}] node {f - g} (m-3-3)

                (m-2-3) edge[swap, dashed, color={rgb,255:red,214;green,92;blue,92}] node {\theta} (m-3-2)
                    edge[swap] node {(f - g)_2} (m-3-3)

                (m-3-1) edge[hook] node {\iota_B} (m-3-2)
                (m-3-2) edge[two heads] node[swap] {\pi_B} (m-3-3);
        \end{tikzpicture}
    \end{center}

    Let \( P \) be the projective that \( f - g \) factors through. Then from the projective property, there exists a morphism \( \theta: P \to P_B \), which causes the lower triangle to commute.

    Let \( p'_{f - g} := \theta \circ (f - g)_1 \circ \pi_A \). By construction, one has that both \( p_f - p_g \) and \( p'_{f - g} \) are morphisms that would make the right hand square commute.
    
    But since
    \[
        p'_{f - g} \circ \iota_A = \theta \circ (f-g)_1 \circ \pi_A \circ \iota_A = \theta \circ (f-g)_1 \circ 0 = 0,
    \]
    it implies that if the leftmost morphism in the diagram was \( 0 \) and the middle morphism was \( p'_{f - g} \), it would commute. By the previous part of this proof, since the two possible middle morphisms, \( p_f - p_g \) and \( p'_{f - g} \), produce two different leftmost morphisms \( \Omega'(f) - \Omega'(g) \) and \( 0 \), one has that \( \Omega [f] = \Omega [g] \).
\end{proof}

Now that \( \Omega \) is a well defined asssignment of objects and morphisms, it only remains to prove functoriality as well as showing that the identity is mapped to the identity to show that it's a functor.

\begin{lemma}
    \label{lem:stmod_omega_endofunctor}
    \( \Omega \) is an endofunctor on \( \Mc \).
\end{lemma}
\begin{proof}
    To show that \( \Omega \) is a functor it remains to prove functoriality as well as showing that the identity is mapped to the identity.

    First want to show that \( \Omega \) is functorial. Let \( A, B, C \in \Mc \). Then by the definition of \( \Omega \), one has the following commutative diagram
    \begin{center}
        \begin{tikzpicture}
            \diagram{m}{1cm}{2cm} {
                {\Omega A} \& {P_A} \& A \\
                {\Omega B} \& {P_B} \& B \\
                {\Omega C} \& {P_C} \& C \\
            };

            \draw[math]
                (m-1-1) edge[hook] (m-1-2)
                    edge[curve={height=30pt}, swap, color={rgb,255:red,214;green,92;blue,92}] node {\Omega'(f \circ g)} (m-3-1)
                    edge node {\Omega'(g)} (m-2-1)
                (m-1-2) edge[two heads] (m-1-3)
                    edge node {p_g} (m-2-2)
                (m-1-3) edge node {g} (m-2-3)

                (m-2-1) edge[hook] (m-2-2)
                    edge node {\Omega'(f)} (m-3-1)
                (m-2-2) edge[two heads] (m-2-3)
                    edge node {p_f} (m-3-2)
                (m-2-3) edge node {f} (m-3-3)

                (m-3-1) edge[hook] (m-3-2)
                (m-3-2) edge[two heads] (m-3-3);
        \end{tikzpicture}
    \end{center}

    Considering the composition of the vertical morphisms, one ends up with the following commutative diagram
    \begin{center}
        \begin{tikzpicture}
            \diagram{m}{1cm}{2cm} {
                {\Omega A} \& {P_A} \& A \\
                {\Omega C} \& {P_C} \& C. \\
            };

            \draw[math]
                (m-1-1) edge[hook] node {\iota_A} (m-1-2)
                    edge[swap] node {\Omega'(f) \circ \Omega'(g)} (m-2-1)
                (m-1-2) edge[two heads] node {\pi_A} (m-1-3)
                    edge node {p_f \circ p_g} (m-2-2)
                (m-1-3) edge node {f \circ g} (m-2-3)

                (m-2-1) edge[hook] node {\iota_C} (m-2-2)
                (m-2-2) edge[two heads] node {\pi_C} (m-2-3);
        \end{tikzpicture}
    \end{center}
    Since \( \Omega \) is a well-defined assignment of morphisms, this implies
    \[
        \Omega([f]) \circ \Omega([g]) = \class{\Omega'(f) \circ \Omega'(g)} = \Omega \class{f \circ g}.
    \]

    Second, need to show that \( \Omega \Id_A = \Id_{\Omega A} \) in \( \Mc \).

    By a similar argument to above, one can see that every square and triangle in the following diagram commutes
    \begin{center}
        \begin{tikzpicture}
            \diagram{m}{1cm}{2cm} {
                {\Omega A} \& {P_A} \& A \\
                {\Omega A} \& {P_A} \& A. \\
            };

            \draw[math]
                (m-1-1) edge[hook] (m-1-2)
                    edge[swap] node {\Id_{\Omega A}} (m-2-1)
                (m-1-2) edge[two heads] (m-1-3)
                    edge node {\Id_{P_A}} (m-2-2)
                (m-1-3) edge node {\Id_A} (m-2-3)

                (m-2-1) edge[hook] (m-2-2)
                (m-2-2) edge[two heads] (m-2-3);
        \end{tikzpicture}
    \end{center}
    Therefore \( \Omega \class{\Id_A} = \class{\Id_{\Omega A}} \).
\end{proof}

An important detail still missing to show that \( \Omega^{-1} \) could be a shift functor is showing that \( \Omega \) is additive.

\begin{lemma}
    \label{lem:stmod_omega_additive_functor}
    \( \Omega \) is an additive functor.
\end{lemma}
\begin{proof}
    Want to show that \( \Omega(f + g) \sim \Omega(f) + \Omega(g) \).
    
    There are two posssible values of \( \Omega'(f + g) \), namely the ones from either of the two commutative diagrams below
    \begin{center}
        \begin{tikzpicture}
            \diagram{m}{1cm}{2cm} {
                {\Omega A} \& {P_A} \& A \\
                {\Omega B} \& {P_B} \& B \\
            };

            \draw[math]
                (m-1-1) edge[hook] node {\iota_A} (m-1-2)
                    edge[swap] node {\Omega'(f + g)} (m-2-1)
                (m-1-2) edge[two heads] node {\pi_A} (m-1-3)
                    edge node[swap] {p_{f + g}} (m-2-2)
                (m-1-3) edge node[swap] {f + g} (m-2-3)

                (m-2-1) edge[hook] node {\iota_B} (m-2-2)
                (m-2-2) edge[two heads] node {\pi_B} (m-2-3);
        \end{tikzpicture}
        \begin{tikzpicture}
            \diagram{m}{1cm}{2cm} {
                {\Omega A} \& {P_A} \& A \\
                {\Omega B} \& {P_B} \& B. \\
            };

            \draw[math]
                (m-1-1) edge[hook] node {\iota_A} (m-1-2)
                    edge[swap] node {\Omega'(f)+\Omega'(g)} (m-2-1)
                (m-1-2) edge[two heads] node {\pi_A} (m-1-3)
                    edge node[swap] {p_f + p_g} (m-2-2)
                (m-1-3) edge node[swap] {f + g} (m-2-3)

                (m-2-1) edge[hook] node {\iota_B} (m-2-2)
                (m-2-2) edge[two heads] node {\pi_B} (m-2-3);
        \end{tikzpicture}
    \end{center}

    Since \( \Omega [f + g] \) is well defined, this implies \( \Omega [f + g] = \Omega([f]) + \Omega([g]) \).
\end{proof}

Now that \( \Omega \) has been shown to be an additive functor, one can define \( \Sigma \) which will turn out to be the inverse of \( \Omega \) later.

\begin{definition}[The functor \( \Sigma \)]
    \label{def:stmod_sigma}
    Since \( \Mod(R) \) has enough projectives, let \( I \) be a collection of tuples for every object \( A \) consisting of an injective module \( I_A \), an monomorphism \( \kappa_A \), and a module \( \Sigma A \), where \( \kappa_A \) is an monomorphism from \( A \) to \( P_A \), and \( \Sigma A \) is a cokernel of \( \kappa_A \).

    Then define \( \Sigma \) as the assignment of objects and morphisms in \( \Mc \) as follows:
    \begin{itemize}
        \item {
            For any object \( A \) in \( \Mc \), let \( \Sigma A \) be as defined in \( I \) above.
        }
        \item {
            \( \Sigma [f] \) is constructed as follows:

            Consider the following diagram in \( \Mod(R) \)
            \begin{center}
                \begin{tikzpicture}
                    \diagram{m}{1cm}{1cm} {
                        A \& I_A \& \Sigma A \\
                        B \& I_B \& \Sigma B. \\
                    };

                    \draw[math]
                        (m-1-1) edge[hook] node {\kappa_A} (m-1-2)
                            edge node {f} (m-2-1)
                        (m-1-2) edge[two heads] node {\rho_A} (m-1-3)
                            edge node {i_f} (m-2-2)
                        (m-1-3) edge node {\Sigma'(f)} (m-2-3)

                        (m-2-1) edge[hook] node {\kappa_B} (m-2-2)
                        (m-2-2) edge[two heads] node {\rho_B} (m-2-3);
                \end{tikzpicture}
            \end{center}

            One has that for a morphism \( f: A \to B \), there exists a morphism \( i_f \) from the universal property of injective objects such that the left square commutes. Please note that this morphism is \emph{not neccesarily} unique.

            Furthermore, since
            \[
                \kappa_B \circ i_f \circ \rho_A = \kappa_B \circ \rho_B \circ f = 0,
            \]
            one has from the universal cokernel property that there exists a unique morphism, dependant on the choice of \( i_f \), denoted as \( \Sigma'(f) \) from \( \Sigma A \) to \( \Sigma B \) such that the right square in the above diagram commutes. Taking the equivalence class of this morphism with respect to morphisms factoring through projectives yields the morphism defined by the functor, i.e. \( \Sigma [f] := \class{\Sigma'(f)} \).
        }
    \end{itemize}
\end{definition}

Now it remains to show every desired property of \( \Omega \), like being a well-defined, an endofunctor, and additive. However, that is a lot of work, that can luckily be skipped because of the previous legwork on proving the respective properties for \( \Omega \).

\begin{lemma}
    \label{lem:stmod_sigma_well-defined_additive_endofunctor}
    \( \Sigma \) is a well-defined and additive endofunctor on \( \Mc \).
\end{lemma}
\begin{proof}
    The proofs of the various statements are entirely dual to the proofs of \autoref{lem:stmod_omega_f_is_well_defined}, \autoref{lem:stmod_omega_endofunctor} and \autoref{lem:stmod_omega_additive_functor}.
\end{proof}

Now one can finally show that \( \Sigma = \Omega^{-1} \).

\begin{theorem}
    \( \Omega \) is an auto equivalence with inverse \( \Sigma \).
\end{theorem}
\begin{proof}
    This thesis will only show that \( \Id_{\Mc} \) is naturally isomorphic to \( \Sigma\Omega \). The omitted part that \( \Id_{\Mc} \) is naturally isomorphic to \( \Omega\Sigma \) is very similar, and uses many dual properties.

    First show that for any \( A \in \Mc \), there exist an isomorphism from \( A \to \Sigma\Omega A \). Consider the following diagram
    \begin{center}
        \begin{tikzpicture}
            \diagram{m}{1cm}{2cm} {
                \Omega A \& P_A \& A \\
                \Omega A \& I_{\Omega A} \& \Sigma \Omega A \\
                \Omega A \& P_A \& A, \\
            };

            \draw[math]
                (m-1-1) edge[hook] node {\iota_A} (m-1-2)
                    edge[equal] (m-2-1)
                (m-1-2) edge[two heads] node {\pi_A} (m-1-3)
                    edge node {i_{\phi_1}} (m-2-2)
                (m-1-3) edge node {\phi_1} (m-2-3)

                (m-2-1) edge[hook] node {\kappa_{\Omega A}} (m-2-2)
                    edge[equal] (m-3-1)
                (m-2-2) edge[two heads] node {\rho_{\Omega A}} (m-2-3)
                    edge node {i_{\phi_2}} (m-3-2)
                (m-2-3) edge node {\phi_2} (m-3-3)

                (m-3-1) edge[hook] node {\iota_A} (m-3-2)
                (m-3-2) edge[two heads] node {\pi_A} (m-3-3);
        \end{tikzpicture}
    \end{center}
    where \( i_{\phi_1} \) is the injective property morphism induced from \( \iota_A \). Then since \( \Mod(R) \) is an abelian category, \( A \) is a cokernel of \( \iota_A \), and by
    \[
        \rho_{\Omega A} \circ i_{\phi_1} \circ \iota_A = \rho_{\Omega A} \circ \kappa_{\Omega A} = 0,
    \]
    one gets from the cokernel property that there is a uniquely induced morphism \( \phi_1: A \to \Sigma\Omega A \). Then doing the same for the lower rectangle of the diagram, using the fact that every projective is also injective, then one gets the morphisms \( i_{\phi_2} \) and \( \phi_2 \) by similar arguments.

    Then to show that \( \phi_1 \) and \( \phi_2 \) are isomorphisms, consider the following diagram
    \begin{center}
        \begin{tikzpicture}
            \diagram{m}{1cm}{2cm} {
                \Omega A \& P_A \& A \\
                \Omega A \& P_A \& A. \\
            };

            \draw[math]
                (m-1-1) edge[hook] node {\iota_A} (m-1-2)
                    edge[swap] node {\Id_{\Omega A} \circ \Id_{\Omega A} - \Id_{\Omega A} = 0} (m-2-1)
                (m-1-2) edge[two heads] node {\pi_A} (m-1-3)
                    edge[swap] node {i_{\phi_2} \circ i_{\phi_1} - \Id_{P_A}} (m-2-2)
                (m-1-3) edge[swap, color={rgb,255:red,214;green,92;blue,92}] node {\theta} (m-2-2)
                    edge node {\phi_2 \circ \phi_1 - \Id_A} (m-2-3)

                (m-2-1) edge[hook] node {\iota_A} (m-2-2)
                (m-2-2) edge[two heads] node {\pi_A} (m-2-3);
        \end{tikzpicture}
    \end{center}

    Using the previous commutative diagram, one gets that every square commutes. But since
    \[
        (p_{\phi_2} \circ p_{\phi_1} - \Id_{P_A}) \circ \iota_A = \iota_A \circ 0 = 0,
    \]
    then from the cokernel property there exist a morphism \( \theta: A \to P_A \) such that the upper triangle commutes. But then one has that
    \[
        (\phi_2 \circ \phi_1 - \Id_A) \circ \pi_A = \pi_A \circ (p_{\theta_2} \circ p_{\theta_1} - \Id_{P_A}) = \pi_A \circ \theta \circ \pi_A,
    \]
    and since \( \pi_A \) is an epimorphism, one gets that the lower triangle commutes. This implies that \( \phi_2 \circ \phi_1 \sim \Id_A \).
    
    By a similar agument that is omitted for brevity, one can show that \( \phi_1 \circ \phi_2 \sim \Id_{\Sigma\Omega A} \), which means that \( [\phi_1] \) and \( [\phi_2] \) are isomorphisms from \( A \) to \( \Sigma\Omega A \).

    To show that these isomorphisms are natural, let \( [f] \in \Mc(A, B) \) and consider the following two diagrams
    \begin{center}
        \begin{tikzpicture}
            \diagram{m}{1cm}{2cm} {
                \Omega A \& P \& A \\
                \Omega A \& I \& \Sigma \Omega A \\
                \Omega B \& I \& \Sigma \Omega B, \\
            };

            \draw[math]
                (m-1-1) edge[hook] node {\iota_A} (m-1-2)
                    edge[equal] (m-2-1)
                (m-1-2) edge[two heads] node {\pi_A} (m-1-3)
                    edge node {i_{\phi^A_1}} (m-2-2)
                (m-1-3) edge node {\phi^A_1} (m-2-3)

                (m-2-1) edge[hook] node {\kappa_{\Omega A}} (m-2-2)
                    edge node {\Omega'(f)} (m-3-1)
                (m-2-2) edge[two heads] node {\rho_{\Omega A}} (m-2-3)
                    edge node {i_{\Omega'(f)}} (m-3-2)
                (m-2-3) edge node {\Sigma' \circ \Omega'(f)} (m-3-3)

                (m-3-1) edge[hook] node {\kappa_{\Omega B}} (m-3-2)
                (m-3-2) edge[two heads] node {\rho_{\Omega B}} (m-3-3);
        \end{tikzpicture}
    \end{center}
    and
    \begin{center}
        \begin{tikzpicture}
            \diagram{m}{1cm}{2cm} {
                \Omega A \& P \& A \\
                \Omega B \& P \& B \\
                \Omega B \& I \& \Sigma \Omega B, \\
            };

            \draw[math]
                (m-1-1) edge[hook] node {\iota_A} (m-1-2)
                    edge node {\Omega'(f)} (m-2-1)
                (m-1-2) edge[two heads] node {\pi_A} (m-1-3)
                    edge node {p_f} (m-2-2)
                (m-1-3) edge node {f} (m-2-3)

                (m-2-1) edge[hook] node {\iota_B} (m-2-2)
                    edge[equal] (m-3-1)
                (m-2-2) edge[two heads] node {\pi_B} (m-2-3)
                    edge node {i_{\phi_1^B}} (m-3-2)
                (m-2-3) edge node {\phi_1^B} (m-3-3)

                (m-3-1) edge[hook] node {\kappa_{\Omega B}} (m-3-2)
                (m-3-2) edge[two heads] node {\rho_{\Omega B}} (m-3-3);
        \end{tikzpicture}
    \end{center}
    where every small square, and therefore rectangle, commutes.

    These diagrams gives rise to the following commutative diagram
    \begin{center}
        \begin{tikzpicture}
            \diagram{m}{1cm}{2.7cm} {
                \Omega A \& P \& A \\
                \Omega B \& I \& \Sigma \Omega B, \\
            };

            \draw[math]
                (m-1-1) edge[hook] node {\iota_A} (m-1-2)
                    edge[swap] node {\Id_{\Omega B} \circ \Omega'(f) - \Omega'(f) \circ \Id_{\Omega A} = 0} (m-2-1)
                (m-1-2) edge[two heads] node {\pi_A} (m-1-3)
                    edge[swap] node {i_{\phi_1^B} \circ p_f - i_{\Omega'(f)} \circ i_{\phi_1^A}} (m-2-2)
                (m-1-3) edge[swap, color={rgb,255:red,214;green,92;blue,92}] node {\theta} (m-2-2)
                    edge node {\phi_1^B \circ f - \Sigma' \circ \Omega'(f) \circ \phi_1^A} (m-2-3)

                (m-2-1) edge[hook] node[swap] {\kappa_{\Omega B}} (m-2-2)
                (m-2-2) edge[two heads] node[swap] {\rho_{\Omega B}} (m-2-3);
        \end{tikzpicture}
    \end{center}
    where from the cokernel property of \( A \), one gets an induced morphism \( \theta \), and from the epimorphism property of \( \pi_A \), it makes the lower triangle commute. And since \( R \) is assumed to be a frobenius ring, this implies that every injective module over \( R \) is projective and vice versa. This implies
    \[
        \phi_1^B \circ f \sim \Sigma' \circ \Omega'(f) \circ \phi_1^A,
    \]
    which means that \( [\phi_1] \) is a natural isomorphism from \( \Id_{\Mc} \) to \( \Sigma \Omega \), with inverse \( [\phi_2] \).
\end{proof}

An interesting consequence of how \( \Omega \) and \( \Sigma \) are defined is that if one were to chose a different \( P \) or \( I \) in their definitions, then it would yield different, but naturally isomorphic functors. A proof of this statement for \( \Sigma \) can be found in \cite[Remark on p. 13]{Happel_1988}, with the proof for \( \Omega \) being dual.

Before one can show the triangulation of \( \Mc \), one first needs to define the cone of a morphism. This definition as well as definition of the triangulation leans heavily upon \cite[Chapter 1, Subsection 2.5]{Happel_1988}.

\begin{definition}
    \label{def:stmod_cone}
    Let \( f \in \Mod(R)(A, B) \).
    
    Then define the \emph{cone of \( f \)} to be the object \( C_f \) from the following commutative diagram in \( \Mod(R) \)
    \begin{center}
        \begin{tikzpicture}
            \diagram{m}{1cm}{1cm} {
                A \& B \\
                I_A \& C_f \\
                \& \Sigma A \\
            };

            \draw[math]
                (m-1-1) edge node {f} (m-1-2)
                    edge[swap] node {\kappa_A} (m-2-1)
                (m-1-2) edge node {g} (m-2-2)
                    edge[curve={height=-25pt}] node {0} (m-3-2)

                (m-2-1) edge node {\gamma_f} (m-2-2)
                    edge node {\rho_A} (m-3-2)
                (m-2-2) edge node {h} (m-3-2);
        \end{tikzpicture}
    \end{center}
    where \( \tuple{I_A, \kappa_A, \Sigma A} \) is as defined in \autoref{def:stmod_sigma}. Furthermore, \( \tuple{C(f), g, \gamma_f} \) is the pushout of \( \tuple{A, f, \kappa_A} \), and \( h \) is the pushout universal property induced by \( \rho_A \) and \( 0 \).
\end{definition}

WIP: Ta hensyn til forskjellige kjegler frå forskjellig val av representantar. Kan ikkje tenka på det, og sei til slutt at pga. tri. kat. eigenskapar, so er dei isomorfe? Spør MS.

% There is a small problem with the above definition in that the cone needs to be unique up to isomorphism in \( \Mc \). Therefore the following lemma is needed.
% \begin{lemma}
%     WIP: Cone is unique up to isomorphism in \( \Mc \).
% \end{lemma}

There are some important details following this definition that will be useful in future results.

\begin{remark}
    \label{rem:stmod_cone_pushout_properties}
    Use the notation and morphisms from \autoref{def:stmod_cone}.

    By known pushout facts, it follows that since \( \kappa_A \) is a monomorphism, then \( g \) is a monomorphism since \( \ker(g) = \ker(\kappa_A) = 0 \). Likewise it follows that \( \coker(g) \cong \coker(\kappa_A) \cong \Sigma A \), and therefore \( h \) is a cokernel morphism for \( \alpha \).
\end{remark}

Now, one can finally define the triangulation on \( \Mc \).

\begin{definition}
    Let \( \Delta \) be the collection of triangles in \( \Mc \) isomorphic to any triangle on the form
    \begin{center}
        \begin{tikzpicture}
            \diagram{m}{1cm}{1cm} {
                A \& B \& C_f \& \Sigma A \\
            };

            \draw[math]
                (m-1-1) edge node {[f]} (m-1-2)
                (m-1-2) edge node {[g]} (m-1-3)
                (m-1-3) edge node {[h]} (m-1-4);
        \end{tikzpicture}
    \end{center}
    where \( g \) and \( h \) are defined as in \autoref{def:stmod_cone}.
\end{definition}

% TODO: Show Delta in Happel equal to Delta in Zimmermann, and then use the easiest proof. Klarer ikkje å visa at Zimmermann har rett.

Then finally comes the full triangulation result, the proof of which is heavily inspired by \cite[First theorem in Chapter 1, Subsection 2.6]{Happel_1988}.

% TODO: Kommenter at den nøyaktige triangulerte strukturen ikkje blir brukt seinare i oppgåva. Etterpå vert berre eksistensen av ein triangulering brukt.
% TODO: Problem i beviset med at somme argument om at komposisjon av to morpfiar blir null.

\begin{definition}
    The tuple \( \tuple{\Mc, \Sigma, \Delta} \) is a triangulated category.
\end{definition}
\begin{proof}
    Need to prove {\bf TR1} to {\bf TR4} from \autoref{def:triangulated_category}.

    \begin{enumerate}[label={(\bfseries TR\arabic*)}]
        \item {
            \begin{enumerate}
                \item {
                    Need to show that for any \( A, B \) and \( f \in \Mc(A, B) \) that there is a distinguished triangle
                    \begin{center}
                        \begin{tikzpicture}
                            \diagram{m}{1cm}{1cm} {
                                A \& B \& C_f \& \Sigma A. \\
                            };

                            \draw[math]
                                (m-1-1) edge node {f} (m-1-2)
                                (m-1-2) edge node {\iota_f} (m-1-3)
                                (m-1-3) edge node {\pi_f} (m-1-4);
                        \end{tikzpicture}
                    \end{center}

                    This is satisfied directly by the definition of \( \Delta \), with \( \iota_f := g \) and \( \pi_f := h \).
                }
                \item {
                    Need to show that the trivial triangle is distinguished.

                    Calculating \( C_{\Id} \) one gets the pushout
                    \begin{center}
                        \begin{tikzpicture}
                            \diagram{m}{1cm}{1cm} {
                                A \& A \\
                                I_A \& C_{\Id}. \\
                            };

                            \draw[math]
                                (m-1-1) edge node {\Id} (m-1-2)
                                    edge node {\kappa_A} (m-2-1)
                                (m-1-2) edge (m-2-2)

                                (m-2-1) edge node {\gamma_{\Id}} (m-2-2);
                        \end{tikzpicture}
                    \end{center}
                    However, it is a known fact the pushout of an isomorphism is an isomorphism. And so \( \gamma_{\Id} \) is an isomorphism, which implies \( C_{\Id} \cong 0 \) in \( \Mc \) because all injective modules are projective in \( \Mc \).
                }
                \item {
                    Need to show that \( \Delta \) is closed under isomorphisms of triangles. However this follows directly from the fact that isomorphisms of triangles is an equivalence relation.
                }
            \end{enumerate}
        }
        \item {
            Need to show that for any distinguished triangle, the left and right shifted triangle is also distinguished.

            First note that a left/right shifted distinguished triangle will be isomorphic to a left/right shifted standard triangle. Therefore it suffices to check that every left/right shifted standard triangle is distinguished.

            Start på proving the left shift.

            Consider the following standard triangle
            \begin{center}
                \begin{tikzpicture}
                    \diagram{m}{1cm}{1cm} {
                        A \& B \& C_f \& \Sigma A \\
                    };

                    \draw[math]
                        (m-1-1) edge node {[f]} (m-1-2)
                        (m-1-2) edge node {[g]} (m-1-3)
                        (m-1-3) edge node {[h]} (m-1-4);
                \end{tikzpicture}
            \end{center}

            Keep in mind the following commutative diagrams given by the definition of \( \Sigma [f] \) and the above standard triangle respectively:
            \begin{center}
                \begin{tikzpicture}
                    \diagram{m}{1cm}{1cm} {
                        A \& I_A \& \Sigma A \\
                        B \& I_B \& \Sigma B, \\
                    };

                    \draw[math]
                        (m-1-1) edge[hook] node {\kappa_A} (m-1-2)
                            edge node {f} (m-2-1)
                        (m-1-2) edge[two heads] node {\rho_A} (m-1-3)
                            edge node {i_f} (m-2-2)
                        (m-1-3) edge node {\Sigma' f} (m-2-3)

                        (m-2-1) edge[hook] node {\kappa_B} (m-2-2)
                        (m-2-2) edge[two heads] node {\rho_B} (m-2-3);
                \end{tikzpicture}
                and
                \begin{tikzpicture}
                    \diagram{m}{1cm}{1cm} {
                        A \& B \\
                        I_A \& C_f \\
                        \& \Sigma A. \\
                    };

                    \draw[math]
                        (m-1-1) edge node {f} (m-1-2)
                            edge[hook] node {\kappa_A} (m-2-1)
                        (m-1-2) edge node{g} (m-2-2)

                        (m-2-1) edge node {\gamma_f} (m-2-2)
                            edge node {\rho_A} (m-3-2)
                        (m-2-2) edge node {h} (m-3-2);
                \end{tikzpicture}
            \end{center}
            With the above diagrams in mind, consider the following commutative diagram
            \begin{center}
                \begin{tikzpicture}
                    \diagram{m}{1cm}{1cm} {
                        A \& B \\
                        I_A \& C_f \& I_B \& \Sigma B \\
                    };

                    \draw[math]
                        (m-1-1) edge node {f} (m-1-2)
                            edge[hook] node {\kappa_A} (m-2-1)
                        (m-1-2) edge node{g} (m-2-2)
                            edge[hook] node {\kappa_B} (m-2-3)

                        (m-2-1) edge node {\gamma_f} (m-2-2)
                            edge[curve={height=1cm}] node {i_f} (m-2-3)
                        (m-2-2) edge[dashed] node {\phi} (m-2-3)

                        (m-2-3) edge node{\rho_B} (m-2-4);
                \end{tikzpicture}
            \end{center}
            where \( \phi \) is given by the pushout universal property.

            Consider the following equalities that follow from the above commutative diagrams
            \[
                \rho_B \circ \phi \circ g = \rho_B \circ \kappa_B = 0 = \Sigma'(f) \circ h \circ g,
            \]
            and
            \[
                \rho_B \circ \phi \circ \gamma_f = \rho_B \circ i_f = \Sigma'(f) \circ \rho_A = \Sigma'(f) \circ h \circ \gamma_f.
            \]
            This implies that considering the following pushout diagram
            \begin{center}
                \begin{tikzpicture}
                    \diagram{m}{1cm}{1cm} {
                        A \& B \\
                        I_A \& C_f \& \Sigma B \\
                    };

                    \draw[math]
                        (m-1-1) edge node {f} (m-1-2)
                            edge[hook] node {\kappa_A} (m-2-1)
                        (m-1-2) edge node{g} (m-2-2)
                            edge[hook] node {0} (m-2-3)

                        (m-2-1) edge node {\gamma_f} (m-2-2)
                            edge[curve={height=1cm}] node {\rho_B \circ i_f} (m-2-3)
                        (m-2-2) edge[dashed] (m-2-3);
                \end{tikzpicture}
            \end{center}
            would have two different morphisms, \( \rho_B \circ \phi \) and \( \Sigma'(f) \circ h \), that could satisfy the pushout universal property. However, by uniqueness, they therefore have to be the same morphism, i.e
            \[
                \rho_B \circ \phi = \Sigma'(f) \circ h.
            \]
            Finally, consider the following commutative diagram
            \begin{center}
                \begin{tikzpicture}
                    \diagram{m}{1cm}{1cm} {
                        0 \& B \& C_f \& \Sigma A \& 0 \\
                        0 \& I_B \& I_B \oplus \Sigma A \& \Sigma A \& 0 \\
                        \& \& \Sigma B \\
                    };

                    \draw[math]
                        (m-1-1) edge (m-1-2)
                            edge[equal] node {\zeta} (m-2-1)
                        (m-1-2) edge node {g} (m-1-3)
                            edge node {\kappa_B} (m-2-2)
                        (m-1-3) edge node {h} (m-1-4)
                            edge node {
                                \begin{psmallmatrix}
                                    \phi \\
                                    h
                                \end{psmallmatrix}
                            } (m-2-3)
                        (m-1-4) edge (m-1-5)
                                edge[equal] (m-2-4)

                        (m-2-1) edge (m-2-2)
                        (m-2-2) edge node {
                            \begin{psmallmatrix}
                                1 \\
                                0
                            \end{psmallmatrix}
                        } (m-2-3)
                            edge node {\rho_B} (m-3-3)
                        (m-2-3) edge node {
                            \begin{psmallmatrix}
                                0 & 1
                            \end{psmallmatrix}
                        } (m-2-4)
                            edge node {
                                \begin{psmallmatrix}
                                    \rho_B & -\Sigma' f
                                \end{psmallmatrix}
                            } (m-3-3)
                        (m-2-4) edge (m-2-5);
                \end{tikzpicture}
            \end{center}
            where the top row is exact by \autoref{rem:stmod_cone_pushout_properties}, and the second row is split-exact. Since the top right square commutes, it follows that equality is the induced cokernel morphism for \( \coker(g) = \Sigma A \), and since \( \zeta = 0 \) is an epimorphism, it follows that the middle square is in fact a pushout, which implies \( I_B \oplus \Sigma A  \) is uniquely isomorphic to \( C_g \). % TODO: Cite Notat, klassifisering av pushout og pullback i abelsk kategori.
            And since
            \[
                \begin{psmallmatrix}
                    \rho_B & -\Sigma' f
                \end{psmallmatrix}
                \begin{psmallmatrix}
                    \phi \\
                    h
                \end{psmallmatrix}
                =
                \rho_B \circ \phi - \Sigma'(f) \circ h = 0,
            \]
            then the triangle
            \begin{center}
                \begin{tikzpicture}
                    \diagram{m}{1cm}{1cm} {
                        B \& C_f \& I_B \oplus \Sigma A \& \Sigma B \\
                    };

                    \draw[math]
                        (m-1-1) edge node {[g]} (m-1-2)
                        (m-1-2) edge node {\class{
                            \begin{psmallmatrix}
                                \phi \\
                                h
                            \end{psmallmatrix}
                        }} (m-1-3)
                        (m-1-3) edge node {\class{
                            \begin{psmallmatrix}
                                \rho_B & -\Sigma' f
                            \end{psmallmatrix}
                        }} (m-1-4);
                \end{tikzpicture}
            \end{center} 
            is (uniquely isomorphic to) the standard triangle of \( g \), and therefore distinguished. % TODO: Burde visa at standardtriangelet er isomorft til dette triangelet?

            Finally it remains to check if the triangle is isomorphic to the expected triangle. Consider the following diagram in \( \Mc \)
            \begin{center}
                \begin{tikzpicture}
                    \diagram{m}{1cm}{1cm} {
                        B \& C_f \& I_B \oplus \Sigma A \& \Sigma B \\
                        B \& C_f \& \Sigma A \& \Sigma B. \\
                    };

                    \draw[math]
                        (m-1-1) edge node {[g]} (m-1-2)
                            edge[equal] (m-2-1)
                        (m-1-2) edge node {\class{
                            \begin{psmallmatrix}
                                \phi \\
                                h
                            \end{psmallmatrix}
                        }} (m-1-3)
                            edge[equal] (m-2-2)
                        (m-1-3) edge node {\class{
                            \begin{psmallmatrix}
                                \rho_B & -\Sigma' f
                            \end{psmallmatrix}
                        }} (m-1-4)
                            edge node {\class{
                                \begin{psmallmatrix}
                                    0 & 1
                                \end{psmallmatrix}
                            }} (m-2-3)
                        (m-1-4) edge[equal] (m-2-4)

                        (m-2-1) edge node {[g]} (m-2-2)
                        (m-2-2) edge node {[h]} (m-2-3)
                        (m-2-3) edge node {[-\Sigma' f]} (m-2-4);
                \end{tikzpicture}
            \end{center}
            The left and the middle square commute directly, but it remains to check if the right square commutes. In addition one needs to check that \( \class{
                \begin{psmallmatrix}
                    0 & 1
                \end{psmallmatrix}
            } \) is an isomorphism.

            First, checking the commutativity. Consider the difference
            \[
                -\Sigma'(f) \circ
                \begin{psmallmatrix}
                    0 & 1
                \end{psmallmatrix}
                -
                \begin{psmallmatrix}
                    \rho_B & -\Sigma' f
                \end{psmallmatrix}
                =
                \begin{psmallmatrix}
                    0 & -\Sigma' f
                \end{psmallmatrix}
                -
                \begin{psmallmatrix}
                    \rho_B & -\Sigma' f
                \end{psmallmatrix}
                = 
                \begin{psmallmatrix}
                    \rho_B & 0
                \end{psmallmatrix}.
            \]
            One can see that the diagram
            \begin{center}
                \begin{tikzpicture}
                    \diagram{m}{1cm}{1cm} {
                        I_B \oplus \Sigma A \& \& \Sigma B \\
                        \& I_B \\
                    };

                    \draw[math]
                        (m-1-1) edge node {
                            \begin{psmallmatrix}
                                \rho_B & 0
                            \end{psmallmatrix}
                        } (m-1-3)
                            edge node {
                                \begin{psmallmatrix}
                                    1 & 0
                                \end{psmallmatrix}
                            } (m-2-2)

                        (m-2-2) edge node {\rho_B} (m-1-3);
                \end{tikzpicture}
            \end{center}
            commutes, and since every injective module is also projective by definition of frobenius ring, the right hand square above commutes.

            Second, checking that \( \class{
                \begin{psmallmatrix}
                    0 & 1
                \end{psmallmatrix}
            } \) is an isomorphism.

            Consider the morphism \( \class{
                \begin{psmallmatrix}
                    0 \\
                    1
                \end{psmallmatrix}
            } \).
            It is already known that \( \class{
                \begin{psmallmatrix}
                    0 & 1
                \end{psmallmatrix}
            } \circ \class{
                \begin{psmallmatrix}
                    0 \\
                    1
                \end{psmallmatrix}
            } = [\Id_A] \).
            Then it remains to check if
            \[
                \Id_{I_B \oplus \Sigma A} -
                \begin{psmallmatrix}
                    0 \\
                    1
                \end{psmallmatrix}
                \begin{psmallmatrix}
                    0 & 1
                \end{psmallmatrix}
                =
                \begin{psmallmatrix}
                    1 & 0 \\
                    0 & 0
                \end{psmallmatrix}
            \]
            factors through a projective.

            Consider the following diagram
            \begin{center}
                \begin{tikzpicture}
                    \diagram{m}{1cm}{1cm} {
                        I_B \oplus \Sigma A \& \& I_B \oplus \Sigma A \\
                        \& I_B. \\
                    };

                    \draw[math]
                        (m-1-1) edge node {
                            \begin{psmallmatrix}
                                1 & 0 \\
                                0 & 0
                            \end{psmallmatrix}
                        } (m-1-3)
                            edge node {
                                \begin{psmallmatrix}
                                    1 & 0
                                \end{psmallmatrix}
                            } (m-2-2)

                        (m-2-2) edge node {
                            \begin{psmallmatrix}
                                1 \\
                                0
                            \end{psmallmatrix}
                        } (m-1-3);
                \end{tikzpicture}
            \end{center}
            It commutes.

            By the upcoming proof of {\bf TR3} the right rotation follows by \autoref{lem:triangulated_category-TR2-only_one_rotation}. % TODO: Framoverreferanse?
        }
        \item {
            By considering two arbitrary distinguished triangles, any morphism between their components will induce a unique morphism between the component of their standard triangles they are isomorphic to, and vice versa. Therefore it suffices to only check {\bf (TR3)} for standard triangles.

            Need to show that given the following diagram where the top and botton row are standard triangles and the diagram excluding the dotted arrow commutes
            \begin{equation}
                \label{eq:stablemod}
                \begin{tikzpicture}
                    \diagram{m}{1cm}{1cm} {
                        A \& B  \& C_f \& \Sigma A \\
                        A' \& B' \& C_{f'} \& \Sigma A' \\
                    };

                    \draw[math]
                        (m-1-1) edge node {[f]} (m-1-2)
                            edge node {[\alpha]} (m-2-1)
                        (m-1-2) edge node {[g]} (m-1-3)
                            edge node {[\beta]} (m-2-2)
                        (m-1-3) edge node {[h]} (m-1-4)
                            edge[dashed] node {[\phi]} (m-2-3)
                        (m-1-4) edge node {\Sigma [\alpha]} (m-2-4)

                        (m-2-1) edge node {[f']} (m-2-2)
                        (m-2-2) edge node {[g']} (m-2-3)
                        (m-2-3) edge node {[h']} (m-2-4);
                \end{tikzpicture}
            \end{equation}
            that there exist some \( \phi: C_f \to C_{f'} \) in the above diagram such that the entire diagram including \( \phi \), commutes.

            Consider the following commutative diagrams in \( \Mod(R) \) from the definition of the two standard triangles
            \begin{center}
                \begin{tikzpicture}
                    \diagram{m}{1cm}{1cm} {
                        A \& B \\
                        I_A \& C_f \\
                        \& \Sigma A, \\
                    };

                    \draw[math]
                        (m-1-1) edge node {f} (m-1-2)
                            edge[hook] node {\kappa_A} (m-2-1)
                        (m-1-2) edge node{g} (m-2-2)

                        (m-2-1) edge node {\gamma_f} (m-2-2)
                            edge node {\rho_A} (m-3-2)
                        (m-2-2) edge node {h} (m-3-2);
                \end{tikzpicture}
                %
                \begin{tikzpicture}
                    \diagram{m}{1cm}{1cm} {
                        A' \& B' \\
                        I_{A'} \& C_{f'} \\
                        \& \Sigma A'. \\
                    };

                    \draw[math]
                        (m-1-1) edge node {f'} (m-1-2)
                            edge[hook] node {\kappa_{A'}} (m-2-1)
                        (m-1-2) edge node {g'} (m-2-2)

                        (m-2-1) edge node {\gamma_{f'}} (m-2-2)
                            edge node {\rho_{A'}} (m-3-2)
                        (m-2-2) edge node {h'} (m-3-2);
                \end{tikzpicture}
            \end{center}
            then since \( [f'] \circ [\alpha] = [\beta] \circ [f] \) in \( \Mc \), one has that \( f' \circ \alpha - \beta \circ f \) factors through a projective. However, since projectives are injectives, then by the universal property of injective objects applied to \( \kappa_A \), it follows that \( f' \circ \alpha - \beta \circ f \) factors through \( \kappa_A \) and that there is some morphism \( \xi: I_A \to B' \) such that
            \[
                f' \circ \alpha - \beta \circ f = \xi \circ \kappa_A.
            \]

            And considering the following diagram by definition of \( \Sigma [\alpha] \)
            \begin{center}
                \begin{tikzpicture}
                    \diagram{m}{1cm}{1cm} {
                        A \& I_A \& \Sigma A \\
                        A' \& I_{A'} \& \Sigma A', \\
                    };

                    \draw[math]
                        (m-1-1) edge[hook] node {\kappa_A} (m-1-2)
                            edge node {\alpha} (m-2-1)
                        (m-1-2) edge[two heads] node {\rho_A} (m-1-3)
                            edge node {i_{\alpha}} (m-2-2)
                        (m-1-3) edge node {\Sigma'(\alpha)} (m-2-3)

                        (m-2-1) edge[hook] node {\kappa_{A'}} (m-2-2)
                        (m-2-2) edge[two heads] node {\rho_{A'}} (m-2-3);
                \end{tikzpicture}
            \end{center}
            one gets that
            \[
                \gamma_{f'} \circ i_{\alpha} + g' \circ \xi: I_A \to C_{f'}.
            \]
            And since
            \begin{align*}
                g' \circ \beta \circ f &= g' \circ f' \circ \alpha + g' \circ \xi \circ \kappa_A \\
                &= \gamma_{f'} \circ \kappa_{A'} \circ \alpha + g' \circ \xi \circ \kappa_A \\
                &= \gamma_{f'} \circ i_{\alpha} \circ \kappa_A + g' \circ \xi \circ \kappa_A \\
                &= (\gamma_{f'} \circ i_{\alpha} + g' \circ \xi) \circ \kappa_A
            \end{align*}
            then by the pushout property of \( C_f \) it follows that there exist some unique \( \phi \) such that the following diagram commutes
            \begin{center}
                \begin{tikzpicture}
                    \diagram{m}{1cm}{1cm} {
                        A \& B \\
                        I_A \& C_f \\
                        \& \& C_{f'}. \\
                    };

                    \draw[math]
                        (m-1-1) edge node {f} (m-1-2)
                            edge node {\kappa_A} (m-2-1)
                        (m-1-2) edge node {g} (m-2-2)
                            edge[curve={height=-25pt}] node {g' \circ \beta} (m-3-3)

                        (m-2-1) edge node {\gamma_f} (m-2-2)
                            edge[curve={height=25pt}] node[swap] {\gamma_{f'} \circ i_{\alpha} + g' \circ \xi} (m-3-3)
                        (m-2-2) edge[dashed] node {\phi} (m-3-3);
                \end{tikzpicture}
            \end{center}
            Finally, needs to check that this \( \phi \) makes \autoref{eq:stablemod} commute.

            By the commutativity of the pushout diagram, the middle square of \autoref{eq:stablemod} commutes. Then it remains to check if \( [h' \circ \phi] = [\Sigma' (\alpha) \circ h] \). Consider the following pushout diagram that commutes because \( \Sigma'(\alpha) \circ h \circ \gamma_f \circ \kappa_A = \Sigma'(\alpha) \circ \rho_A \circ \kappa_A = 0 \),
            \begin{center}
                \begin{tikzpicture}
                    \diagram{m}{1cm}{1cm} {
                        A \& B \\
                        I_A \& C_f \\
                        \& \& \Sigma A', \\
                    };

                    \draw[math]
                        (m-1-1) edge node {f} (m-1-2)
                            edge node {\kappa_A} (m-2-1)
                        (m-1-2) edge node {g} (m-2-2)
                            edge[curve={height=-25pt}] node {0} (m-3-3)

                        (m-2-1) edge node {\gamma_f} (m-2-2)
                            edge[curve={height=25pt}] node[swap] {\Sigma'(\alpha) \circ h \circ \gamma_f} (m-3-3)
                        (m-2-2) edge[dashed] (m-3-3);
                \end{tikzpicture}
            \end{center}
            and the following two equations
            \[
                h' \circ \phi \circ g = h' \circ g' \circ \beta = 0 = \Sigma'(\alpha) \circ h \circ g,
            \]
            and
            \[
                \Sigma'(\alpha) \circ h \circ \gamma_f = \Sigma'(\alpha) \circ \rho_A = \rho_{A'} \circ i_{\alpha} = h' \circ \gamma_{f'} \circ i_{\alpha} = h' \circ (\phi \circ \gamma_f - g' \circ \xi) = h' \circ \phi \circ \gamma_f
            \]
            implies that there are two choices of dashed line in the above diagram that would make it commute. However, by uniqueness, this implies they are equal and so
            \[
                \Sigma'(\alpha) \circ h = h' \circ \phi.
            \]
        }
        \item {
            By a similar argument as in {\bf (TR3)} it is sufficient to only check {\bf (TR4)} for standard triangles.

            Let the following denote the construction of three standard triangles
            \begin{center}
                \begin{tikzpicture}
                    \diagram{m}{1cm}{1cm} {
                        A \& B \&[-0.5cm] B \& D \&[-0.5cm] A \& D \\
                        I_A \& C_f \& I_B \& C_n \& I_A \& C_{n \circ f} \\
                        \& \Sigma A \& \& \Sigma B \& \& \Sigma A \\
                    };

                    \draw[math]
                        (m-1-1) edge node {f} (m-1-2)
                            edge node {\kappa_A} (m-2-1)
                        (m-1-2) edge node {g} (m-2-2)
                        (m-1-3) edge node {n} (m-1-4)
                            edge node {\kappa_B} (m-2-3)
                        (m-1-4) edge node {m} (m-2-4)
                        (m-1-5) edge node {n \circ f} (m-1-6)
                            edge node {\kappa_A} (m-2-5)
                        (m-1-6) edge node {j} (m-2-6)

                        (m-2-1) edge node {\gamma_f} (m-2-2)
                            edge node {\rho_A} (m-3-2)
                        (m-2-2) edge node {h} (m-3-2)
                        (m-2-3) edge node {\gamma_n} (m-2-4)
                            edge node {\rho_B} (m-3-4)
                        (m-2-4) edge node {k} (m-3-4)
                        (m-2-5) edge node {\gamma_{n \circ f}} (m-2-6)
                            edge node {\rho_A} (m-3-6)
                        (m-2-6) edge node {l} (m-3-6);
                \end{tikzpicture}
            \end{center}
            which fit into the following (almost) commutative diagram in \( \Mod(R) \) excluding the dashed arrows
            \begin{equation}
                \label{eq:stmod_tr4}
                \begin{tikzpicture}
                    \diagram{m}{1cm}{1cm} {
                        A \& B \& C_f \& \Sigma A \\
                        A \& D \& C_{n \circ f} \& \Sigma A \\
                        \& C_n \& C_n \& \Sigma B \\
                        \& \Sigma B \& \Sigma C_f. \\
                    };
        
                    \draw[math]
                        (m-1-1) edge node {f} (m-1-2)
                            edge[equal] (m-2-1)
                        (m-1-2) edge node {g} (m-1-3)
                            edge node {n} (m-2-2)
                        (m-1-3) edge node {h} (m-1-4)
                            edge[dashed] node {\phi} (m-2-3)
                        (m-1-4) edge[equal] (m-2-4)
        
                        (m-2-1) edge node {n \circ f} (m-2-2)
                        (m-2-2) edge node {j} (m-2-3)
                            edge node {m} (m-3-2)
                        (m-2-3) edge node {l} (m-2-4)
                            edge[dashed] node {\psi} (m-3-3)
                        (m-2-4) edge node {\Sigma' f} (m-3-4)

                        (m-3-2) edge[equal] (m-3-3)
                            edge node[swap] {k} (m-4-2)
                        (m-3-3) edge node {k} (m-3-4)
                            edge node {\Sigma'(g) \circ k} (m-4-3)

                        (m-4-2) edge node {\Sigma' g} (m-4-3);
                \end{tikzpicture}
            \end{equation}
            By the \emph{construction} of \( \phi \) in {\bf TR3} there exist some \( \phi: C_f \to C_{n \circ f} \) such that the diagram including \( \phi \) commutes. However note that the \( \xi \) in the construction of \( \phi \) is simply \( 0 \) because here the diagram is in \( \Mod(R) \), and not in \( \Mc \) as is assumed in the construction in {\bf TR3}. Similarly, \( i_\alpha = i_{\Id_A} \) can be assumed to simply be \( \Id_{I_A} \). Spesifically this implies that \( \phi \circ \gamma_f = \gamma_{n \circ f} \).

            By \autoref{rem:stmod_cone_pushout_properties} it follows that \( g \) is a monomorphism. And since \( \kappa_{C_f} \) is also a monomorphism, it follows that
            \[
                \kappa_{C_f} \circ g: B \to I_{C_f}
            \]
            is also a monomorphism. Then by the injective object property of \( (\kappa_B, I_B) \) there exist some morphism \( i: I_{C_f} \to I_B \) diagram excluding the dashed arrow commutes
            \begin{center}
                \begin{tikzpicture}
                    \diagram{m}{1cm}{1cm} {
                        \& B \\
                        I_{C_f} \& I_B. \\
                    };

                    \draw[math]
                        (m-1-2) edge node[swap] {\kappa_{C_f} \circ g} (m-2-1)
                            edge node {\kappa_B} (m-2-2)

                        (m-2-1) edge node {i} (m-2-2)
                        (m-2-2) edge[dashed, curve={height=-25pt}] node {\tilde{i}} (m-2-1);
                \end{tikzpicture}
            \end{center}
            By the same argument, there exist some \( \tilde{i} \) such that the above diagram excluding \( i \) commutes.
            
            Consider the following pushout diagram of \( C_{n \circ f} \)
            \begin{center}
                \begin{tikzpicture}
                    \diagram{m}{1cm}{1cm} {
                        A \& D \\
                        I_A \& C_{n \circ f} \\
                        \& \& C_n, \\
                    };

                    \draw[math]
                        (m-1-1) edge node {n \circ f} (m-1-2)
                            edge node {\kappa_A} (m-2-1)
                        (m-1-2) edge node {j} (m-2-2)
                            edge[curve={height=-25pt}] node {m} (m-3-3)

                        (m-2-1) edge node {\gamma_{n \circ f}} (m-2-2)
                            edge[curve={height=25pt}] node[swap] {\gamma_n \circ i \circ \kappa_{C_f} \circ \gamma_f} (m-3-3)
                        (m-2-2) edge[dashed] node {\psi} (m-3-3);
                \end{tikzpicture}
            \end{center}
            the outer ``square'' commutes because
            \begin{align*}
                m \circ n \circ f &= \gamma_n \circ \kappa_B \circ f \\
                &= \gamma_n \circ i \circ \kappa_{C_f} \circ g \circ f \\
                &= \gamma_n \circ i \circ \kappa_{C_f} \circ \gamma_f \circ \kappa_A \\
            \end{align*}
            and therefore the dashed morphism \( \psi \) exists that makes the pushout diagram commute.

            Then to show that \( \psi \) makes \autoref{eq:stmod_tr4} commute in \( \Mc \), one can see that the square to the left of \( \psi \) commutes by definition, however the square on the right is a little more tricky.

            Consider the following commutative diagram
            \begin{center}
                \begin{tikzpicture}
                    \diagram{m}{1cm}{1cm} {
                        A \& \& \& B \\
                        I_A \& C_f \& I_{C_f} \& I_B \\
                        \Sigma A \& \& \& \Sigma B \\
                    };

                    \draw[math]
                        (m-1-1) edge node {f} (m-1-4)
                            edge node {\kappa_A} (m-2-1)
                        (m-1-4) edge node {\kappa_B} (m-2-4)
                            edge node[swap] {g} (m-2-2)

                        (m-2-1) edge node {\gamma_f} (m-2-2)
                            edge node {\rho_A} (m-3-1)
                        (m-2-2) edge node[swap] {\kappa_{C_f}} (m-2-3)
                        (m-2-3) edge node {i} (m-2-4)
                        (m-2-4) edge node {\rho_B} (m-3-4)

                        (m-3-1) edge node {\widetilde{\Sigma' f}} (m-3-4);
                \end{tikzpicture}
            \end{center}
            where the top square commutes because the top left and top right ``triangles'' of the diagram commute, and \( \widetilde{\Sigma' f} \) is given by the cokernel property of \( \Sigma A \). Then by \autoref{lem:stmod_sigma_well-defined_additive_endofunctor} it follows that \( \widetilde{\Sigma' f} \sim \Sigma' f \).

            Then consider the following diagram in \( \Mod(R) \)
            \begin{center}
                \begin{tikzpicture}
                    \diagram{m}{1cm}{1cm} {
                        A \& D \\
                        I_A \& C_{n \circ f} \\
                        \& \& \Sigma B \\
                    };

                    \draw[math]
                        (m-1-1) edge node {n \circ f} (m-1-2)
                            edge node {\kappa_A} (m-2-1)
                        (m-1-2) edge node {j} (m-2-2)
                            edge[curve={height=-25pt}] node {0} (m-3-3)

                        (m-2-1) edge node {\gamma_{n \circ f}} (m-2-2)
                            edge[curve={height=25pt}] node[swap] {k \circ \gamma_n \circ i \circ \kappa_{C_f} \circ \gamma_f} (m-3-3)
                        (m-2-2) edge[dashed] (m-3-3);
                \end{tikzpicture}
            \end{center}
            one can check that
            \[
                \widetilde{\Sigma'(f)} \circ l \circ j = \widetilde{\Sigma'(f)} \circ 0 = 0 = k \circ m = k \circ \psi \circ j
            \]
            and
            \begin{align*}
                k \circ \psi \circ \gamma_{n \circ f} &= k \circ \gamma_n \circ i \circ \kappa_{C_f} \circ \gamma_f \\
                &= \rho_B \circ i \circ \kappa_{C_f} \circ \gamma_f \\
                &= \widetilde{\Sigma'(f)} \circ \rho_A \\
                &= \widetilde{\Sigma'(f)} \circ l \circ \gamma_{n \circ f}.
            \end{align*}
            By uniqueness of the pushout property and definition of \( \widetilde{\Sigma'(f)} \), one has that \( k \circ \psi = \widetilde{\Sigma'(f)} \circ l \sim \Sigma'(f) \circ l \).
                
            Therefore the square to the right of \( \psi \) commutes in \( \Mc \), and not in \( \Mod(R) \), hence why \autoref{eq:stmod_tr4} was said to be \emph{almost} commuting.

            Finally, it remains to check that
            \begin{center}
                \begin{tikzpicture}
                    \diagram{m}{1cm}{1cm} {
                        C_f \& C_{n \circ f} \& C_n \& \Sigma C_f \\
                    };

                    \draw[math]
                        (m-1-1) edge node {[\phi]} (m-1-2)
                        (m-1-2) edge node {[\psi]} (m-1-3)
                        (m-1-3) edge node {\Sigma([g]) \circ [k]} (m-1-4);
                \end{tikzpicture}
            \end{center}
            is a distinguished triangle.

            Consider the following commutative diagram
            \begin{center}
                \begin{tikzpicture}
                    \diagram{m}{1cm}{1cm} {
                        A \& B \& D \\
                        I_A \& C_f \& C_{n \circ f}. \\
                    };

                    \draw[math]
                        (m-1-1) edge node {f} (m-1-2)
                            edge node {\kappa_A} (m-2-1)
                        (m-1-2) edge node {n} (m-1-3)
                            edge node {g} (m-2-2)
                        (m-1-3) edge node {j} (m-2-3)

                        (m-2-1) edge node {\gamma_f} (m-2-2)
                            edge[curve={height=25pt}] node[swap] {\gamma_{n \circ f}} (m-2-3)
                        (m-2-2) edge node {\phi} (m-2-3);
                \end{tikzpicture}
            \end{center}
            Since the left and the outer square are pushouts this implies that the right square is also a pushout.

            {\bf Marius, kan du venlegst dobbeltsjekka det følgande argumentet om at ei "omrokkering" av morfiar gir ein ny pushout, eg er svært usikker på om det fungerar.}

            Consider the following commutative diagram
            \begin{center}
                \begin{tikzpicture}
                    \diagram{m}{0.5cm}{1cm} {
                        \&[-0.5cm] B \& D \\
                        I_{C_f} \\
                        \& I_B \& C_n. \\
                    };
                    \draw[math]
                        (m-1-2) edge node {n} (m-1-3)
                            edge node[swap] {\kappa_{C_f} \circ g} (m-2-1)
                            edge node {\kappa_B} (m-3-2)
                        (m-1-3) edge node {m} (m-3-3)

                        (m-2-1) edge node[swap] {i} (m-3-2)

                        (m-3-2) edge node {\gamma_n} (m-3-3);
                \end{tikzpicture}
            \end{center}
            By the definition of \( C_n \) the square is a pushout. Rearranging the morphisms a little yields the following commutative diagram
            \begin{center}
                \begin{tikzpicture}
                    \diagram{m}{0.5cm}{1cm} {
                        \&[-0.5cm] B \& D \\
                        I_B \\
                        \& I_{C_f} \& C_n. \\
                    };
                    \draw[math]
                        (m-1-2) edge node {n} (m-1-3)
                            edge node[swap] {\kappa_B} (m-2-1)
                            edge node {\kappa_{C_f} \circ g} (m-3-2)
                        (m-1-3) edge node {m} (m-3-3)

                        (m-2-1) edge node[swap] {\tilde{i}} (m-3-2)

                        (m-3-2) edge node {\gamma_n \circ i} (m-3-3);
                \end{tikzpicture}
            \end{center}
            % \begin{center}
            %     \begin{tikzpicture}
            %         \diagram{m}{1cm}{0.5cm} {
            %             B \& \& D \\
            %             I_{C_f} \& \& C_n \\[-0.5cm]
            %             \& I_B \\
            %         };
            %         \draw[math]
            %             (m-1-1) edge node {n} (m-1-3)
            %                 edge node[swap] {\kappa_{C_f} \circ g} (m-2-1)
            %             (m-1-3) edge node {m} (m-2-3)

            %             (m-2-1) edge node {\gamma_n \circ i} (m-2-3)
            %                 edge node[swap] {i} (m-3-2)

            %             (m-3-2) edge node[swap] {\gamma_n} (m-2-3);
            %     \end{tikzpicture}
            % \end{center}
            where the new square is somehow also a pushout? {\bf (Eg trur happel greier dette fordi han anntek at \( I_B = I_{C_f} \), som medførar at \( i = \Id \). Eg har berre fått dette argumentet til å fungera om \( \gamma_n \circ i \circ \tilde{i} = \gamma_n \), som blir rart å annta med tanke på korleis eg har definert \( \Sigma \) på)}.

            WIP: Show above argument, as well as \( \psi \circ \phi = \gamma_n \circ i \circ \kappa_{C_f} \) to use the below diagram to finish the proof.

            Using the new pushout above, one can construct the following commutative diagram
            \begin{center}
                \begin{tikzpicture}
                    \diagram{m}{1cm}{1cm} {
                        B \& D \\
                        C_f \& C_{n \circ f} \\
                        I_{C_f} \& C_n \\
                    };

                    \draw[math]
                        (m-1-1) edge node {n} (m-1-2)
                            edge node {g} (m-2-1)
                        (m-1-2) edge node {j} (m-2-2)
                            edge[curve={height=-25pt}] node {m} (m-3-2)

                        (m-2-1) edge node {\phi} (m-2-2)
                            edge node {\kappa_{C_f}} (m-3-1)
                        (m-2-2) edge node {\psi} (m-3-2)

                        (m-3-1) edge node {\gamma_n \circ i} (m-3-2);
                \end{tikzpicture}
            \end{center}
        }
    \end{enumerate}
\end{proof}

By \cite[Lemma, Subsection 7.5]{Krause_2007} it follows that \( \Mc \) is in fact what one calls a \emph{algebraic triangulated category}. This will be touched on later in the thesis. % TODO: Specify where it will be touched on and the connection between Krause and Jasso-Muro's definition.