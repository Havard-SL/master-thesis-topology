A triangulated category that will be central in this thesis is the stable module category. Therefore, the definition will be given in more details than the previous two examples of triangulated categories.

Before defining the stable module category, we prove a lemma that will be used to define the morphisms.
\begin{lemma}
    \label{lem:morphisms_factoring_through_projectives_r-submodule}
    Let \( R \) be a commutative ring with identity.

    Let \( G \) be the subset of \( \Mod(R)(A, B) \) consisting of module morphisms that factor through a projective module.

    Then \( G \) is an \( R \)-submodule of \( \Mod(R)(A, B) \).
\end{lemma}
\begin{proof}
    Let \( f \) and \( g \) be two morphisms that factor through the projective objects \( P \) and \( Q \), respectively. Then we have the following commutative diagrams,
    \[
        \begin{aligned}
            \begin{tikzpicture}
                \diagram{m}{1cm}{1cm} {
                    A \& P \& B, \\
                };
    
                \draw[math]
                    (m-1-1) edge node {f_1} (m-1-2)
                        edge[curve={height=25pt}] node[swap] {f} (m-1-3)
                    (m-1-2) edge node {f_2} (m-1-3);
            \end{tikzpicture}
        \end{aligned}
        \hspace{0.5cm}
        \text{ and }
        \hspace{0.5cm}
        \begin{aligned}
            \begin{tikzpicture}
                \diagram{m}{1cm}{1cm} {
                    A \& Q \& B. \\
                };
    
                \draw[math]
                    (m-1-1) edge node {g_1} (m-1-2)
                        edge[curve={height=25pt}] node[swap] {g} (m-1-3)
                    (m-1-2) edge node {g_2} (m-1-3);
            \end{tikzpicture}
        \end{aligned}
    \]

    We can then construct the morphism
    \begin{center}
        \begin{tikzpicture}
            \diagram{m}{1cm}{1cm} {
                A \& {P \oplus Q} \& B. \\
            };

            \draw[math]
                (m-1-1) edge node {
                    \begin{psmallmatrix}
                        f_1 \\
                        g_1
                    \end{psmallmatrix}
                } (m-1-2)
                (m-1-2) edge node {(f_2, g_2)} (m-1-3);
        \end{tikzpicture}
    \end{center}

    Composing these two morphisms, we get the morphism \( f_2 \circ f_1 + g_2 \circ g_1 = f + g \). This morphism factors through \( P \oplus Q \), which is projective since it is a direct sum of projective modules. This implies that \( G \) is closed under addition.
    
    Let \( r \in R \). Then the following diagram commutes,
    \begin{center}
        \begin{tikzpicture}
            \diagram{m}{1cm}{1cm} {
                A \& P \& B, \\
            };

            \draw[math]
                (m-1-1) edge node {rf_1} (m-1-2)
                    edge[curve={height=25pt}] node[swap] {rf} (m-1-3)
                (m-1-2) edge node {f_2} (m-1-3);
        \end{tikzpicture}
    \end{center}
    which implies that \( G \) is also closed under multiplication in \( R \).
\end{proof}

In order to get a triangulated category, we have to use a specific type of ring, as defined below.
\begin{definition}
    A commutative ring with identity is called a \emph{Frobenius ring}, if every injective module is projective, and vice versa.
\end{definition}

The following is the definition of the stable module category.
\begin{definition}
    \label{def:stable_module_category}
    Let \( R \) be a \emph{Frobenius ring}.

    Then the \emph{stable module category over \( R \)}, denoted \( \Mc \), is defined in the following way:
    \begin{enumerate}
        \item {
            The objects are modules over \( R \).
        }
        \item {
            For \( A, B \in \Mc \), let \( G \) be the \( R \)-submodule from \autoref{lem:morphisms_factoring_through_projectives_r-submodule}.
        
            Then let
            \[
                \Mc(A, B) := \Mod(R)(A, B)/G.
            \]
        }
        \item {
            For \( [g] \in \Mc(B, C) \), and \( [f] \in \Mc(A, B) \), let composition be defined as follows
            \[
                [g] \circ [f] := [g \circ f].
            \]
        }
    \end{enumerate}
    This category is called the \emph{stable module category over \( R \)}.
\end{definition}

Other authors typically use the notation \( \StMod(R) \) or \( \underline{\Mod}(R) \) for the stable module category. However, for the sake of brevity this is reduced to simply \( \Mc \) in this thesis.

\begin{lemma}
    Composition in \autoref{def:stable_module_category} is well-defined.
\end{lemma}
\begin{proof}
    We need to check that two different choices of representatives of \( [f] \) and \( [g] \) yield the same value.

    Let \( f + \widetilde{f} \) and \( g + \widetilde{g} \) be two different representatives of \( [f] \) and \( [g] \), with \( \widetilde{f} \) and \( \widetilde{g} \) factoring through some projective modules.

    Then it follows that
    \[
         [g + \widetilde{g}] \circ [f + \widetilde{f}] = [g \circ f] + [\widetilde{g} \circ f] + [g \circ \widetilde{f}] + [\widetilde{g} \circ \widetilde{f}].
    \]
    
    Every term other than \( [g \circ f] \) factors through a projective and is therefore equal to \( 0 \) in \( \Mc(A, C) \).
\end{proof}

In the general definition of the stable module category, it is not required that the ring is a Frobenius ring. However, as will become apparent later on, it is a requirement to have a triangulation. Since we will only be considering stable module categories which are triangulated, this assumption is made from the beginning.

In order to admit a triangulation, we need to prove that the stable module category is additive.

% TODO: Forenkla beviset, biprodukt argumentet spesielt. Ikkje 100% sikker på at Mac Lane seier det eg ynskjer, berre 80%ish.
\begin{lemma}
    \( \Mc \) is an additive category.
\end{lemma}
\begin{proof}
    There are two parts to this proof. First, we show that \( \Mc \) is pre-additive and second, we show that \( \Mc \) has finite products. By \cite[p.\ 251]{Mac_Lane_1995}, this implies that \( \Mc \) is additive.

    To show that \( \Mc \) is pre-additive there are two properties that need to be shown
    \begin{enumerate}
        \item {
            We show that for any \( A, B \in \Mc \) that \( \Mc(A, B) \) is an abelian group.

            This follows immediately from the definition, since \( \Mc(A, B) \) is a quotient module, and is therefore an \( R \)-module and, hence, an abelian group.
        }
        \item {
            We show that composition is bilinear.

            Let \( [g], [g'] \in \Mc(B, C) \) and \( [f], [f'] \in \Mc(A, B) \). Consider the following equation,
            \begin{align*}
                ([g] + [g']) \circ ([f] + [f']) &= [g + g'] \circ [f + f'] \\
                &= \class*{(g + g') \circ (f + f')} \\
                &= \class*{g \circ f + g \circ f' + g' \circ f + g' \circ f'} \\
                &= [g] \circ [f] + [g] \circ [f'] + [g'] \circ [f] + [g'] \circ [f'],
            \end{align*}
            which is precisely bilinearity.
        }
    \end{enumerate}
    Thus, \( \Mc \) is pre-additive.

    Now we want to prove that the usual biproduct in \( \Mod(R) \) induces the product in \( \Mc \).
    
    Consider the commutative diagram for the universal property of \( A \oplus B \) as a product in \( \Mod(R) \). Taking residue classes of the morphisms yields the following commutative diagram in \( \Mc \),
    \begin{center}
        \begin{tikzpicture}
            \diagram{m}{1.5cm}{1.5cm} {
                \& X \\
                A \& A \oplus B \& B. \\
            };

            \draw[math]
                (m-1-2) edge node[swap] {[f_A]} (m-2-1)
                    edge[dashed] node {[f]} (m-2-2)
                    edge node {[f_B]} (m-2-3)

                (m-2-2) edge node[swap] {[\pi_B]} (m-2-3)
                    edge node {[\pi_A]} (m-2-1);
        \end{tikzpicture}
    \end{center}
    Let \( [g]: X \to A \oplus B \) be another morphism that satisfies the universal property.

    Then the following diagram
    \begin{center}
        \begin{tikzpicture}
            \diagram{m}{1.5cm}{1.5cm} {
                \& X \\
                A \& A \oplus B \& B \\
            };

            \draw[math]
                (m-1-2) edge node[swap] {[0]} (m-2-1)
                    edge[dashed] node {[f - g]} (m-2-2)
                    edge node {[0]} (m-2-3)

                (m-2-2) edge node[swap] {[\pi_B]} (m-2-3)
                    edge node {[\pi_A]} (m-2-1);
        \end{tikzpicture}
    \end{center}
    commutes.

    Let \( \iota_A \), and \( \iota_B \) denote the canonical split monomorphisms from the universal property of the coproduct. Since \( A \oplus B \) is a biproduct in \( \Mod(R) \), we have
    \[
        \Id_{A \oplus B} = \iota_A \circ \pi_A + \iota_B \circ \pi_B.
    \]
    Thus,
    \begin{align*}
        [f - g] &= [\Id_{A \oplus B} \circ (f - g)] \\
        &= [\iota_A \circ \pi_A \circ (f - g) + \iota_B \circ \pi_B \circ (f - g)] \\
        &= [\iota_A] \circ [\pi_A \circ (f - g)] + [\iota_B] \circ [\pi_B \circ (f - g)] \\
        &= [0],
    \end{align*}
    which implies that \( [f] = [g] \), and that \( A \oplus B \) is a product in \( \Mc \), and by the statement at the start, also a biproduct.
\end{proof}

The next step in creating a triangulation for \( \Mc \) is to define the shift functor. In the stable module category, the shift functor is the inverse to the ``syzygy functor.''

The syzygy functor as well as the shift functor exists in the category of modules over a commutative ring with identity (e.g., a Frobenius ring) because it has \emph{enough projectives} and \emph{enough injectives}. Having enough projectives means that for any module \( A \), there exists some projective module \( P_A \), as well as an epimorphism \( \pi_A \) from \( P_A \) to \( A \). The choice of \( P_A \) and consequently \( \pi_A \) is not necessarily unique up to isomorphism, and two different choices of \( P_A \) could be non-isomorphic. Similarly, having enough injectives means that for any module \( A \) there exists some injective module \( I_A \) along with a monomorphism \( \iota_A \) from \( A \) to \( I_A \). Equal to the projective case, \( I_A \) is not necessarily unique up to isomorphism and \( \iota_A \) is not necessarily unique.

The definition of the syzygy functor is closely tied to a choice of \( P_A \) and \( \pi_A \) for every object \( A \).

\begin{definition}[The syzygy functor \( \Omega \)]
    \label{def:stmod_omega}
    Let \( R \) be a Frobenius ring.

    Let \( P = \set*{\tuple*{P_A, \pi_A, \Omega A}}_{A \in \Mod(R)} \) be a collection of tuples for every object \( A \), where \( P_A \) is a projective module, \( \pi_A: P_A \twoheadrightarrow A \) an epimorphism, and \( \Omega A \) is a kernel of \( \pi_A \). This exists because \( \Mod(R) \) has enough projectives.

    Then define \( \Omega \) as the assignment of objects and morphisms in \( \Mc \) as follows:
    \begin{itemize}
        \item {
            For any object \( A \) in \( \Mc \), let \( \Omega A \) be as above.
        }
        \item {
            For any \( [f] \in \Mc(A, B) \), let \( \Omega [f] \) be constructed as follows:

            Consider the following diagram in \( \Mod(R) \) excluding the dashed arrows,
            \begin{center}
                \begin{tikzpicture}
                    \diagram{m}{1cm}{1cm} {
                        \Omega A \& P_A \& A \\
                        \Omega B \& P_B \& B. \\
                    };

                    \draw[math]
                        (m-1-1) edge[tailed] node {\iota_A} (m-1-2)
                            edge[dashed] node[swap] {\Omega f} (m-2-1)
                        (m-1-2) edge[two headed] node {\pi_A} (m-1-3)
                            edge[dashed] node {p_f} (m-2-2)
                        (m-1-3) edge node {f} (m-2-3)

                        (m-2-1) edge[tailed] node {\iota_B} (m-2-2)
                        (m-2-2) edge[two headed] node {\pi_B} (m-2-3);
                \end{tikzpicture}
            \end{center}
            Let \( \Omega f: \Omega A \to \Omega B \) be any morphism that makes the above diagram commute for some \( p_f: P_A \to P_B \).

            Then define \( \Omega [f] := \class*{\Omega f} \).
        }
    \end{itemize}
    This is called the \emph{syzygy functor}.
\end{definition}

The goal is to show that \( \Omega: \Mc \to \Mc \) is a well-defined endofunctor, and later showing that it is in fact an additive auto-equivalence of categories.

\begin{lemma}
    \label{lem:stmod_omega_f_is_well_defined}
    \( \Omega \) is a well-defined assignment of morphisms.
\end{lemma}
\begin{proof}
    There are three things that need to be proven. First, the existence of an \( \Omega f \) for any \( f \). Second, we need to show that for any \( p_f \), then \( \Omega \) still yields the same morphism in \( \Mc \). Third, we need to show that if \( [f] = [g] \), then \( \Omega [f] = \Omega [g] \).

    First, we have that for a morphism \( f: A \to B \), there exists a morphism \( p_f \) from the lifting property of projective modules such that the right square in the definition commutes.

    Furthermore, since
    \[
        \pi_B \circ p_f \circ \iota_A = f \circ \pi_A \circ \iota_A = f \circ 0 = 0,
    \]
    we have from the universal kernel property that there exists a morphism, \( \Omega f \), dependent on the choice of \( p_f \), such that the left square in the definition commutes. This is therefore a valid choice of \( \Omega f \).

    Second, let \( p_f \) and \( \widetilde{p_f} \) be two different projective morphisms that yield morphisms \( \Omega f \) and \( \widetilde{\Omega f} \), respectively. Then we have the following commutative diagram excluding the dashed arrow,
    \begin{center}
        \begin{tikzpicture}
            \diagram{m}{1cm}{2cm} {
                \Omega A \& P_A \& A \\
                \Omega B \& P_B \& B. \\
            };

            \draw[math]
                (m-1-1) edge[tailed] node {\iota_A} (m-1-2)
                    edge[swap] node {\Omega f - \widetilde{\Omega f}} (m-2-1)
                (m-1-2) edge[two headed] node {\pi_A} (m-1-3)
                    edge[dashed, swap] node {\phi} (m-2-1)
                    edge node {p_f - \widetilde{p_f}} (m-2-2)
                (m-1-3) edge node {f -f = 0} (m-2-3)

                (m-2-1) edge[tailed] node {\iota_B} (m-2-2)
                (m-2-2) edge[two headed] node {\pi_B} (m-2-3);
        \end{tikzpicture}
    \end{center}

    Since
    \[
        \pi_B \circ \tuple*{p_f - \widetilde{p_f}} = \tuple*{f - f} \circ \pi_A = 0
    \]
    there exists a morphism \( \phi \) induced by the kernel property of \( \Omega B \), such that the lower triangle in the diagram commutes. Then because of the monomorphism property of \( \iota_B \), we also get that the upper triangle commutes. This implies that the morphism \( \Omega f - \widetilde{\Omega f} \) factors through \( P_A \), a projective module. Therefore,
    \[
        [0] = \class*{\Omega f - \widetilde{\Omega f}} = \class*{\Omega f} - \class*{\widetilde{\Omega f}}
    \]
    which implies that \( \Omega [f] \) is independent of the choice of \( p_f \).

    Third, we need to show that if \( [f] = [g] \), then \( \Omega [f] = \Omega [g] \). Consider the following commutative diagram excluding the dashed arrow,
    \begin{center}
        \begin{tikzpicture}
            \diagram{m}{1cm}{3cm} {
                \Omega A \& P_A \& A \\
                \&\& P \\
                \Omega B \& P_B \& B. \\
            };

            \draw[math]
                (m-1-1) edge[tailed] node {\iota_A} (m-1-2)
                    edge node {\Omega f - \Omega g} (m-3-1)
                (m-1-2) edge[two headed] node {\pi_A} (m-1-3)
                    edge node {p_f - p_g} (m-3-2)
                (m-1-3) edge[swap] node {(f - g)_1} (m-2-3)
                    edge[curve={height=-25pt}] node {f - g} (m-3-3)

                (m-2-3) edge[swap, dashed] node {\theta} (m-3-2)
                    edge[swap] node {(f - g)_2} (m-3-3)

                (m-3-1) edge[tailed] node {\iota_B} (m-3-2)
                (m-3-2) edge[two headed] node[swap] {\pi_B} (m-3-3);
        \end{tikzpicture}
    \end{center}

    Let \( P \) be the projective module that \( f - g \) factors through. Then from the projective property, there exists a morphism \( \theta: P \to P_B \), which causes the lower triangle to commute.

    Let \( p_{f - g} := \theta \circ (f - g)_1 \circ \pi_A \). By construction, we have that both \( p_f - p_g \) and \( p'_{f - g} \) are morphisms that would make the right hand square commute.
    
    But since
    \[
        p_{f - g} \circ \iota_A = \theta \circ (f-g)_1 \circ \pi_A \circ \iota_A = \theta \circ (f-g)_1 \circ 0 = 0,
    \]
    the following diagram,
    \begin{center}
        \begin{tikzpicture}
            \diagram{m}{1cm}{1cm} {
                \Omega A \& P_A \& A \\
                \Omega B \& P_B \& B, \\
            };

            \draw[math]
                (m-1-1) edge[tailed] node {\iota_A} (m-1-2)
                    edge node {0} (m-2-1)
                (m-1-2) edge[two headed] node {\pi_A} (m-1-3)
                    edge node {p_{f - g}} (m-2-2)
                (m-1-3) edge node {f - g} (m-2-3)

                (m-2-1) edge[tailed] node {\iota_B} (m-2-2)
                (m-2-2) edge[two headed] node {\pi_B} (m-2-3);
        \end{tikzpicture}
    \end{center}
    commutes.

    However, by the second part of this proof, this implies that \( [0] = [\Omega f - \Omega g] \), and therefore \( \Omega [f] = \Omega [g] \).
\end{proof}

Now that \( \Omega \) is a well-defined assignment of objects and morphisms, it only remains to prove functoriality.

\begin{lemma}
    \label{lem:stmod_omega_endofunctor}
    \( \Omega \) is an endofunctor on \( \Mc \).
\end{lemma}
\begin{proof}
    To show that \( \Omega \) is a functor we must prove functoriality.

    First, we show that \( \Omega \) is preserves composition. Let \( A, B, C \in \Mc \). Then by the definition of \( \Omega \), we have the following commutative diagram,
    \begin{center}
        \begin{tikzpicture}
            \diagram{m}{1cm}{2cm} {
                \Omega A \& P_A \& A \\
                \Omega B \& P_B \& B \\
                \Omega C \& P_C \& C. \\
            };

            \draw[math]
                (m-1-1) edge[tailed] node {\iota_A} (m-1-2)
                    % edge[curve={height=30pt}, swap] node {\Omega (g \circ f)} (m-3-1)
                    edge node {\Omega f} (m-2-1)
                (m-1-2) edge[two headed] node {\pi_A} (m-1-3)
                    edge node {p_f} (m-2-2)
                (m-1-3) edge node {f} (m-2-3)

                (m-2-1) edge[tailed] node {\iota_B} (m-2-2)
                    edge node {\Omega g} (m-3-1)
                (m-2-2) edge[two headed] node {\pi_B} (m-2-3)
                    edge node {p_g} (m-3-2)
                (m-2-3) edge node {g} (m-3-3)

                (m-3-1) edge[tailed] node {\iota_A} (m-3-2)
                (m-3-2) edge[two headed] node {\pi_B} (m-3-3);
        \end{tikzpicture}
    \end{center}

    Considering the composition of the vertical morphisms, we end up with the following commutative diagram
    \begin{center}
        \begin{tikzpicture}
            \diagram{m}{1cm}{2cm} {
                \Omega A \& P_A \& A \\
                \Omega C \& P_C \& C. \\
            };

            \draw[math]
                (m-1-1) edge[tailed] node {\iota_A} (m-1-2)
                    edge[swap] node {\Omega g \circ \Omega f} (m-2-1)
                (m-1-2) edge[two headed] node {\pi_A} (m-1-3)
                    edge node {p_g \circ p_f} (m-2-2)
                (m-1-3) edge node {g \circ f} (m-2-3)

                (m-2-1) edge[tailed] node {\iota_C} (m-2-2)
                (m-2-2) edge[two headed] node {\pi_C} (m-2-3);
        \end{tikzpicture}
    \end{center}
    Since \( \Omega \) is a well-defined assignment of morphisms, this implies
    \[
        (\Omega [g]) \circ (\Omega [f]) = \class*{(\Omega g) \circ (\Omega f)} = \class*{\Omega (g \circ f)} = \Omega \class*{g \circ f}.
    \]

    Second, we need to show that \( \Omega [\Id_A] = [\Id_{\Omega A}] \) in \( \Mc \).

    We can see that every square and triangle in the following diagram,
    \begin{center}
        \begin{tikzpicture}
            \diagram{m}{1cm}{2cm} {
                \Omega A \& P_A \& A \\
                \Omega A \& P_A \& A, \\
            };

            \draw[math]
                (m-1-1) edge[tailed] node {\iota_A} (m-1-2)
                    edge[swap] node {\Id_{\Omega A}} (m-2-1)
                (m-1-2) edge[two headed] node {\pi_A} (m-1-3)
                    edge node {\Id_{P_A}} (m-2-2)
                (m-1-3) edge node {\Id_A} (m-2-3)

                (m-2-1) edge[tailed] node {\iota_A} (m-2-2)
                (m-2-2) edge[two headed] node {\pi_A} (m-2-3);
        \end{tikzpicture}
    \end{center}
    commutes.

    Therefore, since \( \Omega \) is a well-defined assignment of morphisms, \( \Omega \class*{\Id_A} = \class*{\Id_{\Omega A}} \).
\end{proof}

Finally, we need the following result to show that \( \Omega \) is additive.

\begin{lemma}
    \label{lem:stmod_omega_additive_functor}
    \( \Omega \) is an additive functor.
\end{lemma}
\begin{proof}
    We must show that \( \Omega [f + g] = \Omega [f] + \Omega [g] \).
    
    Since the diagrams
    \[
        \begin{aligned}
            \begin{tikzpicture}
                \diagram{m}{1cm}{2cm} {
                    {\Omega A} \& {P_A} \& A \\
                    {\Omega B} \& {P_B} \& B \\
                };

                \draw[math]
                    (m-1-1) edge[tailed] node {\iota_A} (m-1-2)
                        edge node {\Omega (f + g)} (m-2-1)
                    (m-1-2) edge[two headed] node {\pi_A} (m-1-3)
                        edge node {p_{f + g}} (m-2-2)
                    (m-1-3) edge node[swap] {f + g} (m-2-3)

                    (m-2-1) edge[tailed] node {\iota_B} (m-2-2)
                    (m-2-2) edge[two headed] node {\pi_B} (m-2-3);
            \end{tikzpicture}
        \end{aligned}
        \quad
        \text{and}
        \quad
        \begin{aligned}
            \begin{tikzpicture}
                \diagram{m}{1cm}{2cm} {
                    {\Omega A} \& {P_A} \& A \\
                    {\Omega B} \& {P_B} \& B \\
                };

                \draw[math]
                    (m-1-1) edge[tailed] node {\iota_A} (m-1-2)
                        edge node {\Omega f + \Omega g} (m-2-1)
                    (m-1-2) edge[two headed] node {\pi_A} (m-1-3)
                        edge node {p_f + p_g} (m-2-2)
                    (m-1-3) edge node[swap] {f + g} (m-2-3)

                    (m-2-1) edge[tailed] node {\iota_B} (m-2-2)
                    (m-2-2) edge[two headed] node {\pi_B} (m-2-3);
            \end{tikzpicture}
        \end{aligned}
    \]
    both commute, and since \( \Omega \) is a well-defined assignment of morphisms, this implies \( \Omega [f + g] = \Omega [f] + \Omega [g] \).
\end{proof}

We can define another functor, the cosyzygy functor \( \Sigma \), which will turn out to be the inverse of the syzygy functor \( \Omega \) and the shift functor in the triangulation of \( \Mc \).

\begin{definition}[The cosyzygy functor \( \Sigma \)]
    \label{def:stmod_sigma}
    Let \( R \) be a Frobenius ring.

    Let \( I = \set*{\tuple*{I_A, \kappa_A, \Sigma A}}_{A \in \Mod(R)} \) be a collection of tuples for every object \( A \), where \( I_A \) is an injective module, \( \kappa_A: A \rightarrowtail I_A \) a monomorphism, and \( \Sigma A \) is a cokernel of \( \kappa_A \). This exists because \( \Mod(R) \) has enough injectives.

    Then define \( \Sigma \) as the assignment of objects and morphisms in \( \Mc \) as follows:
    \begin{itemize}
        \item {
            For any object \( A \) in \( \Mc \), let \( \Sigma A \) be as above.
        }
        \item {
            For \( [f] \in \Mc(A, B) \), let \( \Sigma [f] \) be constructed as follows:

            Consider the following diagram in \( \Mod(R) \) excluding the dashed arrows,
            \begin{center}
                \begin{tikzpicture}
                    \diagram{m}{1cm}{1cm} {
                        A \& I_A \& \Sigma A \\
                        B \& I_B \& \Sigma B. \\
                    };

                    \draw[math]
                        (m-1-1) edge[tailed] node {\kappa_A} (m-1-2)
                            edge node {f} (m-2-1)
                        (m-1-2) edge[two headed] node {\rho_A} (m-1-3)
                            edge[dashed] node {i_f} (m-2-2)
                        (m-1-3) edge[dashed] node {\Sigma f} (m-2-3)

                        (m-2-1) edge[tailed] node {\kappa_B} (m-2-2)
                        (m-2-2) edge[two headed] node {\rho_B} (m-2-3);
                \end{tikzpicture}
            \end{center}

            Let \( \Sigma f: \Sigma A \to \Sigma B \) be any morphism that makes the above diagram commute for some \( i_f: I_A \to I_B \).

            Then define \( \Sigma [f] := \class*{\Sigma f} \).
        }
    \end{itemize}
    This is called the \emph{cosyzygy functor.}
\end{definition}

It remains to show that every desired property of \( \Sigma \), like being well-defined, an endofunctor, and additive, translates to \( \Sigma \).
\begin{lemma}
    \label{lem:stmod_sigma_well-defined_additive_endofunctor}
    \( \Sigma \) is a well-defined and additive endofunctor on \( \Mc \).
\end{lemma}
\begin{proof}
    The proofs of the various statements are entirely dual to the proofs of \autoref{lem:stmod_omega_f_is_well_defined}, \autoref{lem:stmod_omega_endofunctor} and \autoref{lem:stmod_omega_additive_functor}.
\end{proof}

Now we can finally show that \( \Sigma = \Omega^{-1} \).

\begin{theorem}
    \label{thm:stmod_omega_autoeq}
    \( \Omega \) is an auto-equivalence with inverse \( \Sigma \).
\end{theorem}
\begin{proof}
    We will only show that \( \Id_{\Mc} \) is naturally isomorphic to \( \Sigma \Omega \). The omitted part that \( \Id_{\Mc} \) is naturally isomorphic to \( \Omega \Sigma \) is very similar, and uses many dual properties.

    First, we show that for any \( A \in \Mc \), there exists an isomorphism \( A \to \Sigma \Omega A \). Consider the following diagram excluding the dashed arrows, where the rows come from the definitions of \( \Omega A \) and \( \Sigma \Omega A \),
    \begin{center}
        \begin{tikzpicture}
            \diagram{m}{1cm}{2cm} {
                \Omega A \& P_A \& A \\
                \Omega A \& I_{\Omega A} \& \Sigma \Omega A \\
                \Omega A \& P_A \& A. \\
            };

            \draw[math]
                (m-1-1) edge[tailed] node {\iota_A} (m-1-2)
                    edge[equality] (m-2-1)
                (m-1-2) edge[two headed] node {\pi_A} (m-1-3)
                    edge[dashed] node {i_{\phi_1}} (m-2-2)
                (m-1-3) edge[dashed] node {\phi_1} (m-2-3)

                (m-2-1) edge[tailed] node {\kappa_{\Omega A}} (m-2-2)
                    edge[equality] (m-3-1)
                (m-2-2) edge[two headed] node {\rho_{\Omega A}} (m-2-3)
                    edge[dashed] node {i_{\phi_2}} (m-3-2)
                (m-2-3) edge[dashed] node {\phi_2} (m-3-3)

                (m-3-1) edge[tailed] node {\iota_A} (m-3-2)
                (m-3-2) edge[two headed] node {\pi_A} (m-3-3);
        \end{tikzpicture}
    \end{center}
    There exists some \( i_{\phi_1}: P_A \to I_{\Omega A} \) from the injective property of \( I_{\Omega A} \), induced by \( \iota_A \), which would make the top left square commute. In addition, since \( \Mod(R) \) is an abelian category, \( A \) is a cokernel of \( \iota_A \), and by
    \[
        \rho_{\Omega A} \circ i_{\phi_1} \circ \iota_A = \rho_{\Omega A} \circ \kappa_{\Omega A} = 0,
    \]
    we get from the cokernel property that there is a uniquely induced morphism \( \phi_1: A \to \Sigma\Omega A \), which makes the top right square commute. Then by doing the same for the lower rectangle of the diagram, using the fact that every projective module is also injective, we get the morphisms \( i_{\phi_2} \) and \( \phi_2 \) by similar arguments. This makes the entire diagram, including the dashed arrows, commute.

    To show that \( \phi_1 \) and \( \phi_2 \) are isomorphisms, consider the following commutative diagram excluding the dashed arrow,
    \begin{center}
        \begin{tikzpicture}
            \diagram{m}{1cm}{2cm} {
                \Omega A \& P_A \& A \\
                \Omega A \& P_A \& A. \\
            };

            \draw[math]
                (m-1-1) edge[tailed] node {\iota_A} (m-1-2)
                    edge[swap] node {\Id_{\Omega A} \circ \Id_{\Omega A} - \Id_{\Omega A} = 0} (m-2-1)
                (m-1-2) edge[two headed] node {\pi_A} (m-1-3)
                    edge[swap] node {i_{\phi_2} \circ i_{\phi_1} - \Id_{P_A}} (m-2-2)
                (m-1-3) edge[swap, dashed] node {\theta} (m-2-2)
                    edge node {\phi_2 \circ \phi_1 - \Id_A} (m-2-3)

                (m-2-1) edge[tailed] node {\iota_A} (m-2-2)
                (m-2-2) edge[two headed] node {\pi_A} (m-2-3);
        \end{tikzpicture}
    \end{center}

    Since
    \[
        (i_{\phi_2} \circ i_{\phi_1} - \Id_{P_A}) \circ \iota_A = \iota_A \circ 0 = 0,
    \]
    we have from the cokernel property that there exists a morphism \( \theta: A \to P_A \) such that the upper triangle commutes. Thus,
    \[
        (\phi_2 \circ \phi_1 - \Id_A) \circ \pi_A = \pi_A \circ (i_{\theta_2} \circ i_{\theta_1} - \Id_{P_A}) = \pi_A \circ \theta \circ \pi_A,
    \]
    and since \( \pi_A \) is an epimorphism, we get that the lower triangle commutes. This implies that \( [\phi_2 \circ \phi_1 - \Id_A] = [0] \), which implies \( [\phi_2 \circ \phi_1] = [\Id_A] \).
    
    By a similar argument that is omitted for brevity, we can show that \( [\phi_1 \circ \phi_2] = [\Id_{\Sigma\Omega A}] \), which means that \( [\phi_1] \) and \( [\phi_2] \) are isomorphisms from \( A \) to \( \Sigma\Omega A \).

    Finally, to show that these isomorphisms are natural, let \( [f] \in \Mc(A, B) \) and consider the following two commutative diagrams
    \begin{center}
        \begin{tikzpicture}
            \diagram{m}{1cm}{2cm} {
                \Omega A \& P_A \& A \\
                \Omega A \& I_{\Omega A} \& \Sigma \Omega A \\
                \Omega B \& I_{\Omega B} \& \Sigma \Omega B, \\
            };

            \draw[math]
                (m-1-1) edge[tailed] node {\iota_A} (m-1-2)
                    edge[equality] (m-2-1)
                (m-1-2) edge[two headed] node {\pi_A} (m-1-3)
                    edge node {i_{\phi^A_1}} (m-2-2)
                (m-1-3) edge node {\phi^A_1} (m-2-3)

                (m-2-1) edge[tailed] node {\kappa_{\Omega A}} (m-2-2)
                    edge node {\Omega f} (m-3-1)
                (m-2-2) edge[two headed] node {\rho_{\Omega A}} (m-2-3)
                    edge node {i_{\Omega f}} (m-3-2)
                (m-2-3) edge node {\Sigma \Omega f} (m-3-3)

                (m-3-1) edge[tailed] node {\kappa_{\Omega B}} (m-3-2)
                (m-3-2) edge[two headed] node {\rho_{\Omega B}} (m-3-3);
        \end{tikzpicture}
    \end{center}
    and
    \begin{center}
        \begin{tikzpicture}
            \diagram{m}{1cm}{2cm} {
                \Omega A \& P_A \& A \\
                \Omega B \& P_B \& B \\
                \Omega B \& I_{\Omega B} \& \Sigma \Omega B. \\
            };

            \draw[math]
                (m-1-1) edge[tailed] node {\iota_A} (m-1-2)
                    edge node {\Omega f} (m-2-1)
                (m-1-2) edge[two headed] node {\pi_A} (m-1-3)
                    edge node {p_f} (m-2-2)
                (m-1-3) edge node {f} (m-2-3)

                (m-2-1) edge[tailed] node {\iota_B} (m-2-2)
                    edge[equality] (m-3-1)
                (m-2-2) edge[two headed] node {\pi_B} (m-2-3)
                    edge node {i_{\phi_1^B}} (m-3-2)
                (m-2-3) edge node {\phi_1^B} (m-3-3)

                (m-3-1) edge[tailed] node {\kappa_{\Omega B}} (m-3-2)
                (m-3-2) edge[two headed] node {\rho_{\Omega B}} (m-3-3);
        \end{tikzpicture}
    \end{center}

    These diagrams give rise to the following commutative diagram, excluding the dashed arrow,
    \begin{center}
        \begin{tikzpicture}
            \diagram{m}{1cm}{2.7cm} {
                \Omega A \& P_A \& A \\
                \Omega B \& I_{\Omega B} \& \Sigma \Omega B, \\
            };

            \draw[math]
                (m-1-1) edge[tailed] node {\iota_A} (m-1-2)
                    edge[swap] node {\Id_{\Omega B} \circ (\Omega f) - (\Omega f) \circ \Id_{\Omega A} = 0} (m-2-1)
                (m-1-2) edge[two headed] node {\pi_A} (m-1-3)
                    edge[swap] node {i_{\phi_1^B} \circ p_f - i_{\Omega f} \circ i_{\phi_1^A}} (m-2-2)
                (m-1-3) edge[swap, dashed] node {\theta} (m-2-2)
                    edge node {\phi_1^B \circ f - (\Sigma \Omega f) \circ \phi_1^A} (m-2-3)

                (m-2-1) edge[tailed] node[swap] {\kappa_{\Omega B}} (m-2-2)
                (m-2-2) edge[two headed] node[swap] {\rho_{\Omega B}} (m-2-3);
        \end{tikzpicture}
    \end{center}
    where from the cokernel property of \( A \), we get an induced morphism \( \theta \). Furthermore, from the epimorphism property of \( \pi_A \), the lower triangle commutes. Since \( R \) is assumed to be a Frobenius ring, \( I_{\Omega B} \) is also projective. This implies
    \[
        [\phi_1^B \circ f] = [(\Sigma \Omega f) \circ \phi_1^A],
    \]
    which means that \( [\phi_1] \) is a natural isomorphism from \( \Id_{\Mc} \) to \( \Sigma \Omega \), with inverse \( [\phi_2] \).

    As mentioned at the start, the proof that \( \Id_{\Mc} \) is naturally isomorphic to \( \Omega \Sigma \) is very similar to the above proof, but using dual properties.
\end{proof}

An interesting consequence of how \( \Omega \) and \( \Sigma \) are defined is that if we were to choose a different \( P \) or \( I \) in their definitions, then it would yield different, but naturally isomorphic functors. A proof of this statement for \( \Sigma \) can be found in \cite[p.\ 13]{Happel_1988}, with the proof for \( \Omega \) being dual.

Before we can show the triangulation of \( \Mc \), we first need to define the cone of a morphism. The definition of a cone, as well as the definition of the distinguished triangles leans heavily upon \cite[Section 1.2.5]{Happel_1988}.

\begin{definition}
    \label{def:stmod_cone}
    Let \( f \in \Mod(R)(A, B) \).
    
    Then define a \emph{cone of \( f \)} to be a pushout object of the following diagram
    \begin{center}
        \begin{tikzpicture}
            \diagram{m}{1cm}{1cm} {
                A \& B \\
                I_A, \\
            };

            \draw[math]
                (m-1-1) edge node {f} (m-1-2)
                    edge node[swap] {\kappa_A} (m-2-1);
        \end{tikzpicture}
    \end{center}
    and denote it by \( C_f \).

    This pushout also defines two morphisms \( g \) and \( \gamma_f \) which fit into the following pushout square,
    \begin{center}
        \begin{tikzpicture}
            \diagram{m}{1cm}{1cm} {
                A \& B \\
                I_A \& C_f. \\
            };

            \draw[math]
                (m-1-1) edge node {f} (m-1-2)
                    edge node[swap] {\kappa_A} (m-2-1)
                (m-1-2) edge node {g} (m-2-2)

                (m-2-1) edge node {\gamma_f} (m-2-2);
        \end{tikzpicture}
    \end{center}
\end{definition}

Following the definition of a cone, we can define the standard triangles of \( \Mc \) as induced from some triangles in \( \Mod(R) \). The following remark walks through the construction, which will be very useful in proofs.

\begin{remark}
    \label{rem:stmod_cone}
    Let \( f \in \Mod(R)(A, B) \), and let \( C_f, g \) and \( \gamma_f \) be as in \autoref{def:stmod_cone}.
    
    Consider the following commutative diagram excluding the dashed arrows,
    \begin{center}
        \begin{tikzpicture}
            \diagram{m}{1cm}{1cm} {
                A \& B \\
                I_A \& C_f \\
                \& \Sigma A. \\
            };

            \draw[math]
                (m-1-1) edge node {f} (m-1-2)
                    edge[swap] node {\kappa_A} (m-2-1)
                (m-1-2) edge node {g} (m-2-2)
                    edge[curve={height=-25pt}] node {0} (m-3-2)

                (m-2-1) edge node {\gamma_f} (m-2-2)
                    edge node {\rho_A} (m-3-2)
                (m-2-2) edge[dashed] node {h} (m-3-2);
        \end{tikzpicture}
    \end{center}
    Then there exists some morphism \( h: C_f \to \Sigma A \) from the pushout property of \( C_f \).

    This diagram will form the backbone of the distinguished triangles in \( \Mc \).
\end{remark}

Now, we can finally define the triangulation on \( \Mc \).

\begin{definition}
    \label{def:stmod_delta}
    Let \( \Delta \) be the collection of triangles in \( \Mc \) isomorphic to any triangle of the form
    \begin{center}
        \begin{tikzpicture}
            \diagram{m}{1cm}{1cm} {
                A \& B \& C_f \& \Sigma A \\
            };

            \draw[math]
                (m-1-1) edge node {[f]} (m-1-2)
                (m-1-2) edge node {[g]} (m-1-3)
                (m-1-3) edge node {[h]} (m-1-4);
        \end{tikzpicture}
    \end{center}
    for any \( f \in \Mod(R)(A, B) \), and where \( C_f \), \( g \), and \( h \) are as defined in \autoref{rem:stmod_cone}.
\end{definition}

There are some important details from the definition of standard triangles that will be important in proving that \( \Mc \) has a triangulation.

\begin{remark}
    \label{rem:stmod_cone_pushout_properties}
    Consider the objects and morphisms in the above remark.

    Since \( \Mod(R) \) is an abelian category, it follows that since \( \kappa_A \) is a monomorphism, then \( g \) is also a monomorphism as \( \ker(g) = \ker(\kappa_A) = 0 \). Likewise, it follows that \( \coker(g) \cong \coker(\kappa_A) \cong \Sigma A \), and therefore \( h \) is a cokernel morphism of \( g \).
\end{remark}

Since morphisms in \( \Mc \) are residue classes and can therefore have multiple representatives, there are multiple cones for each morphism. Each cone yields a different standard triangle, and there are therefore multiple standard triangles for each morphism. This turns out to not be a problem since it does not conflict with the definition of a triangulated category. In particular, the use of the word ``standard triangle'' and ``cone'' above, still aligns with the definitions given in the definition of a triangulated category.

The following lemma is needed to prove {\bf (TR4)}, however, it also hints to the fact that we could have chosen \( \Delta \) differently, as is done in \cite[Definition 4.16]{Johan_Bachelor}.
\begin{lemma}
    \label{lem:stmod_pushout_different_injectives_isomorphic}
    Let
    \[
        \begin{aligned}
            \begin{tikzpicture}
                \diagram{m}{1cm}{1cm} {
                    A \& B \\
                    I \& C \\
                };
    
                \draw[math]
                    (m-1-1) edge node {f} (m-1-2)
                        edge node[swap] {\kappa} (m-2-1)
                    (m-1-2) edge node {g} (m-2-2)
    
                    (m-2-1) edge node {\gamma} (m-2-2);
            \end{tikzpicture}
        \end{aligned}
        \hspace{0.5cm}
        \text{ and }
        \hspace{0.5cm}
        \begin{aligned}
            \begin{tikzpicture}
                \diagram{m}{1cm}{1cm} {
                    A \& B \\
                    I' \& C' \\
                };
    
                \draw[math]
                    (m-1-1) edge node {f} (m-1-2)
                        edge node[swap] {\kappa'} (m-2-1)
                    (m-1-2) edge node {g'} (m-2-2)
    
                    (m-2-1) edge node {\gamma'} (m-2-2);
            \end{tikzpicture}
        \end{aligned}
    \]
    be two pushout squares in \( \Mod(R) \), with \( \kappa \) and \( \kappa' \) monomorphisms into injective objects \( I \) and \( I' \).

    Then there exists an isomorphism \( [\alpha]: C \to C' \) in \( \Mc \).
    
    In addition, given a morphism \( i: I \to I' \), such that \( i \circ \kappa = \kappa' \), \( \alpha \) has the following properties:
    \begin{itemize}
        \item \( \alpha \circ g = g' \), and
        \item \( \alpha \circ \gamma = \gamma' \circ i \).
    \end{itemize}
\end{lemma}
\begin{proof}
    Since both \( I \) and \( I' \) are injective objects and \( \kappa \) and \( \kappa' \) are monomorphisms, there exists morphisms \( i \) and \( i' \) such that the following two diagrams,
    \[
        \begin{aligned}
            \begin{tikzpicture}
                \diagram{m}{1cm}{0.5cm} {
                    \& A \\
                    I \& \& I', \\
                };

                \draw[math]
                    (m-1-2) edge node[swap] {\kappa} (m-2-1)
                        edge node {\kappa'} (m-2-3)

                    (m-2-1) edge node {i} (m-2-3);
            \end{tikzpicture}
        \end{aligned}
        \hspace{0.5cm}
        \text{ and }
        \hspace{0.5cm}
        \begin{aligned}
            \begin{tikzpicture}
                \diagram{m}{1cm}{0.5cm} {
                    \& A \\
                    I \& \& I', \\
                };

                \draw[math]
                    (m-1-2) edge node[swap] {\kappa} (m-2-1)
                        edge node {\kappa'} (m-2-3)

                    (m-2-3) edge node {i'} (m-2-1);
            \end{tikzpicture}
        \end{aligned}
    \]
    commute.

    Using those morphisms, we can create the following commutative diagram, where \( \alpha \) and \( \beta \) are the induced morphisms from the pushout property of \( C \) and \( C' \) respectively, 
    \begin{center}
        \begin{tikzpicture}
            \diagramorigin{m}{1cm}{1.5cm} {
                A \& B \\
                I \& C \\
                I' \& \& C' \\
                I \& \& \& C. \\
            };

            \draw[math]
                (m-1-1) edge node {f} (m-1-2)
                    edge[curve={height=50pt}] node {\kappa} (m-4-1)
                    edge[curve={height=25pt}] node {\kappa'} (m-3-1)
                    edge node {\kappa} (m-2-1)
                (m-1-2) edge node {g} (m-2-2)
                    edge[curve={height=-25pt}] node {g'} (m-3-3)
                    edge[curve={height=-50pt}] node {g} (m-4-4)

                (m-2-1) edge node {\gamma} (m-2-2)
                    edge node {i} (m-3-1)
                (m-2-2) edge[dashed] node {\alpha} (m-3-3)

                (m-3-1) edge node {\gamma'} (m-3-3)
                    edge node {i'} (m-4-1)
                (m-3-3) edge[dashed] node {\beta} (m-4-4)

                (m-4-1) edge node {\gamma} (m-4-4);
        \end{tikzpicture}
    \end{center}
    We want to show that \( [\beta \circ \alpha] = [\Id_C] \).

    By the definition of a pushout, the following sequence is exact
    \begin{center}
        \begin{tikzpicture}
            \diagram{m}{1cm}{1cm} {
                A \& B \oplus I \& C. \\
            };

            \draw[math]
                (m-1-1) edge node {
                    \begin{pmatrix}
                        f \\
                        \kappa
                    \end{pmatrix}
                } (m-1-2)
                (m-1-2) edge node {
                    \begin{pmatrix}
                        g & \gamma
                    \end{pmatrix}
                } (m-1-3);
        \end{tikzpicture}
    \end{center}
    Consider the following commutative diagram, excluding the dashed arrow,
    \begin{diagramlabel}[\label{diag:C-iso}]
        \begin{tikzpicture}
            \diagram{m}{1cm}{1cm} {
                A \& B \oplus I \& C \\
                \& I \& C. \\
            };

            \draw[math]
                (m-1-1) edge node {
                    \begin{psmallmatrix}
                        f \\
                        \kappa
                    \end{psmallmatrix}
                } (m-1-2)
                (m-1-2) edge node {
                    \begin{psmallmatrix}
                        g & \gamma
                    \end{psmallmatrix}
                } (m-1-3)
                (m-1-2) edge node[swap] {
                    \begin{psmallmatrix}
                        0 & i' \circ i - \Id_I
                    \end{psmallmatrix}
                } (m-2-2)
                (m-1-3) edge[dashed] node[swap] {\delta} (m-2-2)
                    edge node {\beta \circ \alpha - \Id_C} (m-2-3)

                (m-2-2) edge node {\gamma} (m-2-3);
        \end{tikzpicture}
    \end{diagramlabel}

    Since
    \[
        \begin{pmatrix}
            0 & i' \circ i - \Id_I
        \end{pmatrix}
        \circ
        \begin{pmatrix}
            f \\
            \kappa
        \end{pmatrix}
        = 0,
    \]
    then by the cokernel property of \( C \) there exists a morphism \( \delta \) such that the top triangle in \autoref{diag:C-iso} commutes. However, since \( 
        \begin{psmallmatrix}
            g & \gamma
        \end{psmallmatrix}
    \) is an epimorphism by definition, it follows that the bottom triangle also commutes, which implies \( [\beta \circ \alpha] = [\Id_C] \).

    We can prove that \( [\alpha \circ \beta] = [\Id_{C'}] \) in a similar way.
\end{proof}

We can now prove that \( \tuple*{\Mc, \Sigma, \Delta} \) is triangulated, where our proof is inspired by \cite[p.\ 16]{Happel_1988} and \cite[Theorem 4.18]{Johan_Bachelor}.

The details of the proof will not be important for the rest of the thesis, but is included mainly because we will be using \( \Mc \) for examples later on.

\begin{example}
    \label{example:stable_module_category_triangulated}
    The tuple \( \tuple*{\Mc, \Sigma, \Delta} \) is a triangulated category.
\end{example}
\begin{proof} % TODO: Kanskje formuler pushout-unikheit eigenskapen på ein kortare måte, kanskje eit lemma?
    We need to prove {\bf (TR1)} -- {\bf (TR4)} from \autoref{def:triangulated_category}.
    \begin{enumerate}[label={(\bfseries TR\arabic*)}]
        \item {
            \begin{enumerate}
                \item {
                    Let \( [f] \in \Mc(A, B) \).
                    
                    Then by the definition of the distinguished triangles in \( \Mc \), the following triangle,
                    \begin{center}
                        \begin{tikzpicture}
                            \diagram{m}{1cm}{1cm} {
                                A \& B \& C_f \& \Sigma A, \\
                            };

                            \draw[math]
                                (m-1-1) edge node {[f]} (m-1-2)
                                (m-1-2) edge node {[g]} (m-1-3)
                                (m-1-3) edge node {[h]} (m-1-4);
                        \end{tikzpicture}
                    \end{center}
                    is distinguished.
                }
                \item {
                    Let \( A \in \Mc \).
                    
                    By definition, \( C_{\Id_A} \) is the pushout
                    \begin{center}
                        \begin{tikzpicture}
                            \diagram{m}{1cm}{1cm} {
                                A \& A \\
                                I_A \& C_{\Id}. \\
                            };

                            \draw[math]
                                (m-1-1) edge node {\Id} (m-1-2)
                                    edge node {\kappa_A} (m-2-1)
                                (m-1-2) edge (m-2-2)

                                (m-2-1) edge node {\gamma_{\Id}} (m-2-2);
                        \end{tikzpicture}
                    \end{center}
                    Since the pushout of an isomorphism is an isomorphism, \( \gamma_{\Id} \) is an isomorphism, which implies \( C_{\Id} \cong 0 \) in \( \Mc \) because all injective modules are projective in \( \Mc \). This yields the trivial triangle.
                }
                \item {
                    \( \Delta \) is closed under isomorphisms of triangles by definition.
                }
            \end{enumerate}
        }
        \item {
            We need to show \( (\Leftarrow) \) and \( (\Rightarrow) \). By \autoref{lem:triangulated_category-TR2-only_one_rotation}, assuming the upcoming proof of {\bf (TR3)} as well as the previous proof of {\bf (TR1)}, then \( (\Leftarrow) \) is implied by \( (\Rightarrow) \).
            
            Therefore, it is sufficient to only prove \( (\Rightarrow) \).

            First note that a left rotated distinguished triangle will be isomorphic to a left rotated standard triangle. Therefore, it suffices to check that every left rotated standard triangle is distinguished.
            
            We will prove this by picking an arbitrary distinguished triangle, and use it to create a new distinguished triangle which is isomorphic to the left rotated standard triangle.

            Consider the following standard triangle
            \begin{center}
                \begin{tikzpicture}
                    \diagram{m}{1cm}{1cm} {
                        A \& B \& C_f \& \Sigma A \\
                    };

                    \draw[math]
                        (m-1-1) edge node {[f]} (m-1-2)
                        (m-1-2) edge node {[g]} (m-1-3)
                        (m-1-3) edge node {[h]} (m-1-4);
                \end{tikzpicture}
            \end{center}

            Recall the following commutative diagrams given by the definition of \( \Sigma [f] \) (\autoref{def:stmod_sigma}) and the construction of the above standard triangle (\autoref{rem:stmod_cone}),
            \[
                \begin{aligned}
                    \begin{tikzpicture}
                        \diagram{m}{1cm}{1cm} {
                            A \& I_A \& \Sigma A \\
                            B \& I_B \& \Sigma B, \\
                        };

                        \draw[math]
                            (m-1-1) edge[tailed] node {\kappa_A} (m-1-2)
                                edge node {f} (m-2-1)
                            (m-1-2) edge[two headed] node {\rho_A} (m-1-3)
                                edge node {i_f} (m-2-2)
                            (m-1-3) edge node {\Sigma f} (m-2-3)

                            (m-2-1) edge[tailed] node {\kappa_B} (m-2-2)
                            (m-2-2) edge[two headed] node {\rho_B} (m-2-3);
                    \end{tikzpicture}
                \end{aligned}
                \hspace{0.5cm}
                \text{and}
                \hspace{0.5cm}
                \begin{aligned}
                    \begin{tikzpicture}
                        \diagram{m}{1cm}{1cm} {
                            A \& B \\
                            I_A \& C_f \\
                            \& \Sigma A. \\
                        };

                        \draw[math]
                            (m-1-1) edge node {f} (m-1-2)
                                edge[tailed] node {\kappa_A} (m-2-1)
                            (m-1-2) edge node{g} (m-2-2)

                            (m-2-1) edge node {\gamma_f} (m-2-2)
                                edge node {\rho_A} (m-3-2)
                            (m-2-2) edge node {h} (m-3-2);
                    \end{tikzpicture}
                \end{aligned}
            \]
            With the above diagrams in mind, consider the following commutative diagram excluding the dashed arrow,
            \begin{center}
                \begin{tikzpicture}
                    \diagram{m}{1cm}{1cm} {
                        A \& B \\
                        I_A \& C_f \& I_B \& \Sigma B, \\
                    };

                    \draw[math]
                        (m-1-1) edge node {f} (m-1-2)
                            edge[tailed] node {\kappa_A} (m-2-1)
                        (m-1-2) edge node{g} (m-2-2)
                            edge[tailed] node {\kappa_B} (m-2-3)

                        (m-2-1) edge node {\gamma_f} (m-2-2)
                            edge[curve={height=1cm}] node {i_f} (m-2-3)
                        (m-2-2) edge[dashed] node {\phi} (m-2-3)

                        (m-2-3) edge node{\rho_B} (m-2-4);
                \end{tikzpicture}
            \end{center}
            where there exists some \( \phi \), given by the pushout universal property.

            By the commutativity of the above diagrams, we have
            \[
                \rho_B \circ \phi \circ g = \rho_B \circ \kappa_B = 0 = \Sigma f \circ h \circ g,
            \]
            and
            \[
                \rho_B \circ \phi \circ \gamma_f = \rho_B \circ i_f = \Sigma f \circ \rho_A = \Sigma f \circ h \circ \gamma_f.
            \]
            This implies that the following pushout diagram
            \begin{center}
                \begin{tikzpicture}
                    \diagram{m}{1cm}{1cm} {
                        A \& B \\
                        I_A \& C_f \& \Sigma B \\
                    };

                    \draw[math]
                        (m-1-1) edge node {f} (m-1-2)
                            edge[tailed] node {\kappa_A} (m-2-1)
                        (m-1-2) edge node{g} (m-2-2)
                            edge[tailed] node {0} (m-2-3)

                        (m-2-1) edge node {\gamma_f} (m-2-2)
                            edge[curve={height=1cm}] node {\rho_B \circ i_f} (m-2-3)
                        (m-2-2) edge[dashed] (m-2-3);
                \end{tikzpicture}
            \end{center}
            has two different morphisms, \( \rho_B \circ \phi \) and \( \Sigma f \circ h \), that could satisfy the pushout universal property. By uniqueness, they have to be the same morphism, i.e.,
            \[
                \rho_B \circ \phi = \Sigma f \circ h.
            \]
            Finally, consider the following commutative diagram
            \begin{center}
                \begin{tikzpicture}
                    \diagram{m}{1cm}{1cm} {
                        0 \& B \& C_f \& \Sigma A \& 0 \\
                        0 \& I_B \& I_B \oplus \Sigma A \& \Sigma A \& 0 \\
                        \& \& \Sigma B \\
                    };

                    \draw[math]
                        (m-1-1) edge (m-1-2)
                            edge[equality] (m-2-1)
                        (m-1-2) edge node {g} (m-1-3)
                            edge node {\kappa_B} (m-2-2)
                        (m-1-3) edge node {h} (m-1-4)
                            edge node {
                                \begin{psmallmatrix}
                                    \phi \\
                                    h
                                \end{psmallmatrix}
                            } (m-2-3)
                        (m-1-4) edge (m-1-5)
                                edge[equality] (m-2-4)

                        (m-2-1) edge (m-2-2)
                        (m-2-2) edge node {
                            \begin{psmallmatrix}
                                1 \\
                                0
                            \end{psmallmatrix}
                        } (m-2-3)
                            edge node {\rho_B} (m-3-3)
                        (m-2-3) edge node {
                            \begin{psmallmatrix}
                                0 & 1
                            \end{psmallmatrix}
                        } (m-2-4)
                            edge node {
                                \begin{psmallmatrix}
                                    \rho_B & -\Sigma f
                                \end{psmallmatrix}
                            } (m-3-3)
                        (m-2-4) edge (m-2-5);
                \end{tikzpicture}
            \end{center}
            where the top row is exact by \autoref{rem:stmod_cone_pushout_properties}, and the second row is split-exact. It follows that the middle square is in fact a pushout, which implies that \( I_B \oplus \Sigma A  \) is uniquely isomorphic to \( C_g \).
            
            Since
            \[
                \begin{psmallmatrix}
                    \rho_B & -\Sigma f
                \end{psmallmatrix}
                \begin{psmallmatrix}
                    \phi \\
                    h
                \end{psmallmatrix}
                =
                \rho_B \circ \phi - \Sigma f \circ h = 0,
            \]
            the triangle
            \begin{center}
                \begin{tikzpicture}
                    \diagram{m}{1cm}{1cm} {
                        B \& C_f \& I_B \oplus \Sigma A \& \Sigma B \\
                    };

                    \draw[math]
                        (m-1-1) edge node {[g]} (m-1-2)
                        (m-1-2) edge node {\class*{
                            \begin{psmallmatrix}
                                \phi \\
                                h
                            \end{psmallmatrix}
                        }} (m-1-3)
                        (m-1-3) edge node {\class*{
                            \begin{psmallmatrix}
                                \rho_B & -\Sigma f
                            \end{psmallmatrix}
                        }} (m-1-4);
                \end{tikzpicture}
            \end{center} 
            is a standard triangle of \( g \), and therefore distinguished.

            Finally, it remains to check if the triangle is isomorphic to the expected triangle. Consider the following diagram in \( \Mc \)
            \begin{center}
                \begin{tikzpicture}
                    \diagram{m}{1cm}{1cm} {
                        B \& C_f \& I_B \oplus \Sigma A \& \Sigma B \\
                        B \& C_f \& \Sigma A \& \Sigma B. \\
                    };

                    \draw[math]
                        (m-1-1) edge node {[g]} (m-1-2)
                            edge[equality] (m-2-1)
                        (m-1-2) edge node {\class*{
                            \begin{psmallmatrix}
                                \phi \\
                                h
                            \end{psmallmatrix}
                        }} (m-1-3)
                            edge[equality] (m-2-2)
                        (m-1-3) edge node {\class*{
                            \begin{psmallmatrix}
                                \rho_B & -\Sigma f
                            \end{psmallmatrix}
                        }} (m-1-4)
                            edge node {\class*{
                                \begin{psmallmatrix}
                                    0 & 1
                                \end{psmallmatrix}
                            }} (m-2-3)
                        (m-1-4) edge[equality] (m-2-4)

                        (m-2-1) edge node {[g]} (m-2-2)
                        (m-2-2) edge node {[h]} (m-2-3)
                        (m-2-3) edge node {[-\Sigma f]} (m-2-4);
                \end{tikzpicture}
            \end{center}
            The left and the middle square commute directly, but it remains to check if the right square commutes. Additionally, we need to check that \( \class*{
                \begin{psmallmatrix}
                    0 & 1
                \end{psmallmatrix}
            } \) is an isomorphism.

            First, we check the commutativity. Consider the difference
            \[
                \begin{psmallmatrix}
                    \rho_B & -\Sigma f
                \end{psmallmatrix}
                -
                (-\Sigma f) \circ
                \begin{psmallmatrix}
                    0 & 1
                \end{psmallmatrix}
                =
                \begin{psmallmatrix}
                    \rho_B & -\Sigma f
                \end{psmallmatrix}
                +
                \begin{psmallmatrix}
                    0 & \Sigma f
                \end{psmallmatrix}
                =
                \begin{psmallmatrix}
                    \rho_B & 0
                \end{psmallmatrix}.
            \]
            We can see that the diagram
            \begin{center}
                \begin{tikzpicture}
                    \diagram{m}{1cm}{1cm} {
                        I_B \oplus \Sigma A \& \& \Sigma B \\
                        \& I_B \\
                    };

                    \draw[math]
                        (m-1-1) edge node {
                            \begin{psmallmatrix}
                                \rho_B & 0
                            \end{psmallmatrix}
                        } (m-1-3)
                            edge node[swap] {
                                \begin{psmallmatrix}
                                    1 & 0
                                \end{psmallmatrix}
                            } (m-2-2)

                        (m-2-2) edge node[swap] {\rho_B} (m-1-3);
                \end{tikzpicture}
            \end{center}
            commutes, and since every injective module is also projective,
            \[
                \class*{
                    \begin{psmallmatrix}
                        \rho_B & -\Sigma f
                    \end{psmallmatrix}
                }
                =
                \class*{
                    -(\Sigma f) \circ
                    \begin{psmallmatrix}
                        0 & 1
                    \end{psmallmatrix}
                }.
            \]

            Second, we check that \( \class*{
                \begin{psmallmatrix}
                    0 & 1
                \end{psmallmatrix}
            } \) is an isomorphism.

            Consider the morphism \( \class*{
                \begin{psmallmatrix}
                    0 \\
                    1
                \end{psmallmatrix}
            } \).
            Note that \( \class*{
                \begin{psmallmatrix}
                    0 & 1
                \end{psmallmatrix}
            } \circ \class*{
                \begin{psmallmatrix}
                    0 \\
                    1
                \end{psmallmatrix}
            } = [\Id_A] \).
            It remains to check if
            \[
                \Id_{I_B \oplus \Sigma A} -
                \begin{psmallmatrix}
                    0 \\
                    1
                \end{psmallmatrix}
                \begin{psmallmatrix}
                    0 & 1
                \end{psmallmatrix}
                =
                \begin{psmallmatrix}
                    1 & 0 \\
                    0 & 0
                \end{psmallmatrix}
            \]
            factors through a projective module.

            This follows from the following commutative diagram,
            \begin{center}
                \begin{tikzpicture}
                    \diagram{m}{1cm}{1cm} {
                        I_B \oplus \Sigma A \& \& I_B \oplus \Sigma A \\
                        \& I_B. \\
                    };

                    \draw[math]
                        (m-1-1) edge node {
                            \begin{psmallmatrix}
                                1 & 0 \\
                                0 & 0
                            \end{psmallmatrix}
                        } (m-1-3)
                            edge node[swap] {
                                \begin{psmallmatrix}
                                    1 & 0
                                \end{psmallmatrix}
                            } (m-2-2)

                        (m-2-2) edge node[swap] {
                            \begin{psmallmatrix}
                                1 \\
                                0
                            \end{psmallmatrix}
                        } (m-1-3);
                \end{tikzpicture}
            \end{center}
        }
        \item {
            By considering two arbitrary distinguished triangles, any morphism between their components will induce a unique morphism between the component of their standard triangles that they are isomorphic to, and vice versa. Therefore, it suffices to only check {\bf (TR3)} for standard triangles.

            In addition, by the argument made in {\bf (TR2)}, we can not assume {\bf (TR2)} in this proof, as that would yield a circular argument.

            We need to show that given the following commutative diagram, excluding the dashed arrow, where the top and bottom row are standard triangles,
            \begin{diagramlabel}[\label{eq:stablemod}]
                \begin{tikzpicture}
                    \diagram{m}{1cm}{1cm} {
                        A \& B  \& C_f \& \Sigma A \\
                        D \& E \& C_l \& \Sigma D, \\
                    };

                    \draw[math]
                        (m-1-1) edge node {[f]} (m-1-2)
                            edge node {[\alpha]} (m-2-1)
                        (m-1-2) edge node {[g]} (m-1-3)
                            edge node {[\beta]} (m-2-2)
                        (m-1-3) edge node {[h]} (m-1-4)
                            edge[dashed] node {[\phi]} (m-2-3)
                        (m-1-4) edge node {\Sigma [\alpha]} (m-2-4)

                        (m-2-1) edge node {[l]} (m-2-2)
                        (m-2-2) edge node {[m]} (m-2-3)
                        (m-2-3) edge node {[n]} (m-2-4);
                \end{tikzpicture}
            \end{diagramlabel}
            that there exists some \( \phi: C_f \to C_l \) such that the entire diagram, including \( \phi \), commutes.

            Consider the following commutative diagrams in \( \Mod(R) \) from the definition of the two standard triangles above,
            \[
                \begin{aligned}
                    \begin{tikzpicture}
                        \diagram{m}{1cm}{1cm} {
                            A \& B \\
                            I_A \& C_f \\
                            \& \Sigma A, \\
                        };
    
                        \draw[math]
                            (m-1-1) edge node {f} (m-1-2)
                                edge[tailed] node {\kappa_A} (m-2-1)
                            (m-1-2) edge node{g} (m-2-2)
    
                            (m-2-1) edge node {\gamma_f} (m-2-2)
                                edge node {\rho_A} (m-3-2)
                            (m-2-2) edge node {h} (m-3-2);
                    \end{tikzpicture}
                \end{aligned}
                \hspace{0.5cm}
                \text{ and }
                \hspace{0.5cm}
                \begin{aligned}
                    \begin{tikzpicture}
                        \diagram{m}{1cm}{1cm} {
                            D \& E \\
                            I_{D} \& C_{l} \\
                            \& \Sigma D. \\
                        };
    
                        \draw[math]
                            (m-1-1) edge node {l} (m-1-2)
                                edge[tailed] node {\kappa_{D}} (m-2-1)
                            (m-1-2) edge node {m} (m-2-2)
    
                            (m-2-1) edge node {\gamma_{l}} (m-2-2)
                                edge node {\rho_{D}} (m-3-2)
                            (m-2-2) edge node {n} (m-3-2);
                    \end{tikzpicture}
                \end{aligned}
            \]
            Since \( [l] \circ [\alpha] = [\beta] \circ [f] \) in \( \Mc \), we have that \( l \circ \alpha - \beta \circ f \) factors through a projective module. However, since projectives are injectives, then by the universal property of injective objects applied to \( \kappa_A \), it follows that \( l \circ \alpha - \beta \circ f \) factors through \( \kappa_A \) and that there is some morphism \( \xi: I_A \to E \) such that
            \[
                l \circ \alpha - \beta \circ f = \xi \circ \kappa_A.
            \]

            Consider the following diagram by definition of \( \Sigma [\alpha] \)
            \begin{center}
                \begin{tikzpicture}
                    \diagram{m}{1cm}{1cm} {
                        A \& I_A \& \Sigma A \\
                        D \& I_{D} \& \Sigma D. \\
                    };

                    \draw[math]
                        (m-1-1) edge[tailed] node {\kappa_A} (m-1-2)
                            edge node {\alpha} (m-2-1)
                        (m-1-2) edge[two headed] node {\rho_A} (m-1-3)
                            edge node {i_{\alpha}} (m-2-2)
                        (m-1-3) edge node {\Sigma \alpha} (m-2-3)

                        (m-2-1) edge[tailed] node {\kappa_{D}} (m-2-2)
                        (m-2-2) edge[two headed] node {\rho_{D}} (m-2-3);
                \end{tikzpicture}
            \end{center}

            Since
            \begin{align*}
                m \circ \beta \circ f &= m \circ l \circ \alpha - m \circ \xi \circ \kappa_A \\
                &= \gamma_{l} \circ \kappa_{D} \circ \alpha - m \circ \xi \circ \kappa_A \\
                &= \gamma_{l} \circ i_{\alpha} \circ \kappa_A - m \circ \xi \circ \kappa_A \\
                &= (\gamma_{l} \circ i_{\alpha} - m \circ \xi) \circ \kappa_A,
            \end{align*}
            then by the pushout property of \( C_f \) it follows that there exists some unique \( \phi \) such that the following diagram commutes
            \begin{center}
                \begin{tikzpicture}
                    \diagram{m}{1cm}{1cm} {
                        A \& B \\
                        I_A \& C_f \\
                        \& \& C_{l}. \\
                    };

                    \draw[math]
                        (m-1-1) edge node {f} (m-1-2)
                            edge node {\kappa_A} (m-2-1)
                        (m-1-2) edge node {g} (m-2-2)
                            edge[curve={height=-25pt}] node {m \circ \beta} (m-3-3)

                        (m-2-1) edge node {\gamma_f} (m-2-2)
                            edge[curve={height=25pt}] node[swap] {\gamma_{l} \circ i_{\alpha} - m \circ \xi} (m-3-3)
                        (m-2-2) edge[dashed] node {\phi} (m-3-3);
                \end{tikzpicture}
            \end{center}
            In particular, \( \phi \circ \gamma_f = \gamma_{l} \circ i_{\alpha} - m \circ \xi \).

            Finally, we check that this \( \phi \) makes \autoref{eq:stablemod} commute.

            By the commutativity of the pushout diagram, the middle square of \autoref{eq:stablemod} commutes. Then it remains to check \( [n \circ \phi] = [(\Sigma \alpha) \circ h] \). Consider the following diagram, excluding the dashed arrow,
            \begin{center}
                \begin{tikzpicture}
                    \diagram{m}{1cm}{1cm} {
                        A \& B \\
                        I_A \& C_f \\
                        \& \& \Sigma D. \\
                    };

                    \draw[math]
                        (m-1-1) edge node {f} (m-1-2)
                            edge node {\kappa_A} (m-2-1)
                        (m-1-2) edge node {g} (m-2-2)
                            edge[curve={height=-25pt}] node {0} (m-3-3)

                        (m-2-1) edge node {\gamma_f} (m-2-2)
                            edge[curve={height=25pt}] node[swap] {(\Sigma \alpha) \circ h \circ \gamma_f} (m-3-3)
                        (m-2-2) edge[dashed] (m-3-3);
                \end{tikzpicture}
            \end{center}
            The above diagram commutes because \( (\Sigma \alpha) \circ h \circ \gamma_f \circ \kappa_A = (\Sigma \alpha) \circ \rho_A \circ \kappa_A = 0 \).

            Consider the following equations
            \[
                n \circ \phi \circ g = n \circ m \circ \beta = 0 = (\Sigma \alpha) \circ h \circ g,
            \]
            and
            \begin{align*}
                (\Sigma \alpha) \circ h \circ \gamma_f &= (\Sigma \alpha) \circ \rho_A \\
                &= \rho_{D} \circ i_{\alpha} \\
                &= n \circ \gamma_{l} \circ i_{\alpha} \\
                &= n \circ (\phi \circ \gamma_f + m \circ \xi) \\
                &= n \circ \phi \circ \gamma_f.
            \end{align*}
                
            These imply that there are two choices of dashed line in the above diagram that would make it commute. However, by uniqueness, this implies they are equal and therefore
            \[
                (\Sigma \alpha) \circ h = n \circ \phi.
            \]
        }
        \item {
            % TODO: Swap m og n?
            By a similar argument as in {\bf (TR3)} it is sufficient to only check {\bf (TR4)} for standard triangles.

            Consider three standard triangles
            \begin{center}
                \begin{tikzpicture}
                    \diagram{m}{0.7cm}{1cm} {
                        A \& B \& C_f \& \Sigma A, \\
                        B \& D \& C_n \& \Sigma B, \\
                    };

                    \draw[math]
                        (m-1-1) edge node {[f]} (m-1-2)
                        (m-1-2) edge node {[g]} (m-1-3)
                        (m-1-3) edge node {[h]} (m-1-4)

                        (m-2-1) edge node {[n]} (m-2-2)
                        (m-2-2) edge node {[m]} (m-2-3)
                        (m-2-3) edge node {[k]} (m-2-4);
                \end{tikzpicture}
            \end{center}
            and
            \begin{center}
                \begin{tikzpicture}
                    \diagram{m}{0.7cm}{1cm} {
                        A \& D \& C_{n \circ f} \& \Sigma A, \\
                    };
                    
                    \draw[math]
                        (m-1-1) edge node {[n \circ f]} (m-1-2)
                        (m-1-2) edge node {[j]} (m-1-3)
                        (m-1-3) edge node {[l]} (m-1-4);
                \end{tikzpicture}
            \end{center}
            which fit into the following commutative diagram excluding the dashed arrows,
            \begin{center}
                \begin{tikzpicture}
                    \diagram{m}{1cm}{1cm} {
                        A \& B \& C_f \& \Sigma A \\
                        A \& D \& C_{n \circ f} \& \Sigma A \\
                        \& C_n \& C_n \& \Sigma B \\
                        \& \Sigma B \& \Sigma C_f. \\
                    };
        
                    \draw[math]
                        (m-1-1) edge node {[f]} (m-1-2)
                            edge[equality] (m-2-1)
                        (m-1-2) edge node {[g]} (m-1-3)
                            edge node {[n]} (m-2-2)
                        (m-1-3) edge node {[h]} (m-1-4)
                            edge[dashed] node {[\phi]} (m-2-3)
                        (m-1-4) edge[equality] (m-2-4)
        
                        (m-2-1) edge node {[n \circ f]} (m-2-2)
                        (m-2-2) edge node {[j]} (m-2-3)
                            edge node {[m]} (m-3-2)
                        (m-2-3) edge node {[l]} (m-2-4)
                            edge[dashed] node {[\psi]} (m-3-3)
                        (m-2-4) edge node {\Sigma [f]} (m-3-4)

                        (m-3-2) edge[equality] (m-3-3)
                            edge node[swap] {[k]} (m-4-2)
                        (m-3-3) edge node {[k]} (m-3-4)
                            edge node {[(\Sigma g) \circ k]} (m-4-3)

                        (m-4-2) edge node {\Sigma [g]} (m-4-3);
                \end{tikzpicture}
            \end{center}
            We need to show that there exist morphisms \( [\phi] \) and \( [\psi] \) that fit into the above commutative diagram, such that
            \begin{diagramlabel}[\label{tri:stmod_tr4}]
                \begin{tikzpicture}
                    \diagram{m}{1cm}{1cm} {
                        C_f \& C_{n \circ f} \& C_n \& \Sigma C_f \\
                    };

                    \draw[math]
                        (m-1-1) edge node {[\phi]} (m-1-2)
                        (m-1-2) edge node {[\psi]} (m-1-3)
                        (m-1-3) edge node {[(\Sigma g) \circ k]} (m-1-4);
                \end{tikzpicture}
            \end{diagramlabel}
            is a distinguished triangle.

            In order to construct \( [\phi] \) and \( [\psi] \), we will work with the following commutative diagram, excluding the dashed arrows, in \( \Mod(R) \)
            \begin{diagramlabel}[\label{diag:stmod_tr4}]
                \begin{tikzpicture}
                    \diagram{m}{1cm}{1cm} {
                        A \& B \& C_f \& \Sigma A \\
                        A \& D \& C_{n \circ f} \& \Sigma A \\
                        \& C_n \& C_n \& \Sigma B \\
                        \& \Sigma B \& \Sigma C_f. \\
                    };
        
                    \draw[math]
                        (m-1-1) edge node {f} (m-1-2)
                            edge[equality] (m-2-1)
                        (m-1-2) edge node {g} (m-1-3)
                            edge node {n} (m-2-2)
                        (m-1-3) edge node {h} (m-1-4)
                            edge[dashed] node {\phi} (m-2-3)
                        (m-1-4) edge[equality] (m-2-4)
        
                        (m-2-1) edge node {n \circ f} (m-2-2)
                        (m-2-2) edge node {j} (m-2-3)
                            edge node {m} (m-3-2)
                        (m-2-3) edge node {l} (m-2-4)
                            edge[dashed] node {\psi} (m-3-3)
                        (m-2-4) edge node {\Sigma f} (m-3-4)

                        (m-3-2) edge[equality] (m-3-3)
                            edge node[swap] {k} (m-4-2)
                        (m-3-3) edge node {k} (m-3-4)
                            edge node {(\Sigma g) \circ k} (m-4-3)

                        (m-4-2) edge node {\Sigma g} (m-4-3);
                \end{tikzpicture}
            \end{diagramlabel}
            If we can find some \( \phi \) and \( \psi \) along with appropriate \( \Sigma f \) such that the above diagram commutes, then all we would have left to prove is that \autoref{tri:stmod_tr4} is a distinguished triangle.

            Consider the following commutative diagrams from the constructions of the three standard triangles mentioned above (\autoref{rem:stmod_cone}),
            \begin{center}
                \begin{tikzpicture}
                    \diagram{m}{1cm}{1cm} {
                        A \& B \& B \& D \& \& A \& D \\
                        I_A \& C_f \& I_B \& C_n \& \text{and} \& I_A \& C_{n \circ f} \\
                        \& \Sigma A, \& \& \Sigma B, \& \& \& \Sigma A, \\
                    };

                    \draw[math]
                        (m-1-1) edge node {f} (m-1-2)
                            edge node {\kappa_A} (m-2-1)
                        (m-1-2) edge node {g} (m-2-2)
                        (m-1-3) edge node {n} (m-1-4)
                            edge node {\kappa_B} (m-2-3)
                        (m-1-4) edge node {m} (m-2-4)
                        (m-1-6) edge node {n \circ f} (m-1-7)
                            edge node {\kappa_A} (m-2-6)
                        (m-1-7) edge node {j} (m-2-7)

                        (m-2-1) edge node {\gamma_f} (m-2-2)
                            edge node {\rho_A} (m-3-2)
                        (m-2-2) edge node {h} (m-3-2)
                        (m-2-3) edge node {\gamma_n} (m-2-4)
                            edge node {\rho_B} (m-3-4)
                        (m-2-4) edge node {k} (m-3-4)
                        (m-2-6) edge node {\gamma_{n \circ f}} (m-2-7)
                            edge node {\rho_A} (m-3-7)
                        (m-2-7) edge node {l} (m-3-7);
                \end{tikzpicture}
            \end{center}
            as well as the following short exact sequence from the definition of \( \Sigma C_f \),
            \begin{center}
                \begin{tikzpicture}
                    \diagram{m}{1cm}{1cm} {
                        C_f \& I_{C_f} \& \Sigma C_f. \\
                    };

                    \draw[math]
                        (m-1-1) edge[tailed] node {\kappa_{C_f}} (m-1-2)
                        (m-1-2) edge[two headed] node {\rho_{C_f}} (m-1-3); 
                \end{tikzpicture}
            \end{center}

            Similar to what was done in the proof of {\bf (TR3)}, let \( \phi \) be the (dashed) pushout morphism arising from the following commutative diagram excluding the dashed arrow
            \begin{center}
                \begin{tikzpicture}
                    \diagram{m}{1cm}{1cm} {
                        A \& B \\
                        I_A \& C_f \\
                        \& \& C_{n \circ f}. \\
                    };

                    \draw[math]
                        (m-1-1) edge node {f} (m-1-2)
                            edge node {\kappa_A} (m-2-1)
                        (m-1-2) edge node {g} (m-2-2)
                            edge[curve={height=-25pt}] node {j \circ n} (m-3-3)
                        
                        (m-2-1) edge node {\gamma_f} (m-2-2)
                            edge[curve={height=25pt}] node[swap] {\gamma_{n \circ f}} (m-3-3)
                        (m-2-2) edge[dashed] node {\phi} (m-3-3);
                \end{tikzpicture}
            \end{center}
            By definition, the square left of \( \phi \) in \autoref{diag:stmod_tr4} commutes. In order to check if the square to the right commutes, consider the following commutative diagram excluding the dashed arrow
            \begin{center}
                \begin{tikzpicture}
                    \diagram{m}{1cm}{1cm} {
                        A \& B \\
                        I_A \& C_f \\
                        \& \& \Sigma A. \\
                    };

                    \draw[math]
                        (m-1-1) edge node {f} (m-1-2)
                            edge node {\kappa_A} (m-2-1)
                        (m-1-2) edge node {g} (m-2-2)
                            edge[curve={height=-25pt}] node {l \circ j \circ n} (m-3-3)
                        
                        (m-2-1) edge node {\gamma_f} (m-2-2)
                            edge[curve={height=25pt}] node[swap] {l \circ \gamma_{n \circ f}} (m-3-3)
                        (m-2-2) edge[dashed] (m-3-3);
                \end{tikzpicture}
            \end{center}
            Consider the following equations
            \[
                l \circ \phi \circ g = l \circ j \circ n = n \circ 0 = 0 = h \circ g,
            \]
            and
            \[
                l \circ \phi \circ \gamma_f = l \circ \gamma_{n \circ f} = \rho_A = h \circ \gamma_f,
            \]
            which, by the pushout morphism uniqueness, implies that
            \[
                l \circ \phi = h.
            \]
            Therefore, the square to the right of \( \phi \) in \autoref{diag:stmod_tr4} also commutes.

            In order to construct \( \psi \) we have to first define two morphisms which will be used in its definition.

            By \autoref{rem:stmod_cone_pushout_properties}, \( g \) is a monomorphism. Since \( \kappa_{C_f} \) is also a monomorphism, it follows that
            \[
                \kappa_{C_f} \circ g: B \to I_{C_f}
            \]
            is also a monomorphism. Then by the injective object property of \( I_B \), there exists some morphism \( i: I_{C_f} \to I_B \) which makes the following diagram commute,
            \begin{center}
                \begin{tikzpicture}
                    \diagram{m}{1cm}{0.5cm} {
                        \& B \\
                        I_{C_f} \& \& I_B. \\
                    };

                    \draw[math]
                        (m-1-2) edge node[swap] {\kappa_{C_f} \circ g} (m-2-1)
                            edge node {\kappa_B} (m-2-3)

                        (m-2-1) edge node {i} (m-2-3);
                \end{tikzpicture}
            \end{center}
            By the injective property of \( I_{C_f} \), there also exists some \( \tilde{i} \) such that the following diagram commutes,
            \begin{center}
                \begin{tikzpicture}
                    \diagram{m}{1cm}{0.5cm} {
                        \& B \\
                        I_{C_f} \& \& I_B. \\
                    };

                    \draw[math]
                        (m-1-2) edge node[swap] {\kappa_{C_f} \circ g} (m-2-1)
                            edge node {\kappa_B} (m-2-3)

                        (m-2-3) edge node {\tilde{i}} (m-2-1);
                \end{tikzpicture}
            \end{center}
            
            Now we can move on to defining \( \psi \).

            Consider the following diagram excluding the dashed arrow,
            \begin{center}
                \begin{tikzpicture}
                    \diagram{m}{1cm}{1cm} {
                        A \& D \\
                        I_A \& C_{n \circ f} \\
                        \& \& C_n, \\
                    };

                    \draw[math]
                        (m-1-1) edge node {n \circ f} (m-1-2)
                            edge node {\kappa_A} (m-2-1)
                        (m-1-2) edge node {j} (m-2-2)
                            edge[curve={height=-25pt}] node {m} (m-3-3)

                        (m-2-1) edge node {\gamma_{n \circ f}} (m-2-2)
                            edge[curve={height=25pt}] node[swap] {\gamma_n \circ i \circ \kappa_{C_f} \circ \gamma_f} (m-3-3)
                        (m-2-2) edge[dashed] node {\psi} (m-3-3);
                \end{tikzpicture}
            \end{center}
            which commutes because
            \begin{equation}
                \label{diag:stmod_tr4_m_n_f_equal_something_else}
                \begin{aligned}
                    m \circ n \circ f &= \gamma_n \circ \kappa_B \circ f \\
                    &= \gamma_n \circ i \circ \kappa_{C_f} \circ g \circ f \\
                    &= \gamma_n \circ i \circ \kappa_{C_f} \circ \gamma_f \circ \kappa_A.
                \end{aligned}
            \end{equation}
            Thus, by the pushout property, there exists a morphism \( \psi: C_{n \circ f} \to C_n \) such that the above diagram including the dashed arrow commutes.

            To show that \( \psi \) makes \autoref{diag:stmod_tr4} commute in \( \Mc \), we can see that the square to the left of \( \psi \) commutes by definition, and so it remains to choose some \( \Sigma f \) such that the square to the right also commutes.

            Consider the following diagram excluding the dashed arrow,
            \begin{center}
                \begin{tikzpicture}
                    \diagram{m}{1cm}{1cm} {
                        A \& I_A \& \Sigma A \\
                        \& C_f \\
                        \& I_{C_f} \\
                        B \& I_B \& \Sigma B, \\
                    };

                    \draw[math]
                        (m-1-1) edge node {\kappa_A} (m-1-2)
                            edge node[swap] {f} (m-4-1)
                        (m-1-2) edge node {\rho_A} (m-1-3)
                            edge node {\gamma_f} (m-2-2)
                        (m-1-3) edge[dashed] node {\Sigma f} (m-4-3)

                        (m-2-2) edge node {\kappa_{C_f}} (m-3-2)

                        (m-3-2) edge node {i} (m-4-2)

                        (m-4-1) edge node {\kappa_B} (m-4-2)
                            edge node {g} (m-2-2)
                        (m-4-2) edge node {\rho_B} (m-4-3);
                \end{tikzpicture}
            \end{center}
            which commutes because the top left ``trapezoid'' and the bottom left ``triangle'' in the diagram commute. Then since
            \[
                \rho_B \circ i \circ \kappa_{C_f} \circ \gamma_f \circ \kappa_A = \rho_B \circ \kappa_B = 0,
            \]
            it follows by the cokernel property of \( \Sigma A \) that there exists some morphism, \( \Sigma f \), which is denoted as such because it fits with \autoref{def:stmod_sigma}.

            Then consider the following commutative diagram excluding the dashed arrow,
            \begin{center}
                \begin{tikzpicture}
                    \diagram{m}{1cm}{1cm} {
                        A \& D \\
                        I_A \& C_{n \circ f} \\
                        \& \& \Sigma B, \\
                    };

                    \draw[math]
                        (m-1-1) edge node {n \circ f} (m-1-2)
                            edge node {\kappa_A} (m-2-1)
                        (m-1-2) edge node {j} (m-2-2)
                            edge[curve={height=-25pt}] node {0} (m-3-3)

                        (m-2-1) edge node {\gamma_{n \circ f}} (m-2-2)
                            edge[curve={height=25pt}] node[swap] {k \circ \gamma_n \circ i \circ \kappa_{C_f} \circ \gamma_f} (m-3-3)
                        (m-2-2) edge[dashed] (m-3-3);
                \end{tikzpicture}
            \end{center}
            where
            \[
                (\Sigma f) \circ l \circ j = (\Sigma f) \circ 0 = 0 = k \circ m = k \circ \psi \circ j
            \]
            and
            \begin{align*}
                k \circ \psi \circ \gamma_{n \circ f} &= k \circ \gamma_n \circ i \circ \kappa_{C_f} \circ \gamma_f \\
                &= \rho_B \circ i \circ \kappa_{C_f} \circ \gamma_f \\
                &= (\Sigma f) \circ \rho_A \\
                &= (\Sigma f) \circ l \circ \gamma_{n \circ f}.
            \end{align*}
            Thus, by uniqueness of the pushout property it follows that \( k \circ \psi = (\Sigma f) \circ l \), and the square to the right of \( \psi \) in \autoref{diag:stmod_tr4} commutes.

            Finally, it remains to check that
            \begin{center}
                \begin{tikzpicture}
                    \diagram{m}{1cm}{1.5cm} {
                        C_f \& C_{n \circ f} \& C_n \& \Sigma C_f \\
                    };

                    \draw[math]
                        (m-1-1) edge node {[\phi]} (m-1-2)
                        (m-1-2) edge node {[\psi]} (m-1-3)
                        (m-1-3) edge node {(\Sigma[g]) \circ [k]} (m-1-4);
                \end{tikzpicture}
            \end{center}
            is a distinguished triangle.

            We will construct a standard triangle of \( [\phi] \) and then show that this triangle is isomorphic to the above triangle.

            Consider the following commutative diagram
            \begin{center}
                \begin{tikzpicture}
                    \diagram{m}{1cm}{1cm} {
                        A \& B \& D \\
                        I_A \& C_f \& C_{n \circ f}. \\
                    };

                    \draw[math]
                        (m-1-1) edge node {f} (m-1-2)
                            edge node {\kappa_A} (m-2-1)
                        (m-1-2) edge node {n} (m-1-3)
                            edge node {g} (m-2-2)
                        (m-1-3) edge node {j} (m-2-3)

                        (m-2-1) edge node {\gamma_f} (m-2-2)
                            edge[curve={height=25pt}] node[swap] {\gamma_{n \circ f}} (m-2-3)
                        (m-2-2) edge node {\phi} (m-2-3);
                \end{tikzpicture}
            \end{center}
            Since the left and the outer square are pushouts this implies that the right square is also a pushout.

            Consider the following pushout diagram of \( (B, n, \kappa_{C_f} \circ g) \),
            \begin{diagramlabel}[\label{diag:stmod_tr4_C}]
                \begin{tikzpicture}
                    \diagram{m}{1cm}{1cm} {
                        B \& D \\
                        I_{C_f} \& C. \\
                    };

                    \draw[math]
                        (m-1-1) edge node {n} (m-1-2)
                            edge node[swap] {\kappa_{C_f} \circ g} (m-2-1)
                        (m-1-2) edge node {\delta} (m-2-2)

                        (m-2-1) edge node {\gamma} (m-2-2);
                \end{tikzpicture}
            \end{diagramlabel}

            Using the new pushout, consider the following commutative diagram excluding the dashed arrow,
            \begin{center}
                \begin{tikzpicture}
                    \diagram{m}{1cm}{1cm} {
                        B \& D \\
                        C_f \& C_{n \circ f} \\
                        \& \& C. \\
                    };

                    \draw[math]
                        (m-1-1) edge node {n} (m-1-2)
                            edge node[swap] {g} (m-2-1)
                        (m-1-2) edge node {j} (m-2-2)
                            edge[curve={height=-25pt}] node {\delta} (m-3-3)

                        (m-2-1) edge node {\phi} (m-2-2)
                            edge[curve={height=25pt}] node[swap] {\gamma \circ \kappa_{C_f}} (m-3-3)
                        (m-2-2) edge[dashed] node {x} (m-3-3);
                \end{tikzpicture}
            \end{center}
            By the pushout property, there exists a morphism \( x: C_{n \circ f} \to C \) such that the above diagram including \( x \) commutes.
            
            Consider the following commutative diagram,
            \begin{diagramlabel}[\label{diag:stmod_tr4_show_psi_phi_equal_something_else}]
                \begin{tikzpicture}
                    \diagram{m}{1cm}{1cm} {
                        B \& D \\
                        C_f \& C_{n \circ f} \\
                        I_{C_f} \& C. \\
                    };
                    
                    \draw[math]
                        (m-1-1) edge node {n} (m-1-2)
                            edge node[swap] {g} (m-2-1)
                        (m-1-2) edge node[swap] {j} (m-2-2)
                            edge[curve={height=-25pt}] node {\delta} (m-3-2)

                        (m-2-1) edge node {\phi} (m-2-2)
                            edge node[swap] {\kappa_{C_f}} (m-3-1)
                        (m-2-2) edge node[swap] {x} (m-3-2)

                        (m-3-1) edge node {\gamma} (m-3-2);
                \end{tikzpicture}
            \end{diagramlabel}

            Since the outer rectangle and upper square of \autoref{diag:stmod_tr4_show_psi_phi_equal_something_else} are pushouts, then the lower square is also a pushout.

            This implies that there exists a morphism \( p \) by the pushout property, such that the following diagram
            \begin{center}
                \begin{tikzpicture}
                    \diagram{m}{1cm}{1cm} {
                        C_f \& C_{n \circ f} \\
                        I_{C_f} \& C \\
                        \& \Sigma C_f, \\
                    };

                    \draw[math]
                        (m-1-1) edge node {\phi} (m-1-2)
                            edge node[swap] {\kappa_{C_f}} (m-2-1)
                        (m-1-2) edge node[swap] {x} (m-2-2)
                            edge[curve={height=-25pt}] node {0} (m-3-2)

                        (m-2-1) edge node {\gamma} (m-2-2)
                            edge node[swap] {\rho_{C_f}} (m-3-2)
                        (m-2-2) edge node[swap] {p} (m-3-2);
                \end{tikzpicture}
            \end{center}
            commutes and yields a standard triangle
            \begin{center}
                \begin{tikzpicture}
                    \diagram{m}{1cm}{1cm} {
                        C_f \& C_{n \circ f} \& C \& \Sigma C_f. \\
                    };

                    \draw[math]
                        (m-1-1) edge node {[\phi]} (m-1-2)
                        (m-1-2) edge node {[x]} (m-1-3)
                        (m-1-3) edge node {[p]} (m-1-4);
                \end{tikzpicture}
            \end{center}
            
            Consider the two pushout diagrams:
            \[
                \begin{aligned}
                    \begin{tikzpicture}
                        \diagram{m}{1cm}{1cm} {
                            B \& D \\
                            I_B \& C_n, \\
                        };
    
                        \draw[math]
                            (m-1-1) edge node {n} (m-1-2)
                                edge node[swap] {\kappa_B} (m-2-1)
                            (m-1-2) edge node {m} (m-2-2)
    
                            (m-2-1) edge node {\gamma_n} (m-2-2);
                    \end{tikzpicture}
                \end{aligned}
                \hspace{0.5cm}
                \text{ and }
                \hspace{0.5cm}
                \begin{aligned}
                    \begin{tikzpicture}
                        \diagram{m}{1cm}{1cm} {
                            B \& D \\
                            I_{C_f} \& C. \\
                        };
    
                        \draw[math]
                            (m-1-1) edge node {n} (m-1-2)
                                edge node[swap] {\kappa_{C_f} \circ g} (m-2-1)
                            (m-1-2) edge node {\delta} (m-2-2)
    
                            (m-2-1) edge node {\gamma} (m-2-2);
                    \end{tikzpicture}
                \end{aligned}  
            \]
            By \autoref{lem:stmod_pushout_different_injectives_isomorphic} it follows that there exists some \( \alpha: C \to C_n \) such that \( [\alpha] \) is an isomorphism in \( \Mc \), and
            \begin{itemize}
                \item \( \alpha \circ \delta = m \), and
                \item \( \alpha \circ \gamma = \gamma_n \circ i \).
            \end{itemize}

            The final step is then to show that the following diagram is an isomorphism of triangles
            \begin{diagramlabel}[\label{diag:stmod_tr4_iso}]
                \begin{tikzpicture}
                    \diagram{m}{1cm}{1cm} {
                        C_f \& C_{n \circ f} \& C \& \Sigma C_f \\
                        C_f \& C_{n \circ f} \& C_n \& \Sigma C_f. \\
                    };

                    \draw[math]
                        (m-1-1) edge node {[\phi]} (m-1-2)
                            edge[equality] (m-2-1)
                        (m-1-2) edge node {[x]} (m-1-3)
                            edge[equality] (m-2-2)
                        (m-1-3) edge node {[p]} (m-1-4)
                            edge node[swap] {[\alpha]} (m-2-3)
                        (m-1-4) edge[equality] (m-2-4)

                        (m-2-1) edge node {[\phi]} (m-2-2)
                        (m-2-2) edge node {[\psi]} (m-2-3)
                        (m-2-3) edge node {[(\Sigma g) \circ k]} (m-2-4);
                \end{tikzpicture}
            \end{diagramlabel}

            The leftmost square clearly commutes.

            In order to prove that the middle square commutes, consider the following commutative diagram excluding the dashed arrow,
            \begin{center}
                \begin{tikzpicture}
                    \diagram{m}{1cm}{1cm} {
                        A \& B \\
                        I_A \& C_f \\
                        \& \& C_n, \\
                    };

                    \draw[math]
                        (m-1-1) edge node {f} (m-1-2)
                            edge node[swap] {\kappa_A} (m-2-1)
                        (m-1-2) edge node {g} (m-2-2)
                            edge[curve={height=-25pt}] node {m \circ n} (m-3-3)

                        (m-2-1) edge node {\gamma_f} (m-2-2)
                            edge[curve={height=25pt}] node[swap] {\gamma_n \circ i \circ \kappa_{C_f} \circ \gamma_f} (m-3-3)
                        (m-2-2) edge[dashed] (m-3-3);
                \end{tikzpicture}
            \end{center}
            where the inner square is a pushout and the outer ``square'' commutes by \eqref{diag:stmod_tr4_m_n_f_equal_something_else}.

            Note,
            \begin{align*}
                \psi \circ \phi \circ g &= \psi \circ j \circ n \\
                &= m \circ n \\
                &= \gamma_n \circ \kappa_B \\
                &= \gamma_n \circ i \circ \kappa_{C_f} \circ g,
            \end{align*}
            and that
            \begin{align*}
                \psi \circ \phi \circ \gamma_f &= \psi \circ \gamma_{n \circ f} \\
                &= \gamma_n \circ i \circ \kappa_{C_f} \circ \gamma_f,
            \end{align*}
            which implies by the uniqueness of the pushout morphism that
            \[
                \psi \circ \phi = \gamma_n \circ i \circ \kappa_{C_f}.
            \]

            With the above equality in mind, consider the following commutative diagram excluding the dashed arrow,
            \begin{center}
                \begin{tikzpicture}
                    \diagram{m}{1cm}{1cm} {
                        B \& D \\
                        C_f \& C_{n \circ f} \\
                        \& \& C_n. \\
                    };

                    \draw[math]
                        (m-1-1) edge node {n} (m-1-2)
                            edge node[swap] {g} (m-2-1)
                        (m-1-2) edge node {j} (m-2-2) 
                            edge[curve={height=-25pt}] node {m} (m-3-3)

                        (m-2-1) edge node {\phi} (m-2-2)
                            edge[curve={height=25pt}] node[swap] {\alpha \circ \gamma \circ \kappa_{C_f}} (m-3-3)
                        (m-2-2) edge[dashed] (m-3-3);
                \end{tikzpicture}
            \end{center}
            Consider the following two equations,
            \[
                \alpha \circ x \circ j = \alpha \circ \delta = m = \psi \circ j,
            \]
            and
            \[
                \alpha \circ x \circ \phi = \alpha \circ \gamma \circ \kappa_{C_f} = \gamma_n \circ i \circ \kappa_{C_f} = \psi \circ \phi,
            \]
            it follows that by the uniqueness of the pushout that \( \alpha \circ x = \psi \), and so the middle square of \autoref{diag:stmod_tr4_iso} commutes.
            
            It remains to show that the rightmost square commutes, which requires some additional tools.

            Let \( E := \coker(\delta)\). Since \autoref{diag:stmod_tr4_C} is a pushout, we have \( E = \coker(\kappa_{C_f} \circ g) \). Let \( c: I_{C_f} \to E \) be the cokernel morphism of \( \kappa_{C_f} \circ g \).

            Let \( e \) and \( e^{-1} \) be the (dashed) morphisms induced by the cokernel properties which fit in the following commutative diagram,
            \begin{center}
                \begin{tikzpicture}
                    \diagram{m}{1cm}{1cm} {
                        B \& I_B \& \Sigma B \\
                        B \& I_{C_f} \& E \\
                        B \& I_B \& \Sigma B. \\
                    };

                    \draw[math]
                        (m-1-1) edge node {\kappa_B} (m-1-2)
                            edge[equality] (m-2-1)
                        (m-1-2) edge node {\rho_B} (m-1-3)
                            edge node {\tilde{i}} (m-2-2)
                        (m-1-3) edge[dashed] node {e^{-1}} (m-2-3)

                        (m-2-1) edge node {\kappa_{C_f} \circ g} (m-2-2)
                            edge[equality] (m-3-1)
                        (m-2-2) edge node {c} (m-2-3)
                            edge node {i} (m-3-2)
                        (m-2-3) edge[dashed] node {e} (m-3-3)

                        (m-3-1) edge node {\kappa_B} (m-3-2)
                        (m-3-2) edge node {\rho_B} (m-3-3);
                \end{tikzpicture}
            \end{center}
            
            % We can verify that \( [e \circ e^{-1}] = \Sigma [\Id_B] = [\Id_{\Sigma B}] \) by \autoref{lem:stmod_sigma_well-defined_additive_endofunctor}. 
            We have \( [e^{-1} \circ e] = [\Id_E] \) by the following argument:
            
            Consider the following commutative diagram excluding the dashed arrow,
            \begin{center}
                \begin{tikzpicture}
                    \diagram{m}{1cm}{2cm} {
                        B \& I_{C_f} \& E \\
                        B \& I_{C_f} \& E. \\
                    };

                    \draw[math]
                        (m-1-1) edge node {\kappa_{C_f} \circ g} (m-1-2)
                            edge node[swap] {\Id_B - \Id_B = 0} (m-2-1)
                        (m-1-2) edge node {\rho_B} (m-1-3)
                            edge node[swap] {\tilde{i} \circ i - \Id_{I_{C_f}}} (m-2-2)
                        (m-1-3) edge[dashed] (m-2-2)
                            edge node {e^{-1} \circ e - \Id_E} (m-2-3)

                        (m-2-1) edge node {\kappa_{C_f} \circ g} (m-2-2)
                        (m-2-2) edge node {\rho_B} (m-2-3);
                \end{tikzpicture}
            \end{center}
            By the cokernel property of \( E \) there exists some morphism such that the top right triangle including the dashed arrow commutes. Furthermore, since \( \rho_B \) is an epimorphism, the bottom right triangle commutes, which implies \( [e^{-1} \circ e] = [\Id_E] \), since \( I_{C_f} \) is a projective module.

            The above morphisms fit into the following commutative diagram, where \( d \) is the cokernel morphism of \( \delta \),
            \begin{center}
                \begin{tikzpicture}
                    \diagram{m}{1cm}{1cm} {
                        B \& D \\
                        I_{C_f} \& C \\
                        E \& E \\
                        \& \Sigma B. \\
                    };

                    \draw[math]
                        (m-1-1) edge node {n} (m-1-2)
                            edge node {\kappa_{C_f} \circ g} (m-2-1)
                        (m-1-2) edge node {\delta} (m-2-2)
                        
                        (m-2-1) edge node {\gamma} (m-2-2)
                            edge node {c} (m-3-1)
                        (m-2-2) edge node {d} (m-3-2)

                        (m-3-1) edge[equality] (m-3-2)
                        (m-3-2) edge node {e} (m-4-2);
                \end{tikzpicture}
            \end{center}
            
            With the above morphisms in mind, let \( \mu \) be the (dashed) morphism induced by the cokernel property in the following commutative diagram, 
            \begin{center}
                \begin{tikzpicture}
                    \diagram{m}{1cm}{1cm} {
                        B \& I_B \& \Sigma B \\
                        B \& I_{C_f} \& E \\
                        C_f \& I_{C_f} \& \Sigma C_f, \\
                    };

                    \draw[math]
                        (m-1-1) edge node {\kappa_B} (m-1-2)
                            edge[equality] (m-2-1)
                        (m-1-2) edge node {\rho_B} (m-1-3)
                            edge node {\tilde{i}} (m-2-2)
                        (m-1-3) edge node[swap] {e^{-1}} (m-2-3)
                            edge[dashed, curve={height=-25pt}] node {\Sigma g} (m-3-3)

                        (m-2-1) edge node {\kappa_{C_f} \circ g} (m-2-2)
                            edge node {g} (m-3-1)
                        (m-2-2) edge node {c} (m-2-3)
                            edge[equality] (m-3-2)
                        (m-2-3) edge[dashed] node[swap] {\mu} (m-3-3)

                        (m-3-1) edge node {\kappa_{C_f}} (m-3-2)
                        (m-3-2) edge node {\rho_{C_f}} (m-3-3);
                \end{tikzpicture}
            \end{center}
            and let \( \Sigma g := \mu \circ e^{-1} \).

            Before proving \( [(\Sigma g) \circ k \circ \alpha] = [p] \), we first have to prove some relations between the new morphisms and the old ones.

            Since
            \[
                \rho_B \circ i \circ \kappa_{C_f} \circ g = \rho_B \circ \kappa_B = 0,
            \]
            we have the following commutative diagram excluding the dashed arrow,
            \begin{center}
                \begin{tikzpicture}
                    \diagram{m}{1cm}{1cm} {
                        B \& D \\
                        I_{C_f} \& C \\
                        \& \& \Sigma B. \\
                    };

                    \draw[math]
                        (m-1-1) edge node {n} (m-1-2)
                            edge node[swap] {\kappa_{C_f} \circ g} (m-2-1)
                        (m-1-2) edge node {\delta} (m-2-2)
                            edge[curve={height=-25pt}] node {0} (m-3-3)

                        (m-2-1) edge node {\gamma} (m-2-2)
                            edge[curve={height=25pt}] node {\rho_B \circ i} (m-3-3)
                        (m-2-2) edge[dashed] (m-3-3);
                \end{tikzpicture}
            \end{center}

            Both \( e \circ d \) and \( k \circ \alpha \) could fit as the dashed arrow, as
            \[
                k \circ \alpha \circ \delta = k \circ m = 0 = 0 \circ \delta = e \circ d \circ \delta,
            \]
            and
            \[
                k \circ \alpha \circ \gamma = k \circ \gamma_n \circ i = \rho_B \circ i = e \circ c = e \circ d \circ \gamma.
            \]
            Therefore, by uniqueness of the universal property of the pushout, \( e \circ d = k \circ \alpha \).

            Similarly, consider the following commutative diagram excluding the dashed arrow,
            \begin{center}
                \begin{tikzpicture}
                    \diagram{m}{1cm}{1cm} {
                        B \& D \\
                        I_{C_f} \& C \\
                        \& \& \Sigma C_f. \\
                    };

                    \draw[math]
                        (m-1-1) edge node {n} (m-1-2)
                            edge node[swap] {\kappa_{C_f} \circ g} (m-2-1)
                        (m-1-2) edge node {\delta} (m-2-2)
                            edge[curve={height=-25pt}] node {0} (m-3-3)

                        (m-2-1) edge node {\gamma} (m-2-2)
                            edge[curve={height=25pt}] node {\rho_{C_f}} (m-3-3)
                        (m-2-2) edge[dashed] (m-3-3);
                \end{tikzpicture}
            \end{center}

            We can verify that \( p \) already fits as a dashed arrow, since \( p \circ \delta = p \circ x \circ j = 0 \circ j = 0 \). However, \( \mu \circ d \) also fits, as
            \[
                \mu \circ d \circ \delta = \mu \circ 0 = 0,
            \]
            and
            \[
               \mu \circ d \circ \gamma = \mu \circ c = \rho_{C_f}.
            \]
            Therefore, by the uniqueness of the universal property of the pushout, \( p = \mu \circ d \).

            Combining everything, we have
            \[
                [(\Sigma g) \circ k \circ \alpha] = [(\Sigma g) \circ e \circ d] =[\mu \circ e^{-1} \circ e \circ d] = [\mu \circ d] = [p],
            \]
            which finishes the proof, as \( [\alpha] \) is an isomorphism. \qedhere
        }
    \end{enumerate}
\end{proof}

Thus, we have proven that the stable (infinitely generated) module category is triangulated. A natural follow-up question is therefore if there exists a stable finitely generated module category.
\begin{remark}
    This subsection has only mentioned (infinitely generated) modules. However, the same definitions, functors, and proofs work for the finitely generated modules, \( \mod(R) \). This is because a Frobenius ring is Noetherian, and \( \mod(R) \) is therefore an abelian category. In addition, \( \mod(R) \) also has enough projectives and injectives, and therefore no proof or definition uses any properties of \( \Mod(R) \), that \( \mod(R) \) does not have. This yields a triangulated category, \( \Stmod(R) \) which is the stable module category over \( \mod(R) \).
\end{remark}

By \cite[Lemma, Section 7.5]{Krause_2007} it follows that \( \StMod(R) \) and \( \Stmod(R) \) are in fact what we call an ``algebraic triangulated category.'' Note that his definition might not be equivalent to our definition of an algebraic triangulated category in general, which we will define in \autoref{section:alg_tri_cats}.