In this subsection the goal is to define what an algebraic triangulated category is by first defining what \( H^0(\dgM) \) is, and then defining the DG-Yoneda embedding, and how any DG-category can be seen as a subcategory of the category of DG-modules.

The following definition is very similar to \autoref{def:H_bullet_dg_category}. However unlike the cohomology category, this category is not enriched over \( C \).

\begin{definition}[0th cohomology category of a DG-category]
    \label{def:0_th_cohomology_of_dg_cat}
    Let \( \Cc \) be a DG-category over \( R \).

    Then let \( H^0(\Cc) \) be the following category defined as follows
    \begin{enumerate}
        \item {
            Let \( \Obj(H^0(\Cc)) := \Obj(\Cc) \).
        }
        \item {
            Let \( H^0(\Cc)(A, B) := H^0(\Cc(A, B)) \).
        }
        \item {
            Let \( A, B, C \in H^0(\Cc) \) with \( [f] \in H^0(A, B) \) and \( [g] \in H^0(B, C) \).

            Then let composition be as follows
            \begin{align*}
                \circ_{H^0(\Cc)}: H^0(\Cc)(B, C) \times H^0(\Cc)(A, B) &\to H^0(\Cc)(A, C) \\
                [g] \times [f] &\mapsto \class*{ \circ_{\Cc}(g \otimes f) }
            \end{align*}
        }
    \end{enumerate}
\end{definition}

An interesting consequence of \autoref{def:0_th_cohomology_of_dg_cat}, is that we can write the chain homotopy category, \( \K(\Mod(R)) \), defined in \autoref{def:chain_homotopy_cat}, as simply \( H^0(\C_{\dg}) \). Unwinding the definitions, \( H^0([A, B]) \) is exactly every chain homotopy class from \( A \) to \( B \).

The goal is to show that \( H^0\tuple*{\dgM} \) is triangulated. To do that we need to define a shift functor and a class of distinguished triangles. The following is a definition for a shift functor on \( \C_{\dg} \), which will later induce a shift functor on \( \dgM \).

\begin{definition}[Shift in \( \C_{\dg} \)]
    \label{def:sigma_c_dg}
    Let the DG-functor \( \Sigma_{\C_{\dg}} \) be defined as follows:
    \begin{enumerate}
        \item {
            For \( A \in \C_{\dg} \), and \( \Sigma \) the ordinary shift functor on \( \C \), let
            \[
                \Sigma_{\C_{\dg}}(A) := \Sigma A
            \]
        }
        \item {
            Let \( A, B \in \C_{\dg} \) and \( n \in \Zb \).

            Then define \( \Sigma_{\C_{\dg}, n} \) as follows
            \begin{align*}
                \Sigma_{\C_{\dg}, n}: \C_{\dg}(A, B)_n &\to \C_{\dg}(\Sigma A, \Sigma B)_n \\
                f &\mapsto (-1)^n f.
            \end{align*}
        }
    \end{enumerate}
    Then \( \Sigma_{\C_{\dg}} \) is called the \emph{shift functor of \( \C_{\dg} \)}. When the context is uncertain, we will denote it as \( \Sigma_{\C_{\dg}} \), and when the context is certain we will use \( \Sigma \).
\end{definition}

The reason \( \Sigma_{\C_{\dg}} \) is a DG-functor is given by the following remark. 

\begin{remark}
    In order to show that \( \Sigma \) from \autoref{def:sigma_c_dg} is a DG-functor, need to show the following three properties:
    \begin{enumerate}
        \item {
            Firstly, need to show that
            \[
                \Sigma_{A, B}: \C_{\dg}(A, B) \to \C_{\dg}(\Sigma A, \Sigma B)
            \]
            is a chain morphism by checking if it commutes with the differential.

            WIP
        }
        \item {
            Secondly, need to show the ``functoriality'' of \( \Sigma \).

            WIP
        }
        \item {
            Thirdly, need to show that the unit morphism commutes with \( \Sigma \).

            WIP
        }
    \end{enumerate}
\end{remark}

We can extend \( \Sigma_{\C_{\dg}} \) onto \( \dgM \) in the following way.

\begin{definition}
    \label{def:sigma_dgmod}
    Let the DG-functor \( \Sigma_{\dgM} \) be defined as follows:
    \begin{enumerate}
        \item {
            For any \( F \in \dgM \), let
            \[
                \Sigma_{\dgM} F := \Sigma_{\C_{\dg}} \circ F.
            \]
        }
        \item {
            Let \( F, G \in \dgM \), and \( n \in \Zb \).
            
            Then define \( \Sigma_{\dgM, n} \) as follows
            \begin{align*}
                \Sigma_{\dgM, n}: \dgM(F, G)_n &\to \dgM(\Sigma F, \Sigma G)_n \\
                \eta &\mapsto (-1)^n \eta.
            \end{align*}
        }
    \end{enumerate}
    Then \( \Sigma_{\dgM} \) is called the \emph{shift functor of \( \dgM \)}. When the context is uncertain, we will denote it as \( \Sigma_{\dgM} \), and when the context is certain we will use \( \Sigma \).
\end{definition}

\begin{remark}
    Due to our notation of product not having a order, as talked about in \autoref{not:coprod_prod_forvirring}, it follows that \( d_{\Sigma F} = -d_F = \Sigma d_F\) for any DG-module \( F \).

    This is because
    \[
        d_{\Sigma F} = \tuple{d_{\Sigma F A, i}}_{A, i} = \tuple{-d_{F A, i + 1}}_{A, i} = \tuple{-d_{F A, i}}_{A, i} = -d_F = \Sigma d_F.
    \]
\end{remark}

By the following remark, \( \Sigma_{\dgM} \) is a DG-functor.

\begin{remark}
    TODO
\end{remark}

In order to get the shift functor on \( H^0(\dgM) \), we have to take the \( 0 \)th cohomology of this functor. It is defined as follows.

\begin{definition}[\( H^0 \)-induced functor]
    \label{def:H^0-induced_functor}
    Let \( \Ac \) and \( \Bc \) be two DG-categories, and let \( F: \Ac \to \Bc \) be a DG-functor between them.

    Then define the functor \( H^0(F) \) as follows.
    \begin{align*}
        H^0(F): H^0(\Ac) &\to H^0(\Bc) \\
        A &\mapsto F(A)
    \end{align*}
    where for any \( f \in H^0(\Ac)(A, B) \)
    \begin{align*}
        H^0(F)_{A, B}: H^0(\Ac)(A, B) &\to H^0(\Bc)(F(A), F(B)) \\
        [f] &\mapsto [F(f)]
    \end{align*}

    This is called the \emph{\( H^0 \)-induced functor of \( F \)}.
\end{definition}

Which leads us to the definition of the shift functor on \( H^0(\dgM) \).

\begin{definition}[\( \Sigma \) on \( H^0(\dgM) \)]
    \label{def:sigma_h_0_dgmod}
    Let \( \Cc \) be a small DG-category, and let \( \Sigma \) be as defined in \autoref{def:sigma_dgmod}.

    Then the functor
    \[
        \Sigma_{H^0(\dgM)} := H^0(\Sigma): H^0(\dgM) \to H^0(\dgM)
    \]
    is called the \emph{shift functor of \( H^0(\dgM) \)}. When the context is uncertain, we will denote it as \( \Sigma_{H^0(\dgM)} \), and when the context is certain we will use \( \Sigma \).
\end{definition}

The overloading of the notation \( \Sigma \) makes the notation much more concice, and in certain situations and makes equations much more clean while being correct.

% TODO: Utdjup, bevis?
A helpful consequence of the definition of all the \( \Sigma \)s is that they are all automorphisms.
\begin{remark}
    \label{rem:dgm_sigma_automorphism}
    The DG-endofuctors \( \Sigma_{\C_{\dg}} \) and \( \Sigma_{H^0(\dgM)} \) are automorphisms.
\end{remark}

The following is a definition for a cone in \( \dgM \), which will be a part of the triangulation of \( H^0(\dgM) \) afterwards.

\begin{definition}[Cone in \( \dgM \)]
    \label{def:dgm_cone}
    Let \( F, G \in \dgM \), and let \( \eta \in Z^0(\dgM(F, G)) \). Then we can define the following DG-functor, denoted \( C_{\eta} \) as follows
    \begin{itemize}
        \item {
            For any \( A \in \Cc^{\op} \), define \( C_{\eta}(A) \) to be the chain complex with the underlying \( R \)-modules of \( (\Sigma F A) \oplus (G A) \), but with the differential defined as
            \[
                d_{C_{\eta}A} :=
                \begin{pmatrix}
                    d_{\Sigma F A} & 0 \\
                    \eta_A & d_{GA}
                \end{pmatrix}
                =
                \begin{pmatrix}
                    -d_{F A} & 0 \\
                    \eta_A & d_{GA}
                \end{pmatrix}
            \]

            This can also be stated as
            \[
                d_{C_{\eta}} =
                \begin{pmatrix}
                    -d_F & 0 \\
                    \eta & d_G
                \end{pmatrix}
            \]
        }
        \item {
            For any \( f \in \Cc^{\op}(A, B)_n \) let
            \begin{align*}
                C_{\eta, A, B, n}: \Cc^{\op}(A, B)_n &\to \C_{\dg}(C_{\eta} A, C_{\eta} B)_n = \C_{\dg}((\Sigma F A) \oplus (G A), (\Sigma F B) \oplus (G B))_n \\
                f &\mapsto (\Sigma F f) \oplus (G f) =
                \begin{pmatrix}
                    \Sigma F f & 0 \\
                    0 & G f
                \end{pmatrix}.
            \end{align*}
        }
    \end{itemize}
\end{definition}

The above definition is well-defined by the following remark.
\begin{remark}
    In order to check if \( C_{\eta} \) in \autoref{def:dgm_cone} is a DG-functor, we have to check that for any \( A, B \in \Cc \),
    \[
        C_{\eta, A, B}: \Cc^{\op}(A, B) \to \C_{\dg}(C_{\eta} A, C_{\eta} B)
    \]
    is a chain morphism, i.e., we want to show that
    \[
        d_{\C_{\dg}(C_{\eta} A, C_{\eta} B)}(C_{\eta} f) = C_{\eta} d_{\Cc^{\op}(A, B)}(f).
    \]

    Consider the following equation
    \begin{align*}
        d_{\C_{\dg}(C_{\eta} A, C_{\eta} B)}&(C_{\eta} f) = d_{C_{\eta} B} \circ
        \begin{pmatrix}
            \Sigma F f & 0 \\
            0 & G f
        \end{pmatrix}
        - (-1)^n
        \begin{pmatrix}
            \Sigma F f & 0 \\
            0 & G f
        \end{pmatrix}
        \circ d_{C_{\eta} A} \\
        &=
        \begin{pmatrix}
            -(-1)^n d_{F B} \circ F f & 0 \\
            (-1)^n \eta_B \circ F f & d_{G B} \circ G f
        \end{pmatrix} \\
        &\hspace{0.4cm} - (-1)^n
        \begin{pmatrix}
            -(-1)^n (F f) \circ d_{F A} & 0 \\
            (G f) \circ \eta_A & (G f) \circ d_{G A}
        \end{pmatrix} \\
        &=
        \begin{pmatrix}
            (-1)^{n + 1}\tuple*{d_{F B} \circ F f - (-1)^n (F f) \circ d_{F A}} & 0 \\
            (-1)^n\tuple*{\eta_B \circ F f - (G f) \circ \eta_A} & d_{G B} \circ G f - (-1)^n (G f) \circ d_{G A}
        \end{pmatrix} \\
        &=
        \begin{pmatrix}
            \Sigma d_{\C_{\dg}(F A, F B)}(F f) & 0 \\
            (-1)^{n + 1} 0 & d_{\C_{\dg}(G A, G B)}(G f)
        \end{pmatrix} \\
        &=
        \begin{pmatrix}
            \Sigma F d_{\Cc^{\op}(A, B)}(f) & 0 \\
            0 & G d_{\Cc^{\op}(A, B)}(f)
        \end{pmatrix} \\
        &= \C_{\eta} d_{\Cc^{\op}(A, B)}(f).
    \end{align*}
    % TODO: Unit diagram og functorial diagram?
\end{remark}

For a triangulated category, we also need to define the triangulation. The triangulation of \( H^0(\dgM) \) is as follows.
\begin{definition}
    \label{def:delta_H_0_dgmod}
    % TODO: Vis at desse er DG-nat. trans.
    Let \( \iota \) be a degree \( 0 \) DG-morphism in \( \dgM \) with components
    \[
        \iota := \tuple{ \iota_{A, i} }_{A \in \Cc, i \in \Zb},
    \]
    where
    \[
        \iota_{A, i}: (G A)_i \rightarrowtail (C_{\eta} A)_i = (\Sigma F A)_i \oplus (G A)_i
    \]
    are the inclusion morphisms. Similarly, let \( \pi \) be a degree \( 0 \) DG-morphism in \( \dgM \) with components
    \[
        \pi := \tuple{ \pi_{A, i} }_{A \in \Cc, i \in \Zb},
    \]
    where
    \[
        \pi_{A, i}: (C_{\eta} A)_i = (\Sigma F A)_i \oplus (G A)_i \twoheadrightarrow (\Sigma F A)_i
    \]
    are the porjection morphisms.

    Then let \( \Delta \) be the class of triangles isomorphic to any triangle in \( H^0(\dgM) \) on the form
    \begin{center}
        \begin{tikzpicture}
            \diagram{m}{1cm}{1cm} {
                F \& G \& C_{\eta} \& \Sigma F. \\
            };

            \draw[math]
                (m-1-1) edge node {[\eta]} (m-1-2)
                (m-1-2) edge node {[\iota]} (m-1-3)
                (m-1-3) edge node {[\pi]} (m-1-4);
        \end{tikzpicture}
    \end{center}
\end{definition}

All in all this implies that \( H^0(\dgM) \) is triangulated. The proof for this statment can be found in TODO:CITE.
\begin{theorem}
    Let \( \Sigma \) be the shift functor as defined in \autoref{def:sigma_dgmod}, and let \( \Delta \) be as defined in \autoref{def:delta_H_0_dgmod}.

    Then \( \tuple*{H^0(\dgM), \Sigma, \Delta} \) is a triangulated category.
\end{theorem}

% USIKKER START

% \begin{definition}[Acyclic DG-module]
%     Let \( A \in \dgM \).

%     Then \( A \) is called \emph{acyclic} if for any \( X \in \Cc^{\op} \), we have that \( A(X) \in \C_{\dg} \) is acyclic, i.e. \( H^*(A(X)) = 0 \).
% \end{definition}

% \begin{definition}[DG-projective module]
%     Let \( P \in \dgM \) be a DG-module.

%     Then \( P \) is called a \emph{DG-projective module over \( \Cc \)} if:
    
%     For any DG-module \( A \in \dgM \) and any epimorphism \( f \in \dgM(A, P) \) where \( \ker(f) \in \dgM \) is acyclic. Then \( f \) is split.
% \end{definition}

% % TODO: Various definitions and idiosyncracies. Which is correct?
%     % Acyclic kernel? Projective objects? Spanned?
%     % Krause 07 -> Compact objects, maybe more.
%     % Keller 94 -> Another definition of derived DG category.
% % TODO: Following def from Krause 07, but not explicitly written down. Is it correct?
% \begin{definition}[Derived DG-category]
%     Let \( \Cc \) be a DG-category.

%     Then the \emph{derived DG-category} of \( \Cc \), denoted \( \D(\Cc) \), is defined as the full subcategory of \( H^0(\dgM) \) spanned by the objects of \( \dgM \) that are DG-projective.
% \end{definition}

% % MS-Question: What is coproduct for the derived category?
% % Cite: Jasso--Muro p.31, only a statement, no proof
% \begin{proposition}
%     \( \D(\Cc) \) is closed under arbitrary coproduct.
% \end{proposition}

% % Cite: Krause 07 p. 29
% \begin{definition}[Compact objects of a category]
%     Let \( \Tc \) be a triangulated category with arbitrary coproduct. Let \( X \in \Tc \).
    
%     If for any index set \( I \) and any morphism \( f: X \to \coprod_{i \in I} Y_i \), there is a finite index set \( J \subseteq I \) such that \( f \) factors through \( \coprod_{j \in J} Y_j \).
    
%     Then define \( X \) as a \emph{compact object in \( \Cc \)}.
% \end{definition}

% \begin{definition}[Perfect derived DG-category \( \D^c(\Cc) \)]
%     Let \( \D(\Cc) \) be the derived DG-category of \( \Cc \).

%     Then define \( \D^c(\Cc) \) to be the full subcategory of \( \D(\Cc) \) consisting of all compact objects in \( \D(\Cc) \). This is called the \emph{perfect derived DG-category of \( \Cc \)}.
% \end{definition}

% \begin{proposition}
%     \( \D^c(\Cc) \) is a triangulated sub category of \( H^0(\dgM) \).
% \end{proposition}

% USIKKER SLUTT

The next step is to define what an algebraic triangulated category is by relating triangulated categories to \( H^0(\dgM) \). We will accomplis this by using the DG-Yoneda embedding from \cite[Corollary 6.3.6]{Borceux_1994}, which is defined as follows.
\begin{definition}[DG-Yoneda embedding]
    \label{def:DG_Yoneda_embedding}
    Let \( \Cc \) be a small DG-category.
    
    Then let \( \mathbf{h} \) be the DG-functor defined as follows
    \begin{align*}
        \mathbf{h}: \Cc &\to \dgM \\
        A &\mapsto \Cc(-, A)
    \end{align*}

    This DG-functor is called the \emph{DG-Yoneda embedding of \( \Cc \)}.
\end{definition}

The DG-Yoneda embedding is staded without proof because it is too technical to include in this thesis. For a proof of its well-definedness see \cite[Corollary 6.3.6]{Borceux_1994}. We will only be using it in the defintion of an algebraic triangulated category in this thesis.

The DG-Yoneda embedding shows that any DG-category can be seen as a full subcategory of the category of DG-modules. In fact, this embedding can be extended to also work on \( H^0(\Cc) \), where it turns out to be exact.

Then the following theorem shows that taking \( H^0(\mathbf{h}) \) yields a fully faithful exact functor from \( H^0(\Cc) \) to \( \D^c(\Cc) \).

\begin{theorem}
    Let \( \Cc \) be a small DG-category, and let \( \mathbf{h} \) be as in \autoref{def:DG_Yoneda_embedding}.

    Then the functor
    \[
        H^0(\mathbf{h}): H^0(\Cc) \to H^0(\dgM)
    \]
    is full, faithful, and exact.
\end{theorem}
The proof of the above statement can be found in TODO.

Then we can define what a ``pre-triangulated DG-category'' is, which is the final important piece in this definition of algebraic triangulated categories. It must not be confused with a pre-triangulated (non-DG) category which is a category that satisfies the triangulated axioms {\bf (TR1)}, {\bf (TR2)}, and {\bf (TR3)}, but not {\bf (TR4)}.
\begin{definition}[pre-triangulated DG-category]
    Let \( \Cc \) be a small DG-category.

    Then \( \Cc \) is called a \emph{pre-triangulated DG-category} if the image of the (fully faithful and exact) functor \( H^0(\mathbf{h}): H^0(\Cc) \to H^0(\dgM) \) is a triangulated subcategory of \( H^0(\dgM) \).
\end{definition}

Finally we can define what an algebraic triangulated category is.
\begin{definition}[Algebraic triangulated category]
    \label{def:alg_tri_cat}
    Let \( \Tc \) be a triangulated category.

    Then \( \Tc \) is called an \emph{algebraic triangulated category} if there exist a pre-triangulated DG-category, \( \Cc \), called a \emph{DG-enhancement of \( \Tc \)}, such that \( H^0(\Cc) \) is equivalent as triangulated categories to \( \Tc \).
\end{definition}

There are many different and equivalent definitions of algebraic triangulated categories. The reason we use the above definition is because it closely ties the algebraic triangulated category to the category of DG-modules. This in turn allows us to use properties of DG-modules in order to make it possible for Toda brackets to equal Massey products.

% MS-Question: Er definisjonen av algebraisk for spesifikk? Eg trur ikkje er treng at alg.tri.kat. er full underkategori av D^c(\Cc).
