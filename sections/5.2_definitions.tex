In this subsection we want to define what an algebraic triangulated category is using the notion of DG-enhancements. We will start by defining \( H^0(\dgM) \), and then using the DG-Yoneda embedding, which we will also define, we can consider any DG-category as a subcategory of the category of DG-modules.

The following definition is very similar to \autoref{def:H_bullet_dg_category}. However unlike the cohomology category, this category is not enriched over \( \C \), it is simply a category in the usual sense.

\begin{definition}[0th cohomology category of a DG-category]
    \label{def:0_th_cohomology_of_dg_cat}
    Let \( \Cc \) be a DG-category over \( R \).

    Then let \( H^0(\Cc) \) be the following category defined as follows
    \begin{enumerate}
        \item {
            Let \( \Obj(H^0(\Cc)) := \Obj(\Cc) \).
        }
        \item {
            Let \( H^0(\Cc)(A, B) := H^0(\Cc(A, B)) \).
        }
        \item {
            Let \( A, B, C \in H^0(\Cc) \) with \( [f] \in H^0(A, B) \) and \( [g] \in H^0(B, C) \).

            Then let composition be as follows
            \begin{align*}
                c_{H^0(\Cc)}: H^0(\Cc)(B, C) \times H^0(\Cc)(A, B) &\to H^0(\Cc)(A, C) \\
                [g] \times [f] &\mapsto \class*{ c_{\Cc}(g \otimes f) }
            \end{align*}
        }
        \item {
            For \( A \in \Cc \), and \( u_A^{\Cc} \) the unit morphism for \( A \) in \( \Cc \), let the identity morphism in \( H^0(\Cc)(A, A) \) be defined as
            \[
                \Id_A := [u_{A, 0}^{\Cc} (1)].
            \]
        }
    \end{enumerate}
\end{definition}

Composition in the above category is well-defined and associative because it is simply \( c_0 \) from \autoref{def:H_bullet_dg_category}. It remains to show that the identity is well-defined.

\begin{remark}
    In order to show that the identity in the above definition is well-defined, we have to prove that \( [u_{A, 0} (1)] \) is a cycle, and that \( [u_{A, 0} (1)] \) acts like the identity in \( H^0(\Cc) \).

    First, note that since \( u_A^{\Cc} \) is a chain morphism, and \( d_I = 0 \), for any \( n \in \Zb \),
    \[
        d_n^{\Cc(A, A)} \circ u_{A, n}^{\Cc} = u_{A, n}^{\Cc} \circ d_n^I = 0,
    \]
    and so \( d_0^{\Cc(A, A)} \circ u_{A, 0}^{\Cc} (1) = 0 \), and \( u_{A, 0}^{\Cc} (1) \) is a cycle.

    Second, it follows that for \( [f] \in H^0(\Cc)(A, A) \), we have from the ``unit axiom'' in the definition of an enriched category (\cite[Diagram 6.10]{Borceux_1994}) restricted onto the elementary tensors \( 1 \otimes f \) and \( f \otimes 1 \), that
    \begin{align*}
        \Id_A \circ [f] &= \class*{ c_{\Cc}(u_{A, 0}^{\Cc} (1) \otimes f) } \\
        &= [f] \\
        &= \class*{ f \otimes c_{\Cc}(u_{A, 0}^{\Cc} (1)) } \\
        &= [f] \circ \Id_A.
    \end{align*}

    This implies that \( u_{A, 0}^{\Cc} (1) \) is the unique identity in \( H^0(\Cc)(A, A) \).
\end{remark}

\autoref{def:0_th_cohomology_of_dg_cat} connects the theory of DG-categories to the chain homotopy category.

\begin{remark}
    \label{rem:c_dg_h_0_is_chain_homotopy_cat}
    A consequence of \autoref{def:0_th_cohomology_of_dg_cat}, is that we can write the chain homotopy category, \( \K(\Mod(R)) \), defined in \autoref{def:chain_homotopy_cat}, as simply \( H^0(\C_{\dg}) \).
    
    Unpacking the definitions, \( H^0(\C_{\dg})(A, B) \) is exactly every chain homotopy class from \( A \) to \( B \), and both composition and identities all match.
\end{remark}

The goal is to show that \( H^0\tuple*{\dgM} \) is triangulated. To do that we need to define a shift functor and a class of distinguished triangles.

Before we define the shift functor, we can simplify the notation further, to make the definitions more intuitive and easier to work with.

\begin{notation}
    \label{not:prod_coprod_no_order}
    We will consider products and coproducts to have no order. To illustrate, consider the following example.

    Consider two chain complexes \( A \), and \( B \), we can construct the set \( \prod_{j \in \Zb} \Mod(R)(A_{j - 5}, B_j) \). We consider this set as \emph{equal} to \( \prod_{j \in \Zb} \Mod(R)(A_j, B_{j + 5}) \) even though the \( j \)'s do not match. Therefore, an element \( f \in \Mod(R)(A_{j - 5}, B_j) \) is also an element of \( \prod_{j \in \Zb} \Mod(R)(A_j, B_{j + 5}) \).
\end{notation}

The following is a definition for a shift functor on \( \C_{\dg} \), which will later induce a shift functor on \( \dgM \).

\begin{definition}[Shift in \( \C_{\dg} \)]
    \label{def:sigma_c_dg}
    Let the DG-functor \( \Sigma_{\C_{\dg}} \) be defined as follows:
    \begin{enumerate}
        \item {
            For \( A \in \C_{\dg} \), and \( \Sigma \) the ordinary shift functor on \( \C \), let
            \[
                \Sigma_{\C_{\dg}}(A) := \Sigma A
            \]
        }
        \item {
            Let \( A, B \in \C_{\dg} \) and \( n \in \Zb \).

            Then define \( \Sigma_n^{\C_{\dg}} \) as follows
            \begin{align*}
                \Sigma_n^{\C_{\dg}}: \C_{\dg}(A, B)_n &\to \C_{\dg}(\Sigma A, \Sigma B)_n \\
                f &\mapsto (-1)^n f.
            \end{align*}
        }
    \end{enumerate}
    Then \( \Sigma_{\C_{\dg}} \) is called the \emph{shift functor of \( \C_{\dg} \)}. When the context is uncertain, we will denote it as \( \Sigma_{\C_{\dg}} \), and when the context is certain we will use \( \Sigma \).
\end{definition}

By \autoref{not:prod_coprod_no_order}, the chain morphism \( \Sigma_n^{\C_{\dg}} \) makes sense, because
\[
    \C_{\dg}(\Sigma A, \Sigma B)_n  = \prod_{i \in \Zb} \Mod(R)(A_{i + 1}, B_{i + 1 + n}) = \prod_{i \in \Zb} \Mod(R)(A_i, B_{i + n}) = \C_{\dg}(A, B)_n,
\]
and so \( f \in \C_{\dg}(A, B)_n \) if and only if \( f \in \C_{\dg}(\Sigma A, \Sigma B)_n \).

Note that even though the above statement is true, \( \C_{\dg}(A, B) \neq \C_{\dg}(\Sigma A, \Sigma B) \) in general because the differentials differ by a sign, which the following remark explains.

\begin{remark}
    \label{rem:c_dg_sigma_d_equal_minus_d}
    As a result of the above definition as well as \autoref{not:prod_coprod_no_order}, we get that for any \( A, B \in \C \),
    \[
        d_{\Sigma A} = - d_A = \Sigma d_A,
    \]
    and
    \[
        d_{[A, B]} = - d_{[\Sigma A, \Sigma B]}.
    \]

    The first equation holds as \( d_A \in \C_{\dg}(A, A)_1 \), and so \( \Sigma d_A = - d_A \). In addition,
    \[
        d_{\Sigma A} = (d_n^{\Sigma A})_{n \in \Zb} = (- d_{n + 1}^A)_{n \in \Zb} = - (d_n^A)_{n \in \Zb} = - d_A.
    \]

    The second equation holds for \( f \in [A, B]_n \), as
    \begin{align*}
        d_{[A, B]} (f) &= d_B \circ f - (-1)^n f \circ d_A \\
        &= - (d_{\Sigma B} \circ f - (-1)^n f \circ d_{\Sigma A}) \\
        &= - d_{[\Sigma A, \Sigma B]} (f).
    \end{align*}
\end{remark}

The reason \( \Sigma_{\C_{\dg}} \) is a DG-functor is given by the following remark. 

\begin{remark}
    In order to show that \( \Sigma \) from \autoref{def:sigma_c_dg} is a DG-functor, we need to show the following three properties mentioned by \cite[Definition 6.2.3]{Borceux_1994}:
    \begin{enumerate}
        \item {
            We need to show that
            \[
                \Sigma_{A, B}: \C_{\dg}(A, B) \to \C_{\dg}(\Sigma A, \Sigma B)
            \]
            is a chain morphism by checking if it commutes with the differential.

            Fix some \( n \in \Zb \), we want to show that \( \Sigma \circ d_{[A, B]} = d_{[\Sigma A, \Sigma B]} \circ \Sigma \).

            Consider the following equation
            \begin{align*}
                \Sigma_{n + 1} \circ d_n^{[A, B]} &= (-1)^{n + 1} d_n^{[A, B]} \\
                &= (-1)^n d_n^{[\Sigma A, \Sigma B]} \\
                &= d_n^{[\Sigma A, \Sigma B]} \circ \Sigma_n.
            \end{align*}
        }
        \item {
            We need to show that the following diagram,
            \begin{center}
                \begin{tikzpicture}
                    \diagram{m}{1cm}{1cm} {
                        \C_{\dg}(B, C) \otimes \C_{\dg}(A, B) \& \C_{\dg}(A, C) \\
                        \C_{\dg}(\Sigma B,\Sigma C) \otimes \C_{\dg}(\Sigma A,\Sigma B) \& \C_{\dg}(\Sigma A, \Sigma C), \\
                    };

                    \draw[math]
                        (m-1-1) edge node {c} (m-1-2)
                            edge node[swap] {\Sigma_{B, C} \otimes \Sigma_{A, B}} (m-2-1)
                        (m-1-2) edge node {\Sigma_{A, C}} (m-2-2)

                        (m-2-1) edge node {c} (m-2-2);
                \end{tikzpicture}
            \end{center}
            commutes.

            Let \( n \in \Zb \), it is sufficient to show that
            \[
                \Sigma_n^{A, C} \circ c_n = c_n \circ (\Sigma_{B, C} \otimes \Sigma_{A, C})_n.
            \]

            Let \( i + j = n \), with \( f \in \C_{\dg}(A, B)_i, \) and \( g \in \C_{\dg}(B, C)_j \). We have
            \begin{align*}
                \Sigma_n^{A, C} \circ c_n (g \otimes f) &= (-1)^n c_n (g \otimes f) \\
                &= (-1)^i (-1)^j c_n (g \otimes f) \\
                &= c_n\tuple*{((-1)^j g) \otimes ((-1)^i f)} \\
                &= c_n\tuple*{(\Sigma_j^{B, C} g) \otimes (\Sigma_i^{A, B} f)} \\
                &= c_n \circ (\Sigma_{B, C} \otimes \Sigma_{A, B})_n (g \otimes f).
            \end{align*}
            By the uniqueness of \autoref{lem:map_out_of_tensor_unique} we get that
            \[
                \Sigma_n^{A, C} \circ c_n = c_n \circ (\Sigma_{B, C} \otimes \Sigma_{A, B})_n.
            \]
        }
        \item {
            We need to show that the following diagram, for any \( A \in \C_{\dg} \),
            \begin{center}
                \begin{tikzpicture}
                    \diagram{m}{1cm}{1cm} {
                        I \& \C_{\dg}(A, A) \\
                        \& \C_{\dg}(\Sigma A, \Sigma A) \\
                    };

                    \draw[math]
                        (m-1-1) edge node {u_A} (m-1-2)
                            edge node[swap] {u_{\Sigma A}} (m-2-2)
                        (m-1-2) edge node {\Sigma_{A, A}} (m-2-2);
                \end{tikzpicture}
            \end{center}
            commutes.

            For \( n \neq 0 \), the diagram commutes, since \( u_n^A = 0 \) for any \( A \).

            For \( n = 0 \), and any \( r \in R \), we have the following equation
            \[
                \Sigma \circ u_0^A (r) = \Sigma (r \Id_{A_i})_{i \in \Zb} = (r \Id_{A_i})_{i \in \Zb} = u_0^{\Sigma A} (r),
            \]
            which implies that the diagram commutes.
        }
    \end{enumerate}
\end{remark}

We can extend \( \Sigma_{\C_{\dg}} \) onto \( \dgM \) in the following way.

\begin{definition}
    \label{def:sigma_dgmod}
    Let the DG-functor \( \Sigma_{\dgM} \) be defined as follows:
    \begin{enumerate}
        \item {
            For any \( F \in \dgM \), let
            \[
                \Sigma_{\dgM} F := \Sigma_{\C_{\dg}} \circ F.
            \]
        }
        \item {
            Let \( F, G \in \dgM \), and \( n \in \Zb \).
            
            Then define \( \Sigma_{F, G, n}^{\dgM} \) as follows
            \begin{align*}
                \Sigma_{F, G, n}^{\dgM}: \dgM(F, G)_n &\to \dgM(\Sigma F, \Sigma G)_n \\
                \eta &\mapsto (-1)^n \eta.
            \end{align*}
        }
    \end{enumerate}
    Then \( \Sigma_{\dgM} \) is called the \emph{shift functor of \( \dgM \)}. When the context is uncertain, we will denote it as \( \Sigma_{\dgM} \), and when the context is certain we will use \( \Sigma \).
\end{definition}

% TODO: Legg til at Sigma kan verta definert på prod_A C_dg
Since
\[
    \dgM(\Sigma F, \Sigma G)_n \subseteq \prod_{A \in \Cc} \C_{\dg}(\Sigma F A, \Sigma G A)_n = \prod_{A \in \Cc} \C_{\dg}(F A, G A)_n,
\]
and for any \( f \in \Cc^{\op}(B, C) \),
\begin{align*}
    (-1)^{|f||\eta|}(\Sigma G f) \circ \eta_{B} &= \eta_{C} \circ (\Sigma F f) \\
    &\Updownarrow \\
    (-1)^{|f||\eta| + |f|} (G f) \circ \eta_{B} &= (-1)^{|f|} \eta_{C} \circ (F f) \\
    &\Updownarrow \\
    (-1)^{|f||\eta|}(G f) \circ \eta_{B} &= \eta_{C} \circ (F f), \\
\end{align*}
this implies \( \eta \in \dgM(\Sigma F, \Sigma G)_n \), and thus \( \Sigma_{F, G, n}^{\dgM} \) is a well-defined morphism.

We can also consider \( \Sigma_{F, G, n}^{\dgM} \) as follows,
\[
    \Sigma_{F, G, n}^{\dgM} = \prod_{A \in \Cc} \Sigma_{F A, G A, n}^{\C_{\dg}}.
\] 
This is a morphism from \( \prod_{A \in \Cc} \C_{\dg}(F A, G A)_n \) to \( \prod_{A \in \Cc} \C_{\dg}(\Sigma F A, \Sigma G A)_n \).

Before we prove that \( \Sigma_{\dgM} \) is a DG-functor, we have the following remark, which mirrors \autoref{rem:c_dg_sigma_d_equal_minus_d}.

\begin{remark}
    We have that for any \( F, G \in \dgM \),
    \[
        d_{\Sigma F} = -d_F = \Sigma d_F
    \]
    and
    \[
        d_{\dgM(\Sigma F, \Sigma G)} = - d_{\dgM(F, G)}.
    \]

    The first equality holds as
    \[
        d_{\Sigma F} = \tuple*{d_{\Sigma F A}}_{A \in \Cc} = \tuple*{-d_{F A}}_{A \in \Cc} = - d_F = \Sigma d_F.
    \]
    The second equality holds, as for any \( \eta \in \dgM(F, G)_n \), we have
    \[
        d_{\dgM(\Sigma F, \Sigma G)} (\eta) = d_{\Sigma G} \circ \eta - (-1)^n \eta \circ d_{\Sigma F} = - (d_G \circ \eta - (-1)^n \eta \circ d_F) = - d_{\dgM(F, G)} (\eta).
    \]
\end{remark}

By the following remark, \( \Sigma_{\dgM} \) is a well-defined DG-functor.

\begin{remark}
    \( \Sigma_{\dgM} \) is well-defined by the following arguments:

    First, consider the following equation for some \( n \in \Zb \),
    \begin{align*}
        d_n^{\dgM(\Sigma F, \Sigma G)} \circ \Sigma_n^{F, G} &= (-1)^n d_n^{\dgM(\Sigma F, \Sigma G)} \\
        &= (-1)^{n + 1} d_n^{\dgM(F, G)} \\
        &= \Sigma_{n + 1}^{F, G} \circ d_n^{\dgM(F, G)}.
    \end{align*}
    This implies that \( \Sigma_{F, G}: \dgM(F, G) \to \dgM(\Sigma F, \Sigma G) \) is a chain morphism.

    Second, we want the following diagram,
    \begin{center}
        \begin{tikzpicture}
            \diagram{m}{1cm}{1cm} {
                \dgM(G, H) \otimes \dgM(F, G) \& \dgM(F, H) \\
                \dgM(\Sigma G,\Sigma H) \otimes \dgM(\Sigma F, \Sigma G) \& \dgM(\Sigma F, \Sigma H), \\
            };

            \draw[math]
                (m-1-1) edge node {c} (m-1-2)
                    edge node[swap] {\Sigma_{G, H} \otimes \Sigma_{F, G}} (m-2-1)
                (m-1-2) edge node {\Sigma_{F, H}} (m-2-2)

                (m-2-1) edge node {c} (m-2-2);
        \end{tikzpicture}
    \end{center}
    to commute.
    
    For any \( i, j \in \Zb \) with \( i + j = n \), \( \eta \in \dgM(F, G)_i \), and \( \mu \in \dgM(G, H)_j \), we have
    \begin{align*}
        \Sigma_n^{F, H} \circ c_n (\mu \otimes \eta) &= (-1)^n c_n (\mu \otimes \eta) \\
        &= (-1)^i (-1)^j c_n (\mu \otimes \eta) \\
        &= c_n ((((-1))^j \mu) \otimes ((-1)^i \eta)) \\
        &= c_n ((\Sigma_j^{G, H} \mu) \otimes (\Sigma_i^{F, G} \eta)) \\
        &= c_n \circ (\Sigma_{G, H} \otimes \Sigma_{F, G})_n (\mu \otimes \eta).
    \end{align*}
    By uniqueness of \autoref{lem:map_out_of_tensor_unique}, this implies \( \Sigma_n^{F, H} \circ c_n = c_n \circ (\Sigma_{G, H} \otimes \Sigma_{F, G})_n \), which implies that the diagram commutes.

    Third, we want the following diagram for any \( F \in \dgM \),
    \begin{center}
        \begin{tikzpicture}
            \diagram{m}{1cm}{1cm} {
                I \& \dgM(F, F) \\
                \& \dgM(\Sigma F, \Sigma F) \\
            };

            \draw[math]
                (m-1-1) edge node {u_F} (m-1-2)
                    edge node[swap] {u_{\Sigma F}} (m-2-2)
                (m-1-2) edge node {\Sigma_{F, F}} (m-2-2);
        \end{tikzpicture}
    \end{center}
    to commute.

    For \( n \neq 0 \), the diagram commutes, since 
    \[
        u_n^F = \prod_{A \in \Cc} u_n^{F A} = 0.
    \]

    For \( n = 0 \), and any \( r \in R \), we have
    \[
        \Sigma \circ u_0^F (r) = \Sigma (r \Id_{A_i})_{i \in \Zb, A \in \Cc} = (r \Id_{A_i})_{i \in \Zb, A \in \Cc} = u_0^{\Sigma F} (r),
    \]
    which implies that the diagram commutes.
\end{remark}

In order to get the shift functor on \( H^0(\dgM) \), we have to take the \( 0 \)th cohomology of this functor. It is defined as follows.

\begin{definition}[\( H^0 \)-induced functor]
    \label{def:H^0-induced_functor}
    Let \( \Ac \) and \( \Bc \) be two DG-categories, and let \( F: \Ac \to \Bc \) be a DG-functor.

    Then define the functor \( H^0(F) \) as follows:
    \begin{align*}
        H^0(F): H^0(\Ac) &\to H^0(\Bc) \\
        A &\mapsto F A
    \end{align*}
    where for any \( f \in H^0(\Ac)(A, B) \)
    \begin{align*}
        H^0(F)_{A, B}: H^0(\Ac)(A, B) &\to H^0(\Bc)(F A, F B) \\
        [f] &\mapsto [F_0^{A, B} f].
    \end{align*}

    This is called the \emph{\( H^0 \)-induced functor of \( F \)}.
\end{definition}

The above definition is well-defined because by definition \( F_{A, B}: \Ac(A, B) \to \Bc(F A, F B) \) is a chain morphism, and therefore maps cycles to cycles and boundaries to boundaries. Furthermore, functoriality is straight forward to prove.

This leads us to the definition of the shift functor on \( H^0(\dgM) \).

\begin{definition}[\( \Sigma \) on \( H^0(\dgM) \)]
    \label{def:sigma_h_0_dgmod}
    Let \( \Cc \) be a small DG-category, and let \( \Sigma \) be as defined in \autoref{def:sigma_dgmod}.

    Then the functor
    \[
        \Sigma_{H^0(\dgM)} := H^0(\Sigma): H^0(\dgM) \to H^0(\dgM)
    \]
    is called the \emph{shift functor of \( H^0(\dgM) \)}. When the context is uncertain, we will denote it as \( \Sigma_{H^0(\dgM)} \), and when the context is certain we will use \( \Sigma \).
\end{definition}

The overloading of the notation \( \Sigma \) makes the notation much more concise, and in certain situations and makes equations much more clean while still being correct.

A helpful consequence of the definition of all the \( \Sigma \)'s is that they are all clearly automorphisms.
\begin{remark}
    \label{rem:dgm_sigma_automorphism}
    The DG-endofunctors \( \Sigma_{\C_{\dg}} \), and \( \Sigma_{\dgM} \), as well as the endofunctor \( \Sigma_{H^0(\dgM)} \) are all automorphisms.
\end{remark}

We want to prove that \( H^0(\dgM) \) is an additive category, which requires a biproduct. The following notation will help in defining and proving it is correct.

\begin{notation}
    \label{not:c_dg_matrix_direct_sum}
    We will consider the direct sum, as well as matrices of DG-morphisms in \( \C_{\dg} \) in the obvious way.

    For example, for a 2×2 matrix, we use the following notation:

    For \( n \in \Zb \),
    \begin{align*}
        f &\in \C_{\dg}(A, C)_n, \\
        h &\in \C_{\dg}(A, D)_n, \\
        g &\in \C_{\dg}(B, D)_n, \\
        \intertext{and}
        k &\in \C_{\dg}(B, C)_n,
    \end{align*}
    denote the DG-morphism
    \[
        \begin{pmatrix}
            f_i & k_i \\
            h_i & g_i
        \end{pmatrix}_{i \in \Zb}
        \in
        \C_{\dg}(A \oplus B, C \oplus D)_n
    \]
    as
    \[
        \begin{pmatrix}
            f & k \\
            h & g
        \end{pmatrix}.
    \]
    If \( k = 0 = h \), then we can also write it as \( f \oplus g \).
\end{notation}

The above notation fits nicely with the following projection and embedding DG-morphisms.
\begin{definition}
    For \( A, B \in \C \), define
    \[
        \pi_A := \tuple*{\pi_{A_i}}_{i \in \Zb} \in \C_{\dg}(A \oplus B, A)_0
    \]
    and
    \[
        \iota_A := \tuple*{\iota_{A_i}}_{i \in \Zb} \in \C_{\dg}(A, A \oplus B)_0.
    \]
\end{definition}

Since \( \C_{\dg} \) is not a ``normal'' category, but a DG-category it does not have an identity morphism in the usual sense. Luckily, it turns out that if we consider composition of DG-morphisms, then the matrices, the projection, and the inclusion DG-morphisms act exactly as expected, where the identity morphism is swapped for \( u_0^A(1) \).

It will become apparent later on that the biproduct in \( H^0(\dgM) \) is inherited from the following construction in \( \dgM \).

\begin{definition}
    \label{def:dgm_biproduct}
    For \( F, G \in \dgM \), let \( F \oplus G \) be the following DG-module:
    \begin{itemize}
        \item {
            For \( A \in \Cc \), let
            \[
                (F \oplus G) A = (F A) \oplus (G A).
            \]
        }
        \item {
            For \( A, B \in \Cc \), and \( n \in \Zb \), let
            \begin{align*}
                (F \oplus G)_n^{A, B}: \Cc^{\op}(A, B)_n &\to \C_{\dg}((F A) \oplus (G A), (F B) \oplus (G B))_n \\
                f &\mapsto (F f) \oplus (G f).
            \end{align*}
        }
    \end{itemize}
\end{definition}

It is not obvious why the previous definition is a DG-functor. The following remark proves that \( F \oplus G \) is a DG-functor.

\begin{remark}
    \( F \oplus G \) is a DG-functor, as:
    \begin{itemize}
        \item {
            For \( n \in \Zb \), and \( f \in \Cc^{\op}(A, B)_n \), we have
            \begin{align*}
                &d_n \circ (F \oplus G)_n^{A, B} (f) \\
                &= d_n ((F f)_i \oplus (G f)_i)_{i \in \Zb} \\
                &= d_{(F B) \oplus (G B)} \circ ((F f)_i \oplus (G f)_i)_{i \in \Zb} - (-1)^n ((F f)_i \oplus (G f)_i)_{i \in \Zb} \circ d_{(F A) \oplus (G A)} \\
                &= \tuple*{d_{i + n}^{(F B) \oplus (G B)} \circ ((F f)_i \oplus (G f)_i) - (-1)^n ((F f)_{i + 1} \oplus (G f)_{i + 1}) \circ d_i^{(F A) \oplus (G A)}}_{i \in \Zb} \\
                &= \tuple*{ (d_{i + n}^{F B} \circ (F f)_i) \oplus (d_{i + n}^{G B} \circ (G f)_i) - (-1)^n ( ( (F f)_{i + 1} \circ d_i^{F A} ) \oplus ( (G f)_{i + 1} \circ d_i^{G A} ) ) }_{i \in \Zb} \\
                &= \tuple*{ \tuple*{d_{i + n}^{F B} \circ (F f)_i - (-1)^n (F f)_{i + 1} \circ d_i^{F A}} \oplus \tuple*{d_{i + n}^{G B} \circ (G f)_i - (-1)^n (G f)_{i + 1} \circ d_i^{G A}} }_{i \in \Zb} \\
                &= \tuple*{ \tuple*{d_n^{[F A, F B]} (F f)}_i \oplus \tuple*{ d_n^{[G A, G B]} (G f)}_i }_{i \in \Zb} \\
                &= \tuple*{ \tuple*{F d_n (f)}_i \oplus \tuple*{ G d_n (f)}_i }_{i \in \Zb} \\
                &= (F \oplus G)_{n + 1}^{A, B} (d_n (f)) \\
                &= (F \oplus G)_{n + 1}^{A, B} \circ d_n (f).
            \end{align*}
            This implies \( (F \oplus G)_{A, B} \) is a chain morphism.
        }
        \item {
            Consider the following diagram of chain complexes,
            \begin{center}
                \begin{tikzpicture}
                    \diagram{m}{1cm}{1cm} {
                        \Cc^{\op}(B, C) \otimes \Cc^{\op}(A, B) \& \Cc^{\op} (A, C) \\
                        \C_{\dg}((F \oplus G) B, (F \oplus G) C) \& \C_{\dg}((F \oplus G) A, (F \oplus G) C). \\
                    };

                    \draw[math]
                        (m-1-1) edge node {c} (m-1-2)
                            edge node[swap] {(F \oplus G)_{B, C} \otimes (F \oplus G)_{A, B}} (m-2-1)
                        (m-1-2) edge node {(F \oplus G)_{A, C}} (m-2-2)

                        (m-2-1) edge node {c} (m-2-2);
                \end{tikzpicture}
            \end{center}

            Let \( i, j \in \Zb, \) with \( i + j = n \), \( f \in \Cc(A, B)_i, \) and \( g \in \Cc(B, C)_j \). From
            \begin{align*}
                &c_n \circ ((F \oplus G)_{B, C} \otimes (F \oplus G)_{A, B})_n (g \otimes f) \\
                &= c_n ( ( (F g)_k \oplus (G g)_k )_{k \in \Zb} \otimes ( (F f)_k \oplus (G f)_k )_{k \in \Zb} ) \\
                &= ( ( (F g)_{k + i} \oplus (G g)_{k + i} ) \circ ( (F f)_k \oplus (G f)_k ) )_{k \in \Zb} \\
                &= ( ((F g)_{k + i} \circ (F f)_k) \oplus ( (G g)_{k + i} \circ (G f)_k ) )_{k \in \Zb} \\
                &= ( ((F g) \circ (F f))_k \oplus ( (G g) \circ (G f) )_k )_{k \in \Zb} \\
                &= ( (F (g \circ f) )_k \oplus ( G (g \circ f) )_k )_{k \in \Zb} \\
                &= (F \oplus G)_n^{A, C} ( g \circ f ) \\
                &= (F \oplus G)_n^{A, C} \circ c_n (g \otimes f),
            \end{align*}
            we have, along with uniqueness from \autoref{lem:map_out_of_tensor_unique}, that the diagram commutes.
        }
        \item {
            Consider the following diagram for \( A \in \Cc \),
            \begin{center}
                \begin{tikzpicture}
                    \diagram{m}{1cm}{1cm} {
                        I \& \Cc^{\op}(A, A) \\
                        \& \C_{\dg}((F \oplus G) A, (F \oplus G) A). \\
                    };

                    \draw[math]
                        (m-1-1) edge node {u_A} (m-1-2)
                            edge node[swap] {u_{(F \oplus G) A}} (m-2-2)
                        (m-1-2) edge node {(F \oplus G)_{A, A}} (m-2-2);
                \end{tikzpicture}
            \end{center}

            For \( n \neq 0 \), \( u_n^A = 0 \) for all \( A \), and so the diagram commutes.

            For \( n = 0 \), and \( r \in R \), we have
            \begin{align*}
                (F \oplus G) \circ u_0^A (r) &= \tuple*{ (F \circ u_0^A (r))_i \oplus (G \circ u_0^A (r))_i }_{i \in \Zb} \\
                &= ((u_0^{F A} (r))_i \oplus (u_0^{G A} (r))_i)_{i \in \Zb} \\
                &= ((r \Id_{(F A)_i})_i \oplus (r \Id_{(G A)_i})_i)_{i \in \Zb} \\
                &= (r \Id_{(F A)_i \oplus (G A)_i})_{i \in \Zb} \\
                &= (r \Id_{((F \oplus G ) A)_i })_{i \in \Zb} \\
                &= u_0^{(F \oplus G ) A} (r).
            \end{align*}
            This implies the diagram commutes.
        }
    \end{itemize}
\end{remark}

Similarly to how there is a projection and embedding DG-morphism in \( \C_{\dg} \) as mentioned after \autoref{not:c_dg_matrix_direct_sum}, there also exist similar DG-morphisms in \( \dgM \).

\begin{remark}
    \label{rem:dgm_pi_iota}
    For \( A, B \in \Mod(R) \), let \( \pi_A: A \oplus B \to A \) be the projection morphism. Then define
    \[
        \pi_F := \tuple*{ \pi_{(F A)_i} }_{i \in \Zb, A \in \Cc} \in \prod_{A \in \Cc} \C_{\dg}((F A) \oplus (G A), F A)_0.
    \]

    For any \( f \in \Cc^{\op}(A, B)_n, \) we have
    \begin{align*}
        \tuple*{ \pi_{(F B)_i} }_{i \in \Zb} \circ ((F \oplus G) f) &= \tuple*{ \pi_{(F B)_i} }_{i \in \Zb} \circ ((F f)_i \oplus (G f)_i)_{i \in \Zb} \\
        &= \tuple*{ \pi_{(F B)_{i + n}} \circ ((F f)_i \oplus (G f)_i) }_{i \in \Zb} \\
        &= \tuple*{ (F f)_i \circ \pi_{(F A)_i} }_{i \in \Zb} \\
        &= (F f) \circ \tuple*{ \pi_{(F A)_i} }_{i \in \Zb} \\
        &= (-1)^{|f| 0}(F f) \circ \tuple*{ \pi_{(F A)_i} }_{i \in \Zb}, \\
    \end{align*}
    which implies \( \pi_F \in \dgM(F \oplus G, F)_0 \), and similarly for \( \pi_G \).

    For \( A, B \in \Mod(R) \), let \( \iota_A: A \to A \oplus B \) be the embedding morphism. Then define
    \[
        \iota_F := \tuple*{ \iota_{(F A)_i} }_{i \in \Zb, A \in \Cc} \in \prod_{A \in \Cc} \C_{\dg}(F A, (F A) \oplus (G A) )_0.
    \]

    For any \( f \in \Cc^{\op}(A, B)_n, \) we have
    \begin{align*}
        ((F \oplus G) f) \circ \tuple*{ \iota_{(F A)_i} }_{i \in \Zb} &= ((F f)_i \oplus (G f)_i)_{i \in \Zb} \circ \tuple*{ \iota_{(F A)_i} }_{i \in \Zb} \\
        &= \tuple*{ ((F f)_i \oplus (G f)_i) \circ \iota_{(F A)_i} }_{i \in \Zb} \\
        &= \tuple*{ \iota_{(F B)_{i + n}} \circ (F f)_i  }_{i \in \Zb} \\
        &= \tuple*{ \iota_{(F B)_{i + n}} }_{i \in \Zb} \circ (F f) \\
        &= (-1)^{|f| 0}\tuple*{ \iota_{(F B)_{i + n}} }_{i \in \Zb} \circ (F f), \\
    \end{align*}
    which implies \( \iota_F \in \dgM(F, F \oplus G)_0 \), and similarly for \( \iota_G \).

    In particular, \( \pi_F \) and \( \iota_F \) also has the following two properties,
    \begin{align*}
        \pi_F \circ \iota_F &= \tuple*{ \tuple*{ \pi_{(F A)_i} }_{i \in \Zb} \circ \tuple*{ \iota_{(F A)_i} }_{i \in \Zb} }_{A \in \Cc} \\
        &= \tuple*{ \pi_{(F A)_i} \circ \iota_{(F A)_i} }_{i \in \Zb, A \in \Cc} \\
        &= \tuple*{ \Id_{(F A)_i} }_{i \in \Zb, A \in \Cc} \\
        &= u_0^F(1),
    \end{align*}
    and
    \begin{align*}
        \iota_F \circ \pi_F + \iota_G \circ \pi_G &= \tuple*{\iota_{(F A)_i} \circ \pi_{(F A)_i} + \iota_{(G A)_i} \circ \pi_{(G A)_i} }_{i \in \Zb, A \in \Cc} \\
        &= \tuple*{ \Id_{(F A)_i \oplus (G A)_i} }_{i \in \Zb, A \in \Cc} \\
        &= \tuple*{ \Id_{((F \oplus G) A)_i} }_{i \in \Zb, A \in \Cc} \\
        &= \tuple*{ u_0^{(F \oplus G) A} (1)}_{A \in \Cc} \\
        &= u_0^{F \oplus G} (1).
    \end{align*}
\end{remark}

Similarly to what was done for \( \C_{\dg} \), we also use the notation of matrices and direct sum for DG-morphisms in \( \prod_{A \in \Cc} \C_{\dg} (F A, G A) \) for any DG-module \( F \) and \( G \).
\begin{notation}
    We will consider the direct sum, as well as matrices of DG-morphisms in \( \prod_{A \in \Cc} \C_{\dg} (F A, G A) \) for any DG-module \( F \) and \( G \) in the obvious way.

    For example, for a 2×2 matrix, we use the following notation:

    For \( n \in \Zb \),
    \begin{align*}
        \eta &\in \prod_{A \in \Cc} \C_{\dg}(F A, H A)_n, \\
        \mu &\in \prod_{A \in \Cc} \C_{\dg}(G A, K A)_n, \\
        \alpha &\in \prod_{A \in \Cc} \C_{\dg}(F A, K A)_n, \\
        \intertext{and}
        \beta &\in \prod_{A \in \Cc} \C_{\dg}(G A, H A)_n,
    \end{align*}
    denote the DG-morphism
    \[
        \begin{pmatrix}
            \eta_A & \beta_A \\
            \alpha_A & \mu_A
        \end{pmatrix}_{A \in \Cc}
        \in
        \prod_{A \in \Cc} \C_{\dg}(F \oplus G, H \oplus K)_n
    \]
    as
    \[
        \begin{pmatrix}
            \eta & \beta \\
            \alpha & \mu
        \end{pmatrix}.
    \]
    Similarly to above, if \( \alpha = 0 = \beta \), we write the above morphism as \( \eta \oplus \mu \).
\end{notation}

Similar to what we observed with \( \C_{\dg} \), we also get that the above notation, along with \( \pi_F \) and \( \iota_F \), when considering composition of DG-morphisms, have all the expected properties of usual composition of matrices, with \( u_0^F(1) \) as the ``identity.'' Two examples of this are the final two properties mentioned in \autoref{rem:dgm_pi_iota}.

Additionally, if every DG-morphism in a matrix is in \( \dgM \), then the matrix is also in \( \dgM \) since the matrix can be described as a sum of morphisms in \( \dgM \).

The following lemma is used to prove that \( H^0(\dgM) \) is additive.

\begin{lemma}
    \label{lem:dgm_pi_iota_cycles}
    Let \( F, G \in \dgM \).
    
    Then both \( \pi_F \) and \( \iota_F \) are cycles.
\end{lemma}
\begin{proof}
    First, consider \( \pi_F \),
    \begin{align*}
        d_0^{\dgM(F \oplus G, F)} ( \pi_F ) &= d_F \circ \pi_F - \pi_F \circ d_{F \oplus G} \\
        &= \tuple*{ d_i^{F A} \circ \pi_i^{F A} - \pi_{i + 1}^{F A} \circ d_i^{(F A) \oplus (G A)} }_{i \in \Zb, A \in \Cc} \\
        &= \tuple*{
            \begin{pmatrix}
                d_i^{F A} & 0
            \end{pmatrix}
            -
            \begin{pmatrix}
                1 & 0
            \end{pmatrix}
            \begin{pmatrix}
                d_i^{F A} & 0 \\
                0 & d_i^{G A}
            \end{pmatrix}
        }_{i \in \Zb, A \in \Cc} \\
        &= \tuple*{
            \begin{pmatrix}
                d_i^{F A} & 0
            \end{pmatrix}
            -
            \begin{pmatrix}
                d_i^{F A} & 0
            \end{pmatrix}
        }_{i \in \Zb, A \in \Cc} \\
        &= 0.
    \end{align*}

    Second, consider \( \iota_F \),
    \begin{align*}
        d_0^{\dgM(F, F \oplus G)} ( \iota_F ) &= d_{F \oplus G} \circ \iota_F - \iota_F \circ d_F \\
        &= \tuple*{ d_i^{(F A) \oplus (G A)} \circ \iota_i^{F A} - \iota_{i + 1}^{F A} \circ d_i^{F A} }_{i \in \Zb, A \in \Cc} \\
        &= \tuple*{
            \begin{pmatrix}
                d_i^{F A} \\
                0
            \end{pmatrix}
            -
            \begin{pmatrix}
                d_i^{F A} \\
                0
            \end{pmatrix}
        }_{i \in \Zb, A \in \Cc} \\
        &= 0. \qedhere
    \end{align*}
\end{proof}

In order for \( H^0(\dgM) \) to have a triangulation, it has to be additive.

\begin{lemma}
    \( H^0(\dgM) \) is an additive category.
\end{lemma}
\begin{proof}
    For any \( F, G \in \dgM \), we have that \( H^0(\dgM)(F, G) \) is an \( R \)-module, and therefore an abelian category. Since composition in \( \dgM \) is degree-wise \( R \)-bilinear, composition in \( H^0(\dgM) \) is also bilinear. This implies that \( H^0(\dgM) \) is pre-additive.

    It remains to prove that \( H^0(\dgM) \) has a finite biproduct:

    Let \( F \oplus G \) be as in \autoref{def:dgm_biproduct}.
    
    By \cite[p.\ 250]{Mac_Lane_1995}, it suffices to prove that the following equalities hold,
    \[
        [\pi_F \circ \iota_F] = [\Id_F], \quad [\pi_G \circ \iota_G] = [\Id_G], \quad \text{and} \quad [\iota_F \circ \pi_F + \iota_G \circ \pi_G] = [\Id_{F \oplus G}].
    \]

    By \autoref{lem:dgm_pi_iota_cycles}, \( \pi_F, \pi_G, \iota_F, \) and \( \iota_F \) are cycles, and so they are all representatives of morphisms in \( H^0(\dgM) \).

    Remember from \autoref{def:0_th_cohomology_of_dg_cat}, we have that for any \( F \in \dgM \), \( \Id_F := [u_0^F ( 1 )] \).

    By the final two properties in \autoref{rem:dgm_pi_iota}, we have that
    \[
        [\pi_F \circ \iota_F] = [u_0^F (1)] = \Id_F,
    \]
    and similarly for \( \Id_G \), as well as
    \[
        [\iota_F \circ \pi_F + \iota_G \circ \pi_G] = [u_0^{F \oplus G} (1)] = \Id_{F \oplus G}.
    \]

    Therefore, \( H^0(\dgM) \) is additive.
\end{proof}

In order to construct the cone, which is a part of the triangulation of \( \dgM \), we need some more tools. The first is some notation that will be helpful for the results.

\begin{notation}
    Let \( n \in \Zb \), and let \( A \) be a chain complex.
    
    Then let \( A_{\bullet+n} \) denote the \( n \)-shifted chain complex of \( A \) but \emph{without} the \( (-1)^n \) sign in front of the differential.
\end{notation}

With the above notation in mind, we get an equality that is essential in creating the cone, as well as showing that Toda brackets equal Massey products.

\begin{lemma}
    \label{lem:shift_one_component_inner_product_chain_complex}
    Let \( A, B \in \C \).

    Then there is a chain isomorphism
    \begin{align*}
        \phi: \class*{\Sigma^{-n} A, B} &\to \class*{A, B}_{\bullet+n} \\
        f &\mapsto f,
    \end{align*}
    i.e.,
    \begin{center}
        \begin{tikzpicture}
            \diagram{m}{1cm}{1cm} {
                \class*{\Sigma^{-n} A, B}: \&[-0.9cm] \cdots \& \class*{\Sigma^{-n} A, B}_{-1} \& \class*{\Sigma^{-n} A, B}_0 \& \class*{\Sigma^{-n} A, B}_1 \& \cdots \\
                \class*{A, B}_{\bullet+n}: \& \cdots \& \class*{A, B}_{n - 1} \& \class*{A, B}_n \& \class*{A, B}_{n + 1} \& \cdots \\
            };

            \draw[math]
                (m-1-1) edge[equality] (m-2-1)
                (m-1-2) edge (m-1-3)
                (m-1-3) edge node {d_{-1}^{\class*{\Sigma^{-n} A, B}}} (m-1-4)
                    edge[equality] (m-2-3)
                (m-1-4) edge node {d_0^{\class*{\Sigma^{-n} A, B}}} (m-1-5)
                    edge[equality] (m-2-4)
                (m-1-5) edge (m-1-6)
                    edge[equality] (m-2-5)

                (m-2-2) edge (m-2-3)
                (m-2-3) edge node {d_{n-1}^{\class*{A, B}}} (m-2-4)
                (m-2-4) edge node {d_n^{\class*{A, B}}} (m-2-5)
                (m-2-5) edge (m-2-6);
        \end{tikzpicture}
    \end{center}
\end{lemma}
\begin{proof}
    For any \( i \in \Zb \), there are two parts that need to be proved.

    First, we need to show that \( \class*{\Sigma^{-n} A, B}_i = \class*{A, B}_{n + i} \), and second, need to show that \( d_i^{\class*{\Sigma^{-n} A, B}} = d_{n + i}^{\class*{A, B}} \).

    We start with the first part.

    Expanding the definitions and using \autoref{not:prod_coprod_no_order}, we have
    \begin{align*}
        \class*{\Sigma^{-n} A, B}_i &= \prod_{j \in \Zb} \Mod(R)((\Sigma^{-n} A)_j, B_{j + i}) \\
        &= \prod_{j \in \Zb} \Mod(R)(A_{j - n}, B_{j + i}) \\
        &= \prod_{j \in \Zb} \Mod(R)(A_j, B_{j + i + n}) \\
        &= [A, B]_{i + n}
    \end{align*}

    For the second part, let \( \tuple*{f_j}_{j \in \Zb} \in [A, B]_{n + i} \), with \( f_j: A_j \to B_{j + n + i} \). Thus,
    \begin{align*}
        d_i^{\class*{\Sigma^{-n} A, B}}\tuple*{f_j}_{j \in \Zb} &= \tuple*{d_{n + i + j}^B \circ f_j - (-1)^i f_{j + 1} \circ d_{j + n}^{\Sigma^{-n} A}}_{j \in \Zb} \\
        &= \tuple*{d_{n + i + j}^B \circ f_j - (-1)^{n + i} f_{j + 1} \circ d_j^A}_{j \in \Zb} \\
        &= d_{n + i}^{\class*{A, B}}(f_j)_{j \in \Zb}.
    \end{align*}
    Therefore, \( \class*{\Sigma^{-n} A, B} \) is canonically isomorphic to \( \class*{A, B}_{\bullet+n} \) with an ``identity'' morphism.
\end{proof}

The above lemma gives the following result, which is what we will use in the definition of Massey products on \( H^0(\dgM) \).

\begin{lemma}
    \label{lem:dgmod_shift_eq_plus}
    Let \( F, G \in \dgM \).

    Then there is a chain isomorphism
    \begin{align*}
        \phi: \dgM(\Sigma^{-i} F, G) &\to \dgM(F, G)_{\bullet + i} \\
        \eta &\mapsto \eta.
    \end{align*}
\end{lemma}
\begin{proof}
    Let \( \phi_A \) be the canonical isomorphism from \autoref{lem:shift_one_component_inner_product_chain_complex}, where
    \[ 
    \phi_A: [\Sigma^{-i} F A, G A] \to [F A, G A]_{\bullet+i},
    \]
    and let \( \phi := \prod_{A \in \Cc} \phi_A \). Then
    \[
        \prod_{A \in \Cc^{\op}} \C_{\dg}(\Sigma^{-i} F A, G A) \stackrel{\phi}{\cong} \prod_{A \in \Cc^{\op}} \C_{\dg}(F A, G A)_{\bullet+i} = \tuple*{\prod_{A \in \Cc^{\op}} \C_{\dg}(F A, G A)}_{\bullet+i}.
    \]
    It remains to show that for any \( n \in \Zb \),
    \[
        \eta \in \tuple*{\prod_{A \in \Cc^{\op}} \C_{\dg}(\Sigma^{-i} F A, G A)}_n
    \]
    satisfies the properties of \( \dgM(\Sigma^{-i} F, G)_n \) if and only if it satisfies the properties of \( \dgM(F, G)_{n + i} \).

    This follows, since for any \( f \in \Cc^{\op}(A, B)_j \) we have
    \[
        (-1)^{nj} \eta_B \circ (\Sigma^{-i} F f) = (-1)^{nj} \eta_B \circ ((-1)^{ji}F f) = (-1)^{(n + i)j}\eta_B \circ (F f),
    \]
    and so
    \[
        (G f) \circ \eta_A = (-1)^{nj} \eta_B \circ (\Sigma^{-i} F f) = (-1)^{(n + i)j}\eta_B \circ (F f). \qedhere
    \]
\end{proof}

The fact that a DG-morphism in \( \C_{\dg} \) and \( \dgM \) can be considered as a DG-morphism with another domain or codomain might create ambiguity in the notation, especially when considering  composition of morphism. However, this turns out to not be a problem.

% TODO: Blir ikkje eigenskapen at \phi er ein kjedeisomorfi brukt? Er ikkje det viktig?
% TODO: Korleis gir dette meining med hensyn på ulike "eigenskapar" som DG-morfiar kan ha? Som i beviset av at toda=massey på h^0 dgM, så er d_X3 circ t + t +circ d_X1 framleis lik d_dgm(X_1, X_3)(t) sjølv om det vart gjort i dgm(X_1, X_3).
% TODO: Sitera til denne når ein gjer "consider..." argument?
\begin{remark}
    \label{rem:dgm_c_dg_super_degree_shift}
    For any \( i \in \Zb \), we have already shown that \( \dgM(F, G)_i = \dgM(\Sigma F, \Sigma G)_i \), and we have also shown that \( \dgM(\Sigma F, G)_i = \dgM(F, G)_{i + 1} \). Combining these two properties we get that for any \( \eta \in \dgM(F, G)_i \), and \( n, k \in \Zb \) we get that \( \eta \) can be considered as a DG-morphism in \( \dgM(\Sigma^n F, \Sigma^k G)_{k - n + i} \).

    Furthermore, since composition in \( \dgM \) essentially boils down to the same degree-wise composition of morphisms in \( \Mod(R) \), two composable morphisms \( \eta \) and \( \mu \) will have the same composition regardless of ``where'' we compose them.

    This makes the statement ``consider \( \mu \circ \eta \) as an element in \( \dgM(\Sigma^n F, \Sigma^k H)_{k - n + |\eta| + |\mu|} \)'' well-defined without specifying what composition morphism we use.

    The same arguments are also true for \( \C_{\dg} \), as well as composition of DG-morphisms in \( \prod_{A \in \Cc} \C_{\dg}(F A, G B) \) for any DG-module \( F \) and \( G \).
\end{remark}

The idea that composition of DG-morphisms in ``equal'' modules is independent of where we do the composing has been implicitly used in multiple equations already, and is a major reason why the equations we are working with can be written as nicely as they do. If composition did not behave this nicely, the convention in \autoref{not:suppress_canonical_isomorphisms} would have been very confusing, but now it makes equations cleaner.

The following is the definition for a cone in \( \dgM \).

\begin{definition}[Cone in \( \dgM \)]
    \label{def:dgm_cone}
    Let \( F, G \in \dgM \), and let \( \eta \in Z^0(\dgM(F, G)) \). Then we can define the following DG-module, denoted \( C_{\eta} \) as follows
    \begin{itemize}
        \item {
            Consider \( \eta \in \dgM(F, G)_0 \) as an element of \( \dgM(\Sigma F, G)_1 \) by using \autoref{lem:dgmod_shift_eq_plus}.

            For any \( A \in \Cc \), define \( C_{\eta}(A) \) to be the chain complex with the underlying \( R \)-modules of \( (\Sigma F A) \oplus (G A) \), but with the differentials
            \[
                d_{C_{\eta}A} :=
                \begin{pmatrix}
                    d_{\Sigma F A} & 0 \\
                    \eta_A & d_{GA}
                \end{pmatrix}
                =
                \begin{pmatrix}
                    -d_{F A} & 0 \\
                    \eta_A & d_{GA}
                \end{pmatrix}.
            \]
        }
        \item {
            For any \( f \in \Cc^{\op}(A, B)_n \), let
            % TODO: Rar C_eta
            \begin{align*}
                {C_{\eta}}_n^{A, B}: \Cc^{\op}(A, B)_n &\to \C_{\dg}(C_{\eta} A, C_{\eta} B)_n \\
                f &\mapsto ((\Sigma F) \oplus G) f = 
                \begin{pmatrix}
                    \Sigma F f & 0 \\
                    0 & G f
                \end{pmatrix}.
            \end{align*}
        }
    \end{itemize}
\end{definition}

We can write the differential of \( C_{\eta} \) as
\[
    d_{C_{\eta}} =
    \begin{pmatrix}
        -d_F & 0 \\
        \eta & d_G
    \end{pmatrix},
\]
where we consider the \( \eta \) in the matrix as an element of \( \dgM(\Sigma F, G)_1 \).

The definition of \( c_{\eta} \) has the following property that makes it easier to work with.

\begin{remark}
    \label{rem:dgm_c_eta_similar_to_sigma_f_plus_g}
    Since the only property that differentiates \( ((\Sigma F) \oplus G) \) and \( C_{\eta} \), is the differentials of the chain complexes
    \[
        ((\Sigma F) \oplus G) A \quad \text{ and } \quad C_{\eta} A,
    \]
    every result that does not rely on the differentials hold for both \( (\Sigma F) \oplus G \) and \( C_{\eta} \).
\end{remark}

We have that \( C_{\eta} \) is a well-defined DG-functor by the following remark.
\begin{remark}
    In order to check if \( C_{\eta} \) in \autoref{def:dgm_cone} is a DG-functor, we have to check the three properties:
    \begin{itemize}
        \item {
            For any \( A, B \in \Cc \), we want to show that
            \[
                {C_{\eta}}_{A, B}: \Cc^{\op}(A, B) \to \C_{\dg}(C_{\eta} A, C_{\eta} B)
            \]
            is a chain morphism, i.e., we want to show that for any \( f \in \Cc^{\op}(A, B)_n \),
            \[
                d_n^{[C_{\eta} A, C_{\eta} B]} (C_{\eta} f) = C_{\eta} d_n^{\Cc^{\op}(A, B)}(f).
            \]

            We have
            \begin{align*}
                &d_n^{[C_{\eta} A, C_{\eta} B]} (C_{\eta} f) \\
                &= d_{C_{\eta} B} \circ
                \begin{pmatrix}
                    \Sigma F f & 0 \\
                    0 & G f
                \end{pmatrix}
                - (-1)^n
                \begin{pmatrix}
                    \Sigma F f & 0 \\
                    0 & G f
                \end{pmatrix}
                \circ d_{C_{\eta} A} \\
                &=
                \begin{pmatrix}
                    -(-1)^n d_{F B} \circ F f & 0 \\
                    (-1)^n \eta_B \circ F f & d_{G B} \circ G f
                \end{pmatrix} - (-1)^n
                \begin{pmatrix}
                    -(-1)^n (F f) \circ d_{F A} & 0 \\
                    (G f) \circ \eta_A & (G f) \circ d_{G A}
                \end{pmatrix} \\
                &=
                \begin{pmatrix}
                    (-1)^{n + 1}\tuple*{d_{F B} \circ F f - (-1)^n (F f) \circ d_{F A}} & 0 \\
                    (-1)^n\tuple*{\eta_B \circ F f - (G f) \circ \eta_A} & d_{G B} \circ G f - (-1)^n (G f) \circ d_{G A}
                \end{pmatrix} \\
                &=
                \begin{pmatrix}
                    \Sigma d_n^{[F A, F B]}(F f) & 0 \\
                    (-1)^{n + 1} 0 & d_n^{[G A, G B]}(G f)
                \end{pmatrix} \\
                &=
                \begin{pmatrix}
                    \Sigma F d_n^{\Cc^{\op}(A, B)}(f) & 0 \\
                    0 & G d_n^{\Cc^{\op}(A, B)}(f)
                \end{pmatrix} \\
                &= \C_{\eta} d_n^{\Cc^{\op}(A, B)}(f).
            \end{align*}
        }
        \item {
            Since the degree-wise composition does not rely upon the differentials in any way, by \autoref{rem:dgm_c_eta_similar_to_sigma_f_plus_g},
            the composition axiom diagram commutes degree-wise. The diagram commutes as chain morphisms as \( {C_{\eta}}_{A, B} \) is a chain morphism.
        }
        \item {
            By the same argument as above, we get that the unit axiom is true for \( C_{\eta} \) by the same arguments as for \( (\Sigma F) \oplus G \).
        }
    \end{itemize}
\end{remark}

Another consequence of \( ((\Sigma F) \oplus G) \) and \( C_{\eta} \) being similar is the following.

\begin{remark}
    \label{rem:dgm_different_dg_morphisms_same_space_give_degree-wise_same_morphisms}
    By \autoref{rem:dgm_c_eta_similar_to_sigma_f_plus_g}, for any \( H \in \dgM \) and any \( i \in \Zb \), we have that \( \dgM(H, (\Sigma F) \oplus G)_i = \dgM(H, C_{\eta})_i \) as well as \( \dgM((\Sigma F) \oplus G, H)_i = \dgM(C_{\eta}, H)_i \), because the differential is not a part of the definition of being a DG-morphism for some \( i \).

    Composition of DG-morphisms to and from \( C_{\eta} \) is also preserved.
\end{remark}

By using the above remark, we can consider the projection as well as the embedding morphisms as morphisms to or from \( C_{\eta} \) instead. We also want to verify some properties for future proofs.

\begin{remark}
    \label{rem:dgm_differentials_of_inclusions_and_projections_of_cone}
    Let \( \iota_G \in \dgM(G, (\Sigma F) \oplus G)_0 \) and \( \pi_{\Sigma F} \in \dgM((\Sigma F) \oplus G, \Sigma F)_0 \) be as in \autoref{rem:dgm_pi_iota}.

    By \autoref{rem:dgm_different_dg_morphisms_same_space_give_degree-wise_same_morphisms} we can consider the above DG-morphisms as DG-morphisms to and from \( C_{\eta} \) instead of \( (\Sigma F) \oplus G \), respectively.

    Additionally, \( \iota_G \) and \( \pi_{\Sigma F} \) are cycles as
    \begin{align*}
        d_0^{\dgM(G, C_{\eta})}(\iota_G) &= d_{C_{\eta}} \circ \iota_G - \iota_G \circ d_G \\
        &=
        \begin{pmatrix}
            -d_F & 0 \\
            \eta & d_G
        \end{pmatrix}
        \begin{pmatrix}
            0 \\
            1
        \end{pmatrix}
        -
        \begin{pmatrix}
            0 \\
            1
        \end{pmatrix}
        \circ d_G \\
        &=
        \begin{pmatrix}
            0 \\
            d_G
        \end{pmatrix}
        -
        \begin{pmatrix}
            0 \\
            d_G
        \end{pmatrix}
        = 0
    \end{align*}
    and
    \begin{align*}
        d_0^{\dgM(C_{\eta}, \Sigma F)}(\pi_{\Sigma F}) &= d_{\Sigma F} \circ \pi_{\Sigma F} - \pi_{\Sigma F} \circ d_{C_{\eta}} \\
        &= -d_F \circ
        \begin{pmatrix}
            1 & 0
        \end{pmatrix}
        -
        \begin{pmatrix}
            1 & 0
        \end{pmatrix}
        \begin{pmatrix}
            -d_F & 0 \\
            \eta & d_G
        \end{pmatrix} \\
        &=
        \begin{pmatrix}
            -d_F & 0
        \end{pmatrix}
        -
        \begin{pmatrix}
            -d_F & 0
        \end{pmatrix}
        = 0.
    \end{align*}

    The differentials of \( \iota_{\Sigma F} \in \dgM(\Sigma F, C_{\eta})_0 \) and \( \pi_G \in \dgM(C_{\eta}, G)_0 \) are
    \begin{align*}
        d_0^{\dgM(\Sigma F, C_{\eta})}(\iota_{\Sigma F}) &= d_{C_{\eta}} \circ \iota_{\Sigma F} - \iota_{\Sigma F} \circ d_{\Sigma F} \\
        &=
        \begin{pmatrix}
            -d_F & 0 \\
            \eta & d_G
        \end{pmatrix}
        \begin{pmatrix}
            1 \\
            0
        \end{pmatrix}
        -
        \begin{pmatrix}
            1 \\
            0
        \end{pmatrix}
        \circ (- d_F) \\
        &=
        \begin{pmatrix}
            -d_F \\
            \eta
        \end{pmatrix}
        +
        \begin{pmatrix}
            d_F \\
            0
        \end{pmatrix} \\
        &=
        \begin{pmatrix}
            0 \\
            \eta
        \end{pmatrix}
        = \iota_G \circ \eta
    \end{align*}
    and
    \begin{align*}
        d_0^{\dgM(C_{\eta}, G)}(\pi_G) &= d_G \circ \pi_G - \pi_G \circ d_{C_{\eta}} \\
        &= d_G \circ
        \begin{pmatrix}
            0 & 1
        \end{pmatrix}
        -
        \begin{pmatrix}
            0 & 1
        \end{pmatrix}
        \begin{pmatrix}
            -d_F & 0 \\
            \eta & d_G
        \end{pmatrix} \\
        &=
        \begin{pmatrix}
            0 & d_G
        \end{pmatrix}
        -
        \begin{pmatrix}
            \eta & d_G
        \end{pmatrix} \\
        &= -
        \begin{pmatrix}
            \eta & 0
        \end{pmatrix}
        = - \eta \circ \pi_{\Sigma F}.
    \end{align*}
\end{remark}

The triangulation of \( H^0(\dgM) \) is as follows.
\begin{definition}
    \label{def:delta_H_0_dgmod}
    Let \( \iota_G \) and \( \pi_{\Sigma F} \) be as in the remark above.

    Then \( \Delta \) is the class of triangles isomorphic to any triangle in \( H^0(\dgM) \) of the form
    \begin{center}
        \begin{tikzpicture}
            \diagram{m}{1cm}{1cm} {
                F \& G \& C_{\eta} \& \Sigma F. \\
            };

            \draw[math]
                (m-1-1) edge node {[\eta]} (m-1-2)
                (m-1-2) edge node {[\iota_G]} (m-1-3)
                (m-1-3) edge node {[\pi_{\Sigma F}]} (m-1-4);
        \end{tikzpicture}
    \end{center}
\end{definition}

Together, this implies that \( H^0(\dgM) \) is triangulated.
\begin{theorem}
    Let \( \Sigma \) be the shift functor as defined in \autoref{def:sigma_dgmod}, and let \( \Delta \) be as defined in \autoref{def:delta_H_0_dgmod}.

    Then \( \tuple*{H^0(\dgM), \Sigma, \Delta} \) is a triangulated category.
\end{theorem}
For a proof of the above theorem, see \cite[p.\ 31]{Jasso-Muro_2023} for the general idea of a proof using the connection of \( H^0(\dgM) \) to a stable Frobenius category, or see \cite[p.\ 97]{Bondal--Kapranov_1991} for a proof based on twisted complexes.

The next step is to define what an algebraic triangulated category is by relating triangulated categories to \( H^0(\dgM) \). We will accomplish this by using the DG-Yoneda embedding from \cite[Corollary 6.3.6]{Borceux_1994}, which is defined as follows.
\begin{definition}[DG-Yoneda embedding]
    \label{def:DG_Yoneda_embedding}
    Let \( \Cc \) be a small DG-category.
    
    Then let \( \mathbf{h}: \Cc \to \dgM \) be the DG-functor defined as follows:
    \begin{itemize}
        \item {
            For \( A \in \Cc \), let
            \[
                \mathbf{h} A := \Cc(?, A).
            \]
        }
        \item {
            For \( f \in \Cc(A, B)_n \), let
            \[
                \mathbf{h}_n^{A, B} (f) := (f)_*,
            \]
            where \( (f)_* \) is post composition with \( f \) as a DG-morphism.
        }
    \end{itemize}

    This DG-functor is called the \emph{DG-Yoneda embedding of \( \Cc \)}.
\end{definition}

Here \( \Cc \) needs to be small, since \( \dgM := \dgMod_{\dg}(\Cc^{\op}, \C_{\dg}) \) is only defined for small \( \Cc \).

For a proof of why the DG-Yoneda embedding is a well-defined DG-functor, see \cite[Corollary 6.3.6]{Borceux_1994}.

% Borceux seie at DG-yoneda isomorfien er ``DG-naturleg'' kva skal det bety, og er det ein eigenskap me treng? Er Yoneda OP?
\begin{remark}
    \label{rem:dg_yoneda_embedding_fully_faithful}
    Since by the DG-Yoneda lemma (\cite[Corollary 6.3.5]{Borceux_1994}), it follows that
    \[
        \mathbf{h}_{A, B}: \Cc(A, B) \to \dgM(\Cc(-, A), \Cc(-, B))
    \]
    is an isomorphism, and since \( \mathbf{h} \) is a DG-functor, we have that we can consider any DG-category as a sort of full ``DG-subcategory'' (which is defined as expected) of the DG-category of DG-modules.
\end{remark}

The following theorem shows that taking \( H^0(\mathbf{h}) \) yields a fully faithful functor from \( H^0(\Cc) \) to \( H^0(\dgM) \).

\begin{lemma}
    Let \( \Cc \) be a small DG-category, and let \( \mathbf{h} \) be as in \autoref{def:DG_Yoneda_embedding}.

    Then the functor
    \[
        H^0(\mathbf{h}): H^0(\Cc) \to H^0(\dgM)
    \]
    is fully faithful.
\end{lemma}
\begin{proof}
    By \autoref{rem:dg_yoneda_embedding_fully_faithful}, we have that for any \( A, B \in \Cc \), \( \mathbf{h}_{A, B} \) is a chain isomorphism, which implies \( H^0(\mathbf{h})_{A, B} \) is an isomorphism.
\end{proof}

We are now ready to define what a ``pre-triangulated DG-category'' is, which is the final important piece in this definition of algebraic triangulated categories. It must not be confused with a pre-triangulated (non-DG) category which is a category that satisfies the triangulated axioms {\bf (TR1)}, {\bf (TR2)}, and {\bf (TR3)}, but not {\bf (TR4)}.
\begin{definition}[Pre-triangulated DG-category]
    \label{def:pre-tri_dg_cat}
    Let \( \Cc \) be a small DG-category.

    Then \( \Cc \) is called a \emph{pre-triangulated DG-category} if \( H^0(\Cc) \) is triangulated, and \( H^0(\mathbf{h}): H^0(\Cc) \to H^0(\dgM) \) is a triangulated functor, such that the image of \( H^0(\mathbf{h}) \) is a triangulated subcategory of \( H^0(\dgM) \).
\end{definition}
In particular, if \( \Cc \) is a pre-triangulated DG-category, then due to \( H^0(\mathbf{h}) \) being fully faithful and mapping objects injectively, it is a triangulated \emph{automorphism} onto its image, which is a full triangulated subcategory of \( H^0(\dgM) \).

The above definition is equivalent to \cite[Definition 3.1.1]{Jasso-Muro_2023}.

Finally, we can define what an algebraic triangulated category is.
\begin{definition}[Algebraic triangulated category]
    \label{def:alg_tri_cat}
    Let \( \Tc \) be a triangulated category.

    Then \( \Tc \) is called an \emph{algebraic triangulated category} if there exists a pre-triangulated DG-category, \( \Cc \), called a \emph{DG-enhancement of \( \Tc \)}, such that \( H^0(\Cc) \) is triangulated equivalent to \( \Tc \).
\end{definition}

There are many equivalent, or slightly different definitions of algebraic triangulated categories. The reason we use the above definition is that it closely relates the algebraic triangulated category to the category of DG-modules. This in turn allows us to use the properties of DG-modules in order to make it possible for Toda brackets to equal Massey products.