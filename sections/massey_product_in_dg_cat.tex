\begin{notation}
    Let \( \Cc \) be any additive category.
    
    Then let \( \C \tuple{\Cc} \) denote the category of chain complexes of objects in \( \Cc \).

    Furthermore let the differential in these chain complexes have \emph{ascending} order, i.e., for \( M_i, M_{i+1} \in \Cc \) which are adjacent objects in a chain complex \( M \in \C \tuple{\Cc} \), the differential would be
    \[
        d_i : M_i \to M_{i + 1}.
    \]
\end{notation}

% TODO: Obvious?
\begin{notation}
    In this section, there will be a lot of products and coproducts of modules mentioned, and so a small note on the notation of elements, and the reasoning behind the notation could be useful.

    Let \( R \) be a commutative ring with identity. Let \( A_i \in \Mod(R) \) and let
    \[
        \iota_i: A_i \to \coprod_{i \in \Zb} A_i
    \]
    denote the canonical split monomorphism by the universal property of the coproduct in \( \Mod(R) \).

    Then for any \( a_i \in A_i \), the element
    \[
        \iota_i(a_i) \in \coprod_{i \in \Zb} A_i
    \]
    is just denoted as
    \[
        a_i \in \coprod_{i \in \Zb} A_i.
    \]
    
    The reasoning for the above notation is twofold. Firstly, it reduces notation while not being ambigous. Secondly, one never talks about a general element of a coproduct. Almost always when talking about what a morphism does to an element of the coproduct, it is what happens to the \( \iota_i(a_i) \)'s, which is a consequence of the universal property of the coproduct.

    In addition, let 
    \[
        \pi_i: \prod_{i \in \Zb} A_i \to A_i
    \]
    be the universal split epimorphism by the universal property of the product in \( \Mod(R) \).
    
    Then for any element \( a \in \prod_{i \in \Zb} A_i \), denote
    \[
        a = \tuple{a_i}_{i \in \Zb} \in \prod_{i \in \Zb} A_i
    \]
    where \( a_i := \pi_i(a) \in A_i \).
    
    The reasoning behind this notation is because the product in \( \Mod(R) \) is the direct product, and because the universal property of the product. That is because it makes it so that when talking about morphisms into the product, the morphism is fully defined by what it does to each degree in \( \prod_{i \in \Zb} A_i \), which is easily shown using the above notation.  
\end{notation}

% Section wide TODO's: Unit morphism in enriched category, notation for elements in a coproduct.

\subsection{Definition of a DG-category}
In this thesis the definition of a DG-Category is based on enriched category theory, as it is a modern approach, and by the opinion of the author it is also the most elegeant approach.

\begin{notation}
    Let \( \Cc \) be any additive category.
    
    Then let \( \C \tuple{\Cc} \) denote the category of chain complexes of objects in \( \Cc \).

    Furthermore let the differential in these chain complexes have \emph{ascending} order. I.e. for \( M_i, M_{i+1} \in \Cc \) which are adjacent objects in a chain complex \( M \in \C \tuple{\Cc} \), the differential would be
    \[
        d_i : M_i \to M_{i + 1}.
    \]
\end{notation}

\begin{definition}[Tensor product of chain complexes over \( \Mod(R) \)]
    \label{def:tensor_product_of_chain_complexes_over_Mod(R)}
    Let \( R \) be a commutative ring with identity. Furthermore let \( A, B \in \C \tuple{\Mod(R)} \).

    Then define the modules
    \[
        (A \otimes B)_n := \coprod_{p + q = n} A_p \otimes B_q
    \]
    which are a part of the chain complex
    \begin{center}
        \begin{tikzpicture}
            \diagram{m}{1cm}{1cm} {
                A \otimes B: \\
            };
        \end{tikzpicture}
        %
        \begin{tikzpicture}
            \diagram{m}{1cm}{1cm} {
                \cdots \& (A \otimes B)_{-1} \& (A \otimes B)_0 \& (A \otimes B)_1 \& \cdots \\
            };

            \draw[math]
                (m-1-1) edge (m-1-2)
                (m-1-2) edge node {d_{-1}} (m-1-3)
                (m-1-3) edge node {d_0} (m-1-4)
                (m-1-4) edge (m-1-5);
        \end{tikzpicture}
    \end{center}

    Where the differentials, \( d_n \), are defined as follows:
    
    Let \( i, j \in \Zb \) with \( i + j = n \), and let
    \[
        \iota_{i, j}: A_i \otimes B_j \hookrightarrow \tuple{A \otimes B}_n
    \]
    be the canonical split-monomorphism. Furthermore let \( a \otimes b \in A_i \otimes B_j \) be an elementary tensor.

    Then the differential is uniquely defined as follows
    \[
        d_n(\iota_{i, j}(a \otimes b)) := \iota_{i + 1, j}\tuple{d_{A, i}(a) \otimes b} + (-1)^{i} \iota_{i, j + 1}\tuple{a \otimes d_{B, j}(b)}.
    \]

    This is called the \emph{tensor product of chain complexes over \( \Mod(R) \)}.
\end{definition}

\begin{remark}
    The definition of the differentials in \autoref{def:tensor_product_of_chain_complexes_over_Mod(R)} is well defined and unique by the following argument.

    Look at the following diagram
    \begin{equation}
        \label{tikz:differential_of_tensor_product_of_chain_complexes_over_Mod(R)}
        \begin{tikzpicture}
            \diagram{m}{2cm}{2cm} {
                A_i \otimes B_j \& \coprod\limits_{p + q = n} A_p \otimes B_q \\
                A_i \times B_j \& \coprod\limits_{p' + q' = n + 1} A_{p'} \otimes B_{q'} \\
            };

            \draw[math]
                (m-1-1) edge[hook] node {\iota_{i, j}} (m-1-2)
                    edge[dashed] node {\alpha_{i, j}} (m-2-2)
                (m-1-2) edge[dashed] node {\beta} (m-2-2)

                (m-2-1) edge node {f} (m-2-2)
                    edge node {\otimes} (m-1-1);
        \end{tikzpicture}
    \end{equation}
    Let \( f \) be the map defined element-wise as follows
    \[
        a \times b \mapsto \iota_{i + 1, j}\tuple{d_{A, i}(a) \otimes b} + (-1)^{i} \iota_{i, j + 1}\tuple{a \otimes d_{B, j}(b)}
    \]
    One can check that this map is \( R \)-balanced, and therefore a morphism. In addition, by the universal property of tensor product in \( \Mod(R) \), \( f \) induces a unique morphism, \( \alpha_{i, j} \), which is induced from the elementary tensors as follows
    \[
        a \otimes b \mapsto \iota_{i + 1, j}\tuple{d_{A, i}(a) \otimes b} + (-1)^{i} \iota_{i, j + 1}\tuple{a \otimes d_{B, j}(b)}.
    \]
    Since this works for any \( i, j \) as long as \( i + j = n \), one can construct \( \alpha_{i, j} \) for every valid \( i, j \) pair.

    Then by using the universal property of the coproduct one gets the unique map \( \beta \) which is by \autoref{tikz:differential_of_tensor_product_of_chain_complexes_over_Mod(R)} uniquely determined by it's actions on elementary tensors in \( A_i \otimes B_j \) in the following way
    \[
        \iota_{i, j}(a \otimes b) \mapsto \iota_{i + 1, j}\tuple{d_{A, i}(a) \otimes b} + (-1)^{i} \iota_{i, j + 1}\tuple{a \otimes d_{B, j}(b)}.
    \]
    Which is exactly equal to \( d_n \).

    By a similar argument as above, it follows that since \( d_{n + 1} \circ d_n \) sends every \( \iota_{i, j}(a \otimes b) \) to \( 0 \), then it has to be the zero map, and \( d_n \) is therefore a differential.
\end{remark}

\begin{definition}[Internal hom of chain complexes over \( \Mod(R) \)]
    \label{def:internal_hom_of_chain_complexes_over_Mod(R)}
    Let \( R \) be a commutative ring with identity. Furthermore let \( A, B \in \C \tuple{\Mod(R)} \).

    Then define the modules
    \[
        \class{A, B}_n := \prod_{j \in \Zb} \Mod(R)(A_j, B_{j + n})
    \]
    which are a part of the chain complex
    \begin{center}
        \begin{tikzpicture}
            \diagram{m}{1cm}{1cm} {
                \class{A, B}: \\
            };
        \end{tikzpicture}
        %
        \begin{tikzpicture}
            \diagram{m}{1cm}{1cm} {
                \cdots \& \class{A, B}_{-1} \& \class{A, B}_0 \& \class{A, B}_1 \& \cdots \\
            };

            \draw[math]
                (m-1-1) edge (m-1-2)
                (m-1-2) edge node {d_{-1}} (m-1-3)
                (m-1-3) edge node {d_0} (m-1-4)
                (m-1-4) edge (m-1-5);
        \end{tikzpicture}
    \end{center}

    Where the differentials, \( d_n \), are defined as follows:
    WIP
    \begin{align*}
        d_i: \prod_{j \in \Zb} \Mod(R)(A_j, B_{j + i}) &\to \prod_{j \in \Zb} \Mod(R)(A_j, B_{j + i + 1}) \\
        f &\mapsto d_B \circ f - (-1)^i f \circ d_A
    \end{align*}
    This is called the \emph{internal hom of chain complexes over \( \Mod(R) \)}.
\end{definition}

% MS-Question: Do I lack data in this definition?
% https://ncatlab.org/nlab/show/category+of+chain+complexes
\begin{fact}[nlab]
    Let \( R \) be a commutative ring with identity, and let \( \otimes \) denote the tensor product on \( \C \tuple{\Mod(R)} \). Furthermore let \( I \) be the chain complex in \( \C \tuple{\Mod(R)} \) consisting solely of \( 0 \)-objects in non-zero degrees, and the \( R \)-module \( R \) in degree 0. 

    Then \( \tuple{\C \tuple{\Mod(R)}, \otimes, I} \) is a symmetric closed monoidal category.
\end{fact}

% This thesis will be using the following definition, which is similar to the one given by Berest--Mehrle, but not restricted to small categories.
% TODO: Cite
\begin{definition}% [DG-category]
    Let \( R \) be a commutative ring with identity.

    Then \( \Cc \) is a \emph{DG-category over \( R \)} if it is a category enriched over \( \C \tuple{\Mod(R)} \).
\end{definition}
This definition also appear in Jasso--Muro p. 29. (TODO: Ref), except they define it for a field and not a commutative ring with identity.



\subsection{Definition of Massey Product in a DG-category}
% MS-Question: Is H^* a full functor on a DG-category? -> It should be full.
% TODO: Decide on notation, class-notation or H^*-notation?
\begin{definition}
    Let \( \Cc \) be a differentially graded category over \( R \).

    Let the following be a diagram in \( \Cc \)
    \begin{center}
        \begin{tikzpicture}
            \diagram{m}{1cm}{1cm} {
                X_1 & X_2 & X_3 & X_4 \\
            };

            \draw[math]
                (m-1-1) edge node {f_1} (m-1-2)
                (m-1-2) edge node {f_2} (m-1-3)
                (m-1-3) edge node {f_3} (m-1-4);
        \end{tikzpicture}
    \end{center}

    Furthermore, let
    \[
        [f_1] = H^* \tuple{f_1} \in H^* \tuple{\Cc \tuple{X_1, X_2}},
    \]
    \[
        [f_2] = H^* \tuple{f_2} \in H^* \tuple{\Cc \tuple{X_2, X_3}},
    \]
    and
    \[
        [f_3] = H^* \tuple{f_3} \in H^* \tuple{\Cc \tuple{X_3, X_4}}.
    \]
    Then let:
    \[
        \set{
            [(-1)^{\abs{f_3} + \abs{f_2}}s f_1 + (-1)^{\abs{f_3} + 1} f_3 t]
            \mid
            d(s) = (-1)^{f_3 + 1} f_3 f_2, \quad
            d(t) = (-1)^{f_2 + 1} f_2 f_1
        }
    \]
    This is a subset of \( H^* \tuple{\Cc \tuple{X_1, X_4}} \), called the \emph{Massey product of \( f_3, f_2 \) and \( f_1 \)}, and is denoted as \( \toda{[f_3], [f_2], [f_1]} \).
\end{definition}

\begin{remark}
    If \( \Cc \) is a differentially graded category over \( R \) with the following diagram
    \begin{center}
        \begin{tikzpicture}
            \diagram{m}{1cm}{1cm} {
                X_1 & X_2 & X_3 & X_4 \\
            };

            \draw[math]
                (m-1-1) edge node {f_1} (m-1-2)
                (m-1-2) edge node {f_2} (m-1-3)
                (m-1-3) edge node {f_3} (m-1-4);
        \end{tikzpicture}
    \end{center}
    where \( \abs{f_1} = d_1, \abs{f_2} = d_2 \), and \( \abs{f_3} = d_3 \).

    Then one gets that
    
\end{remark}

\subsection{What is a pre-triangulated category?}
There are multiple definitions of an algebraic triangulated categoery. In this thesis, the definition used will be from TODO:Cite Jasso--Muro.

% TODO Cite: Jasso--Muro
\begin{definition}[\( \C_{\dg}(\Mod(R)) \)]
    Let \( R \) be a commutative ring with identity.

    Then let \emph{\( \C_{\dg}(\Mod(R)) \)} be a DG-category defined as follows
    \begin{enumerate}
        \item {
            \( \Obj(\C_{\dg}(\Mod(R))) := \Obj(\C(\Mod(R))) \).
        }
        \item {
            For \( A, B \in \C_{\dg}(\Mod(R)) \) let
            \[ 
                \tuple{\C_{\dg}(\Mod(R))(A, B)}_i := \bigoplus_{j \in \Zb} \Mod(R)(A_j, B_{j + i})
            \]
            with
            \[
                \C_{\dg}(\Mod(R))(A, B) = \bigoplus_{i \in \Zb} \tuple{\C_{\dg}(\Mod(R))(A, B)}_i.
            \]

            Let \( d_A \) and \( d_B \) be the differential of \( A \) and \( B \) as objects of \( \C(\Mod(R)) \), respectively.

            % TODO: This is slight abuse of notation (d_B and d_A are defined on A, B, not A_j, B_{j + i}). Should I fix?
            Let the differential of \( \C_{\dg}(\Mod(R))(A, B) \) be defined as follows
            \begin{align*}
                d_i: \tuple{\C_{\dg}(\Mod(R))(A, B)}_i &\to \tuple{\C_{\dg}(\Mod(R))(A, B)}_{i + 1} \\
                f &\mapsto d_B \circ f - (-1)^i f \circ d_A
            \end{align*}
        }
        \item {
            For \( A, B, C \in \C_{\dg}(\Mod(R)) \), let
            \[
                \circ_{\C_{\dg}(\Mod(R))}: \C_{\dg}(\Mod(R))(B, C) \otimes \C_{\dg}(\Mod(R))(A, B) \to \C_{\dg}(\Mod(R))(A, C)
            \]
            % TODO: Explain more?
            be defined as expected.
        }
    \end{enumerate}
\end{definition}

% TODO Cite: Jasso--Muro
% TODO: Elaborate on enriched functor?
\begin{definition}[DG-functor]
    An enriched functor between two DG-categories is called a \emph{DG-functor}.
\end{definition}

% MS-question: Remark below. -> Probably OK.
\begin{remark}
    % Bondal--Kapranov has as definition
    Enriched functor between DG-categories implies it preserves differentials and grading? TODO
\end{remark}

% TODO: DG-categories over the same ring?
% TODO: Need to show that it's a category?
% MS-Question: Correct? Berest--Mehrle (LN) has another def. -> Subscript dg betyr enriched.
\begin{notation}
    For two DG-categories \( \Ac, \Bc \) over the same commutative ring \( R \), let \( \Fun_{\dg}(\Ac, \Bc) \) denote the DG-category over \( R \) of all DG-functors from \( \Ac \) to \( \Bc \).

    TODO: Show DG-structure.
\end{notation}

\begin{definition}[Opposite DG-category]
    Let \( \Cc \) be a DG-category.

    Then let \( \Cc^{op} \) be the DG-category defined as follows
    \begin{enumerate}
        \item {
            \( \Obj(\Cc^{op}) := \Obj(\Cc) \)
        }
        \item {
            For \( A, B \in \Cc^{op} \), let \( \Cc^{op}(A, B) := \Cc(B, A) \).
        }
        \item {
            For \( A, B, C \in \Cc^{op} \), with \( f \in \Cc^{op}(B, C) \) and \( g \in \Cc^{op}(A, B) \) homogeneous elements of degree \( d_f \) and \( d_g \) respectively.

            Let composition be defined as
            \begin{align*}
                \circ_{\Cc^{op}}: \Cc^{op}(B, C) \otimes \Cc^{op}(A, B) &\to \Cc^{op}(A, C) \\
                f \otimes g &\mapsto (-1)^{d_f d_g} \circ_{\Cc} (g \otimes f)
            \end{align*}
            % TODO: Why can I say this? Need some statement saying all elementary tensors are a sum of homogeneous elementary tensors? As well as saying that the set of elementary tensors generate all elements in the tensor product?
            and extended to all other elementary tensors.
        }
    \end{enumerate}
\end{definition}

% TODO: Cite: Jasso--Muro
% No mention of the ring in the notation? -> Implied since DG-category is over a ring!
\begin{definition}[\( \dgMod_{\dg}(\Cc) \)]
    Let \( \Cc \) be a DG-category over \( R \).

    % TODO: Why "Right"?
    Then define the \emph{DG-category of (right) DG \( \Cc \)-modules} as
    \[
        \dgMod_{\dg}(\Cc) := \Fun_{\dg}(\Cc^{op}, \C_{\dg}(\Mod(R))).
    \]
    Objects in \( \dgMod_{\dg}(\Cc) \) are called \emph{DG-modules over \( \Cc \)}.
\end{definition}

% TODO: Should be true according to Jasso--Muro
% TODO: DG-category over R?
% Probably true by the definition considering Fun_dg is a DG cat.
\begin{proposition}
    \( \dgMod_{\dg}(\Cc) \) is a DG-category.
\end{proposition}
\begin{proof}
    TODO
\end{proof}

% TODO: Why is \Cc(-, A) a functor into \C_{\dg}(\Mod(R))?
\begin{definition}[DG Yoneda embedding]
    \label{def:DG_Yoneda_embedding}
    Let \( \Cc \) be a DG-category over \( R \).
    
    Then let \( \mathbf{h} \) be the functor defined as follows
    \begin{align*}
        \mathbf{h}: \Cc &\to \dgMod_{\dg}(\Cc) \\
        A &\mapsto \Cc(-, A)
    \end{align*}

    This functor is called the \emph{DG Yoneda embedding of \( \Cc \)}.
\end{definition}

% TODO: Add Yoneda embedding identifies \Cc with a full subcategory of \dgMod_dg(\Cc)?

% TODO: Could define this for Mod(R)-enriched categories?
\begin{definition}[0th cohomology category of a DG category]
    Let \( \Cc \) be a DG category over \( R \).

    Then let \( H^0(\Cc) \) be the following (enriched over \( \Mod(R) \) TODO) category defined as follows
    \begin{enumerate}
        \item {
            Let \( \Obj(H^0(\Cc)) := \Obj(\Cc) \).
        }
        \item {
            Let \( A, B \in H^0(\Cc) \).

            Then let \( H^0(\Cc)(A, B) := H^0(\Cc(A, B)) \).
        }
        \item {
            Let \( A, B, C \in H^0(\Cc) \) with \( f_1 \in H^0(A, B) \) and \( f_2 \in H^0(B, C) \).

            Then by \autoref{lem:massey_product_in_dg_cat/massey_product_definition/exist_lifting_h_star}, there exists \( g_1 \in \Cc(A, B) \), and \( g_2 \in \Cc(B, C) \) such that \( \class{g_1} = f_1 \) and \( \class{g_2} = f_2 \).

            % TODO: Slight abuse of notation taking a "class" of an element of a chain complex.
            % TODO: Need to show that this is well defined?
            Then let composition be defined on elementary tensors as follows
            \begin{align*}
                \circ_{H^0(\Cc)}: H^0(\Cc)(B, C) \otimes H^0(\Cc)(A, B) &\to H^0(\Cc)(A, C) \\
                f_2 \otimes f_1 &\mapsto \class{ \circ_{\Cc}(g_2 \otimes g_1) }
            \end{align*}
        }
    \end{enumerate}
\end{definition}

% TODO: What are the triangles? Is the shift functor correct on maps?
% TODO: Incorrect/abuse of notation, how does the shift work on maps?
\begin{theorem}
    Let \( \Cc \) be a DG-category over \( R \). Let \( \Sigma_{\C_{\dg}(\Mod(R))} \) be the shift functor on \( \C_{\dg}(\Mod(R)) \).

    Then \( H^0(\dgMod_{\dg}(\Cc)) \) is a triangulated category with the shift functor \( \Sigma(-) = \Sigma_{\C_{\dg}(\Mod(R))} \circ - \).
\end{theorem}
\begin{proof}
    TODO
\end{proof}

% MS-Question: Have seen definition of small category that is that the class of iso classes are small, not the class of objects. What is correct? Are they equivalent? -> Essentially small.

\begin{definition}[Acyclic DG-module]
    Let \( \Cc \) be a DG-category over a commutative ring (with identity) \( R \). Furthermore, let \( A \in \dgMod_{\dg}(\Cc) \) be a DG-module over \( \Cc \).

    Then \( A \) is called \emph{acyclic} if for any \( X \in \Cc \), one has that \( A(X) \in \C_{\dg}(\Mod(R)) \) is acyclic, i.e. \( H^*(A(X)) = 0 \).
\end{definition}

% TODO: Is \dgMod_{\dg}(\Cc) abelian?/Have kernels?
\begin{definition}[DG-projective module]
    Let \( P \in \dgMod_{\dg}(\Cc) \) be a DG-module.

    Then \( P \) is called a \emph{DG-projective module over \( \Cc \)} if:
    
    For any DG-module \( A \in \dgMod_{\dg}(\Cc) \) and any epimorphism \( f \in \dgMod_{dg}(\Cc)(A, P) \) where \( \ker(f) \in \dgMod_{\dg}(\Cc) \) is acyclic. Then \( f \) is split.
\end{definition}

% TODO: Various definitions and idiosyncracies. Which is correct?
    % Acyclic kernel? Projective objects? Spanned?
    % Krause 07 -> Compact objects, maybe more.
    % Keller 94 -> Another definition of derived DG category.
% TODO: Following def from Krause 07, but not explicitly written down. Is it correct?
\begin{definition}[Derived DG-category]
    Let \( \Cc \) be a DG-category.

    Then the \emph{derived DG-category} of \( \Cc \), denoted \( \D(\Cc) \), is defined as the full subcategory of \( H^0(\dgMod_{\dg}(\Cc)) \) spanned by the objects of \( \dgMod_{\dg}(\Cc) \) that are DG-projective.
\end{definition}

% MS-Question: What is coproduct for the derived category?
% Cite: Jasso--Muro p.31, only a statement, no proof
\begin{proposition}
    \( \D(\Cc) \) is closed under arbitrary coproduct.
\end{proposition}
\begin{proof}
    TODO
\end{proof}

% MS-Quastion: Why are small categories sometimes mentioned in def of derived category? Something to do with localization being well defined?

% MS-Question: Arbitrary coproducts <=> infinite coproducts? -> Need triangulated property for this to make sense.
% Cite: Krause 07 p. 29
\begin{definition}[Compact objects of a category]
    Let \( \Cc \) be a triangulated category with arbitrary coproduct. Let \( X \in \Cc \). 
    
    Then if \( X \) has the following property:
    
    For any index set \( I \) and any morphism \( f: X \to \coprod_{i \in I} Y_i \), there is a finite index set \( J \subseteq I \) such that \( f \) factors through \( \coprod_{j \in J} Y_j \).
    
    Then define \( X \) as a \emph{compact object in \( \Cc \)}.
\end{definition}

% TODO: Is the derived category triangulated? Need to be in order for previous def to apply.
\begin{definition}[Perfect derived DG-category \( \D^c(\Cc) \)]
    Let \( \D(\Cc) \) be the derived DG-category of \( \Cc \).

    Then define \( \D^c(\Cc) \) to be the full subcategory of \( \D(\Cc) \) consisting of all compact objects in \( \D(\Cc) \). This is called the \emph{perfect derived DG-category of \( \Cc \)}.
\end{definition}

% TODO: Cite: Jasso--Muro says so
\begin{proposition}
    \( \D^c(\Cc) \) is triangulated.
\end{proposition}
\begin{proof}
    TODO
\end{proof}

% TODO: Is this even a functor? Or well defined? Probably need to show well defined and that every morphism in H^0(A) has a representative in A.
\begin{definition}[\( H^0 \)-induced functor]
    \label{def:H^0-induced_functor}
    Let \( \Ac \) and \( \Bc \) be two DG-categories, and let \( F: \Ac \to \Bc \) be a functor between them.

    Then define the functor \( H^0(F) \) as follows:
    \begin{align*}
        H^0(F): H^0(\Ac) &\to H^0(\Bc) \\
        A &\mapsto F(A) \\
        (H^0(f): A \to B) &\mapsto (H^0(F(f)): F(A) \to F(B)) 
    \end{align*}

    This is called the \( H^0 \)-induced functor of \( F \).
\end{definition}

\begin{theorem}
    \autoref{def:H^0-induced_functor} is a well-defined functor.
\end{theorem}
\begin{proof}
    TODO
\end{proof}

% Want to show that H^0(h) has codomain D^c(\Cc)
\begin{remark}
    Let \( \mathbf{h}: \Cc \to \dgMod_{\dg}(\Cc) \) be the DG-Yoneda embedding from \autoref{def:DG_Yoneda_embedding}.

    Then for any \( A \in \Cc \), one has that \( H^0(\mathbf{h})(A) \) is both DG-projective and compact.
    
    TODO: SHOW!!!

    Therefore one has that the functor \( H^0(\mathbf{h}): H^0(\Cc) \to H^0(\dgMod_{\dg}(\Cc)) \) factors through \( \D^c(\Cc) \). Denote this functor with the same notation:
    \[
        H^0(\mathbf{h}): H^0(\Cc) \to \D^c(\Cc)
    \]
\end{remark}

\begin{remark}
    \( H^0(\mathbf{h}): H^0(\Cc) \to \D^c(\Cc) \) is fully faithful.

    TODO: Prove
\end{remark}

% TODO: Why need small? Probably something with derived.
% TODO: Heilt ordrett nesten frå Jasso--Muro 2023 p. 32, burde kanskje omformulera?
\begin{definition}[pre-triangulated DG-category]
    Let \( \Cc \) be a small DG-category.

    Then \( \Cc \) is called a \emph{pre-triangulated DG-category} if the image of the (fully faithful) functor \( H^0(\mathbf{h}): H^0(\Cc) \to \D^c(\Cc) \) is a triangulated subcategory of \( \D^c(Cc) \).
\end{definition}

\begin{definition}[Algebraic triangulated category]
    Let \( \Tc \) be a triangulated category.

    Then \( \Tc \) is called an \emph{algebraic triangulated category} if there exist a pre-triangulated DG-category, \( \Cc \), such that \( H^0(\Cc) \) is equivalent to \( \Tc \).
\end{definition}

\subsection{Why do Massey product and toda brackets intersect?}
% First need to extend massey-prod definition to H^0

% MS-Question: I Jasso--Muro så er dette berre definert for element av dgMod (Ingen subscript!)
% TODO: This is also the same shift functor that makes H^0(dgmodblabla) triangulated. Probably not neccesary to specify then.
\begin{proposition}
    \label{prop:H^i_dgmod_cong_H^0_with_shift}
    Let \( \Cc \) be a DG-category over \( R \). Let \( \Sigma \) be the shift functor on \( H^0(\dgMod_{\dg}(\Cc)) \). Let \( A, B \in \dgMod_{\dg}(\Cc) \).

    Then there is an isomorphism
    \[
        \phi: H^i(\dgMod_{\dg}(\Cc)(A, B)) \stackrel{\sim}{\to} H^0(\dgMod_{\dg}(\Cc))(A, \Sigma^i(B)).
    \]
\end{proposition}
\begin{proof}
    TODO
\end{proof}

% TODO: Show this is well defined!!
\begin{remark}
    \label{rem:H^0_into_H^*_inclusion}
    Let \( \Cc \) be a DG-category.

    There is a dense and faithful functor \( \iota: H^0(\Cc) \hookrightarrow H^*(\Cc) \) given by
    \begin{align*}
        \iota: H^0(\Cc) &\to H^*(\Cc) \\
        A &\mapsto A \\
        \iota_{A, B}: H^0(\Cc)(A, B) &\to H^*(\Cc)(A, B) \\
        f &\mapsto \tuple{\dots, 0, f, 0, \dots} \quad \text{\( f \) is in degree \( 0 \)}
    \end{align*}

    TODO: Show well defined.
\end{remark}

\begin{definition}[Massey product on \( H^0(\dgMod_{\dg}(\Cc)) \)]
    \label{def:massey_product_H^0(dgMod_dg(C))}
    Let \( \Cc \) be a DG-category. Let the following be a diagram in \( H^0(\dgMod_{\dg}(\Cc)) \)
    \begin{center}
        \begin{tikzpicture}
            \diagram{m}{1cm}{1cm} {
                X_1 & X_2 & X_3 & X_4 \\
            };

            \draw[math]
                (m-1-1) edge node {f_1} (m-1-2)
                (m-1-2) edge node {f_2} (m-1-3)
                (m-1-3) edge node {f_3} (m-1-4);
        \end{tikzpicture}
    \end{center}
    Using the functor \( \iota \) in \autoref{rem:H^0_into_H^*_inclusion} one can view the above diagram as a diagram in \( H^*(\dgMod_{\dg}(\Cc)) \) as follows
    \begin{center}
        \begin{tikzpicture}
            \diagram{m}{1cm}{1cm} {
                X_1 & X_2 & X_3 & X_4 \\
            };

            \draw[math]
                (m-1-1) edge node {\iota(f_1)} (m-1-2)
                (m-1-2) edge node {\iota(f_2)} (m-1-3)
                (m-1-3) edge node {\iota(f_3)} (m-1-4);
        \end{tikzpicture}
    \end{center}
    where all the maps are of degree \( 0 \).

    % MS-Question: Overly complicated and imprecise on domains and stuff.
    By \autoref{rem:massey_product_in_dg_cat/massey_product_definition/massey_product_sum_of_degrees} the massey product of these maps \( \massey{\iota(f_3), \iota(f_2), \iota(f_1)} \) only have non-zero components in \( H^{-1}(\dgMod_{\dg}(\Cc)(X_1, X_4)) \). Then, using the ismorphism \( \phi \) in \autoref{prop:H^i_dgmod_cong_H^0_with_shift} one has that \( \phi(\massey{\iota(f_3), \iota(f_2), \iota(f_1)}) \subseteq H^0(\dgMod_{\dg}(\Cc)(X_1, \Sigma^{-1}X_4)) \). This in turn means that \( \phi(\massey{\iota(f_3), \iota(f_2), \iota(f_1)}) \subseteq \im(\iota_{X_1, \Sigma^{-1}(X_4)}) \). But since \( \iota \) is a faithful and dense funcor, it follows that there is a unique subset \( M \subseteq H^0(\dgMod_{\dg}(\Cc))(X_1, \Sigma^{-1}(X_4)) \) such that \( \iota(M) = \phi(\massey{\iota(f_3), \iota(f_2), \iota(f_1)}) \).

    Then define this \( M \) as the \emph{massey product on \( H^0(\dgMod_{\dg}(\Cc)) \)}.
\end{definition}

% TODO: Specify that the functor is exact?
\begin{definition}[Massey product in an algebraic triangulated category]
    Let \( \Tc \) be an algebraic triangulated category, and let the following be a diagram in \( \Tc \)
    \begin{center}
        \begin{tikzpicture}
            \diagram{m}{1cm}{1cm} {
                X_1 & X_2 & X_3 & X_4 \\
            };

            \draw[math]
                (m-1-1) edge node {f_1} (m-1-2)
                (m-1-2) edge node {f_2} (m-1-3)
                (m-1-3) edge node {f_3} (m-1-4);
        \end{tikzpicture}
    \end{center}

    Since \( \Tc \) is algebraic, it is equivalent to \( H^0(\Cc) \) for some pre-triangulated DG-category \( \Cc \). Furthermore, since \( \Cc \) is a pre-triangulated DG-category, one has that \( H^0(\Cc) \) is equivalent to \( \im(H^0(\mathbf{h})) \). And since \( \im(H^0(\mathbf{h})) \) is a full subcategory of \( H^0(\dgMod_{\dg}(\Cc)) \), the diagram above can be looked at as a diagram in \( H^0(\dgMod_{\dg}(\Cc)) \).

    To recap, one has the relation:
    \[
        \Tc \cong H^0(\Cc) \cong  \stackrel{full}{\subseteq} \D^c(\Cc) \stackrel{full}{\subseteq} \D(\Cc) \stackrel{full}{\subseteq} H^0(\dgMod_{\dg}(\Cc))
    \]

    Let \( F \) denote the functor that takes \( F: \Tc \hookrightarrow H^0(\dgMod_{\dg}(\Cc)) \) by the maps above. Then one has that the above diagram in \( \Tc \) can be viewed as a diagram in \( H^0(\dgMod_{\dg}(\Cc)) \) as follows
    \begin{center}
        \begin{tikzpicture}
            \diagram{m}{1cm}{1cm} {
                F(X_1) & F(X_2) & F(X_3) & F(X_4). \\
            };

            \draw[math]
                (m-1-1) edge node {F(f_1)} (m-1-2)
                (m-1-2) edge node {F(f_2)} (m-1-3)
                (m-1-3) edge node {F(f_3)} (m-1-4);
        \end{tikzpicture}
    \end{center}
    
    On the above diagram one can take the massey product (as in \autoref{def:massey_product_H^0(dgMod_dg(C))}). This yields a subset \( \massey{F(f_3), F(f_2), F(f_1)} \subseteq H^0(\dgMod_{\dg}(\Cc))(F(X_1), \Sigma^{-1}(F(X_4))) \), and since \( \im(H^0(\mathbf{h})) \) is a full subcategory of \( H^0(\dgMod_{\dg}(\Cc)) \), one has that \( \massey{F(f_3), F(f_2), F(f_1)} \subseteq \im(H^0(\mathbf{h}))(F(X_1), \Sigma^{-1}(F(X_4))) \) and therefore isomorphic as a set to a subset of \( \Tc(X_1, \Sigma^{-1}(X_4)) \cong \Tc(\Sigma(X_1), X_4) \). This subset is denoted as \( \massey{f_3, f_2, f_1} \) and is called the \emph{massey product of \( \Tc \)}.
\end{definition}

\begin{theorem}
    Let \( \Tc \) be an algebraic triangulated category. Furthermore let the following be a diagram in \( \Tc \)
    \begin{center}
        \begin{tikzpicture}
            \diagram{m}{1cm}{1cm} {
                X_1 & X_2 & X_3 & X_4 \\
            };

            \draw[math]
                (m-1-1) edge node {f_1} (m-1-2)
                (m-1-2) edge node {f_2} (m-1-3)
                (m-1-3) edge node {f_3} (m-1-4);
        \end{tikzpicture}
    \end{center}
    % MS-question: Massey product ser stygt ut.
    Then \( \toda{f_3, f_2, f_1} = (-1)^{TODO} \massey{f_3, f_2, f_1} \).
\end{theorem}
