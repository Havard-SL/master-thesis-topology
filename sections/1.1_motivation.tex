Triangulated categories were discovered independently by Dieter Puppe and Jean-Louis Verdier in 1962 and 1963, respectively. Verdier introduced his axioms in his doctoral thesis, supervised by Alexander Grothendieck, leading to the attribution for the discovery of triangulated category sometimes being split three ways between Puppe, Verdier, and Grothendieck. Puppe discovered triangulated categories while looking at the stable homotopy category, while Verdier and Grothendieck were motivated from a more algebraic viewpoint by the derived category of an abelian category.

% TODO: Fix claims and references when finished.
As their discovery would imply, triangulated categories are useful in both algebra and topology. Typical algebraic examples of triangulated categories are: the derived categories of abelian category as mentioned above, the stable module category, as well as the homotopy category of chain complexes. On the other hand, typical topological examples of triangulated categories are: the stable homotopy category, and the Spanier--Whitehead category. In \autoref{section:tri_cats} we will define triangulated categories, as well as give three examples. We will touch a little upon the Spanier--Whitehead category (\autoref{subsubsection:spanier_whitehead_cat}) and the chain homotopy category (\autoref{subsubsection:chain_homotopy_cat}), but will mainly focus on the stable module category (\autoref{subsubsection:stable_module_cat}) for use in examples throughout the thesis, along with a proof that it is a triangulated category (\autoref{example:stable_module_category_triangulated}).

% MS-Question: Dwyer-Kan influence on Shipley?
% TODO: Skriv meir samanhengande.
Toda brackets were initially introduced by Hiroshi Toda in 1962 to calculate homotopy groups of spheres. The definition of Toda brackets in any triangulated category is based on the work of Shipley in 2002 \cite[Definition A.2]{Shipley_2002}. She bases her definition on ideas from Cohen in 1968 \cite[p. 308]{Cohen_1968}. Shipley's definition was further refined by Sagave in 2008 \cite[Remark 4.5]{Sagave_2008}. The definitions in this thesis is a more elegant formulation of Sagave's definitions by Christensen and Frankland in 2017 \cite[Definition 3.1]{Christensen-Frankland_2017}. The definitions and some examples of calculations will be discussed in \autoref{section:toda_brackets}.

Toda brackets are useful, but can be hard to compute. Therefore, any additional tools to compute Toda brackets could help a lot in certain cases. One such tool is the ``indeterminacy'', but another tool is ``Massey products''.

Massey products were introduced by William Schumacher Massey in 1958. They can be defined in a DG-category (\autoref{def:massey_product_dg_cat}) and further generalized to be defined in any algebraic triangulated category (\autoref{def:massey_product_alg_tri_cat}). It turns out that in an algebraic triangulated category, Massey products and Toda brackets coincide as follows
\[
    \toda{f_3, f_2, f_1} = \massey{f_3, f_2, f_1},
\]
which will be proved in \autoref{theorem:massey_equals_toda}.

In order to understand why the Toda brackets equal the Massey products, we will define a lot of the necessary prerequisites to understanding the proof, including DG-categories, DG-modules and algebraic triangulated categories. We will not be defining everything in the most general form, but rather narrow down the definitions to the case we are interested in.

Finally, in \autoref{section:applications} we will discuss potential applications of this equality between Massey products and Toda brackets.

By NTNU regulations, it is mandatory to include the following subsection.
\subsubsection{Sustainability statement}
While pure mathematics seemingly has no real world impact, it has been repeatedly proven that pure mathematics appear in other fields of study in the future and provides tools to solve problems that can appear in the real world. Topology in particular, has a deep history connected to optimization problems, and in newer times have been used in data analysis and other fields. This thesis could therefore bring future benefit to the physical sciences as well as data analysis which are key fields for future innovation. Given the potential widespread applicability of pure topology research, this thesis could contribute to goal number 3 for good health and well being, by being a tool that could be used in the medical field, as well as goal 9 for industry, innovation and infrastructure, by creating tools that could potentially be used to drive new innovations.

Of course, given the potential wide applicability of topological techniques in the future, it could contribute to other goals, but the two abovementioned goals, namely goal 3 and 9, are the ones where topology is currently being used to some degree.

