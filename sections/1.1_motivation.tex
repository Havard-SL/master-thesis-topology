Triangulated categories was allegedly discovered independently by Dieter Puppe and Jean-Louis Verdier in 1962 and 1963 respectively. Verdier introduced his axioms in his doctoral thesis, supervised by the famous Alexander Groethendieck, leading to the attribution for the discovery of triangulated category sometimes being split three ways between Puppe, Verdier, and Groethendieck. Puppe discovered triangulated categories while looking at the stable homotopy category, while Verdier and Groethendieck were motivated from a more algebraic viewpoint by the derived categroy of an abelian category.

% TODO: Fix claims and references when finished.
As it's discovery would imply, triangulated categories are useful in both algebra and topology. Notable algebra inspired triangulated categories are: the derived categories of abelian category as mentioned above, the stable module category, as well as the homotopy category of chain complexes. On the other hand, notable topology inspired triangulated categories are: the stable homotopy category, and the Spanier-Whitehead category. This thesis will touch a little upon the Spanier-Whitehead category and the derived category of an abelian category in section 2, but will mainly focus on the stable module category for use in examples throughout the thesis, along with a proof that it's a triangulated category in section 2.

% MS-Question: Dwyer-Kan influence on Shipley?
Toda brackets were initially introduced by Hiroshi Toda in 1962 for use in calculating homotopy groups of spheres. In this thesis a definition of Toda brackets on any triangulated category, as was first introduced by Shipley in 2002 \cite[Definition A.2]{Shipley_2002}, based on ideas from Cohen in 1968 \cite[Definition on bottom of p. 308]{Cohen_1968} and built upon by Sagave in 2008 \cite[Remark 4.5]{Sagave_2008}. The definitions in this thesis is a more elegant formulation of Segave's definitions by Christensen and Frankland in 2017 \cite[Definition 3.1]{Christensen-Frankland_2017}.

Massey products were introduced by William Schumacher Massey in 1958. They can be defined on a DG-category and further generalized to be defined on any algebraic triangulated category. Both DG-category and algebraic triangulated category will be defined on this thesis. In general, it is known that computing Massey products is easier than computing Toda brackets. Luckily, it turns out that on an algebraic triangulated category, Massey products and Toda brackets coincide with a small sign difference as follows
\[
    TODO.
\]

The Adams spectral sequence is a spectral sequence introduced in 1958 by John Frank Adams for use in computing homotopy groups of spectra. Later in 1974, Miller generalized the Adams spectral sequence to be defined on any triangulated category \cite[Chapter I]{Miller_1975}. In 2017, Christen and Frankland \cite[Section 4, Section 6]{Christensen-Frankland_2017} showed that the page \( r \) differential of the Adams spectral sequence on a triangulated category can be expressed with respect to a \( r \)-fold Toda bracket.

Therefore, by using three-fold Massey products, one can potentially calculate the page two differential of an Adams spectral sequence more efficiently than conventional methods.

By NTNU, it is mandatory to include the following subsection.
\subsubsection{Sustainability statement}
While pure mathematics seemingly has no real world impact, it has been repeatedly proven that pure mathematics appear in other fields of study in the future and provides tools to solve problems that can appear in the real world. Topology in particular, has a deep history connected to optimization problems, and in newer times have been used in data analysis and other fields. This thesis could therefore bring future benefit to the physical sciences as well as data analysis which are key fields for future innovation. Given the potential widespread applicability of pure topology research, this thesis could contribute to goal number 3 for good health and well being, by being a tool that could be used in the medical field, as well as goal 9 for industry, innovation and infrastructure, by creating tools that could potentially be used to drive new innovations.

Of course, given the potential wide applicability of topological techniques in the future, it could contribute to other goals, but the two abovementioned goals, namely goal 3 and 9, are the ones where topology is currently being used to some degree.


This thesis will assume that the reader is already familiar with basic homological algebra as well as some module theory.
