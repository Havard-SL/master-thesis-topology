We start by proving a lemma which shows a property of DG-modules which makes it possible to compare Toda brackets and Massey products.

\begin{lemma}
    \label{lem:H^i_dgmod_cong_H^0_with_shift}
    Let \( \Cc \) be a DG-category. Let \( \Sigma \) denote the shift functor on \( H^0(\dgMod_{\dg}(\Cc)) \). And let \( A, B \in \dgMod_{\dg}(\Cc) \).

    Then there are isomorphisms
    \[
        \phi_i: H^i(\dgMod_{\dg}(\Cc)(A, B)) \stackrel{\sim}{\to} H^0(\dgMod_{\dg}(\Cc))(\Sigma^{-i} A, B).
    \]
\end{lemma}
\begin{proof}
    TODO
\end{proof}

Using the above lemma, we can define how to take the Massey product in \( H^0(\dgMod_{\dg}(\Cc)) \).

\begin{definition}[Massey product on \( H^0(\dgMod_{\dg}(\Cc)) \)]
    \label{def:massey_product_H^0(dgMod_dg(C))}
    Let \( \Cc \) be a DG-category, and let \( \phi_i \) be as in \autoref{lem:H^i_dgmod_cong_H^0_with_shift}.
    
    Let the following be a diagram in \( H^0(\dgMod_{\dg}(\Cc)) \)
    \begin{center}
        \begin{tikzpicture}
            \diagram{m}{1cm}{1cm} {
                X_0 \& X_1 \& X_2 \& X_3. \\
            };

            \draw[math]
                (m-1-1) edge node {f_1} (m-1-2)
                (m-1-2) edge node {f_2} (m-1-3)
                (m-1-3) edge node {f_3} (m-1-4);
        \end{tikzpicture}
    \end{center}
    Considering this as a DG-diagram in \( H^{\bullet}(\dgMod_{\dg}(\Cc)) \), with \( |f_i| = 0 \), let the following
    \[
        \phi_{-1}\tuple*{\massey{f_3, f_2, f_1}} \subseteq H^0(\dgMod_{\dg}(\Cc))(\Sigma A, B)
    \]
    be called the \emph{Massey product of \( f_1, f_2, f_3 \) in \( H^0(\dgMod_{\dg}(\Cc)) \)}.
\end{definition}

Using the above definition, we can define the Massey product more generally in an algebraic triangulated category.

\begin{definition}[Massey product in an algebraic triangulated category]
    Let \( \Tc \) be an algebraic triangulated category, let \( \Cc \) be the DG-enhancement of \( \Tc \) with \( \Phi: \Tc \to H^0(\Cc) \) the equivalence by the algebraic triangulated category property.
    
    Let the following be a diagram in \( \Tc \)
    \begin{center}
        \begin{tikzpicture}
            \diagram{m}{1cm}{1cm} {
                X_0 \& X_1 \& X_2 \& X_3. \\
            };

            \draw[math]
                (m-1-1) edge node {f_1} (m-1-2)
                (m-1-2) edge node {f_2} (m-1-3)
                (m-1-3) edge node {f_3} (m-1-4);
        \end{tikzpicture}
    \end{center}
    Consider the following diagram in \( H^0(\Cc) \)
    \begin{center}
        \begin{tikzpicture}
            \diagram{m}{1cm}{1cm} {
                \Phi X_0 \& \Phi X_1 \& \Phi X_2 \& \Phi X_3. \\
            };

            \draw[math]
                (m-1-1) edge node {\Phi f_1} (m-1-2)
                (m-1-2) edge node {\Phi f_2} (m-1-3)
                (m-1-3) edge node {\Phi f_3} (m-1-4);
        \end{tikzpicture}
    \end{center}
    Consider again the following diagram in \( H^0(\dgMod_{\dg}(\Cc)) \)
    \begin{center}
        \begin{tikzpicture}
            \diagram{m}{1cm}{1cm} {
                H^0(\mathbf{h}) \Phi X_0 \& H^0(\mathbf{h}) \Phi X_1 \& H^0(\mathbf{h}) \Phi X_2 \& H^0(\mathbf{h}) \Phi X_3. \\
            };

            \draw[math]
                (m-1-1) edge node {H^0(\mathbf{h}) \Phi f_1} (m-1-2)
                (m-1-2) edge node {H^0(\mathbf{h}) \Phi f_2} (m-1-3)
                (m-1-3) edge node {H^0(\mathbf{h}) \Phi f_3} (m-1-4);
        \end{tikzpicture}
    \end{center}
    Take the Massey product of the above diagram in \( H^0(\dgMod_{\dg}(\Cc)) \) to get
    \[
        \phi_{-1}\tuple*{\massey{H^0(\mathbf{h}) \Phi f_3, H^0(\mathbf{h}) \Phi f_2, H^0(\mathbf{h}) \Phi f_1}} \subseteq H^0(\dgMod_{\dg}(\Cc))(\Sigma H^0(\mathbf{h}) \Phi X_0, H^0(\mathbf{h}) \Phi X_3)
    \]
    Then apply \( \Phi^{-1} H^0(\mathbf{h})^{-1} \) to the result to get
    \[
        \Phi^{-1} H^0(\mathbf{h})^{-1} \phi_{-1}\tuple*{\massey{H^0(\mathbf{h}) \Phi f_3, H^0(\mathbf{h}) \Phi f_2, H^0(\mathbf{h}) \Phi f_1}},
    \]
    which, up to pre and post-composition with isomorphisms, is a subset of \( \Tc(\Sigma X_0, X_3) \).

    This subset is called the \emph{massey product of \( f_1, f_2 \) and \( f_3 \) in an algebraic triangulated category}.
\end{definition}

