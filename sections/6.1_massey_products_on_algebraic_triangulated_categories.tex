The following lemma, is essential in the definition of Massey products on \( H^0(\dgM) \) and therefore any algebraic triangulated category, which will become apparent later.

\begin{lemma}
    \label{cor:H^i_dgmod_cong_H^0_with_shift}
    Let \( F, G \in \dgM \).

    Then there is a canonical isomorphism
    \begin{align*}
        H^0(\phi): H^i(\dgM(F, G)) &\to H^0(\dgM)(\Sigma^{-i} F, G) \\
        \eta &\mapsto \eta.
    \end{align*}
\end{lemma}
\begin{proof}
    Take \( H^0 \) of \( \phi \) in \autoref{lem:dgmod_shift_eq_plus}.
\end{proof}

Using the above lemma, we define how to take the Massey product in \( H^0(\dgM) \) as follows.

\begin{definition}[Massey product on \( H^0(\dgM) \)]
    \label{def:massey_product_H^0(dgMod_dg(C))}
    Let \( \Cc \) be a DG-category.
    
    Let the following be a diagram in \( H^0(\dgM) \)
    \begin{center}
        \begin{tikzpicture}
            \diagram{m}{1cm}{1cm} {
                X_0 \& X_1 \& X_2 \& X_3. \\
            };

            \draw[math]
                (m-1-1) edge node {f_1} (m-1-2)
                (m-1-2) edge node {f_2} (m-1-3)
                (m-1-3) edge node {f_3} (m-1-4);
        \end{tikzpicture}
    \end{center}
    Considering this as a DG-diagram in \( H^{\bullet}(\dgM) \), with \( |f_i| = 0 \), and the canonical isomorphism from \autoref{cor:H^i_dgmod_cong_H^0_with_shift}, let the following
    \[
        \massey{f_3, f_2, f_1} \subseteq H^0(\dgM)(\Sigma A, B)
    \]
    be called the \emph{Massey product of \( f_1, f_2, f_3 \) in \( H^0(\dgM) \)}.
\end{definition}

Using the above definition, we can define the Massey product more generally in an algebraic triangulated category.

\begin{definition}[Massey product in an algebraic triangulated category]
    \label{def:massey_product_alg_tri_cat}
    Let \( \Tc \) be an algebraic triangulated category, let \( \Cc \) be the DG-enhancement of \( \Tc \) with \( \Phi: \Tc \to H^0(\Cc) \) the equivalence by the algebraic triangulated category property. Let \( \phi: \Phi^{-1} H^0(\mathbf{h})^{-1} H^0(\mathbf{h}) \Phi \to \Id_{\Tc} \) be the natural isomorphism arising from the fact that \( H^0(\mathbf{h}) \Phi \) is an equivalence. Furthermore, let \( \eta: H^0(\mathbf{h}) \Phi \Sigma \to \Sigma H^0(\mathbf{h}) \Phi \) be the natural isomorphism from the fact that \( H^0(\mathbf{h}) \Phi \) is a triangulated functor onto its image.
    
    Let the following be a diagram in \( \Tc \)
    \begin{center}
        \begin{tikzpicture}
            \diagram{m}{1cm}{1cm} {
                X_1 \& X_2 \& X_3 \& X_4. \\
            };

            \draw[math]
                (m-1-1) edge node {f_1} (m-1-2)
                (m-1-2) edge node {f_2} (m-1-3)
                (m-1-3) edge node {f_3} (m-1-4);
        \end{tikzpicture}
    \end{center}
    Consider again the following diagram in \( \im(H^0(\mathbf{h}) \Phi) \),
    \begin{center}
        \begin{tikzpicture}
            \diagram{m}{1cm}{1cm} {
                H^0(\mathbf{h}) \Phi X_1 \& H^0(\mathbf{h}) \Phi X_2 \& H^0(\mathbf{h}) \Phi X_3 \& H^0(\mathbf{h}) \Phi X_4. \\
            };

            \draw[math]
                (m-1-1) edge node {H^0(\mathbf{h}) \Phi f_1} (m-1-2)
                (m-1-2) edge node {H^0(\mathbf{h}) \Phi f_2} (m-1-3)
                (m-1-3) edge node {H^0(\mathbf{h}) \Phi f_3} (m-1-4);
        \end{tikzpicture}
    \end{center}
    Take the Massey product of the above diagram in \( H^0(\dgM) \) to get
    \[
        \massey{H^0(\mathbf{h}) \Phi f_3, H^0(\mathbf{h}) \Phi f_2, H^0(\mathbf{h}) \Phi f_1} \subseteq H^0(\dgM)(\Sigma H^0(\mathbf{h}) \Phi X_1, H^0(\mathbf{h}) \Phi X_4)
    \]
    Then pre-compose with \( \eta_{X_1} \) and apply \( \Phi^{-1} H^0(\mathbf{h})^{-1} \) to get
    \[
        \Phi^{-1} H^0(\mathbf{h})^{-1} (\massey{H^0(\mathbf{h}) \Phi f_3, H^0(\mathbf{h}) \Phi f_2, H^0(\mathbf{h}) \Phi f_1} \circ \eta_{X_1}),
    \]
    which, up to pre- and post-composition with \( \phi \), yields
    \[
        \phi_{X_4} \circ \tuple*{\Phi^{-1} H^0(\mathbf{h})^{-1} (\massey{H^0(\mathbf{h}) \Phi f_3, H^0(\mathbf{h}) \Phi f_2, H^0(\mathbf{h}) \Phi f_1} \circ \eta_{X_1})} \circ \phi^{-1}_{\Sigma X_1} \subseteq \Tc(\Sigma X_1, X_4).
    \]

    This subset of \( \Tc(\Sigma X_1, X_4) \) is called the \emph{Massey product of \( f_1, f_2 \) and \( f_3 \) in an algebraic triangulated category}.
\end{definition}

The above definition is well-defined since \( \im(H^0(\mathbf{h}) \Phi) \) is a full and triangulated subcategory of \( H^0(\dgM) \), and so Massey products is well-defined on \( \im(H^0(\mathbf{h}) \Phi) \) since
\begin{align*}
    H^0(\dgM)(\Sigma H^0(\mathbf{h}) \Phi X_1, H^0(\mathbf{h}) \Phi X_4) = \im(H^0(\mathbf{h}) \Phi)(\Sigma H^0(\mathbf{h}) \Phi X_1, H^0(\mathbf{h}) \Phi X_4).
\end{align*}
