\section{Examples of Toda brackets}

Let \( R = \Fb_2C_2 \), with \( g \in C_2 \) being the generator.

Then \( J = \tuple{1 + g} \) is the only ideal of \( R \).

First note that \( J \) is not projective, since the short exact sequence:

\begin{center}
	\begin{tikzpicture}
		\diagram{m}{1cm}{1cm}{
			J & R & J \\
		};
	
		\draw[math]
			(m-1-1) edge[hook] node {\iota} (m-1-2)
			(m-1-2) edge[two heads] node {\phi} (m-1-3);
	\end{tikzpicture}
\end{center}

Where, \( \iota \) is the inclusion, and \( \phi \) is the map:

\begin{align*}
    \phi: R &\to J \\
    0, 1 + g &\mapsto 0 \\
    1, g &\mapsto 1 + g
\end{align*}

But \( \phi \) does not split, since \( \iota \) is the only monomorphism of \( J \) into \( R \), but composes to \( 0 \) with \( \phi \). Therefore \( J \) cannot be projective, since every epimorphism into a projective module splits.

Furthermore, since \( \phi \) is an epimorphism with kernel \( J \), one gets from the third isomorphism theorem that \( \frac{R}{J} \cong J \).

\subsection{Example 1}

The first example I calculated was the Toda bracket of the following diagram:

\begin{center}
	\begin{tikzpicture}
		\diagram{m}{1cm}{1cm}{
				J & J & J & J \\
		};

		\draw[math]
			(m-1-1) edge node {\Id} (m-1-2)
			(m-1-2) edge node {0} (m-1-3)
			(m-1-3) edge node {\Id} (m-1-4);
	\end{tikzpicture}
\end{center}

The cone of \( \Id_J \) is the pushout of \( (\Id_J, \iota_J) \), where \( \iota_J \) is a monomorphism into the injective/projective \( R \). This is by construction \( \frac{R \oplus J}{\sim} \), where \( (0, 1+g) \sim (1+g, 0) \). This is isomorphic to \( R \).

The supension of \( J \) is the cokernel of \( \iota_J \). In the module categoty, this is \( \frac{R}{J} \).

Using the cofiber-cofiber definition, one gets the following diagram:

\begin{center}
	\begin{tikzpicture}
		\diagram{m}{1cm}{1cm} {
			J & J & R & {\frac{R}{J}} \\
			J & J & J & J \\
		};

		\draw[math]
			(m-1-1) edge node {\Id} (m-1-2)
				edge[equal] (m-2-1)
			(m-1-2) edge (m-1-3)
				edge[equal] (m-2-2)
			(m-1-3) edge (m-1-4)
				edge node {\rho} (m-2-3)
			(m-1-4) edge node {\psi} (m-2-4)

			(m-2-1) edge node {\Id} (m-2-2)
			(m-2-2) edge node {0} (m-2-3)
			(m-2-3) edge node {\Id} (m-2-4);
	\end{tikzpicture}
\end{center}

However, since the diagram is in \( \StMod(R) \), one has that \( R \cong 0 \), and therefore \( \rho = 0 \). And from earlier one has that \( \frac{R}{J} \cong J \).

This gives the following diagram in \( \StMod(R) \):

\begin{center}
	\begin{tikzpicture}
		\diagram{m}{1cm}{1cm} {
			J & J & 0 & J \\
			J & J & J & J \\
		};

		\draw[math]
			(m-1-1) edge node {\Id} (m-1-2)
				edge[equal] (m-2-1)
			(m-1-2) edge node {0} (m-1-3)
				edge[equal] (m-2-2)
			(m-1-3) edge node {0} (m-1-4)
				edge node {0} (m-2-3)
			(m-1-4) edge node {\psi} (m-2-4)

			(m-2-1) edge node {\Id} (m-2-2)
			(m-2-2) edge node {0} (m-2-3)
			(m-2-3) edge node {\Id} (m-2-4);
	\end{tikzpicture}
\end{center}

But this shows that any endomorphism on \( J \) makes the rightmost square commute, and is therefore in the Toda bracket (up to pre-composition by an isomorphism \( \frac{R}{J} \to J \)) of \( \toda{\Id_J, 0, \Id_J} \). But since we only care about Toda brackets up to pre/post-composition of isomorphism, one can write that \( \toda{\Id_J, 0, \Id_J} = \Hom_R(J, J) \).

\subsection{Example 2}

Want to show that \( \toda{\Id_J, \Id_J, \Id_J} = \emptyset \) since there is no distinguished triangle one can put in the cofiber-cofiber definition such that the squares commute.

The cone of the identity map is \( 0 \) as seen before (and in general in a triangulated category, the cone of an isomorphism is isomorphic to \( 0 \)), with \( \Sigma(J) \cong J \):

\begin{center}
	\begin{tikzpicture}
		\diagram{m}{1cm}{1cm} {
			J & J & 0 & J \\
			J & J & J & J \\
		};

		\draw[math]
			(m-1-1) edge node {\Id} (m-1-2)
				edge[equal] (m-2-1)
			(m-1-2) edge node {0} (m-1-3)
				edge[equal] (m-2-2)
			(m-1-3) edge node {0} (m-1-4)
				edge[squiggly] node {0} (m-2-3)
			(m-1-4) edge node {\psi} (m-2-4)

			(m-2-1) edge node {\Id} (m-2-2)
			(m-2-2) edge node {\Id} (m-2-3)
			(m-2-3) edge node {\Id} (m-2-4);
	\end{tikzpicture}
\end{center}

There is no squiggly map in the diagram above that can make the middle square commute, unless \( J \cong 0 \).

Therefore \( \toda{\Id_J, \Id_J, \Id_J} = \emptyset \). And one has that (TODO: Cite, Supervisors said so) in general, for the Toda bracket \( \toda{f_3, f_2, f_1} \), if \( f_3 \circ f_2 \neq 0 \) or \( f_2 \circ f_1 \neq 0 \), then the Toda bracket will be empty:

\begin{theorem}
	Let \( f_1, f_2, f_3 \) be three composable maps:

	\begin{center}
		\begin{tikzpicture}
			\diagram{m}{1cm}{1cm} {
				X_1 & X_2 & X_3 & X_4 \\
			};

			\draw[math]
				(m-1-1) edge node {f_1} (m-1-2)
				(m-1-2) edge node {f_2} (m-1-3)
				(m-1-3) edge node {f_3} (m-1-4);
		\end{tikzpicture}
	\end{center}

	such that \( f_2 \circ f_1 \neq 0 \) or \( f_3 \circ f_2 \neq 0 \).

	Then \( \toda{f_3, f_2, f_1} = \emptyset \).
\end{theorem}
\begin{proof}
	TODO
\end{proof}

\subsection{Example 3}

In general, one has that for any triangulated category that if \( \toda{f_3, 0, \Id} \) is a well defined Toda bracket, then it is equal to \( \Hom(\Sigma(X_1), X_3) \). This is because the cone is \( 0 \):

\begin{center}
	\begin{tikzpicture}
		\diagram{m}{1cm}{1cm} {
			{X_1} & {X_1} & 0 & {\Sigma(X_1)} \\
			{X_1} & {X_1} & {X_2} & {X_3} \\
		};

		\draw[math]
			(m-1-1) edge node {\Id} (m-1-2)
				edge[equal] (m-2-1)
			(m-1-2) edge (m-1-3)
				edge[equal] (m-2-2)
			(m-1-3) edge (m-1-4)
				edge (m-2-3)
			(m-1-4) edge node {\phi} (m-2-4)

			(m-2-1) edge node {\Id} (m-2-2)
			(m-2-2) edge node {0} (m-2-3)
			(m-2-3) edge node {f_3} (m-2-4);
	\end{tikzpicture}
\end{center}

So any possible \( \psi \) will make the right square commute.

\subsection{Example 4}

Want to find the Toda bracket of the following maps:

\begin{center}
	\begin{tikzpicture}
		\diagram{m}{1cm}{1cm} {
			J & {J \oplus J} & {J \oplus J} & J \\
		};

		\draw[math]
			(m-1-1) edge node {\begin{psmallmatrix} 1 \\ 1 \end{psmallmatrix}} (m-1-2)
			(m-1-2) edge node {\begin{psmallmatrix} 1 & 1 \\ 1 & 1 \end{psmallmatrix}} (m-1-3)
			(m-1-3) edge node {\begin{psmallmatrix} 1 & 1 \end{psmallmatrix}} (m-1-4);
	\end{tikzpicture}
\end{center}

Using the cofiber-cofiber definition, one gets the following commutative diagram:

\begin{center}
	\begin{tikzpicture}
		\diagram{m}{1cm}{1cm} {
			J & {J \oplus J} & J & J \\
			J & {J \oplus J} & {J \oplus J} & J \\
		};

		\draw[math]
			(m-1-1) edge node {\begin{psmallmatrix} 1 \\ 1 \end{psmallmatrix}} (m-1-2)
				edge[equal] (m-2-1)
			(m-1-2) edge node {{\begin{psmallmatrix} 1 & 1 \end{psmallmatrix}}} (m-1-3)
				edge[equal] (m-2-2)
			(m-1-3) edge (m-1-4)
				edge node {{\begin{psmallmatrix} 1 \\ 1 \end{psmallmatrix}}} (m-2-3)
			(m-1-4) edge node {\psi} (m-2-4)

			(m-2-1) edge node {\begin{psmallmatrix} 1 \\ 1 \end{psmallmatrix}} (m-2-2)
			(m-2-2) edge node {\begin{psmallmatrix} 1 & 1 \\ 1 & 1 \end{psmallmatrix}} (m-2-3)
			(m-2-3) edge node {\begin{psmallmatrix} 1 & 1 \end{psmallmatrix}} (m-2-4);
	\end{tikzpicture}
\end{center}

Where the top row is distinguished.

Here one can see that the only possible map for the middle-right map is to be \( \left(
    \begin{smallmatrix}
        1 \\ 1
    \end{smallmatrix} \right)
    \) in order for the middle square to commute since \( \left(
    \begin{smallmatrix}
        1 & 1
    \end{smallmatrix} \right)
    \) is an epimorphism. Therefore one gets that \( \psi \) can be any endomorphism on \( J \), since the right square always composes to \( 0 \) either way.