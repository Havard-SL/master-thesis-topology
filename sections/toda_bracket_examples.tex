\section{Examples of Toda brackets}

Let \( R = \Fb_2C_2 \), with \( g \in C_2 \) being the generator.

Then \( J = \tuple{1 + g} \) is the only ideal of \( R \).

First note that \( J \) is not projective, since the short exact sequence:

\begin{center}
	\begin{tikzpicture}
		\diagram{m}{1cm}{1cm}{
			J & R & J \\
		};
	
		\draw[math]
			(m-1-1) edge[hook] node {\iota} (m-1-2)
			(m-1-2) edge[two heads] node {\phi} (m-1-3);
	\end{tikzpicture}
\end{center}

Where, \( \iota \) is the inclusion, and \( \phi \) is the map:

\begin{align*}
    \phi: R &\to J \\
    0, 1 + g &\mapsto 0 \\
    1, g &\mapsto 1 + g
\end{align*}

But \( \phi \) does not split, since \( \iota \) is the only monomorphism of \( J \) into \( R \), but composes to \( 0 \) with \( \phi \). Therefore \( J \) cannot be projective, since every epimorphism into a projective module splits.

Furthermore, since \( \phi \) is an epimorphism with kernel \( J \), one gets from the third isomorphism theorem that \( \frac{R}{J} \cong J \).

\subsection{Example 1}

The first example I calculated was the Toda bracket of the following diagram:

\begin{center}
	\begin{tikzpicture}
		\diagram{m}{1cm}{1cm}{
				J & J & J & J \\
		};

		\draw[math]
			(m-1-1) edge node {\Id} (m-1-2)
			(m-1-2) edge node {0} (m-1-3)
			(m-1-3) edge node {\Id} (m-1-4);
	\end{tikzpicture}
\end{center}

The cone of \( \Id_J \) is the pushout of \( (\Id_J, \iota_J) \), where \( \iota_J \) is a monomorphism into the injective/projective module \( R \).

This is by construction \( \frac{R \oplus J}{\sim} \), where \( (0, 1+g) \sim (1+g, 0) \). This is isomorphic to \( R \).

The supension of \( J \) is the cokernel of \( \iota_J \). In \( \Mod(R) \), this is isomorphic to \( \frac{R}{J} \).

Using the cofiber-cofiber definition, one gets the following diagram:

\begin{center}
	\begin{tikzpicture}
		\diagram{m}{1cm}{1cm} {
			J & J & R & {\frac{R}{J}} \\
			J & J & J & J \\
		};

		\draw[math]
			(m-1-1) edge node {\Id} (m-1-2)
				edge[equal] (m-2-1)
			(m-1-2) edge (m-1-3)
				edge[equal] (m-2-2)
			(m-1-3) edge (m-1-4)
				edge node {\rho} (m-2-3)
			(m-1-4) edge node {\psi} (m-2-4)

			(m-2-1) edge node {\Id} (m-2-2)
			(m-2-2) edge node {0} (m-2-3)
			(m-2-3) edge node {\Id} (m-2-4);
	\end{tikzpicture}
\end{center}

However, since the diagram is in \( \StMod(R) \), one has that \( R \cong 0 \), and therefore \( \rho = 0 \). And from earlier one has that \( \frac{R}{J} \cong J \).

This gives the following diagram in \( \StMod(R) \):

\begin{center}
	\begin{tikzpicture}
		\diagram{m}{1cm}{1cm} {
			J & J & 0 & J \\
			J & J & J & J \\
		};

		\draw[math]
			(m-1-1) edge node {\Id} (m-1-2)
				edge[equal] (m-2-1)
			(m-1-2) edge node {0} (m-1-3)
				edge[equal] (m-2-2)
			(m-1-3) edge node {0} (m-1-4)
				edge node {0} (m-2-3)
			(m-1-4) edge node {\psi} (m-2-4)

			(m-2-1) edge node {\Id} (m-2-2)
			(m-2-2) edge node {0} (m-2-3)
			(m-2-3) edge node {\Id} (m-2-4);
	\end{tikzpicture}
\end{center}

But this shows that any endomorphism on \( J \) makes the rightmost square commute, and is therefore in the Toda bracket (up to pre-composition by an isomorphism \( \frac{R}{J} \to J \)) of \( \toda{\Id_J, 0, \Id_J} \). 

Toda brackets are denoted up to pre/post-composition of isomorphism, one can write that \( \toda{\Id_J, 0, \Id_J} = \StMod(R)(J, J) \).

\subsection{Example 2}

Want to show that \( \toda{\Id_J, \Id_J, \Id_J} = \emptyset \) since there is no distinguished triangle one can put in the cofiber-cofiber definition such that the squares commute.

The cone of the identity map is \( 0 \) as seen before (and in general in a triangulated category, the cone of an isomorphism is isomorphic to \( 0 \) (TODO)), and \( \Sigma(J) \cong J \):

\begin{center}
	\begin{tikzpicture}
		\diagram{m}{1cm}{1cm} {
			J & J & 0 & J \\
			J & J & J & J \\
		};

		\draw[math]
			(m-1-1) edge node {\Id} (m-1-2)
				edge[equal] (m-2-1)
			(m-1-2) edge node {0} (m-1-3)
				edge[equal] (m-2-2)
			(m-1-3) edge node {0} (m-1-4)
				edge[squiggly] node {0} (m-2-3)
			(m-1-4) edge node {\psi} (m-2-4)

			(m-2-1) edge node {\Id} (m-2-2)
			(m-2-2) edge node {\Id} (m-2-3)
			(m-2-3) edge node {\Id} (m-2-4);
	\end{tikzpicture}
\end{center}

There is no squiggly map in the diagram above that can make the middle square commute, unless \( J \cong 0 \).

Therefore \( \toda{\Id_J, \Id_J, \Id_J} = \emptyset \). And one has that (TODO: Cite, Supervisors said so) in general, for the Toda bracket \( \toda{f_3, f_2, f_1} \), if \( f_3 \circ f_2 \neq 0 \) or \( f_2 \circ f_1 \neq 0 \), then the Toda bracket will be empty:

\begin{theorem}
	Let \( f_1, f_2, f_3 \) be three composable maps in any triangulated category, \( \Tc \):

	\begin{center}
		\begin{tikzpicture}
			\diagram{m}{1cm}{1cm} {
				X_1 & X_2 & X_3 & X_4 \\
			};

			\draw[math]
				(m-1-1) edge node {f_1} (m-1-2)
				(m-1-2) edge node {f_2} (m-1-3)
				(m-1-3) edge node {f_3} (m-1-4);
		\end{tikzpicture}
	\end{center}

	such that \( f_2 \circ f_1 \neq 0 \) or \( f_3 \circ f_2 \neq 0 \).

	Then \( \toda{f_3, f_2, f_1} = \emptyset \).
\end{theorem}
\begin{proof}
	Assume that \( \toda{f_3, f_2, f_1} \neq \emptyset \). Then from the definition of Toda bracket there exists maps \(\alpha, \beta, \phi, \psi \) such that the following diagram commutes:

	\begin{center}
		\begin{tikzpicture}
			\diagram{m}{1cm}{1cm} {
				X_1 & X_2 & C_{f_1} & \Sigma(X_1) \\
				X_1 & X_2 & X_3 & X_4 \\
			};

			\draw[math]
				(m-1-1) edge node {f_1} (m-1-2)
					edge[equal]	(m-2-1)
				(m-1-2) edge node {\alpha} (m-1-3)
					edge[equal] (m-2-2)
				(m-1-3) edge node {\beta} (m-1-4)
					edge node {\phi} (m-2-3)
				(m-1-4) edge node {\psi} (m-2-4)

				(m-2-1) edge node {f_1} (m-2-2)
				(m-2-2) edge node {f_2} (m-2-3)
				(m-2-3) edge node {f_3} (m-2-4);
		\end{tikzpicture}
	\end{center}

	Split the proof into two different contradictions:

	\begin{itemize}
		\item{
			Case 1:

			Assume that \( f_2 \circ f_1 \neq 0 \). Then one has that
			\[
				\phi \circ \alpha \circ f_1 = \phi \circ 0 = 0
			\]
			since \( \alpha, f_1 \) are two composable maps from the same distinguished triangle. (TODO: Cite or prove?)

			But from commutativity of the diagram one also has
			\[
				\phi \circ \alpha \circ f_1 = f_2 \circ f_1 \neq 0.
			\]
			Which is a contradiction, so \( f_2 \circ f_1 = 0 \).
		}
		\item{
			Case 2:

			Assume that \( f_3 \circ f_2 \neq 0 \). Then one has that
			\[
				0 = \psi \circ 0 = \psi \circ \beta \circ \alpha = f_3 \circ \phi \circ \alpha = f_3 \circ f_2 \neq 0.
			\]
			Which is also a contradiction.
		}
	\end{itemize}

	Therefore both \( f_2 \circ f_1 = 0 \) and \( f_3 \circ f_2 = 0 \) if \( \toda{f_3, f_2, f_1} \neq \emptyset \), which is contrapositive to the statement in the theorem.

\end{proof}

\subsection{Example 3}

Let \( \toda{f_3, 0, \Id} \) be a well defined Toda bracket. Then one has the following diagram:

\begin{center}
	\begin{tikzpicture}
		\diagram{m}{1cm}{1cm} {
			{X_1} & {X_1} & 0 & {\Sigma(X_1)} \\
			{X_1} & {X_1} & {X_2} & {X_3} \\
		};

		\draw[math]
			(m-1-1) edge node {\Id} (m-1-2)
				edge[equal] (m-2-1)
			(m-1-2) edge (m-1-3)
				edge[equal] (m-2-2)
			(m-1-3) edge (m-1-4)
				edge (m-2-3)
			(m-1-4) edge node {\phi} (m-2-4)

			(m-2-1) edge node {\Id} (m-2-2)
			(m-2-2) edge node {0} (m-2-3)
			(m-2-3) edge node {f_3} (m-2-4);
	\end{tikzpicture}
\end{center}

Here one has that any possible \( \psi: \Sigma(X_1) \to X_3 \) will make the right square commute. Therefore \( \toda{f_3, 0, \Id} = \Tc(\Sigma(X_1), X_3) \).

\subsection{Example 4}

Want to find the Toda bracket of the following maps using the cofiber-cofiber definition:

\begin{center}
	\begin{tikzpicture}
		\diagram{m}{1cm}{1cm} {
			J & {J \oplus J} & {J \oplus J} & J \\
		};

		\draw[math]
			(m-1-1) edge node {\begin{psmallmatrix} 1 \\ 1 \end{psmallmatrix}} (m-1-2)
			(m-1-2) edge node {\begin{psmallmatrix} 1 & 1 \\ 1 & 1 \end{psmallmatrix}} (m-1-3)
			(m-1-3) edge node {\begin{psmallmatrix} 1 & 1 \end{psmallmatrix}} (m-1-4);
	\end{tikzpicture}
\end{center}

First need to find the standard triangle of \( \begin{psmallmatrix} 1 \\ 1 \end{psmallmatrix}: J \to J \oplus J \):

The cone is defined as the pushout of the following diagram:

\begin{center}
	\begin{tikzpicture}
		\diagram{m}{1cm}{1cm} {
			J & R \\
			J \oplus J \\
		};

		\draw[math]
			(m-1-1) edge[hook] node {\iota_J} (m-1-2)
				edge[swap] node {\begin{psmallmatrix} 1 \\ 1 \end{psmallmatrix}} (m-2-1);
	\end{tikzpicture}
\end{center}

Where \( \iota_J \) is a monomorphism into a injective module. Here, this map is chosen to be the only monomorphism from \( J \to R \), namely \( \iota \).

In the category of modules, the pushout becomes:

\begin{center}
	\begin{tikzpicture}
		\diagram{m}{1cm}{1cm} {
			J & R \\
			J \oplus J & \frac{J \oplus J \oplus R}{\sim} \\
		};

		\draw[math]
			(m-1-1) edge[hook] node {\iota_J} (m-1-2)
				edge[swap] node {\begin{psmallmatrix} 1 \\ 1 \end{psmallmatrix}} (m-2-1)
			(m-1-2) edge node {\gamma} (m-2-2)

			(m-2-1) edge[hook] node {\rho} (m-2-2);
	\end{tikzpicture}
\end{center}

Where \( (1 + g, 1 + g, 0) \sim (0, 0, 1 + g) \).

The map \( \rho \) is given as the composition:
\begin{center}
	\begin{tikzpicture}
		\diagram{m}{1cm}{1cm} {
			J \oplus J & J \oplus J \oplus R & \frac{J \oplus J \oplus R}{\sim} \\
		};

		\draw[math]
			(m-1-1) edge[hook] node {i} (m-1-2)
			(m-1-2) edge[two heads] node {\pi} (m-1-3);
	\end{tikzpicture}
\end{center}

Where \( i \) is the embedding, and \( \pi \) is the quotient epimorphism.

One can check that the pushout is isomorphic to \( J \oplus R \) via the map:

\begin{align*}
	\alpha: \frac{J \oplus J \oplus R}{\sim} &\to J \oplus R \\
	(0, 0, r) &\mapsto (0, r) \\
	(1 + g, 0, r) &\mapsto (1 + g, r) \\
	(0, 1 + g, r) &\mapsto (1 + g, r) \\
	(1+ + g, 1 + g, r) &\mapsto (0, r) \\
\end{align*}

Therefore, by checking the map \( \alpha \circ \rho \) one can see that it becomes the map:
\[ 
	(\begin{psmallmatrix}
		1 & 1 \\
	\end{psmallmatrix}, 0):  J \oplus J \to J \oplus R
\]

Doing a similar argument for \( \gamma \), one gets that it is simply the embedding \( \pi \).

Furthermore, the map \( J \oplus R \to \Sigma(J) \cong \frac{R}{J} \) is given as the unique pushout map \( \beta \), satisfying the following commutative diagram:

\begin{center}
	\begin{tikzpicture}
		\diagram{m}{1cm}{2cm} {
			J & R & \frac{R}{J} \\
			J \oplus J & J \oplus R \\
		};

		\draw[math]
			(m-1-1) edge[hook] node {\iota_J} (m-1-2)
				edge[swap] node {\begin{psmallmatrix} 1 \\ 1 \end{psmallmatrix}} (m-2-1)
			(m-1-2) edge[two heads] node {\pi_J} (m-1-3)
				edge node {\pi} (m-2-2)

			(m-2-1) edge[hook] node {(\begin{psmallmatrix} 1 & 1 \\ \end{psmallmatrix}, 0)} (m-2-2)
				edge[curve={height=60pt}] node {0} (m-1-3)
			(m-2-2) edge node {\beta} (m-1-3);
	\end{tikzpicture}
\end{center}

One can check that a candidate for the map \( \beta \) is \( (0, \pi) \). And since \( \beta \) is unique, this is the only map making the diagram commute.

Therefore the standard triangle becomes:

\begin{center}
	\begin{tikzpicture}
		\diagram{m}{1cm}{2cm} {
			J & J \oplus J & J \oplus R & \frac{R}{J} \\
		};

		\draw[math]
			(m-1-1) edge node {\begin{psmallmatrix} 1 \\ 1 \end{psmallmatrix}} (m-1-2)
			(m-1-2) edge node {(\begin{psmallmatrix} 1 & 1 \\ \end{psmallmatrix}, 0)} (m-1-3)
			(m-1-3) edge node {(0, \pi)} (m-1-4);
	\end{tikzpicture}
\end{center}

However, since the objects and morphisms are in \( \StMod(R) \), one has that \( R \cong 0 \), and by using the isomorphism \( \frac{R}{J} \cong J \), it becomes:

\begin{center}
	\begin{tikzpicture}
		\diagram{m}{1cm}{1cm} {
			J & J \oplus J & J & J \\
		};

		\draw[math]
			(m-1-1) edge node {\begin{psmallmatrix} 1 \\ 1 \end{psmallmatrix}} (m-1-2)
			(m-1-2) edge node {\begin{psmallmatrix} 1 & 1 \\ \end{psmallmatrix}} (m-1-3)
			(m-1-3) edge node {0} (m-1-4);
	\end{tikzpicture}
\end{center}

Using the cofiber-cofiber definition, one gets the following commutative diagram:

\begin{center}
	\begin{tikzpicture}
		\diagram{m}{1cm}{1cm} {
			J & {J \oplus J} & J & J \\
			J & {J \oplus J} & {J \oplus J} & J \\
		};

		\draw[math]
			(m-1-1) edge node {\begin{psmallmatrix} 1 \\ 1 \end{psmallmatrix}} (m-1-2)
				edge[equal] (m-2-1)
			(m-1-2) edge node {{\begin{psmallmatrix} 1 & 1 \end{psmallmatrix}}} (m-1-3)
				edge[equal] (m-2-2)
			(m-1-3) edge node {0} (m-1-4)
				edge node {\phi} (m-2-3)
			(m-1-4) edge node {\psi} (m-2-4)

			(m-2-1) edge node {\begin{psmallmatrix} 1 \\ 1 \end{psmallmatrix}} (m-2-2)
			(m-2-2) edge node {\begin{psmallmatrix} 1 & 1 \\ 1 & 1 \end{psmallmatrix}} (m-2-3)
			(m-2-3) edge node {\begin{psmallmatrix} 1 & 1 \end{psmallmatrix}} (m-2-4);
	\end{tikzpicture}
\end{center}

Where the top row is distinguished.

Firstly, using the fact that \( \begin{psmallmatrix}
		1 & 1 \\
	\end{psmallmatrix}: J \oplus J \to J \) is an epimorphism, one gets that \( \phi \circ \begin{psmallmatrix}
		1 & 1 \\
	\end{psmallmatrix} = \begin{psmallmatrix}
		1 \\ 1 \\
	\end{psmallmatrix} \begin{psmallmatrix}
		1 & 1 \\
	\end{psmallmatrix} \implies \phi = \begin{psmallmatrix}
		1 \\ 1 \\
	\end{psmallmatrix} \) by the epimorphism property.

And secondly one has that \( \begin{psmallmatrix}
		1 & 1 \\
	\end{psmallmatrix} \circ \phi = \begin{psmallmatrix}
		1 & 1 \\
	\end{psmallmatrix} \circ \begin{psmallmatrix}
		1 \\ 1 \\
	\end{psmallmatrix} = 0 \), so the Toda-bracket is non-empty.

Finally one can see that for any \( \psi: J \to J \), the right square will commute, and so the toda bracket becomes \( \StMod(R)(J, J) \).