\section{Properties of \texorpdfstring{\( \Stmod(\Fb_3 C_3) \)}{Stmod(F\_3C\_3)}} % TODO: Not boldface, the math-operator and \Fb doesnt become bold.
% TODO another class possible if P = M^n?

% Isomorphic to path-algebra with relations

\begin{definition}
    Let \( \Fb_3 \) be the finite field with \( 3 \) elements.
\end{definition}

\begin{definition}
    Let \( C_3 \) be the cyclic group with \( 3 \) elements.
\end{definition}

\begin{definition}
    Let \( \Fb_3 C_3 \) be the group algebra of \( \Fb_3 \) over \( C_3 \).
\end{definition}

\begin{theorem}
    Let \( \Gamma = \)
    \begin{tikzpicture}
        \diagram{m}{1cm}{1cm} {
            1 \\
        };

        \draw[math]
            (m-1-1) edge[in=150, out=30, looseness=4.8] node[swap] {\alpha} (m-1-1);
    \end{tikzpicture}
    be a quiver.
    Let \( g \in \Fb_3 C_3 \) be the generator of \( C_3 \).
    
    Then the algebra homomorphism defined as follows
    \begin{align*}
        \phi: \Fb_3 C_3 &\to \frac{\Fb_3\Gamma}{(\alpha^3)} \\
        1 &\mapsto e \\
        g &\mapsto \alpha + 1
    \end{align*}
    is an isomorphism from \( \Fb_3 C_3 \) to \( \frac{\Fb_3\Gamma}{(\alpha^3)} \).    
\end{theorem}
\begin{proof}
    TODO
\end{proof}

\begin{theorem}
    One has that \( \Fb_3 C_3 \) has three distinct indecomposable modules up to isomorphism.
    
    They are isomorphic to \( \frac{\Fb_3C_3}{(g - 1)}, \frac{\Fb_3C_3}{(g - 1)^2}, \) and \( \Fb_3C_3 \), respectively.
\end{theorem}
\begin{proof}
    TODO
\end{proof}

\begin{theorem} \label{thm:f_3c_3_mu}
    Let \( \mu: \frac{\Fb_3C_3}{(g - 1)^2} \to \frac{\Fb_3C_3}{(g - 1)} \) be defined as TODO

    Then \( \Hom_{\Fb_3C_3}\tuple{\frac{\Fb_3C_3}{(g - 1)^2}, \frac{\Fb_3C_3}{(g - 1)}} = \set{0, \mu} \)
\end{theorem}
\begin{proof}
    TODO
\end{proof}

\begin{lemma} \label{lem:classify_m}
    One has that \( \frac{\Fb_3C_3}{(g - 1)^2} = \set{[\pm 1], [\pm g], [\pm g^2], [\pm (g - 1)]} \)
\end{lemma}
\begin{proof}
    TODO
\end{proof}

\begin{theorem} \label{thm:f_3c_3_nu} % NEEDED TODO
    Let 
    \begin{align*}
        \nu: \frac{\Fb_3C_3}{(g - 1)} &\to \frac{\Fb_3C_3}{(g - 1)^2} \\
        [\tilde{a}1] \mapsto [\tilde{a}(g - 1)]
    \end{align*}

    Then \( \nu \) is well defined, and \( \Stmod(\Fb_3C_3)\tuple{\frac{\Fb_3C_3}{(g - 1)}, \frac{\Fb_3C_3}{(g - 1)^2}} = \set{0, \pm \nu} \)
\end{theorem}
\begin{proof}
    First, want to prove that \( \nu \) is well defined.

    Class-representation:

    Let \( \bar{a} \in \frac{\Fb_3C_3}{(g - 1)} \). Then one has that \( \bar{a} = [a1 + bg + cg^2] = [a1 + b1 + c1] = [(a + b + c)1] \). And so any element of \( \frac{\Fb_3C_3}{(g - 1)} \) can be written as \( [\tilde{a}1] \) for som \( \tilde{a} \in \Fb_3 \).

    Group homomorphism:

    Let \( g \in \Fb_3C_3 \) be the generator of \( C_3 \). Then \( \nu([a1] + [b1]) = \nu([(a + b)1]) = [(a + b)(g - 1)] = [a(g - 1) + b(g - 1)] = [a(g - 1)] + [b(g - 1)] = \nu([a]) + \nu([b]) \).

    Module homomorphism:

    Let \( g \in \Fb_3C_3 \) be the generator of \( C_3 \). Then one has that for any \( r = a + bg + cg^2 \in \Fb_3C_3 \) that \( r[\tilde{a}1] = [r\tilde{a}1] = [(a + b + c)\tilde{a}1] \), and
    \begin{align*}
        r\nu([\tilde{a}1]) &= r[\tilde{a}(g - 1)] \\
        &= [r\tilde{a}(g - 1)] \\
        &= [(a + bg + cg^2)\tilde{a}(g - 1)] \\
        &\vdots \\
        &= [(a + b + c)\tilde{a}(g - 1)] \\
        &= \nu(r[\tilde{a}1])
    \end{align*}

    Independent of choice of class-representation:

    Let \( [a] = [b] \) in \( \frac{\Fb_3C_3}{(g - 1)} \). Then \( [a] - [b] = [r(g-1)] \) for some \( r \in \Fb_3C_3 \).

    Then \( \nu([a]) - \nu([b]) = \nu([a] - [b]) = \nu([r(g - 1)]) = [r(g - 1)^2] = [0] \).

    Secondly, want to show that \( \Stmod(\Fb_3C_3)\tuple{\frac{\Fb_3C_3}{(g - 1)}, \frac{\Fb_3C_3}{(g - 1)^2}} = \set{0, \pm \nu} \).

    Let \( f \in \Stmod(\Fb_3C_3)\tuple{\frac{\Fb_3C_3}{(g - 1)}, \frac{\Fb_3C_3}{(g - 1)^2}} \), and let \( r \in \Fb_3C_3 \).
    
    Then \( f \) is a module homomorphism, and one therefore has that \( f([a1]) = a1 f([1]) \). And so \( f \) is fully determined by the value of \( f([1]) \). And from \autoref{lem:classify_m} on has that \( \frac{\Fb_3C_3}{(g - 1)^2} = \set{[\pm (g - 1)], [\pm 1], [\pm g], [\pm g^2]} \).

    However one can see that if \( [1] \) is sent to \( [\pm 1], [\pm g] \), or \( [\pm g^2] \), then \( f([0]) = f([g - 1]) = (g - 1)f([1]) \neq [0] = [g^2 + g + 1] = [(g - 1)^2] \), which would imply that \( f \) is \emph{not} a group homomorphism.

    Therefore \( [1] \) has to be sent to either \( [(g - 1)] \) or \( [-(g - 1)] \). These maps, being \( \nu \) and \( -\nu \) respectivly.
\end{proof}

\begin{theorem} \label{thm:f_3c_3_mu_circ_nu_zero}
    Let \( \mu \) and \( \nu \) be as in \autoref{thm:f_3c_3_mu} and \autoref{thm:f_3c_3_nu} respectively.

    Then \( \mu \circ \nu = 0 \).
\end{theorem}
\begin{proof}
    TODO
\end{proof}

\begin{theorem} \label{thm:f_3c_3_decomposition}
    Any object in \( \Stmod(\Fb_3C_3) \) is isomorphic to a direct sum \( \tuple{ \frac{\Fb_3C_3}{(g - 1)^2} }^n \oplus \tuple{ \frac{\Fb_3C_3}{(g - 1)} }^m \) for \( n, m \in \Nb_0 \), where taking the power of \( 0 \) gives the zero object. \sloppy
\end{theorem}
\begin{proof}
    TODO
\end{proof}

\begin{theorem} % Is this even true?? TODO
    Let \( \Ac \) be an additive category.
    
    Then 
    \begin{enumerate}
        \item {
            The functor \( \Hom(-_1 \oplus -_2, -_3): \Ac^{op} \times \Ac^{op} \times \Ac \to \Ab \) is naturally isomorphic to the functor \( \Hom(-_1, -_3) \oplus \Hom(-_2, -_3) \). With the isomorphism

            \begin{center}
                \begin{tikzpicture}
                    \diagram{m}{1cm}{2cm} {
                        \Hom(-_1 \oplus -_2, -_3) & \Hom(-_1, -_3) \oplus \Hom(-_2, -_3) \\
                    };

                    \draw[math]
                        (m-1-1) edge[curve={height=-25pt}] node {\begin{psmallmatrix} (i_{-_1})^* \\ (i_{-_2})^* \end{psmallmatrix}} (m-1-2)
                        (m-1-2) edge[curve={height=-25pt}] node {\begin{psmallmatrix} (p_{-_1})^* & (p_{-_2})^* \end{psmallmatrix}} (m-1-1);
                \end{tikzpicture}
            \end{center}
        }
        \item {
            The functor \( \Hom(-_1, -_2 \oplus -_3): \Ac^{op} \times \Ac \times \Ac \to \Ab \) is naturally isomorphic to the functor \( \Hom(-_1, -_2) \oplus \Hom(-_1, -_3) \). With the isomorphism

            \begin{center}
                \begin{tikzpicture}
                    \diagram{m}{1cm}{2cm} {
                        \Hom(-_1, -_2 \oplus -_3) & \Hom(-_1, -_2) \oplus \Hom(-_1, -_3) \\
                    };

                    \draw[math]
                        (m-1-1) edge[curve={height=-25pt}] node {\begin{psmallmatrix} (p_{-_2})_* \\ (p_{-_3})_* \end{psmallmatrix}} (m-1-2)
                        (m-1-2) edge[curve={height=-25pt}] node {\begin{psmallmatrix} (i_{-_2})_* & (i_{-_3})_* \end{psmallmatrix}} (m-1-1);
                \end{tikzpicture}
            \end{center}
        }
    \end{enumerate}
\end{theorem}
\begin{proof}
    TODO: 
    
    https://ncatlab.org/nlab/show/additive+functor

    https://ncatlab.org/nlab/show/hom-functor+preserves+limits
\end{proof}

% Remark abuse of post-composition notation? TODO

\begin{theorem} \label{thm:hom_direct_sum_map_nice}
    Let \( \Ac \) be an additive category, and let \( A, B, C \in \Ac \). Let \( f: B \to C \). Let \( n \in \Nb \).

    Then:
    \begin{enumerate}
        \item {
            The following diagram commutes
            \begin{center}
                \begin{tikzpicture}
                    \diagram{m}{3cm}{2cm} {
                        \Ac(A, B^n) & \Ac(A, C^n) \\
                        \Ac(A, B)^n & \Ac(A, C)^n \\
                    };
        
                    \draw[math]
                        (m-1-1) edge node {(f^n)_*} (m-1-2)
                            edge node { \begin{psmallmatrix} (p_1^B)_* \\ (p_2^B)_* \\ \vdots \\ (p_n^B)_* \end{psmallmatrix} } (m-2-1)
                        (m-1-2) edge node { \begin{psmallmatrix} (p_1^C)_* \\ (p_2^C)_* \\ \vdots \\ (p_n^C)_* \end{psmallmatrix} } (m-2-2)
                        
                        (m-2-1) edge node {(f_*)^n} (m-2-2);
                \end{tikzpicture}
            \end{center}
        }
        \item {
            The following diagram commutes
            \begin{center}
                \begin{tikzpicture}
                    \diagram{m}{3cm}{2cm} {
                        \Ac(A^n, B) & \Ac(A^n, C) \\
                        \Ac(A, B)^n & \Ac(A, C)^n \\
                    };
        
                    \draw[math]
                        (m-1-1) edge node {f_*} (m-1-2)
                            edge node { \begin{psmallmatrix} (i_1^A)^* \\ (i_2^A)^* \\ \vdots \\ (i_n^A)^* \end{psmallmatrix} } (m-2-1)
                        (m-1-2) edge node { \begin{psmallmatrix} (i_1^A)^* \\ (i_2^A)^* \\ \vdots \\ (i_n^A)^* \end{psmallmatrix} } (m-2-2)
                        
                        (m-2-1) edge node {(f_*)^n} (m-2-2);
                \end{tikzpicture}
            \end{center}
        }
    \end{enumerate}
\end{theorem}
\begin{proof}
    TODO: Notes, example projective class.
\end{proof}


% Show that there only is one non-trivial map between the indecomposable non-projective modules (Might not be true)

% Show that Sigma(S) = M, Sigma(M) = S.

% Show that S -> M -> S is exact.
% Show that S -> M -> S -> M is dist.

\begin{theorem} \label{thm:F_functor}
    Let \( F: \Stmod(\Fb_3C_3) \to \Stmod(\Fb_3C_3) \) be a ``functor'' that takes any object \( A \in \Stmod(\Fb_3C_3) \) and maps it to its decomposition by \autoref{thm:f_3c_3_decomposition}, and morphisms are induced by the isomorphisms from the decomposition. I.e. there are some \( n, m \in \Nb_0 \) such that \( A \mapsto F(A) = \tuple{ \frac{\Fb_3C_3}{(g - 1)^2} }^n \oplus \tuple{ \frac{\Fb_3C_3}{(g - 1)} }^m \). And furthermore for \( f \in \Stmod(\Fb_3C_3)\tuple{A, B} \), one has \( f \mapsto F(f) = \phi_A^{-1} \circ f \circ \phi_B \), where \( \phi_A \) and \( \phi_B \) are the chosen isomorphisms between \( A \) and it's decomposition, and \( B \) and it's decomposition, respectively.

    Then \( F \) is a well defined functor.
\end{theorem}
\begin{proof}
    TODO
\end{proof}

% Independent of choice of F, possibly, due to krull-remak-schmidt uniqe decomposition? TODO
\begin{definition} % Unholy definition.
    Let the functor \( F \) be as in \autoref{thm:F_functor}.

    Define the set \( P_B \) as follows:

    For any object \( A, B \in \Stmod(\Fb_3C_3) \), and for any \( f \in \Stmod(\Fb_3C_3)\tuple{A, B} \).

    Then \( f \in P_B \iff \)

    For
    \begin{align*}
        F(f) &\in \Stmod(\Fb_3C_3)\tuple{\tuple{ \frac{\Fb_3C_3}{(g - 1)^2} }^{n_A} \oplus \tuple{ \frac{\Fb_3C_3}{(g - 1)} }^{m_A}, \tuple{ \frac{\Fb_3C_3}{(g - 1)^2} }^{n_B} \oplus \tuple{ \frac{\Fb_3C_3}{(g - 1)} }^{m_B}} \\
        &\stackrel{\phi}{\cong} \Stmod(\Fb_3C_3)\tuple{ \tuple{ \frac{\Fb_3C_3}{(g - 1)^2} }, \tuple{ \frac{\Fb_3C_3}{(g - 1)^2} } }^{n_An_B} \\
        &\oplus \Stmod(\Fb_3C_3)\tuple{ \tuple{ \frac{\Fb_3C_3}{(g - 1)^2} }, \tuple{ \frac{\Fb_3C_3}{(g - 1)} } }^{n_Am_B} \\
        &\oplus \Stmod(\Fb_3C_3)\tuple{ \tuple{ \frac{\Fb_3C_3}{(g - 1)} }, \tuple{ \frac{\Fb_3C_3}{(g - 1)^2} } }^{m_An_B} \\
        &\oplus \Stmod(\Fb_3C_3)\tuple{ \tuple{ \frac{\Fb_3C_3}{(g - 1)} }, \tuple{ \frac{\Fb_3C_3}{(g - 1)} } }^{m_Am_B}
    \end{align*}
    With \( p_1, p_2, p_3, p_4 \) being the projection maps down to the four main components.

    One has that all of the following are true:
    \begin{enumerate}
        \item {
            \( p_1 \circ \phi \circ F(f) = \oplus_{n_An_B} \set{0, \cdot(\pm (g - 1))} \).
        }
        \item {
            \( p_3 \circ \phi \circ F(f) = 0 \).
        }
        \item {
            \( p_4 \circ \phi \circ F(f) = 0 \).
        }
    \end{enumerate}
\end{definition}

\begin{remark}
    Definitely needf to remark on the previous definition.... 
    
    Any morphism can not have any component from S to S, or S to M, as well as they can only have one certain component from M to M.

    TODO: Fix
\end{remark}

\begin{theorem} % Major rewrite of first part! TODO
    Let \( \Pc = \set{ \tuple{ \frac{\Fb_3C_3}{(g - 1)} }^n \mid n \in \Nb } \union \set{ 0 } \). Let \( \Nc = P_B \union \set{ 0 } \).

    Then \( \tuple{ \Pc, \Nc} \) is a projective class in \( \Stmod(\Fb_3C_3) \).
\end{theorem}
\begin{proof}
    Need to show that \( \tuple{ \Pc, \Nc } \) satisfies the three properties in \autoref{def:projective_class}.

    \begin{enumerate}
        \item {
            \( \tuple{ \Rightarrow } \) Let \( f \in \Nc \).

            If \( f = 0 \), then the statement is true.

            Assume \( f \neq 0 \). Then \( f = \tuple{ \mu }^n \in \Nc \) for some \( n \in \Nb \).
            
            Let \( P \in \Pc \)
            
            If \( P = 0 \), then the statement is true.

            Assume \( P = \tuple{ \frac{\Fb_3C_3}{(g - 1)} }^m \) for some \( m \in \Nb \).

            Then using \autoref{thm:hom_direct_sum_map_nice}, the following map
            \[
                \tuple{ \tuple{ \mu }^n }_* : \Stmod(\Fb_3C_3)\tuple{P^m, \tuple{\frac{\Fb_3C_3}{(g - 1)^2}}^n } \to \Stmod(\Fb_3C_3)\tuple{P^m, \tuple{\frac{\Fb_3C_3}{(g - 1)}}^n }.
            \]
            can be written as
            \[
                \tuple{ \tuple{ \mu }^n }_* = \phi \circ (\mu_*)^{nm} \circ \psi
            \]
            where \( \phi \) and \( \psi \) are isomorphisms.

            From \autoref{thm:f_3c_3_nu} one has that \( \Stmod(\Fb_3C_3)\tuple{ \frac{\Fb_3C_3}{(g - 1)}, \frac{\Fb_3C_3}{(g - 1)^2} } = \set{ 0, \pm \nu } \).

            And from \autoref{thm:f_3c_3_mu_circ_nu_zero} one gets that applying \( (\mu)_* \) to either \( 0 \) or \( \pm \nu \) gives the zero map.

            Then
            \[
                \tuple{ \tuple{ \mu }^n }_* = \phi \circ (\mu_*)^{nm} \circ \psi = \phi \circ 0^{nm} \circ \psi = 0
            \]

            Since the choice of \( P \in \Pc \) was arbitrary. One must have that it is correct for every \( P \in \Pc \). And likewise for any \( f \in \Nc \).

            \( ( \Leftarrow ) \) Let \( A, B \in \Stmod(\Fb_3C_3) \) be two arbitrary objects, and let \( f \in \Stmod(\Fb_3C_3)\tuple{ A, B } \).

            If \( f = 0 \), then the statement is true. 
            
            Assume \( f \neq 0 \) (if possible). % Allowed to say this? TODO

            From \autoref{thm:f_3c_3_decomposition}, one has integers \( n_A, m_A, n_B, m_b \in \Nc_0 \) such that \( A \cong \tuple{ \frac{\Fb_3C_3}{(g - 1)^2} }^{n_A} \oplus \tuple{ \frac{\Fb_3C_3}{(g - 1)} }^{m_A} \), and \( B \cong \tuple{ \frac{\Fb_3C_3}{(g - 1)^2} }^{n_B} \oplus \tuple{ \frac{\Fb_3C_3}{(g - 1)} }^{m_B} \).

            TODO show that f is in Nc.
        }
    \end{enumerate}
\end{proof}