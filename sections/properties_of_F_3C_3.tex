\section{Properties of \texorpdfstring{\( \Stmod{\Fb_3 C_3} \)}{Stmod(F\_3C\_3)}} % TODO: Not boldface, the math-operator and \Fb doesnt become bold.
% TODO another class possible if P = M^n?

% Isomorphic to path-algebra with relations

\begin{notation}
    Let \( \Fb_3 \) be the finite field with \( 3 \) elements.
\end{notation}

\begin{notation}
    Let \( C_3 \) be the cyclic group with \( 3 \) elements.
\end{notation}

\begin{notation}
    Let \( \Fb_3 C_3 \) be the group algebra of \( \Fb_3 \) over \( C_3 \).
\end{notation}

\begin{lemma}
    Let \( \Gamma = \)
    \begin{tikzpicture}
        \diagram{m}{1cm}{1cm} {
            1 \\
        };

        \draw[math]
            (m-1-1) edge[in=150, out=30, looseness=4.8] node[swap] {\alpha} (m-1-1);
    \end{tikzpicture}
    be a quiver.
    Let \( g \in \Fb_3 C_3 \) be the generator of \( C_3 \).
    
    Then the algebra homomorphism defined as follows
    \begin{align*}
        \phi: \Fb_3 C_3 &\to \frac{\Fb_3\Gamma}{(\alpha^3)} \\
        1 &\mapsto e \\
        g &\mapsto \alpha + 1
    \end{align*}
    is an isomorphism from \( \Fb_3 C_3 \) to \( \frac{\Fb_3\Gamma}{(\alpha^3)} \).    
\end{lemma}
\begin{proof}
    TODO
\end{proof}

\begin{definition}
    Let \( S := \frac{\Fb_3C_3}{(g - 1)} \), let \( M := \frac{\Fb_3C_3}{(g - 1)^2} \), and let \( P := {\Fb_3C_3}_{\Fb_3C_3} \).
\end{definition}

\begin{lemma}
    One has that \( \Fb_3 C_3 \) has three distinct indecomposable modules up to isomorphism.
    
    They are isomorphic to \( S, M, \) and \( P \) with \( S \) a simple module, and \( P \) a projective module.

    Furthermore, neither \( S \) or \( M \) are projective modules.
\end{lemma}
\begin{proof}
    TODO
\end{proof}

\begin{definition}
    Let \( \Mc := \Stmod{\Fb_3C_3} \).
\end{definition}

\begin{lemma} \label{thm:f_3c_3_mu}
    Let \( \mu: M \to S \) be defined as TODO

    Then \( \Mc\tuple{M, S} = \set{0, \pm\mu} \)
\end{lemma}
\begin{proof}
    TODO
\end{proof}

\begin{lemma} \label{lem:classify_m}
    One has that \( M = \set{[\pm 1], [\pm g], [\pm g^2], [\pm (g - 1)]} \)
\end{lemma}
\begin{proof}
    TODO
\end{proof}

\begin{lemma} \label{thm:f_3c_3_nu} % NEEDED TODO
    Let 
    \begin{align*}
        \nu: S &\to M \\
        [\tilde{a}1] \mapsto [\tilde{a}(g - 1)]
    \end{align*}

    Then \( \nu \) is well defined, and \( \Mc\tuple{S, M} = \set{0, \pm \nu} \)
\end{lemma}
\begin{proof}
    First, want to prove that \( \nu \) is well defined.

    Class-representation:

    Let \( \bar{a} \in S \). Then one has that \( \bar{a} = [a1 + bg + cg^2] = [a1 + b1 + c1] = [(a + b + c)1] \). And so any element of \( S \) can be written as \( [\tilde{a}1] \) for som \( \tilde{a} \in \Fb_3 \).

    Group homomorphism:

    Let \( g \in \Fb_3C_3 \) be the generator of \( C_3 \). Then \( \nu([a1] + [b1]) = \nu([(a + b)1]) = [(a + b)(g - 1)] = [a(g - 1) + b(g - 1)] = [a(g - 1)] + [b(g - 1)] = \nu([a]) + \nu([b]) \).

    Module homomorphism:

    Let \( g \in \Fb_3C_3 \) be the generator of \( C_3 \). Then one has that for any \( r = a + bg + cg^2 \in \Fb_3C_3 \) that \( r[\tilde{a}1] = [r\tilde{a}1] = [(a + b + c)\tilde{a}1] \), and
    \begin{align*}
        r\nu([\tilde{a}1]) &= r[\tilde{a}(g - 1)] \\
        &= [r\tilde{a}(g - 1)] \\
        &= [(a + bg + cg^2)\tilde{a}(g - 1)] \\
        &\vdots \\
        &= [(a + b + c)\tilde{a}(g - 1)] \\
        &= \nu(r[\tilde{a}1])
    \end{align*}

    Independent of choice of class-representation:

    Let \( [a] = [b] \) in \( S \). Then \( [a] - [b] = [r(g-1)] \) for some \( r \in \Fb_3C_3 \).

    Then \( \nu([a]) - \nu([b]) = \nu([a] - [b]) = \nu([r(g - 1)]) = [r(g - 1)^2] = [0] \).

    Secondly, want to show that \( \Mc\tuple{S, M} = \set{0, \pm \nu} \).

    Let \( f \in \Mc\tuple{S, M} \), and let \( r \in \Fb_3C_3 \).
    
    Then \( f \) is a module homomorphism, and one therefore has that \( f([a1]) = a1 f([1]) \). And so \( f \) is fully determined by the value of \( f([1]) \). 
    
    And from \autoref{lem:classify_m} on has that \( M = \set{[\pm (g - 1)], [\pm 1], [\pm g], [\pm g^2]} \).

    However one can see that if \( [1] \) is sent to \( [\pm 1], [\pm g] \), or \( [\pm g^2] \), then \( f([0]) = f([g - 1]) = (g - 1)f([1]) \neq [0] = [g^2 + g + 1] = [(g - 1)^2] \), which would imply that \( f \) is \emph{not} a group homomorphism.

    Therefore \( [1] \) has to be sent to either \( [(g - 1)] \) or \( [-(g - 1)] \). These maps, being \( \nu \) and \( -\nu \) respectivly.
\end{proof}

\begin{lemma} \label{thm:f_3c_3_mu_circ_nu_zero}
    Let \( \mu \) and \( \nu \) be as in \autoref{thm:f_3c_3_mu} and \autoref{thm:f_3c_3_nu} respectively.

    Then \( \mu \circ \nu = 0 \).
\end{lemma}
\begin{proof}
    TODO
\end{proof}

\begin{lemma} \label{lem:g-1_circ_nu_equals_zero}
    Let \( \nu \) be as in \autoref{thm:f_3c_3_nu}. Let
    \begin{align*}
        \cdot(g - 1): M &\to M \\
        [a1 + bg] &\mapsto [(g - 1)(a1 + bg)]
    \end{align*}

    Then \( \cdot(g - 1) \circ \nu = 0 \) 
\end{lemma}
\begin{proof}
    TODO
\end{proof}

\begin{lemma} \label{thm:f_3c_3_decomposition}
    Any object in \( \Mc \) is isomorphic to a direct sum \( \tuple{ S }^n \oplus \tuple{ M }^m \) for \( n, m \in \Nb_0 \), where taking the power of \( 0 \) gives the zero object.
\end{lemma}
\begin{proof}
    TODO
\end{proof}

\begin{notation}
    Let \( \Ac \) be an additive category, and let \( A, B \in \Ac \).

    Then denote the projection from \( A \oplus B \) to \( A \) by \( p_{A \oplus B}^A \), and denote the inclusion of \( A \) into \( A \oplus B \) by \( i_A^{A \oplus B} \).

    Furthermore, for any \( n \in \Nb \), and any \( i \in \set{1, 2, \dots, n} \), denote the projection from \( A^n \) to the \( i \)-th summand of \( A^n \) by \( p_{A^n}^{A_i} \). Similarly, denote the inclusion of the \( i \)-th summand of \( A^n \) into \( A^n \) by \( i_{A_i}^{A^n} \).
\end{notation}

\begin{lemma} % Is this even true?? TODO
    Let \( \Ac \) be an additive category.
    
    Then 
    \begin{enumerate}
        \item {
            The functor \( \Ac(-_1 \oplus -_2, -_3): \Ac^{op} \times \Ac^{op} \times \Ac \to \Ab \) is naturally isomorphic to the functor \( \Ac(-_1, -_3) \oplus \Ac(-_2, -_3) \). With the isomorphism

            \begin{center}
                \begin{tikzpicture}
                    \diagram{m}{1cm}{2cm} {
                        \Ac(-_1 \oplus -_2, -_3) & \Ac(-_1, -_3) \oplus \Ac(-_2, -_3) \\
                    };

                    \draw[math]
                        (m-1-1) edge[curve={height=-25pt}] node {\begin{psmallmatrix} (i_{-_1}^{-_1 \oplus -_2})^* \\ (i_{-_2}^{-_1 \oplus -_2})^* \end{psmallmatrix}} (m-1-2)
                        (m-1-2) edge[curve={height=-25pt}] node {\begin{psmallmatrix} (p_{-_1 \oplus -_2}^{-_1})^* & (p_{-_1 \oplus -_2}^{-_2})^* \end{psmallmatrix}} (m-1-1);
                \end{tikzpicture}
            \end{center}
        }
        \item {
            The functor \( \Ac(-_1, -_2 \oplus -_3): \Ac^{op} \times \Ac \times \Ac \to \Ab \) is naturally isomorphic to the functor \( \Ac(-_1, -_2) \oplus \Ac(-_1, -_3) \). With the isomorphism

            \begin{center}
                \begin{tikzpicture}
                    \diagram{m}{1cm}{2cm} {
                        \Ac(-_1, -_2 \oplus -_3) & \Ac(-_1, -_2) \oplus \Ac(-_1, -_3) \\
                    };

                    \draw[math]
                        (m-1-1) edge[curve={height=-25pt}] node {\begin{psmallmatrix} (p_{-_2 \oplus -_3}^{-_2})_* \\ (p_{-_2 \oplus -_3}^{-_3})_* \end{psmallmatrix}} (m-1-2)
                        (m-1-2) edge[curve={height=-25pt}] node {\begin{psmallmatrix} (i_{-_2}^{-_2 \oplus -_3})_* & (i_{-_3}^{-_2 \oplus -_3})_* \end{psmallmatrix}} (m-1-1);
                \end{tikzpicture}
            \end{center}
        }
    \end{enumerate}
\end{lemma}
\begin{proof}
    TODO: 
    
    https://ncatlab.org/nlab/show/additive+functor

    https://ncatlab.org/nlab/show/hom-functor+preserves+limits
\end{proof}

% Remark abuse of post-composition notation? TODO

% Problems with same n on lhs and rhs? TODO
\begin{lemma} \label{thm:hom_direct_sum_map_nice}
    Let \( \Ac \) be an additive category, and let \( A, B, C \in \Ac \). Let \( f: B \to C \). Let \( n \in \Nb \).

    Then:
    \begin{enumerate}
        \item {
            The following diagram commutes
            \begin{center}
                \begin{tikzpicture}
                    \diagram{m}{3cm}{2cm} {
                        \Ac(A, B^n) & \Ac(A, C^n) \\
                        \Ac(A, B)^n & \Ac(A, C)^n \\
                    };
        
                    \draw[math]
                        (m-1-1) edge node {(f^n)_*} (m-1-2)
                            edge node { \begin{psmallmatrix} (p_{B^n}^{B_1})_* \\ (p_{B^n}^{B_2})_* \\ \vdots \\ (p_{B^n}^{B_n})_* \end{psmallmatrix} } (m-2-1)
                        (m-1-2) edge node { \begin{psmallmatrix} (p_{C^n}^{C_1})_* \\ (p_{C^n}^{C_2})_* \\ \vdots \\ (p_{C^n}^{C_n})_* \end{psmallmatrix} } (m-2-2)
                        
                        (m-2-1) edge node {(f_*)^n} (m-2-2);
                \end{tikzpicture}
            \end{center}
        }
        \item {
            The following diagram commutes
            \begin{center}
                \begin{tikzpicture}
                    \diagram{m}{3cm}{2cm} {
                        \Ac(A^n, B) & \Ac(A^n, C) \\
                        \Ac(A, B)^n & \Ac(A, C)^n \\
                    };
        
                    \draw[math]
                        (m-1-1) edge node {f_*} (m-1-2)
                            edge node { \begin{psmallmatrix} (i_{A_1}^{A^n})^* \\ (i_{A_2}^{A^n})^* \\ \vdots \\ (i_{A_n}^{A^n})^* \end{psmallmatrix} } (m-2-1)
                        (m-1-2) edge node { \begin{psmallmatrix} (i_{A_1}^{A^n})^* \\ (i_{A_2}^{A^n})^* \\ \vdots \\ (i_{A_n}^{A^n})^* \end{psmallmatrix} } (m-2-2)
                        
                        (m-2-1) edge node {(f_*)^n} (m-2-2);
                \end{tikzpicture}
            \end{center}
        }
    \end{enumerate}
\end{lemma}
\begin{proof}
    TODO: Notes, example projective class.
\end{proof}


% Show that there only is one non-trivial map between the indecomposable non-projective modules (Might not be true)

% Show that Sigma(S) = M, Sigma(M) = S.

% Show that S -> M -> S is exact.
% Show that S -> M -> S -> M is dist.

\begin{definition} \label{thm:F_functor} % Assignment OK? TODO
    Define \( F: \Mc \to \Mc \) be an assignment that takes any object \( A \in \Mc \) and maps it to its decomposition by \autoref{thm:f_3c_3_decomposition}, and morphisms are induced by the isomorphisms from the decomposition. 
    
    I.e. there are some \( n, m \in \Nb_0 \) such that \( A \mapsto F(A) = S^n \oplus M^m \). And furthermore for \( f \in \Mc\tuple{A, B} \), one has \( f \mapsto F(f) = \phi_A^{-1} \circ f \circ \phi_B \), where \( \phi_A \) and \( \phi_B \) are the chosen isomorphisms between \( A \) and it's decomposition, and \( B \) and it's decomposition, respectively.
\end{definition}

\begin{lemma}
    The assignment \( F: \Mc \to \Mc \) from \autoref{thm:F_functor} is a functor.
\end{lemma}
\begin{proof}
    TODO
\end{proof}

\begin{lemma} \label{lem:projection_unique} % Is even true?? TODO
    Let \( \Ac \) be an additive category. Let \( A, B, C \in \Ac \).

    Then \( p_{A \oplus B}^A \circ p_{A \oplus B \oplus C}^{A \oplus B} = p_{A \oplus B \oplus C}^A \), and \( i_{A \oplus B}^{A \oplus B \oplus C} \circ i_A^{A \oplus B} = i_A^{A \oplus B \oplus C} \).
\end{lemma}
\begin{proof}
    TODO
\end{proof}

\begin{lemma} \label{lem:only_non_surjective_M_to_M}
    One has that \( 0 \) and \( \pm(\cdot(g - 1)) \) are the only non-isomorphism morphisms in \( \Mc\tuple{M, M} \).
\end{lemma}
\begin{proof}
    TODO
\end{proof}

\begin{remark} \label{rem:big_iso}
    By using \autoref{thm:hom_direct_sum_map_nice} one gets that the following diagram
    \begin{center}
        \begin{tikzpicture}
            \diagram{m}{1cm}{1cm} {
                \Mc\tuple{S^{n_A} \oplus M^{m_A}, S^{n_B} \oplus M^{m_B}} \\
                \Mc\tuple{S^{n_A}, S^{n_B} \oplus M^{m_B}} \oplus \Mc\tuple{M^{m_A}, S^{n_B} \oplus M^{m_B}} \\
                \Mc\tuple{S^{n_A}, S^{n_B}} \oplus \Mc\tuple{S^{n_A}, M^{m_B}} \oplus \Mc\tuple{M^{m_A}, S^{n_B}} \oplus \Mc\tuple{M^{m_A}, M^{m_B}} \\
                \Mc\tuple{S, S}^{n_A n_B} \oplus \Mc\tuple{S, M}^{n_A m_B} \oplus \Mc\tuple{M, S}^{m_A n_B} \oplus \Mc\tuple{M, M}^{m_A m_B} \\
            };

            \draw[math]
                (m-1-1) edge node[marking, below] {\sim} node {g_1} (m-2-1)

                (m-2-1) edge node[marking, below] {\sim} node {g_2} (m-3-1)

                (m-3-1) edge node[marking, below] {\sim} node {g_3} (m-4-1);
        \end{tikzpicture}
    \end{center}
    
    
    Composing, one gets the isomorphism \( \phi = g_3 \circ g_2 \circ g_1 \) such that
    \begin{align*}
        &\Mc\tuple{S^{n_A} \oplus M^{m_A}, S^{n_B} \oplus M^{m_B}} \\
        &\stackrel{\phi}{\cong} \Mc\tuple{ S, S }^{n_An_B} \\
        &\oplus \Mc\tuple{ S, M }^{n_Am_B} \\
        &\oplus \Mc\tuple{ M, S }^{m_An_B} \\
        &\oplus \Mc\tuple{ M, M }^{m_Am_B}.
    \end{align*}

    Where
    \[
        g_1 =
        \begin{pmatrix}
            \tuple{ i_{S^{n_A}}^{S^{n_A} \oplus M^{m_A}} }^* \\
            \tuple{ i_{M^{m_A}}^{S^{n_A} \oplus M^{m_A}} }^*
        \end{pmatrix},
    \]
    \[
        g_2 =
        \begin{pmatrix}
            \tuple{ p_{S^{n_B} \oplus M^{m_B}}^{S^{n_B}} }_* \\
            \tuple{ p_{S^{n_B} \oplus M^{m_B}}^{M^{m_B}} }_*
        \end{pmatrix}
        \oplus
        \begin{pmatrix}
            \tuple{ p_{S^{n_B} \oplus M^{m_B}}^{S^{n_B}} }_* \\
            \tuple{ p_{S^{n_B} \oplus M^{m_B}}^{M^{m_B}} }_*
        \end{pmatrix}
    \]
    \begin{multline*}
        g_3 =
        \begin{pmatrix}
            (p_{S^{n_B}}^{S_1})_* \circ (i_{S_1}^{S^{n_A}})^* \\
            (p_{S^{n_B}}^{S_2})_* \circ (i_{S_1}^{S^{n_A}})^* \\
            \vdots \\
            (p_{S^{n_B}}^{S_{n_B}})_* \circ (i_{S_{n_A}}^{S^{n_A}})^*
        \end{pmatrix}
        \oplus
        \begin{pmatrix}
            (p_{M^{m_B}}^{M_1})_* \circ (i_{S_1}^{S^{n_A}})^* \\
            (p_{M^{m_B}}^{M_2})_* \circ (i_{S_1}^{S^{n_A}})^* \\
            \vdots \\
            (p_{M^{m_B}}^{M_{m_B}})_* \circ (i_{S_{n_A}}^{S^{n_A}})^*
        \end{pmatrix} \\
        \oplus
        \begin{pmatrix}
            (p_{S^{n_B}}^{S_1})_* \circ (i_{M_1}^{M^{m_A}})^* \\
            (p_{S^{n_B}}^{S_2})_* \circ (i_{M_1}^{M^{m_A}})^* \\
            \vdots \\
            (p_{S^{n_B}}^{S_{n_B}})_* \circ (i_{M_{m_A}}^{M^{m_A}})^*
        \end{pmatrix}
        \oplus
        \begin{pmatrix}
            (p_{M^{m_B}}^{M_1})_* \circ (i_{M_1}^{M^{m_A}})^* \\
            (p_{M^{m_B}}^{M_2})_* \circ (i_{M_1}^{M^{m_A}})^* \\
            \vdots \\
            (p_{M^{m_B}}^{M_{m_B}})_* \circ (i_{M_{m_A}}^{M^{m_A}})^*
        \end{pmatrix}
    \end{multline*}

    Where from \autoref{lem:projection_unique} one can calculate
    \[
        \phi = g_3 \circ g_2 \circ g_1 =
        \begin{pmatrix} % Wrong indexes, need some n_B and m_B's TODO
            \tuple{ p_{S^{n_B} \oplus M^{m_B}}^{S_1} }_* \circ \tuple{ i_{S_1}^{S^{n_A} \oplus M^{m_A}} }^* \\
            \vdots \\
            \tuple{ p_{S^{n_B} \oplus M^{m_B}}^{S_{n_B}} }_* \circ \tuple{ i_{S_{n_A}}^{S^{n_A} \oplus M^{m_A}} }^* \\
            \tuple{ p_{S^{n_B} \oplus M^{m_B}}^{M_1} }_* \circ \tuple{ i_{S_1}^{S^{n_A} \oplus M^{m_A}} }^* \\
            \vdots \\
            \tuple{ p_{S^{n_B} \oplus M^{m_B}}^{M_{m_B}} }_* \circ \tuple{ i_{S_{n_A}}^{S^{n_A} \oplus M^{m_A}} }^* \\
            \tuple{ p_{S^{n_B} \oplus M^{m_B}}^{S_1} }_* \circ \tuple{ i_{M_1}^{S^{n_A} \oplus M^{m_A}} }^* \\
            \vdots \\
            \tuple{ p_{S^{n_B} \oplus M^{m_B}}^{S_{n_B}} }_* \circ \tuple{ i_{M_{m_A}}^{S^{n_A} \oplus M^{m_A}} }^* \\
            \tuple{ p_{S^{n_B} \oplus M^{m_B}}^{M_1} }_* \circ \tuple{ i_{M_1}^{S^{n_A} \oplus M^{m_A}} }^* \\
            \vdots \\
            \tuple{ p_{S^{n_B} \oplus M^{m_B}}^{M_{m_B}} }_* \circ \tuple{ i_{M_{m_A}}^{S^{n_A} \oplus M^{m_A}} }^* \\
        \end{pmatrix}.
    \]

    Similarly one has that \( \phi^{-1} = g_1^{-1} \circ g_2^{-1} \circ g_3^{-1} \), where
    \[
        g_1^{-1} =
        \begin{pmatrix}
            \tuple{ p_{S^{n_A} \oplus M^{m_A}}^{S^{n_A}} }^*, & \tuple{ p_{S^{n_A} \oplus M^{m_A}}^{M^{m_A}} }^*
        \end{pmatrix},
    \]
    \[
        g_2^{-1} =
        \begin{pmatrix}
            \tuple{ i_{S^{n_B}}^{S^{n_B} \oplus M^{m_B}} }_*, & \tuple{ i_{M^{m_B}}^{S^{n_B} \oplus M^{m_B}} }_*
        \end{pmatrix}
        \oplus
        \begin{pmatrix}
            \tuple{ i_{S^{n_B}}^{S^{n_B} \oplus M^{m_B}} }_*, & \tuple{ i_{M^{m_B}}^{S^{n_B} \oplus M^{m_B}} }_*
        \end{pmatrix},
    \]
    and
    \begin{multline*}
        g_3^{-1} =
        \begin{pmatrix}
            \tuple{ p_{S^{n_A}}^{S_1} }^* \circ \tuple{ i_{S_1}^{S^{n_B}} }_*, & \tuple{ p_{S^{n_A}}^{S_1} }^* \circ \tuple{ i_{S_2}^{S^{n_B}} }_*, & \dots, & \tuple{ p_{S^{n_A}}^{S_{n_A}} }^* \circ \tuple{ i_{S_{n_B}}^{S^{n_B}} }_*
        \end{pmatrix} \\
        \oplus
        \begin{pmatrix}
            \tuple{ p_{S^{n_A}}^{S_1} }^* \circ \tuple{ i_{M_1}^{M^{m_B}} }_*, & \tuple{ p_{S^{n_A}}^{S_1} }^* \circ \tuple{ i_{M_2}^{M^{m_B}} }_*, & \dots, & \tuple{ p_{S^{n_A}}^{S_{n_A}} }^* \circ \tuple{ i_{M_{m_B}}^{M^{m_B}} }_*
        \end{pmatrix} \\
        \oplus
        \begin{pmatrix}
            \tuple{ p_{M^{m_A}}^{M_1} }^* \circ \tuple{ i_{S_1}^{S^{n_B}} }_*, & \tuple{ p_{M^{m_A}}^{M_1} }^* \circ \tuple{ i_{S_2}^{S^{n_B}} }_*, & \dots, & \tuple{ p_{M^{m_A}}^{M_{m_A}} }^* \circ \tuple{ i_{S_{n_B}}^{S^{n_B}} }_*
        \end{pmatrix} \\
        \oplus
        \begin{pmatrix}
            \tuple{ p_{M^{m_A}}^{M_1} }^* \circ \tuple{ i_{M_1}^{M^{m_B}} }_*, & \tuple{ p_{M^{m_A}}^{M_1} }^* \circ \tuple{ i_{M_2}^{M^{m_B}} }_*, & \dots, & \tuple{ p_{M^{m_A}}^{M_{m_A}} }^* \circ \tuple{ i_{M_{m_B}}^{M^{m_B}} }_*
        \end{pmatrix}.
    \end{multline*}

    This gives
    \begin{multline*}
        \phi^{-1} =  g_1^{-1} \circ g_2^{-1} \circ g_3^{-1} = \\
        \biggl(
            \tuple{ p_{S^{n_A} \oplus M^{m_A}}^{S_1} }^* \circ \tuple{ i_{S_1}^{S^{n_B} \oplus M^{m_B}} }_*, \tuple{ p_{S^{n_A} \oplus M^{m_A}}^{S_1} }^* \circ \tuple{ i_{S_2}^{S^{n_B} \oplus M^{m_B}} }_*, \dots, \tuple{ p_{S^{n_A} \oplus M^{m_A}}^{S_{n_A}} }^* \circ \tuple{ i_{S_{n_B}}^{S^{n_B} \oplus M^{m_B}} }_*, \\
            \tuple{ p_{S^{n_A} \oplus M^{m_A}}^{S_1} }^* \circ \tuple{ i_{M_1}^{S^{n_B} \oplus M^{m_B}} }_*, \tuple{ p_{S^{n_A} \oplus M^{m_A}}^{S_1} }^* \circ \tuple{ i_{M_2}^{S^{n_B} \oplus M^{m_B}} }_*, \dots, \tuple{ p_{S^{n_A} \oplus M^{m_A}}^{S_{n_A}} }^* \circ \tuple{ i_{M_{m_B}}^{S^{n_B} \oplus M^{m_B}} }_*, \\
            \tuple{ p_{S^{n_A} \oplus M^{m_A}}^{M_1} }^* \circ \tuple{ i_{S_1}^{S^{n_B} \oplus M^{m_B}} }_*, \tuple{ p_{S^{n_A} \oplus M^{m_A}}^{M_1} }^* \circ \tuple{ i_{S_2}^{S^{n_B} \oplus M^{m_B}} }_*, \dots, \tuple{ p_{S^{n_A} \oplus M^{m_A}}^{M_{m_A}} }^* \circ \tuple{ i_{S_{n_B}}^{S^{n_B} \oplus M^{m_B}} }_*, \\
            \tuple{ p_{S^{n_A} \oplus M^{m_A}}^{M_1} }^* \circ \tuple{ i_{M_1}^{S^{n_B} \oplus M^{m_B}} }_*, \tuple{ p_{S^{n_A} \oplus M^{m_A}}^{M_1} }^* \circ \tuple{ i_{M_2}^{S^{n_B} \oplus M^{m_B}} }_*, \dots, \tuple{ p_{S^{n_A} \oplus M^{m_A}}^{M_{m_A}} }^* \circ \tuple{ i_{M_{m_B}}^{S^{n_B} \oplus M^{m_B}} }_*
        \biggr)
    \end{multline*}

    Furthermore let \( p_{S, S}, p_{S, M}, p_{M, S} \) and \( p_{M, M} \) be the projection maps from
    \[
        \Mc\tuple{S, S}^{n_A n_B} \oplus \Mc\tuple{S, M}^{n_A m_B} \oplus \Mc\tuple{M, S}^{m_A n_B} \oplus \Mc\tuple{M, M}^{m_A m_B}
    \]
    to \( \Mc\tuple{S, S}^{n_A n_B}, \Mc\tuple{S, M}^{n_A m_B}, \Mc\tuple{M, S}^{m_A n_B}, \) and \( \Mc\tuple{M, M}^{m_A m_B}  \) respectively.
\end{remark}

% Independent of choice of F, possibly, due to krull-remak-schmidt uniqe decomposition? TODO
\begin{definition} \label{def:unholy} % Unholy definition.
    Let the functor \( F \) be as in \autoref{thm:F_functor}.  Let, \( \phi, p_{S, S}, p_{S, M}, p_{M, S}, p_{M, M} \) be as in \autoref{rem:big_iso}.

    Define the set \( P_B \) as follows:

    For any object \( A, B \in \Mc \), and for any \( f \in \Mc\tuple{A, B} \).

    Then \( f \in P_B \iff \)

    All of the following are true:
    \begin{enumerate}
        \item {
            \( p_{S, S} \circ \phi \circ F(f) = 0 \).
        }
        \item {
            \( p_{S, M} \circ \phi \circ F(f) = 0 \).
        }
        \item {
            \( p_{M, M} \circ \phi \circ F(f) = \set{0, \cdot(\pm (g - 1))}^{m_A m_B} \).
        }
    \end{enumerate}
\end{definition}

\begin{remark}
    Definitely needf to remark on the previous definition.... 
    
    Any morphism can not have any component from S to S, or S to M, as well as they can only have one certain component from M to M.

    TODO: Fix
\end{remark}

\begin{remark} \label{rem:F_properties}
    Given the following commutative diagram by the definition of \( F \)
    \begin{center}
        \begin{tikzpicture}
            \diagram{m}{1cm}{1cm} {
                A & B \\
                S^{n_A} \oplus M^{m_A} & S^{n_B} \oplus M^{m_B} \\
            };

            \draw[math]
                (m-1-1) edge node {f} (m-1-2)
                    edge node {\psi_A} node[marking, below] {\sim} (m-2-1)
                (m-1-2) edge node {\psi_B} node[marking, below] {\sim} (m-2-2)

                (m-2-1) edge node {F(f)} (m-2-2);
        \end{tikzpicture}
    \end{center}
    Applying the functor \( \Mc(S^j \oplus M^k, -) \) to this, one gets the following commutative diagram
    \begin{center}
        \begin{tikzpicture}
            \diagram{m}{1cm}{2cm} {
                \Mc(S^j \oplus M^k, A) & \Mc(S^j \oplus M^k, B) \\
                \Mc\tuple{S^j \oplus M^k, S^{n_A} \oplus M^{m_A} } & \Mc\tuple{S^j \oplus M^k, S^{n_B} \oplus M^{m_B} } \\
            };

            \draw[math]
                (m-1-1) edge node {f_*} (m-1-2)
                    edge node {(\psi_A)_*} node[marking, below] {\sim} (m-2-1)
                (m-1-2) edge node {(\psi_B)_*} node[marking, below] {\sim} (m-2-2)

                (m-2-1) edge node {F(f)_*} (m-2-2);
        \end{tikzpicture}
    \end{center}
    Expanding downwards (and flipping it over to make it fit), using \autoref{rem:big_iso}, one gets the following commutative diagram
    \begin{equation}
        \begin{tikzpicture}[every node/.style={scale=0.85}] \label{tikz:f_star}
            \diagram{m}{1cm}{1cm} {
                \Mc\tuple{S^j \oplus M^k, S^{n_A} \oplus M^{m_A} } & \Mc\tuple{S, S}^{j n_A} \oplus \Mc\tuple{S, M}^{j m_A} \oplus \Mc\tuple{M, S}^{k n_A} \oplus \Mc\tuple{M, M}^{k m_A} \\
                \Mc\tuple{S^j \oplus M^k, S^{n_B} \oplus M^{m_B} } & \Mc\tuple{S, S}^{j n_B} \oplus \Mc\tuple{S, M}^{j m_B} \oplus \Mc\tuple{M, S}^{k n_B} \oplus \Mc\tuple{M, M}^{k m_B} \\
            };

            \draw[math]
                (m-1-1) edge node {\phi_A} node[marking, below] {\sim} (m-1-2)
                    edge node {F(f)_*} (m-2-1)
                (m-1-2) edge node {\phi_B \circ F(f)_* \circ \phi_A^{-1}} (m-2-2)
                    
                (m-2-1) edge node {\phi_B} node[marking, below] {\sim} (m-2-2);
        \end{tikzpicture}
    \end{equation}
    Furthermore, one has
    \[
        \phi_B \circ F(f)_* \circ \phi_A^{-1} =
        TODO
    \]
    where 
\end{remark}

% Possible to simplify proof by saying that postcomposing with a map in Nc is the same as precomposing with any map from the hom-sets?
% Seems so obvious that there should be a simpler way to do things. TODO
% Is i_A^A+B+C = i_A+B^A+B+C \circ i_A^A+B? TODO
\begin{example} % Major rewrite of first part! TODO
    Let \( \Pc = \set{ S^n \mid n \in \Nb } \union \set{ 0 } \). Let \( \Nc = P_B \union \set{ 0 } \).

    Then \( \tuple{ \Pc, \Nc} \) is a projective class in \( \Mc \).
\end{example}
\begin{proof}
    Need to show that \( \tuple{ \Pc, \Nc } \) satisfies the three properties in \autoref{def:projective_class}.

    \begin{enumerate}
        \item {
            \( \tuple{ \Rightarrow } \) Let \( f \in \Nc \).

            If \( f = 0 \), then the statement is true.

            Assume \( f \neq 0 \). Then \( f \in \Mc\tuple{A, B} \) for two non-zero modules \( A, B \in \Mc \), and satisfying \autoref{def:unholy}.
            
            Let \( \tilde{P} \in \Pc \)
            
            If \( \tilde{P} = 0 \), then the statement is true.

            Assume \( \tilde{P} = S^k \) for some \( k \in \Nb \).

            Then from \autoref{thm:hom_direct_sum_map_nice}, one gets the following commutative diagram
            \begin{center}
                \begin{tikzpicture}
                    \diagram{m}{1cm}{1cm} {
                        \Mc\tuple{S^k, A} & \Mc\tuple{S^k, B} \\
                        \Mc\tuple{S, A}^k & \Mc\tuple{S, B}^k \\
                    };

                    \draw[math]
                        (m-1-1) edge node {f_*} (m-1-2)
                            edge node[marking, above] {\sim} (m-2-1)
                        (m-1-2) edge node[marking, above] {\sim} (m-2-2)

                        (m-2-1) edge node {(f_*)^k} (m-2-2);
                \end{tikzpicture}
            \end{center}
            It suffices to check that \( f_*: \Mc\tuple{S, A} \to \Mc\tuple{S, B} \) is zero.

            Using \autoref{rem:F_properties} with \( j = 1 \) and \( k = 0 \), one gets the following morphisms:
            \[
                \phi_A^{-1} = \begin{pmatrix}
                    \tuple{ i_{S_1}^{S^{n_A} \oplus M^{m_A}} }_* & \dots & \tuple{ i_{S_{n_A}}^{S^{n_A} \oplus M^{m_A}} }_* & \tuple{ i_{M_1}^{S^{n_A} \oplus M^{m_A}} }_* & \dots & \tuple{ i_{M_{m_A}}^{S^{n_A} \oplus M^{m_A}} }_*
                \end{pmatrix}
            \]
            and
            \[
                \phi_B = \begin{pmatrix}
                    \tuple{ p_{S^{n_B} \oplus M^{m_B}}^{S_1} }_* \\
                    \vdots \\
                    \tuple{ p_{S^{n_B} \oplus M^{m_B}}^{S_{n_B}} }_* \\
                    \tuple{ p_{S^{n_B} \oplus M^{m_B}}^{M_1} }_* \\
                    \vdots \\
                    \tuple{ p_{S^{n_B} \oplus M^{m_B}}^{M_{m_B}} }_*
                \end{pmatrix}.
            \]
            This gives
            \[
                \phi_B \circ F(f)_* \circ \phi_A^{-1} =
                \begin{pmatrix}
                    F_1 & F_2 \\
                    F_3 & F_4
                \end{pmatrix}
            \]
            where
            \[
                F_1 = \begin{pmatrix}
                    \tuple{ p_{S^{n_B} \oplus M^{m_B}}^{S_1} \circ F(f) \circ i_{S_1}^{S^{n_A} \oplus M^{m_A}} }_* & \dots & \tuple{ p_{S^{n_B} \oplus M^{m_B}}^{S_1} \circ F(f) \circ i_{S_{n_A}}^{S^{n_A} \oplus M^{m_A}} }_* \\
                    \vdots & \ddots & \vdots \\
                    \tuple{ p_{S^{n_B} \oplus M^{m_B}}^{S_{n_B}} \circ F(f) \circ i_{S_1}^{S^{n_A} \oplus M^{m_A}} }_* & \dots & \tuple{ p_{S^{n_B} \oplus M^{m_B}}^{S_{n_B}} \circ F(f) \circ i_{S_{n_A}}^{S^{n_A} \oplus M^{m_A}} }_* \\
                \end{pmatrix},
            \]
            \[
                F_2 = \begin{pmatrix}
                    \tuple{ p_{S^{n_B} \oplus M^{m_B}}^{S_1} \circ F(f) \circ i_{M_1}^{S^{n_A} \oplus M^{m_A}} }_* & \dots & \tuple{ p_{S^{n_B} \oplus M^{m_B}}^{S_1} \circ F(f) \circ i_{M_{m_A}}^{S^{n_A} \oplus M^{m_A}} }_* \\
                    \vdots & \ddots & \vdots \\
                    \tuple{ p_{S^{n_B} \oplus M^{m_B}}^{S_{n_B}} \circ F(f) \circ i_{M_1}^{S^{n_A} \oplus M^{m_A}} }_* & \dots & \tuple{ p_{S^{n_B} \oplus M^{m_B}}^{S_{n_B}} \circ F(f) \circ i_{M_{m_A}}^{S^{n_A} \oplus M^{m_A}} }_* \\
                \end{pmatrix},
            \]
            \[
                F_3 = \begin{pmatrix}
                    \tuple{ p_{S^{n_B} \oplus M^{m_B}}^{M_1} \circ F(f) \circ i_{S_1}^{S^{n_A} \oplus M^{m_A}} }_* & \dots & \tuple{ p_{S^{n_B} \oplus M^{m_B}}^{M_1} \circ F(f) \circ i_{S_{n_A}}^{S^{n_A} \oplus M^{m_A}} }_* \\
                    \vdots & \ddots & \vdots \\
                    \tuple{ p_{S^{n_B} \oplus M^{m_B}}^{M_{m_B}} \circ F(f) \circ i_{S_1}^{S^{n_A} \oplus M^{m_A}} }_* & \dots & \tuple{ p_{S^{n_B} \oplus M^{m_B}}^{M_{m_B}} \circ F(f) \circ i_{S_{n_A}}^{S^{n_A} \oplus M^{m_A}} }_* \\
                \end{pmatrix},
            \]
            and
            \[
                F_4 = \begin{pmatrix}
                    \tuple{ p_{S^{n_B} \oplus M^{m_B}}^{M_1} \circ F(f) \circ i_{M_1}^{S^{n_A} \oplus M^{m_A}} }_* & \dots & \tuple{ p_{S^{n_B} \oplus M^{m_B}}^{M_1} \circ F(f) \circ i_{M_{m_A}}^{S^{n_A} \oplus M^{m_A}} }_* \\
                    \vdots & \ddots & \vdots \\
                    \tuple{ p_{S^{n_B} \oplus M^{m_B}}^{M_{m_B}} \circ F(f) \circ i_{M_1}^{S^{n_A} \oplus M^{m_A}} }_* & \dots & \tuple{ p_{S^{n_B} \oplus M^{m_B}}^{M_{m_B}} \circ F(f) \circ i_{M_{m_A}}^{S^{n_A} \oplus M^{m_A}} }_* \\
                \end{pmatrix}.
            \]
            But from the definition of \( P_B \), one has that
            \[
                \phi \circ F(f) = \begin{pmatrix}
                     p_{S^{n_A} \oplus M^{m_A}}^{S_1} \circ F(f) \circ i_{S_1}^{S^{n_A} \oplus M^{m_A}} \\
                    \vdots \\
                    p_{S^{n_A} \oplus M^{m_A}}^{S_{n_A}} \circ F(f) \circ i_{S_{n_A}}^{S^{n_A} \oplus M^{m_A}} \\
                    p_{S^{n_A} \oplus M^{m_A}}^{M_1} \circ F(f) \circ i_{S_1}^{S^{n_A} \oplus M^{m_A}} \\
                    \vdots \\
                    p_{S^{n_A} \oplus M^{m_A}}^{M_{m_A}} \circ F(f) \circ i_{S_{n_A}}^{S^{n_A} \oplus M^{m_A}} \\
                    p_{S^{n_A} \oplus M^{m_A}}^{S_1} \circ F(f) \circ i_{M_1}^{S^{n_A} \oplus M^{m_A}} \\
                    \vdots \\
                    p_{S^{n_A} \oplus M^{m_A}}^{S_{n_A}} \circ F(f) \circ i_{M_{m_A}}^{S^{n_A} \oplus M^{m_A}} \\
                    p_{S^{n_A} \oplus M^{m_A}}^{M_1} \circ F(f) \circ i_{M_1}^{S^{n_A} \oplus M^{m_A}} \\
                    \vdots \\
                    p_{S^{n_A} \oplus M^{m_A}}^{M_{m_A}} \circ F(f) \circ i_{M_{m_A}}^{S^{n_A} \oplus M^{m_A}}
                \end{pmatrix}
                = \begin{pmatrix}
                    0 \\
                    \vdots \\
                    0 \\
                    0 \\
                    \vdots \\
                    0 \\
                    p_{S^{n_A} \oplus M^{m_A}}^{S_1} \circ F(f) \circ i_{M_1}^{S^{n_A} \oplus M^{m_A}} \\
                    \vdots \\
                    p_{S^{n_A} \oplus M^{m_A}}^{S_{n_A}} \circ F(f) \circ i_{M_{m_A}}^{S^{n_A} \oplus M^{m_A}} \\
                    \set{0, \cdot(\pm(g - 1))} \\
                    \vdots \\
                    \set{0, \cdot(\pm(g - 1))}
                \end{pmatrix}.
            \]
            This implies that
            \[
                F_1 = 0, F_3 = 0,
            \]
            and
            \[
                F_4 = \begin{pmatrix}
                    \set{0, \cdot(\pm(g - 1))} & \dots & \set{0, \cdot(\pm(g - 1))} \\
                    \vdots & \ddots & \vdots \\
                    \set{0, \cdot(\pm(g - 1))} & \dots & \set{0, \cdot(\pm(g - 1))}
                \end{pmatrix}.
            \]
            Using this information about \( \psi_B \circ F(f)_* \circ \psi_A^{-1} \), observe its pointwise values for any \( \tuple{g_1, \dots, g_{n_A}, h_1, \dots, h_{m_A}} \in \Mc(S, S)^{n_A} \oplus \Mc(S, M)^{m_A} \).
            
            First note that by \autoref{thm:f_3c_3_nu}, one has that \( h_i = \set{0, \pm \nu} \) for all \( i \).

            Also note that by \autoref{lem:g-1_circ_nu_equals_zero} one has
            \[
                F_4 \begin{pmatrix}
                    \set{0, \pm \nu} \\
                    \vdots \\
                    \set{0, \pm \nu}
                \end{pmatrix} = 0.
            \]
            And by \autoref{thm:f_3c_3_mu} one has that
            \[
                p_{S^{n_B} \oplus M^{m_B}}^S \circ F(f) \circ i_M^{S^{n_A} \oplus M^{m_A}} \in \Mc(M, M) = \set{0, \pm \mu}
            \]
            And furthermore by \autoref{thm:f_3c_3_mu_circ_nu_zero} this implies that
            \[
                F_2 \begin{pmatrix}
                    \set{0, \pm \nu} \\
                    \vdots \\
                    \set{0, \pm \nu}
                \end{pmatrix} = 0.
            \]

            All in all, this implies
            \begin{multline*}
                \psi_B \circ F(f)_* \circ \psi_A^{-1} \tuple{g_1, \dots, g_{n_A}, h_1, \dots, h_{m_A}} \\
                = \begin{pmatrix}
                    F_1 & F_2 \\
                    F_3 & F_4
                \end{pmatrix} \tuple{g_1, \dots, g_{n_A}, \set{0, \pm \nu}, \dots, \set{0, \pm \nu}} \\
                = \begin{pmatrix}
                    0 & F_2 \begin{psmallmatrix}
                        \set{0, \pm \nu} \\
                        \vdots \\
                        \set{0, \pm \nu}
                    \end{psmallmatrix} \\
                    0 & F_4 \begin{psmallmatrix}
                        \set{0, \pm \nu} \\
                        \vdots \\
                        \set{0, \pm \nu}
                    \end{psmallmatrix} \\
                \end{pmatrix} 
                = 0.
            \end{multline*}

            So by commutativity of \autoref{tikz:f_star} as well as the vertical maps all being isomorphisms, it follows that \( f_* = 0 \).

            \( ( \Leftarrow ) \) Show this implication by a counter-positive argument.

            Assume \( f: A \to B \) with \( f \neq \Nc \)

            Then first of all, \( f \neq 0 \).

            Want to show that for all \( \tilde{P} \in \Pc \), one has that \( f_*: \Mc(\tilde{P}, A) \to \Mc(\tilde{P}, B)\) is non-zero.

            If \( \tilde{P} = 0 \), then \( f_* \) is zero, so assume that \( \tilde{P} \neq 0 \). Then there is some \( k \in \Nb \) such that \( \tilde{P} = S^k \).

            Using the diagram and maps from \autoref{tikz:f_star}, want to show that \( \psi_B \circ F(f)_* \circ \psi_A^{-1} \) is pointwise non-zero.

            In order to prove this, split the cases up by which point in \autoref{def:unholy} that we assume \( f \) not to fulfill:
            \begin{enumerate}
                \item {
                    Assume that \( p_{S, S} \circ \phi \circ F(f) \neq 0 \).

                    Then there is some \( j, k \in \Nb \) such that \( p_{S^{n_B} \oplus M^{m_B}}^{S_j} \circ F(f) \circ i_{S_k}^{S^{n_A} \oplus M^{m_A}} \neq 0 \) is non-zero.

                    Then there is a \( (n_A + m_A) \)-tuple that has all zeroes, except for \( \Id_S \) in the \( k \)-th coordinate. I.e. \( \alpha = \tuple{0, \dots, 0, \Id_S, 0, \dots, 0} \). Such that
                    \[
                        \psi_B \circ F(f)_* \circ \psi_A^{-1} \tuple{\alpha}
                        =
                        \begin{pmatrix}
                            F_1 & F_2 \\
                            F_3 & F_4
                        \end{pmatrix}
                        \tuple{0, \dots, 0, \Id_S, 0, \dots, 0}
                        = \beta
                    \]
                    where in the \( j \)-th coordinate of \( \beta \) there is a non-zero term \( p_{S^{n_B} \oplus M^{m_B}}^{S_j} \circ F(f) \circ i_{S_k}^{S^{n_A} \oplus M^{m_A}} \).

                    Then \( f \not\in \Nc \). 
                }
                \item {
                    Assume that \( p_{S, M} \circ \phi \circ F(f) \neq 0 \).

                    Then there is some \( j, k \in \Nb \) such that \( p_{S^{n_B} \oplus M^{m_B}}^{M_j} \circ F(f) \circ i_{S_k}^{S^{n_A} \oplus M^{m_A}} \) is non-zero.

                    Then there is a \( (n_A + m_A) \)-tuple that has all zeroes, except for \( \Id_S \) in the \( k \)-th coordinate. I.e. \( \alpha = \tuple{0, \dots, 0, \Id_S, 0, \dots, 0} \). Such that
                    \[
                        \psi_B \circ F(f)_* \circ \psi_A^{-1} \tuple{\alpha}
                        =
                        \begin{pmatrix}
                            F_1 & F_2 \\
                            F_3 & F_4
                        \end{pmatrix}
                        \tuple{0, \dots, 0, \Id_S, 0, \dots, 0}
                        = \beta
                    \]
                    where in the \( j + n_B \)-th coordinate of \( \beta \) there is a non-zero term \( p_{S^{n_B} \oplus M^{m_B}}^{M_j} \circ F(f) \circ i_{S_k}^{S^{n_A} \oplus M^{m_A}} \).
                }
                \item {
                    Assume that \( p_{M, M} \circ \phi \circ F(f) \neq \set{0, \cdot(\pm (g - 1))}^{m_A m_B} \).

                    By \autoref{lem:only_non_surjective_M_to_M} there is some \( j, k \in \Nb \) such that \( p_{S^{n_B} \oplus M^{m_B}}^{M_j} \circ F(f) \circ i_{S_k}^{M^{n_A} \oplus M^{m_A}} \) is an isomorphism.

                    Then there is a \( (n_A + m_A) \)-tuple that has all zeroes, except for \( \nu \) in the \( n_A + k \)-th coordinate. I.e. \( \alpha = \tuple{0, \dots, 0, \nu, 0, \dots, 0} \). Such that
                    \[
                        \psi_B \circ F(f)_* \circ \psi_A^{-1} \tuple{\alpha}
                        =
                        \begin{pmatrix}
                            F_1 & F_2 \\
                            F_3 & F_4
                        \end{pmatrix}
                        \tuple{0, \dots, 0, \nu, 0, \dots, 0}
                        = \beta
                    \]
                    where in the \( j + n_B \)-th coordinate of \( \beta \) there is a term \( p_{S^{n_B} \oplus M^{m_B}}^{M_j} \circ F(f) \circ i_{S_k}^{S^{n_A} \oplus M^{m_A}} \circ \nu \) that is non-zero because it is a non-zero morphism composed with an isomorphism.
                }
            \end{enumerate} 
        }
        \item {
            \( (\Rightarrow) \) Let \( \tilde{P} \in \Pc \).

            % If \( \tilde{P} = 0 \), then its true.

            % Assume \( \tilde{P} \neq 0 \), i.e. there is some \( k \in \Nb \) such that \( \tilde{P} = S^k \).

            Let \( A, B \in \Mc \), and let \( f \in \Mc(A, B) \intersect \Nc \). Then by point 1, \( (\Rightarrow) \) previously, one gets that \( f_* = 0 \).
            
            \( (\Leftarrow) \) Want to show this by a counter positive argument.

            Assume \( \tilde{P} \not\in \Pc \). That implies there is some \( k \in \Nb \) and \( j \in \Nb_0 \) such that \( \tilde{P} = S^j \oplus M^k \).

            If \( f = 0 \), then it's true. Assume \( f \neq 0 \).

            From \autoref{rem:F_properties}, it is sufficient to show that the map \( \phi_B \circ F(f)_* \circ \phi_A^{-1} \) is non-zero.


        }
    \end{enumerate}
\end{proof}