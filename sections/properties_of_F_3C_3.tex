\section{Properties of \texorpdfstring{\( \Stmod(\Fb_3 C_3) \)}{Stmod(F\_3C\_3)}} % TODO: Not boldface, the math-operator and \Fb doesnt become bold.

% Isomorphic to path-algebra with relations

\begin{definition}
    Let \( \Fb_3 \) be the finite field with \( 3 \) elements.
\end{definition}

\begin{definition}
    Let \( C_3 \) be the cyclic group with \( 3 \) elements.
\end{definition}

\begin{definition}
    Let \( \Fb_3 C_3 \) be the group algebra of \( \Fb_3 \) over \( C_3 \).
\end{definition}

\begin{theorem}
    Let \( \Gamma = \)
    \begin{tikzpicture}
        \diagram{m}{1cm}{1cm} {
            1 \\
        };

        \draw[math]
            (m-1-1) edge[in=150, out=30, looseness=4.8] node[swap] {\alpha} (m-1-1);
    \end{tikzpicture}
    be a quiver.
    Let \( g \in \Fb_3 C_3 \) be the generator of \( C_3 \).
    
    Then the algebra homomorphism defined as follows
    \begin{align*}
        \phi: \Fb_3 C_3 &\to \frac{\Fb_3\Gamma}{(\alpha^3)} \\
        1 &\mapsto e \\
        g &\mapsto \alpha + 1
    \end{align*}
    is an isomorphism from \( \Fb_3 C_3 \) to \( \frac{\Fb_3\Gamma}{(\alpha^3)} \).    
\end{theorem}
\begin{proof}
    TODO
\end{proof}

\begin{theorem}
    One has that \( \Fb_3 C_3 \) has three distinct indecomposable modules up to isomorphism.
    
    They are isomorphic to \( \frac{\Fb_3C_3}{(1 + g)}, \frac{\Fb_3C_3}{(1 + g)^2}, \) and \( \Fb_3C_3 \), respectively.
\end{theorem}
\begin{proof}
    TODO
\end{proof}

\begin{theorem} \label{thm:f_3c_3_mu}
    Let \( \mu: \frac{\Fb_3C_3}{(1 + g)^2} \to \frac{\Fb_3C_3}{(1 + g)} \) be defined as TODO

    Then \( \Hom_{\Fb_3C_3}\tuple{\frac{\Fb_3C_3}{(1 + g)^2}, \frac{\Fb_3C_3}{(1 + g)}} = \set{0, \mu} \)
\end{theorem}
\begin{proof}
    TODO
\end{proof}

\begin{theorem}
    Any object in \( \Stmod(\Fb_3C_3) \) is isomorphic to an object of the form \( \tuple{ \frac{\Fb_3C_3}{(1 + g)^2} }^n \oplus \tuple{ \frac{\Fb_3C_3}{(1 + g)} }^m \) for \( n, m \in \Nb_0 \), where taking the power of \( 0 \) gives the zero object.
\end{theorem}
\begin{proof}
    TODO
\end{proof}

\begin{theorem} % Naturality?
    Let \( \Ac \) be an additive category.
    
    Then 
    \begin{enumerate}
        \item {
            The functor \( \Hom(-_1 \oplus -_2, -_3): \Ac^{op} \times \Ac^{op} \times \Ac \to \Ab \) is naturally isomorphic to the functor \( \Hom(-_1, -_3) \oplus \Hom(-_2, -_3) \). With the isomorphism

            \begin{center}
                \begin{tikzpicture}
                    \diagram{m}{1cm}{1cm} {
                        \Hom(-_1 \oplus -_2, -_3) & \Hom(-_1, -_3) \oplus \Hom(-_2, -_3) \\
                    };

                    \draw[math]
                        (m-1-1) edge[curve={height=-25pt}] node {\begin{psmallmatrix} (i_{-_1})^* \\ (i_{-_2})^* \end{psmallmatrix}} (m-1-2)
                        (m-1-2) edge[curve={height=-25pt}] node {\begin{psmallmatrix} (p_{-_1})^* & (p_{-_2})^* \end{psmallmatrix}} (m-1-1);
                \end{tikzpicture}
            \end{center}

            \begin{align*}
                \phi_{\tuple{-_1, -_2, -_3}}: \Hom(-_1 \oplus -_2, -_3) &\to \Hom(-_1, -_3) \oplus \Hom(-_2, -_3) \\
            \end{align*}
        }
        \item {
            The functor \( \Hom(-_1, -_2 \oplus -_3): \Ac^{op} \times \Ac \times \Ac \to \Ab \) is naturally isomorphic to the functor \( \Hom(-_1, -_2) \oplus \Hom(-_1, -_3) \).
        }
    \end{enumerate}
\end{theorem}
\begin{proof}
    TODO: 
    
    https://ncatlab.org/nlab/show/additive+functor

    https://ncatlab.org/nlab/show/hom-functor+preserves+limits
\end{proof}


% Show that there only is one non-trivial map between the indecomposable non-projective modules (Might not be true)

% Show that Sigma(S) = M, Sigma(M) = S.

% Show that S -> M -> S is exact.
% Show that S -> M -> S -> M is dist.

\begin{theorem}
    Let \( \Pc = \set{ \tuple{ \frac{\Fb_3C_3}{(1 + g)} }^n \mid n \in \Nb } \union \set{ 0 } \). Let \( \mu: \frac{\Fb_3C_3}{(1 + g)^2} \to \frac{\Fb_3C_3}{(1 + g)} \) be as in \autoref{thm:f_3c_3_mu} and let \( \Nc = \set{ \tuple{ \mu }^n \mid n \in \Nb } \union \set{ 0 } \).

    Then \( \tuple{ \Pc, \Nc} \) is a projective class in \( \Stmod(\Fb_3C_3) \).
\end{theorem}
\begin{proof}
    Need to show that \( \tuple{ \Pc, \Nc } \) satisfies the three properties in \autoref{def:projective_class}.

    \begin{enumerate}
        \item {
            \( \tuple{ \Rightarrow } \) Fix any \( n \in \Nb \). Let \( f = \tuple{ \mu }^n \in \Nc \).
            Then fix any \( m \in \Nb \), and let \( P = \tuple{ \frac{\Fb_3C_3}{(1 + g)} }^m \).

            Then 
            \[
                \tuple{ \tuple{ \mu }^n }_* : \Stmod(\Fb_3C_3)\tuple{P^m, \tuple{\frac{\Fb_3C_3}{(1 + g)^2}}^n } \to \Stmod(\Fb_3C_3)\tuple{P^m, \tuple{\frac{\Fb_3C_3}{(1 + g)}}^n }.
            \]
            % Need naturality? TODO
            
        }
    \end{enumerate}
\end{proof}