In broad terms, a DG-module is a functor from some DG category into an a specific DG-category. The following definition, which is based on \cite[p. 29]{Jasso-Muro_2023}, is the category that will be the co-domain of these DG-modules.

% TODO: Verify definition with borceux definition of composition p. 295
\begin{definition}[\( \C_{\dg} \)]
    \label{def:c_dg_mod_r}
    Let \( R \) be a commutative ring with identity.

    Then let \emph{\( \C_{\dg}(\Mod(R)) \)}, or shortened to \emph{\( \C_{\dg} \)}, be the DG-category defined as follows
    \begin{enumerate}
        \item {
            \( \Obj(\C_{\dg}) := \Obj(\C) \).
        }
        \item {
            Let \( A, B \in \C_{\dg} \). Let \( \class*{A, B} \) denote the internal hom of \( \C \) (\autoref{def:internal_hom_of_chain_complexes_over_Mod(R)}) with respect to \( A, B \) as objects from \( \C \).

            Then let \( \C_{\dg}(A, B) := \class*{A, B} \).
        }
        \item {
            For \( A, B, C \in \C_{\dg} \), let
            \[
                c_{\C_{\dg}}: \C_{\dg}(B, C) \otimes \C_{\dg}(A, B) \to \C_{\dg}(A, C)
            \]
            be defined as the chain morphism where \( c_{\C_{\dg}, n} \) is uniquely defined on
            \[
                \tuple*{g_p}_{p \in \Zb} \in [B, C]_i \text{ and } \tuple*{f_q}_{q \in \Zb} \in [A, B]_j
            \]
            as follows
            \begin{align*}
                c_{\C_{\dg}, n}: \tuple*{ \C_{\dg}(B, C) \otimes \C_{\dg}(A, B) }_n &\to \C_{\dg}(A, C)_n \\
                \tuple*{g_p}_{p \in \Zb} \otimes \tuple*{f_q}_{q \in \Zb} &\mapsto \tuple*{g_{p + j} \circ f_p}_{p \in \Zb}.
            \end{align*}
        }
        \item {
            Let \( A \in \C_{\dg} \).

            Then the unit morphism,
            \[
                u_A: I \to \C_{\dg}(A, A),
            \]
            is the chain morphism where in degree \( 0 \) it is
            \begin{align*}
                u_{A, 0} : R &\to [A, A]_0 \\
                r &\mapsto \tuple{r\Id_{A_i}}_{i \in \Zb}
            \end{align*}
            and \( 0 \) in every other degree.
        }
    \end{enumerate}
\end{definition}

The following remark shows why the composition above is well-defined.

\begin{remark}
    The composition definition in \autoref{def:c_dg_mod_r} is well defined by the following two arguments:

    \begin{enumerate}
        \item {
            Firstly, the morphisms \( c_{\C_{\dg}, n} \) are uniqely defined by \autoref{lem:map_out_of_tensor_unique} where the \( g_{i, j} \)'s are as follows
            \begin{align*}
                g_{i, j}: \prod_{p \in \Zb} \Mod(R)(B_p, C_{p + i}) \times \prod_{p \in \Zb} \Mod(R)(A_p, B_{p + j}) &\to \prod_{p \in \Zb} \Mod(R)(A_p, C_{p + i + j}) \\
                \tuple*{ g_p }_{p \in \Zb} \times \tuple*{ f_q }_{q \in \Zb} &\mapsto \tuple*{ g_{p + j} \circ f_p }_{p \in \Zb}.
            \end{align*}
            And these can be checked to be \( R \)-bilinear.
        }
        \item {
            Secondly, need to check that the \( c_{\C_{\dg}, n} \)'s form a chain morphism.

            Look at the following equation (with shortened notation for brevity) 
            \begin{align*}
                &( d_{(A, C), n} \circ c_n - c_{n + 1} \circ d_{(B, C) \otimes (A, B), n} )\tuple*{ \tuple*{ g_p }_{p \in \Zb} \otimes \tuple*{ f_q }_{q \in \Zb} } \\
                &= d_{(A, C), n} \circ c_n \tuple*{ \tuple*{ g_p }_{p \in \Zb} \otimes \tuple*{ f_q }_{q \in \Zb} } - c_{n + 1} \circ d_{(B, C) \otimes (A, B), n}\tuple*{ \tuple*{ g_p }_{p \in \Zb} \otimes \tuple*{ f_q }_{q \in \Zb} } \\
                \intertext{by definition of composition as well as differential of tensor product it follows that}
                &= d_{(A, C), n}\tuple*{
                    \tuple*{ g_{p + j} \circ f_p }_{p \in \Zb}
                } \\
                &\hspace{0.4cm} - c_{n + 1} \tuple*{
                    d_{(B, C), i}\tuple*{ \tuple*{ g_p }_{p \in \Zb} } \otimes \tuple*{ f_q }_{q \in \Zb}
                    + (-1)^i \tuple*{ g_p }_{p \in \Zb} \otimes d_{(A, B), j} \tuple*{ \tuple*{ f_q }_{q \in \Zb} }
                } \\
                \intertext{then by the definition of the differential of the internal hom}
                &= \tuple*{
                    d_{C, p + n} \circ g_{p + j} \circ f_p - (-1)^n g_{p + j + 1} \circ f_{p + 1} \circ d_{A, p}
                }_{p \in \Zb} \\
                &\hspace{0.4cm} - c_{n + 1} (
                    \tuple*{
                        d_{C, p + i} \circ g_p - (-1)^i g_{p + 1} \circ d_{B, p}
                    }_{p \in \Zb} \otimes \tuple*{ f_q }_{q \in \Zb} \\
                    &\hspace{0.8cm}+ (-1)^i \tuple*{ g_p }_{p \in \Zb} \otimes \tuple*{
                        d_{B, q + j} \circ f_q - (-1)^j f_{q + 1} \circ d_{A, q}
                    }_{q \in \Zb}
                ) \\
                \intertext{then by the fact that composition is an \( R \)-homomorphism}
                &= \tuple*{
                    d_{C, p + n} \circ g_{p + j} \circ f_p - (-1)^n \circ g_{p + j + 1} \circ f_{p + 1} \circ d_{A, p}
                }_{p \in \Zb} \\
                &\hspace{0.4cm} - c_{n + 1} \tuple*{
                    \tuple*{
                        d_{C, p + i} \circ g_p - (-1)^i g_{p + 1} \circ d_{B, p}
                    }_{p \in \Zb} \otimes \tuple*{ f_q }_{q \in \Zb}
                } \\
                &\hspace{0.4cm} - (-1)^i c_{n + 1} \tuple*{
                    \tuple*{ g_p }_{p \in \Zb} \otimes \tuple*{
                        d_{B, q + j} \circ f_q - (-1)^j f_{q + 1} \circ d_{A, q}
                    }_{q \in \Zb}
                } \\
                \intertext{then by the definition of composition}
                &= \tuple*{
                    d_{C, p + n} \circ g_{p + j} \circ f_p
                }_{p \in \Zb} - (-1)^n \tuple*{
                    g_{p + j + 1} \circ f_{p + 1} \circ d_{A, p}
                }_{p \in \Zb} \\
                &\hspace{0.4cm} - \tuple*{
                    d_{C, p + i + j} \circ g_{p + j} \circ f_p
                }_{p \in \Zb} + (-1)^i \tuple*{
                    g_{p + j + 1} \circ d_{B, p + j} \circ f_p
                }_{p \in \Zb} \\
                &\hspace{0.4cm} - (-1)^i \tuple*{
                    g_{p + j + 1} \circ d_{B, p + j} \circ f_p
                }_{p \in \Zb} + (-1)^{i + j} \tuple*{
                    g_{p + j + 1} \circ f_{p + 1} \circ d_{A, p}
                }_{p \in \Zb} \\
                &= 0.
            \end{align*}
            By \autoref{lem:map_out_of_tensor_unique}, this shows that the morphism
            \[
                d_{(A, C), n} \circ c_n - c_{n + 1} \circ d_{(B, C) \otimes (A, B), n}: \\
                \tuple*{ \C_{\dg}(B, C) \otimes \C_{\dg}(A, B) }_n \to \C_{\dg}(A, C)_{n + 1}
            \]         
            corresponds to the \( g_{i,j} \)'s where \( g_{i, j} = 0 \). However, by uniqueness, this implies that
            \[
                d_{(A, C), n} \circ c_n - c_{n + 1} \circ d_{(B, C) \otimes (A, B), n} = 0.
            \]
        }
    \end{enumerate}
\end{remark}

One can also through ardous calculations using \autoref{rem:tensor_prod_internal_hom_adjoint} find that the definition of composition in \autoref{def:c_dg_mod_r} is the same as in \cite[p. 295]{Borceux_1994}.

An interesting consequence of how \( \C_{\dg} \) is defined is that the differential can be treated as a DG-morphism. This is discussed in the following remark.
\begin{remark}
    \label{rem:c_dg_differential}
    For any \( A \in \C_{\dg} \), consider the degree \( 1 \) DG-morphism
    \[
        d_A := \tuple{ d_{A, j} }_{j \in \Zb} \in \C_{\dg}(A, A)_1.
    \]
    Then the differential of \( f \in \C_{\dg}(A, B)_i \) could be written more simply as
    \[
        d_{\C_{\dg}(A, B), i}: f \mapsto d_B \circ f - (-1)^i f \circ d_A.
    \]
    This is because if \( f = \tuple{f_j}_{j \in \Zb} \), then
    \[
        d_B \circ f = \tuple{d_{B, j} }_j \circ \tuple{ f_j }_j = \tuple{ d_{B, j + i} \circ f_j }_j
    \]
    and likewise
    \[
        f \circ d_A = \tuple{f_{j + 1} \circ d_{A, j}}_j
    \]
    which yields exactly the differnetial stated in \autoref{def:c_dg_mod_r}.
\end{remark}

Similar to regular modules over a non-commutative ring, DG-modules can be defined as both right or left sided modules. In this thesis, only right DG-modules will be used. Because of that, the following definition is necessary.

\begin{definition}[Opposite DG-category, \( \Cc^{\op} \)]
    \label{def:opposite_dg_category}
    Let \( \Cc \) be a DG-category.

    Then let \( \Cc^{op} \) be the DG-category defined as follows
    \begin{enumerate}
        \item {
            \( \Obj(\Cc^{op}) := \Obj(\Cc) \)
        }
        \item {
            For \( A, B \in \Cc^{op} \), let \( \Cc^{op}(A, B) := \Cc(B, A) \).
        }
        \item {
            Let \( A, B, C \in \Cc^{op} \) and let \( s \) be as in \autoref{rem:symmetry_tensor_product_of_chain_complex}.
            
            Then define composition as the chain morphism
            \[
                c_{\Cc^{\op}} :=  c_{\Cc} \circ s: \Cc^{\op} (B, C) \otimes \Cc^{\op} (A, B) \to \Cc^{\op} (A, C)
            \]
        }
        \item {
            Let the unit morphism be the same as in \( \Cc \).
        }
    \end{enumerate}
    This is called the \emph{opposite DG-category of \( \Cc \)}.
\end{definition}

Another piece of the definition of a DG-module is to define what a functor between DG-categories is. That is the following definition.

\begin{definition}[DG-functor]
    An enriched functor between two DG-categories as defined in \cite[Definition 6.2.3]{Borceux_1994} is called a \emph{DG-functor}.
\end{definition}

An example of a DG-functor is the shift functor on \( \C_{\dg} \), which is defined in \autoref{def:sigma_c_dg}.

Finally one can define the DG-functor category, which is the final piece in defining DG-modules.

\begin{definition}[DG-functor category, \( \Fun_{\dg}(\Ac, \Bc) \)]
    \label{def:dg_functor_category}
    Let \( \Ac \) and \( \Bc \) be DG-categories over a commutative ring with identity, \( R \). In addition, let \( \Ac \) be small.

    Then let \( \Fun_{\dg}(\Ac, \Bc) \) be the following DG-category:
    \begin{enumerate}
        \item{
            Let \( \Obj(\Fun_{\dg}(\Ac, \Bc)) \) be the class of every DG-functor from \( \Ac \) to \( \Bc \).
        }
        \item{
            For \( F, G \in \Fun_{\dg}(\Ac, \Bc) \), let \( \Fun_{\dg}(\Ac, \Bc)(F, G) \) be defined as in \cite[Proposition 6.3.1]{Borceux_1994}, in particular, \( \Fun_{\dg}(\Ac, \Bc)(F, G) \) is a sub chain complex of \( \prod_{A \in \Ac} \Bc(FA, GA) \).
        }
        \item {
            For any \( i, j \in \Zb \), let
            \[
                \tuple{\eta_{i, A}}_A \in \tuple{\prod_{A \in \Ac} \Bc(FA, GA)}_i
            \]
            and
            \[
                \tuple{\mu_{j, A}}_A \in \tuple{\prod_{A \in \Ac} \Bc(GA, HA)}_j.
            \]

            Then define composition element-wise as follows
            \[
                c_{i + j}: \tuple{\mu_{j, A}}_A \otimes \tuple{\eta_{i, A}}_A \mapsto \tuple{\mu_{j, A} \circ \eta_{i, A}}_A.
            \]
            Since \( \Fun_{\dg}(\Ac, \Bc)(F, G) \) is a sub chain complex of \( \prod_{A \in \Ac} \Bc(FA, GA) \), composition can be restricted to \( \Fun_{\dg}(\Ac, \Bc)(F, G) \).

            TODO: Er det veldefinert?
        }
        \item {
            Let unit morphisms be defined as WIP.
        }
    \end{enumerate}
    This is called the \emph{DG-functor category from \( \Ac \) to \( \Bc \)}.
\end{definition}

In this thesis, given some \( n \), elements in \( \Fun_{\dg}(\Ac, \Bc)_n \) are denoted as \emph{DG-natural transformations}. In other litterature, this name is often associated with another type of natural transformation between DG-functors that is not a part of a chain complex. Borceux talks about these kinds of ``DG-natural transformations'' in \cite[Definition 6.2.4]{Borceux_1994}. That is not relevant for this thesis, and therefore we stick to our definition.

The definition of (right) DG-modules is the following.

\begin{definition}[DG-modules, \( \dgMod_{\dg}(\Cc) \)]
    Let \( \Cc \) be a small DG-category over \( R \).

    % TODO: Why "Right"?
    Then define the \emph{DG-category of (right) DG-\( \Cc \)-modules} as
    \[
        \dgMod_{\dg}(\Cc) := \Fun_{\dg}(\Cc^{op}, \C_{\dg}).
    \]
    Objects in \( \dgMod_{\dg}(\Cc) \) are called \emph{DG-modules over \( \Cc \)}.
\end{definition}

Bofore moving on, it is helpful to try and properly understand the structure of a DG-module. The definition in \autoref{def:dg_functor_category} is abstract and hard to understand exactly how the morphisms, i.e., the ``DG-natural transformations'', look like and what properties they have. The following remark goes through the calculations laid forth by Borceux for DG-modules.

\begin{remark}[Functor category structure from Borceux]
    The goal of this remark is to understand what the morphism structure of \( \dgMod_{\dg}(\Cc) \) is, and what properties it has from the enriched category theory perspective based on \cite{Borceux_1994}.
    
    Let \( F, G \in \dgMod_{\dg}(\Cc) \) and let
    \[
        \tuple{ \eta_{i, A} }_{A \in \Cc^{\op}} \in \tuple*{ \prod_{A \in \Cc^{\op}} \C_{\dg}(F(A), G(A)) }_i
    \]
    and let
    \[
        f_j \in \Cc^{\op}(A', A'')_j.
    \]
    Construct the following composition of morphisms, where the right hand side is the element-wise action on the \( n \)-th component on an arbitrary elementary tensor with \( i +j = n \) (this is enough information to deduce it's action on any element by \autoref{lem:map_out_of_tensor_unique})
    \begin{diagramlabel}[\label{eq:functor_category_borceux}]
        \newcommand{\height}{1cm}
        %
        \mmznext{meaning to context=\height}
        \begin{tikzpicture}
            \diagram{m}{\height}{1cm} {
                \tuple*{ \prod_{A \in \Cc^{\op}} \C_{\dg}(F(A), G(A)) } \otimes \Cc^{\op}(A', A'') \\
                \C_{\dg}(F(A'), G(A')) \otimes \C_{\dg}(G(A'), G(A'')) \\
                \C_{\dg}(G(A'), G(A'')) \otimes \C_{\dg}(F(A'), G(A')) \\
                \C_{\dg}(F(A'), G(A'')) \\
            };

            \draw[math]
                (m-1-1) edge node {\pi_{A'} \otimes G_{A', A''}} (m-2-1)

                (m-2-1) edge node {s} (m-3-1)

                (m-3-1) edge node {c_{\C_{\dg}}} (m-4-1);
        \end{tikzpicture}
        %
        \mmznext{meaning to context=\height}
        \begin{tikzpicture}
            \diagram{m}{\height}{1cm} {
                \tuple{ \eta_{i, A} }_{A \in \Cc^{\op}} \otimes f_j  \\
                \eta_{i, A'} \otimes G(f_j) \\
                (-1)^{ij} G(f_j) \otimes \eta_{i, A'} \\
                (-1)^{ij} G(f_j) \circ \eta_{i, A'} \\
            };

            \path[math]
                ([yshift=-2.5mm]m-1-1.south) edge[maps to] (m-2-1)

                (m-2-1) edge[maps to] (m-3-1)

                (m-3-1) edge[maps to] (m-4-1);
        \end{tikzpicture}
    \end{diagramlabel}
    NOTE: The composition, \( \circ \), in bottom right of the above diagram is the composition as defined for DG-diagrams.

    Name the entire above composition of chain complex morphisms for \( \psi_{A', A''} \).

    % For any \( i, j \in \Zb \) let \( \iota_{i,j} \) denote the inclusion
    % \begin{multline*}
    %     \iota_{i, j}: \tuple*{ \prod_{A \in \Cc^{\op}} \C_{\dg}(F(A), G(A)) }_i \otimes \Cc^{\op}(A', A'')_j \\
    %     \rightarrowtail \tuple*{ \tuple*{ \prod_{A \in \Cc^{\op}} \C_{\dg}(F(A), G(A)) } \otimes \Cc^{\op}(A', A'') }_{i + j}
    % \end{multline*}

    Take the adjoint of \autoref{eq:functor_category_borceux} morphism gives the chain complex morphism \( \phi_{A', A''} \), where
    \begin{center}
        \newcommand{\height}{1cm}
        %
        \mmznext{meaning to context=\height}
        \begin{tikzpicture}
            \diagram{m}{\height}{1cm} {
                \prod\limits_{A \in \Cc^{\op}} \C_{\dg}(F(A), G(A)) \\
                \left[ \Cc^{\op}(A', A''), \C_{\dg}(F(A'), G(A'')) \right] \\
            };

            \draw[math]
                (m-1-1) edge node {\phi_{A', A''}} (m-2-1);
        \end{tikzpicture}
        %
        \mmznext{meaning to context=\height}
        \begin{tikzpicture}
            \diagram{m}{\height}{1cm} {
                \tuple{ \eta_{i, A} }_{A \in \Cc^{\op}} \\
                \tuple{ \psi_{A', A'', k} \tuple{ \tuple{ \eta_{i, A} }_{A \in \Cc^{\op}} \otimes ? } }_{k \in \Zb}. \\
            };

            \draw[math]
                (m-1-1) edge[maps to] (m-2-1);
        \end{tikzpicture}
    \end{center}
    NOTE: The notation causes some ambiguity here and so it is neccesary to specify that in the bottom right of the above diagram, \( ? \in \Cc^{\op}(A', A'')_{k - i} \).

    Finally take the product of this map over all \( A', A'' \in \Cc^{\op} \) to get the morphism
    \begin{center}
        \newcommand{\height}{1cm}
        %
        \mmznext{meaning to context=\height}
        \begin{tikzpicture}
            \diagram{m}{\height}{1cm} {
                \prod\limits_{A \in \Cc^{\op}} \C_{\dg}(F(A), G(A)) \\
                \prod\limits_{A', A'' \in \Cc^{\op}} \left[ \Cc^{\op}(A', A''), \C_{\dg}(F(A'), G(A'')) \right] \\
            };

            \draw[math]
                (m-1-1) edge node {\prod\limits_{A', A'' \in \Cc^{\op}} \phi_{A', A''}} (m-2-1);
        \end{tikzpicture}
        %
        \mmznext{meaning to context=\height}
        \begin{tikzpicture}
            \diagram{m}{\height}{1cm} {
                \tuple{ \eta_{i, A} }_{A \in \Cc^{\op}} \\
                \tuple{ \tuple{ \psi_{A', A'', k} \tuple{ \tuple{ \eta_{i, A} }_{A \in \Cc^{\op}} \otimes ? } }_{k \in \Zb} }_{A', A'' \in \Cc^{\op}}. \\
            };

            \draw[math]
                (m-1-1) edge[maps to] (m-2-1);
        \end{tikzpicture}
    \end{center}
    Doing a similar construction as in \autoref{eq:functor_category_borceux} look at the composition
    \begin{center}
        \newcommand{\height}{1cm}
        %
        \mmznext{meaning to context=\height}
        \begin{tikzpicture}
            \diagram{m}{\height}{1cm} {
                \tuple*{ \prod\limits_{A \in \Cc^{\op}} \C_{\dg}(F(A), G(A)) } \otimes \Cc^{\op}(A', A'') \\
                \C_{\dg}(F(A''), G(A'')) \otimes \C_{\dg}(F(A'), F(A'')) \\
                \C_{\dg}(F(A'), G(A'')) \\
            };

            \draw[math]
                (m-1-1) edge node {\pi_{A''} \otimes F_{A', A''}} (m-2-1)

                (m-2-1) edge node {\circ} (m-3-1);
        \end{tikzpicture}
        %
        \mmznext{meaning to context=\height}
        \begin{tikzpicture}
            \diagram{m}{\height}{1cm} {
                \tuple{ \eta_{i, A} }_{A \in \Cc^{\op}} \otimes f_j  \\
                \eta_{i, A''} \otimes F(f_j) \\
                \eta_{i, A''} \circ F(f_j) \\
            };

            \draw[math]
                (m-1-1) edge[maps to] (m-2-1)

                (m-2-1) edge[maps to] (m-3-1);
        \end{tikzpicture}
    \end{center}
    Using this as \( \widetilde{\psi}_{A', A''} \) continue the construction as before by taking the adjoint and the product to end up with the map \( \widetilde{\phi}_{A', A''} \).

    Then \cite[Proposition 6.3.1]{Borceux_1994} states that \( \dgMod_{\dg}(\Cc)(F, G) \) is the equalizer of the following diagram
    \begin{center}
        \begin{tikzpicture}
            \diagram{m}{1cm}{1cm} {
                \prod\limits_{A \in \Cc^{\op}} \C_{\dg}(F(A), G(A)) \\
                \prod\limits_{A', A'' \in \Cc^{\op}} \left[ \Cc^{\op}(A', A''), \C_{\dg}(F(A'), G(A'')) \right] \\
            };

            \path[math] ([xshift=2.5mm]m-1-1.south) edge node {\prod\limits_{A', A'' \in \Cc^{\op}} \phi_{A', A''}} ([xshift=2.5mm]m-2-1.north);
            \draw[math] ($(m-1-1.south) + (-0.25, 0)$) to node[swap] {\prod\limits_{A', A'' \in \Cc^{\op}} \widetilde{\phi}_{A', A''}} ($(m-2-1.north) + (-0.25, 0)$);
        \end{tikzpicture}
    \end{center}
    Then the question remains, what properties would this equalizer have?
    
    Consider the chain subcomplex
    \[
        H \stackrel{\iota}{\rightarrowtail} \prod_{A \in \Cc^{\op}} \C_{\dg}(F(A), G(A))
    \]
    where
    \[
        H := \ker\tuple*{\prod\limits_{A', A'' \in \Cc^{\op}} \phi_{A', A''} - \prod_{A', A'' \in \Cc^{\op}} \widetilde{\phi}_{A', A''}}.
    \]
    This is the equalizer by the kernel property.

    Then \( H \) would for any \( i \in \Zb \) contain the elements
    \[
        \tuple{ \eta_{i, A} }_{A \in \Cc^{\op}} \in \tuple*{ \prod_{A \in \Cc^{\op}} \C_{\dg}(F(A), G(A)) }_i
    \]
    such that
    \begin{align*}
        \tuple*{ \prod\limits_{A', A'' \in \Cc^{\op}} \phi_{A', A''} }_j \tuple{ \tuple{ \eta_{i, A} }_{A \in \Cc^{\op}} } &= \tuple*{ \prod_{A', A'' \in \Cc^{\op}} \widetilde{\phi}_{A', A''} }_j \tuple{ \tuple{ \eta_{i, A} }_{A \in \Cc^{\op}} } \\
        &\Updownarrow \\
        \tuple{ \tuple{ \psi_{A', A'', k}\tuple{ \tuple{ \eta_{i, A} }_{A \in \Cc^{\op}} \otimes ? } }_{k \in \Zb} }_{A', A'' \in \Cc^{\op}} &= \tuple{ \tuple{ \widetilde{\psi}_{A', A'', k}\tuple{ \tuple{ \eta_{i, A} }_{A \in \Cc^{\op}} \otimes ? } }_{k \in \Zb} }_{A', A'' \in \Cc^{\op}}
        \intertext{Which are equal if for any \( A', A'' \in \Cc^{\op} \) and any \( k \in \Zb \) and any \( f \in \Cc^{\op}\tuple*{A', A''}_{k - i} \) one has the following}
        \psi_{A', A'', k}\tuple{ \tuple{ \eta_{i, A} }_{A \in \Cc^{\op}} \otimes f } &= \widetilde{\psi}_{A', A'', k} \tuple{ \tuple{ \eta_{i, A} }_{A \in \Cc^{\op}} \otimes f } \\
        &\Updownarrow \\
        (-1)^{(k - i)*i}G(f) \circ \eta_{i, A'} &= \eta_{i, A''} \circ F(f) \\
    \end{align*}
\end{remark}

In plain english, this means that the morphism space \( \dgMod_{\dg}(\Cc)(F, G) \) has the following properties:
\begin{enumerate}
    \item {
        The structure of \( \dgMod_{\dg}(\Cc)(F, G) \) is as a sub chain complex of
        \[
            \prod_{A \in \Cc^{\op}} \C_{\dg}(F(A), G(A)).
        \]
        Which is a chain complex where in the \( i \)-th component, one has
        \[
            \tuple*{ \prod_{A \in \Cc^{\op}} \C_{\dg}(F(A), G(A)) }_i = \prod_{A \in \Cc^{\op}} \C_{\dg}(F(A), G(A))_i
        \]
        where every element is denoted as \( \tuple{ \eta_{i, A} }_{A \in \Cc^{\op}} \).

        Let \( \widetilde{d_A} \) be the differential of \( \C_{\dg}(F(A), G(A)) \), then the differential of
        \[
            \prod_{A \in \Cc^{\op}} \C_{\dg}(F(A), G(A))
        \]
        is
        \begin{align*}
            \prod_{A \in \Cc^{\op}} \widetilde{d_A} : \prod_{A \in \Cc^{\op}} \C_{\dg}(F(A), G(A))_j &\to \prod_{A \in \Cc^{\op}} \C_{\dg}(F(A), G(A))_{j + 1} \\
            \tuple{ \eta_{j, A} }_{A \in \Cc^{\op}} &\mapsto \tuple{ \widetilde{d_A}\tuple{ \eta_{j, A} } }_{A \in \Cc^{\op}} \\
            &= \tuple*{ d_{G(A)} \circ \eta_{j, A} - (-1)^j \eta_{j, A} \circ d_{F(A)} }_{A \in \Cc^{\op}}.
        \end{align*}
    }
    \item {
        In addition, \( \dgMod_{\dg}(\Cc)(F, G) \) has the property that for any
        \[
            \tuple{ \eta_{i, A} }_{A \in \Cc^{\op}} \in \dgMod_{\dg}(\Cc)(F, G)_i,
        \]
        and any \( f \in \Cc^{\op}(A', A'')_j \), one has that
        \[
            G_{A', A'', j}(f) \circ \eta_{i, A'} = (-1)^{ij} \eta_{i, A''} \circ F_{A', A'', j}(f).
        \]
    }
\end{enumerate}

Similar to how we could simplify the notation for the differential of \( \C_{\dg} \) in \autoref{rem:c_dg_differential}, we can similarly simplify the differential of \( \dgMod_{\dg}(\Cc) \), as is discussed in the following remark.

\begin{remark}
    For any \( A \in \C_{\dg} \), let \( d_A \) be as in \autoref{rem:c_dg_differential}.

    Then for any \( F \in \dgMod_{\dg}(\Cc) \) consider the degree \( 1 \) DG-morphism
    \[
        d_F := \tuple{d_{FA}}_A \in \prod_{A \in \Cc} \C_{\dg}(FA, FA).
    \]

    It follows that for any \( \eta_i = \tuple{\eta_{i, A}}_A \in \dgMod_{\dg}(\Cc)(F, G) \) we have
    \[
        d_{\dgMod_{\dg}(\Cc)(F, G), i}(\eta_i) = d_G \circ \eta_i - (-1)^i \eta_i \circ d_F
    \]
    because
    \[
        d_G \circ \eta_i = \tuple{d_{G A}}_A \circ \tuple{\eta_{i, A}}_A = \tuple{d_{G A} \circ \eta_{i, A}}_A
    \]
    and similarly
    \[
        \eta_i \circ d_F = \tuple{\eta_{i, A} \circ d_{F A}}_A.
    \]

    TODO: Er \( d_F \) i \( \dgMod \)?
\end{remark}

% TODO: Skriv noko om at litteraturen er ueinig om definisjonen av DG-nat-trans.
