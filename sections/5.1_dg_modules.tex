In broad terms, a DG-module is an enriched functor from some DG-category into a specific DG-category. The following definition, which is based on \cite[p.\ 29]{Jasso-Muro_2023}, is the category that will be the codomain of these DG-modules.

\begin{definition}[\( \C_{\dg} \)]
    \label{def:c_dg_mod_r}
    Let \( R \) be a commutative ring with identity.

    Then let \emph{\( \C_{\dg}(\Mod(R)) \)}, abbreviated to \emph{\( \C_{\dg} \)}, be the DG-category defined as follows
    \begin{enumerate}
        \item {
            \( \Obj(\C_{\dg}) := \Obj(\C) \).
        }
        \item {
            Let \( A, B \in \C_{\dg} \), and let \( \class*{A, B} \) denote the internal Hom of \( \C \) (\autoref{def:internal_hom_of_chain_complexes_over_Mod(R)}) with respect to \( A, B \in \C \).

            Then let \( \C_{\dg}(A, B) := \class*{A, B} \).
        }
        \item {
            For \( A, B, C \in \C_{\dg} \), let
            \[
                c_{\C_{\dg}}: \C_{\dg}(B, C) \otimes \C_{\dg}(A, B) \to \C_{\dg}(A, C)
            \]
            be defined as the chain morphism where \( c_n^{\C_{\dg}} \) is uniquely defined on
            \[
                \tuple*{g_p}_{p \in \Zb} \in [B, C]_i \quad \text{and} \quad \tuple*{f_q}_{q \in \Zb} \in [A, B]_j
            \]
            as follows
            \begin{align*}
                c_n^{\C_{\dg}}: \tuple*{ \C_{\dg}(B, C) \otimes \C_{\dg}(A, B) }_n &\to \C_{\dg}(A, C)_n \\
                \tuple*{g_p}_{p \in \Zb} \otimes \tuple*{f_q}_{q \in \Zb} &\mapsto \tuple*{g_{p + j} \circ f_p}_{p \in \Zb}.
            \end{align*}
        }
        \item {
            Let \( A \in \C_{\dg} \).

            Then the unit morphism,
            \[
                u_A: I \to \C_{\dg}(A, A),
            \]
            is the chain morphism where in degree \( 0 \) it is
            \begin{align*}
                u_0^A : R &\to [A, A]_0 \\
                r &\mapsto \tuple*{r\Id_{A_i}}_{i \in \Zb}
            \end{align*}
            and \( 0 \) in every other degree.
        }
    \end{enumerate}
\end{definition}

The following remark shows why the composition above is well-defined.

\begin{remark}
    The composition definition in \autoref{def:c_dg_mod_r} is well-defined by the following two arguments:

    \begin{enumerate}
        \item {
            First, the morphisms \( c_n^{\C_{\dg}} \) are uniquely defined by \autoref{lem:map_out_of_tensor_unique} where the \( g_{i, j} \) are as follows
            \begin{align*}
                g_{i, j}: \prod_{p \in \Zb} \Mod(R)(B_p, C_{p + i}) \times \prod_{p \in \Zb} \Mod(R)(A_p, B_{p + j}) &\to \prod_{p \in \Zb} \Mod(R)(A_p, C_{p + i + j}) \\
                \tuple*{ g_p }_{p \in \Zb} \times \tuple*{ f_q }_{q \in \Zb} &\mapsto \tuple*{ g_{p + j} \circ f_p }_{p \in \Zb}.
            \end{align*}
            These can be checked to be \( R \)-bilinear.
        }
        \item {
            Second, we need to check that the \( c_{\C_{\dg}, n} \) form a chain morphism.

            Consider the following equation
            \begin{align*}
                &( d_n^{[A, C]} \circ c_n - c_{n + 1} \circ d_n^{[B, C] \otimes [A, B]} )\tuple*{ \tuple*{ g_p }_{p \in \Zb} \otimes \tuple*{ f_q }_{q \in \Zb} } \\
                &= d_n^{[A, C]} \circ c_n \tuple*{ \tuple*{ g_p }_{p \in \Zb} \otimes \tuple*{ f_q }_{q \in \Zb} } - c_{n + 1} \circ d_n^{[B, C] \otimes [A, B]}\tuple*{ \tuple*{ g_p }_{p \in \Zb} \otimes \tuple*{ f_q }_{q \in \Zb} } \\
                \intertext{by definition of composition as well as differential of tensor product,}
                &= d_n^{[A, C]}\tuple*{ g_{p + j} \circ f_p }_{p \in \Zb} \\
                &\hspace{0.4cm} - c_{n + 1} \tuple*{
                    d_i^{[B, C]} \tuple*{ g_p }_{p \in \Zb} \otimes \tuple*{ f_q }_{q \in \Zb}
                    + (-1)^i \tuple*{ g_p }_{p \in \Zb} \otimes d_j^{[A, B]} \tuple*{ f_q }_{q \in \Zb}
                } \\
                \intertext{by the definition of the differential of the internal Hom,}
                &= \tuple*{
                    d_{p + n}^C \circ g_{p + j} \circ f_p - (-1)^n g_{p + j + 1} \circ f_{p + 1} \circ d_p^A
                }_{p \in \Zb} \\
                &\hspace{0.4cm} - c_{n + 1} \bigg(
                    \tuple*{
                        d_{p + i}^C \circ g_p - (-1)^i g_{p + 1} \circ d_p^B
                    }_{p \in \Zb} \otimes \tuple*{ f_q }_{q \in \Zb} \\
                    &\hspace{0.8cm}+ (-1)^i \tuple*{ g_p }_{p \in \Zb} \otimes \tuple*{
                        d_{q + j}^B \circ f_q - (-1)^j f_{q + 1} \circ d_q^A
                    }_{q \in \Zb}
                \bigg) \\
                \intertext{by composition being a \( \Mod(R) \) morphism,}
                &= \tuple*{
                    d_{p + n}^C \circ g_{p + j} \circ f_p - (-1)^n \circ g_{p + j + 1} \circ f_{p + 1} \circ d_p^A
                }_{p \in \Zb} \\
                &\hspace{0.4cm} - c_{n + 1} \tuple*{
                    \tuple*{
                        d_{p + i}^C \circ g_p - (-1)^i g_{p + 1} \circ d_p^B
                    }_{p \in \Zb} \otimes \tuple*{ f_q }_{q \in \Zb}
                } \\
                &\hspace{0.4cm} - (-1)^i c_{n + 1} \tuple*{
                    \tuple*{ g_p }_{p \in \Zb} \otimes \tuple*{
                        d_{q + j}^B \circ f_q - (-1)^j f_{q + 1} \circ d_q^A
                    }_{q \in \Zb}
                } \\
                \intertext{by the definition of composition,}
                &= \tuple*{
                    d_{p + n}^C \circ g_{p + j} \circ f_p
                }_{p \in \Zb} - (-1)^n \tuple*{
                    g_{p + j + 1} \circ f_{p + 1} \circ d_p^A
                }_{p \in \Zb} \\
                &\hspace{0.4cm} - \tuple*{
                    d_{p + i + j}^C \circ g_{p + j} \circ f_p
                }_{p \in \Zb} + (-1)^i \tuple*{
                    g_{p + j + 1} \circ d_{p + j}^B \circ f_p
                }_{p \in \Zb} \\
                &\hspace{0.4cm} - (-1)^i \tuple*{
                    g_{p + j + 1} \circ d_{p + j}^B \circ f_p
                }_{p \in \Zb} + (-1)^{i + j} \tuple*{
                    g_{p + j + 1} \circ f_{p + 1} \circ d_p^A
                }_{p \in \Zb} \\
                &= 0.
            \end{align*}
            By \autoref{lem:map_out_of_tensor_unique}, this shows that the morphism
            \[
                d_n^{[A, C]} \circ c_n - c_{n + 1} \circ d_n^{[B, C] \otimes [A, B]}: \\
                \tuple*{ \C_{\dg}(B, C) \otimes \C_{\dg}(A, B) }_n \to \C_{\dg}(A, C)_{n + 1}
            \]         
            corresponds to the \( g_{i,j} \) where \( g_{i, j} = 0 \). However, by uniqueness, this implies that
            \[
                d_n^{[A, C]} \circ c_n - c_{n + 1} \circ d_n^{[B, C] \otimes [A, B]} = 0.
            \]
        }
    \end{enumerate}
\end{remark}

We can also through arduous calculations using \autoref{rem:tensor_prod_internal_hom_adjoint} find that the definition of composition in \autoref{def:c_dg_mod_r} is the same as in \cite[p.\ 295]{Borceux_1994}.

An interesting consequence of how \( \C_{\dg} \) is defined is that the differential can be treated as a DG-morphism. This is discussed in the following remark.
\begin{remark}
    \label{rem:c_dg_differential}
    For any \( A \in \C_{\dg} \), consider the degree \( 1 \) DG-morphism
    \[
        d_A := \tuple*{ d_j^A }_{j \in \Zb} \in \C_{\dg}(A, A)_1.
    \]
    Then the differential of \( f \in \C_{\dg}(A, B)_i \) could be written more simply as
    \[
        d_i^{[A, B]}: f \mapsto d_B \circ f - (-1)^i f \circ d_A.
    \]
    This is because if \( f = \tuple*{f_j}_{j \in \Zb} \), then
    \[
        d_B \circ f = \tuple*{d_j^B }_{j \in \Zb} \circ \tuple*{ f_j }_{j \in \Zb} = \tuple*{ d_{j + i}^B \circ f_j }_{j \in \Zb}
    \]
    and likewise
    \[
        f \circ d_A = \tuple*{f_{j + 1} \circ d_j^A}_{j \in \Zb}
    \]
    which yields exactly the differential stated in \autoref{def:c_dg_mod_r}.
\end{remark}

Similar to how we could define the differential as a degree \( 1 \) DG-morphism in \( \C_{\dg} \), we could define a degree \( 0 \) DG-morphism \( \Id_A = \tuple*{ \Id_{A_i} }_{i \in \Zb} = u_{A, 0}(1) \). However, this could be a source of confusion as while it would act like the identity when composing DG-morphisms, enriched categories do not generally have identity morphisms, they have unit morphisms which sort of work like identities, but not exactly. Hence, we avoid denoting any DG-morphisms as ``\( \Id \)'' both now and later on.

Similar to regular modules over a non-commutative ring, DG-modules can be defined as both right or left sided modules. We will consider only right DG-modules. In order to define right DG-modules, we need to define the opposite category of a DG-category.

\begin{definition}[Opposite DG-category, \( \Cc^{\op} \)]
    \label{def:opposite_dg_category}
    Let \( \Cc \) be a DG-category.

    Then let \( \Cc^{op} \) be the DG-category defined as follows
    \begin{enumerate}
        \item {
            \( \Obj(\Cc^{op}) := \Obj(\Cc) \)
        }
        \item {
            For \( A, B \in \Cc^{op} \), let \( \Cc^{op}(A, B) := \Cc(B, A) \).
        }
        \item {
            Let \( A, B, C \in \Cc^{op} \) and let \( s \) be as in \autoref{rem:symmetry_tensor_product_of_chain_complex}.
            
            Then define composition as the chain morphism
            \[
                c_{\Cc^{\op}} :=  c_{\Cc} \circ s: \Cc^{\op} (B, C) \otimes \Cc^{\op} (A, B) \to \Cc^{\op} (A, C)
            \]
        }
        \item {
            Let the unit morphism be the same as in \( \Cc \).
        }
    \end{enumerate}
    This is called the \emph{opposite DG-category of \( \Cc \)}.
\end{definition}

Another piece of the definition of a DG-module is to define what a functor between DG-categories is.

\begin{definition}[DG-functor]
    An enriched functor between two DG-categories as defined in \cite[Definition 6.2.3]{Borceux_1994} is called a \emph{DG-functor}.
\end{definition}

There are multiple examples in the following subsection (\autoref{sec:alg_tri_cat_def}) of DG-functors, such as the shift functor on \( \C_{\dg} \), the shift functor of the not yet defined \( \dgM \), as well as \( F \oplus G \) for two DG-functors \( F \) and \( G \).

Finally, we can define the DG-functor category, which is the final piece in defining DG-modules.

\begin{definition}[DG-functor category, \( \Fun_{\dg}(\Ac, \Bc) \)]
    \label{def:dg_functor_category}
    Let \( \Ac \) and \( \Bc \) be DG-categories over a commutative ring with identity, \( R \). In addition, let \( \Ac \) be small.

    Then let \( \Fun_{\dg}(\Ac, \Bc) \) be the DG-category defined in \cite[Proposition 6.3.1]{Borceux_1994}.

    This is called the \emph{(enriched) DG-functor category from \( \Ac \) to \( \Bc \)}.
\end{definition}

In this thesis, given some \( n \), elements in \( \Fun_{\dg}(\Ac, \Bc)_n \) are denoted as \emph{DG-natural transformations}. In other literature, this name is often associated with another type of natural transformation between DG-functors that are not part of the (enriched) DG-functor category. Borceux talks about these kinds of ``DG-natural transformations'' in \cite[Definition 6.2.4]{Borceux_1994}.

The reason for why \( \Ac \) has to be small is because the DG-natural transformations are defined by taking a product of \( R \)-modules over every object in \( \Ac \). This is not defined if \( \Ac \) is big, and is also the reason we only define DG-categories over the infinitely generated \( R \)-modules, as \( \mod(R) \) is not a complete category, but \( \Mod(R) \) is. This will become clearer, when we explicitly calculate what the DG-natural transformations look like for (right) DG-modules in \autoref{rem:dgm_natural_trans_structure}.

Also of note is that Borceux does not define composition or the unit morphism in the DG-functor category, and so we will define them manually for (right) DG-modules in \autoref{rem:dgm_composition_and_unit_morphisms}.

The definition of (right) DG-modules is the following.

\begin{definition}[DG-modules, \( \dgMod_{\dg}(\Cc) \), \( \dgM \)]
    Let \( \Cc \) be a small DG-category over \( R \).

    Then define the \emph{DG-category of (right) DG-\( \Cc \)-modules} as
    \[
        \dgMod_{\dg}(\Cc) := \Fun_{\dg}(\Cc^{op}, \C_{\dg}).
    \]
    Objects in \( \dgMod_{\dg}(\Cc) \) are called \emph{DG-modules over \( \Cc \)}.

    From now on, we will use the shorthand \( \dgM \) for \( \dgMod_{\dg}(\Cc) \).
\end{definition}

\autoref{def:dg_functor_category} is abstract, hard to understand, and does not define composition or the unit morphism of \( \dgM \). In order to define composition and the unit morphism, and in order to make future proofs easier, we use the following remark to show how the DG-natural transformations look like in \( \dgM \).

\begin{remark}[Functor category structure from Borceux]
    \label{rem:dgm_natural_trans_structure}
    Let \( F, G \in \dgM \). In order to show what \( \dgM(F, G) \) looks like, we have to first take the adjoint of the morphisms in the diagram \cite[Diagram 6.21]{Borceux_1994} with the symmetry morphism applied to one of the diagrams to make it fit. Then we have to take the product of the morphisms we get, and finally take the equalizer of the two morphisms.
    
    Let
    \[
        \eta = \tuple*{ \eta_A }_{A \in \Cc} \in \tuple*{ \prod_{A \in \Cc} \C_{\dg}(F A, G A) }_i = \prod_{A \in \Cc} \tuple*{ \C_{\dg}(F A, G A) }_i,
    \]
    which exists because \( \Ac \) is assumed to be small, and because \( \Mod(R) \) is a complete category when \( R \) is commutative, which we assume.

    Furthermore, let
    \[
        f \in \Cc^{\op}(B, C)_j
    \]
    for some \( i, j \in \Zb \).

    Both \( \eta \) and \( f \) will be used as arbitrary elements to see what the chain morphisms in their appropriate degree does to any element. This is usually enough information to deduce its action on any element, by e.g., \autoref{lem:map_out_of_tensor_unique}.

    Consider the following composition of chain morphisms, where the right-hand side is the element-wise action on the \( (i + j) \)th component on an arbitrary elementary tensor.
    \begin{diagramlabel}[\label{diag:functor_category_borceux}]
        \newcommand{\height}{1cm}
        %
        \mmznext{meaning to context=\height}
        \begin{tikzpicture}
            \diagram{m}{\height}{1cm} {
                \tuple*{ \prod\limits_{A \in \Cc} \C_{\dg}(F A, G A) } \otimes \Cc^{\op}(B, C) \\
                \C_{\dg}(F B, G B) \otimes \C_{\dg}(G B, G C) \\
                \C_{\dg}(G B, G C) \otimes \C_{\dg}(F B, G B) \\
                \C_{\dg}(F B, G C) \\
            };

            \draw[math]
                (m-1-1) edge node {\pi_B \otimes G_{B, C}} (m-2-1)

                (m-2-1) edge node {s} (m-3-1)

                (m-3-1) edge node {c_{\C_{\dg}}} (m-4-1);
        \end{tikzpicture}
        %
        \mmznext{meaning to context=\height}
        \begin{tikzpicture}
            \diagram{m}{\height}{1cm} {
                \eta \otimes f  \\
                \eta_{B} \otimes (G f) \\
                (-1)^{ij} (G f) \otimes \eta_{B} \\
                (-1)^{ij} (G f) \circ \eta_{B} \\
            };

            \path[math]
                ([yshift=-2.5mm]m-1-1.south) edge[maps to] (m-2-1)

                (m-2-1) edge[maps to] (m-3-1)

                (m-3-1) edge[maps to] (m-4-1);
        \end{tikzpicture}
    \end{diagramlabel}

    We write the entire above composition of chain complex morphisms as \( \psi_{B, C} \).

    Taking the adjoint of \( \psi_{B, C} \) gives the chain complex morphism \( \phi_{B, C} \), where
    \begin{center}
        \newcommand{\height}{1cm}
        %
        \mmznext{meaning to context=\height}
        \begin{tikzpicture}
            \diagram{m}{\height}{1cm} {
                \prod\limits_{A \in \Cc} \C_{\dg}(F A, G A) \\
                \left[ \Cc^{\op}(B, C), \C_{\dg}(F B, G C) \right] \\
            };

            \draw[math]
                (m-1-1) edge node {\phi_{B, C}} (m-2-1);
        \end{tikzpicture}
        %
        \mmznext{meaning to context=\height}
        \begin{tikzpicture}
            \diagram{m}{\height}{1cm} {
                \eta \\
                \tuple*{ \psi_k^{B, C} \tuple*{ \eta \otimes ? } }_{k \in \Zb}. \\
            };

            \draw[math]
                (m-1-1) edge[maps to] (m-2-1);
        \end{tikzpicture}
    \end{center}
    % NOTE: In the bottom right of the above diagram, \( ? \in \Cc^{\op}(B, C)_{k - i} \).

    Finally, take the product of this map over all \( B, C \in \Cc^{\op} \) to get the morphism
    \begin{center}
        \newcommand{\height}{1cm}
        %
        \mmznext{meaning to context=\height}
        \begin{tikzpicture}
            \diagram{m}{\height}{1cm} {
                \prod\limits_{A \in \Cc} \C_{\dg}(F A, G A) \\
                \prod\limits_{B, C \in \Cc} \left[ \Cc^{\op}(B, C), \C_{\dg}(F B, G C) \right] \\
            };

            \draw[math]
                (m-1-1) edge node {\prod\limits_{B, C \in \Cc} \phi_{B, C}} (m-2-1);
        \end{tikzpicture}
        %
        \mmznext{meaning to context=\height}
        \begin{tikzpicture}
            \diagram{m}{\height}{1cm} {
                \eta \\
                \tuple*{ \psi_k^{B, C} \tuple*{ \eta \otimes ? } }_{k \in \Zb, B, C \in \Cc}. \\
            };

            \draw[math]
                (m-1-1) edge[maps to] (m-2-1);
        \end{tikzpicture}
    \end{center}

    By a similar construction as in \autoref{diag:functor_category_borceux}, consider the diagram,
    \begin{center}
        \newcommand{\height}{1cm}
        %
        \mmznext{meaning to context=\height}
        \begin{tikzpicture}
            \diagram{m}{\height}{1cm} {
                \tuple*{ \prod\limits_{A \in \Cc} \C_{\dg}(F A, G A) } \otimes \Cc^{\op}(B, C) \\
                \C_{\dg}(F C, G C) \otimes \C_{\dg}(F B, F C) \\
                \C_{\dg}(F B, G C) \\
            };

            \draw[math]
                (m-1-1) edge node {\pi_{C} \otimes F_{B, C}} (m-2-1)

                (m-2-1) edge node {\circ} (m-3-1);
        \end{tikzpicture}
        %
        \mmznext{meaning to context=\height}
        \begin{tikzpicture}
            \diagram{m}{\height}{1cm} {
                \eta \otimes f  \\
                \eta_{C} \otimes (F f) \\
                \eta_{C} \circ (F f). \\
            };

            \draw[math]
                (m-1-1) edge[maps to] (m-2-1)

                (m-2-1) edge[maps to] (m-3-1);
        \end{tikzpicture}
    \end{center}
    Define the composition of the above chain morphisms as \( \widetilde{\psi}_{B, C} \) and continue the construction as before by taking the adjoint and the product to end up with the morphism \( \widetilde{\phi}_{B, C} \).

    Then \cite[Proposition 6.3.1]{Borceux_1994} states that \( \dgM(F, G) \) is the equalizer of the following diagram,
    \begin{center}
        \begin{tikzpicture}
            \diagram{m}{1.5cm}{1cm} {
                \prod\limits_{A \in \Cc} \C_{\dg}(F A, G A) \\
                \prod\limits_{B, C \in \Cc} \left[ \Cc^{\op}(B, C), \C_{\dg}(F B, G C) \right]. \\
            };

            \path[math] ([xshift=2.5mm]m-1-1.south) edge node {\prod\limits_{B, C \in \Cc} \phi_{B, C}} ([xshift=2.5mm]m-2-1.north);
            \draw[math] ($(m-1-1.south) + (-0.25, 0)$) to node[swap] {\prod\limits_{B, C \in \Cc} \widetilde{\phi}_{B, C}} ($(m-2-1.north) + (-0.25, 0)$);
        \end{tikzpicture}
    \end{center}

    In an abelian category, such as \( \C \), the equalizer becomes
    \[
        \dgM(F, G) := \ker \tuple*{\prod\limits_{B, C \in \Cc} \phi_{B, C} - \prod\limits_{B, C \in \Cc} \widetilde{\phi}_{B, C}}.
    \]
\end{remark}

Now that we have a notion of what the DG-natural transformations look like, we can consider what properties they have.

\begin{remark}
    What properties does \( \dgM(F, G) \) have?

    First, notice that \( \dgM(F, G) \) is as a sub chain complex of
    \[
        \prod_{A \in \Cc} \C_{\dg}(F A, G A),
    \]
    which is a chain complex where in the \( i \)th component, we have
    \[
        \tuple*{ \prod_{A \in \Cc} \C_{\dg}(F A, G A) }_i = \prod_{A \in \Cc} \C_{\dg}(F A, G A)_i.
    \]
    This chain complex exists, because \( \Cc \) is assumed to be a small category, and \( \Mod(R) \) is a complete category when \( R \) is commutative, which we assume in the definition.

    The differential of \( \prod_{A \in \Cc} \C_{\dg}(F A, G A) \), denoted \( d \), is in degree \( j \) the morphism
    \begin{align*}
        d_j: \prod_{A \in \Cc} \C_{\dg}(F A, G A)_j &\to \prod_{A \in \Cc} \C_{\dg}(F A, G A)_{j + 1} \\
        \eta &\mapsto \tuple*{ d_j^{[F A, G A]}\tuple*{ \eta_A } }_{A \in \Cc} \\
        &= \tuple*{ d_{G A} \circ \eta_A - (-1)^j \eta_A \circ d_{F A} }_{A \in \Cc}.
    \end{align*}

    Second, notice that every element \( \eta \in \dgM(F, G)_j \) satisfies the property
    \[
        \tuple*{ \prod\limits_{B, C \in \Cc} \phi_{B, C} }_j (\eta) = \tuple*{ \prod_{B, C \in \Cc} \widetilde{\phi}_{B, C} }_j (\eta),
    \]
    which is equivalent to
    \[
        \tuple*{ \psi_k^{B, C}\tuple*{ \eta \otimes ? } }_{k \in \Zb, B, C \in \Cc} = \tuple*{ \widetilde{\psi}_k^{B, C}\tuple*{ \eta \otimes ? } }_{k \in \Zb, B, C \in \Cc}.
    \]
    The above equality is equivalent to the statement: For any \( B, C \in \Cc \), \( k \in \Zb \) and \( f \in \Cc^{\op}(B, C)_{k - i} \), the following equality holds
    \[
        \psi_k^{B, C}\tuple*{ \eta \otimes f } = \widetilde{\psi}_k^{B, C} \tuple*{ \eta \otimes f },
    \]
    which is exactly the following equation,
    \[
        (-1)^{|f||\eta|}(G f) \circ \eta_B = \eta_C \circ (F f).
    \]
\end{remark}

The above remark says that in order to prove some DG-morphism \( \eta \) is in \( \dgM(F, G)_{|\eta|} \), we only have to prove that for any \( B, C \in \Cc \) and \( f \in \Cc^{\op}(B, C)_{|f|} \), the following equality holds
\[
        (-1)^{|f||\eta|}(G f) \circ \eta_B = \eta_C \circ (F f).
\]
This will be useful in defining composition and the unit morphism of \( \dgM \).

\begin{remark}
    \label{rem:dgm_composition_and_unit_morphisms}
    For any \( i, j \in \Zb \), let
    \[
        \eta = \tuple*{\eta_A}_{A \in \Cc} \in \dgM(F, G)_i
    \]
    and
    \[
        \mu = \tuple*{\mu_A}_{A \in \Cc} \in \dgM(G, H)_j.
    \]

    Then define composition (uniquely) as follows,
    \begin{align*}
        c: \dgM(G, H)_j \otimes \dgM(F, G)_i &\to \dgM(F, H)_{i + j} \\
        \mu \otimes \eta &\mapsto \tuple*{\mu_A \circ \eta_A}_{A \in \Cc}.
    \end{align*}

    Likewise, define the unit morphism as follows:
    
    Let \( u_A^{\C_{\dg}} \) denote the unit morphism for \( A \in \C_{\dg} \). Then the unit morphism for \( F \in \dgM \) is defined as the following product,
    \[
        u_F^{\dgM} = \prod_{A \in \Cc} u_{F A}^{\C_{\dg}} : I \to \dgM(F, F) \subseteq \prod_{A \in \Cc} \C_{\dg}(F A, F A).
    \]

    Now we want to verify that composition and the unit morphisms are well-defined and correct.

    We have to verify that composition is well-defined, that composition is a chain morphism, that the codomain of composition is in \( \dgM \), and fourth, that the codomain of the unit morphisms is in \( \dgM \).

    First, notice that by \autoref{lem:map_out_of_tensor_unique}, it follows that for every degree \( n \), \( c_n \) is a unique and well-defined morphism.

    Second, we check if \( c \) is a chain morphism. Let \( n \in \Zb \), \( i, j \in \Zb \), with \( i + j = n \),
    \[
        \eta = \tuple*{\eta_A}_{A \in \Cc} \in \dgM(F, G)_i,
    \]
    and
    \[
        \mu = \tuple*{\mu_A}_{A \in \Cc} \in \dgM(G, H)_j.
    \]
    % TODO: Kan ein forenkla med å bruka at komposisjon i C_dg er kjedemorfi? Får ein rar konsekvens, sjå Fun_dg s. 3 i RM. Stemmer beviset?
    % TODO: Indeks og domene på differentsial?
    We have
    \begin{align*}
        &d_n \circ c_n (\mu \otimes \eta) \\
        &= d_n \tuple*{\mu_A \circ \eta_A}_{A \in \Cc} \\
        &= \tuple*{d_{H A} \circ \mu_A \circ \eta_A - (-1)^{i + j} \mu_A \circ \eta_A \circ d_{F A}}_{A \in \Cc} \\
        &= \tuple*{d_{H A} \circ \mu_A \circ \eta_A - (-1)^i \mu_A \circ d_{G A} \circ \eta_A + (-1)^i \mu_A \circ d_{G A} \circ \eta_A - (-1)^{i + j} \mu_A \circ \eta_A \circ d_{F A}}_{A \in \Cc} \\
        &= c_{n + 1} \tuple*{ (d_{H A} \circ \mu_A - (-1)^i \mu_A \circ d_{G A})_{A \in \Cc} \otimes \eta + (-1)^i \mu \otimes (d_{G A} \circ \eta_A - (-1)^j \eta_A \circ d_{F A})_{A \in \Cc} } \\
        &= c_{n + 1} \tuple*{ d(\mu) \otimes \eta + (-1)^i \mu \otimes d(\eta) } \\
        &= c_{n + 1} \circ d_n (\mu \otimes \eta).
    \end{align*}
    Then by uniqueness of \autoref{lem:map_out_of_tensor_unique}, it follows that \( d_n \circ c_n = c_{n + 1} \circ d_n \), and composition is therefore a chain morphism.
    
    Third, we want to verify that the codomain of the composition is in \( \dgM \). Let \( \mu \) and \( \eta \) be as above. It is sufficient to verify that \( \mu \circ \eta \in \dgM(F, H)_n \).

    Let \( f \in \Cc^{\op}(B, C) \), and consider the following equation,
    \begin{align*}
        (\mu \circ \eta)_C \circ (F f) &= \mu_C \circ \eta_C \circ (F f) \\
        &= \mu_C \circ (-1)^{|f||\eta|} (G f) \circ \eta_B \\
        &= (-1)^{|f||\mu|}(-1)^{|f||\eta|} (H f) \circ \mu_B \circ \eta_B \\
        &= (-1)^{|f|(|\mu| + |\eta|)} (H f) \circ (\mu \circ \eta)_B,
    \end{align*}
    which is exactly the property equivalent with being an element of \( \dgM(F, H)_n \).

    Fourth, we need to verify that the codomain of \( u_F \) is \( \dgM(F, F) \):
    
    This is equivalent to checking if for any \( r \in R \), we have
    \[
        \tuple*{ r \Id_{A_i} }_{i \in \Zb, A \in \Cc} \in \dgM(F, F)_0,
    \]
    and checking that for \( n \neq 0 \), we have
    \[
        0 \in \dgM(F, F)_n.
    \]

    First, for any \( f \in \Cc^{\op}(A, B)_n \), consider
    \begin{align*}
        (F f) \circ (r \Id_{A_i})_{i \in \Zb} &= r ((F f)_i \circ \Id_{A_i})_{i \in \Zb} \\
        &= r ((F f)_i)_{i \in \Zb} \\
        &= r (\Id_{B_{i + n}} \circ (F f)_i)_{i \in \Zb} \\
        &= (r \Id_{B_i})_{i \in \Zb} \circ (F f),
    \end{align*}
    which implies \( \tuple*{ r \Id_{A_i} }_{i \in \Zb, A \in \Cc} \in \dgM(F, F)_0 \).

    Second, we get that for any \( n \in \Zb \), any \( F, G \in \dgM, \) and any \( f \in \Cc^{\op}(A, B)_n, \) the following equality holds
    \[
        (-1)^{|f|0}G(f) \circ 0 = 0 = 0 \circ F(f).
    \]
    Therefore, for any \( n \in \Zb \),
    \[
        0 \in \dgM(F, F)_n,
    \]
    which implies the second property, and the codomain of the unit morphisms is therefore in \( \dgM \).
\end{remark}

Similar to how we could simplify the notation for the differential of \( \C_{\dg} \) in \autoref{rem:c_dg_differential}, we can similarly simplify the differential of \( \dgM \).

\begin{remark}
    For any \( A \in \C_{\dg} \), let \( d_A \) be as in \autoref{rem:c_dg_differential}.

    Then for any \( F \in \dgM \) consider the degree \( 1 \) DG-morphism
    \[
        d_F := \tuple*{d_{FA}}_{A \in \Cc} \in \prod_{A \in \Cc} \C_{\dg}(FA, FA)_1.
    \]

    It follows that for any \( \eta = \tuple*{\eta_A}_{A \in \Cc} \in \dgM(F, G)_i \) we have
    \[
        d_i^{\dgM(F, G)}(\eta) = d_G \circ \eta - (-1)^i \eta \circ d_F
    \]
    as
    \begin{align*}
        d_i^{\dgM(F, G)}(\eta) &= \tuple*{ d_{G A} \circ \eta_A - (-1)^i \eta_A \circ d_{F A} }_{A \in \Cc} \\
        &= \tuple*{ d_{G A} \circ \eta_A }_{A \in \Cc} - (-1)^i \tuple*{ \eta_A \circ d_{F A} }_{A \in \Cc} \\
        &= \tuple*{d_{G A}}_{A \in \Cc} \circ \tuple*{\eta_A}_{A \in \Cc} - (-1)^i \tuple*{\eta_A}_{A \in \Cc} \circ \tuple*{d_{F A}}_{A \in \Cc} \\
        &= d_G \circ \eta - (-1)^i \eta \circ d_F.
    \end{align*}
\end{remark}

Note that \( d_F \) is \emph{not} an element of \( \dgM(F, F)_1 \) in general, because for any \( f \in \Cc^{\op}(B, C)_n \), we have
\[
    d_{F C} \circ (F f) - (-1)^n (F f) \circ d_{F B} = d_n^{[F B, F C]}(F f) = F d_n^{\Cc(B, C)} (f),
\]
which, in order for \( d_F \) to be a DG-natural transformation, implies that \( F d_n^{\Cc(B, C)} (f) = 0 \) for any \( f \), which will only be true if every DG-morphism in \( \Cc \) is a cycle, or if \( F \) maps all boundaries to \( 0 \), both of which are not true in general.