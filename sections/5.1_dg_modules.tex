In broad terms, a DG-module is a functor from some DG category into an a specific DG-category. The following definition, which is based on \cite[p. 29]{Jasso-Muro_2023}, is the category that will be the co-domain of these DG-modules.

% TODO: Verify definition with borceux definition of composition p. 295
\begin{definition}[\( \C_{\dg} \)]
    \label{def:c_dg_mod_r}
    Let \( R \) be a commutative ring with identity.

    Then let \emph{\( \C_{\dg}(\Mod(R)) \)}, or shortened to \emph{\( \C_{\dg} \)}, be the DG-category defined as follows
    \begin{enumerate}
        \item {
            \( \Obj(\C_{\dg}) := \Obj(\C) \).
        }
        \item {
            Let \( A, B \in \C_{\dg} \). Let \( \class*{A, B} \) denote the internal hom of \( \C \) (\autoref{def:internal_hom_of_chain_complexes_over_Mod(R)}) with respect to \( A, B \) as objects from \( \C \).

            Then let \( \C_{\dg}(A, B) := \class*{A, B} \).
        }
        \item {
            For \( A, B, C \in \C_{\dg} \), let
            \[
                c_{\C_{\dg}}: \C_{\dg}(B, C) \otimes \C_{\dg}(A, B) \to \C_{\dg}(A, C)
            \]
            be defined as the chain morphism where \( c_{\C_{\dg}, n} \) is uniquely defined on
            \[
                \tuple*{g_p}_{p \in \Zb} \in [B, C]_i \text{ and } \tuple*{f_q}_{q \in \Zb} \in [A, B]_j
            \]
            as follows
            \begin{align*}
                c_{\C_{\dg}, n}: \tuple*{ \C_{\dg}(B, C) \otimes \C_{\dg}(A, B) }_n &\to \C_{\dg}(A, C)_n \\
                \tuple*{g_p}_{p \in \Zb} \otimes \tuple*{f_q}_{q \in \Zb} &\mapsto \tuple*{g_{p + j} \circ f_p}_{p \in \Zb}.
            \end{align*}
        }
        \item {
            Let \( A \in \C_{\dg} \).

            Then the unit morphism,
            \[
                u_A: I \to \C_{\dg}(A, A),
            \]
            is the chain morphism where in degree \( 0 \) it is
            \begin{align*}
                u_{A, 0} : R &\to [A, A]_0 \\
                r &\mapsto \tuple{r\Id_{A_i}}_{i \in \Zb}
            \end{align*}
            and \( 0 \) in every other degree.
        }
    \end{enumerate}
\end{definition}

The following remark shows why the composition above is well-defined.

\begin{remark}
    The composition definition in \autoref{def:c_dg_mod_r} is well defined by the following two arguments:

    \begin{enumerate}
        \item {
            Firstly, the morphisms \( c_{\C_{\dg}, n} \) are uniqely defined by \autoref{lem:map_out_of_tensor_unique} where the \( g_{i, j} \)'s are as follows
            \begin{align*}
                g_{i, j}: \prod_{p \in \Zb} \Mod(R)(B_p, C_{p + i}) \times \prod_{p \in \Zb} \Mod(R)(A_p, B_{p + j}) &\to \prod_{p \in \Zb} \Mod(R)(A_p, C_{p + i + j}) \\
                \tuple*{ g_p }_{p \in \Zb} \times \tuple*{ f_q }_{q \in \Zb} &\mapsto \tuple*{ g_{p + j} \circ f_p }_{p \in \Zb}.
            \end{align*}
            And these can be checked to be \( R \)-bilinear.
        }
        \item {
            Secondly, need to check that the \( c_{\C_{\dg}, n} \)'s form a chain morphism.

            Look at the following equation (with shortened notation for brevity) 
            \begin{align*}
                &( d_{(A, C), n} \circ c_n - c_{n + 1} \circ d_{(B, C) \otimes (A, B), n} )\tuple*{ \tuple*{ g_p }_{p \in \Zb} \otimes \tuple*{ f_q }_{q \in \Zb} } \\
                &= d_{(A, C), n} \circ c_n \tuple*{ \tuple*{ g_p }_{p \in \Zb} \otimes \tuple*{ f_q }_{q \in \Zb} } - c_{n + 1} \circ d_{(B, C) \otimes (A, B), n}\tuple*{ \tuple*{ g_p }_{p \in \Zb} \otimes \tuple*{ f_q }_{q \in \Zb} } \\
                \intertext{by definition of composition as well as differential of tensor product it follows that}
                &= d_{(A, C), n}\tuple*{
                    \tuple*{ g_{p + j} \circ f_p }_{p \in \Zb}
                } \\
                &\hspace{0.4cm} - c_{n + 1} \tuple*{
                    d_{(B, C), i}\tuple*{ \tuple*{ g_p }_{p \in \Zb} } \otimes \tuple*{ f_q }_{q \in \Zb}
                    + (-1)^i \tuple*{ g_p }_{p \in \Zb} \otimes d_{(A, B), j} \tuple*{ \tuple*{ f_q }_{q \in \Zb} }
                } \\
                \intertext{then by the definition of the differential of the internal hom}
                &= \tuple*{
                    d_{C, p + n} \circ g_{p + j} \circ f_p - (-1)^n g_{p + j + 1} \circ f_{p + 1} \circ d_{A, p}
                }_{p \in \Zb} \\
                &\hspace{0.4cm} - c_{n + 1} (
                    \tuple*{
                        d_{C, p + i} \circ g_p - (-1)^i g_{p + 1} \circ d_{B, p}
                    }_{p \in \Zb} \otimes \tuple*{ f_q }_{q \in \Zb} \\
                    &\hspace{0.8cm}+ (-1)^i \tuple*{ g_p }_{p \in \Zb} \otimes \tuple*{
                        d_{B, q + j} \circ f_q - (-1)^j f_{q + 1} \circ d_{A, q}
                    }_{q \in \Zb}
                ) \\
                \intertext{then by the fact that composition is an \( R \)-homomorphism}
                &= \tuple*{
                    d_{C, p + n} \circ g_{p + j} \circ f_p - (-1)^n \circ g_{p + j + 1} \circ f_{p + 1} \circ d_{A, p}
                }_{p \in \Zb} \\
                &\hspace{0.4cm} - c_{n + 1} \tuple*{
                    \tuple*{
                        d_{C, p + i} \circ g_p - (-1)^i g_{p + 1} \circ d_{B, p}
                    }_{p \in \Zb} \otimes \tuple*{ f_q }_{q \in \Zb}
                } \\
                &\hspace{0.4cm} - (-1)^i c_{n + 1} \tuple*{
                    \tuple*{ g_p }_{p \in \Zb} \otimes \tuple*{
                        d_{B, q + j} \circ f_q - (-1)^j f_{q + 1} \circ d_{A, q}
                    }_{q \in \Zb}
                } \\
                \intertext{then by the definition of composition}
                &= \tuple*{
                    d_{C, p + n} \circ g_{p + j} \circ f_p
                }_{p \in \Zb} - (-1)^n \tuple*{
                    g_{p + j + 1} \circ f_{p + 1} \circ d_{A, p}
                }_{p \in \Zb} \\
                &\hspace{0.4cm} - \tuple*{
                    d_{C, p + i + j} \circ g_{p + j} \circ f_p
                }_{p \in \Zb} + (-1)^i \tuple*{
                    g_{p + j + 1} \circ d_{B, p + j} \circ f_p
                }_{p \in \Zb} \\
                &\hspace{0.4cm} - (-1)^i \tuple*{
                    g_{p + j + 1} \circ d_{B, p + j} \circ f_p
                }_{p \in \Zb} + (-1)^{i + j} \tuple*{
                    g_{p + j + 1} \circ f_{p + 1} \circ d_{A, p}
                }_{p \in \Zb} \\
                &= 0.
            \end{align*}
            By \autoref{lem:map_out_of_tensor_unique}, this shows that the morphism
            \[
                d_{(A, C), n} \circ c_n - c_{n + 1} \circ d_{(B, C) \otimes (A, B), n}: \\
                \tuple*{ \C_{\dg}(B, C) \otimes \C_{\dg}(A, B) }_n \to \C_{\dg}(A, C)_{n + 1}
            \]         
            corresponds to the \( g_{i,j} \)'s where \( g_{i, j} = 0 \). However, by uniqueness, this implies that
            \[
                d_{(A, C), n} \circ c_n - c_{n + 1} \circ d_{(B, C) \otimes (A, B), n} = 0.
            \]
        }
    \end{enumerate}
\end{remark}

One can also through ardous calculations using \autoref{rem:tensor_prod_internal_hom_adjoint} find that the definition of composition in \autoref{def:c_dg_mod_r} is the same as in \cite[p. 295]{Borceux_1994}.

Similar to regular modules over a non-commutative ring, DG-modules can be defined as both right or left sided modules. In this thesis, only right DG-modules will be used. Because of that, the following definition is necessary.

\begin{definition}[Opposite DG-category, \( \Cc^{\op} \)]
    \label{def:opposite_dg_category}
    Let \( \Cc \) be a DG-category.

    Then let \( \Cc^{op} \) be the DG-category defined as follows
    \begin{enumerate}
        \item {
            \( \Obj(\Cc^{op}) := \Obj(\Cc) \)
        }
        \item {
            For \( A, B \in \Cc^{op} \), let \( \Cc^{op}(A, B) := \Cc(B, A) \).
        }
        \item {
            Let \( A, B, C \in \Cc^{op} \) and let \( s \) be as in \autoref{rem:symmetry_tensor_product_of_chain_complex}.
            
            Then define composition as the chain morphism
            \[
                c_{\Cc^{\op}} :=  c_{\Cc} \circ s: \Cc^{\op} (B, C) \otimes \Cc^{\op} (A, B) \to \Cc^{\op} (A, C)
            \]
        }
        \item {
            Let the unit morphism be the same as in \( \Cc \).
        }
    \end{enumerate}
    This is called the \emph{opposite DG-category of \( \Cc \)}.
\end{definition}

Another piece of the definition of a DG-module is to define what a functor between DG-categories is. That is the following definition.

\begin{definition}[DG-functor]
    An enriched functor between two DG-categories as defined in \cite[Definition 6.2.3]{Borceux_1994} is called a \emph{DG-functor}.
\end{definition}

An example of a DG-functor is the ``shift functor'' on DG-modules. It is defined in TODO:Ref.

Finally one can define the DG-functor category, which is the final piece in defining DG-modules.

\begin{definition}[DG-functor category, \( \Fun_{\dg}(\Ac, \Bc) \)]
    \label{def:dg_functor_category}
    Let \( \Ac \) and \( \Bc \) be DG-categories over a commutative ring with identity, \( R \). In addition, let \( \Ac \) be small.

    Then let \( \Fun_{\dg}(\Ac, \Bc) \) be the following DG-category:
    \begin{enumerate}
        \item{
            Let \( \Obj(\Fun_{\dg}(\Ac, \Bc)) \) be the class of every DG-functor from \( \Ac \) to \( \Bc \).
        }
        \item{
            For \( F, G \in \Fun_{\dg}(\Ac, \Bc) \), let \( \Fun_{\dg}(\Ac, \Bc)(F, G) \) be defined as in \cite[Proposition 6.3.1]{Borceux_1994}.
        }
        \item {
            Let composition be defined as WIP.
        }
    \end{enumerate}
    This is called the \emph{DG-functor category from \( \Ac \) to \( \Bc \)}.
\end{definition}

% TODO: ADD THINGS BELOW THIS LINE WHERE THEY FIT

% \begin{definition}[\( \Sigma_{\C_{\dg}(\Mod(R))} \)]
%     \label{def:sigma_c_dg}
%     Let the DG-functor \( \Sigma_{\C_{\dg}(\Mod(R))} \) be defined as follows
%     \begin{enumerate}
%         \item {
%             For \( A \in \C_{\dg}(\Mod(R)) \), let
%             \[
%                 \Sigma_{\C_{\dg}(\Mod(R))}(A) := \Sigma_{\C(\Mod(R))}(A)
%             \]
%         }
%         \item {
%             For any \( A, B \in \C_{\dg}(\Mod(R)) \), let
%             \begin{align*}
%                 \Sigma_{\C_{\dg}(\Mod(R))}: \C_{\dg}(\Mod(R))(A, B) \to \C_{\dg}(\Mod(R))(\Sigma_{\C_{\dg}(\Mod(R))}(A), \Sigma_{\C_{\dg}(\Mod(R))}(B))
%             \end{align*}
%             be the chain homomorphism where the \( n \)-th map is
%             \begin{align*}
%                 \Sigma_{\C_{\dg}(\Mod(R)), n}: \C_{\dg}(\Mod(R))(A, B)_n &\to \C_{\dg}(\Mod(R))(\Sigma A, \Sigma B)_n \\
%                 \tuple*{ f_p }_{p \in \Zb} &\mapsto (-1)^n \tuple*{ f_{p + 1} }_{p \in \Zb}.
%             \end{align*}
%         }
%     \end{enumerate}
% \end{definition}
% \begin{remark}
%     In order to show that \( \Sigma_{\C_{\dg}(\Mod(R))} \) from \autoref{def:sigma_c_dg} is a DG-functor, need to show the following three properties:
%     \begin{enumerate}
%         \item {
%             Firstly, need to show that
%             \[
%                 \Sigma_{\C_{\dg}(\Mod(R))}: \C_{\dg}(\Mod(R))(A, B) \to \C_{\dg}(\Mod(R))(\Sigma_{\C_{\dg}(\Mod(R))}(A), \Sigma_{\C_{\dg}(\Mod(R))}(B))
%             \]
%             is a chain morphism by checking if it commutes with the differential.

%             WIP
%         }
%         \item {
%             Secondly, need to show the ``functoriality'' of \( \Sigma_{\C_{\dg}(\Mod(R))} \).

%             WIP
%         }
%         \item {
%             Thirdly, need to show that the unit morphism commutes with \( \Sigma_{\C_{\dg}(\Mod(R))} \).

%             WIP
%         }
%     \end{enumerate}
% \end{remark}