An example of a triangulated category is the Spanier--Whitehead category. It is a \emph{topological triangulated category}. The definition of topological triangulated categories is ``… any triangulated category which is equivalent to a full triangulated subcategory of the homotopy category of a stable model category'' \cite[p.\ 6]{Schwede_2010}. In the same paragraph as the above quote, Schwede also explains why the Spanier--Whitehead category is topological. It will become clear that the Spanier--Whitehead category is an example of a triangulated category inspired by topology, and so the designation as a topological triangulated category fits (morally). However, the definition of a topological triangulated category is not relevant for this thesis, and so we will not go into further details. 
% Schwede says "For us" før sitatet. Betyr det at det ikkje er ein definitsjon, men blir implisert av definisjonen? Er homotopikategori av ein cofibration category betre def?

First, some notation.
\begin{notation}
    Let \( X \) and \( Y \) be two based topological spaces, with \( x_0 \) the basepoint of \( X \).

    We write \( \class*{ X, Y } \) for the set of basepoint-preserving homotopy classes of continuous basepoint-preserving functions from \( X \) to \( Y \).

    Let \( I \) denote the closed unit interval with basepoint \( 0 \).

    We write \( \Sigma X \) for the reduced suspension of \( X \), i.e., the quotient space
    \[
        X \times I / \sim
    \]
    where \( (x, 0) \sim (x_0, t) \sim (x, 1) \) for all \( x \in X \)  and \( t \in I \).
\end{notation}

We can now define the Spanier--Whitehead category. The triangulation will be described afterwards.

The Spanier--Whitehead category is motivated by the Freudenthal suspension theorem and historically led to the definition of the stable homotopy category. The reason the Freudenthal suspension theorem is so central to the Spanier--Whitehead category is the following corollary, whose proof can be found in \cite[Remark 5.2]{Daria_Bachelor}.

\begin{corollary}
    Let \( X \) and \( Y \) be based and finite CW-complexes, and let \( n, m \in \Nb \).
    
    Then the colimit
    \[
        \colim_{q \to \infty} \class*{ \Sigma^{n + q} X, \Sigma^{m + q} Y }
    \]
    is attained for a finite \( q \in \Nb \).
\end{corollary}

The above corollary of the Freudenthal suspension theorem is crucial to understanding why the Spanier--Whitehead category is defined as it is. The following definition is based on \cite[Definition 2]{Schwede_2010}.

\begin{definition}[Spanier--Whitehead category]
    \label{def:sw-cat}
    Let \( SW \) be the category with the following properties:
    \begin{enumerate}
        \item {
            Objects in \( SW \) are pairs consisting of a based CW-complex, \( X \), and an integer, \( n \), i.e., \( \tuple*{X, n} \).
        }
        \item {
            Let \( \Sigma \) denote the reduced suspension of a topological space.

            Morphisms in \( SW \) are the following colimits of abelian groups
            \[
                SW\tuple*{ (X, n), (Y, m) } := \colim_{\stackrel{q \to \infty}{q \geq \max(|n| + 2, |m|)}} \class*{ \Sigma^{n + q}(X), \Sigma^{m + q}(Y) }
            \]
        }
        \item {
            Composition is the induced colimit morphism from ordinary composition in each degree.
        }
    \end{enumerate}

    Then \( SW \) is called the \emph{Spanier--Whitehead category}.
\end{definition}

The reason for limiting the \( q \) in the colimit in part 2, is to make sure that the colimit is well-defined and taken over abelian groups. This is because \( \class*{\Sigma^n X, Y} \) is only an abelian group if \( n \geq 2 \). 

Composition is well-defined by functoriality of the reduced suspension functor.

It can be shown that the Spanier--Whitehead category is an additive category \cite[Proposition 5.7]{Daria_Bachelor}.

To define the triangulated structure of the Spanier--Whitehead category it is necessary to define the shift functor and the class of distinguished triangles.

The definition of the shift functor is dependent on the following property.
\begin{lemma}
    \label{lem:sw_colim_canonical_iso}
    There is a unique, canonical isomorphism,
    \[
        SW\tuple*{ (X, n), (Y, m) } \stackrel{\sim}{\to} SW\tuple*{ (X, n + 1), (Y, m + 1) }.
    \]
\end{lemma}
\begin{proof}
    From the definition of the colimit, \( SW\tuple*{ (X, n), (Y, m) } \) is a colimit of the same diagram as \( SW\tuple*{ (X, n + 1), (Y, m + 1) } \), and vice versa.
\end{proof}

% Marius: Her har eg fortrengt den kanoniske isomorfien.
The following defines the shift functor.
\begin{definition}[Shift functor in \( SW \)]
    \label{def:sw-shift}
    Let \( \Sigma_{SW} \) be the following assignment of objects and morphisms in \( SW \).

    For any \( (X, n) \in SW \), let
    \[
        \Sigma_{SW} (X, n) := (X, n + 1),
    \] 
    and for \( f \in SW((X, n), (Y, m)) \), let
    \[
        \Sigma_{SW} f := f,
    \]
    where we have implicitly applied the canonical isomorphism from the previous lemma.

    We will denote \( \Sigma_{SW} \) as simply \( \Sigma \) when it is clear from the context.
\end{definition}

The shift functor in \autoref{def:sw-shift} is an automorphism.

As the notation would imply, in \( SW \), we have that \( \Sigma_{SW} (X, n) = (X, n + 1) \cong ( \Sigma X, n ) \).

Before defining the distinguished triangles, we first define what a ``mapping cone'' is.
\begin{definition}[Mapping cone \( C_f \)]
    Let \( f: X \to Y \) be a based continuous map between based topological spaces, with \( x_0 \) being the basepoint of \( X \). Let \( I \) be the closed unit interval.

    Then let the \emph{mapping cylinder}, denoted \( M_f \), be the quotient space
    \[
        (X \times I) \sqcup Y / \sim
    \]
    with the relation \( (x, 0) = f(x) \) for all \( x \in X \).

    In addition, let the \emph{mapping cone}, denoted \( C_f \), be defined as the quotient space
    \[
        M_f / \sim
    \]
    with the relation \( (x, 1) = (x', 1) \) for all \( x, x' \in X \) as well as \( (x_0, t) \sim (x_0, 0) \) for all \( t \in I \).
\end{definition}

For a continuous based map \( f: X \to Y \), let \( \iota_f: Y \rightarrowtail C_f \) be the inclusion of \( Y \) into \( C_f \). In addition, let \( \pi_f: C_f \twoheadrightarrow \Sigma X \) map the subspace \( Y \) of \( C_f \) to \( x_0 \), and any point \( (x, t) \) in the \( X \times I \) subspace is mapped to \( (x, t) \in \Sigma X \).

The following definition of the distinguished triangles in \( SW \) omits a lot of details as it is not necessary for our purposes. More details can be found in \cite[Definition 4.7 and 5.8]{Daria_Bachelor}.

\begin{definition}[Distinguished triangles in \( SW \)]
    \label{def:sw-dist_triangles}
    Let \( \Delta \) be the collection of triangles in \( SW \) satisfying the following property.
    
    A triangle
    \[
        (X, n) \to (Y, m) \to (Z, l) \to \Sigma (X, n)
    \]
    is in \( \Delta \) if and only if there is some continuous based map \( f: A \to B \) between two based CW-complexes, and an even number \( k \), such that the following triangle in the homotopy category of based CW-complexes
    \[
        \Sigma^{n + k} X \to \Sigma^{m + k} Y \to \Sigma^{l + k} Z \to \Sigma^{n + k + 1} X
    \]
    is isomorphic to
    \begin{center}
        \begin{tikzpicture}
            \diagram{m}{1cm}{1cm} {
                A \& B \& C_f \& \Sigma A. \\
            };

            \draw[math]
                (m-1-1) edge node {[f]} (m-1-2)
                (m-1-2) edge node {[\iota_f]} (m-1-3)
                (m-1-3) edge node {[\pi_f]} (m-1-4);
        \end{tikzpicture}
    \end{center}
\end{definition}

The reason for restricting \( k \) to an even number is in order to prove the right rotation axiom as well as proving that the trivial triangle is distinguished.

Finally, we can define \( SW \) as a triangulated category.

\begin{example}
    Let \( SW \) be as in \autoref{def:sw-cat}, let \( \Sigma: SW \to SW \) be as in \autoref{def:sw-shift} and let \( \Delta \) be as in \autoref{def:sw-dist_triangles}.

    Then \( \tuple*{SW, \Sigma, \Delta} \) is a triangulated category.
\end{example}
For a proof of this, see \cite[Theorem 5.9]{Daria_Bachelor}.

From this example of a triangulated category we can see where a lot of the notation in the definition of a triangulated category comes from. It is no coincidence that it is common to use the same symbol for both reduced suspension and for the shift functor (\( \Sigma \)), and the same symbol for the mapping cone as well as the cone in a triangulated category (\( C_f \)).