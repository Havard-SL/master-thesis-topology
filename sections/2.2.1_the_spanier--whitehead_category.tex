The first example of a triangulated category is the Spanier--Whitehead category. It is an example of a \emph{topological triangulated category}. The definition of topological triangulated categories is ``... any triangulated category which is equivalent to a full triangulated subcategory of the homotopy category of a stable model category.'' \cite[At the top of p. 6 in the standalone article]{Schwede_2010}. In the same paragraph as the above quote, Schwede also explains why the Spanier--Whitehead category is topological. It will become clear that the Spanier--Whitehead category is an example of a triangulated category inspired by topology, and so the designation as a topological triangulated category fits (morally). However, the definition of a topological triangulated category is not relevant for this thesis, and so we won't go into furter details. 
% Schwede says "For us" før sitatet. Betyr det at det ikkje er ein definitsjon, men blir implisert av definisjonen? Er homotopikategori av ein cofibration category betre def?

First, some notation.
\begin{notation}
    Let \( X \) and \( Y \) be two pointed topological spaces.

    Then let \( \class{ X, Y } \) denote the basepoint-preserving homotopy class of continous basepoint-preserving functions from \( X \) to \( Y \).
\end{notation}

We can now define the \emph{underlying} Spanier--Whitehead category. The triangulated structure will be descirbed afterwards.

The Spanier--Whitehead category is motivated by the Freudental suspension theorem and historically lead to the definition of the stable homotopy category. The reason why the Freudental suspension theorem is so central to the Spanier--Whitehead is the following corollary, whose proof can be found in \cite[Remark 5.2]{Daria_Bachelor}.

\begin{corollary}
    \label{cor:sw_freudenthal_suspension}
    Let \( X \) and \( Y \) be pointed and finite CW-complexes, let \( n, m \in \Nb \) and let \( \Sigma \) denote the reduced suspension of a topological space.
    
    Then the colimit
    \[
        \colim_{q \to \infty} \class{ \Sigma^{n + q}(X), \Sigma^{m + q}(Y) }
    \]
    is attained after a finite \( q \in \Nb \).
\end{corollary}

The above corollary is crucial to understanding why the Spanier--Whitehead category is defined as it is. The following definition is based on \cite[Definition 2]{Schwede_2010}.

\begin{definition}[Spanier--Whitehead category]
    \label{def:sw-cat}
    Let \( SW \) be the category with the following properties:
    \begin{enumerate}
        \item {
            Objects in \( SW \) are tuples of a pointed CW-complexes and an integer, \( \tuple{X, n} \).
        }
        \item {
            Let \( \Sigma \) denote the reduced suspension of a topological space.

            Morphisms in \( SW \) are the following colimits of abelian groups
            \[
                SW\tuple{ (X, n), (Y, m) } := \colim_{\stackrel{q \to \infty}{q \geq \max(|n| + 2, |m|)}} \class{ \Sigma^{n + q}(X), \Sigma^{m + q}(Y) }
            \]
        }
        \item {
            Composition is defined by the usual composition, but for a high enough \( q \) such that by \autoref{cor:sw_freudenthal_suspension} the colimit is attained for both morphisms. Then the resulting composed morphism is embedded into the colimit by taking repeated suspensions if necessary.
        }
    \end{enumerate}

    Then \( SW \) is called the \emph{Spanier--Whitehead category}.
\end{definition}

The reason for limiting the \( q \) in the colimit in item 2, is to make sure that the colimit is well-defined and taken over abelian groups. This is because \( \class{\Sigma^n X, Y} \) is only an abelian group if \( n \geq 2 \).

It can be shown that the Spanier--Whitehead category is an additive category \cite[Proposition 5.7]{Daria_Bachelor}.

To see the triangulated structure of the Spanier--Whitehead category it is necessary to define the shift functor and the class of distinguished triangles.

\begin{definition}[Shift functor in \( SW \)]
    \label{def:sw-shift}
    Let \( \Sigma \) be the following assignment of objects and morphisms in \( SW \).

    Let
    \[
        \Sigma(X, n) := (X, n + 1),
    \] 
    and for \( f: (X, n) \to (Y, m) \), let
    \[
        \Sigma f := f: (X, n + 1) \to (Y, m + 1).
    \]

    This can be shown to be a functor, and an automorphism.
\end{definition}

As the notation would imply, in \( SW \), one has that \( \Sigma(X, n) = (X, n + 1) \cong ( \Sigma X, n ) \).

The following definition of the distinguished triangles to make \( SW \) triangulated omits a lot of details as it is not necessary later in this thesis. For those who are interested, more details can be found in \cite[Definition 5.8, Definition 4.7]{Daria_Bachelor}.

\begin{definition}[Distinguished triangles in \( SW \)]
    \label{def:sw-dist_triangles}
    Let \( \Delta \) be the collection of triangles in \( SW \) satisfying the following property.

    A triangle
    \[
        (X, n) \to (Y, m) \to (Z, l) \to (X, n + 1)
    \]
    is in \( \Delta \) if and only if there is some even number \( k \), such that the following triangle in the homotopy category of pointed CW-complexes
    \[
        \Sigma^{n + k} X \to \Sigma^{m + k} Y \to \Sigma^{l + k} Z \to \Sigma^{n + k + 1} X
    \]
    is isomorphic as a triangle in the homotopy category of pointed CW-complexes to a triangle of the form
    \[
        A \stackrel{f}{\to} B \to C_f \to \Sigma A
    \]
    where \( C_f \) is the mapping cone of \( f \).
\end{definition}

Finally one can define \( SW \) as a triangulated category.

\begin{example}
    Let \( SW \) be as in \autoref{def:sw-cat}, let \( \Sigma: SW \to SW \) be as in \autoref{def:sw-shift} and let \( \Delta \) be as in \autoref{def:sw-dist_triangles}.

    Then \( \tuple{SW, \Sigma, \Delta} \) is a triangulated category.
\end{example}

For a proof of the Spanier--Whitehead category being triangulated, see \cite[Theorem 5.9]{Daria_Bachelor}.

From this example of a triangulated category one can see where a lot of the notation in the definition of a triangulated category comes from. There is no coincidence that it is common to use the same symbol for both reduced suspension and for the shift functor (\( \Sigma \)), and the same symbol for the mapping cone as well as the cone in a triangulated category (\( C_f \)).
