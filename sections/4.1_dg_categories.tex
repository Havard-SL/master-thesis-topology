Before defining Massey products on the cohomology category of a DG-category, we first have to define what a DG-category is.

We will use the definition of DG-categories based on enriched category theory, as it is both modern, and is helpful for defining the DG-functor category in subsequent sections. The enriched category theory in this section is mainly inspired by \cite[Section 6.2]{Borceux_1994}.

The first definition necessary for understanding DG-categories, is the tensor product of chain complexes over modules. However, this definition uses coproducts, and we will use special notation for elements in the coproduct.

\begin{notation}
    \label{not:coprod}
    Let \( R \) be a commutative ring with identity. Let \( A_i \in \Mod(R) \) and let
    \[
        \iota_i: A_i \rightarrowtail \coprod_{i \in \Zb} A_i
    \]
    denote the canonical split monomorphism by the universal property of the coproduct in \( \Mod(R) \).

    Then for any \( a_i \in A_i \), the element
    \[
        \iota_i(a_i) \in \coprod_{i \in \Zb} A_i
    \]
    is simply denoted as
    \[
        a_i \in \coprod_{i \in \Zb} A_i.
    \]
\end{notation}

The reasoning for the above notation is twofold. First, it reduces notation while not being ambiguous. Second, we never use a general element of a coproduct. Almost always when considering what a morphism does to an element of the coproduct, it is what happens to the \( \iota_i(a_i) \), which is a consequence of the universal property of the coproduct.

The following is the definition of the tensor product of chain complexes.

\begin{definition}[Tensor product of \( \C \)]
    \label{def:tensor_product_of_chain_complexes_over_Mod(R)}
    Let \( R \) be a commutative ring with identity.
    
    Let \( ?_1 \otimes ?_2 \) be the following assignment of objects and morphisms in \( \C \times \C \):
    \begin{itemize}
        \item {
            For \( A, B \in \C \), and for any  \( n \in \Zb \) define the modules
            \[
                (A \otimes B)_n := \coprod_{p + q = n} A_p \otimes B_q
            \]
            which are a part of the chain complex
            \begin{center}
                \begin{tikzpicture}
                    \diagram{m}{1cm}{1cm} {
                        A \otimes B: \\
                    };
                \end{tikzpicture}
                %
                \begin{tikzpicture}
                    \diagram{m}{1cm}{1cm} {
                        \cdots \& \tuple*{A \otimes B}_{-1} \& \tuple*{A \otimes B}_0 \& \tuple*{A \otimes B}_1 \& \cdots \\
                    };

                    \draw[math]
                        (m-1-1) edge (m-1-2)
                        (m-1-2) edge node {d_{-1}} (m-1-3)
                        (m-1-3) edge node {d_0} (m-1-4)
                        (m-1-4) edge (m-1-5);
                \end{tikzpicture}
            \end{center}
            where the differentials, \( d_n \), are defined as follows:
            
            Let \( i + j = n \) and \( a \otimes b \in A_i \otimes B_j \) be an elementary tensor.

            Then the differential is (uniquely) defined by the following assignments
            \[
                d_n(a \otimes b) := d_{A, i}(a) \otimes b + (-1)^{i} a \otimes d_{B, j}(b).
            \]
        }
        \item {
            For morphisms \( f \in \C(A, A') \) and \( g \in \C(B, B') \), let \( f \otimes g \) be the following chain morphism,
            \begin{center}
                \begin{tikzpicture}
                    \diagram{m}{1cm}{1cm} {
                        A \otimes B: \&[-0.7cm] \cdots \& \tuple*{A \otimes B}_{-1} \& \tuple*{A \otimes B}_0 \& \tuple*{A \otimes B}_1 \& \cdots \\
                        A' \otimes B': \& \cdots \& \tuple*{A' \otimes B'}_{-1} \& \tuple*{A' \otimes B'}_0 \& \tuple*{A' \otimes B'}_1 \& \cdots \\
                    };

                    \draw[math]
                        (m-1-1) edge node {f \otimes g} (m-2-1)
                        (m-1-2) edge (m-1-3)
                        (m-1-3) edge node {d_{-1}} (m-1-4)
                            edge node {\coprod_{i + j = -1} f_i \otimes g_j} (m-2-3)
                        (m-1-4) edge node {d_0} (m-1-5)
                            edge node {\coprod_{i + j = 0} f_i \otimes g_j} (m-2-4)
                        (m-1-5) edge (m-1-6)
                            edge node {\coprod_{i + j = 1} f_i \otimes g_j} (m-2-5)

                        (m-2-2) edge (m-2-3)
                        (m-2-3) edge node {d_{-1}} (m-2-4)
                        (m-2-4) edge node {d_0} (m-2-5)
                        (m-2-5) edge (m-2-6);
                \end{tikzpicture}
            \end{center}
        }
    \end{itemize}

    This is called the \emph{tensor product of \( \C \)}.
\end{definition}

In order to see that the above definition of the differential on \( A \otimes B \) is well-defined, we need the following lemma, which will be useful later as well.

\begin{lemma}
    \label{lem:map_out_of_tensor_unique}
    Let \( A, B \in \C \), \( C \in \Mod(R) \), and \( n \in \Zb \).

    Then a set of \( R \)-bilinear morphisms, \( \set{g_{i, j}}_{i + j = n} \), where
    \[
        g_{i, j}: A_i \times B_j \to C;
    \]
    uniquely define a morphism
    \[
        f \in \Mod(R)\tuple*{(A \otimes B)_n, C}
    \]
    where for any \( a \in A_i \), and \( B \in B_j \), we have
    \[
        f(a \otimes b) = g_{i, j}(a, b).
    \]
\end{lemma}
\begin{proof}
    Consider at the following diagram,
    \begin{diagramlabel}[\label{tikz:differential_of_tensor_product_of_chain_complexes_over_Mod(R)}]
        \begin{tikzpicture}
            \diagram{m}{2cm}{2cm} {
                A_i \otimes B_j \& \coprod\limits_{p + q = n} A_p \otimes B_q \\
                A_i \times B_j \& C, \\
            };

            \draw[math]
                (m-1-1) edge[tailed] node {\iota_{i, j}} (m-1-2)
                    edge[dashed] node {\alpha_{i, j}} (m-2-2)
                (m-1-2) edge[dashed] node {\beta} (m-2-2)

                (m-2-1) edge node {g_{i, j}} (m-2-2)
                    edge node {\otimes} (m-1-1);
        \end{tikzpicture}
    \end{diagramlabel}
    where \( \iota_{i, j} \) is as defined in \autoref{not:coprod}, and where the \( g_{i, j} \) are \( R \)-bilinenar.

    Then by the universal property of tensor product in \( \Mod(R) \), \( g_{i, j} \) induces a unique morphism, \( \alpha_{i, j} \), which is induced from the elementary tensors as follows
    \[
        \alpha_{i, j}: a \otimes b \mapsto g_{i, j}(a, b).
    \]

    Since this works for any \( i, j \) as long as \( i + j = n \), we can construct \( \alpha_{i, j} \) for every valid \( i, j \) pair.

    Then by using the universal property of the coproduct we get the unique map \( \beta \) which is by \autoref{tikz:differential_of_tensor_product_of_chain_complexes_over_Mod(R)} uniquely determined by its actions on elementary tensors \( a \otimes b \in A_i \otimes B_j \) in the following way
    \[
        \beta: a \otimes b \mapsto g_{i, j}(a, b),
    \]
    which is the desired and unique \( f \).
\end{proof}

Given some \( f \in \Mod(R)\tuple*{ (A \otimes B)_n, C } \) we have that for any \( i + j = n \), \( g_{i, j}(a, b) := f(a \otimes b) \) is an \( R \)-bilinear morphism, and therefore \( \set{g_{i, j}}_{i + j = n} \) uniquely define \( f \). In a way, the \( g_{i, j} \)'s ``represent'' \( f \).

The following remark shows that the tensor product of \( \C \) in \autoref{def:tensor_product_of_chain_complexes_over_Mod(R)} is a well-defined bifunctor.

WIP

\begin{remark}
    \label{rem:c_tensor_bifunctor}
    The definition of the differentials in \autoref{def:tensor_product_of_chain_complexes_over_Mod(R)} is well-defined and unique by the following argument.

    We can check that for \( i + j = n \) that
    \begin{align*}
        g_{i, j}: A_i \times B_j &\to (A \otimes B)_{n + 1} \\
        (a, b) &\mapsto d_{A, i}(a) \otimes b + (-1)^i a \otimes d_{B, j}(b)
    \end{align*}
    is \( R \)-bilinear.

    Then by \autoref{lem:map_out_of_tensor_unique} it follows that \( d_n \) is uniquely defined.

    Similarly, by seeing that
    \[
        d_{n + 1} \circ d_n: a \otimes b \mapsto 0
    \]
    and the \( 0 \) map is \( R \)-bilinear, so \( d_{n + 1} \circ d_n \) is uniquely defined. Since the zero map would also send \( a \otimes b \) to \( 0 \), then by uniqueness, \( d_{n + 1} \circ d_n = 0 \).

    In order to verify that the assignment of morphisms is well-defined, we have to check if \( f \otimes g \) is a chain morphism.
    
    Let \( i + j = n \in \Zb \), with \( a \in A_i \) and \( b \in B_j \). Consider the following equation,
    \begin{align*}
        d_n \circ (f \otimes g)_n (a \otimes b) &= d_n \circ (f_i(a) \otimes g_j(b)) \\
        &= (d_{A, i} \circ f_i(a)) \otimes g_j(b) + (-1)^i f_i(a) \otimes (d_{B, j} \circ g_j(b)) \\
        &= (f_{i + 1} \circ d_{A, i} (a)) \otimes g_j (b) + (-1)^i f_i(a) \otimes (g_{j + 1} \circ d_{B, j}(b)) \\
        &= (f \otimes g)_{n + 1} (d_{A, i}(a) \otimes b) + (-1)^i (f \otimes g)_{n + 1} (a \otimes d_{B, j}(b)) \\
        &= (f \otimes g)_{n + 1} (d_{A, i} (a) \otimes b + (-1)^i a \otimes d_{B, j} (b)) \\
        &= (f \otimes g)_{n + 1} \circ d_n (a \otimes b).
    \end{align*}
    Then by \autoref{lem:map_out_of_tensor_unique} it follows that
    \[
        d_n \circ (f \otimes g)_n = (f \otimes g)_{n + 1} \circ d_n.
    \]

    All that remains is to check bifunctoriality.

    First, it is easy to see that \( \Id \otimes \Id \) is the identity morphism.
    
    For \( f \otimes g: A \otimes B \to A' \otimes B' \), \( m \otimes n: A' \otimes B' \to A'' \otimes B'' \), and any \( n \in \Zb \) the composition becomes,
    \begin{align*}
        (m \otimes n)_n \circ (f \otimes g)_n &= (\coprod_{i + j = n} m_i \otimes n_j) \circ (\coprod_{i + j = n} f_i \otimes g_j) \\
        &= \coprod_{i + j = n} (m_i \otimes n_j) \circ (f_i \otimes g_j) \\
        &= \coprod_{i + j = n} (m_i \circ f_i) \otimes (n_j \circ g_j) \\
        &= ((m \circ f) \otimes (n \circ g))_n.
    \end{align*}

    Thus, \( ?_1 \otimes ?_2 \) is a bifunctor.
\end{remark}

An important property of the tensor product of chain complexes is symmetry. This is shown in the following remark.

\begin{remark}[Symmetry of tensor product in \( \C \)]
    \label{rem:symmetry_tensor_product_of_chain_complex}
    For \( A, B \in \C \), there exists an isomorphism
    \[
        s = \tuple*{s_n}_{n \in \Zb}: A \otimes B \to B \otimes A
    \]
    that for any \( n \) and \( i + j = n \) with \( a \in A_i \) and \( b \in B_j \) is defined on the \( n \)th component as follows,
    \[
        s_n: a \otimes b \mapsto (-1)^{ij} b \otimes a.
    \]

    By \autoref{lem:map_out_of_tensor_unique} every \( s_n \) is well-defined. It is also clear that every \( s_n \) is an isomorphism by checking injectivity and surjectivity. It only remains to verify that \( s = \tuple*{s_n}_{n \in \Zb} \) is a chain morphism.

    Consider the following difference
    \begin{align*}
        &s_{n + 1} \circ d_n(a \otimes b) - d_n \circ s_n (a \otimes b) \\
        &= s_{n + 1} \tuple*{d_{A, i}(a) \otimes b + (-1)^i a \otimes d_{B, j}(b)} - d_n \tuple*{ (-1)^{ij} b \otimes a } \\
        &= (-1)^{(i + 1)j} b \otimes d_{A, i}(a) + (-1)^{i + i(j + 1)} d_{B, j}(b) \otimes a - (-1)^{ij} d_{B, j}(b) \otimes a - (-1)^{j + ij} b \otimes d_{A, i}(a) \\
        &= (-1)^{(i + 1)j} b \otimes d_{A, i}(a) - (-1)^{(i + 1)j} b \otimes d_{A, i}(a) + (-1)^{ij} d_{B, j}(b) \otimes a - (-1)^{ij} d_{B, j}(b) \otimes a \\
        &= 0,
    \end{align*}
    which implies that the morphism \( s_{n + 1} \circ d_n - d_n \circ s_n \) sends every elementary tensor to \( 0 \), which again implies that the morphism is equal to \( 0 \).
\end{remark}

Another definition closely linked to the tensor product in \( \C \) is the internal Hom, which is defined using products, that we also use a special notation for.
\begin{notation}
    \label{not:prod}
    Let 
    \[
        \pi_i: \prod_{i \in \Zb} A_i \twoheadrightarrow A_i
    \]
    be the universal split epimorphism by the universal property of the product in \( \Mod(R) \).
    
    Then for any element \( a \in \prod_{i \in \Zb} A_i \), denote \( a \) as
    \[
        a = \tuple*{a_i}_{i \in \Zb}, \: \text{where} \: a_i := \pi_i(a) \in A_i.
    \]
\end{notation}

The reasoning behind this notation is twofold. First, the product in \( \Mod(R) \) is the direct product, and direct products are usually denoted this way. Second, by the universal property of the product, a morphism out of \( \prod_{i \in \Zb} A_i \) is fully dependent on what the resulting value in each degree is, and the above notation captures that property.

As above, we also need some notation for pre- and post-composition.
\begin{notation}
    Let \( \Cc \) be any locally small category, and let \( f \in \Cc(A, B) \).

    Then denote post-composition by \( f \) on any morphism with codomain \( A \) as \( (f)_* \).

    Likewise, denote pre-composition by \( f \) on any morphism with domain \( B \) as \( (f)^* \).
\end{notation}
In \( \Mod(R) \), the pre- and post-composition maps are \( R \)-morphisms on the Hom modules.

The following is the definition of the internal Hom of chain complexes. 
\begin{definition}[Internal hom of \( \C \)]
    \label{def:internal_hom_of_chain_complexes_over_Mod(R)}
    Let \( R \) be a commutative ring with identity.
    
    Let \( [?_1, ?_2] \) be the following assignment of objects and morphisms in \( \C^{\op} \times \C \):

    \begin{itemize}
        \item {
            Let \( A, B \in \C \).

            Then, for any \( n \in \Zb \) define the modules
            \[
                \class*{A, B}_n := \prod_{i \in \Zb} \Mod(R)(A_i, B_{i + n})
            \]
            which are a part of the chain complex
            \begin{center}
                \begin{tikzpicture}
                    \diagram{m}{1cm}{1cm} {
                        \class*{A, B}: \&[-0.7cm] \cdots \& \class*{A, B}_{-1} \& \class*{A, B}_0 \& \class*{A, B}_1 \& \cdots \\
                    };

                    \draw[math]
                        (m-1-2) edge (m-1-3)
                        (m-1-3) edge node {d_{-1}} (m-1-4)
                        (m-1-4) edge node {d_0} (m-1-5)
                        (m-1-5) edge (m-1-6);
                \end{tikzpicture}
            \end{center}
            where the differentials, \( d_n \), are defined as follows,
            \begin{align*}
                d_n : \class*{A, B}_n &\to \class*{A, B}_{n + 1} \\
                \tuple*{f_i}_{i \in \Zb} &\mapsto \tuple*{d_{B, i + n} \circ f_i - (-1)^n f_{i + 1} \circ d_{A, i}}_{i \in \Zb}.
            \end{align*}
        }
        \item {
            For morphisms \( f: \in \C^{\op}(A, A') \) and \( g \in \C(B, B') \), let \( [f, g] \) be the following chain morphism,
            \begin{center}
                \begin{tikzpicture}
                    \diagram{m}{1cm}{1cm} {
                        \class*{A, B}: \&[-0.7cm] \cdots \& \class*{A, B}_{-1} \& \class*{A, B}_0 \& \class*{A, B}_1 \& \cdots \\
                        \class*{A', B'}: \& \cdots \& \class*{A', B'}_{-1} \& \class*{A', B'}_0 \& \class*{A', B'}_1 \& \cdots \\
                    };

                    \draw[math]
                        (m-1-1) edge node {f \otimes g} (m-2-1)
                        (m-1-2) edge (m-1-3)
                        (m-1-3) edge node {d_{-1}} (m-1-4)
                            edge node {\prod_{i \in \Zb} (g_{i - 1})_* \circ (f_i)^*} (m-2-3)
                        (m-1-4) edge node {d_0} (m-1-5)
                            edge node {\prod_{i \in \Zb} (g_i)_* \circ (f_i)^*} (m-2-4)
                        (m-1-5) edge (m-1-6)
                            edge node {\prod_{i \in \Zb} (g_{i + 1})_* \circ (f_i)^*} (m-2-5)

                        (m-2-2) edge (m-2-3)
                        (m-2-3) edge node {d_{-1}} (m-2-4)
                        (m-2-4) edge node {d_0} (m-2-5)
                        (m-2-5) edge (m-2-6);
                \end{tikzpicture}
            \end{center}
        }
    \end{itemize}
    
    This is called the \emph{internal Hom of \( \C \)}.
\end{definition}

By the following remark, internal Hom of \( \C \) is a well-defined bifunctor.
\begin{remark}
    The definition of the differentials of the internal Hom are well-defined because the only thing that needs to be checked is if the \( d_n \) are differentials, which is easy to prove.

    It remains to show that \( [f, g] \) is a chain morphism, as well as bifunctoriality.

    Let \( n \in \Zb \), and \( \tuple*{h_i}_{i \in \Zb} \in [A, B]_n \). Consider the equation,
    \begin{align*}
        d_n \circ [f, g]_n (h_i)_{i \in \Zb} &= d_n \circ \tuple*{g_{i + n} \circ h_i \circ f_i}_{i \in \Zb} \\
        &= \tuple*{d_{B', i + n} \circ g_{i + n} \circ h_i \circ f_i - (-1)^n g_{i + n + 1} \circ h_{i + 1} \circ f_{i + 1} \circ d_{A', i}}_{i \in \Zb} \\
        &= \tuple*{g_{i + n + 1} \circ d_{B, i + n} \circ h_i \circ f_i - (-1)^n g_{i + n + 1} \circ h_{i + 1} \circ d_{A, i} \circ g_i}_{i \in \Zb} \\
        &= \tuple*{(g_{i + n + 1})_* \circ (f_i)^* \circ \tuple*{d_{B, i + n} \circ h_i - (-1)^n h_{i + 1} \circ d_{A, i}}}_{i \in \Zb} \\
        &= \tuple*{\prod_{i \in \Zb} (g_{i + n + 1})_* \circ (f_i)^*} \circ d_n (h_i)_{i \in \Zb} \\
        &= [f, g]_{n + 1} \circ d_n (h_i)_{i \in \Zb}.
    \end{align*}
    Thus, \( [f, g] \) is a chain morphism.

    It is easy to see that \( [\Id, \Id] \) is the identity morphism.

    Finally, for \( [f, g]: [A, B] \to [A', B'] \), \( [v, w]: [A', B'] \to [A'', B''] \), and any \( n \in \Zb \), consider the following equation,
    \begin{align*}
        [f, g]_n \circ [v, w]_n &= \tuple*{\prod_{i \in \Zb} (g_{i + n})_* \circ (f_i)^*} \circ \tuple*{\prod_{i \in \Zb} (w_{i + n})_* \circ (v_i)^*} \\
        &= \prod_{i \in \Zb} \tuple*{(g_{i + n})_* \circ (f_i)^* \circ (w_{i + n})_* \circ (v_i)^*} \\
        &= \prod_{i \in \Zb} \tuple*{(g_{i + n} \circ w_{i + n})_* \circ(f_i \circ_{\op} v_i)^*} \\
        &= [f \circ_{\op} v, g \circ w]_n.
    \end{align*}

    Thus, \( [?_1, ?_2] \) is a well-defined bifunctor.
\end{remark}

As the names would imply, the tensor product and internal Hom of chain complexes over modules are adjoint in the usual sense. This is shown in the following remark.

\begin{remark}[Tensor product and internal Hom adjunction in \( \C \)]
    \label{rem:tensor_prod_internal_hom_adjoint}
    Let \( A, B, \) and \( C \) be chain complexes in \( \C \) for some commutative ring \( R \), and let
    \[
        f \in \C\tuple*{A \otimes B, C}.
    \]
    
    Let \( f = \tuple*{f_n}_{n \in \Zb} \) where \( f_n \in \Mod(R)\tuple*{ \tuple*{A \otimes B}_n, C_n} \) are the individual level-wise morphisms of the chain morphism \( f \).

    Then, unpacking the definitions, we have that \( f \) looks like the following diagram
    \begin{center}
        \begin{tikzpicture}
            \diagram{m}{1cm}{1cm} {
                \cdots \& \coprod\limits_{i + j = -1} A_i \otimes B_j \& \coprod\limits_{i + j = 0} A_i \otimes B_j \& \coprod\limits_{i + j = 1} A_i \otimes B_j \& \cdots \\
                \cdots \& C_{-1} \& C_0 \& C_1 \& \cdots \\
            };

            \draw[math]
                (m-1-1) edge (m-1-2)
                (m-1-2) edge node {d_{A \otimes B, -1}} (m-1-3)
                    edge node {f_{-1}} (m-2-2)
                (m-1-3) edge node {d_{A \otimes B, 0}} (m-1-4)
                    edge node {f_0} (m-2-3)
                (m-1-4) edge (m-1-5)
                    edge node {f_1} (m-2-4)

                (m-2-1) edge (m-2-2)
                (m-2-2) edge node {d_{C, -1}} (m-2-3)
                (m-2-3) edge node {d_{C, 0}} (m-2-4)
                (m-2-4) edge (m-2-5);
        \end{tikzpicture}
    \end{center}
    Likewise, the adjoint has to look like the following diagram
    \begin{center}
        \begin{tikzpicture}
            \diagram{m}{1cm}{0.70cm} {
                \cdots \&[-0.5cm] A_{-1} \& A_0 \& A_1 \&[-0.5cm] \cdots \\
                \cdots \& \prod\limits_{j \in \Zb} \Mod(R)(B_j, C_{j - 1}) \& \prod\limits_{j \in \Zb} \Mod(R)(B_j, C_j) \& \prod\limits_{j \in \Zb} \Mod(R)(B_j, C_{j + 1}) \& \cdots \\
            };

            \draw[math]
                (m-1-1) edge (m-1-2)
                (m-1-2) edge node {d_{A, -1}} (m-1-3)
                    edge node {?_{-1}} (m-2-2)
                (m-1-3) edge node {d_{A, 0}} (m-1-4)
                    edge node {?_0} (m-2-3)
                (m-1-4) edge (m-1-5)
                    edge node {?_1} (m-2-4)

                (m-2-1) edge (m-2-2)
                (m-2-2) edge node {d_{\class*{B, C}, -1}} (m-2-3)
                (m-2-3) edge node {d_{\class*{B, C}, 0}} (m-2-4)
                (m-2-4) edge (m-2-5);
        \end{tikzpicture}
    \end{center}

    Let \( \iota_{i, j} \) be as defined in \autoref{not:coprod}.

    Then take the Hom-tensor adjoint in \( \Mod(R) \) of the morphism
    \[
        f_{i + j} \circ \iota_{i, j}: A_i \otimes B_j \to C_{i + j}.
    \]
    This yields a morphism
    \begin{align*}
        \phi_{f, i, j}: A_i &\to \Mod(R)(B_j, C_{j + i}) \\
        a &\mapsto f_{i + j}\tuple*{ a \otimes ? }.
    \end{align*}
    Then by the universal property of the product there is some unique morphism
    \[
        \phi_{f, i} := \prod_{j \in \Zb} \phi_{f, i, j}: A_i \to \prod_{j \in \Zb} \Mod(R)\tuple*{B_j, C_{j + i}}.
    \]
    Collecting these morphisms yields a morphism, which is a candidate for the adjoint of \( f \), namely
    \[
        \phi_f := \tuple*{\phi_{f, i}}_{i \in \Zb}.
    \]
\end{remark}

The proof that the above construction is correct is in the following.
\begin{proof}[Proof that the construction in \autoref{rem:tensor_prod_internal_hom_adjoint} is the adjoint]\phantom{hei}
    
    In order to show that \autoref{rem:tensor_prod_internal_hom_adjoint} is the proper adjoint definition, we need to show the following properties:
    \begin{enumerate}
        \item {
            First, we want to show that \( \phi_f \) is a chain morphism.

            We need to check that for any \( i \in \Zb \) that the following diagram,
            \begin{center}
                \begin{tikzpicture}
                    \diagram{m}{1cm}{1cm} {
                        A_i \& A_{i + 1} \\
                        \class*{B, C}_i \& \class*{B, C}_{i + 1}, \\
                    };

                    \draw[math]
                        (m-1-1) edge node {d_{A, i}} (m-1-2)
                            edge node {\phi_{f, i}} (m-2-1)
                        (m-1-2) edge node {\phi_{f, i + 1}} (m-2-2)

                        (m-2-1) edge node {d_{\class*{B, C}, i}} (m-2-2);
                \end{tikzpicture}
            \end{center}
            commutes.

            % TODO: Fiks intertext?
            Pick an arbitrary \( a \in A_i \) and consider the following equation
            \begin{align*}
                &\phi_{f, i + 1} \circ d_{A, i}(a) - d_{\class*{B, C}, i} \circ \phi_{f, i}(a) \\
                &= \tuple*{ f_{i + j + 1}\tuple*{d_{A, i}(a) \otimes ?} }_{j \in \Zb}
                - d_{\class*{B, C}, i} \tuple*{ \tuple*{ f_{i + j}\tuple*{a \otimes ?} }_{j \in \Zb} } \\
                \intertext{by expanding out the definition of \( d_{\class*{B, C}, i} \) it follows that}
                &= \tuple*{ f_{i + j + 1}\tuple*{d_{A, i}(a) \otimes ?}
                - d_{C, i + j} \circ f_{i + j}\tuple*{a \otimes ?}
                + (-1)^i f_{i + j + 1}\tuple*{a \otimes d_{B, j}(?)} }_{j \in \Zb} \\
                \intertext{by consolidating the two terms that post-compose by \( f_{i + j + 1} \) it follows that}
                &= \tuple*{ f_{i + j + 1}\bigl( \tuple*{d_{A, i}(a) \otimes ?}
                + (-1)^i\tuple*{ a \otimes d_{B, j}(?) } \bigr)
                - d_{C, i + j} \circ f_{i + j}\tuple*{ a \otimes ? } }_{j \in \Zb} \\
                \intertext{by the definition of the differential of \( A \otimes B \) it follows that}
                &= \tuple*{ f_{i + j + 1} \circ d_{A \otimes B, i + j} \tuple*{ a \otimes ? }
                - d_{C, i + j} \circ f_{i + j} ( a \otimes ? ) }_{j \in \Zb} \\
                \intertext{by \( f \) being a chain homomorphism from \( A \otimes B \) to \( C \) it follows that}
                &= 0.
            \end{align*}
        }
        \item {
            Second, we want to show that the assignment \( f \mapsto \phi_f \) is an isomorphism of groups.

            First, notice that the construction of \( \phi_f \) is fully reversible:

            From \( \phi_{f, i} \), we get \( \phi_{f, i, j} \) by post composing with the canonical projection morphisms as mentioned in \autoref{not:prod}.

            Then, from \( \phi_{f_i, j} \) we get \( f_{i + j} \circ \iota_{i, j} \) by taking the inverse Hom-tensor adjunction.

            From \( f_{i + j} \circ \iota_{i, j} \) we can uniquely get \( f_{i + j} \) by the universal property of coproducts.

            By a similar argument, doing the reverse construction as above and going back again yields the same morphism, therefore \( f \mapsto \phi_f \) is a bijection.

            Finally, notice that every step of the construction is \( R \)-linear, therefore \( f \mapsto \phi_f \) is an isomorphism of modules.
        }
        \item {
            Third, we want to show there is a natural isomorphism
            \[
                \C(?_1 \otimes ?_2, ?_3) \cong \C(?_1, \left[ ?_2, ?_3 \right])
            \]
            where the natural isomorphisms are \( f \mapsto \phi_f \).

            Let
            \[
                f \times g \times h \in \C^{\op} \times \C^{\op} \times \C (A \times B \times C, A' \times B' \times C').
            \]

            We want to show that the following square commutes,
            \begin{center}
                \begin{tikzpicture}
                    \diagram{m}{1cm}{2cm} {
                        \C(A \otimes B, C) \& \C(A' \otimes B', C') \\
                        \C(A, [B, C]) \& \C(A', [B', C']), \\
                    };

                    \draw[math]
                        (m-1-1) edge node {\C(f \otimes g, h)} (m-1-2)
                            edge (m-2-1)
                        (m-1-2) edge (m-2-2)

                        (m-2-1) edge node {\C(f, [g, h])} (m-2-2);
                \end{tikzpicture}
            \end{center}
            where the vertical morphisms are doing the construction above.

            Consider a chain morphism \( x = \tuple*{x_i}_{i \in \Zb} \in \C(A \otimes B, C) \). We want to show that
            \[
                \phi_{\C(f \otimes g, h) (x)} = \C(f, [g, h]) (\phi_x).
            \]

            We start by calculating \( \phi_{\C(f \otimes g, h) (x)} \).

            First, note that
            \[
                \C(f \otimes g, h)(x) = h \circ x \circ (f \otimes g).
            \]
            Let \( i, j \in \Zb \), then
            \[
                (h \circ x \circ (f \otimes g))_{i + j} = h_{i + j} \circ x_{i + j} \circ (\coprod_{p + q = i + j} f_p \otimes g_q).
            \]
            Pre-composing with \( \iota_{i, j} \) yields
            \[
                h_{i + j} \circ x_{i + j} \circ (\coprod_{p + q = i + j} f_p \otimes g_q) \circ \iota_{i, j} = h_{i + j} \circ x_{i + j} \circ (f_i \otimes g_j).
            \]
            Taking the adjoint, yields the following morphism
            \[
                a' \mapsto h_{i + j} \circ x_{i + j} \circ (f_i \otimes g_j)(a' \otimes ?).
            \]
            This morphism is equal to
            \[
                a' \mapsto h_{i + j} \circ x_{i + j} \circ (f_i (a') \otimes g_j(?)).
            \]
            Which again is equal to the morphism
            \[
                a' \mapsto h_{i + j} \circ (g_j)^* \circ \phi_{x, i, j} \circ f_i(a').
            \]
            Which we will write as follows,
            \[
                \phi_{\C(f \otimes g, h)(x), i, j} = h_{i + j} \circ (g_j)^* \circ \phi_{x, i, j} \circ f_i.
            \]
            Taking the product over all the \( j \), yields
            \begin{align*}
                \phi_{\C(f \otimes g, h) (x), i} &= \prod_{j \in \Zb} h_{i + j} \circ (g_j)^* \circ \phi_{x, i, j} \circ f_i \\
                &= \tuple*{\prod_{j \in \Zb} h_{i + j} \circ (g_j)^* \circ \phi_{x, i, j}} \circ f_i \\
                &= \tuple*{\prod_{j \in \Zb} (h_{i + j})_* \circ (g_j)^*} \circ \tuple*{\prod_{j \in \Zb} \phi_{x, i, j}} \circ f_i \\
                &= [g, h]_i \circ \phi_{x, i} \circ f_i \\
                &= \C(f, [g, h])_i (\phi_x).
            \end{align*}
            Finally, combining all the degrees to create a chain morphism,
            \[
                \phi_{\C(f \otimes g, h) (x)} = \tuple*{\phi_{\C(f \otimes g, h) (x), i}}_{i \in \Zb} = \tuple*{\C(f, [g, h])_i (\phi_x)}_{i \in \Zb} = \C(f, [g, h]) (\phi_x). \qedhere
            \]
        }
    \end{enumerate}
\end{proof}

Together, the above definitions and remarks gives the structure of a symmetric monoidal category.

\begin{lemma}[\( \C \) is symmetric monoidal closed]
    \label{lem:c_sym_mon_closed}
    Let \( R \) be a commutative ring with identity, and let \( \otimes \) denote the tensor product on \( \C \). Furthermore, let \( I \) be the chain complex in \( \C \) consisting solely of \( 0 \)-objects except for the \( R \)-module \( R \) in index \( 0 \).

    Then \( \tuple*{\C, \otimes, I} \) is a symmetric monoidal closed category.
\end{lemma}

% TODO: Dobbeltsjekk detaljar.
We will omit proving the above lemma in full detail, the following is just a sketch of the proof.
\begin{proof}[Sketch of proof]
    First, we prove it has every property of a monoidal category, as stated in \cite[Definition 6.1.1]{Borceux_1994}.

    We start with proving the data:
    \begin{enumerate}
        \item {
            \( \C \) is a category.
        }
        \item {
            By \autoref{rem:c_tensor_bifunctor}, the tensor product is a bifunctor.
        }
        \item {
            \( I \) is the unit.
        }
        \item {
            Consider the following diagram,
            \begin{center}
                \begin{tikzpicture}[scale=0.8, every node/.style={scale=0.8}]
                    \diagram{m}{0.6cm}{0.2cm} {
                        \cdots \& \coprod\limits_{i + j = -1} \tuple*{\coprod\limits_{p + q = i} A_p \otimes B_q} \otimes C_j \& \coprod\limits_{i + j = 0} \tuple*{\coprod\limits_{p + q = i} A_p \otimes B_q} \otimes C_j \& \coprod\limits_{i + j = 1} \tuple*{\coprod\limits_{p + q = i} A_p \otimes B_q} \otimes C_j \& \cdots \\
                        \& \coprod\limits_{i + j + k = -1} (A_i \otimes B_j) \otimes C_k \& \coprod\limits_{i + j + k = 0} (A_i \otimes B_j) \otimes C_k \& \coprod\limits_{i + j + k = 1} (A_i \otimes B_j) \otimes C_k \\
                        \& \coprod\limits_{i + j + k = -1} A_i \otimes (B_j \otimes C_k) \& \coprod\limits_{i + j + k = 0} A_i \otimes (B_j \otimes C_k) \& \coprod\limits_{i + j + k = 1} A_i \otimes (B_j \otimes C_k) \\
                        \cdots \& \coprod\limits_{i + j = -1} A_i \otimes \tuple*{\coprod\limits_{p + q = j} B_p \otimes C_q} \& \coprod\limits_{i + j = -1} A_i \otimes \tuple*{\coprod\limits_{p + q = j} B_p \otimes C_q} \& \coprod\limits_{i + j = -1} A_i \otimes \tuple*{\coprod\limits_{p + q = j} B_p \otimes C_q} \& \cdots \\
                    };

                    \draw[math]
                        (m-1-1) edge (m-1-2)
                        (m-1-2) edge (m-1-3)
                            edge (m-2-2)
                        (m-1-3) edge (m-1-4)
                            edge (m-2-3)
                        (m-1-4) edge (m-1-5)
                            edge (m-2-4)

                        (m-2-2) edge (m-3-2)
                        (m-2-3) edge (m-3-3)
                        (m-2-4) edge (m-3-4)

                        (m-3-2) edge (m-4-2)
                        (m-3-3) edge (m-4-3)
                        (m-3-4) edge (m-4-4)

                        (m-4-1) edge (m-4-2)
                        (m-4-2) edge (m-4-3)
                        (m-4-3) edge (m-4-4)
                        (m-4-4) edge (m-4-5);
                \end{tikzpicture}
            \end{center}

            where the vertical morphisms are the expected natural morphisms, and the horizontal morphisms are the differentials.

            By uniqueness of a three-tensor version of \autoref{lem:map_out_of_tensor_unique}, it can be shown that the above diagram commutes. Let the composition of the vertical morphisms be the associativity chain morphism.
        }
        \item[5 \& 6.] {
            Let \( A \in \C \). Then \( I \otimes A \) is the chain complex,
            \begin{center}
                \begin{tikzpicture}
                    \diagram{m}{1cm}{1.5cm} {
                        \cdots \& R \otimes A_{-1} \& R \otimes A_0 \& R \otimes A_1 \& \cdots \\
                    };

                    \draw[math]
                        (m-1-1) edge (m-1-2)
                        (m-1-2) edge node {\Id_R \otimes d_{A, -1}} (m-1-3)
                        (m-1-3) edge node {\Id_R \otimes d_{A, 0}} (m-1-4)
                        (m-1-4) edge (m-1-5);
                \end{tikzpicture}
            \end{center}
            and the chain complex \( A \otimes I \) is as expected.

            Let \( l_{A_i}: R \otimes A_i \to A_i \) be the left unit isomorphism in \( \Mod(R) \), and let \( r_{A_i}: A_i \otimes R \to A_i \) be the right unit isomorphism in \( \Mod(R) \).

            In lieu of the differential of the above chain complex, it is obvious that the tuple \( \tuple*{l_{A_i}}_{i \in \Zb} \) constitute a chain morphism from \( I \otimes A \) to \( A \). The same argument can be made for \( \tuple*{r_{A_i}}_{i \in \Zb} \) and \( A \otimes I \).
        }
    \end{enumerate}
    We now prove the data:
    \begin{enumerate}
        \item {
            The associativity isomorphism is natural, because it is the composition of natural chain isomorphisms. 
        }
        \item[2 \& 3.] {
            The left and right identity in \( \C \) are natural because the left and right identity in \( \Mod(R) \) are natural.
        }
        \item[4.] {
            By uniqueness of a four-tensor version of \autoref{lem:map_out_of_tensor_unique}, it follows that \cite[Diagram 6.1]{Borceux_1994} commutes.
        }
        \item[5.] {
            By uniqueness of a three-tensor version of \autoref{lem:map_out_of_tensor_unique}, \cite[Diagram 6.2]{Borceux_1994} commutes.
        }
    \end{enumerate}

    Second, we prove the category is symmetric, as stated in \cite[Definition 6.1.2]{Borceux_1994}.

    Let \( s \) be as in \autoref{rem:symmetry_tensor_product_of_chain_complex}.

    \begin{enumerate}
        \item {
            Let \( f \times g \in \C \times \C (A \times B, A' \times B') \).

            We want to show that the following diagram,
            \begin{center}
                \begin{tikzpicture}
                    \diagram{m}{1cm}{1cm} {
                        A \otimes B \& B \otimes A \\
                        A' \otimes B' \& B' \otimes A', \\
                    };
                    
                    \draw[math]
                        (m-1-1) edge node {s} (m-1-2)
                            edge node {f \otimes g} (m-2-1)
                        (m-1-2) edge node {g \otimes f} (m-2-2)

                        (m-2-1) edge node {s} (m-2-2);
                \end{tikzpicture}
            \end{center}
            commutes.

            Let \( a \in A_i \) and \( b \in B_j \), and consider the following equation,
            \begin{align*}
                (g \otimes f)_{i + j} \circ s_{i + j} (a \otimes b) &= (g \otimes f)_{i + j} ((-1)^{ij}b \otimes a) \\
                &= (-1)^{ij} g_j(b) \otimes f_i(a) \\
                &= s_{i + j}(f_i(a) \otimes g_j(b)) \\
                &= s_{i + j} \circ (f \otimes g)_{i + j} (a \otimes b).
            \end{align*}
            By the above equation, along with the uniqueness of \autoref{lem:map_out_of_tensor_unique}, the above diagram commutes.
        }
        \item {
            Since both paths of \cite[Diagram 6.3]{Borceux_1994} can be represented with the same \( g_{i, j, k} \) in a three-tensor version of \autoref{lem:map_out_of_tensor_unique},
            \[
                g_{i, j, k}: (a, b, c) \mapsto (-1)^{i(j + k)}b \otimes (c \otimes a),
            \]
            then the diagram commutes by uniqueness.
        }
        \item[3 \& 4.] {
            Both \cite[Diagram 6.4]{Borceux_1994}, and \cite[Diagram 6.5]{Borceux_1994} commutes by uniqueness of \autoref{lem:map_out_of_tensor_unique}.
        }
    \end{enumerate}

    Third, we prove the category is closed, i.e., the tensor product is adjoint to some functor, which is true by the proof associated to \autoref{rem:tensor_prod_internal_hom_adjoint}.
\end{proof}

Finally, we can define what a DG-category is.

\begin{definition}[DG-category]
    \label{def:dg_cat}
    Let \( R \) be a commutative ring with identity.

    Then \( \Cc \) is a \emph{DG-category over \( R \)} if it is a category enriched over \( \C \).
\end{definition}

This definition also appears in \cite[p.\ 29]{Jasso-Muro_2023}, except they define it for a field and not a commutative ring with identity as is done in this thesis.
