{\color{red}
IKKJE RETT DETTE, IKKJE FERDIG

Let \( \Ac \) be the full DG-subcategory of \( \C_{\dg} \), where the objects are exact chain complexes consisting only of injective/projective modules.

Then \( \Mc \) is equivalent to \( H^0(\Ac) \) through the functor \( \Phi \), which is the following functor:

For any \( A \in \Mc \), consider the projective and injective resolutions:
\[
    \cdots \to 0 \to A \to I_1 \to I_2 \to \cdots
\]
\[
    \cdots \to P_2 \to P_1 \to A \to 0 \to \cdots
\]

Gluing them together yields the following exact chain complex, where \( I_1 \) is in degree \( 0 \).
\[
    E: \cdots \to P_2 \to P_1 \to I_1 \to I_2 \to \cdots
\]

Then define \( \Phi(A) := E \).

WIP
}
% MS-Question https://math.stackexchange.com/questions/177495/massey-products-in-the-adams-spectral-sequence#1088096

% MS-Notes som er nemd i stackexchange seier massey er vanskelegare enn Toda.

% TODO: Bruka indeterminancy til å hjelpa med berekningane?

% Korleis får ein ein small katyegori med Krause def?

% Massey er heilt forferdeleg å reikna ut. Trur eg manglar grunnleggande info.


% Remark på forteikn.