A goal of this thesis is to let the theorems and definitions contain all the neccesary assumptions without referencing this subsection. Despite that, there are some conventions that are used throughout this thesis that could be nice to mention at the start.

\begin{notation}
    \( R \) is \emph{always} a commutative ring with identity in this thesis.
\end{notation}

\begin{notation}
    In this thesis, every chain complex will have \emph{ascending} order. This is sometimes called a \emph{cochain complex} in other litterature.
\end{notation}

\begin{notation}
    In this thesis, \( \Mod(R) \) will refer to \emph{infinitely generated} \( R \)-modules.

    Similarly, \( \mod(R) \) will refer to \emph{finitely generated} \( R \)-modules.
\end{notation}

\begin{notation}
    Hooked arrows represent monomorphisms, and double-headed arrows represent epimorphisms.
\end{notation}