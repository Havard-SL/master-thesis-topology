This thesis will assume that the reader is already familiar with basic homological algebra as well as some module theory.

A goal of this thesis is to let the theorems and definitions contain all the neccesary assumptions without needlessly confusing a returning reader. Despite that, there are some conventions that are used throughout this thesis that could be nice to mention at the very start.

\begin{notation}
    \( R \) will \emph{always} denote a commutative ring with identity.
\end{notation}

\begin{notation}
    Every chain complex will have \emph{ascending} order. This is sometimes called a \emph{cochain complex} in other litterature.
\end{notation}

\begin{notation}
    \( \Mod(R) \) will refer to \emph{infinitely generated} \( R \)-modules.

    Similarly, \( \mod(R) \) will refer to \emph{finitely generated} \( R \)-modules.
\end{notation}

\begin{notation}
    Hooked arrows represent monomorphisms, and double-headed arrows represent epimorphisms.
\end{notation}

% TODO: Clean up superflous notation.