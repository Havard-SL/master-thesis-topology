This thesis will assume that the reader is already familiar with basic homological algebra as well as some module theory.

Some assumptions in results will be implied by definitions and not be explicitly stated in theorems. Despite that, there are some conventions that are used throughout this thesis.

First is the properties of the ring.
\begin{notation}
    \( R \) will \emph{always} denote a commutative ring with identity.
\end{notation}

Second, the following is the chain complex convention.
\begin{notation}
    \label{not:chain_complex}
    Every chain complex will have \emph{ascending} order. This is sometimes called a \emph{cochain complex} in other litterature.

    Differentials are indexed according to the order of their domain. I.e. for a chain complex \( A = \tuple{A_i}_{i \in \Zb} \), the \( n \)-th differential has the same index as its domain, \( d_n: A_n \to A_{n + 1} \).
\end{notation}

Third, the module notation is as follows.
\begin{notation}
    \( \Mod(R) \) will refer to \( R \)-modules, and \( \mod(R) \) will refer to \emph{finitely generated} \( R \)-modules.
\end{notation}

Fourth,the following arrow conventions will be used.
\begin{notation}
    Tailed arrows \( (\rightarrowtail) \) represent monomorphisms, and double-headed arrows \( (\twoheadrightarrow) \) represent epimorphisms.
\end{notation}

% TODO: Clean up superflous notation.