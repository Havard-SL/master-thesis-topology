Triangulated categories were discovered independently by Dieter Puppe and Jean-Louis Verdier in 1962 and 1963, respectively. Verdier introduced his axioms in his doctoral thesis, which was supervised by Alexander Grothendieck. This lead to the attribution for the discovery of triangulated categories sometimes being split three ways between Puppe, Verdier, and Grothendieck. Puppe discovered triangulated categories while looking at the stable homotopy category, while Verdier and Grothendieck were motivated from a more algebraic viewpoint via the derived category of an abelian category.

As their discovery would imply, triangulated categories are useful in both algebra and topology. Typical algebraic examples of triangulated categories are: the derived category of an abelian category, the stable module category, and the homotopy category of chain complexes. On the other hand, typical topological examples of triangulated categories are: the stable homotopy category, and the Spanier--Whitehead category. In \autoref{section:tri_cats} we will define and give three examples of triangulated categories. We will touch a little upon the Spanier--Whitehead category (\autoref{subsubsection:spanier_whitehead_cat}) and the chain homotopy category (\autoref{subsubsection:chain_homotopy_cat}), but will mainly focus on the stable module category (\autoref{subsubsection:stable_module_cat}) for use in examples throughout the thesis. For this reason, we prove that the stable module category is a triangulated category (\autoref{example:stable_module_category_triangulated}).

Toda brackets were initially introduced by Hiroshi Toda in 1962 to calculate homotopy groups of spheres. The definition of Toda brackets in any triangulated category is based on the work of Shipley in 2002 \cite[Definition A.2]{Shipley_2002}. She bases her definition on ideas from Cohen in 1968 \cite[p.\ 308]{Cohen_1968}. Shipley's definition was further refined by Sagave in 2008 \cite[Remark 4.5]{Sagave_2008}. The definitions in this thesis are a more elegant formulation of Sagave's definitions by Christensen and Frankland in 2017 \cite[Definition 3.1]{Christensen-Frankland_2017}. The definitions and some examples of calculations will be discussed in \autoref{section:toda_brackets}.

Toda brackets are useful, but can be hard to compute. Therefore, any additional tools to compute Toda brackets could help a lot in certain cases. One such tool is the ``indeterminacy,'' and another tool is ``Massey products.''

Massey products were introduced by William Schumacher Massey in 1958, and is a useful tool in many calculations in cohomology. Massey products can be defined on the cohomology category of a DG-category (\autoref{def:massey_product_dg_cat}) and further generalized to be defined in any algebraic triangulated category (\autoref{def:massey_product_alg_tri_cat}).

Our main result, \autoref{theorem:massey_equals_toda}, proves that Toda brackets and Massey products are equal, i.e.,
\[
    \toda{f_3, f_2, f_1} = \massey{f_3, f_2, f_1},
\]
in algebraic triangulated categories.

In order to prove this, we will define and prove a lot of the necessary prerequisites. We will sketch a proof of why the category of chain complexes over a commutative ring with identity is symmetric monoidal closed. Then we will define DG-categories, the category of DG-modules, algebraic triangulated categories, and many necessary prerequisites for the proof. We will not be defining everything in the most general form, but rather narrow down the definitions to the cases we are interested in.

We will also calculate some examples of Toda brackets and Massey products in the stable module category to test if they coincide as expected.

Finally, we will end the thesis with a potential application of this equality between Massey products and Toda brackets.

By NTNU regulations, it is mandatory to include the following statement.
\subsection{Sustainability statement}
While pure mathematics seemingly has no real world impact, it has been repeatedly proven that pure mathematics appear in other fields of study in the future and provides tools to solve problems that can appear in the real world. Topology in particular, has a deep history connected to optimization problems, and in newer times have been used in data analysis and other fields. This thesis could bring future benefit to the physical sciences as well as data analysis which are key fields for future innovation. Given the potential widespread applicability of pure topology research, this thesis could contribute greatly to the UN's sustainability goal number 3 and 9. First, goal number 3 for good health and well-being, as topology is already being used for data analysis in medical research. Second, goal number 9 for industry, innovation, and infrastructure, since topology can be used to further innovation.

Of course, given the potential wide applicability of topological techniques, it could contribute to other goals. However, the two aforementioned sustainability goals, goal 3 and 9, are the ones where topology is currently being used to some degree.

\subsection{Notation}
This thesis will assume that the reader is already familiar with basic homological algebra as well as some module theory.

Some assumptions in results will be implied by definitions and not be explicitly stated in theorems. However, there are some conventions that are used throughout this thesis.

First, the following is the chain complex convention.
\begin{notation}
    \label{not:chain_complex}
    Every chain complex will have \emph{ascending} order. This is sometimes called a \emph{cochain complex} in other literature.

    Differentials are indexed according to the order of their domain, i.e., for a chain complex \( A = \tuple*{A_i}_{i \in \Zb} \), the \( n \)th differential has the same index as its domain, \( d_n: A_n \to A_{n + 1} \).
    
    We try to keep to the same convention of indexing with respect to the domain when considering morphisms between graded objects.
\end{notation}

Second, the module notation is as follows.
\begin{notation}
    \( \Mod(R) \) will refer to \( R \)-modules, and \( \mod(R) \) will refer to \emph{finitely generated} \( R \)-modules.
\end{notation}

Third, the following arrow conventions will be used.
\begin{notation}
    Tailed arrows \( (\rightarrowtail) \) represent monomorphisms, and double-headed arrows \( (\twoheadrightarrow) \) represent epimorphisms.
\end{notation}

Fourth, certain canonical isomorphisms will be omitted in the notation.
\begin{notation}
    \label{not:suppress_canonical_isomorphisms}
    We will mention when there is a canonical isomorphism, but will suppress them in calculations for clarity.

    For example, given a category \( \Cc \), \( \Cc(A, B) \) is \emph{canonically isomorphic} to \( \Cc^{\op}(B, A) \), since by definition \( \Cc^{\op}(B, A) = \Cc(A, B) \).

    However, in calculations we will write, for example, \( f \in \Cc(A, B) \iff f \in \Cc^{\op}(B, A) \).
\end{notation}

Fifth, we differ between ``morphisms'' and ``maps.''
\begin{notation}
    A ``morphism'' is a morphism in some category, while a ``map'' is a function on the underlying sets.
\end{notation}