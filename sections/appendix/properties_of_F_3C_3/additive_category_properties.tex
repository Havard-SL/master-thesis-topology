\begin{notation}
    Let \( \Ac \) be an additive category, and let \( A, B \in \Ac \).

    Then denote the projection from \( A \oplus B \) to \( A \) by \( p_{A \oplus B}^A \), and denote the inclusion of \( A \) into \( A \oplus B \) by \( i_A^{A \oplus B} \).

    Furthermore, for any \( n \in \Nb \), and any \( i \in \set{1, 2, \dots, n} \), denote the projection from \( A^n \) to the \( i \)-th summand of \( A^n \) by \( p_{A^n}^{A_i} \). Similarly, denote the inclusion of the \( i \)-th summand of \( A^n \) into \( A^n \) by \( i_{A_i}^{A^n} \).
\end{notation}

% TODO: Look into if this is true. Category thery definitions.
\begin{lemma} \label{lem:hom_natural_iso}
    Let \( \Ac \) be an additive category.
    
    Then 
    \begin{enumerate}
        \item {
            The functor \( \Ac(-_1 \oplus -_2, -_3): \Ac^{op} \times \Ac^{op} \times \Ac \to \Ab \) is naturally isomorphic to the functor \( \Ac(-_1, -_3) \oplus \Ac(-_2, -_3) \). With the isomorphism

            \begin{center}
                \begin{tikzpicture}
                    \diagram{m}{1cm}{2cm} {
                        \Ac(-_1 \oplus -_2, -_3) \& \Ac(-_1, -_3) \oplus \Ac(-_2, -_3) \\
                    };

                    \draw[math]
                        (m-1-1) edge[curve={height=-25pt}] node {\begin{psmallmatrix} (i_{-_1}^{-_1 \oplus -_2})^* \\ (i_{-_2}^{-_1 \oplus -_2})^* \end{psmallmatrix}} (m-1-2)
                        (m-1-2) edge[curve={height=-25pt}] node {\begin{psmallmatrix} (p_{-_1 \oplus -_2}^{-_1})^* & (p_{-_1 \oplus -_2}^{-_2})^* \end{psmallmatrix}} (m-1-1);
                \end{tikzpicture}
            \end{center}
        }
        \item {
            The functor \( \Ac(-_1, -_2 \oplus -_3): \Ac^{op} \times \Ac \times \Ac \to \Ab \) is naturally isomorphic to the functor \( \Ac(-_1, -_2) \oplus \Ac(-_1, -_3) \). With the isomorphism

            \begin{center}
                \begin{tikzpicture}
                    \diagram{m}{1cm}{2cm} {
                        \Ac(-_1, -_2 \oplus -_3) \& \Ac(-_1, -_2) \oplus \Ac(-_1, -_3) \\
                    };

                    \draw[math]
                        (m-1-1) edge[curve={height=-25pt}] node {\begin{psmallmatrix} (p_{-_2 \oplus -_3}^{-_2})_* \\ (p_{-_2 \oplus -_3}^{-_3})_* \end{psmallmatrix}} (m-1-2)
                        (m-1-2) edge[curve={height=-25pt}] node {\begin{psmallmatrix} (i_{-_2}^{-_2 \oplus -_3})_* & (i_{-_3}^{-_2 \oplus -_3})_* \end{psmallmatrix}} (m-1-1);
                \end{tikzpicture}
            \end{center}
        }
    \end{enumerate}
\end{lemma}
\begin{proof}
    TODO: 
    
    https://ncatlab.org/nlab/show/additive+functor

    https://ncatlab.org/nlab/show/hom-functor+preserves+limits
\end{proof}

% TODO: Remark abuse of post-composition notation?

\begin{lemma} \label{lem:hom_split_over_direct_sum_n_naturally}
    Let \( \Ac \) be an additive category, and let \( n \in \Nb \).
    
    Then 
    \begin{enumerate}
        \item {
            The functor
            \[
                \Ac\tuple{\bigoplus_{i = 1}^n -_i, -_{n + 1}}: \tuple{\Ac^{op}}^n \times \Ac \to \Ab
            \]
            is naturally isomorphic to the functor
            \[
                \bigoplus_{i = 1}^n \Ac\tuple{-_i, -_{n + 1}}
            \]
            with the isomorphism
            \begin{center}
                \begin{tikzpicture}
                    \diagram{m}{1cm}{2cm} {
                        \Ac\tuple{\bigoplus_{i = 1}^n -_i, -_{n + 1}} \&
                        \bigoplus_{i = 1}^n \Ac\tuple{-_i, -_{n + 1}} \\
                    };

                    \draw[math]
                        (m-1-1) edge[curve={height=-25pt}] node {
                            \begin{psmallmatrix}
                                (i_{-_1}^{\bigoplus_{i = 1}^n -_i})^* \\
                                (i_{-_2}^{\bigoplus_{i = 1}^n -_i})^* \\
                                \vdots \\
                                (i_{-_n}^{\bigoplus_{i = 1}^n -_i})^*
                            \end{psmallmatrix}
                            } (m-1-2)
                        (m-1-2) edge[curve={height=-25pt}] node {
                            \begin{psmallmatrix}
                                (p_{\bigoplus_{i = 1}^n -_i}^{-_1})^* &
                                (p_{\bigoplus_{i = 1}^n -_i}^{-_2})^* &
                                \dots &
                                (p_{\bigoplus_{i = 1}^n -_i}^{-_n})^*
                            \end{psmallmatrix}
                            } (m-1-1);
                \end{tikzpicture}
            \end{center}
        }
        \item {
            The functor
            \[
                \Ac\tuple{-_1, \bigoplus_{i = 2}^{n + 1} -_i}: \Ac^{op} \times \tuple{\Ac}^n \to \Ab
            \]
            is naturally isomorphic to the functor
            \[
                \bigoplus_{i = 2}^{n + 1}\Ac\tuple{-_1, -_i}
            \]
            with the isomorphism
            \begin{center}
                \begin{tikzpicture}
                    \diagram{m}{1cm}{2cm} {
                        \Ac\tuple{-_1, \bigoplus_{i = 2}^{n + 1} -_i} \&
                        \bigoplus_{i = 2}^{n + 1}\Ac\tuple{-_1, -_i} \\
                    };

                    \draw[math]
                        (m-1-1) edge[curve={height=-25pt}] node {
                            \begin{psmallmatrix}
                                (p_{\bigoplus_{i = 2}^{n + 1} -_i}^{-_2})_* \\
                                (p_{\bigoplus_{i = 2}^{n + 1} -_i}^{-_3})_* \\
                                \vdots \\
                                (p_{\bigoplus_{i = 2}^{n + 1} -_i}^{-_{n + 1}})_*
                            \end{psmallmatrix}
                            } (m-1-2)
                        (m-1-2) edge[curve={height=-25pt}] node {
                            \begin{psmallmatrix}
                                (i_{-_2}^{\bigoplus_{i = 2}^{n + 1} -_i})_* &
                                (i_{-_3}^{\bigoplus_{i = 2}^{n + 1} -_i})_* &
                                \dots &
                                (i_{-_{n + 1}}^{\bigoplus_{i = 2}^{n + 1} -_i})_*
                            \end{psmallmatrix}
                            } (m-1-1);
                \end{tikzpicture}
            \end{center}
        }
    \end{enumerate}
\end{lemma}
\begin{proof}
    TODO: Itarately apply \autoref{lem:hom_natural_iso}.
\end{proof}

\begin{lemma} \label{thm:hom_direct_sum_map_nice}
    Let \( \Ac \) be an additive category, and let \( A, B, C \in \Ac \). Let \( f: B \to C \). Let \( n \in \Nb \).

    Then:
    \begin{enumerate}
        \item {
            The following diagram commutes
            \begin{center}
                \begin{tikzpicture}
                    \diagram{m}{3cm}{2cm} {
                        \Ac(A, B^n) \& \Ac(A, C^n) \\
                        \Ac(A, B)^n \& \Ac(A, C)^n \\
                    };
        
                    \draw[math]
                        (m-1-1) edge node {(f^n)_*} (m-1-2)
                            edge node { \begin{psmallmatrix} (p_{B^n}^{B_1})_* \\ (p_{B^n}^{B_2})_* \\ \vdots \\ (p_{B^n}^{B_n})_* \end{psmallmatrix} } (m-2-1)
                        (m-1-2) edge node { \begin{psmallmatrix} (p_{C^n}^{C_1})_* \\ (p_{C^n}^{C_2})_* \\ \vdots \\ (p_{C^n}^{C_n})_* \end{psmallmatrix} } (m-2-2)
                        
                        (m-2-1) edge node {(f_*)^n} (m-2-2);
                \end{tikzpicture}
            \end{center}
        }
        \item {
            The following diagram commutes
            \begin{center}
                \begin{tikzpicture}
                    \diagram{m}{3cm}{2cm} {
                        \Ac(A^n, B) \& \Ac(A^n, C) \\
                        \Ac(A, B)^n \& \Ac(A, C)^n \\
                    };
        
                    \draw[math]
                        (m-1-1) edge node {f_*} (m-1-2)
                            edge node { \begin{psmallmatrix} (i_{A_1}^{A^n})^* \\ (i_{A_2}^{A^n})^* \\ \vdots \\ (i_{A_n}^{A^n})^* \end{psmallmatrix} } (m-2-1)
                        (m-1-2) edge node { \begin{psmallmatrix} (i_{A_1}^{A^n})^* \\ (i_{A_2}^{A^n})^* \\ \vdots \\ (i_{A_n}^{A^n})^* \end{psmallmatrix} } (m-2-2)
                        
                        (m-2-1) edge node {(f_*)^n} (m-2-2);
                \end{tikzpicture}
            \end{center}
        }
    \end{enumerate}
\end{lemma}
\begin{proof}
    TODO: Notes, example projective class.
\end{proof}