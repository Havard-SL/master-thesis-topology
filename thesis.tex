\documentclass[a4paper, 12pt]{article}

% Oversett visse ord til norsk.
% \usepackage[nynorsk]{babel}

% For å kunna skriva æøå i tekstar MERK: Blir automatisk ubrukeleg med lualatex og fontspec
%\usepackage[utf8]{inputenc}

\usepackage[T1]{fontenc}

% Fiksa margin
\usepackage[margin=2cm]{geometry}

% Fiksar datoformatet på tiitelen
\usepackage[ddmmyyyy]{datetime} 


\usepackage{amssymb}

% For visse mattesymbol, typ \mathbb
\usepackage{amsmath}

% Bilete
\usepackage{graphicx}

% For kodesnuttar og resultat
%\usepackage{minted}

% Kan endra på korleis listar ser ut
\usepackage{enumitem}

% For autoref
\usepackage[hidelinks,colorlinks=true]{hyperref} 

% For fargar på ting ein referer til i autoref
\hypersetup{allcolors=[rgb]{0,0.31,0.62}}

% For teorem, definisjon, bevis enviornments.
\usepackage{amsthm} 

\usepackage{thmtools} 

% For svgar
%\usepackage{svg}

% Set svg mappo
%\svgpath{svg/}

% For å laga / halda styr på flytopbjekt
% \usepackage{float}

% \usepackage[subsection]{placeins}

% Fjernar indents ved nye avsnitt, men gjer linjeavstanden kortare (Kanskje)
\usepackage{parskip} 

% Lualatex font greie
\usepackage{fontspec}

\usepackage{unicode-math}

% Set fontar som blir brukt
\setmathfont{Latin Modern Math} % Dette er standardfonten
\setmathfont[range=\setminus]{Asana Math}
% \setmainfont{Atkinson Hyperlegible}
% \setmainfont{GFS Neohellenic Math}
% \setmainfont{Fira Sans}
% \setmathfont{Fira Math}
% \setmathfont[range=\setminus]{Asana Math}

\usepackage{quiver}

%For kommutative diagram med tikz
% \usepackage{tikz-cd}

% Thaule tikzcd erstatning
%\usepackage{tikz}
%\usetikzlibrary{matrix}
%\newcommand{\diagram}[3]{\matrix (#1) [matrix of math nodes,row
%  sep={#2},column sep={#3},text height=1.5ex,text
%  depth=0.25ex]}

% Ny type lista med ganske perfekt spacing
\newlist{plist}{enumerate}{5}
\setlist[plist]{align=left, itemindent = 0cm, labelsep = 0cm, labelindent = 0cm}
\setlist[plist,1]{label=\arabic*, font=\bf\Large}
\setlist[plist,2]{label*=.\arabic*, labelwidth=1.25cm, leftmargin=1.25cm}
\setlist[plist,3]{label*=.\arabic*, labelwidth=1.5cm, leftmargin=1.5cm}

% Teoremstil
\theoremstyle{plain}
\newtheorem{theorem}{Theorem}[section]
\newtheorem{proposition}[theorem]{Proposition}
\newtheorem{corollary}[theorem]{Corollary}
\newtheorem{lemma}[theorem]{Lemma}

% Definisjonstil
\theoremstyle{definition}
\newtheorem{definition}[theorem]{Definition}
\newtheorem{example}[theorem]{Example}
\newtheorem{remark}[theorem]{Remark}

% Praktiske forkortningar
\newcommand{\Rb}{\mathbb{R}}
\newcommand{\Qb}{\mathbb{Q}}
\newcommand{\Zb}{\mathbb{Z}}
\newcommand{\Nb}{\mathbb{N}}
\newcommand{\Nc}{\mathcal{N}}
\newcommand{\Tc}{\mathcal{T}}

% Praktiske omformuleringar
\newcommand{\intersect}{\mathop{\cap}\limits}
\newcommand{\union}{\mathop{\cup}\limits}

% Praktiske kommandoar
% \newcommand{\gr}[1]{ \lvert #1 \rvert } % Geometrisk realisering
\newcommand{\set}[1]{\left\{#1\right\}} % mengd
\newcommand{\tuple}[1]{\left(#1\right)} % tuppel

% Nødvendige nye operatorar for bacheloroppgåva
% \DeclareMathOperator{\Cech}{Cech} % Cech-kompleks
% \DeclareMathOperator{\Sd}{Sd} % Barysentrisk Oppdeling
% \DeclareMathOperator{\bst}{bst} % Lukka barysentrisk stjerna
% \DeclareMathOperator{\Sk}{Sk} % Skjelett
\DeclareMathOperator{\Id}{Id} % Identitetsavbildinga
\DeclareMathOperator{\StMod}{StMod}
\DeclareMathOperator{\Obj}{Obj}
\DeclareMathOperator{\Hom}{Hom}
\DeclareMathOperator{\Mod}{Mod}
\DeclareMathOperator{\coker}{coker}

\title{Master thesis TODO}
\author{Håvard Skjetne Lilleheie}

\begin{document}

\maketitle

\section{Stable module category is triangulated}

% TODO: Significantly rewrite the entire section. Additivity can directly show most of the results given in this section. Also, many proofs are very similar.
\begin{definition}\label{def:stable_module_category}
    Let \( G \) be a group. Let \( R \) be a \( G \) algebra over the field \( K \), i.e. \( R = KG \) with the free module structure, and normal multiplication of group and field elements.

    Then the ``stable module category over \( R \)'', denoted \( \Tc := \StMod(R) \) is defined in the following way:
    \begin{enumerate}
        \item \( \Obj(\Tc) := \Obj(\Mod(R)) \).
        \item \( \Hom_{\Tc}(A, B) := \Hom_R(A, B)/\set{\text{maps that factor through a projective}} \)
    \end{enumerate}
\end{definition}

\begin{theorem}
    The definition in \autoref{def:stable_module_category} is well defined, and it is an additive category.
\end{theorem}
\begin{proof}
    First, need to check that the set of maps that factor through a projective is an \( R \) submodule of \( \Hom_R(A, B) \).

    Let, \( f \) and \( g \) be two maps that factor through the projectives \( P \) and \( Q \) respectively. Then we have the following diagrams:

    \[\begin{tikzcd}
        A & P & B
        \arrow["{f_1}", from=1-1, to=1-2]
        \arrow["{f_2}", from=1-2, to=1-3]
    \end{tikzcd}\]

    Where \( f_2 \circ f_1 = f \), and

    \[\begin{tikzcd}
        A & Q & B
        \arrow["{g_1}", from=1-1, to=1-2]
        \arrow["{g_2}", from=1-2, to=1-3]
    \end{tikzcd}\]

    Where \( g_2 \circ g_1 = g \).

    Can then construct the map

    \[\begin{tikzcd}
        A & {P \oplus Q} & B
        \arrow["{(f_1, g_1)^T}", from=1-1, to=1-2]
        \arrow["{(f_2, g_2)}", from=1-2, to=1-3]
    \end{tikzcd}\]

    Composing these two maps, one gets the map \( f_2 \circ f_1 + g_2 \circ g_1 = f + g \). This maps factors thorugh \( P \oplus Q \), which is projective since it's a direct sum of projective modules.

    Therefore, the set of homomorphisms that factor through a projective is closed under addition. And multiplying with a ring element still factors through the same projective, since every map is an \( R \) homomorphism. Therefore the set of maps that factor through a projective is an \( R \) submodule.

    Therefore \( \Hom_{\Tc}(A, B) \) is an abelian group, and the outstanding properties of an additive category is inherited from \( \Mod(R) \) as well. (TODO: Prove the unproved properties!)
\end{proof}

\begin{definition}
    Let \( A \in \Obj(\Tc) \).

    Let \( \Omega \) an endofunctor on \( \Tc \). Where \( \Omega(A) \) is given by choosing a projecive module \( P \) for every \( A \) with an endomorphism \( \pi_A \) from \( P \) to \( A \), and taking the kernel of that map. I.e \( \Omega(A) = \ker(\pi_A) \).
\end{definition}

\begin{remark}
    From the definition of \( \Omega \), \( \Omega(f) \) is constructed as follows:

    Looking at the following commutative diagram:

    % https://q.uiver.app/#q=WzAsNixbMCwwLCJcXE9tZWdhKEEpIl0sWzAsMSwiXFxPbWVnYShCKSJdLFsxLDAsIlBfQSJdLFsxLDEsIlBfQiJdLFsyLDAsIkEiXSxbMiwxLCJCIl0sWzAsMSwiXFxPbWVnYShmKSIsMl0sWzAsMiwiXFxpb3RhX0EiLDAseyJzdHlsZSI6eyJ0YWlsIjp7Im5hbWUiOiJob29rIiwic2lkZSI6InRvcCJ9fX1dLFsyLDQsIlxccGlfQSIsMCx7InN0eWxlIjp7ImhlYWQiOnsibmFtZSI6ImVwaSJ9fX1dLFsxLDMsIlxcaW90YV9CIiwwLHsic3R5bGUiOnsidGFpbCI6eyJuYW1lIjoiaG9vayIsInNpZGUiOiJ0b3AifX19XSxbMyw1LCJcXHBpX0IiLDAseyJzdHlsZSI6eyJoZWFkIjp7Im5hbWUiOiJlcGkifX19XSxbNCw1LCJmIl0sWzIsMywiaF9mIl1d
    \[\begin{tikzcd}
        {\Omega(A)} & {P_A} & A \\
        {\Omega(B)} & {P_B} & B
        \arrow["{\Omega(f)}"', from=1-1, to=2-1]
        \arrow["{\iota_A}", hook, from=1-1, to=1-2]
        \arrow["{\pi_A}", two heads, from=1-2, to=1-3]
        \arrow["{\iota_B}", hook, from=2-1, to=2-2]
        \arrow["{\pi_B}", two heads, from=2-2, to=2-3]
        \arrow["f", from=1-3, to=2-3]
        \arrow["{p_f}", from=1-2, to=2-2]
    \end{tikzcd}\]

    One has that for a map \( f: A \to B \), one gets the map \( p_f \) from the lifitng property of projective modules. Please note that this map is \emph{not neccesarily} unique.

    Furthermore, since \( \pi_B \circ p_f \circ \iota_A = f \circ \pi_A \circ \iota_A = f \circ 0 = 0 \), one has from the universal kernel property that there is a \emph{unique} map (given \( p_f \)) \( \Omega(f) \) from \( \Omega(A) \) to \( \Omega(B) \), which is the map defined by the functor.
\end{remark}

\begin{lemma}
    One has that \( \Omega \) is a functor.
\end{lemma}
\begin{proof}
    % Need to check the following:
    % 1) F(f o g) = F(f) o F(g)
    % 2) F(1) = 1

    First want to show that \( \Omega \) is functorial. Let \( A, B, C \in \Obj(\Tc) \). Then one can create the following diagram using the notation from before:

    % https://q.uiver.app/#q=WzAsOSxbMCwwLCJcXE9tZWdhKEEpIl0sWzAsMSwiXFxPbWVnYShCKSJdLFswLDIsIlxcT21lZ2EoQykiXSxbMSwwLCJQX0EiXSxbMSwxLCJQX0IiXSxbMSwyLCJQX0MiXSxbMiwwLCJBIl0sWzIsMSwiQiJdLFsyLDIsIkMiXSxbNiw3LCJnIl0sWzcsOCwiZiJdLFs1LDgsIiIsMCx7InN0eWxlIjp7ImhlYWQiOnsibmFtZSI6ImVwaSJ9fX1dLFs0LDcsIiIsMCx7InN0eWxlIjp7ImhlYWQiOnsibmFtZSI6ImVwaSJ9fX1dLFszLDYsIiIsMix7InN0eWxlIjp7ImhlYWQiOnsibmFtZSI6ImVwaSJ9fX1dLFswLDMsIiIsMix7InN0eWxlIjp7InRhaWwiOnsibmFtZSI6Imhvb2siLCJzaWRlIjoidG9wIn19fV0sWzEsNCwiIiwwLHsic3R5bGUiOnsidGFpbCI6eyJuYW1lIjoiaG9vayIsInNpZGUiOiJ0b3AifX19XSxbMiw1LCIiLDAseyJzdHlsZSI6eyJ0YWlsIjp7Im5hbWUiOiJob29rIiwic2lkZSI6InRvcCJ9fX1dLFszLDQsImhfZyJdLFs0LDUsImhfZiIsMl0sWzMsNSwiaF97ZiBcXGNpcmMgZ30iLDEseyJsYWJlbF9wb3NpdGlvbiI6NzAsImN1cnZlIjotMywiY29sb3VyIjpbMCw2MCw2MF19LFswLDYwLDYwLDFdXSxbMCwxLCJcXE9tZWdhKGcpIiwyXSxbMSwyLCJcXE9tZWdhKGYpIiwyXSxbMCwyLCJcXE9tZWdhKGYgXFxjaXJjIGcpIiwxLHsibGFiZWxfcG9zaXRpb24iOjIwLCJjdXJ2ZSI6LTMsImNvbG91ciI6WzAsNjAsNjBdfSxbMCw2MCw2MCwxXV0sWzYsOCwiZiBcXGNpcmMgZyIsMSx7ImN1cnZlIjotMywiY29sb3VyIjpbMCw2MCw2MF19LFswLDYwLDYwLDFdXV0=
    \[\begin{tikzcd}
        {\Omega(A)} & {P_A} & A \\
        {\Omega(B)} & {P_B} & B \\
        {\Omega(C)} & {P_C} & C
        \arrow["g", from=1-3, to=2-3]
        \arrow["f", from=2-3, to=3-3]
        \arrow[two heads, from=3-2, to=3-3]
        \arrow[two heads, from=2-2, to=2-3]
        \arrow[two heads, from=1-2, to=1-3]
        \arrow[hook, from=1-1, to=1-2]
        \arrow[hook, from=2-1, to=2-2]
        \arrow[hook, from=3-1, to=3-2]
        \arrow["{p_g}", from=1-2, to=2-2]
        \arrow["{p_f}"', from=2-2, to=3-2]
        \arrow["{p_{f \circ g}}"{description, pos=0.7}, color={rgb,255:red,214;green,92;blue,92}, curve={height=-18pt}, from=1-2, to=3-2]
        \arrow["{\Omega(g)}"', from=1-1, to=2-1]
        \arrow["{\Omega(f)}"', from=2-1, to=3-1]
        \arrow["{\Omega(f \circ g)}"{description, pos=0.2}, color={rgb,255:red,214;green,92;blue,92}, curve={height=-18pt}, from=1-1, to=3-1]
        \arrow["{f \circ g}"{description}, color={rgb,255:red,214;green,92;blue,92}, curve={height=-18pt}, from=1-3, to=3-3]
    \end{tikzcd}\]

    Then, one gets the following commuting diagram:

    % https://q.uiver.app/#q=WzAsNixbMCwwLCJcXE9tZWdhKEEpIl0sWzAsMSwiXFxPbWVnYShDKSJdLFsxLDAsIlBfQSJdLFsxLDEsIlBfQyJdLFszLDAsIkEiXSxbMywxLCJDIl0sWzMsNSwiIiwwLHsic3R5bGUiOnsiaGVhZCI6eyJuYW1lIjoiZXBpIn19fV0sWzIsNCwiIiwyLHsic3R5bGUiOnsiaGVhZCI6eyJuYW1lIjoiZXBpIn19fV0sWzAsMiwiIiwyLHsic3R5bGUiOnsidGFpbCI6eyJuYW1lIjoiaG9vayIsInNpZGUiOiJ0b3AifX19XSxbMSwzLCIiLDAseyJzdHlsZSI6eyJ0YWlsIjp7Im5hbWUiOiJob29rIiwic2lkZSI6InRvcCJ9fX1dLFsyLDMsImhfe2YgXFxjaXJjIGd9IC0gaF9mIFxcY2lyYyBoX2ciXSxbMCwxLCJcXE9tZWdhKGYgXFxjaXJjIGcpIC0gXFxPbWVnYShmKSBcXGNpcmMgXFxPbWVnYShnKSIsMl0sWzQsNSwiZiBcXGNpcmMgZyAtIGYgXFxjaXJjIGcgPSAwIl0sWzIsMSwiXFxwaGkiLDIseyJzdHlsZSI6eyJib2R5Ijp7Im5hbWUiOiJkYXNoZWQifX19XV0=
    \[\begin{tikzcd}
        {\Omega(A)} & {P_A} && A \\
        {\Omega(C)} & {P_C} && C
        \arrow[two heads, from=2-2, to=2-4]
        \arrow[two heads, from=1-2, to=1-4]
        \arrow[hook, from=1-1, to=1-2]
        \arrow[hook, from=2-1, to=2-2]
        \arrow["{p_{f \circ g} - p_f \circ p_g}", from=1-2, to=2-2]
        \arrow["{\Omega(f \circ g) - \Omega(f) \circ \Omega(g)}"', from=1-1, to=2-1]
        \arrow["{f \circ g - f \circ g = 0}", from=1-4, to=2-4]
        \arrow["\phi"', dashed, from=1-2, to=2-1]
    \end{tikzcd}\]

    But this implies that \( \pi_C \circ (p_{f \circ g} - p_f \circ p_g) = 0 \), which inducec a map by the kernel property \( \phi: P_A \to \Omega(C) \). Such that the lower triangle commutes. And since, \( \iota_C \) is a monomorphism, one gets that the upper triangle also commutes. And therefore \( \Omega(f \circ g) - \Omega(f) \circ \Omega(g) \) factors through a projective, and therefore \( \Omega(f \circ g) \sim \Omega(f) \circ \Omega(g) \).

    Second, need to show that \( \Omega(\Id_A) = \Id_{\Omega(A)} \) in \( \Tc \).

    By the same argument like above, one can see that every square and triangle in the following diagram also commutes:

    % https://q.uiver.app/#q=WzAsNixbMCwwLCJcXE9tZWdhKEEpIl0sWzAsMSwiXFxPbWVnYShBKSJdLFsxLDAsIlBfQSJdLFsxLDEsIlBfQSJdLFszLDAsIkEiXSxbMywxLCJBIl0sWzAsMiwiIiwwLHsic3R5bGUiOnsidGFpbCI6eyJuYW1lIjoiaG9vayIsInNpZGUiOiJ0b3AifX19XSxbMiw0LCIiLDAseyJzdHlsZSI6eyJoZWFkIjp7Im5hbWUiOiJlcGkifX19XSxbMyw1LCIiLDIseyJzdHlsZSI6eyJoZWFkIjp7Im5hbWUiOiJlcGkifX19XSxbMSwzLCIiLDIseyJzdHlsZSI6eyJ0YWlsIjp7Im5hbWUiOiJob29rIiwic2lkZSI6InRvcCJ9fX1dLFswLDEsIlxcT21lZ2EoSWRfQSkgLSBJZF97XFxPbWVnYShBKX0iLDJdLFsyLDMsImhfe0lkX0F9IC0gSWRfe1BfQX0iXSxbNCw1LCJJZF9BIC0gSWRfQSA9IDAiXSxbMiwxLCJcXHBoaSIsMix7InN0eWxlIjp7ImJvZHkiOnsibmFtZSI6ImRhc2hlZCJ9fX1dXQ==
    \[\begin{tikzcd}
        {\Omega(A)} & {P_A} && A \\
        {\Omega(A)} & {P_A} && A
        \arrow[hook, from=1-1, to=1-2]
        \arrow[two heads, from=1-2, to=1-4]
        \arrow[two heads, from=2-2, to=2-4]
        \arrow[hook, from=2-1, to=2-2]
        \arrow["{\Omega(Id_A) - Id_{\Omega(A)}}"', from=1-1, to=2-1]
        \arrow["{p_{Id_A} - Id_{P_A}}", from=1-2, to=2-2]
        \arrow["{Id_A - Id_A = 0}", from=1-4, to=2-4]
        \arrow["\phi"', dashed, from=1-2, to=2-1]
    \end{tikzcd}\]

    And therefore \( \Omega(\Id_A) \sim Id_{\Omega(A)} \).

\end{proof}

\begin{lemma}
    Let \( A, B \in \Tc \), then for \( f, g \in \Hom_{\Tc}(A, B) \), one has that \( \Omega(f + g) = \Omega(f) + \Omega(g) \) in \( \Tc \). I.e. \( \Omega \) is additive.
\end{lemma}
\begin{proof}
    Want to show that \( \Omega(f + g) \sim \Omega(f) + \Omega(g) \).
    
    One has that for any morphisms \( f, g \in \Hom_{\Tc}(A, B) \), from the definition of \( \Tc \), that \( f = g \) in \( \Tc \) if \( f - g \) factors through a projective.

    With that in mind, look at the following diagram:

    % https://q.uiver.app/#q=WzAsNixbMCwwLCJcXE9tZWdhKEEpIl0sWzAsMSwiXFxPbWVnYShCKSJdLFsxLDAsIlBfQSJdLFsxLDEsIlBfQiJdLFszLDEsIkIiXSxbMywwLCJBIl0sWzAsMSwiXFxPbWVnYShmK2cpLVxcT21lZ2EoZiktXFxPbWVnYShnKSIsMl0sWzAsMiwiXFxpb3RhX0EiLDAseyJzdHlsZSI6eyJ0YWlsIjp7Im5hbWUiOiJob29rIiwic2lkZSI6InRvcCJ9fX1dLFsxLDMsIlxcaW90YV9CIiwwLHsic3R5bGUiOnsidGFpbCI6eyJuYW1lIjoiaG9vayIsInNpZGUiOiJ0b3AifX19XSxbMyw0LCJcXHBpX0IiLDAseyJzdHlsZSI6eyJoZWFkIjp7Im5hbWUiOiJlcGkifX19XSxbMiwzLCJoX3tmICsgZ30gLSBoX2YgLSBoX2ciXSxbMiwxLCJcXHBoaSIsMix7InN0eWxlIjp7ImJvZHkiOnsibmFtZSI6ImRhc2hlZCJ9fX1dLFs1LDQsImYgKyBnIC0gZiAtIGcgPSAwIl0sWzIsNSwiXFxwaV9BIiwwLHsic3R5bGUiOnsiaGVhZCI6eyJuYW1lIjoiZXBpIn19fV1d
    \[\begin{tikzcd}
        {\Omega(A)} & {P_A} && A \\
        {\Omega(B)} & {P_B} && B
        \arrow["{\Omega(f+g)-\Omega(f)-\Omega(g)}"', from=1-1, to=2-1]
        \arrow["{\iota_A}", hook, from=1-1, to=1-2]
        \arrow["{\iota_B}", hook, from=2-1, to=2-2]
        \arrow["{\pi_B}", two heads, from=2-2, to=2-4]
        \arrow["{p_{f + g} - p_f - p_g}", from=1-2, to=2-2]
        \arrow["\phi"', dashed, from=1-2, to=2-1]
        \arrow["{f + g - f - g = 0}", from=1-4, to=2-4]
        \arrow["{\pi_A}", two heads, from=1-2, to=1-4]
    \end{tikzcd}\]

    Starting from the leftmost side, want to show that \( \Omega(f + g) - \Omega(f) - \Omega(g) \) factors through a projective.

    Firstly, one can observe that \( \iota_B \circ (\Omega(f+g)-\Omega(f)-\Omega(g)) = \iota_B \circ \Omega(f + g) - \iota_B \circ \Omega(f) - \iota_B \circ \Omega(g) = p_{f + g} \circ \iota_A - p_{f} \circ \iota_A - p_{g} \circ \iota_A = (p_{f + g} - p_f - p_g) \circ \iota_A \). So the map \( p_{f + g} - p_f - p_g: P_A \to P_B \) makes the left square commute.
    
    Secondly, one can see that \( \pi_B \circ (p_{f + g} - p_f - p_g) = \pi_B \circ p_{f + g} - \pi_B \circ p_f - \pi_B \circ  p_g = (f + g) \circ \pi_A - f \circ \pi_A - g \circ \pi_A = (f + g - f - g) \circ \pi_A = 0 \circ \pi_A = 0 \). 
    
    But then from the kernel property there is an induced and unique map \( \phi: P_A \to \Omega(B) \) such that \( \iota_B \circ \phi = p_{f + g} - p_f - p_g \). But from the commutativity of the left square, one has that \( \iota_B \circ (\Omega(f+g)-\Omega(f)-\Omega(g)) = \iota_B \circ \phi \circ \iota_A \). Furthermore, since \( \iota_A \) is a monomorphism, one gets that \( \Omega(f+g)-\Omega(f)-\Omega(g) = \phi \circ \iota_A \).

    But that implies that \( \Omega(f+g)-\Omega(f)-\Omega(g) \) factors thorugh a projective, and therefore \( \Omega(f + g) \sim \Omega(f) + \Omega(g) \).
\end{proof}

\begin{lemma}
    The definition of \( \Omega \) is well defined.
\end{lemma}
\begin{proof}
    There are two things that need to be proven. Firstly, in the construction of \( \Omega(f) \), one have to chose a map \( p_f \) from the projective property. Need to show that if one choses another projective map, that the \( \Omega \) still yields the same map in \( \Tc \). Secondly, one need to show that if \( f \sim g \), then \( \Omega(f) \sim \Omega(g) \).

    To prove the first part, let \( p_f \) and \( p_f' \) be two different projective maps that give the maps \( \Omega(f) \) and \( \Omega(f)' \) respectively. Then one has the following commutative diagram:

    % https://q.uiver.app/#q=WzAsNixbMCwwLCJcXE9tZWdhKEEpIl0sWzAsMSwiXFxPbWVnYShCKSJdLFsxLDAsIlBfQSJdLFsxLDEsIlBfQiJdLFsyLDAsIkEiXSxbMiwxLCJCIl0sWzAsMiwiIiwyLHsic3R5bGUiOnsidGFpbCI6eyJuYW1lIjoiaG9vayIsInNpZGUiOiJ0b3AifX19XSxbMiw0LCIiLDIseyJzdHlsZSI6eyJoZWFkIjp7Im5hbWUiOiJlcGkifX19XSxbMSwzLCIiLDIseyJzdHlsZSI6eyJ0YWlsIjp7Im5hbWUiOiJob29rIiwic2lkZSI6InRvcCJ9fX1dLFszLDUsIiIsMix7InN0eWxlIjp7ImhlYWQiOnsibmFtZSI6ImVwaSJ9fX1dLFs0LDUsImYgLSBmID0gMCJdLFsyLDMsImhfZiAtIGgnX2YiXSxbMCwxLCJcXE9tZWdhKGYpIC0gXFxPbWVnYShmKSciLDJdLFsyLDEsIlxccGhpIiwwLHsic3R5bGUiOnsiYm9keSI6eyJuYW1lIjoiZGFzaGVkIn19fV1d
    \[\begin{tikzcd}
        {\Omega(A)} & {P_A} & A \\
        {\Omega(B)} & {P_B} & B
        \arrow[hook, from=1-1, to=1-2]
        \arrow[two heads, from=1-2, to=1-3]
        \arrow[hook, from=2-1, to=2-2]
        \arrow[two heads, from=2-2, to=2-3]
        \arrow["{f - f = 0}", from=1-3, to=2-3]
        \arrow["{p_f - p'_f}", from=1-2, to=2-2]
        \arrow["{\Omega(f) - \Omega(f)'}"', from=1-1, to=2-1]
        \arrow["\phi", dashed, from=1-2, to=2-1]
    \end{tikzcd}\]

    Using the same argument as always, one gets that \( \Omega(f) \sim \Omega(f)' \). (I also think this follows directly from additivity.)

    Secondly, need to show that if \( f \sim g \), then \( \Omega(f) \sim \Omega(g) \). Look at the following diagram:

    % https://q.uiver.app/#q=WzAsNyxbMCwwLCJcXE9tZWdhKEEpIl0sWzAsMiwiXFxPbWVnYShCKSJdLFsxLDAsIlBfQSJdLFsxLDIsIlBfQiJdLFszLDAsIkEiXSxbMywyLCJCIl0sWzMsMSwiUCJdLFswLDIsIiIsMix7InN0eWxlIjp7InRhaWwiOnsibmFtZSI6Imhvb2siLCJzaWRlIjoidG9wIn19fV0sWzIsNCwiIiwyLHsic3R5bGUiOnsiaGVhZCI6eyJuYW1lIjoiZXBpIn19fV0sWzEsMywiIiwyLHsic3R5bGUiOnsidGFpbCI6eyJuYW1lIjoiaG9vayIsInNpZGUiOiJ0b3AifX19XSxbMyw1LCIiLDIseyJzdHlsZSI6eyJoZWFkIjp7Im5hbWUiOiJlcGkifX19XSxbMiwzLCJoX3AgXFxjaXJjIChmLWcpXzEgXFxjaXJjIFxccGlfQSJdLFswLDEsIjAiLDJdLFs0LDYsIihmLWcpXzEiLDJdLFs2LDUsIihmLWcpXzIiLDJdLFs0LDUsImYgLSBnIiwwLHsiY3VydmUiOi0zfV0sWzYsMywiaF9wIiwyLHsiY29sb3VyIjpbMCw2MCw2MF0sInN0eWxlIjp7ImJvZHkiOnsibmFtZSI6ImRhc2hlZCJ9fX0sWzAsNjAsNjAsMV1dXQ==
    \[\begin{tikzcd}
        {\Omega(A)} & {P_A} && A \\
        &&& P \\
        {\Omega(B)} & {P_B} && B
        \arrow[hook, from=1-1, to=1-2]
        \arrow[two heads, from=1-2, to=1-4]
        \arrow[hook, from=3-1, to=3-2]
        \arrow[two heads, from=3-2, to=3-4]
        \arrow["{\theta \circ (f-g)_1 \circ \pi_A}", from=1-2, to=3-2]
        \arrow["0"', from=1-1, to=3-1]
        \arrow["{(f-g)_1}"', from=1-4, to=2-4]
        \arrow["{(f-g)_2}"', from=2-4, to=3-4]
        \arrow["{f - g}", curve={height=-18pt}, from=1-4, to=3-4]
        \arrow["{\theta}"', color={rgb,255:red,214;green,92;blue,92}, dashed, from=2-4, to=3-2]
    \end{tikzcd}\]

    Let \( P \) be the projective that \( f - g \) factors through. Then from the projective propertive, one gets a map \( \theta: P \to P_B \). Then the diagram commutes. But then \( \theta \circ (f-g)_1 \circ \pi_A \circ \iota_A = \theta \circ (f-g)_1 \circ 0 = 0 \), and so the diagram commutes with \( 0: \Omega(A) \to \Omega(B) \). But since, \( \theta \circ (f-g)_1 \circ \pi_A \) is another choice of \( p_{f - g} \), from the previous part of the proof since \( \Omega \) is independent of the choice of \( h \)-maps, one gets that \( \Omega(f - g) \sim 0 \). And from additivity, one has that \( \Omega(f - g) \sim \Omega(f) - \Omega(g) \), one then gets that \( \Omega(f) \sim \Omega(g) \).
\end{proof}

% Need to show the following:
    % Sigma well defined:
    % 1) Independant of choice of P/I
    % 2) Independant of choice of projective/injective map
    % 3) Given two equivalent maps, is the image the same?
    % Sigma additive.
    % Then both \( \Sigma \) and \( \Omega \), are additive automorphisms with \( \Sigma^{-1} = \Omega \).
\begin{remark}
    Since \( R \) is a \( G \)-algebra over a field \( K \), it is known that every projective module is injective and vice versa.
\end{remark}

\begin{definition}
    Let \( A \in \Obj(\Tc) \).

    Let \( \Sigma \) be an endofunctor on \( \Tc \), where \( \Sigma(A) \) is given by choosing for every \( A \) a injective module \( I \), and a monomorphism from \( \kappa_A: A \to I \). Then taking the cokernel of that map. I.e \( \Sigma(A) = \coker(\kappa_A) \).
\end{definition}

\begin{lemma}
    One has that \( \Sigma \) is a well defined and additive functor.
\end{lemma}
\begin{proof}
    Very similar proofs as for \( \Omega \), but using the injective module property as well as the cokernel property. (TODO)
\end{proof}

\begin{theorem}
    One has that \( \Omega \) is an auto equivalence with \( \Omega^{-1} = \Sigma \).
\end{theorem}
\begin{proof}
    I will only show that \( \Id_{\Tc} \) is naturally isomorphic to \( \Sigma\Omega \). I claim (TODO) that the proof of the other part is very similar, but uses many dual properties.

    Let \( A \in \Obj(\Tc) \).

    First show that there exist a (not neccesarily unique, but for any object, just choose one) isomorphism from \( A \to \Sigma\Omega(A) \). Consider the following diagram:

    % https://q.uiver.app/#q=WzAsOSxbMCwwLCJcXE9tZWdhKEEpIl0sWzIsMCwiUF9BIl0sWzQsMCwiQSJdLFswLDEsIlxcT21lZ2EoQSkiXSxbMiwxLCJJX3tcXE9tZWdhKEEpfSJdLFs0LDEsIlxcU2lnbWFcXE9tZWdhKEEpIl0sWzAsMiwiXFxPbWVnYShBKSJdLFsyLDIsIlBfQSJdLFs0LDIsIkEiXSxbMSwyLCJcXHBpX0EiLDAseyJzdHlsZSI6eyJoZWFkIjp7Im5hbWUiOiJlcGkifX19XSxbMCwxLCJcXGlvdGFfQSIsMCx7InN0eWxlIjp7InRhaWwiOnsibmFtZSI6Imhvb2siLCJzaWRlIjoidG9wIn19fV0sWzMsNCwiXFxrYXBwYV97XFxPbWVnYShBKX0iLDAseyJzdHlsZSI6eyJ0YWlsIjp7Im5hbWUiOiJob29rIiwic2lkZSI6InRvcCJ9fX1dLFs0LDUsIlxccmhvX3tcXE9tZWdhKEEpfSIsMCx7InN0eWxlIjp7ImhlYWQiOnsibmFtZSI6ImVwaSJ9fX1dLFs2LDcsIlxcaW90YV9BIiwwLHsic3R5bGUiOnsidGFpbCI6eyJuYW1lIjoiaG9vayIsInNpZGUiOiJ0b3AifX19XSxbNyw4LCJcXHBpX0EiLDAseyJzdHlsZSI6eyJoZWFkIjp7Im5hbWUiOiJlcGkifX19XSxbMCwzLCIiLDIseyJsZXZlbCI6Miwic3R5bGUiOnsiaGVhZCI6eyJuYW1lIjoibm9uZSJ9fX1dLFszLDYsIiIsMSx7ImxldmVsIjoyLCJzdHlsZSI6eyJoZWFkIjp7Im5hbWUiOiJub25lIn19fV0sWzEsNCwiaV97XFxwaGlfMX0iLDJdLFs0LDcsImlfe1xccGhpXzJ9IiwyXSxbMiw1LCJcXHBoaV8xIiwyXSxbNSw4LCJcXHBoaV8yIiwyXV0=
    \[\begin{tikzcd}
        {\Omega(A)} && {P_A} && A \\
        {\Omega(A)} && {I_{\Omega(A)}} && {\Sigma\Omega(A)} \\
        {\Omega(A)} && {P_A} && A
        \arrow["{\pi_A}", two heads, from=1-3, to=1-5]
        \arrow["{\iota_A}", hook, from=1-1, to=1-3]
        \arrow["{\kappa_{\Omega(A)}}", hook, from=2-1, to=2-3]
        \arrow["{\rho_{\Omega(A)}}", two heads, from=2-3, to=2-5]
        \arrow["{\iota_A}", hook, from=3-1, to=3-3]
        \arrow["{\pi_A}", two heads, from=3-3, to=3-5]
        \arrow[Rightarrow, no head, from=1-1, to=2-1]
        \arrow[Rightarrow, no head, from=2-1, to=3-1]
        \arrow["{i_{\phi_1}}"', from=1-3, to=2-3]
        \arrow["{i_{\phi_2}}"', from=2-3, to=3-3]
        \arrow["{\phi_1}"', from=1-5, to=2-5]
        \arrow["{\phi_2}"', from=2-5, to=3-5]
    \end{tikzcd}\]

    Where \( i_{\phi_1} \) is the injective property map induced from \( \iota_A \). Then, since \( \rho_{\Omega(A)} \circ i_{\phi_1} \circ \iota_A = \rho_{\Omega(A)} \circ \kappa_{\Omega(A)} = 0 \), and since \( \Mod(R) \) is an abelian category, one has that the cokernel of a kernel of a epimorphism is isomorphic to the codomain of the epimorphism. Therefore one has that \( A \) is the cokernel of \( \iota_A \). Therefore, from the cokernel property, there is a uniquely induced map \( \phi_1: A \to \Sigma\Omega(A) \). Then doing the same for the lower rectangle of the diagram, using the fact that every projective is also injective, then one gets the map \( \phi_2 \).

    Then to show that \( \phi_1 \) and \( \phi_2 \) are isomorphisms, look at the following diagram:

    % https://q.uiver.app/#q=WzAsNixbMCwwLCJcXE9tZWdhKEEpIl0sWzAsMSwiXFxPbWVnYShBKSJdLFsyLDAsIlBfQSJdLFsyLDEsIlBfQSJdLFs0LDAsIkEiXSxbNCwxLCJBIl0sWzQsNSwiXFxwaGlfMiBcXGNpcmMgXFxwaGlfMSAtIFxcSWRfQSJdLFsyLDQsIlxccGlfQSIsMCx7InN0eWxlIjp7ImhlYWQiOnsibmFtZSI6ImVwaSJ9fX1dLFswLDIsIlxcaW90YV9BIiwwLHsic3R5bGUiOnsidGFpbCI6eyJuYW1lIjoiaG9vayIsInNpZGUiOiJ0b3AifX19XSxbMSwzLCJcXGlvdGFfQSIsMCx7InN0eWxlIjp7InRhaWwiOnsibmFtZSI6Imhvb2siLCJzaWRlIjoidG9wIn19fV0sWzMsNSwiXFxwaV9BIiwwLHsic3R5bGUiOnsiaGVhZCI6eyJuYW1lIjoiZXBpIn19fV0sWzIsMywicF97XFxwaGlfMn0gXFxjaXJjIHBfe1xccGhpXzF9IC0gXFxJZF9QIiwyXSxbMCwxLCJcXElkX3tcXE9tZWdhKEEpfSBcXGNpcmMgXFxJZF97XFxPbWVnYShBKX0gLSBcXElkX3tcXE9tZWdhKEEpfSA9IDAiLDJdLFs0LDMsIlxcdGhldGEiLDIseyJjb2xvdXIiOlswLDYwLDYwXX0sWzAsNjAsNjAsMV1dXQ==
    \[\begin{tikzcd}
        {\Omega(A)} && {P_A} && A \\
        {\Omega(A)} && {P_A} && A
        \arrow["{\phi_2 \circ \phi_1 - \Id_A}", from=1-5, to=2-5]
        \arrow["{\pi_A}", two heads, from=1-3, to=1-5]
        \arrow["{\iota_A}", hook, from=1-1, to=1-3]
        \arrow["{\iota_A}", hook, from=2-1, to=2-3]
        \arrow["{\pi_A}", two heads, from=2-3, to=2-5]
        \arrow["{p_{\phi_2} \circ p_{\phi_1} - \Id_P}"', from=1-3, to=2-3]
        \arrow["{\Id_{\Omega(A)} \circ \Id_{\Omega(A)} - \Id_{\Omega(A)} = 0}"', from=1-1, to=2-1]
        \arrow["\theta"', color={rgb,255:red,214;green,92;blue,92}, from=1-5, to=2-3]
    \end{tikzcd}\]

    Using the previous commutative diagram, one gets that every square commutes. But then \( (p_{\phi_2} \circ p_{\phi_1} - \Id_{P_A}) \circ \iota_A = \iota_A \circ 0 = 0 \). Then from the cokernel property there exist a map \( \theta: A \to P_A \) such that \( \theta \circ \pi_A = p_{\theta_2} \circ p_{\theta_1} - \Id_P \). But then one has that \( (\phi_2 \circ \phi_1 - \Id_A) \circ \pi_A = \pi_A \circ (p_{\theta_2} \circ p_{\theta_1} - \Id_P) = \pi_A \circ \theta \circ \pi_A \). But since \( \pi_A \) is an epimorphism, one gets that \( \phi_2 \circ \phi_1 - \Id_A = \pi_A \circ \theta \), which means that \( \phi_2 \circ \phi_1 \sim \Id_A \).
    
    Similarly (TODO) one can show that \( \phi_1 \circ \phi_2 \sim Id_A \), which means that \( \phi_1 \) and \( \phi_2 \) are isomorphisms from \( A \) to \( \Sigma\Omega(A) \).

    For \( A, B \in \Tc \), let \( f \in \Hom_R(A, B) \).

    To show that these isomorphisms are natural, look at the following two diagrams:

    % https://q.uiver.app/#q=WzAsOSxbMCwwLCJcXE9tZWdhKEEpIl0sWzIsMCwiUCJdLFs0LDAsIkEiXSxbMCwxLCJcXE9tZWdhKEEpIl0sWzIsMSwiSSJdLFs0LDEsIlxcU2lnbWFcXE9tZWdhKEEpIl0sWzAsMiwiXFxPbWVnYShCKSJdLFsyLDIsIkkiXSxbNCwyLCJcXFNpZ21hXFxPbWVnYShCKSJdLFswLDMsIiIsMCx7ImxldmVsIjoyLCJzdHlsZSI6eyJoZWFkIjp7Im5hbWUiOiJub25lIn19fV0sWzAsMSwiXFxpb3RhX0EiLDAseyJzdHlsZSI6eyJ0YWlsIjp7Im5hbWUiOiJob29rIiwic2lkZSI6InRvcCJ9fX1dLFsxLDIsIlxccGlfQSIsMCx7InN0eWxlIjp7ImhlYWQiOnsibmFtZSI6ImVwaSJ9fX1dLFszLDQsIlxca2FwcGFfe1xcT21lZ2EoQSl9IiwwLHsic3R5bGUiOnsidGFpbCI6eyJuYW1lIjoiaG9vayIsInNpZGUiOiJ0b3AifX19XSxbNCw1LCJcXHJob197XFxPbWVnYShBKX0iLDAseyJzdHlsZSI6eyJoZWFkIjp7Im5hbWUiOiJlcGkifX19XSxbMSw0LCJwX3tcXHBoaV5BXzF9IiwyXSxbMiw1LCJcXHBoaV5BXzEiLDJdLFszLDYsIlxcT21lZ2EoZikiLDJdLFs2LDcsIlxca2FwcGFfe1xcT21lZ2EoQil9IiwwLHsic3R5bGUiOnsidGFpbCI6eyJuYW1lIjoiaG9vayIsInNpZGUiOiJ0b3AifX19XSxbNyw4LCJcXHJob197XFxPbWVnYShCKX0iLDAseyJzdHlsZSI6eyJoZWFkIjp7Im5hbWUiOiJlcGkifX19XSxbNCw3LCJpX3tcXE9tZWdhKGYpfSIsMl0sWzUsOCwiXFxTaWdtYShcXE9tZWdhKGYpKSIsMl1d
    \[\begin{tikzcd}
        {\Omega(A)} && P && A \\
        {\Omega(A)} && I && {\Sigma\Omega(A)} \\
        {\Omega(B)} && I && {\Sigma\Omega(B)}
        \arrow[Rightarrow, no head, from=1-1, to=2-1]
        \arrow["{\iota_A}", hook, from=1-1, to=1-3]
        \arrow["{\pi_A}", two heads, from=1-3, to=1-5]
        \arrow["{\kappa_{\Omega(A)}}", hook, from=2-1, to=2-3]
        \arrow["{\rho_{\Omega(A)}}", two heads, from=2-3, to=2-5]
        \arrow["{p_{\phi^A_1}}"', from=1-3, to=2-3]
        \arrow["{\phi^A_1}"', from=1-5, to=2-5]
        \arrow["{\Omega(f)}"', from=2-1, to=3-1]
        \arrow["{\kappa_{\Omega(B)}}", hook, from=3-1, to=3-3]
        \arrow["{\rho_{\Omega(B)}}", two heads, from=3-3, to=3-5]
        \arrow["{i_{\Omega(f)}}"', from=2-3, to=3-3]
        \arrow["{\Sigma(\Omega(f))}"', from=2-5, to=3-5]
    \end{tikzcd}\]

    And:

    % https://q.uiver.app/#q=WzAsOSxbMCwwLCJcXE9tZWdhKEEpIl0sWzIsMCwiUCJdLFs0LDAsIkEiXSxbMCwxLCJcXE9tZWdhKEIpIl0sWzIsMSwiUCJdLFs0LDEsIkIiXSxbMCwyLCJcXE9tZWdhKEIpIl0sWzIsMiwiSSJdLFs0LDIsIlxcU2lnbWFcXE9tZWdhKEIpIl0sWzIsNSwiZiJdLFswLDMsIlxcT21lZ2EoZikiXSxbMyw2LCIiLDAseyJsZXZlbCI6Miwic3R5bGUiOnsiaGVhZCI6eyJuYW1lIjoibm9uZSJ9fX1dLFswLDEsIlxcaW90YV9BIiwwLHsic3R5bGUiOnsidGFpbCI6eyJuYW1lIjoiaG9vayIsInNpZGUiOiJ0b3AifX19XSxbMSwyLCJcXHBpX0EiLDAseyJzdHlsZSI6eyJoZWFkIjp7Im5hbWUiOiJlcGkifX19XSxbMyw0LCJcXGlvdGFfQiIsMCx7InN0eWxlIjp7InRhaWwiOnsibmFtZSI6Imhvb2siLCJzaWRlIjoidG9wIn19fV0sWzQsNSwiXFxwaV9CIiwwLHsic3R5bGUiOnsiaGVhZCI6eyJuYW1lIjoiZXBpIn19fV0sWzYsNywiXFxrYXBwYV97XFxPbWVnYShCKX0iLDAseyJzdHlsZSI6eyJ0YWlsIjp7Im5hbWUiOiJob29rIiwic2lkZSI6InRvcCJ9fX1dLFs3LDgsIlxccmhvX3tcXE9tZWdhKEIpfSIsMCx7InN0eWxlIjp7ImhlYWQiOnsibmFtZSI6ImVwaSJ9fX1dLFs1LDgsIlxccGhpXzFeQiJdLFs0LDcsImlfe1xccGhpXzFeQn0iXSxbMSw0LCJwX2YiXV0=
    \[\begin{tikzcd}
        {\Omega(A)} && P && A \\
        {\Omega(B)} && P && B \\
        {\Omega(B)} && I && {\Sigma\Omega(B)}
        \arrow["f", from=1-5, to=2-5]
        \arrow["{\Omega(f)}", from=1-1, to=2-1]
        \arrow[Rightarrow, no head, from=2-1, to=3-1]
        \arrow["{\iota_A}", hook, from=1-1, to=1-3]
        \arrow["{\pi_A}", two heads, from=1-3, to=1-5]
        \arrow["{\iota_B}", hook, from=2-1, to=2-3]
        \arrow["{\pi_B}", two heads, from=2-3, to=2-5]
        \arrow["{\kappa_{\Omega(B)}}", hook, from=3-1, to=3-3]
        \arrow["{\rho_{\Omega(B)}}", two heads, from=3-3, to=3-5]
        \arrow["{\phi_1^B}", from=2-5, to=3-5]
        \arrow["{i_{\phi_1^B}}", from=2-3, to=3-3]
        \arrow["{p_f}", from=1-3, to=2-3]
    \end{tikzcd}\]

    Where every small square, and therefore rectangle, commutes.

    This gives rise to the following commutative diagram:

    % https://q.uiver.app/#q=WzAsNixbMCwwLCJcXE9tZWdhKEEpIl0sWzQsMCwiUCJdLFs4LDAsIkEiXSxbMCwxLCJcXE9tZWdhKEIpIl0sWzQsMSwiSSJdLFs4LDEsIlxcU2lnbWFcXE9tZWdhKEIpIl0sWzIsNSwiXFxwaGlfMV5CIFxcY2lyYyBmIC0gXFxTaWdtYVxcT21lZ2EoZikgXFxjaXJjIFxccGhpXzFeQSJdLFsxLDQsImlfe1xccGhpXzFeQn0gXFxjaXJjIHBfZiAtIGlfe1xcT21lZ2EoZil9IFxcY2lyYyBwX3tcXHBoaV8xXkF9Il0sWzAsMywiXFxJZF97XFxPbWVnYShCKX0gXFxjaXJjIFxcT21lZ2EoZikgLSBcXE9tZWdhKGYpIFxcY2lyYyBcXElkX3tcXE9tZWdhKEEpfSA9IDAgIl0sWzAsMSwiIiwxLHsic3R5bGUiOnsidGFpbCI6eyJuYW1lIjoiaG9vayIsInNpZGUiOiJ0b3AifX19XSxbMSwyLCIiLDEseyJzdHlsZSI6eyJoZWFkIjp7Im5hbWUiOiJlcGkifX19XSxbMyw0LCIiLDEseyJzdHlsZSI6eyJ0YWlsIjp7Im5hbWUiOiJob29rIiwic2lkZSI6InRvcCJ9fX1dLFs0LDUsIiIsMSx7InN0eWxlIjp7ImhlYWQiOnsibmFtZSI6ImVwaSJ9fX1dLFsyLDQsIlxcdGhldGEiLDIseyJsYWJlbF9wb3NpdGlvbiI6NDAsImN1cnZlIjotMSwiY29sb3VyIjpbMCw2MCw2MF19LFswLDYwLDYwLDFdXV0=
    \[\begin{tikzcd}
        {\Omega(A)} &&&& P &&&& A \\
        {\Omega(B)} &&&& I &&&& {\Sigma\Omega(B)}
        \arrow["{\phi_1^B \circ f - \Sigma\Omega(f) \circ \phi_1^A}", from=1-9, to=2-9]
        \arrow["{i_{\phi_1^B} \circ p_f - i_{\Omega(f)} \circ p_{\phi_1^A}}", from=1-5, to=2-5]
        \arrow["{\Id_{\Omega(B)} \circ \Omega(f) - \Omega(f) \circ \Id_{\Omega(A)} = 0 }", from=1-1, to=2-1]
        \arrow[hook, from=1-1, to=1-5]
        \arrow[two heads, from=1-5, to=1-9]
        \arrow[hook, from=2-1, to=2-5]
        \arrow[two heads, from=2-5, to=2-9]
        \arrow["\theta"'{pos=0.4}, color={rgb,255:red,214;green,92;blue,92}, curve={height=-6pt}, from=1-9, to=2-5]
    \end{tikzcd}\]

    But from the cokernel property of \( A \), one gets an induced map \( \theta \), and from the epimorphism property, it makes the lower triangle commute, which implies \( \phi_1^B \circ f \sim \Sigma\Omega(f) \circ \phi_1^A \), which means that it is natural. And since the choice of \( \phi_1 \) is arbitrary, it is independent of the choice of \( \phi_1 \).
\end{proof}

I found two different definitions of distinguished triangles in \( \StMod(R) \) that are most likely equivalent (TODO):

\begin{definition}
    Definition from Zimmermann:

    Let \( \Delta \) be every triangle isomorphic to a triangle of the form of the top row of this triangle:

    % https://q.uiver.app/#q=WzAsNixbMCwwLCJBIl0sWzEsMCwiQiJdLFsyLDAsIkMiXSxbMCwxLCJBIl0sWzEsMSwiSV9BIl0sWzMsMCwiXFxTaWdtYShBKSJdLFswLDEsIlxcYWxwaGEiLDAseyJzdHlsZSI6eyJ0YWlsIjp7Im5hbWUiOiJob29rIiwic2lkZSI6InRvcCJ9fX1dLFsxLDIsIlxcYmV0YSIsMCx7InN0eWxlIjp7ImhlYWQiOnsibmFtZSI6ImVwaSJ9fX1dLFswLDMsIiIsMix7ImxldmVsIjoyLCJzdHlsZSI6eyJoZWFkIjp7Im5hbWUiOiJub25lIn19fV0sWzMsNCwiIiwyLHsic3R5bGUiOnsidGFpbCI6eyJuYW1lIjoiaG9vayIsInNpZGUiOiJ0b3AifX19XSxbNCw1LCIiLDIseyJzdHlsZSI6eyJoZWFkIjp7Im5hbWUiOiJlcGkifX19XSxbMSw0LCJcXGlvdGFfZiIsMl0sWzIsNSwiXFxnYW1tYSJdXQ==
    \[\begin{tikzcd}
        A & B & C & {\Sigma(A)} \\
        A & {I_A}
        \arrow["\alpha", hook, from=1-1, to=1-2]
        \arrow["\beta", two heads, from=1-2, to=1-3]
        \arrow[Rightarrow, no head, from=1-1, to=2-1]
        \arrow[hook, from=2-1, to=2-2]
        \arrow[two heads, from=2-2, to=1-4]
        \arrow["{\iota}"', from=1-2, to=2-2]
        \arrow["\gamma", from=1-3, to=1-4]
    \end{tikzcd}\]

    Where \( A, B, C \) with \( \alpha, \beta \) makes a short exact sequence, and \( \iota \) is given by the injective property of \( I_A \), and \( \gamma \) is the cokernel induced map.

    Definition from Martirosian:

    Let \( \Delta \) be any triangle isomorphic to a triangle of the form of the top row of the following diagram:

    % https://q.uiver.app/#q=WzAsNSxbMCwwLCJBIl0sWzEsMCwiQiJdLFsyLDAsIkNfXFxhbHBoYSJdLFszLDAsIlxcU2lnbWEoQSkiXSxbMSwxLCJJX0EiXSxbMCwxLCJcXGFscGhhIl0sWzEsMiwiXFxiZXRhIl0sWzIsMywiXFxnYW1tYSJdLFswLDQsIlxcaW90YV9BIiwyXSxbNCwyLCJcXHJobyIsMl0sWzEsMywiMCIsMCx7ImN1cnZlIjotNH1dLFs0LDMsIlxccGlfQSIsMCx7ImN1cnZlIjoyfV1d
    \[\begin{tikzcd}
        A & B & {C_\alpha} & {\Sigma(A)} \\
        & {I_A}
        \arrow["\alpha", from=1-1, to=1-2]
        \arrow["\beta", from=1-2, to=1-3]
        \arrow["\gamma", from=1-3, to=1-4]
        \arrow["{\iota_A}"', from=1-1, to=2-2]
        \arrow["\rho"', from=2-2, to=1-3]
        \arrow["0", curve={height=-24pt}, from=1-2, to=1-4]
        \arrow["{\pi_A}", curve={height=12pt}, from=2-2, to=1-4]
    \end{tikzcd}\]

    Where \( C_\alpha \) is the pushout of \( A, B, I_A \), and \( \gamma \) is given by the pushout universal property.
\end{definition}

\begin{remark}
    I think (TODO) the definitions are equivalent because a pushout can be expressed as a short exact sequence \( A \to B \oplus I_A \to C_\alpha \), and since \( B \oplus I_A \simeq B \) in \( \Tc \), then it can be possible to connect the two definitions. 
\end{remark}

\begin{theorem}
    One has that \( \StMod(R) \) is a triangulated category with \( \Sigma \) as the suspension and \( \Delta \) as the distinguished triangles.
\end{theorem}
\begin{proof}
    TODO
\end{proof}

\section{What is Toda Brackets?}

Besed on Christensen \& Frankland article (TODO).

\begin{definition}
    Given a triangulated category \( \Tc \), given the following diagram in \( \Tc \):

    % https://q.uiver.app/#q=WzAsNCxbMCwwLCJYXzAiXSxbMSwwLCJYXzEiXSxbMiwwLCJYXzIiXSxbMywwLCJYXzMiXSxbMCwxLCJmXzEiXSxbMSwyLCJmXzIiXSxbMiwzLCJmXzMiXV0=
    \[\begin{tikzcd}
        {X_0} & {X_1} & {X_2} & {X_3}
        \arrow["{f_1}", from=1-1, to=1-2]
        \arrow["{f_2}", from=1-2, to=1-3]
        \arrow["{f_3}", from=1-3, to=1-4]
    \end{tikzcd}\]

    one can define the \emph{three-fold Toda bracket of \( f_3, f_2, f_1 \)}, denoted \( \langle f_3, f_2, f_1 \rangle \), a subset of \( \Hom_{\Tc}(\Sigma(X_0), X_3) \) in three different ways:

    \begin{enumerate}
        \item {
            Iterated cofiber Toda bracket:

            Every possible \( \psi \) that makes the following diagram commute:

            % https://q.uiver.app/#q=WzAsOCxbMCwwLCJYXzAiXSxbMSwwLCJYXzEiXSxbMywwLCJcXFNpZ21hKFhfMCkiXSxbMiwwLCJDX2YiXSxbMCwxLCJYXzAiXSxbMSwxLCJYXzEiXSxbMiwxLCJYXzIiXSxbMywxLCJYXzMiXSxbMCwxLCJmXzEiXSxbMSwzXSxbMywyXSxbMCw0LCIiLDIseyJsZXZlbCI6Miwic3R5bGUiOnsiaGVhZCI6eyJuYW1lIjoibm9uZSJ9fX1dLFsxLDUsIiIsMix7ImxldmVsIjoyLCJzdHlsZSI6eyJoZWFkIjp7Im5hbWUiOiJub25lIn19fV0sWzMsNl0sWzIsNywiXFxwc2kiXSxbNCw1LCJmXzEiXSxbNSw2LCJmXzIiXSxbNiw3LCJmXzMiXV0=
            \[\begin{tikzcd}
                {X_0} & {X_1} & {Y} & {\Sigma(X_0)} \\
                {X_0} & {X_1} & {X_2} & {X_3}
                \arrow["{f_1}", from=1-1, to=1-2]
                \arrow[from=1-2, to=1-3]
                \arrow[from=1-3, to=1-4]
                \arrow[Rightarrow, no head, from=1-1, to=2-1]
                \arrow[Rightarrow, no head, from=1-2, to=2-2]
                \arrow[from=1-3, to=2-3]
                \arrow["\psi", from=1-4, to=2-4]
                \arrow["{f_1}", from=2-1, to=2-2]
                \arrow["{f_2}", from=2-2, to=2-3]
                \arrow["{f_3}", from=2-3, to=2-4]
            \end{tikzcd}\]

            Where the top row is distinguished.
        }
    \item {
            Fiber cofiber Toda bracket:

            Every composite \( \beta \circ \Sigma(\alpha) \)that makes the following diaram commute:

            % https://q.uiver.app/#q=WzAsOCxbMCwwLCJYXzAiXSxbMSwwLCJYXzEiXSxbMSwxLCJYXzEiXSxbMiwxLCJYXzIiXSxbMiwyLCJYXzIiXSxbMywyLCJYXzMiXSxbMywxLCJDX3tmXzJ9Il0sWzAsMSwiXFxTaWdtYV57LTF9KENfe2ZfMn0pIl0sWzAsMSwiZl8xIl0sWzEsMiwiIiwwLHsibGV2ZWwiOjIsInN0eWxlIjp7ImhlYWQiOnsibmFtZSI6Im5vbmUifX19XSxbMiwzLCJmXzIiXSxbMyw0LCIiLDAseyJsZXZlbCI6Miwic3R5bGUiOnsiaGVhZCI6eyJuYW1lIjoibm9uZSJ9fX1dLFs0LDUsImZfMyJdLFszLDZdLFs3LDJdLFswLDcsIlxcYWxwaGEiLDJdLFs2LDUsIlxcYmV0YSIsMl1d
            \[\begin{tikzcd}
                {X_0} & {X_1} \\
                {\Sigma^{-1}(Y)} & {X_1} & {X_2} & {Y} \\
                && {X_2} & {X_3}
                \arrow["{f_1}", from=1-1, to=1-2]
                \arrow[Rightarrow, no head, from=1-2, to=2-2]
                \arrow["{f_2}", from=2-2, to=2-3]
                \arrow[Rightarrow, no head, from=2-3, to=3-3]
                \arrow["{f_3}", from=3-3, to=3-4]
                \arrow[from=2-3, to=2-4]
                \arrow[from=2-1, to=2-2]
                \arrow["\alpha"', from=1-1, to=2-1]
                \arrow["\beta"', from=2-4, to=3-4]
            \end{tikzcd}\]

            Where the middle row is distinguished.
        }
    \item {
        Iterated fiber Toda bracket:

        Every map \( \Sigma(\delta) \), where \( \delta \) makes the following diagram commute:

        % https://q.uiver.app/#q=WzAsOCxbMCwwLCJYXzAiXSxbMSwwLCJYXzEiXSxbMiwwLCJYXzIiXSxbMywwLCJYXzMiXSxbMiwxLCJYXzIiXSxbMywxLCJYXzMiXSxbMCwxLCJcXFNpZ21hXnstMX0oWF8zKSJdLFsxLDEsIkNfe2ZfM30iXSxbMCwxLCJmXzEiXSxbMSwyLCJmXzIiXSxbMiwzLCJmXzMiXSxbNCw1LCJmXzMiXSxbNyw0XSxbNiw3XSxbMCw2LCJcXGRlbHRhIiwyXSxbMSw3XSxbMiw0LCIiLDEseyJsZXZlbCI6Miwic3R5bGUiOnsiaGVhZCI6eyJuYW1lIjoibm9uZSJ9fX1dLFszLDUsIiIsMSx7ImxldmVsIjoyLCJzdHlsZSI6eyJoZWFkIjp7Im5hbWUiOiJub25lIn19fV1d
        \[\begin{tikzcd}
            {X_0} & {X_1} & {X_2} & {X_3} \\
            {\Sigma^{-1}(X_3)} & {Y} & {X_2} & {X_3}
            \arrow["{f_1}", from=1-1, to=1-2]
            \arrow["{f_2}", from=1-2, to=1-3]
            \arrow["{f_3}", from=1-3, to=1-4]
            \arrow["{f_3}", from=2-3, to=2-4]
            \arrow[from=2-2, to=2-3]
            \arrow[from=2-1, to=2-2]
            \arrow["\delta"', from=1-1, to=2-1]
            \arrow[from=1-2, to=2-2]
            \arrow[Rightarrow, no head, from=1-3, to=2-3]
            \arrow[Rightarrow, no head, from=1-4, to=2-4]
        \end{tikzcd}\]

        Where the bottom row is distinguished.
    }
    \end{enumerate}

    Which all gives rise to the same subset of \( \Hom_{\Tc}(\Sigma(X_0), X_3) \).
\end{definition}

\begin{remark}
    Note that every \( Y \) in the above definition is isomorphic to a cone of the respective map.
\end{remark}

\subsection{Counterexample from article (TODO)}

For \( R = kG \) a group algebra, one has that in \( \StMod(R) \), the toda bracket is not equal to the toda bracket with the negative triangulation on \( \StMod(R) \).

\begin{definition}
    Negative triangulation TODO
\end{definition}

\subsection{Examples of Toda brackets}

\end{document}
