\documentclass[a4paper, 12pt]{article}

% Right-justify text. Creates some overfull hbox-es, and works weirdly with microtype.
% \usepackage[document]{ragged2e}

% Make language-specific tweaks, like changing section/theorem words or adding more hyphonation points.
\usepackage[english]{babel}

% For å kunna skriva æøå i tekstar MERK: Blir automatisk ubrukeleg med lualatex og fontspec
% \usepackage[utf8]{inputenc}

\usepackage[T1]{fontenc}

% Adds hyphenation points and aims to improve paragraph rendering.
% Use [activate=false] to disable some features that causes issues with ragged2e. Not sure if setting this option disables every feature in the package entirely, or if it's still worth keeping then.
\usepackage{microtype}

% Enables memoization
% extract=no means that latexmk need to do the extracting itself. This is controlled by the latexmkrc file.
\usepackage[extract=no]{memoize}

% Uncomment the following to recompile every tikz-diagram. NOTE: Recompile is a little borked with latexmk. TODO: Write issue to memoize.
% \mmzset{
%     recompile,
% }

% Fiksa margin
\usepackage[margin=2cm]{geometry}

% Fiksar datoformatet på tiitelen
\usepackage[ddmmyyyy]{datetime}

\usepackage{amssymb}

% For visse mattesymbol, typ \mathbb
\usepackage{amsmath}

% Bilete
\usepackage{graphicx}

% For kodesnuttar og resultat
% \usepackage{minted}

% Kan endra på korleis listar ser ut
\usepackage{enumitem}

% For autoref
\usepackage[hidelinks,colorlinks=true]{hyperref} 

% For fargar på ting ein referer til i autoref
\hypersetup{allcolors=[rgb]{0,0.31,0.62}}

% For not needing to compile twice with hyperref (?)
\usepackage{bookmark}

% For teorem, definisjon, bevis enviornments.
\usepackage{amsthm}

% For meir avanserte teoremkonstruksjonar.
\usepackage{thmtools}

% For psmallmatrix.
\usepackage{mathtools}

% For boksar rundt tekst.
\usepackage{tcolorbox}
\tcbuselibrary{skins} % For å ha meir fancy boksar, trengs for "enhanced".
\tcbuselibrary{breakable} % For å ha breakable boksar.

% For svgar
% \usepackage{svg}

% Set svg mappo
% \svgpath{svg/}

% Fjernar indents ved nye avsnitt, men gjer linjeavstanden kortare (Kanskje)
\usepackage{parskip}

% Lualatex font greie
\usepackage{fontspec}

\usepackage[warnings-off={mathtools-colon,mathtools-overbracket}]{unicode-math}

% TikZ!
\usepackage{tikz}

% Set fontar som blir brukt
\setmathfont{Latin Modern Math} % Dette er standardfonten
\setmathfont[range=\setminus]{Asana Math} % Somehow setting this changes tikz-cd arrow style
% \setmainfont{Atkinson Hyperlegible}
% \setmainfont{GFS Neohellenic Math}
% \setmainfont{Fira Sans}
% \setmathfont{Fira Math}
% \setmathfont[range=\setminus]{Asana Math}

\usetikzlibrary{matrix}

\usetikzlibrary{cd}

% In order to remove 1 pixel line at end of equal arrows.
\usetikzlibrary{nfold}

% From quiver:
% `pathmorphing` is necessary to draw squiggly arrows.
\usetikzlibrary{decorations.pathmorphing}

% From quiver:
% `calc` is necessary to draw curved arrows.
\usetikzlibrary{calc}

% Externalize TikZ diagrams to save compilation time.
% NOTE: Doesn't work with tikz-cd.
% \usetikzlibrary{external}
% \tikzexternalize[prefix=tikz/]

% \usepackage{memoize}
% \mmzset{memo dir}

% Marius Thaule TikZ matrix.
\newcommand{\diagram}[3]{\matrix[ampersand replacement = \&] (#1) [matrix of math nodes,row
  sep={#2},column sep={#3},text height=1.5ex,text
  depth=0.25ex]}
% Modified to fix the distance between nodes. Useful for diagrams where there are diagonal arrows that should be paralell.
\newcommand{\diagramorigin}[3]{\matrix[ampersand replacement = \&] (#1) [matrix of math nodes,row
  sep={#2},column sep={#3, between origins},text height=1.5ex,text
  depth=0.25ex]}

% Set style of TikZ pictures to tikz-cd style.
\tikzset{every picture/.append style={commutative diagrams/every diagram}}
% \tikzset{math/.style = {commutative diagrams/every arrow,
%   commutative diagrams/every label,
%   execute at begin node=\(, execute at end node=\)}
% }
\tikzset{math/.style = {
  execute at begin node=\(, execute at end node=\)}
}

% Shortcuts to tikz-cd styles.
\tikzset{hook/.style = {commutative diagrams/hook}}
\tikzset{dashed/.style = {commutative diagrams/dashed}}
\tikzset{two heads/.style = {commutative diagrams/two heads}}
\tikzset{equal/.style = {commutative diagrams/equal, nfold}}
\tikzset{squiggly/.style = {commutative diagrams/squiggly}}
\tikzset{marking/.style = {commutative diagrams/marking}}
\tikzset{maps to/.style = {commutative diagrams/maps to}}
\tikzset{shift right/.style = {commutative diagrams/shift right}}
\tikzset{shift left/.style = {commutative diagrams/shift left}}

% From quiver:
% A TikZ style for curved arrows of a fixed height, due to AndréC.
\tikzset{curve/.style={settings={#1},to path={(\tikztostart)
    .. controls ($(\tikztostart)!\pv{pos}!(\tikztotarget)!\pv{height}!270:(\tikztotarget)$)
    and ($(\tikztostart)!1-\pv{pos}!(\tikztotarget)!\pv{height}!270:(\tikztotarget)$)
    .. (\tikztotarget)\tikztonodes}},
    settings/.code={\tikzset{quiver/.cd,#1}
        \def\pv##1{\pgfkeysvalueof{/tikz/quiver/##1}}},
    quiver/.cd,pos/.initial=0.35,height/.initial=0
}

% Needed for suspension style.
\usetikzlibrary{decorations.markings}

% Marks the arrows with a suspension style.
\tikzset{
  suspension/.style = {postaction = decorate,
      decoration = {
          markings,
          mark = at position 0.3 with {\draw[-] (0,-0.075) -- (0,0.075);}
      },
  },
}
    

% Ny type lista med ganske perfekt spacing
\newlist{plist}{enumerate}{5}
\setlist[plist]{align=left, itemindent = 0cm, labelsep = 0cm, labelindent = 0cm}
\setlist[plist,1]{label=\arabic*, font=\bf\Large}
\setlist[plist,2]{label*=.\arabic*, labelwidth=1.25cm, leftmargin=1.25cm}
\setlist[plist,3]{label*=.\arabic*, labelwidth=1.5cm, leftmargin=1.5cm}

% Teoremstil
\theoremstyle{plain}
\newtheorem{theorem}{Theorem}[subsection]
\newtheorem{proposition}[theorem]{Proposition}
\newtheorem{corollary}[theorem]{Corollary}
\newtheorem{lemma}[theorem]{Lemma}

% Definisjonstil
\theoremstyle{definition}
\newtheorem{definition}[theorem]{Definition}
\newtheorem{example}[theorem]{Example}
\newtheorem{remark}[theorem]{Remark}
\newtheorem{construction}[theorem]{Construction}
\newtheorem{notation}[theorem]{Notation}
\newtheorem{fact}[theorem]{Fact}
\newtheorem{question}[theorem]{Question}

% Fancy box engine?
% \newcommand{\createbox}[1]{
%     \tcolorboxenvironment{#1}{
%         breakable, % Makes boxes breakable.
%         enhanced jigsaw, % Required for making breaks fancy.
%         arc = 0mm, % Corners.
%         % sharp corners, % make the corners sharp
%         oversize, % Makes the box not "squish" the text, but rather extend the box into the margins.
%         colback = white, % Changes the background colour.
%         % parskip, % Behave better with parskip package (not sure what it does)
%         % beforeafter skip balanced = 0pt, % Change spacing before and after boxes.
%         parbox = false, % Make text formatting similar to the outside.
%         before upper=\vspace{-2\parskip}, % Fix for extra parskip that comes in the start of the box.
%     }
% }

% Simple box engine?
\newcommand{\createbox}[1]{
    \tcolorboxenvironment{#1}{
        breakable, % Makes boxes breakable.
        empty, % Empty skin.
        arc = 0mm, % Corners.
        % sharp corners, % make the corners sharp
        oversize, % Makes the box not "squish" the text, but rather extend the box into the margins.
        % colback = white, % Changes the background colour.
        % parskip, % Behave better with parskip package (not sure what it does)
        % beforeafter skip balanced = 0pt, % Change spacing before and after boxes.
        parbox = false, % Make text formatting similar to the outside.
        before upper=\vspace{-2\parskip}, % Fix for extra parskip that comes in the start of the box.
        borderline={1pt}{0pt}{black}
    }
}

\createbox{theorem}
\createbox{proposition}
\createbox{corollary}
\createbox{lemma}
\createbox{definition}
\createbox{example}
\createbox{remark}
\createbox{construction}
\createbox{notation}
\createbox{fact}
\createbox{question}

% TODO: Make proof connect to the theorem box. Maybe it will look better? Then it would be difficult to have text in between.
\createbox{proof}

% Blackboard shortcuts
\newcommand{\Fb}{{\mathbb{F}}}
\newcommand{\Nb}{{\mathbb{N}}}
\newcommand{\Qb}{{\mathbb{Q}}}
\newcommand{\Rb}{{\mathbb{R}}}
\newcommand{\Zb}{{\mathbb{Z}}}

% Caligraphy shortcuts
\newcommand{\Ac}{{\mathcal{A}}}
\newcommand{\Bc}{{\mathcal{B}}}
\newcommand{\Cc}{{\mathcal{C}}}
\newcommand{\Ic}{{\mathcal{I}}}
\newcommand{\Kc}{{\mathcal{K}}}
\newcommand{\Mc}{{\mathcal{M}}}
\newcommand{\Nc}{{\mathcal{N}}}
\newcommand{\Pc}{{\mathcal{P}}}
\newcommand{\Tc}{{\mathcal{T}}}

% Set management shortcuts
\newcommand{\intersect}{\mathop{\cap}\limits}
\newcommand{\union}{\mathop{\cup}\limits}
\newcommand{\directsum}{\mathop{\oplus}\limits}

% Shorthands
\newcommand{\abs}[1]{ \lvert #1 \rvert }
\newcommand{\set}[1]{ \left\{ #1 \right\} }
\newcommand{\tuple}[1]{ \left( #1 \right) }
\newcommand{\toda}[1]{ \langle #1 \rangle }

\newcommand{\Stmod}[1]{\stablemod\tuple{ #1 }}

% New math operators
\DeclareMathOperator{\Id}{Id}
\DeclareMathOperator{\StMod}{StMod}
\DeclareMathOperator{\stablemod}{Stmod}
\DeclareMathOperator{\Obj}{Obj}
\DeclareMathOperator{\Hom}{Hom}
\DeclareMathOperator{\Mod}{Mod}
% \DeclareMathOperator{\mod}{mod}
\DeclareMathOperator{\coker}{coker}
\DeclareMathOperator{\im}{im}
\DeclareMathOperator{\Ab}{Ab}
\DeclareMathOperator{\Fun}{Fun}
\DeclareMathOperator{\dg}{dg}
\DeclareMathOperator{\C}{C}
\DeclareMathOperator{\dgMod}{dgMod}


\title{Master thesis TODO}
\author{Håvard Skjetne Lilleheie}

\begin{document}

\maketitle

\tableofcontents

\section{Toda bracket}
Toda brackets is a subset of morphisms in a triangulated category that satisfies certain properties. An interesting property of Toda brackets are that they are correlated with the differential in the second page of the Adams spectral sequence, which is the reason this thesis covers them. The details surrounding Toda brackets will become important moving on, and is the reason they are being properly defined in this thesis.

\subsection{What is a Toda bracket?}
The following three definitions are all based on \cite[Definition 3.1]{Christensen-Frankland_2017}.

Given a triangulated category \( \Tc \), given the following diagram in \( \Tc \):

    \begin{center}
        \begin{tikzpicture}
            \diagram{m}{1cm}{1cm} {
                {X_0} \& {X_1} \& {X_2} \& {X_3} \\
            };

            \draw[math]
                (m-1-1) edge node {f_1} (m-1-2)
                (m-1-2) edge node {f_2} (m-1-3)
                (m-1-3) edge node {f_3} (m-1-4);
        \end{tikzpicture}
    \end{center}

    one can define the \emph{three-fold Toda bracket of \( f_3, f_2, f_1 \)} in three different (but actually identical, see \autoref{prop:toda-bracket-definitions-coincide}) ways:

\begin{definition}[Iterated cofiber Toda bracket]
    \label{def:iterated-cofiber-toda-bracket}
    Let the set of every possible \( \psi \in \Tc(\Sigma(X_0), X_3) \) that makes the following diagram commute
    \begin{center}
        \begin{tikzpicture}
            \diagram{m}{1cm}{1cm} {
                {X_0} \& {X_1} \& {Y} \& {\Sigma(X_0)} \\
                {X_0} \& {X_1} \& {X_2} \& {X_3} \\
            };

            \draw[math]
                (m-1-1) edge node {f_1} (m-1-2)
                    edge[equal] (m-2-1)
                (m-1-2) edge (m-1-3)
                    edge[equal] (m-2-2)
                (m-1-3) edge (m-1-4)
                    edge (m-2-3)
                (m-1-4) edge node {\psi} (m-2-4)

                (m-2-1) edge node {f_1} (m-2-2)
                (m-2-2) edge node {f_2} (m-2-3)
                (m-2-3) edge node {f_3} (m-2-4);
        \end{tikzpicture}
    \end{center}
    where the top row is distinguished, be denoted as \( \toda{f_3, f_2, f_1}_{\cc} \). This is called the \emph{three-fold iterated cofiber Toda Bracket of \( f_3, f_2, f_1 \)}.
\end{definition}

\begin{definition}[Fiber-cofiber Toda bracket]
    \label{def:fiber-cofiber-toda-bracket}
    Let the set of every composite \( \beta \circ \Sigma(\alpha) \in \Tc(\Sigma(X_0), X_3) \) that makes the following diaram commute
    \begin{center}
        \begin{tikzpicture}
            \diagram{m}{1cm}{1cm} {
                {X_0} \& {X_1} \\
                {\Sigma^{-1}(Y)} \& {X_1} \& {X_2} \& {Y} \\
                \&\& {X_2} \& {X_3} \\
            };

            \draw[math]
                (m-1-1) edge node {f_1} (m-1-2)
                    edge node {\alpha} (m-2-1)
                (m-1-2) edge[equal] (m-2-2)

                (m-2-1) edge (m-2-2)
                (m-2-2) edge node {f_2} (m-2-3)
                (m-2-3) edge (m-2-4)
                    edge[equal] (m-3-3)
                (m-2-4) edge node {\beta} (m-3-4)

                (m-3-3) edge node {f_3} (m-3-4);
        \end{tikzpicture}
    \end{center}
    where the middle row is distinguished, be denoted as \( \toda{f_3, f_2, f_1}_{\fc} \). This is called the \emph{three-fold fiber-cofiber Toda Bracket of \( f_3, f_2, f_1 \)}.
\end{definition}

\begin{definition}[Iterated fiber Toda bracket]
    \label{def:iterated-fiber-toda-bracket}
    Let the set of every morphism \( \Sigma(\delta) \in \Tc(\Sigma(X_0), X_3) \), where \( \delta \) makes the following diagram commute
    \begin{center}
        \begin{tikzpicture}
            \diagram{m}{1cm}{1cm} {
                {X_0} \& {X_1} \& {X_2} \& {X_3} \\
                {\Sigma^{-1}(X_3)} \& {Y} \& {X_2} \& {X_3} \\
            };

            \draw[math]
                (m-1-1) edge node {f_1} (m-1-2)
                    edge node {\delta} (m-2-1)
                (m-1-2) edge node {f_2} (m-1-3)
                    edge (m-2-2)
                (m-1-3) edge node {f_3} (m-1-4)
                    edge[equal] (m-2-3)
                (m-1-4) edge[equal] (m-2-4)

                (m-2-1) edge (m-2-2)
                (m-2-2) edge (m-2-3)
                (m-2-3) edge node {f_3} (m-2-4);
        \end{tikzpicture}
    \end{center}
    where the bottom row is distinguished, be denoted as \( \toda{f_3, f_2, f_1}_{\ff} \). This is called the \emph{three-fold iterated fiber Toda Bracket of \( f_3, f_2, f_1 \)}.
\end{definition}

\begin{proposition}
    \label{prop:toda-bracket-definitions-coincide}
    All three of the toda brackets definitions above (\autoref{def:iterated-cofiber-toda-bracket}, \autoref{def:fiber-cofiber-toda-bracket}, \autoref{def:iterated-fiber-toda-bracket}) are equal.

    I.e:
    \[
        \toda{f_3, f_2, f_1}_{\cc} = \toda{f_3, f_2, f_1}_{\fc} = \toda{f_3, f_2, f_1}_{\ff}.
    \]
\end{proposition}
\begin{proof}
    TODO: check  \( \subseteq \) of the proof in \cite[Proposition 3.3]{Christensen-Frankland_2017}
\end{proof}

\begin{definition}
    As a consequence of \autoref{prop:toda-bracket-definitions-coincide} one can uniquely define the \emph{Toda bracket of \( f_3, f_2, f_1 \)} as either \( \toda{f_3, f_2, f_1}_{\cc} \), \( \toda{f_3, f_2, f_1}_{\fc} \), or \( \toda{f_3, f_2, f_1}_{\ff} \), and it is denoted simply as \emph{\( \toda{f_3, f_2, f_1} \)}.
\end{definition}

% TODO: Cut?
% \begin{remark}
%     Note that every \( Y \) in the definitions \autoref{def:iterated-cofiber-toda-bracket}, \autoref{def:fiber-cofiber-toda-bracket} and \autoref{def:iterated-fiber-toda-bracket}, are isomorphic to a cone of the respective map.
% \end{remark}


\subsection{Examples of Toda brackets}
Let \( R = \Fb_2C_2 \), with \( g \in C_2 \) being the generator.

Then \( J = \tuple{1 + g} \) is the only ideal of \( R \).

First note that \( J \) is not projective, since the short exact sequence:

\begin{center}
	\begin{tikzpicture}
		\diagram{m}{1cm}{1cm}{
			J \& R \& J \\
		};
	
		\draw[math]
			(m-1-1) edge[hook] node {\iota} (m-1-2)
			(m-1-2) edge[two heads] node {\phi} (m-1-3);
	\end{tikzpicture}
\end{center}

Where, \( \iota \) is the inclusion, and \( \phi \) is the map:

\begin{align*}
    \phi: R &\to J \\
    0, 1 + g &\mapsto 0 \\
    1, g &\mapsto 1 + g
\end{align*}

But \( \phi \) does not split, since \( \iota \) is the only monomorphism of \( J \) into \( R \), but composes to \( 0 \) with \( \phi \). Therefore \( J \) cannot be projective, since every epimorphism into a projective module splits.

Furthermore, since \( \phi \) is an epimorphism with kernel \( J \), one gets from the third isomorphism theorem that \( \frac{R}{J} \cong J \).

\subsection{Example 1}

The first example I calculated was the Toda bracket of the following diagram:

\begin{center}
	\begin{tikzpicture}
		\diagram{m}{1cm}{1cm}{
				J \& J \& J \& J \\
		};

		\draw[math]
			(m-1-1) edge node {\Id} (m-1-2)
			(m-1-2) edge node {0} (m-1-3)
			(m-1-3) edge node {\Id} (m-1-4);
	\end{tikzpicture}
\end{center}

The cone of \( \Id_J \) is the pushout of \( (\Id_J, \iota_J) \), where \( \iota_J \) is a monomorphism into the injective/projective module \( R \).

This is by construction \( \frac{R \oplus J}{\sim} \), where \( (0, 1+g) \sim (1+g, 0) \). This is isomorphic to \( R \).

The supension of \( J \) is the cokernel of \( \iota_J \). In \( \Mod(R) \), this is isomorphic to \( \frac{R}{J} \).

Using the cofiber-cofiber definition, one gets the following diagram:

\begin{center}
	\begin{tikzpicture}
		\diagram{m}{1cm}{1cm} {
			J \& J \& R \& {\frac{R}{J}} \\
			J \& J \& J \& J \\
		};

		\draw[math]
			(m-1-1) edge node {\Id} (m-1-2)
				edge[equal] (m-2-1)
			(m-1-2) edge (m-1-3)
				edge[equal] (m-2-2)
			(m-1-3) edge (m-1-4)
				edge node {\rho} (m-2-3)
			(m-1-4) edge node {\psi} (m-2-4)

			(m-2-1) edge node {\Id} (m-2-2)
			(m-2-2) edge node {0} (m-2-3)
			(m-2-3) edge node {\Id} (m-2-4);
	\end{tikzpicture}
\end{center}

However, since the diagram is in \( \StMod(R) \), one has that \( R \cong 0 \), and therefore \( \rho = 0 \). And from earlier one has that \( \frac{R}{J} \cong J \).

This gives the following diagram in \( \StMod(R) \):

\begin{center}
	\begin{tikzpicture}
		\diagram{m}{1cm}{1cm} {
			J \& J \& 0 \& J \\
			J \& J \& J \& J \\
		};

		\draw[math]
			(m-1-1) edge node {\Id} (m-1-2)
				edge[equal] (m-2-1)
			(m-1-2) edge node {0} (m-1-3)
				edge[equal] (m-2-2)
			(m-1-3) edge node {0} (m-1-4)
				edge node {0} (m-2-3)
			(m-1-4) edge node {\psi} (m-2-4)

			(m-2-1) edge node {\Id} (m-2-2)
			(m-2-2) edge node {0} (m-2-3)
			(m-2-3) edge node {\Id} (m-2-4);
	\end{tikzpicture}
\end{center}

But this shows that any endomorphism on \( J \) makes the rightmost square commute, and is therefore in the Toda bracket (up to pre-composition by an isomorphism \( \frac{R}{J} \to J \)) of \( \toda{\Id_J, 0, \Id_J} \). 

Toda brackets are denoted up to pre/post-composition of isomorphism, one can write that \( \toda{\Id_J, 0, \Id_J} = \StMod(R)(J, J) \).

\subsection{Example 2}

Want to show that \( \toda{\Id_J, \Id_J, \Id_J} = \emptyset \) since there is no distinguished triangle one can put in the cofiber-cofiber definition such that the squares commute.

The cone of the identity map is \( 0 \) as seen before (and in general in a triangulated category, the cone of an isomorphism is isomorphic to \( 0 \) (TODO)), and \( \Sigma(J) \cong J \):

\begin{center}
	\begin{tikzpicture}
		\diagram{m}{1cm}{1cm} {
			J \& J \& 0 \& J \\
			J \& J \& J \& J \\
		};

		\draw[math]
			(m-1-1) edge node {\Id} (m-1-2)
				edge[equal] (m-2-1)
			(m-1-2) edge node {0} (m-1-3)
				edge[equal] (m-2-2)
			(m-1-3) edge node {0} (m-1-4)
				edge[squiggly] node {0} (m-2-3)
			(m-1-4) edge node {\psi} (m-2-4)

			(m-2-1) edge node {\Id} (m-2-2)
			(m-2-2) edge node {\Id} (m-2-3)
			(m-2-3) edge node {\Id} (m-2-4);
	\end{tikzpicture}
\end{center}

There is no squiggly map in the diagram above that can make the middle square commute, unless \( J \cong 0 \).

Therefore \( \toda{\Id_J, \Id_J, \Id_J} = \emptyset \). And one has that (TODO: Cite, Supervisors said so) in general, for the Toda bracket \( \toda{f_3, f_2, f_1} \), if \( f_3 \circ f_2 \neq 0 \) or \( f_2 \circ f_1 \neq 0 \), then the Toda bracket will be empty:

\begin{theorem}
	Let \( f_1, f_2, f_3 \) be three composable maps in any triangulated category, \( \Tc \):

	\begin{center}
		\begin{tikzpicture}
			\diagram{m}{1cm}{1cm} {
				X_1 \& X_2 \& X_3 \& X_4 \\
			};

			\draw[math]
				(m-1-1) edge node {f_1} (m-1-2)
				(m-1-2) edge node {f_2} (m-1-3)
				(m-1-3) edge node {f_3} (m-1-4);
		\end{tikzpicture}
	\end{center}

	such that \( f_2 \circ f_1 \neq 0 \) or \( f_3 \circ f_2 \neq 0 \).

	Then \( \toda{f_3, f_2, f_1} = \emptyset \).
\end{theorem}
\begin{proof}
	Assume that \( \toda{f_3, f_2, f_1} \neq \emptyset \). Then from the definition of Toda bracket there exists maps \(\alpha, \beta, \phi, \psi \) such that the following diagram commutes:

	\begin{center}
		\begin{tikzpicture}
			\diagram{m}{1cm}{1cm} {
				X_1 \& X_2 \& C_{f_1} \& \Sigma(X_1) \\
				X_1 \& X_2 \& X_3 \& X_4 \\
			};

			\draw[math]
				(m-1-1) edge node {f_1} (m-1-2)
					edge[equal]	(m-2-1)
				(m-1-2) edge node {\alpha} (m-1-3)
					edge[equal] (m-2-2)
				(m-1-3) edge node {\beta} (m-1-4)
					edge node {\phi} (m-2-3)
				(m-1-4) edge node {\psi} (m-2-4)

				(m-2-1) edge node {f_1} (m-2-2)
				(m-2-2) edge node {f_2} (m-2-3)
				(m-2-3) edge node {f_3} (m-2-4);
		\end{tikzpicture}
	\end{center}

	Split the proof into two different contradictions:

	\begin{itemize}
		\item{
			Case 1:

			Assume that \( f_2 \circ f_1 \neq 0 \). Then one has that
			\[
				\phi \circ \alpha \circ f_1 = \phi \circ 0 = 0
			\]
			since \( \alpha, f_1 \) are two composable maps from the same distinguished triangle. (TODO: Cite or prove?)

			But from commutativity of the diagram one also has
			\[
				\phi \circ \alpha \circ f_1 = f_2 \circ f_1 \neq 0.
			\]
			Which is a contradiction, so \( f_2 \circ f_1 = 0 \).
		}
		\item{
			Case 2:

			Assume that \( f_3 \circ f_2 \neq 0 \). Then one has that
			\[
				0 = \psi \circ 0 = \psi \circ \beta \circ \alpha = f_3 \circ \phi \circ \alpha = f_3 \circ f_2 \neq 0.
			\]
			Which is also a contradiction.
		}
	\end{itemize}

	Therefore both \( f_2 \circ f_1 = 0 \) and \( f_3 \circ f_2 = 0 \) if \( \toda{f_3, f_2, f_1} \neq \emptyset \), which is contrapositive to the statement in the theorem.

\end{proof}

\subsection{Example 3}

Let \( \toda{f_3, 0, \Id} \) be a well defined Toda bracket. Then one has the following diagram:

\begin{center}
	\begin{tikzpicture}
		\diagram{m}{1cm}{1cm} {
			{X_1} \& {X_1} \& 0 \& {\Sigma(X_1)} \\
			{X_1} \& {X_1} \& {X_2} \& {X_3} \\
		};

		\draw[math]
			(m-1-1) edge node {\Id} (m-1-2)
				edge[equal] (m-2-1)
			(m-1-2) edge (m-1-3)
				edge[equal] (m-2-2)
			(m-1-3) edge (m-1-4)
				edge (m-2-3)
			(m-1-4) edge node {\phi} (m-2-4)

			(m-2-1) edge node {\Id} (m-2-2)
			(m-2-2) edge node {0} (m-2-3)
			(m-2-3) edge node {f_3} (m-2-4);
	\end{tikzpicture}
\end{center}

Here one has that any possible \( \psi: \Sigma(X_1) \to X_3 \) will make the right square commute. Therefore \( \toda{f_3, 0, \Id} = \Tc(\Sigma(X_1), X_3) \).

\subsection{Example 4}

Want to find the Toda bracket of the following maps using the cofiber-cofiber definition:

\begin{center}
	\begin{tikzpicture}
		\diagram{m}{1cm}{1cm} {
			J \& {J \oplus J} \& {J \oplus J} \& J \\
		};

		\draw[math]
			(m-1-1) edge node {\begin{psmallmatrix} 1 \\ 1 \end{psmallmatrix}} (m-1-2)
			(m-1-2) edge node {\begin{psmallmatrix} 1 & 1 \\ 1 & 1 \end{psmallmatrix}} (m-1-3)
			(m-1-3) edge node {\begin{psmallmatrix} 1 & 1 \end{psmallmatrix}} (m-1-4);
	\end{tikzpicture}
\end{center}

First need to find the standard triangle of \( \begin{psmallmatrix} 1 \\ 1 \end{psmallmatrix}: J \to J \oplus J \):

The cone is defined as the pushout of the following diagram:

\begin{center}
	\begin{tikzpicture}
		\diagram{m}{1cm}{1cm} {
			J \& R \\
			J \oplus J \\
		};

		\draw[math]
			(m-1-1) edge[hook] node {\iota_J} (m-1-2)
				edge[swap] node {\begin{psmallmatrix} 1 \\ 1 \end{psmallmatrix}} (m-2-1);
	\end{tikzpicture}
\end{center}

Where \( \iota_J \) is a monomorphism into a injective module. Here, this map is chosen to be the only monomorphism from \( J \to R \), namely \( \iota \).

In the category of modules, the pushout becomes:

\begin{center}
	\begin{tikzpicture}
		\diagram{m}{1cm}{1cm} {
			J \& R \\
			J \oplus J \& \frac{J \oplus J \oplus R}{\sim} \\
		};

		\draw[math]
			(m-1-1) edge[hook] node {\iota_J} (m-1-2)
				edge[swap] node {\begin{psmallmatrix} 1 \\ 1 \end{psmallmatrix}} (m-2-1)
			(m-1-2) edge node {\gamma} (m-2-2)

			(m-2-1) edge[hook] node {\rho} (m-2-2);
	\end{tikzpicture}
\end{center}

Where \( (1 + g, 1 + g, 0) \sim (0, 0, 1 + g) \).

The map \( \rho \) is given as the composition:
\begin{center}
	\begin{tikzpicture}
		\diagram{m}{1cm}{1cm} {
			J \oplus J \& J \oplus J \oplus R \& \frac{J \oplus J \oplus R}{\sim} \\
		};

		\draw[math]
			(m-1-1) edge[hook] node {i} (m-1-2)
			(m-1-2) edge[two heads] node {\pi} (m-1-3);
	\end{tikzpicture}
\end{center}

Where \( i \) is the embedding, and \( \pi \) is the quotient epimorphism.

One can check that the pushout is isomorphic to \( J \oplus R \) via the map:

\begin{align*}
	\alpha: \frac{J \oplus J \oplus R}{\sim} &\to J \oplus R \\
	(0, 0, r) &\mapsto (0, r) \\
	(1 + g, 0, r) &\mapsto (1 + g, r) \\
	(0, 1 + g, r) &\mapsto (1 + g, r) \\
	(1+ + g, 1 + g, r) &\mapsto (0, r) \\
\end{align*}

Therefore, by checking the map \( \alpha \circ \rho \) one can see that it becomes the map:
\[ 
	(\begin{psmallmatrix}
		1 & 1 \\
	\end{psmallmatrix}, 0):  J \oplus J \to J \oplus R
\]

Doing a similar argument for \( \gamma \), one gets that it is simply the embedding \( \pi \).

Furthermore, the map \( J \oplus R \to \Sigma(J) \cong \frac{R}{J} \) is given as the unique pushout map \( \beta \), satisfying the following commutative diagram:

\begin{center}
	\begin{tikzpicture}
		\diagram{m}{1cm}{2cm} {
			J \& R \& \frac{R}{J} \\
			J \oplus J \& J \oplus R \\
		};

		\draw[math]
			(m-1-1) edge[hook] node {\iota_J} (m-1-2)
				edge[swap] node {\begin{psmallmatrix} 1 \\ 1 \end{psmallmatrix}} (m-2-1)
			(m-1-2) edge[two heads] node {\pi_J} (m-1-3)
				edge node {\pi} (m-2-2)

			(m-2-1) edge[hook] node {(\begin{psmallmatrix} 1 & 1 \\ \end{psmallmatrix}, 0)} (m-2-2)
				edge[curve={height=60pt}] node {0} (m-1-3)
			(m-2-2) edge node {\beta} (m-1-3);
	\end{tikzpicture}
\end{center}

One can check that a candidate for the map \( \beta \) is \( (0, \pi) \). And since \( \beta \) is unique, this is the only map making the diagram commute.

Therefore the standard triangle becomes:

\begin{center}
	\begin{tikzpicture}
		\diagram{m}{1cm}{2cm} {
			J \& J \oplus J \& J \oplus R \& \frac{R}{J} \\
		};

		\draw[math]
			(m-1-1) edge node {\begin{psmallmatrix} 1 \\ 1 \end{psmallmatrix}} (m-1-2)
			(m-1-2) edge node {(\begin{psmallmatrix} 1 & 1 \\ \end{psmallmatrix}, 0)} (m-1-3)
			(m-1-3) edge node {(0, \pi)} (m-1-4);
	\end{tikzpicture}
\end{center}

However, since the objects and morphisms are in \( \StMod(R) \), one has that \( R \cong 0 \), and by using the isomorphism \( \frac{R}{J} \cong J \), it becomes:

\begin{center}
	\begin{tikzpicture}
		\diagram{m}{1cm}{1cm} {
			J \& J \oplus J \& J \& J \\
		};

		\draw[math]
			(m-1-1) edge node {\begin{psmallmatrix} 1 \\ 1 \end{psmallmatrix}} (m-1-2)
			(m-1-2) edge node {\begin{psmallmatrix} 1 & 1 \\ \end{psmallmatrix}} (m-1-3)
			(m-1-3) edge node {0} (m-1-4);
	\end{tikzpicture}
\end{center}

Using the cofiber-cofiber definition, one gets the following commutative diagram:

\begin{center}
	\begin{tikzpicture}
		\diagram{m}{1cm}{1cm} {
			J \& {J \oplus J} \& J \& J \\
			J \& {J \oplus J} \& {J \oplus J} \& J \\
		};

		\draw[math]
			(m-1-1) edge node {\begin{psmallmatrix} 1 \\ 1 \end{psmallmatrix}} (m-1-2)
				edge[equal] (m-2-1)
			(m-1-2) edge node {{\begin{psmallmatrix} 1 & 1 \end{psmallmatrix}}} (m-1-3)
				edge[equal] (m-2-2)
			(m-1-3) edge node {0} (m-1-4)
				edge node {\phi} (m-2-3)
			(m-1-4) edge node {\psi} (m-2-4)

			(m-2-1) edge node {\begin{psmallmatrix} 1 \\ 1 \end{psmallmatrix}} (m-2-2)
			(m-2-2) edge node {\begin{psmallmatrix} 1 & 1 \\ 1 & 1 \end{psmallmatrix}} (m-2-3)
			(m-2-3) edge node {\begin{psmallmatrix} 1 & 1 \end{psmallmatrix}} (m-2-4);
	\end{tikzpicture}
\end{center}

Where the top row is distinguished.

Firstly, using the fact that \( \begin{psmallmatrix}
		1 & 1 \\
	\end{psmallmatrix}: J \oplus J \to J \) is an epimorphism, one gets that \( \phi \circ \begin{psmallmatrix}
		1 & 1 \\
	\end{psmallmatrix} = \begin{psmallmatrix}
		1 \\ 1 \\
	\end{psmallmatrix} \begin{psmallmatrix}
		1 & 1 \\
	\end{psmallmatrix} \implies \phi = \begin{psmallmatrix}
		1 \\ 1 \\
	\end{psmallmatrix} \) by the epimorphism property.

And secondly one has that \( \begin{psmallmatrix}
		1 & 1 \\
	\end{psmallmatrix} \circ \phi = \begin{psmallmatrix}
		1 & 1 \\
	\end{psmallmatrix} \circ \begin{psmallmatrix}
		1 \\ 1 \\
	\end{psmallmatrix} = 0 \), so the Toda-bracket is non-empty.

Finally one can see that for any \( \psi: J \to J \), the right square will commute, and so the toda bracket becomes \( \StMod(R)(J, J) \).


\section{Massey product on a DG-category}
\subsection{What is a DG-category?}
\begin{notation}
    Let \( \Cc \) be any additive category.
    
    Then let \( C \tuple{\Cc} \) denote the category of chain complexes of objects in \( \Cc \).

    Furthermore let the differential in these chain complexes have \emph{ascending} order. I.e. for \( M_i, M_{i+1} \in \Cc \) which are adjacent objects in a chain complex \( M \in C \tuple{\Cc} \), the differential would be
    \[
        d_i : M_i \to M_{i + 1}.
    \]
\end{notation}

\begin{notation}
    Let \( R \) be a commutative ring with identity. Let \( A \in C \tuple{\Mod(R)} \).

    Then the degree of a \emph{homogenous} element \( a \in A \) is denoted \emph{\( \abs{a} \)}.
\end{notation}

% TODO: Prove that the resulting chain complex is a chain complex? At least add a refrence.
\begin{definition}
    \label{def:massey_product_in_dg_cat/what_is_a_dg_cat/tensor_product_of_chain_complexes}
    Let \( R \) be a commutative ring with identity. Furthermore let \( A, B \in C \tuple{\Mod(R)} \).

    Then define the modules
    \[
        (A \otimes B)_n := \bigoplus_{p + q = n} A_p \otimes B_q
    \]
    which are a part of the chain complex
    \[
        A \otimes B := \bigoplus_n \tuple{A \otimes B}_n
    \]
    with the differential (for \( a \) a homogenous element of \( A \))
    \[
        d(a \otimes b) := d(a) \otimes b + (-1)^{|a|} a \otimes d(b).
    \]

    This is called the \emph{tensor product of chain complexes over \( \Mod(R) \)}.
\end{definition}

% I have found three different definitions of a DG-category:

% \begin{definition}[DG-category, Bondal--Kapranov 1991]
%     \label{def:massey_product_in_dg_cat/what_is_a_dg_cat/dg_cat_bondal--kapranov_1991}
%     A \emph{DG-category} \( \Cc \) is a category satisfying the following criteria:
%     \begin{enumerate}
%         \item \( \Cc \) is pre-additive.
%         \item The hom-sets of \( \Cc \) are objects in \( C \tuple{Ab} \).
%         \item{
%             Composition is done with the tensor product, and should be a \( C \tuple{\Ab} \)-morphism.

%             I.e:
%             \[
%                 \circ_{A,B,C}: \Cc \tuple{B, C} \otimes \Cc \tuple{A, B} \to \Cc \tuple{A, C}
%             \]
%             should be a \( C \tuple{\Ab} \)-morphism.
%         }
%         \item For all \( A \in \Cc \), \( d(\Id_A) = 0 \).
%     \end{enumerate}
% \end{definition}

% \begin{definition}[DG-category, Keller 1994]
%     \label{def:massey_product_in_dg_cat/what_is_a_dg_cat/dg_cat_keller_1994}
%     Let \( R \) be a commutative ring.

%     A \emph{DG-category} \( \Cc \) is a category satisfying the following criteria:
%     \begin{enumerate}
%         \item The hom-sets of \( \Cc \) are objects in \( C \tuple{\Mod \tuple{R} } \).
%         \item{
%             Composition is done with the tensor product, and should be a \( C \tuple{\Mod \tuple{R} } \)-morphism.

%             I.e:
%             \[
%                 \circ_{A, B, C}: \Cc \tuple{B, C} \otimes \Cc \tuple{A, B} \to \tuple{A, C}
%             \]
%             should be a \( C \tuple{\Mod \tuple{R} } \)-morphism.
%         }
%         \item {
%             For \( f \in \Cc \tuple{A, B} \) any element, and \( g \in \Cc \tuple{B, C} \) a homogenous element, then

%             \[
%                 d(g \circ f) = d(g) \circ f + (-1)^{|g|} g \circ d(f).
%             \]
%         }
%     \end{enumerate}
% \end{definition}

% \begin{definition}[DG-category, Berest--Mehrle 2017 (Lecture notes)]
%     \label{def:massey_product_in_dg_cat/what_is_a_dg_cat/dg_cat_berest--mehrle_2017}
%     Let \( R \) be a commutative ring.

%     A \emph{DG-category} \( \Cc \) is a small category enriched in the category \( C \tuple{\Mod(R)} \).

%     This is excplicitly that \( \Cc \) consists of the following data:
%     \begin{enumerate}
%         \item A set of objects \( Ob(\Cc) \).
%         \item Every pair of objects \( A, B \in \Cc \) correspond to an object in \( C \tuple{\Mod(R)} \).
%         \item{
%             For every triple \( A, B, C \in \Cc \) there is a \( C \tuple{\Mod(R)} \) homomorphism \( \circ_{A,B,C} \):
%             \[
%                 \circ_{A, B, C}: \Cc(B, C) \otimes \Cc(A, B) \to \Cc(A, C)
%             \]
%         }
%         \item For all \( A \in \Cc \) there is an identity morphism \( \Id_A \in \Cc(A, A) \) with the expected properties.
%     \end{enumerate}
% \end{definition}

% TODO: Fix this remark since it doesn't make a lot of sense since the Leibniz rule hasn't been established as a correct rule yet.
% \begin{remark}
%     \label{rem:massey_product_in_dg_cat/what_is_a_dg_cat/d_of_id_is_zero}
%     If one has that \( \Id_A \) is the identity element in the chain complex \( \Cc\tuple{A, A} \). Then \( \abs{\Id_A} = 0 \) and it follows that
%     \[
%         d(\Id_A) = d(\Id_A \circ \Id_A) = d(\Id_A) \circ \Id_A + (-1)^0 \Id_A \circ d(\Id_A) = 2d(\Id_A)
%     \]
%     which implies that \( d(\Id_A) = 0 \).
% \end{remark}

% TODO: Remove or fix this remark.
% \begin{remark}
%     Every definition of a DG-category given above seem to be very similar in many aspects.

%     In particular, \autoref{def:massey_product_in_dg_cat/what_is_a_dg_cat/dg_cat_bondal--kapranov_1991} is a special case of \autoref{def:massey_product_in_dg_cat/what_is_a_dg_cat/dg_cat_keller_1994}, where \( R = \Zb \).

%     Also, \emph{I think} many of the points of the definitions are redundant. The supsecting redudnat points are as follows:
%     \begin{itemize}
%         \item {
%             In \autoref{def:massey_product_in_dg_cat/what_is_a_dg_cat/dg_cat_bondal--kapranov_1991} both point 1 and point 4 are redundant. Since objects in \( C \tuple{\Ab} \) are abelian groups, the category is already pre-additive.
            
%             And by sign conventions (TODO: Ref/explain/find out/understand/help!), point 2 and 3 imply a leibniz-like rule as in \autoref{def:massey_product_in_dg_cat/what_is_a_dg_cat/dg_cat_keller_1994} point 3, which by \autoref{rem:massey_product_in_dg_cat/what_is_a_dg_cat/d_of_id_is_zero} implies point 4.
%         }
%         \item {
%             In \autoref{def:massey_product_in_dg_cat/what_is_a_dg_cat/dg_cat_keller_1994} point 3 seems to follow from sign conventions (TODO: Ref/explain/find out/understand/help!).
%         }
%     \end{itemize}

%     In view of the above redundant points, \emph{I believe} that \autoref{def:massey_product_in_dg_cat/what_is_a_dg_cat/dg_cat_berest--mehrle_2017} implies the other definitions.
% \end{remark}

% MS-Question: Do I lack data in this definition?
% https://ncatlab.org/nlab/show/category+of+chain+complexes
\begin{fact}[nlab]
    Let \( R \) be a commutative ring with identity, and let \( \otimes \) denote the tensor product on \( C \tuple{\Mod(R)} \). Furthermore let \( I \) be the chain complex in \( C \tuple{\Mod(R)} \) consisting solely of \( 0 \)-objects in non-zero degrees, and the \( R \)-module \( R \) in degree 0. 

    Then \( \tuple{C \tuple{\Mod(R)}, \otimes, I} \) is a symmetric monoidal category.
\end{fact}

% This thesis will be using the following definition, which is similar to the one given by Berest--Mehrle, but not restricted to small categories.
% TODO: Cite
\begin{definition}% [DG-category]
    Let \( R \) be a commutative ring with identity.

    Then \( \Cc \) is a \emph{DG-category over \( R \)} if it is a category enriched over \( C \tuple{\Mod(R)} \).
\end{definition}
This definition also appear in Jasso--Muro p. 29. (TODO: Ref), except they define it for a field and not a commutative ring with identity.

% https://mathoverflow.net/questions/17951/what-tensor-product-of-chain-complexes-satisfies-the-usual-universal-property
% \begin{remark}
%     TODO: Connection between cartesian product and tensor product.

%     Let
%     \[
%         \phi: A \oplus B \to C.
%     \]
%     For any element \( a \in A \), one can look at the morphism resulting from fixing one of the factors of \( \phi \). Denote this by
%     \[
%         \phi_a: B \to C
%     \]
%     where \( \phi_a(b) = \phi(a, b) \).
    
%     This gives a morphism
%     \[
%         \psi: A \to \Hom(B, C).
%     \]

%     However, by tensor left-adjunction, this \( \phi \) corresponds to a morphism
%     \[
%         \tilde{\psi}: A \otimes B \to C.
%     \]
% \end{remark}

% \begin{definition}
%     Let \( A \) be an associative algebra with identity over a commutative ring with identity \( R \). Furthermore, let \( A \) have a graded ring structure. Let \( R \cdot 1_A \subseteq A_0 \).

%     Then \( A \) is called a \emph{graded algebra over \( R \)}.
% \end{definition}

% \begin{definition}
%     Let \( A \) be a graded algebra over \( R \).

%     If there is a 'graded algebra over \( R \)' -endomorphism of degree \( 1 \), denoted \( d_A \), (i.e. for any \( i \in \Zb \), \( d_A |_{A_i}: A_i \to A_{i + 1} \))
%     with the following properties:
%     \begin{enumerate}
%         \item One has that \( d \circ d = 0 \).
%         \item {
%             For \( a, b \in A \) with \( a \) homogenous, one has that
%             \[
%                 d\tuple{a \cdot b}
%                 =
%                 d\tuple{a} \cdot b + \tuple{-1}^{\abs{a}}a \cdot d\tuple{b}.
%             \]
%             }
%     \end{enumerate}

%     Then \( A \) is called a \emph{differentially graded algebra over \( R \)}.
% \end{definition}

% \begin{theorem}
%     Let \( \Cc \) be a DG-category over \( R \).

%     Then for any \( A \in \Cc \) one has that \( \Cc \tuple{A, A} \) is a differentially graded algebra over \( R \).
% \end{theorem}

% \begin{proof}
%     TODO
% \end{proof}


\section{Appendix}
This section is for general results that apply to multiple parts in the document. In my opinion these might fit as appendixes.

\subsection{Stable module category is triangulated}
% TODO: Significantly rewrite the entire section. Additivity can directly show most of the results given in this section. Also, many proofs are very similar.
\begin{definition}\label{def:stable_module_category}
    Let \( G \) be a group. Let \( R \) be a \( G \) algebra over the field \( K \), i.e. \( R = KG \) with the free module structure, and normal multiplication of group and field elements.

    Then the ``stable module category over \( R \)'', denoted \( \Tc := \StMod(R) \) is defined in the following way:
    \begin{enumerate}
        \item \( \Obj(\Tc) := \Obj(\Mod(R)) \).
        \item \( \Tc(A, B) := \Mod(R)(A, B)/\set{\text{maps that factor through a projective}} \)
    \end{enumerate}
\end{definition}

\begin{theorem}
    The definition in \autoref{def:stable_module_category} is well defined, and it is an additive category.
\end{theorem}
\begin{proof}
    First, need to check that the set of maps that factor through a projective is an \( R \) submodule of \( \Mod(R)(A, B) \).

    Let, \( f \) and \( g \) be two maps that factor through the projectives \( P \) and \( Q \) respectively. Then we have the following diagrams:

    \begin{center} % error meldingen på neste feil endre seg om ein bruke ampersand replacement med memoize!?!?! TODO
        \begin{tikzpicture}
            \diagram{m}{1cm}{1cm} {
                A \& P \& B \\
            };

            \draw[math]
                (m-1-1) edge node {f_1} (m-1-2)
                (m-1-2) edge node {f_2} (m-1-3);
        \end{tikzpicture}
    \end{center}

    Where \( f_2 \circ f_1 = f \), and

    \begin{center}
        \begin{tikzpicture}
            \diagram{m}{1cm}{1cm} {
                A \& Q \& B \\
            };

            \draw[math]
                (m-1-1) edge node {g_1} (m-1-2)
                (m-1-2) edge node {g_2} (m-1-3);
        \end{tikzpicture}
    \end{center}

    Where \( g_2 \circ g_1 = g \).

    Can then construct the map

    \begin{center}
        \begin{tikzpicture}
            \diagram{m}{1cm}{1cm} {
                A \& {P \oplus Q} \& B \\
            };

            \draw[math]
                (m-1-1) edge node {(f_1, g_1)^T} (m-1-2)
                (m-1-2) edge node {(f_2, g_2)} (m-1-3);
        \end{tikzpicture}
    \end{center}

    Composing these two maps, one gets the map \( f_2 \circ f_1 + g_2 \circ g_1 = f + g \). This maps factors thorugh \( P \oplus Q \), which is projective since it's a direct sum of projective modules.

    Therefore, the set of homomorphisms that factor through a projective is closed under addition. And multiplying with a ring element still factors through the same projective, since every map is an \( R \) homomorphism. Therefore the set of maps that factor through a projective is an \( R \) submodule.

    Therefore \( \Tc(A, B) \) is an abelian group, and the outstanding properties of an additive category is inherited from \( \Mod(R) \) as well. (TODO: Prove the unproved properties!)
\end{proof}

\begin{definition}
    Let \( A \in \Obj(\Tc) \).

    Let \( \Omega \) an endofunctor on \( \Tc \). Where \( \Omega(A) \) is given by choosing a projecive module \( P \) for every \( A \) with an endomorphism \( \pi_A \) from \( P \) to \( A \), and taking the kernel of that map. I.e \( \Omega(A) = \ker(\pi_A) \).
\end{definition}

\begin{remark}
    From the definition of \( \Omega \), \( \Omega(f) \) is constructed as follows:

    Looking at the following commutative diagram:

    \begin{center}
        \begin{tikzpicture}
            \diagram{m}{1cm}{1cm} {
                {\Omega(A)} \& {P_A} \& A \\
                {\Omega(B)} \& {P_B} \& B \\
            };

            \draw[math]
                (m-1-1) edge[hook] node {\iota_A} (m-1-2)
                    edge node {\Omega(f)} (m-2-1)
                (m-1-2) edge[two heads] node {\pi_A} (m-1-3)
                    edge node {p_f} (m-2-2)
                (m-1-3) edge node {f} (m-2-3)

                (m-2-1) edge[hook] node {\iota_B} (m-2-2)
                (m-2-2) edge[two heads] node {\pi_B} (m-2-3);
        \end{tikzpicture}
    \end{center}

    One has that for a map \( f: A \to B \), one gets the map \( p_f \) from the lifitng property of projective modules. Please note that this map is \emph{not neccesarily} unique.

    Furthermore, since \( \pi_B \circ p_f \circ \iota_A = f \circ \pi_A \circ \iota_A = f \circ 0 = 0 \), one has from the universal kernel property that there is a \emph{unique} map (given \( p_f \)) \( \Omega(f) \) from \( \Omega(A) \) to \( \Omega(B) \), which is the map defined by the functor.
\end{remark}

\begin{lemma}
    One has that \( \Omega \) is a functor.
\end{lemma}
\begin{proof}
    % Need to check the following:
    % 1) F(f o g) = F(f) o F(g)
    % 2) F(1) = 1

    First want to show that \( \Omega \) is functorial. Let \( A, B, C \in \Obj(\Tc) \). Then one can create the following diagram using the notation from before:

    \begin{center}
        \begin{tikzpicture}
            \diagram{m}{1cm}{2cm} {
                {\Omega(A)} \& {P_A} \& A \\
                {\Omega(B)} \& {P_B} \& B \\
                {\Omega(C)} \& {P_C} \& C \\
            };

            \draw[math]
                (m-1-1) edge[hook] (m-1-2)
                    edge[curve={height=30pt}, swap, color={rgb,255:red,214;green,92;blue,92}] node {\Omega(f \circ g)} (m-3-1)
                    edge node {\Omega(g)} (m-2-1)
                (m-1-2) edge[two heads] (m-1-3)
                    edge[curve={height=30pt}, swap, color={rgb,255:red,214;green,92;blue,92}] node[pos=0.3] {p_{f \circ g}} (m-3-2)
                    edge node {p_g} (m-2-2)
                (m-1-3) edge node {g} (m-2-3)
                    edge[curve={height=-30pt}, color={rgb,255:red,214;green,92;blue,92}] node {f \circ g} (m-3-3)

                (m-2-1) edge[hook] (m-2-2)
                    edge node {\Omega(f)} (m-3-1)
                (m-2-2) edge[two heads] (m-2-3)
                    edge node {p_f} (m-3-2)
                (m-2-3) edge node {f} (m-3-3)

                (m-3-1) edge[hook] (m-3-2)
                (m-3-2) edge[two heads] (m-3-3);
        \end{tikzpicture}
    \end{center}

    Then, one gets the following commuting diagram:

    \begin{center}
        \begin{tikzpicture}
            \diagram{m}{1cm}{2cm} {
                {\Omega(A)} \& {P_A} \& A \\
                {\Omega(C)} \& {P_C} \& C \\
            };

            \draw[math]
                (m-1-1) edge[hook] (m-1-2)
                    edge[swap] node {\Omega(f \circ g) - \Omega(f) \circ \Omega(g)} (m-2-1)
                (m-1-2) edge[two heads] (m-1-3)
                    edge[dashed, swap] node {\phi} (m-2-1)
                    edge node {p_{f \circ g} - p_f \circ p_g} (m-2-2)
                (m-1-3) edge node {f \circ g - f \circ g = 0} (m-2-3)

                (m-2-1) edge[hook] (m-2-2)
                (m-2-2) edge[two heads] (m-2-3);
        \end{tikzpicture}
    \end{center}

    But this implies that \( \pi_C \circ (p_{f \circ g} - p_f \circ p_g) = 0 \), which inducec a map by the kernel property \( \phi: P_A \to \Omega(C) \). Such that the lower triangle commutes. And since, \( \iota_C \) is a monomorphism, one gets that the upper triangle also commutes. And therefore \( \Omega(f \circ g) - \Omega(f) \circ \Omega(g) \) factors through a projective, and therefore \( \Omega(f \circ g) \sim \Omega(f) \circ \Omega(g) \).

    Second, need to show that \( \Omega(\Id_A) = \Id_{\Omega(A)} \) in \( \Tc \).

    By the same argument like above, one can see that every square and triangle in the following diagram also commutes:

    \begin{center}
        \begin{tikzpicture}
            \diagram{m}{1cm}{2cm} {
                {\Omega(A)} \& {P_A} \& A \\
                {\Omega(A)} \& {P_A} \& A \\
            };

            \draw[math]
                (m-1-1) edge[hook] (m-1-2)
                    edge[swap] node {\Omega(Id_A) - Id_{\Omega(A)}} (m-2-1)
                (m-1-2) edge[two heads] (m-1-3)
                    edge[dashed, swap] node {\phi} (m-2-1)
                    edge node {p_{Id_A} - Id_{P_A}} (m-2-2)
                (m-1-3) edge node {Id_A - Id_A = 0} (m-2-3)

                (m-2-1) edge[hook] (m-2-2)
                (m-2-2) edge[two heads] (m-2-3);
        \end{tikzpicture}
    \end{center}

    And therefore \( \Omega(\Id_A) \sim Id_{\Omega(A)} \).

\end{proof}

\begin{lemma}
    Let \( A, B \in \Tc \), then for \( f, g \in \Tc(A, B) \), one has that \( \Omega(f + g) = \Omega(f) + \Omega(g) \) in \( \Tc \). I.e. \( \Omega \) is additive.
\end{lemma}
\begin{proof}
    Want to show that \( \Omega(f + g) \sim \Omega(f) + \Omega(g) \).
    
    One has that for any morphisms \( f, g \in \Tc(A, B) \), from the definition of \( \Tc \), that \( f = g \) in \( \Tc \) if \( f - g \) factors through a projective.

    With that in mind, look at the following diagram:

    \begin{center}
        \begin{tikzpicture}
            \diagram{m}{1cm}{2cm} {
                {\Omega(A)} \& {P_A} \& A \\
                {\Omega(B)} \& {P_B} \& B \\
            };

            \draw[math]
                (m-1-1) edge[hook] node {\iota_A} (m-1-2)
                    edge[swap] node {\Omega(f+g)-\Omega(f)-\Omega(g)} (m-2-1)
                (m-1-2) edge[two heads] node {\pi_A} (m-1-3)
                    edge[dashed, swap] node {\phi} (m-2-1)
                    edge node {p_{f + g} - p_f - p_g} (m-2-2)
                (m-1-3) edge node {f + g - f - g = 0} (m-2-3)

                (m-2-1) edge[hook] node {\iota_B} (m-2-2)
                (m-2-2) edge[two heads] node {\pi_B} (m-2-3);
        \end{tikzpicture}
    \end{center}

    Starting from the leftmost side, want to show that \( \Omega(f + g) - \Omega(f) - \Omega(g) \) factors through a projective.

    Firstly, one can observe that \( \iota_B \circ (\Omega(f+g)-\Omega(f)-\Omega(g)) = \iota_B \circ \Omega(f + g) - \iota_B \circ \Omega(f) - \iota_B \circ \Omega(g) = p_{f + g} \circ \iota_A - p_{f} \circ \iota_A - p_{g} \circ \iota_A = (p_{f + g} - p_f - p_g) \circ \iota_A \). So the map \( p_{f + g} - p_f - p_g: P_A \to P_B \) makes the left square commute.
    
    Secondly, one can see that \( \pi_B \circ (p_{f + g} - p_f - p_g) = \pi_B \circ p_{f + g} - \pi_B \circ p_f - \pi_B \circ  p_g = (f + g) \circ \pi_A - f \circ \pi_A - g \circ \pi_A = (f + g - f - g) \circ \pi_A = 0 \circ \pi_A = 0 \). 
    
    But then from the kernel property there is an induced and unique map \( \phi: P_A \to \Omega(B) \) such that \( \iota_B \circ \phi = p_{f + g} - p_f - p_g \). But from the commutativity of the left square, one has that \( \iota_B \circ (\Omega(f+g)-\Omega(f)-\Omega(g)) = \iota_B \circ \phi \circ \iota_A \). Furthermore, since \( \iota_A \) is a monomorphism, one gets that \( \Omega(f+g)-\Omega(f)-\Omega(g) = \phi \circ \iota_A \).

    But that implies that \( \Omega(f+g)-\Omega(f)-\Omega(g) \) factors thorugh a projective, and therefore \( \Omega(f + g) \sim \Omega(f) + \Omega(g) \).
\end{proof}

\begin{lemma}
    The definition of \( \Omega \) is well defined.
\end{lemma}
\begin{proof}
    There are two things that need to be proven. Firstly, in the construction of \( \Omega(f) \), one have to chose a map \( p_f \) from the projective property. Need to show that if one choses another projective map, that the \( \Omega \) still yields the same map in \( \Tc \). Secondly, one need to show that if \( f \sim g \), then \( \Omega(f) \sim \Omega(g) \).

    To prove the first part, let \( p_f \) and \( p_f' \) be two different projective maps that give the maps \( \Omega(f) \) and \( \Omega(f)' \) respectively. Then one has the following commutative diagram:

    \begin{center}
        \begin{tikzpicture}
            \diagram{m}{1cm}{2cm} {
                {\Omega(A)} \& {P_A} \& A \\
                {\Omega(B)} \& {P_B} \& B \\
            };

            \draw[math]
                (m-1-1) edge[hook] node {\iota_A} (m-1-2)
                    edge[swap] node {\Omega(f) - \Omega(f)'} (m-2-1)
                (m-1-2) edge[two heads] node {\pi_A} (m-1-3)
                    edge[dashed, swap] node {\phi} (m-2-1)
                    edge node {p_f - p'_f} (m-2-2)
                (m-1-3) edge node {f -f = 0} (m-2-3)

                (m-2-1) edge[hook] node {\iota_B} (m-2-2)
                (m-2-2) edge[two heads] node {\pi_B} (m-2-3);
        \end{tikzpicture}
    \end{center}

    Using the same argument as always, one gets that \( \Omega(f) \sim \Omega(f)' \). (I also think this follows directly from additivity.)

    Secondly, need to show that if \( f \sim g \), then \( \Omega(f) \sim \Omega(g) \). Look at the following diagram:

    \begin{center}
        \begin{tikzpicture}
            \diagram{m}{1cm}{3cm} {
                \Omega(A) \& P_A \& A \\
                \&\& P \\
                \Omega(B) \& P_B \& B \\
            };

            \draw[math]
                (m-1-1) edge[hook] (m-1-2)
                    edge node {0} (m-3-1)
                (m-1-2) edge[two heads] (m-1-3)
                    edge node {\theta \circ (f - g)_1 \circ \pi_A} (m-3-2)
                (m-1-3) edge[swap] node {(f - g)_1} (m-2-3)
                    edge[curve={height=-25pt}] node {f - g} (m-3-3)

                (m-2-3) edge[swap, dashed, color={rgb,255:red,214;green,92;blue,92}] node {\theta} (m-3-2)
                    edge[swap] node {(f - g)_2} (m-3-3)

                (m-3-1) edge[hook] (m-3-2)
                (m-3-2) edge[two heads] (m-3-3);
        \end{tikzpicture}
    \end{center}

    Let \( P \) be the projective that \( f - g \) factors through. Then from the projective propertive, one gets a map \( \theta: P \to P_B \). Then the diagram commutes. But then \( \theta \circ (f-g)_1 \circ \pi_A \circ \iota_A = \theta \circ (f-g)_1 \circ 0 = 0 \), and so the diagram commutes with \( 0: \Omega(A) \to \Omega(B) \). But since, \( \theta \circ (f-g)_1 \circ \pi_A \) is another choice of \( p_{f - g} \), from the previous part of the proof since \( \Omega \) is independent of the choice of \( h \)-maps, one gets that \( \Omega(f - g) \sim 0 \). And from additivity, one has that \( \Omega(f - g) \sim \Omega(f) - \Omega(g) \), one then gets that \( \Omega(f) \sim \Omega(g) \).
\end{proof}

% Need to show the following:
    % Sigma well defined:
    % 1) Independant of choice of P/I
    % 2) Independant of choice of projective/injective map
    % 3) Given two equivalent maps, is the image the same?
    % Sigma additive.
    % Then both \( \Sigma \) and \( \Omega \), are additive automorphisms with \( \Sigma^{-1} = \Omega \).
\begin{remark}
    Since \( R \) is a \( G \)-algebra over a field \( K \), it is known that every projective module is injective and vice versa.
\end{remark}

\begin{definition}
    Let \( A \in \Obj(\Tc) \).

    Let \( \Sigma \) be an endofunctor on \( \Tc \), where \( \Sigma A \) is given by choosing for every \( A \) a injective module \( I \), and a monomorphism from \( \kappa_A: A \to I \). Then taking the cokernel of that map. I.e \( \Sigma A = \coker(\kappa_A) \).
\end{definition}

\begin{lemma}
    One has that \( \Sigma \) is a well defined and additive functor.
\end{lemma}
\begin{proof}
    Very similar proofs as for \( \Omega \), but using the injective module property as well as the cokernel property. (TODO)
\end{proof}

\begin{theorem}
    One has that \( \Omega \) is an auto equivalence with \( \Omega^{-1} = \Sigma \).
\end{theorem}
\begin{proof}
    I will only show that \( \Id_{\Tc} \) is naturally isomorphic to \( \Sigma\Omega \). I claim (TODO) that the proof of the other part is very similar, but uses many dual properties.

    Let \( A \in \Obj(\Tc) \).

    First show that there exist a (not neccesarily unique, but for any object, just choose one) isomorphism from \( A \to \Sigma\Omega(A) \). Consider the following diagram:

    \begin{center}
        \begin{tikzpicture}
            \diagram{m}{1cm}{2cm} {
                \Omega(A) \& P_A \& A \\
                \Omega(A) \& I_{\Omega(A)} \& \Sigma \circ \Omega(A) \\
                \Omega(A) \& P_A \& A \\
            };

            \draw[math]
                (m-1-1) edge[hook] node {\iota_A} (m-1-2)
                    edge[equal] (m-2-1)
                (m-1-2) edge[two heads] node {\pi_A} (m-1-3)
                    edge node {i_{\phi_1}} (m-2-2)
                (m-1-3) edge node {\phi_1} (m-2-3)

                (m-2-1) edge[hook] node {\kappa_{\Omega(A)}} (m-2-2)
                    edge[equal] (m-3-1)
                (m-2-2) edge[two heads] node {\rho_{\Omega(A)}} (m-2-3)
                    edge node {i_{\phi_2}} (m-3-2)
                (m-2-3) edge node {\phi_2} (m-3-3)

                (m-3-1) edge[hook] node {\iota_A} (m-3-2)
                (m-3-2) edge[two heads] node {\pi_A} (m-3-3);
        \end{tikzpicture}
    \end{center}

    Where \( i_{\phi_1} \) is the injective property map induced from \( \iota_A \). Then, since \( \rho_{\Omega(A)} \circ i_{\phi_1} \circ \iota_A = \rho_{\Omega(A)} \circ \kappa_{\Omega(A)} = 0 \), and since \( \Mod(R) \) is an abelian category, one has that the cokernel of a kernel of a epimorphism is isomorphic to the codomain of the epimorphism. Therefore one has that \( A \) is the cokernel of \( \iota_A \). Therefore, from the cokernel property, there is a uniquely induced map \( \phi_1: A \to \Sigma\Omega(A) \). Then doing the same for the lower rectangle of the diagram, using the fact that every projective is also injective, then one gets the map \( \phi_2 \).

    Then to show that \( \phi_1 \) and \( \phi_2 \) are isomorphisms, look at the following diagram:

    \begin{center}
        \begin{tikzpicture}
            \diagram{m}{1cm}{2cm} {
                \Omega(A) \& P_A \& A \\
                \Omega(A) \& P_A \& A \\
            };

            \draw[math]
                (m-1-1) edge[hook] node {\iota_A} (m-1-2)
                    edge[swap] node {\Id_{\Omega(A)} \circ \Id_{\Omega(A)} - \Id_{\Omega(A)} = 0} (m-2-1)
                (m-1-2) edge[two heads] node {\pi_A} (m-1-3)
                    edge[swap] node {p_{\phi_2} \circ p_{\phi_1} - \Id_P} (m-2-2)
                (m-1-3) edge[swap, color={rgb,255:red,214;green,92;blue,92}] node {\theta} (m-2-2)
                    edge node {\phi_2 \circ \phi_1 - \Id_A} (m-2-3)

                (m-2-1) edge[hook] node {\iota_A} (m-2-2)
                (m-2-2) edge[two heads] node {\pi_A} (m-2-3);
        \end{tikzpicture}
    \end{center}

    Using the previous commutative diagram, one gets that every square commutes. But then \( (p_{\phi_2} \circ p_{\phi_1} - \Id_{P_A}) \circ \iota_A = \iota_A \circ 0 = 0 \). Then from the cokernel property there exist a map \( \theta: A \to P_A \) such that \( \theta \circ \pi_A = p_{\theta_2} \circ p_{\theta_1} - \Id_P \). But then one has that \( (\phi_2 \circ \phi_1 - \Id_A) \circ \pi_A = \pi_A \circ (p_{\theta_2} \circ p_{\theta_1} - \Id_P) = \pi_A \circ \theta \circ \pi_A \). But since \( \pi_A \) is an epimorphism, one gets that \( \phi_2 \circ \phi_1 - \Id_A = \pi_A \circ \theta \), which means that \( \phi_2 \circ \phi_1 \sim \Id_A \).
    
    Similarly (TODO) one can show that \( \phi_1 \circ \phi_2 \sim Id_A \), which means that \( \phi_1 \) and \( \phi_2 \) are isomorphisms from \( A \) to \( \Sigma\Omega(A) \).

    For \( A, B \in \Tc \), let \( f \in \Mod(R)(A, B) \).

    To show that these isomorphisms are natural, look at the following two diagrams:

    \begin{center}
        \begin{tikzpicture}
            \diagram{m}{1cm}{2cm} {
                \Omega(A) \& P \& A \\
                \Omega(A) \& I \& \Sigma \circ \Omega(A) \\
                \Omega(B) \& I \& \Sigma \circ \Omega(B) \\
            };

            \draw[math]
                (m-1-1) edge[hook] node {\iota_A} (m-1-2)
                    edge[equal] (m-2-1)
                (m-1-2) edge[two heads] node {\pi_A} (m-1-3)
                    edge node {p_{\phi^A_1}} (m-2-2)
                (m-1-3) edge node {\phi^A_1} (m-2-3)

                (m-2-1) edge[hook] node {\kappa_{\Omega(A)}} (m-2-2)
                    edge node {\Omega(f)} (m-3-1)
                (m-2-2) edge[two heads] node {\rho_{\Omega(A)}} (m-2-3)
                    edge node {i_{\Omega(f)}} (m-3-2)
                (m-2-3) edge node {\Sigma \circ \Omega(f)} (m-3-3)

                (m-3-1) edge[hook] node {\kappa_{\Omega(B)}} (m-3-2)
                (m-3-2) edge[two heads] node {\rho_{\Omega(B)}} (m-3-3);
        \end{tikzpicture}
    \end{center}

    And:

    \begin{center}
        \begin{tikzpicture}
            \diagram{m}{1cm}{2cm} {
                \Omega(A) \& P \& A \\
                \Omega(B) \& P \& B \\
                \Omega(B) \& I \& \Sigma \circ \Omega(B) \\
            };

            \draw[math]
                (m-1-1) edge[hook] node {\iota_A} (m-1-2)
                    edge node {\Omega(f)} (m-2-1)
                (m-1-2) edge[two heads] node {\pi_A} (m-1-3)
                    edge node {p_f} (m-2-2)
                (m-1-3) edge node {f} (m-2-3)

                (m-2-1) edge[hook] node {\iota_B} (m-2-2)
                    edge[equal] (m-3-1)
                (m-2-2) edge[two heads] node {\pi_B} (m-2-3)
                    edge node {i_{\phi_1^B}} (m-3-2)
                (m-2-3) edge node {\phi_1^B} (m-3-3)

                (m-3-1) edge[hook] node {\kappa_{\Omega(B)}} (m-3-2)
                (m-3-2) edge[two heads] node {\rho_{\Omega(B)}} (m-3-3);
        \end{tikzpicture}
    \end{center}

    Where every small square, and therefore rectangle, commutes.

    This gives rise to the following commutative diagram:

    \begin{center}
        \begin{tikzpicture}
            \diagram{m}{1cm}{3cm} {
                \Omega(A) \& P \& A \\
                \Omega(B) \& I \& \Sigma \circ \Omega(B) \\
            };

            \draw[math]
                (m-1-1) edge[hook] (m-1-2)
                    edge[swap] node {\Id_{\Omega(B)} \circ \Omega(f) - \Omega(f) \circ \Id_{\Omega(A)} = 0} (m-2-1)
                (m-1-2) edge[two heads] (m-1-3)
                    edge[swap] node {i_{\phi_1^B} \circ p_f - i_{\Omega(f)} \circ p_{\phi_1^A}} (m-2-2)
                (m-1-3) edge[swap, color={rgb,255:red,214;green,92;blue,92}] node {\theta} (m-2-2)
                    edge node {\phi_1^B \circ f - \Sigma\Omega(f) \circ \phi_1^A} (m-2-3)

                (m-2-1) edge[hook] (m-2-2)
                (m-2-2) edge[two heads] (m-2-3);
        \end{tikzpicture}
    \end{center}

    But from the cokernel property of \( A \), one gets an induced map \( \theta \), and from the epimorphism property, it makes the lower triangle commute, which implies \( \phi_1^B \circ f \sim \Sigma\Omega(f) \circ \phi_1^A \), which means that it is natural. And since the choice of \( \phi_1 \) is arbitrary, it is independent of the choice of \( \phi_1 \).
\end{proof}

I found two different definitions of distinguished triangles in \( \StMod(R) \) that are most likely equivalent (TODO):

\begin{definition}
    Definition from Zimmermann:

    Let \( \Delta \) be every triangle isomorphic to a triangle of the form of the top row of this triangle:

    \begin{center}
        \begin{tikzpicture}
            \diagram{m}{1cm}{1cm} {
                A \& B \& C \& \Sigma A \\
                A \& I_A \\
            };

            \draw[math]
                (m-1-1) edge[hook] node {\alpha} (m-1-2)
                    edge[equal] (m-2-1)
                (m-1-2) edge[two heads] node {\beta} (m-1-3)
                    edge[swap] node {\iota} (m-2-2)
                (m-1-3) edge node {\gamma} (m-1-4)

                (m-2-1) edge[hook] (m-2-2)
                (m-2-2) edge[two heads] (m-1-4);
        \end{tikzpicture}
    \end{center}

    Where \( A, B, C \) with \( \alpha, \beta \) makes a short exact sequence, and \( \iota \) is given by the injective property of \( I_A \), and \( \gamma \) is the cokernel induced map.

    Definition from Martirosian:

    Let \( \Delta \) be any triangle isomorphic to a triangle of the form of the top row of the following diagram:

    \begin{center}
        \begin{tikzpicture}
            \diagram{m}{1cm}{1cm} {
                A \& B \& C_\alpha \& \Sigma A \\
                \& I_A \\
            };

            \draw[math]
                (m-1-1) edge node {\alpha} (m-1-2)
                    edge[swap] node {\iota_A} (m-2-2)
                (m-1-2) edge node {\beta} (m-1-3)
                    edge[curve={height=-25pt}] node {0} (m-1-4)
                (m-1-3) edge node {\gamma} (m-1-4)

                (m-2-2) edge[curve={height=15pt}] node {\pi_A} (m-1-4)
                    edge node {\rho} (m-1-3);
        \end{tikzpicture}
    \end{center}

    Where \( C_\alpha \) is the pushout of \( A, B, I_A \), and \( \gamma \) is given by the pushout universal property.
\end{definition}

\begin{remark} % Not neccesarily true remark, because the first definition might only work for finite dimensional algebras over fields.
    I think (TODO) the definitions are equivalent because a pushout can be expressed as a short exact sequence \( A \to B \oplus I_A \to C_\alpha \), and since \( B \oplus I_A \simeq B \) in \( \Tc \), then it can be possible to connect the two definitions. 
\end{remark}

\begin{theorem}
    One has that \( \StMod(R) \) is a triangulated category with \( \Sigma \) as the suspension and \( \Delta \) as the distinguished triangles.
\end{theorem}
\begin{proof}
    TODO
\end{proof}



\section{To be used}

This section is for parts that I have yet to use, but would like to implement into the thesis in some way.

\subsection{Projective and Injective classes}
\begin{definition}[Projective class, projective object] \label{def:projective_class}
    Let \( \Tc \) be a triangulated category.

    Let \( (\Pc, \Nc) \) be a tuple where \( \Pc \) is a class of objects, and \( \Nc \) is a class of morphisms, satisfying the following properties:

    \begin{enumerate}
        \item {Let \( f \in \Tc(X, Y) \).
        
        Then \( f \in \Nc \) if and only if for all \( P \in \Pc \) one has that \( f_*: \Tc(P, X) \to \Tc(P, Y) \) is the zero map.}

        \item {Let \( P \in \Tc \).
        
        Then \( P \in \Pc \) if and only if for all \( f \in \Tc(X, Y) \intersect \Nc \), one has that \( f_*: \Tc(P, X) \to \Tc(P, Y) \) is the zero map.}

        \item {For every \( X \in \Tc \) there exists objects \( Y \in \Tc \) and \( P \in \Pc \) along with a morphism \( f \in \Tc(X, Y) \intersect \Nc \) such that there exists a distinguished triangle on the form \( P \to X \stackrel{f}{\to} Y \to \Sigma(P) \).}
    \end{enumerate}

    Then \( (\Pc, \Nc) \) is called a \emph{projective class in \( \Tc \)}, and an object \( P \in \Pc \) is called \emph{projective}. % Projective over a projective class? TODO
\end{definition}

\begin{example}
    Let \( \Tc \) be any triangulated category. Let \( \Pc = \set{0} \), and \( \Nc = \) every morphism in \( \Tc \).

    Then \( (\Pc, \Nc) \) is a projective class in \( \Tc \).
\end{example}
\begin{proof}
    Check every property:
    \begin{enumerate}
        \item {
            For any morphism \( f \in \Tc(X, Y) \) one has that \( \Tc(0, Y) \) is the just the zero map, and so the statement holds.
        }
        \item {
            \( (\Leftarrow) \) For any object \( P \in \Tc \) one has that \( f = \Id_P \in \Tc(P, P) \), but \( (\Id_P)_*: \Tc(P, P) \to \Tc(P, P) \) is the zero map if and only if \( P = 0 \).

            \( (\Rightarrow) \) If \( P = 0 \), then for any \( f \in \Tc(X, Y) \), one has that \( \Tc(0, Y) \) is the zero map, so the statement holds. 
        }
        \item {
            Let \( X \in \Tc \), then the (right-shifted) trivial triangle
            \begin{center}
                \begin{tikzpicture}
                    \diagram{m}{1cm}{1cm} {
                        0 \& X \& X \& 0 \\
                    };

                    \draw[math]
                        (m-1-1) edge (m-1-2)
                        (m-1-2) edge node {\Id_X} (m-1-3)
                        (m-1-3) edge (m-1-4);
                \end{tikzpicture}
            \end{center}
            satisfies this property.
        }
    \end{enumerate}
\end{proof}

\begin{remark}
    For any projective class \( ( \Pc, \Nc ) \) in \( \Tc \), one must have \( 0 \in \Pc \), since it always satisfies property 2 in \autoref{def:projective_class}, and \( 0 \in \Nc \), since it always satisfies property 1.
\end{remark}

\begin{definition}[Stable projective class]
    Let \( (\Pc, \Nc) \) be a projective class in \( \Tc \).

    If for any \( P \in \Pc \) and \( n \in \Zb \) one has that \( \Sigma^n(P) \in \Pc \).
    
    Then \( (\Pc, \Nc) \) is called a \emph{stable projective class}.
\end{definition}

% Projective class stable under coproduct and retracts? TODO

\begin{definition}[\( \Pc \)-epic, \( \Pc \)-monic]
    Let \( f \in \Tc(X, Y) \) and let \( (\Pc, \Nc) \) be a projective class in \( \Tc \).

    Then one has the following definitions:

    % surjective or epimorphism? Injective or monomorphism?
    \begin{enumerate}
        \item {If for all \( P \in \Pc \), one has that \( f_*: \Tc(P, X) \to \Tc(P, Y) \) is surjective.
        
        Then \( f \) is called \emph{\( \Pc \)-epic}.}
        
        \item {If for all \( P \in \Pc \), one has that \( f_*: \Tc(P, X) \to \Tc(P, Y) \) is injective.
        
        Then \( f \) is called \emph{\( \Pc \)-monic}.}
    \end{enumerate}
\end{definition}

% Equivalent to cofiber map being P-null, or fiber map being P-null. TODO

\begin{definition}[Injective class, injective object]
    Let \( \Tc \) be a triangulated category.

    If the tuple \( (\Ic, \Nc) \) is a projective class in \( \Tc^{op} \).
    
    Then \( (\Ic, \Nc) \) is called an \emph{injective class in \( \Tc \)}, and an object \( I \in \Ic \) is called \emph{injective}.
\end{definition}

% Explicit definition. TODO

% Stable under products and retracts? TODO.

\begin{definition}[Stable injective class]
    Let \( (\Ic, \Nc) \) be an injective class in \( \Tc \).

    If for any \( I \in \Ic \) and \( n \in \Zb \) one has that \( \Sigma^n(I) \in \Ic \).
    
    Then \( (\Ic, \Nc) \) is called a \emph{stable injective class}.
\end{definition}

\begin{definition}[\( \Ic \)-monic, \( \Ic \)-epic]
    Let \( f \in \Tc(X, Y) \) and let \( (\Ic, \Nc) \) be an injective class in \( \Tc \).

    Then one has the following definitions:

    \begin{enumerate} % surjective or epimorphism? Injective or monomorphism?
        \item {If for all \( I \in \Ic \), one has that \( f_*: \Tc(I, X) \to \Tc(I, Y) \) is surjective.
        
        Then \( f \) is called \emph{\( \Ic \)-monic}.}
        
        \item {If for all \( I \in \Ic \), one has that \( f_*: \Tc(I, X) \to \Tc(I, Y) \) is injective.
        
        Then \( f \) is called \emph{\( \Ic \)-epic}.}
    \end{enumerate}
\end{definition}

% Equivalent to fiber map being I-null, or cofiber map being I-null. TODO

% Remark -> Connection to lifitng and extension property.

% Convention neccesary?

\begin{definition}[Adams resolution w.r.t. a projective class] \label{def:adams_resolution_projective_class} % Need stable? TODO
    Let \( (\Pc, \Nc) \) be a projective class in \( \Tc \). Let \( X_0 \in \Tc \).

    Then given objects and morphisms in \( \Tc \) fitting in the following diagram:

    \begin{center}
        \begin{tikzpicture}
            \diagram{m}{1cm}{1cm} {
                X_0 \&\& X_1 \&\& X_2 \&\& X_3 \& \dots \\
                \& P_0 \&\& P_1 \&\& P_2 \\
            };

            \draw[math]
                (m-1-1) edge node {i_0} (m-1-3)
                (m-1-3) edge node {i_1} (m-1-5)
                    edge[suspension] node {\delta_0} (m-2-2)
                (m-1-5) edge node {i_2} (m-1-7)
                    edge[suspension] node {\delta_1} (m-2-4)
                (m-1-7) edge (m-1-8)
                    edge[suspension] node {\delta_2} (m-2-6)

                (m-2-2) edge node {p_0} (m-1-1)
                (m-2-4) edge node {p_1} (m-1-3)
                (m-2-6) edge node {p_2} (m-1-5);
        \end{tikzpicture}
    \end{center}

    Where every \( i_n \in \Nc \), and \( P_n \in \Pc \), and marked arrows denote degree shifting maps, with \( \delta_n \in \Tc(X_{n + 1}, \Sigma(P_n)) \). And where every triangle on the form:

    \begin{center}
        \begin{tikzpicture}
            \diagram{m}{1cm}{1cm} {
                P_n \& X_n \& X_{n + 1} \& \Sigma(P_n) \\
            };

            \draw[math]
                (m-1-1) edge node {p_n} (m-1-2)
                (m-1-2) edge node {i_n} (m-1-3)
                (m-1-3) edge node {\delta_n} (m-1-4);
        \end{tikzpicture}
    \end{center}

    is distinguished.

    This is called an \emph{Adams resolution of \( X_0 \) with respect to a projective class \( (\Pc, \Nc) \)}.
\end{definition}

\begin{definition}[Adams resolution w.r.t. an injective class] \label{def:adams_resolution_injective_class} % Need stable? TODO
    Let \( (\Ic, \Nc) \) be an injective class in \( \Tc \). Let \( Y_0 \in \Tc \).

    Then given objects and morphisms in \( \Tc \) fitting in the following diagram:

    \begin{center}
        \begin{tikzpicture}
            \diagram{m}{1cm}{1cm} {
                Y_0 \&\& Y_1 \&\& Y_2 \&\& Y_3 \& \dots \\
                \& I_0 \&\& I_1 \&\& I_2 \\
            };

            \draw[math]
                (m-1-1) edge[swap] node {p_0} (m-2-2)
                (m-1-3) edge[swap] node {i_0} (m-1-1)
                    edge[swap] node {p_1} (m-2-4)
                (m-1-5) edge[swap] node {i_1} (m-1-3)
                    edge[swap] node {p_2} (m-2-6)
                (m-1-7) edge[swap] node {i_2} (m-1-5)
                (m-1-8) edge (m-1-7)

                (m-2-2) edge[suspension] node[swap] {\delta_0} (m-1-3)
                (m-2-4) edge[suspension] node[swap] {\delta_1} (m-1-5)
                (m-2-6) edge[suspension] node[swap] {\delta_2} (m-1-7);
        \end{tikzpicture}
    \end{center}

    Where every \( i_n \in \Nc \), and \( I_n \in \Ic \), and marked arrows denote degree shifting maps, with \( \delta_n \in \Tc(I_n, \Sigma(Y_{n + 1})) \). And where every triangle on the form:

    \begin{center}
        \begin{tikzpicture}
            \diagram{m}{1cm}{2cm} {
                \Sigma^{-1}(I_n) \& Y_{n + 1} \& Y_n \& I_n \\
            };

            \draw[math]
                (m-1-1) edge node {\Sigma^{-1}(\delta_s)} (m-1-2)
                (m-1-2) edge node {i_n} (m-1-3)
                (m-1-3) edge node {p_n} (m-1-4);
        \end{tikzpicture}
    \end{center}

    is distinguished.

    This is called an \emph{Adams resolution of \( Y_0 \) with respect to an injective class \( (\Ic, \Nc) \)}.
\end{definition}

\begin{theorem} % Stable? TODO
    Let \( (\Pc, \Nc) \) be a projective class in \( \Tc \).

    Then for any object \( X \in \Tc \), there exists an Adams Resolution of \( X \) with respect to the projective class \( (\Pc, \Nc) \).
\end{theorem}
\begin{proof}
    Let \( X_0 = X \).

    Then by \autoref{def:projective_class} property 3, one has that there exists an object \( X_1 \in \Tc \) and object \( P_0 \in \Pc \), and three morphisms \( i_0, p_0, \delta_0 \), such that the following is a distinguished triangle
    \[
        P_0 \stackrel{p_0}{\longrightarrow} X_0 \stackrel{i_0}{\longrightarrow} X_1 \stackrel{\delta_0}{\longrightarrow} \Sigma(P_0)
    \]
    and \( i_0 \in \Nc \).

    Repeat this process with \( X_1, X_2, \dots, X_n \), until one gets every map \( i_n, p_n, \delta_n \) for any \( n \in \Nb \).

    These maps fit into the diagram seen in \autoref{def:adams_resolution_projective_class}.
\end{proof}

\begin{theorem} \label{thm:exists_adams_resolution_injective_class} % Stable? TODO
    Let \( (\Ic, \Nc) \) be an injective class in \( \Tc \).

    Then for any object \( Y \in \Tc \), there exists an Adams resolution of \( Y \) with respect to the injective class \( (\Ic, \Nc) \).
\end{theorem}
\begin{proof}
    TODO
\end{proof}

\begin{construction} \label{construction:adams_spectral_sequence} 
    Let \( (\Ic, \Nc) \) be an injective class in \( \Tc \).

    For any \( Y \in \Tc \), one can construct an Adams resolution of \( Y \) with respect to the injective class \( (\Ic, \Nc) \) by \autoref{thm:exists_adams_resolution_injective_class}, using the notation from \autoref{def:adams_resolution_injective_class}.

    For any \( X \in \Tc, t \in \Zb, s \in \Nb_0 \), let:
    \[ 
        \phi: \Tc(\Sigma^{t - s}(X), \Sigma(Y_{s + 1})) \stackrel{\cong}{\longrightarrow} \Tc(\Sigma^{t - s - 1}(X), Y_{s + 1}) 
    \]
    Denote the natural isomorphism from \( \Sigma \)'s automorphism property.

    Furthermore, let
    \begin{align*}
        (i_s)_*: \Tc(\Sigma^{t - s}(X), Y_s) &\to \Tc(\Sigma^{t - s}(X), Y_{s - 1}) \\
        (p_s)_*: \Tc(\Sigma^{t - s}(X), Y_s) &\to \Tc(\Sigma^{t - s}(X), I_s) \\
        (\phi \circ \delta_s)_*: \Tc(\Sigma^{t - s}(X), I_s) &\to \Tc(\Sigma^{t - s - 1}(X), Y_{s + 1})
    \end{align*}
    be maps given by post-composition by \( i_s, p_s, \phi \circ \delta_s \) respectively.

    Let \( i, p, \delta \) be morphisms that fit in the following diagram:

    \begin{center}
        \begin{tikzpicture}
            \diagram{m}{2cm}{1cm} {
                \directsum_{s \in \Nb_0, t \in \Zb}\Tc(\Sigma^{t - s}(X), Y_s) \&\& \directsum_{s \in \Nb_0, t \in \Zb}\Tc(\Sigma^{t - s}(X), Y_s) \\
                \&  \directsum_{s \in \Nb_0, t \in \Zb}\Tc(\Sigma^{t - s}(X), I_s) \\
            };

            % Very sus 0-length arrow created when using anchors.
            % tips=proper or use "to" instead of "edge" (only on final arrow, or else there is no head).
            \draw[math, tips=proper]
                (m-1-1) edge node {i} (m-1-3)
                (m-1-3) edge node {p} (m-2-2.north east)

                (m-2-2.north west) edge node {\delta} (m-1-1);
        \end{tikzpicture}
    \end{center}

    Where \( i \) is a direct sum of either \( 0 \) on the coordinates with \( s = 0 \), or \( (i_s)_* \) for some \( s \neq 0 \), depending on what fits. Likewise, \( p \) is a direct sum of \( (p_s)_* \), and \( \delta \) is a direct sum of \( (\phi \circ \delta_s)_* \), for \( s \) such that it fits.
\end{construction}

\begin{theorem}
    The diagram in \autoref{construction:adams_spectral_sequence} is an exact couple.
\end{theorem}
\begin{proof} % Simplify proof, not dependant on Adams resolution structure at all? TODO
    Want to prove that the diagram is exact in the three corners.

    Firstly, note that since \( i, p, \delta \) are all direct summands of maps, it is sufficient to check that it is exact for every ``level'' of the direct sum.

    Therefore, fix any \( t \in \Zb \) and \( s \in \Nb_0 \).

    Focusing on the top right corner one needs to show that the maps \( i \) and \( p \) are exact given these \( t \) and \( s \).

    Firstly note that
    \begin{center}
        \begin{tikzpicture}
            \diagram{m}{1cm}{1cm} {
                \Sigma^{-1}(I_s) \& Y_{s+1} \& Y_s \& I_s \\
            };

            \draw[math]
                (m-1-1) edge node {\Sigma^{-1}(\delta_s)} (m-1-2)
                (m-1-2) edge node {i_s} (m-1-3)
                (m-1-3) edge node {p_s} (m-1-4);
        \end{tikzpicture}
    \end{center}
    is a distinguished triangle by \autoref{def:adams_resolution_injective_class}. Then one can construct the following long exact sequence:
    \begin{center}
        \begin{tikzpicture}
            \diagram{m}{1cm}{1cm} {
                \dots \& \Tc(\Sigma^{t-s}(X), Y_{s+1}) \& \Tc(\Sigma^{t-s}(X), Y_s) \& \Tc(\Sigma^{t-s}(X), I_s) \& \dots \\
            };

            \draw[math]
                (m-1-1) edge (m-1-2)
                (m-1-2) edge node {(i_s)_*} (m-1-3)
                (m-1-3) edge node {(p_s)_*} (m-1-4)
                (m-1-4) edge (m-1-5);
        \end{tikzpicture}
    \end{center}

    Which is exactly how the maps \( i \) and \( p \) would interact on the coordinate with \( s \) and \( t \) fixed, and is therefore exact.

    Focusing on the bottom corner, it is very similar:

    The following triangle is distinguished
    \begin{center}
        \begin{tikzpicture}
            \diagram{m}{1cm}{1cm} {
                Y_{s+1} \& Y_s \& I_s \& \Sigma(Y_{s+1}) \\
            };

            \draw[math]
                (m-1-1) edge node {i_s} (m-1-2)
                (m-1-2) edge node {p_s} (m-1-3)
                (m-1-3) edge node {\delta_s} (m-1-4);
        \end{tikzpicture}
    \end{center}
    since it is a left-rotated version of the distinguished triangle above.

    Again, this gives rise to the following long exact sequence
    \begin{center}
        \begin{tikzpicture}
            \diagram{m}{1cm}{1cm} {
                \dots \& \Tc(\Sigma^{t-s}(X), Y_s) \& \Tc(\Sigma^{t-s}(X), I_s) \& \Tc(\Sigma^{t-s}(X), \Sigma(Y_{s + 1})) \& \dots \\
            };

            \draw[math]
                (m-1-1) edge (m-1-2)
                (m-1-2) edge node {(p_s)_*} (m-1-3)
                (m-1-3) edge node {(\delta_s)_*} (m-1-4)
                (m-1-4) edge (m-1-5);
        \end{tikzpicture}
    \end{center}
    Changing the right map to \( (\phi \circ \delta_s)_* \), one gets the following sequence
    \begin{center}
        \begin{tikzpicture}
            \diagram{m}{1cm}{1cm} {
                \dots \& \Tc(\Sigma^{t-s}(X), Y_s) \& \Tc(\Sigma^{t-s}(X), I_s) \& \Tc(\Sigma^{t-s-1}(X), Y_{s + 1}) \& \dots \\
            };

            \draw[math]
                (m-1-1) edge (m-1-2)
                (m-1-2) edge node {(p_s)_*} (m-1-3)
                (m-1-3) edge node {\phi \circ (\delta_s)_*} (m-1-4)
                (m-1-4) edge (m-1-5);
        \end{tikzpicture}
    \end{center}
    Which is still exact in the middle, because \( \ker((\delta_s)_*) = \ker((\phi \circ \delta_s)_*) \), since \( \phi \) is an isomorphism.

    And since the middle part is exactly how the maps \( p \) and \( \delta \) would meet for given \( s \) and \( t \), the diagram is also exact at the bottom.

    Lastly at the top left corner one has to divide the problem into two cases:

    If \( s \neq 0 \):

    Then the following triangle is distinguished and exists since \( s \geq 1 \)
    \begin{center}
        \begin{tikzpicture}
            \diagram{m}{1cm}{1cm} {
                I_{s - 1} \& \Sigma(Y_s) \& \Sigma(Y_{s - 1}) \& \Sigma(I_{s - 1}) \\
            };

            \draw[math]
                (m-1-1) edge node[above=5pt] {\delta_{s - 1}} (m-1-2)
                (m-1-2) edge node[above=5pt] {\Sigma(i_{s - 1})} (m-1-3)
                (m-1-3) edge node[above=5pt] {\Sigma(p_{s - 1})} (m-1-4);
        \end{tikzpicture}
    \end{center}
    because it is the same as the triangles above, but with the indexes reduced by one.

    Using this, one can construct the following long exact sequence
    \begin{center}
        \begin{tikzpicture}
            \diagram{m}{1cm}{1cm} {
                \dots \& \Tc(\Sigma^{t-s}(X), I_{s - 1}) \& \Tc(\Sigma^{t-s}(X), \Sigma(Y_s)) \& \Tc(\Sigma^{t - s}(X), \Sigma(Y_{s - 1})) \& \dots \\
            };

            \draw[math]
                (m-1-1) edge (m-1-2)
                (m-1-2) edge node[above=5pt] {(\delta_{s - 1})_*} (m-1-3)
                (m-1-3) edge node[above=5pt] {(\Sigma(i_{s - 1}))_*} (m-1-4)
                (m-1-4) edge (m-1-5);
        \end{tikzpicture}
    \end{center}

    Let \( \psi \) be the natural isomorphism
    \[
        \psi: \Tc(\Sigma^{t-s}(X), \Sigma(Y_{s - 1})) \stackrel{\cong}{\longrightarrow} \Tc(\Sigma^{t - s - 1}(X), Y_{s - 1}).
    \]

    Then by changing the maps, one gets the following sequence
    \begin{center}
        \begin{tikzpicture}
            \diagram{m}{1cm}{1cm} {
                \dots \& \Tc(\Sigma^{t-s}(X), I_{s - 1}) \& \Tc(\Sigma^{t - s - 1}(X), Y_s) \& \Tc(\Sigma^{t - s - 1}(X), Y_{s - 1}) \& \dots \\
            };

            \draw[math]
                (m-1-1) edge (m-1-2)
                (m-1-2) edge node[above=5pt] {\phi \circ (\delta_{s - 1})_*} (m-1-3)
                (m-1-3) edge node[above=5pt] {\psi \circ (\Sigma(i_{s - 1}))_* \circ \phi^{-1}} (m-1-4)
                (m-1-4) edge (m-1-5);
        \end{tikzpicture}
    \end{center}
    Where the middle part is still exact, because
    % Need lemma for last part?
    \begin{align*}
        \im((\delta_{s-1})_*) &= \ker((\Sigma(i_{s-1}))_*) \\
        &\Updownarrow \\
        \phi(\im((\delta_{s-1})_*)) &= \phi(\ker((\Sigma(i_{s-1}))_*)) \\
        \im(\phi \circ (\delta_{s-1})_*) &= \phi(\ker((\Sigma(i_{s-1}))_*)) \\
        &= \ker((\Sigma(i_{s-1}))_* \circ \phi^{-1})
    \end{align*}
    and since \( \ker(\psi \circ (\Sigma(i_{s-1}))_* \circ \phi^{-1}) = \ker((\Sigma(i_{s-1}))_* \circ \phi^{-1}) \), since \( \psi \) is an isomorphism.

    Furthermore, by the definition of \( \phi \) and \( \psi \), one has the following commutative diagram
    \begin{center}
        \begin{tikzpicture}
            \diagram{m}{1cm}{1cm} {
                \Sigma^{t - s}( X ) \& \Sigma( Y_s ) \& \Sigma( Y_{s - 1} ) \\
                \Sigma^{t - s - 1}( X ) \& Y_s \& Y_{s - 1} \\
            };

            \draw[math]
                (m-1-1) edge node {\Sigma( f )} (m-1-2)
                    edge node {\eta_1} (m-2-1)
                (m-1-2) edge node {\Sigma( i_s )} (m-1-3)
                    edge node {\eta_2} (m-2-2)
                (m-1-3) edge node {\eta_3} (m-2-3)

                (m-2-1) edge node {f} (m-2-2)
                (m-2-2) edge node {i_s} (m-2-3);
        \end{tikzpicture}
    \end{center}
    Where the \( \eta_j \)'s are the natural isomorphisms that define both \( \phi \) and \( \psi \).
    
    I.e. for any \( f \in \Tc(\Sigma^{t - s - 1}(X), Y_s) \) and \( g \in \Tc(\Sigma^{t - s - 1}, Y_{s - 1}) \), one has
    \[
        \phi: \Sigma( f ) \mapsto f = \eta_2 \circ \Sigma( f ) \circ \eta_1^{-1}
    \]
    and
    \[
        \psi: \Sigma( g ) \mapsto g = \eta_3 \circ \Sigma( g ) \circ \eta_1^{-1}.
    \]

    Then look at \( \psi \circ (\Sigma(i_{s-1}))_* \circ \phi^{-1} \) pointwise
    \begin{align*}
        \psi \circ (\Sigma(i_{s-1}))_* \circ \phi^{-1} ( f ) &= \psi \circ (\Sigma(i_{s-1}))_* ( \eta_2^{-1} \circ f \circ \eta_1 ) \\
        &= \psi ( \Sigma(i_{s - 1}) \circ \eta_2^{-1} \circ f \circ \eta_1 ) \\
        &= \eta_3 \circ \Sigma(i_{s - 1}) \circ \eta_2^{-1} \circ f \circ \eta_1 \circ \eta_1^{-1} \\
        &= \eta_3 \circ \Sigma(i_{s - 1}) \circ \eta_2^{-1} \circ f \\
        \intertext{By commutativity of the above diagram, this becomes}
        &= i_{s - 1} \circ f
    \end{align*}

    Which means that \( \psi \circ (\Sigma(i_{s-1}))_* \circ \phi^{-1} = ( i_{s - 1} )_* \). One thereforore gets the following sequence that is identical to the one above
    \begin{center}
        \begin{tikzpicture}
            \diagram{m}{1cm}{1cm} {
                \dots \& \Tc(\Sigma^{t-s}(X), I_{s - 1}) \& \Tc(\Sigma^{t - s - 1}(X), Y_s) \& \Tc(\Sigma^{t - s - 1}(X), Y_{s - 1}) \& \dots \\
            };

            \draw[math]
                (m-1-1) edge (m-1-2)
                (m-1-2) edge node[above=5pt] {\phi \circ (\delta_{s - 1})_*} (m-1-3)
                (m-1-3) edge node[above=5pt] {i_{s - 1}} (m-1-4)
                (m-1-4) edge (m-1-5);
        \end{tikzpicture}
    \end{center}
    Which is exactly how the maps \( \delta \) and \( i \) would meet for given \( s \) and \( t \), and is exact in the middle, implying the top left corner of the above diagram would be exact.

    And finally, if \( s = 0 \):

    The image of \( (\phi \circ \delta_n)_* \) for any \( n \) would never have codomain \( Y_0 \), except for the zero map by the definition, and therefore there is no image onto \( \Tc(\Sigma^t( X ), Y_0) \). There is also no map \( i_n \) with domain \( \Tc(\Sigma^t( X ), Y_0) \), except for the zero map \( i_0 = 0 \). And so, it is exact by default.

    Therefore one has that the top left corner is also exact for the given \( t \) and \( s \).

    By conclusion, since the choice of \( s \) and \( t \) was arbitrary, every corner is exact for any \( s \in \Nb_0 \) and \( t \in \Zb \), and by the argument at the start of the proof, the theorem follows.
\end{proof}

\begin{definition}[Adams spectral sequence]
    Let \( (\Ic, \Nc) \) be a stable injective class in \( \Tc \).

    Then

    TODO: Finish
\end{definition}


\subsection{Two examples of projective classes in \texorpdfstring{\( \Stmod{\Fb_3 C_3} \)}{Stmod(F\_3C\_3)}}
\subsubsection{Preliminaries}

\begin{definition} \label{thm:F_functor}
    Define \( F: \Mc \to \Mc \) to be an assignment that takes any object \( A \in \Mc \) and maps it to its decomposition by \autoref{thm:f_3c_3_decomposition}, and morphisms are induced by the isomorphisms from the decomposition. 
    
    I.e. there are some \( n, m \in \Nb_0 \) such that \( A \mapsto F(A) = S^n \oplus M^m \). And furthermore for \( f \in \Mc\tuple{A, B} \), one has \( f \mapsto F(f) := \psi_A^{-1} \circ f \circ \psi_B \), where \( \psi_A \) and \( \psi_B \) are the chosen isomorphisms between \( A \) and it's decomposition, and \( B \) and it's decomposition, respectively.
\end{definition}

\begin{lemma}
    The assignment \( F: \Mc \to \Mc \) from \autoref{thm:F_functor} is a functor.
\end{lemma}
\begin{proof}
    TODO
\end{proof}

\begin{lemma} \label{lem:projection_unique}
    Let \( \Ac \) be an additive category. Let \( A, B, C \in \Ac \).

    Then \( p_{A \oplus B}^A \circ p_{A \oplus B \oplus C}^{A \oplus B} \) is ``equal'' to \( p_{A \oplus B \oplus C}^A \) up to pre-composing with an isomorphism, which is omitted from the notation.
    
    Likewise \( i_{A \oplus B}^{A \oplus B \oplus C} \circ i_A^{A \oplus B} \) is ``equal'' to \( i_A^{A \oplus B \oplus C} \) up to pre-composing with an isomorphism, which is omitted from the notation.
\end{lemma}
\begin{proof}
    TODO
\end{proof}

\begin{remark} \label{rem:big_iso}
    By using \autoref{lem:hom_split_over_direct_sum_n_naturally} one gets that the following diagram
    \begin{center}
        \begin{tikzpicture}
            \diagram{m}{1cm}{1cm} {
                \Mc\tuple{S^{n_A} \oplus M^{m_A}, S^{n_B} \oplus M^{m_B}} \\
                \Mc\tuple{S^{n_A}, S^{n_B} \oplus M^{m_B}} \oplus \Mc\tuple{M^{m_A}, S^{n_B} \oplus M^{m_B}} \\
                \Mc\tuple{S^{n_A}, S^{n_B}} \oplus \Mc\tuple{S^{n_A}, M^{m_B}} \oplus \Mc\tuple{M^{m_A}, S^{n_B}} \oplus \Mc\tuple{M^{m_A}, M^{m_B}} \\
                \Mc\tuple{S, S}^{n_A n_B} \oplus \Mc\tuple{S, M}^{n_A m_B} \oplus \Mc\tuple{M, S}^{m_A n_B} \oplus \Mc\tuple{M, M}^{m_A m_B} \\
            };

            \draw[math]
                (m-1-1) edge node[marking, below] {\sim} node {g_1} (m-2-1)

                (m-2-1) edge node[marking, below] {\sim} node {g_2} (m-3-1)

                (m-3-1) edge node[marking, below] {\sim} node {g_3} (m-4-1);
        \end{tikzpicture}
    \end{center}
    
    
    Composing, one gets the isomorphism \( \phi = g_3 \circ g_2 \circ g_1 \) such that
    \begin{align*}
        &\Mc\tuple{S^{n_A} \oplus M^{m_A}, S^{n_B} \oplus M^{m_B}} \\
        &\stackrel{\phi}{\cong} \Mc\tuple{ S, S }^{n_An_B} \\
        &\oplus \Mc\tuple{ S, M }^{n_Am_B} \\
        &\oplus \Mc\tuple{ M, S }^{m_An_B} \\
        &\oplus \Mc\tuple{ M, M }^{m_Am_B}.
    \end{align*}

    Where
    \[
        g_1 =
        \begin{psmallmatrix}
            \tuple{ i_{S^{n_A}}^{S^{n_A} \oplus M^{m_A}} }^* \\
            \tuple{ i_{M^{m_A}}^{S^{n_A} \oplus M^{m_A}} }^*
        \end{psmallmatrix},
    \]
    \[
        g_2 =
        \begin{psmallmatrix}
            \tuple{ p_{S^{n_B} \oplus M^{m_B}}^{S^{n_B}} }_* \\
            \tuple{ p_{S^{n_B} \oplus M^{m_B}}^{M^{m_B}} }_*
        \end{psmallmatrix}
        \oplus
        \begin{psmallmatrix}
            \tuple{ p_{S^{n_B} \oplus M^{m_B}}^{S^{n_B}} }_* \\
            \tuple{ p_{S^{n_B} \oplus M^{m_B}}^{M^{m_B}} }_*
        \end{psmallmatrix}
    \]
    and
    \begin{multline*}
        g_3 =
        \begin{psmallmatrix}
            (p_{S^{n_B}}^{S_1})_* \circ (i_{S_1}^{S^{n_A}})^* \\
            (p_{S^{n_B}}^{S_2})_* \circ (i_{S_1}^{S^{n_A}})^* \\
            \vdots \\
            (p_{S^{n_B}}^{S_{n_B}})_* \circ (i_{S_{n_A}}^{S^{n_A}})^*
        \end{psmallmatrix}
        \oplus
        \begin{psmallmatrix}
            (p_{M^{m_B}}^{M_1})_* \circ (i_{S_1}^{S^{n_A}})^* \\
            (p_{M^{m_B}}^{M_2})_* \circ (i_{S_1}^{S^{n_A}})^* \\
            \vdots \\
            (p_{M^{m_B}}^{M_{m_B}})_* \circ (i_{S_{n_A}}^{S^{n_A}})^*
        \end{psmallmatrix}
        \oplus
        \begin{psmallmatrix}
            (p_{S^{n_B}}^{S_1})_* \circ (i_{M_1}^{M^{m_A}})^* \\
            (p_{S^{n_B}}^{S_2})_* \circ (i_{M_1}^{M^{m_A}})^* \\
            \vdots \\
            (p_{S^{n_B}}^{S_{n_B}})_* \circ (i_{M_{m_A}}^{M^{m_A}})^*
        \end{psmallmatrix}
        \oplus
        \begin{psmallmatrix}
            (p_{M^{m_B}}^{M_1})_* \circ (i_{M_1}^{M^{m_A}})^* \\
            (p_{M^{m_B}}^{M_2})_* \circ (i_{M_1}^{M^{m_A}})^* \\
            \vdots \\
            (p_{M^{m_B}}^{M_{m_B}})_* \circ (i_{M_{m_A}}^{M^{m_A}})^*
        \end{psmallmatrix}.
    \end{multline*}

    And from \autoref{lem:projection_unique} one can calculate
    \[
        \phi = g_3 \circ g_2 \circ g_1 =
        \begin{psmallmatrix}
            \tuple{ p_{S^{n_B} \oplus M^{m_B}}^{S_1} }_* \circ \tuple{ i_{S_1}^{S^{n_A} \oplus M^{m_A}} }^* \\
            \tuple{ p_{S^{n_B} \oplus M^{m_B}}^{S_2} }_* \circ \tuple{ i_{S_1}^{S^{n_A} \oplus M^{m_A}} }^* \\
            \vdots \\
            \tuple{ p_{S^{n_B} \oplus M^{m_B}}^{S_{n_B}} }_* \circ \tuple{ i_{S_{n_A}}^{S^{n_A} \oplus M^{m_A}} }^* \\
            \tuple{ p_{S^{n_B} \oplus M^{m_B}}^{M_1} }_* \circ \tuple{ i_{S_1}^{S^{n_A} \oplus M^{m_A}} }^* \\
            \tuple{ p_{S^{n_B} \oplus M^{m_B}}^{M_2} }_* \circ \tuple{ i_{S_1}^{S^{n_A} \oplus M^{m_A}} }^* \\
            \vdots \\
            \tuple{ p_{S^{n_B} \oplus M^{m_B}}^{M_{m_B}} }_* \circ \tuple{ i_{S_{n_A}}^{S^{n_A} \oplus M^{m_A}} }^* \\
            \tuple{ p_{S^{n_B} \oplus M^{m_B}}^{S_1} }_* \circ \tuple{ i_{M_1}^{S^{n_A} \oplus M^{m_A}} }^* \\
            \tuple{ p_{S^{n_B} \oplus M^{m_B}}^{S_2} }_* \circ \tuple{ i_{M_1}^{S^{n_A} \oplus M^{m_A}} }^* \\
            \vdots \\
            \tuple{ p_{S^{n_B} \oplus M^{m_B}}^{S_{n_B}} }_* \circ \tuple{ i_{M_{m_A}}^{S^{n_A} \oplus M^{m_A}} }^* \\
            \tuple{ p_{S^{n_B} \oplus M^{m_B}}^{M_1} }_* \circ \tuple{ i_{M_1}^{S^{n_A} \oplus M^{m_A}} }^* \\
            \tuple{ p_{S^{n_B} \oplus M^{m_B}}^{M_2} }_* \circ \tuple{ i_{M_1}^{S^{n_A} \oplus M^{m_A}} }^* \\
            \vdots \\
            \tuple{ p_{S^{n_B} \oplus M^{m_B}}^{M_{m_B}} }_* \circ \tuple{ i_{M_{m_A}}^{S^{n_A} \oplus M^{m_A}} }^*
        \end{psmallmatrix}.
    \]

    Similarly one has that \( \phi^{-1} = g_1^{-1} \circ g_2^{-1} \circ g_3^{-1} \), where
    \[
        g_1^{-1} =
        \begin{psmallmatrix}
            \tuple{ p_{S^{n_A} \oplus M^{m_A}}^{S^{n_A}} }^*, & \tuple{ p_{S^{n_A} \oplus M^{m_A}}^{M^{m_A}} }^*
        \end{psmallmatrix},
    \]
    \[
        g_2^{-1} =
        \begin{psmallmatrix}
            \tuple{ i_{S^{n_B}}^{S^{n_B} \oplus M^{m_B}} }_*, & \tuple{ i_{M^{m_B}}^{S^{n_B} \oplus M^{m_B}} }_*
        \end{psmallmatrix}
        \oplus
        \begin{psmallmatrix}
            \tuple{ i_{S^{n_B}}^{S^{n_B} \oplus M^{m_B}} }_*, & \tuple{ i_{M^{m_B}}^{S^{n_B} \oplus M^{m_B}} }_*
        \end{psmallmatrix},
    \]
    and
    \begin{multline*}
        g_3^{-1} =
        \begin{psmallmatrix}
            \tuple{ p_{S^{n_A}}^{S_1} }^* \circ \tuple{ i_{S_1}^{S^{n_B}} }_*, & \tuple{ p_{S^{n_A}}^{S_1} }^* \circ \tuple{ i_{S_2}^{S^{n_B}} }_*, & \dots, & \tuple{ p_{S^{n_A}}^{S_{n_A}} }^* \circ \tuple{ i_{S_{n_B}}^{S^{n_B}} }_*
        \end{psmallmatrix} \\
        \oplus
        \begin{psmallmatrix}
            \tuple{ p_{S^{n_A}}^{S_1} }^* \circ \tuple{ i_{M_1}^{M^{m_B}} }_*, & \tuple{ p_{S^{n_A}}^{S_1} }^* \circ \tuple{ i_{M_2}^{M^{m_B}} }_*, & \dots, & \tuple{ p_{S^{n_A}}^{S_{n_A}} }^* \circ \tuple{ i_{M_{m_B}}^{M^{m_B}} }_*
        \end{psmallmatrix} \\
        \oplus
        \begin{psmallmatrix}
            \tuple{ p_{M^{m_A}}^{M_1} }^* \circ \tuple{ i_{S_1}^{S^{n_B}} }_*, & \tuple{ p_{M^{m_A}}^{M_1} }^* \circ \tuple{ i_{S_2}^{S^{n_B}} }_*, & \dots, & \tuple{ p_{M^{m_A}}^{M_{m_A}} }^* \circ \tuple{ i_{S_{n_B}}^{S^{n_B}} }_*
        \end{psmallmatrix} \\
        \oplus
        \begin{psmallmatrix}
            \tuple{ p_{M^{m_A}}^{M_1} }^* \circ \tuple{ i_{M_1}^{M^{m_B}} }_*, & \tuple{ p_{M^{m_A}}^{M_1} }^* \circ \tuple{ i_{M_2}^{M^{m_B}} }_*, & \dots, & \tuple{ p_{M^{m_A}}^{M_{m_A}} }^* \circ \tuple{ i_{M_{m_B}}^{M^{m_B}} }_*
        \end{psmallmatrix}.
    \end{multline*}

    This gives (a wide, one index tall matrix)
    \begin{multline*}
        \phi^{-1} =  g_1^{-1} \circ g_2^{-1} \circ g_3^{-1} = \\
        \biggl(
            \begin{smallmatrix}
                \tuple{ p_{S^{n_A} \oplus M^{m_A}}^{S_1} }^* \circ \tuple{ i_{S_1}^{S^{n_B} \oplus M^{m_B}} }_* &
                \tuple{ p_{S^{n_A} \oplus M^{m_A}}^{S_1} }^* \circ \tuple{ i_{S_2}^{S^{n_B} \oplus M^{m_B}} }_* &
                \dots &
                \tuple{ p_{S^{n_A} \oplus M^{m_A}}^{S_{n_A}} }^* \circ \tuple{ i_{S_{n_B}}^{S^{n_B} \oplus M^{m_B}} }_*
            \end{smallmatrix}
            \\
            \begin{smallmatrix}
                \tuple{ p_{S^{n_A} \oplus M^{m_A}}^{S_1} }^* \circ \tuple{ i_{M_1}^{S^{n_B} \oplus M^{m_B}} }_* &
                \tuple{ p_{S^{n_A} \oplus M^{m_A}}^{S_1} }^* \circ \tuple{ i_{M_2}^{S^{n_B} \oplus M^{m_B}} }_* &
                \dots &
                \tuple{ p_{S^{n_A} \oplus M^{m_A}}^{S_{n_A}} }^* \circ \tuple{ i_{M_{m_B}}^{S^{n_B} \oplus M^{m_B}} }_*
            \end{smallmatrix}
            \\
            \begin{smallmatrix}
                \tuple{ p_{S^{n_A} \oplus M^{m_A}}^{M_1} }^* \circ \tuple{ i_{S_1}^{S^{n_B} \oplus M^{m_B}} }_* &
                \tuple{ p_{S^{n_A} \oplus M^{m_A}}^{M_1} }^* \circ \tuple{ i_{S_2}^{S^{n_B} \oplus M^{m_B}} }_* &
                \dots &
                \tuple{ p_{S^{n_A} \oplus M^{m_A}}^{M_{m_A}} }^* \circ \tuple{ i_{S_{n_B}}^{S^{n_B} \oplus M^{m_B}} }_*
            \end{smallmatrix}
            \\
            \begin{smallmatrix}
                \tuple{ p_{S^{n_A} \oplus M^{m_A}}^{M_1} }^* \circ \tuple{ i_{M_1}^{S^{n_B} \oplus M^{m_B}} }_* &
                \tuple{ p_{S^{n_A} \oplus M^{m_A}}^{M_1} }^* \circ \tuple{ i_{M_2}^{S^{n_B} \oplus M^{m_B}} }_* &
                \dots &
                \tuple{ p_{S^{n_A} \oplus M^{m_A}}^{M_{m_A}} }^* \circ \tuple{ i_{M_{m_B}}^{S^{n_B} \oplus M^{m_B}} }_*
            \end{smallmatrix}
        \biggr)
    \end{multline*}
\end{remark}

\begin{remark} \label{rem:F_properties}
    Given the following commutative diagram by the definition of \( F \)
    \begin{center}
        \begin{tikzpicture}
            \diagram{m}{1cm}{1cm} {
                A \& B \\
                S^{n_A} \oplus M^{m_A} \& S^{n_B} \oplus M^{m_B} \\
            };

            \draw[math]
                (m-1-1) edge node {f} (m-1-2)
                    edge node {\psi_A} node[marking, below] {\sim} (m-2-1)
                (m-1-2) edge node {\psi_B} node[marking, below] {\sim} (m-2-2)

                (m-2-1) edge node {F(f)} (m-2-2);
        \end{tikzpicture}
    \end{center}
    Applying the functor \( \Mc(S^j \oplus M^k, -) \) to this, one gets the following commutative diagram
    \begin{center}
        \begin{tikzpicture}
            \diagram{m}{1cm}{2cm} {
                \Mc(S^j \oplus M^k, A) \& \Mc(S^j \oplus M^k, B) \\
                \Mc\tuple{S^j \oplus M^k, S^{n_A} \oplus M^{m_A} } \& \Mc\tuple{S^j \oplus M^k, S^{n_B} \oplus M^{m_B} } \\
            };

            \draw[math]
                (m-1-1) edge node {f_*} (m-1-2)
                    edge node {(\psi_A)_*} node[marking, below] {\sim} (m-2-1)
                (m-1-2) edge node {(\psi_B)_*} node[marking, below] {\sim} (m-2-2)

                (m-2-1) edge node {F(f)_*} (m-2-2);
        \end{tikzpicture}
    \end{center}
    Expanding downwards (and flipping it over to make it fit), using \autoref{rem:big_iso}, one gets the following commutative diagram
    \begin{center}
        \begin{tikzpicture}[every node/.style={scale=0.93}] \label{tikz:f_star}
            \diagram{m}{1cm}{1cm} {
                \Mc\tuple{S^j \oplus M^k, S^{n_A} \oplus M^{m_A} } \& \Mc\tuple{S, S}^{j n_A} \oplus \Mc\tuple{S, M}^{j m_A} \oplus \Mc\tuple{M, S}^{k n_A} \oplus \Mc\tuple{M, M}^{k m_A} \\
                \Mc\tuple{S^j \oplus M^k, S^{n_B} \oplus M^{m_B} } \& \Mc\tuple{S, S}^{j n_B} \oplus \Mc\tuple{S, M}^{j m_B} \oplus \Mc\tuple{M, S}^{k n_B} \oplus \Mc\tuple{M, M}^{k m_B} \\
            };

            \draw[math]
                (m-1-1) edge node {\phi_A} node[marking, below] {\sim} (m-1-2)
                    edge node {F(f)_*} (m-2-1)
                (m-1-2) edge node {\phi_B \circ F(f)_* \circ \phi_A^{-1}} (m-2-2)
                    
                (m-2-1) edge node {\phi_B} node[marking, below] {\sim} (m-2-2);
        \end{tikzpicture}
    \end{center}
    If one were to calculate \( \phi_B \circ F(f)_* \circ \phi_A^{-1} \), one would get something like This
    \[
        \phi_B \circ F(f)_* \circ \phi_A^{-1} =
        \begin{pmatrix}
            A & B & C & D \\
            E & F & G & H \\
            I & J & K & L \\
            M & N & O & P
        \end{pmatrix}
    \]
    Where these ``blocks'': \( A, B, C, D \), etc. corresponds to what \( F(f)_* \) does on the part that goes from one main summand in the domain to another main summand in the codomain.
    
    But checking if \( F(f)_* \) is zero by looking at the every ``block'' of \( \phi_B \circ F(f)_* \circ \phi_A^{-1} \) is overkill, since many of the elements in the big matrix are actually zero. This is because \( F(f)_* \) is simply post-composing while there is no pre-composing going on.
    
    One can see that many blocks are all zero, since there is no part of \( F(f)_* \) that could possibly go from e.g. \( \Mc(S, S) \) to \( \Mc(M, S) \), since it is only post composing. This makes \( C, D, G \) and \( H \) as well as \( I, J, M \) and \( N \), all zero. But in addition to making half of the blocks completely zero, most of the non-zero blocks themselves are zero, which will be explored in the next paragraph.
    
    Let \( \tilde{A}, \tilde{B} \in \set{ S, M } \) and let \( a, b \in \Nb \) (such that it makes sense below)

    Then define
    \[
        F(f)_{\tilde{A}_a}^{\tilde{B}_b} := p_{S^{n_B} \oplus M^{m_B}}^{\tilde{B}_b} \circ F(f) \circ i_{\tilde{A}_a}^{S^{n_A} \oplus M^{m_A}}
    \]
    and
    \[
        \iota_{\tilde{A}_a}^{\tilde{B}_b} := p_{S^j \oplus M^k}^{\tilde{B}_b} \circ i_{\tilde{A}_a}^{S^j \oplus M^k}.
    \]

    Then take for example block \( A \), it would look something like this
    \[
        A =
        \begin{psmallmatrix}
            \tuple{ \iota_{S_1}^{S_1} }^* \circ \tuple{ F(f)_{S_1}^{S_1} }_* &
            \tuple{ \iota_{S_1}^{S_1} }^* \circ \tuple{ F(f)_{S_2}^{S_1} }_* &
            \dots &
            \tuple{ \iota_{S_1}^{S_j} }^* \circ \tuple{ F(f)_{S_{n_A}}^{S_1} }_* \\
            \tuple{ \iota_{S_1}^{S_1} }^* \circ \tuple{ F(f)_{S_1}^{S_2} }_* &
            \tuple{ \iota_{S_1}^{S_1} }^* \circ \tuple{ F(f)_{S_2}^{S_2} }_* &
            \dots &
            \tuple{ \iota_{S_1}^{S_j} }^* \circ \tuple{ F(f)_{S_{n_A}}^{S_2} }_* \\
            \vdots &
            \vdots &
            \ddots &
            \vdots \\
            \tuple{ \iota_{S_j}^{S_1} }^* \circ \tuple{ F(f)_{S_1}^{S_{n_B}} }_* &
            \tuple{ \iota_{S_j}^{S_1} }^* \circ \tuple{ F(f)_{S_2}^{S_{n_B}} }_* &
            \dots &
            \tuple{ \iota_{S_j}^{S_j} }^* \circ \tuple{ F(f)_{S_{n_A}}^{S_{n_B}} }_*
        \end{psmallmatrix}
    \]
    However, one has that \( \iota_{\tilde{A}_a}^{\tilde{B}_b} \) is zero unless \( a = b \) and \( A = B \), and the identity map otherwise. This makes most of the entries in \( A \), except for certain ``blocks'' on the diagonal, equal to \( 0 \).

    In fact, one can write
    \begin{multline*}
        A = \oplus_{i = 1}^{j}
        \begin{psmallmatrix}
            \tuple{ \iota_{S_i}^{S_i} }^* \circ \tuple{ F(f)_{S_1}^{S_1} }_* &
            \tuple{ \iota_{S_i}^{S_i} }^* \circ \tuple{ F(f)_{S_2}^{S_1} }_* &
            \dots &
            \tuple{ \iota_{S_i}^{S_i} }^* \circ \tuple{ F(f)_{S_{n_A}}^{S^1} }_* \\
            \tuple{ \iota_{S_i}^{S_i} }^* \circ \tuple{ F(f)_{S_1}^{S_2} }_* &
            \tuple{ \iota_{S_i}^{S_i} }^* \circ \tuple{ F(f)_{S_2}^{S_2} }_* &
            \dots &
            \tuple{ \iota_{S_i}^{S_i} }^* \circ \tuple{ F(f)_{S_{n_A}}^{S^2} }_* \\
            \vdots & \vdots & \ddots & \vdots \\
            \tuple{ \iota_{S_i}^{S_i} }^* \circ \tuple{ F(f)_{S_1}^{S_{n_B}} }_* &
            \tuple{ \iota_{S_i}^{S_i} }^* \circ \tuple{ F(f)_{S_2}^{S_{n_B}} }_* &
            \dots &
            \tuple{ \iota_{S_i}^{S_i} }^* \circ \tuple{ F(f)_{S_{n_A}}^{S^{n_B}} }_* \\
        \end{psmallmatrix} \\
        = \oplus_{i = 1}^{j}
        \begin{psmallmatrix}
            \tuple{ F(f)_{S_1}^{S_1} }_* &
            \tuple{ F(f)_{S_2}^{S_1} }_* &
            \dots &
            \tuple{ F(f)_{S_{n_A}}^{S^1} }_* \\
            \tuple{ F(f)_{S_1}^{S_2} }_* &
            \tuple{ F(f)_{S_2}^{S_2} }_* &
            \dots &
            \tuple{ F(f)_{S_{n_A}}^{S^2} }_* \\
            \vdots & \vdots & \ddots & \vdots \\
            \tuple{ F(f)_{S_1}^{S_{n_B}} }_* &
            \tuple{ F(f)_{S_2}^{S_{n_B}} }_* &
            \dots &
            \tuple{ F(f)_{S_{n_A}}^{S^{n_B}} }_* \\
        \end{psmallmatrix}
    \end{multline*}
    For \( A, B, C \in \set{S, M} \) and
    \[
        a = 
        \begin{cases}
            j & A = S \\
            k & A = M
        \end{cases},
        b =
        \begin{cases}
            n_A & B = S \\
            m_A & B = M
        \end{cases},
        c =
        \begin{cases}
            n_B & C = S \\
            m_B & C = M
        \end{cases},
    \]
    let
    \[
        L_{A, B, C} = \oplus_{i = 1}^{a}
        \begin{psmallmatrix}
            \tuple{ F(f)_{B_1}^{C_1} }_* &
            \tuple{ F(f)_{B_2}^{C_1} }_* &
            \dots &
            \tuple{ F(f)_{B_b}^{C_1} }_* \\
            \tuple{ F(f)_{B_1}^{C_2} }_* &
            \tuple{ F(f)_{B_2}^{C_2} }_* &
            \dots &
            \tuple{ F(f)_{B_b}^{C_2} }_* \\
            \vdots & \vdots & \ddots & \vdots \\
            \tuple{ F(f)_{B_1}^{C_c} }_* &
            \tuple{ F(f)_{B_2}^{C_c} }_* &
            \dots &
            \tuple{ F(f)_{B_b}^{C_c} }_* \\
        \end{psmallmatrix}.
    \]
    Then one gets that
    \[
        \phi_B \circ F(f)_* \circ \phi_A^{-1} =
        \begin{pmatrix}
            L_{S, S, S} & L_{S, M, S} \\
            L_{S, S, M } & L_{S, M, M}
        \end{pmatrix}
        \oplus
        \begin{pmatrix}
            L_{M, S, S} & L_{M, M, S} \\
            L_{M, S, M} & L_{M, M, M}
        \end{pmatrix}.
    \]
\end{remark}

\begin{remark} \label{rem:phi_and_L_connection}
    One may wonder if there is a more direct connection between the \( \phi \) in \autoref{rem:big_iso}, and the map
    \[
        \phi_B \circ F(f)_* \circ \phi_A^{-1} =
        \begin{pmatrix}
            L_{S, S, S} & L_{S, M, S} \\
            L_{S, S, M } & L_{S, M, M}
        \end{pmatrix}
        \oplus
        \begin{pmatrix}
            L_{M, S, S} & L_{M, M, S} \\
            L_{M, S, M} & L_{M, M, M}
        \end{pmatrix}.
    \]
    from \autoref{rem:F_properties}, and in fact, there is one major similarity which will be used in the proceeding examples.

    One can write
    \[
        \phi \tuple{F(f)}
        =
        \begin{psmallmatrix}
            \tuple{ p_{S^{n_B} \oplus M^{m_B}}^{S_1} }_* \circ \tuple{ i_{S_1}^{S^{n_A} \oplus M^{m_A}} }^* \\
            \tuple{ p_{S^{n_B} \oplus M^{m_B}}^{S_2} }_* \circ \tuple{ i_{S_1}^{S^{n_A} \oplus M^{m_A}} }^* \\
            \vdots \\
            \tuple{ p_{S^{n_B} \oplus M^{m_B}}^{S_{n_B}} }_* \circ \tuple{ i_{S_{n_A}}^{S^{n_A} \oplus M^{m_A}} }^* \\
            \tuple{ p_{S^{n_B} \oplus M^{m_B}}^{M_1} }_* \circ \tuple{ i_{S_1}^{S^{n_A} \oplus M^{m_A}} }^* \\
            \tuple{ p_{S^{n_B} \oplus M^{m_B}}^{M_2} }_* \circ \tuple{ i_{S_1}^{S^{n_A} \oplus M^{m_A}} }^* \\
            \vdots \\
            \tuple{ p_{S^{n_B} \oplus M^{m_B}}^{M_{m_B}} }_* \circ \tuple{ i_{S_{n_A}}^{S^{n_A} \oplus M^{m_A}} }^* \\
            \tuple{ p_{S^{n_B} \oplus M^{m_B}}^{S_1} }_* \circ \tuple{ i_{M_1}^{S^{n_A} \oplus M^{m_A}} }^* \\
            \tuple{ p_{S^{n_B} \oplus M^{m_B}}^{S_2} }_* \circ \tuple{ i_{M_1}^{S^{n_A} \oplus M^{m_A}} }^* \\
            \vdots \\
            \tuple{ p_{S^{n_B} \oplus M^{m_B}}^{S_{n_B}} }_* \circ \tuple{ i_{M_{m_A}}^{S^{n_A} \oplus M^{m_A}} }^* \\
            \tuple{ p_{S^{n_B} \oplus M^{m_B}}^{M_1} }_* \circ \tuple{ i_{M_1}^{S^{n_A} \oplus M^{m_A}} }^* \\
            \tuple{ p_{S^{n_B} \oplus M^{m_B}}^{M_2} }_* \circ \tuple{ i_{M_1}^{S^{n_A} \oplus M^{m_A}} }^* \\
            \vdots \\
            \tuple{ p_{S^{n_B} \oplus M^{m_B}}^{M_{m_B}} }_* \circ \tuple{ i_{M_{m_A}}^{S^{n_A} \oplus M^{m_A}} }^*
        \end{psmallmatrix}
        \tuple{F(f)}
        =
        \begin{psmallmatrix}
            F(f)_{S_1}^{S_1} \\
            F(f)_{S_1}^{S_2} \\
            \vdots \\
            F(f)_{S_{n_B}}^{S_{n_A}} \\
            F(f)_{S_1}^{M_1} \\
            F(f)_{S_1}^{M_2} \\
            \vdots \\
            F(f)_{S_{n_B}}^{M_{m_A}} \\
            F(f)_{M_1}^{S_1} \\
            F(f)_{M_1}^{S_2} \\
            \vdots \\
            F(f)_{M_{m_B}}^{S_{n_A}} \\
            F(f)_{M_1}^{M_1} \\
            F(f)_{M_1}^{M_2} \\
            \vdots \\
            F(f)_{M_{m_B}}^{M_{m_A}}
        \end{psmallmatrix}
    \]
    In other words, one can think of taking \( \phi(F(f)) \) as ``stretching out'' the elements of \( \phi_B \circ F(f)_* \circ \phi_A^{-1} \).
\end{remark}


\subsubsection{Examples}

\begin{definition} \label{def:unholy}
    Let the functor \( F \) be as in \autoref{thm:F_functor}.  Let, \( \phi \) be as in \autoref{rem:big_iso}, and use the notation from \autoref{rem:F_properties}.

    Define the set \( P_S \) as follows:

    For any object \( A, B \in \Mc \), and for any \( f \in \Mc\tuple{A, B} \).

    Then \( f \in P_S \iff \)

    All of the following are true:
    \begin{enumerate}
        \item \( F(f)_{S_a}^{S_b} = 0 \) for all \( a, b \).
        \item \( F(f)_{S_a}^{M_b} = 0 \) for all  \( a, b \).
        \item \( F(f)_{M_a}^{M_b} \in \set{0, \cdot(\pm (g - 1))} \) for all \( a, b \).
    \end{enumerate}
\end{definition}

% TODO: Mention the connection to phi as mentioned in the remark above? Might be overkill.
\begin{remark}
    Definitely need to remark on the previous definition.... 
    
    Any morphism can not have any component from S to S, or S to M, as well as they can only have one certain component from M to M.

    TODO: Fix
\end{remark}

% TODO: Possible to simplify proof by saying that postcomposing with a map in Nc is the same as precomposing with any map from the hom-sets?
% TODO: Seems so obvious that there should be a simpler way to do things.
\begin{example}
    Let \( \Pc = \set{ S^n \mid n \in \Nb } \). Let \( \Nc = P_S \).

    Then \( \tuple{ \Pc, \Nc} \) is a projective class in \( \Mc \).
\end{example}
\begin{proof}
    Need to show that \( \tuple{ \Pc, \Nc } \) satisfies the three properties in \autoref{def:projective_class}.

    \begin{enumerate}
        \item {
            \( \tuple{ \Rightarrow } \) Let \( f \in \Nc \).

            If \( f = 0 \), then the statement is true.

            Assume \( f \neq 0 \). Then \( f \in \Mc\tuple{A, B} \) for two non-zero modules \( A, B \in \Mc \), and satisfying \autoref{def:unholy}.
            
            Let \( \tilde{P} \in \Pc \)
            
            If \( \tilde{P} = 0 \), then the statement is true.

            Assume \( \tilde{P} = S^j \) for some \( j \in \Nb \).

            Then from \autoref{thm:hom_direct_sum_map_nice}, one gets the following commutative diagram
            \begin{center}
                \begin{tikzpicture}
                    \diagram{m}{1cm}{1cm} {
                        \Mc\tuple{S^j, A} \& \Mc\tuple{S^j, B} \\
                        \Mc\tuple{S, A}^j \& \Mc\tuple{S, B}^j \\
                    };

                    \draw[math]
                        (m-1-1) edge node {f_*} (m-1-2)
                            edge node[marking, above] {\sim} (m-2-1)
                        (m-1-2) edge node[marking, above] {\sim} (m-2-2)

                        (m-2-1) edge node {(f_*)^j} (m-2-2);
                \end{tikzpicture}
            \end{center}
            It suffices to check that \( f_*: \Mc\tuple{S, A} \to \Mc\tuple{S, B} \) is zero.

            Using \autoref{rem:F_properties} with \( j = 1 \) and \( k = 0 \), one gets
            % \[
            %     \phi_A^{-1} = \begin{pmatrix}
            %         \tuple{ i_{S_1}^{S^{n_A} \oplus M^{m_A}} }_* &
            %         \dots &
            %         \tuple{ i_{S_{n_A}}^{S^{n_A} \oplus M^{m_A}} }_* &
            %         \tuple{ i_{M_1}^{S^{n_A} \oplus M^{m_A}} }_* &
            %         \dots &
            %         \tuple{ i_{M_{m_A}}^{S^{n_A} \oplus M^{m_A}} }_*
            %     \end{pmatrix}
            % \]
            % and
            % \[
            %     \phi_B = \begin{pmatrix}
            %         \tuple{ p_{S^{n_B} \oplus M^{m_B}}^{S_1} }_* \\
            %         \vdots \\
            %         \tuple{ p_{S^{n_B} \oplus M^{m_B}}^{S_{n_B}} }_* \\
            %         \tuple{ p_{S^{n_B} \oplus M^{m_B}}^{M_1} }_* \\
            %         \vdots \\
            %         \tuple{ p_{S^{n_B} \oplus M^{m_B}}^{M_{m_B}} }_*
            %     \end{pmatrix}.
            % \]
            This gives
            \[
                \phi_B \circ F(f)_* \circ \phi_A^{-1} =
                \begin{pmatrix}
                    L_{S, S, S} & L_{S, M, S} \\
                    L_{S, S, M} & L_{S, M, M}
                \end{pmatrix}
            \]
            where
            \[
                L_{S, S, S} =
                \begin{pmatrix}
                    \tuple{ F(f)_{S_1}^{S_1} }_* &
                    \dots &
                    \tuple{ F(f)_{S_{n_A}}^{S_1} }_* \\
                    \vdots & \ddots & \vdots \\
                    \tuple{ F(f)_{S_1}^{S_{n_B}} }_* &
                    \dots &
                    \tuple{ F(f)_{S_{m_A}}^{S_{n_B}} }_* \\
                \end{pmatrix},
            \]
            \[
                L_{S, M, S} =
                \begin{pmatrix}
                    \tuple{ F(f)_{M_1}^{S_1} }_* &
                    \dots &
                    \tuple{ F(f)_{M_{m_A}}^{S_1} }_* \\
                    \vdots & \ddots & \vdots \\
                    \tuple{ F(f)_{M_1}^{S_{n_B}} }_* &
                    \dots &
                    \tuple{ F(f)_{M_{m_A}}^{S_{n_B}} }_* \\
                \end{pmatrix},
            \]
            \[
                L_{S, S, M} =
                \begin{pmatrix}
                    \tuple{ F(f)_{S_1}^{M_1} }_* &
                    \dots &
                    \tuple{ F(f)_{S_{n_A}}^{M_1} }_* \\
                    \vdots & \ddots & \vdots \\
                    \tuple{ F(f)_{S_1}^{M_{m_B}} }_* &
                    \dots &
                    \tuple{ F(f)_{S_{n_A}}^{M_{m_B}} }_* \\
                \end{pmatrix},
            \]
            and
            \[
                L_{S, M, M} =
                \begin{pmatrix}
                    \tuple{ F(f)_{M_1}^{M_1} }_* &
                    \dots &
                    \tuple{ F(f)_{M_{m_A}}^{M_1} }_* \\
                    \vdots & \ddots & \vdots \\
                    \tuple{ F(f)_{M_1}^{M_{m_B}} }_* &
                    \dots &
                    \tuple{ F(f)_{M_{m_A}}^{M_{m_B}} }_* \\
                \end{pmatrix}.
            \]
            But from the definition of \( P_S \), using \autoref{rem:phi_and_L_connection} one has that
            \[
                \phi \circ F(f) =
                \begin{pmatrix}
                    F(f)_{S_1}^{S_1} \\
                    \vdots \\
                    F(f)_{S_{n_A}}^{S_{n_A}} \\
                    F(f)_{S_1}^{M_1} \\
                    \vdots \\
                    F(f)_{S_{n_A}}^{M_{m_A}} \\
                    F(f)_{M_1}^{S_1} \\
                    \vdots \\
                    F(f)_{M_{m_A}}^{S_{n_A}} \\
                    F(f)_{M_1}^{M_1} \\
                    \vdots \\
                    F(f)_{M_{m_A}}^{M_{m_A}}
                \end{pmatrix}
                =
                \begin{pmatrix}
                    0 \\
                    \vdots \\
                    0 \\
                    0 \\
                    \vdots \\
                    0 \\
                    F(f)_{M_1}^{S_1} \\
                    \vdots \\
                    F(f)_{M_{m_A}}^{S_{n_A}} \\
                    \set{0, \cdot(\pm(g - 1))} \\
                    \vdots \\
                    \set{0, \cdot(\pm(g - 1))}
                \end{pmatrix}.
            \]
            This implies that
            \[
                L_{S, S, S} = 0, L_{S, S, M} = 0,
            \]
            and
            \[
                L_{S, M, M} =
                \begin{pmatrix}
                    \set{0, \cdot(\pm(g - 1))} & \dots & \set{0, \cdot(\pm(g - 1))} \\
                    \vdots & \ddots & \vdots \\
                    \set{0, \cdot(\pm(g - 1))} & \dots & \set{0, \cdot(\pm(g - 1))}
                \end{pmatrix}.
            \]
            Using this information about \( \phi_B \circ F(f)_* \circ \phi_A^{-1} \), observe its pointwise values for any 
            \[
                \tuple{g_1, \dots, g_{n_A}, h_1, \dots, h_{m_A}} \in \Mc(S, S)^{n_A} \oplus \Mc(S, M)^{m_A}.
            \]
            First note that by \autoref{thm:f_3c_3_nu}, one has that \( h_i = \set{0, \pm \nu} \) for all \( i \).

            Also note that by \autoref{lem:g-1_circ_nu_equals_zero} one has
            \[
                L_{S, M, M}
                \begin{pmatrix}
                    \set{0, \pm \nu} \\
                    \vdots \\
                    \set{0, \pm \nu}
                \end{pmatrix} = 0.
            \]
            And by \autoref{thm:f_3c_3_mu} one has that
            \[
                F(f)_{M_a}^{S_b} = p_{S^{n_B} \oplus M^{m_B}}^{S_b} \circ F(f) \circ i_{M_a}^{S^{n_A} \oplus M^{m_A}} \in \Mc(M, M) = \set{0, \pm \mu}
            \]
            And furthermore by \autoref{thm:f_3c_3_mu_circ_nu_zero} this implies that
            \[
                L_{S, M, S}
                \begin{pmatrix}
                    \set{0, \pm \nu} \\
                    \vdots \\
                    \set{0, \pm \nu}
                \end{pmatrix} = 0.
            \]

            All in all, this implies
            \begin{multline*}
                \phi_B \circ F(f)_* \circ \phi_A^{-1} \tuple{g_1, \dots, g_{n_A}, h_1, \dots, h_{m_A}} \\
                =
                \begin{pmatrix}
                    L_{S, S, S} & L_{S, M, S} \\
                    L_{S, S, M} & L_{S, M, M}
                \end{pmatrix}
                \tuple{g_1, \dots, g_{n_A}, \set{0, \pm \nu}, \dots, \set{0, \pm \nu}} \\
                =
                \begin{pmatrix}
                    0 & L_{S, M, S}
                    \begin{psmallmatrix}
                        \set{0, \pm \nu} \\
                        \vdots \\
                        \set{0, \pm \nu}
                    \end{psmallmatrix} \\
                    0 & L_{S, M, M}
                    \begin{psmallmatrix}
                        \set{0, \pm \nu} \\
                        \vdots \\
                        \set{0, \pm \nu}
                    \end{psmallmatrix} \\
                \end{pmatrix} 
                = 0.
            \end{multline*}

            So by \autoref{rem:F_properties} as well as the vertical maps all being isomorphisms, it follows that \( f_* = 0 \).

            \( ( \Leftarrow ) \) Show this implication by a counter-positive argument.

            Assume \( f: A \to B \) with \( f \not\in \Nc \)

            Want to show that there exist a \( \tilde{P} \in \Pc \) such that \( f_*: \Mc(\tilde{P}, A) \to \Mc(\tilde{P}, B)\) is non-zero.

            Assume therefore that \( \tilde{P} = S \).

            Therefore, using \( j = 1 \) and the notation from \autoref{rem:F_properties}, want to show that \( \phi_B \circ F(f)_* \circ \phi_A^{-1} \) is pointwise non-zero.

            In order to prove this, split the cases up by which point in \autoref{def:unholy} that we assume \( f \) not to fulfill:
            \begin{enumerate}
                \item {
                    Assume that there is some \( a, b \in \Nb \) such that
                    \[
                        F(f)_{S_a}^{S_b}
                    \]
                    is non-zero.

                    Then there is a \( (n_A + m_A) \)-tuple that has all zeroes, except for \( \Id_S \) in the \( a \)-th coordinate. I.e. \( \alpha = \tuple{0, \dots, 0, \Id_S, 0, \dots, 0} \). Such that
                    \[
                        \phi_B \circ F(f)_* \circ \phi_A^{-1} \tuple{\alpha}
                        =
                        \begin{pmatrix}
                            L_{S, S, S} & L_{S, M, S} \\
                            L_{S, S, M} & L_{S, M, M}
                        \end{pmatrix}
                        \tuple{0, \dots, 0, \Id_S, 0, \dots, 0}
                        = \beta
                    \]
                    where in the \( b \)-th coordinate of \( \beta \) there is a non-zero term
                    \[
                        F(f)_{S_a}^{S_b} \circ \Id_S = F(f)_{S_a}^{S_b}.
                    \]
                    Then \( f \not\in \Nc \). 
                }
                \item {
                    Assume that there is some \( a, b \in \Nb \) such that
                    \[
                        F(f)_{S_a}^{M_b}
                    \]
                    is non-zero.

                    Then there is a \( (n_A + m_A) \)-tuple that has all zeroes, except for \( \Id_S \) in the \( a \)-th coordinate. I.e. \( \alpha = \tuple{0, \dots, 0, \Id_S, 0, \dots, 0} \). Such that
                    \[
                        \phi_B \circ F(f)_* \circ \phi_A^{-1} \tuple{\alpha}
                        =
                        \begin{pmatrix}
                            L_{S, S, S} & L_{S, M, S} \\
                            L_{S, S, M} & L_{S, M, M}
                        \end{pmatrix}
                        \tuple{0, \dots, 0, \Id_S, 0, \dots, 0}
                        = \beta
                    \]
                    where in the \( ( n_B + b ) \)-th coordinate of \( \beta \) there is a non-zero term
                    \[
                        F(f)_{S_a}^{M_b} \circ \Id_S = F(f)_{S_a}^{M_b}
                    \]
                    Then \( f \not\in \Nc \).
                }
                \item {
                    Assume that there is some \( a, b \in \Nb \) such that
                    \[
                        F(f)_{M_a}^{M_b} \not\in \set{0, \pm(g - 1)}
                    \]

                    By \autoref{lem:only_non_surjective_M_to_M} that means that \( F(f)_{M_a}^{M_b} \) is an isomorphism.

                    Then there is an \( (n_A + m_A) \)-tuple that has all zeroes, except for \( \nu \) in the \( (n_A + a) \)-th coordinate. I.e. \( \alpha = \tuple{0, \dots, 0, \nu, 0, \dots, 0} \). Such that
                    \[
                        \phi_B \circ F(f)_* \circ \phi_A^{-1} \tuple{\alpha}
                        =
                        \begin{pmatrix}
                            L_{S, S, S} & L_{S, M, S} \\
                            L_{S, S, M} & L_{S, M, M}
                        \end{pmatrix}
                        \tuple{0, \dots, 0, \nu, 0, \dots, 0}
                        = \beta
                    \]
                    where in the \( (n_B + b) \)-th coordinate of \( \beta \) there is a term
                    \[
                        F(f)_{M_a}^{M_b} \circ \nu
                    \]
                    that is non-zero because it is a non-zero morphism composed with an isomorphism.

                    Therefore \( f \not\in \Nc \).
                }
            \end{enumerate} 
        }
        \item {
            \( (\Rightarrow) \) Let \( \tilde{P} \in \Pc \).

            Let \( A, B \in \Mc \), and let \( f \in \Mc(A, B) \intersect \Nc \). Then by point 1, \( (\Rightarrow) \) previously, one gets that \( f_* = 0 \).
            
            \( (\Leftarrow) \) Want to show this by a counter positive argument.

            Assume \( \tilde{P} \not\in \Pc \). That implies there is some \( j \in \Nb_0 \) and \( k \in \Nb \) such that \( \tilde{P} = S^j \oplus M^k \).

            Want to find some \( f \in \Tc(A, B) \intersect \Nc \) such that
            \[
                f_*: \Tc(P, A) \to \Tc(P, B)
            \]
            is non-zero.

            From \autoref{rem:F_properties}, it is sufficient to show that the map \( \phi_B \circ F(f)_* \circ \phi_A^{-1} \) is non-zero.

            Let \( A \cong B \cong M \) and let \( F(f) = \cdot(g - 1) \). Then \( n_A = n_B = 0 \) and \( m_A = m_B = 1 \) gives that
            \[
                \phi_B \circ F(f)_* \circ \phi_A^{-1} =
                \begin{pmatrix}
                    L_{S, S, S} & L_{S, M, S} \\
                    L_{S, S, M } & L_{S, M, M}
                \end{pmatrix}
                \oplus
                \begin{pmatrix}
                    L_{M, S, S} & L_{M, M, S} \\
                    L_{M, S, M} & L_{M, M, M}
                \end{pmatrix}
            \]
            turns into
            \[
                \phi_B \circ F(f)_* \circ \phi_A^{-1}
                = L_{S, M, M} \oplus L_{M, M, M}
                = \tuple{ \oplus_{i=1}^j F(f)_* }
                \oplus
                \tuple{ \oplus_{i=1}^k F(f)_* }.
            \]
            Also, by assumption,
            \[
                F(f) = \cdot(g-1) \in \set{0, \cdot( \pm(g - 1) )} \subseteq \Nc.
            \]
            However, look at the \( (j + k) \)-tuple \( \alpha = \tuple{ 0, \dots, 0, \Id_M, 0, \dots, 0 } \), that is only non-zero at coordinate \( j + 1 \).
            
            Then
            \[
                \phi_B \circ F(f)_* \circ \phi_A^{-1} (\alpha)
                =
                \tuple{ \oplus_{i=1}^j (\cdot(g-1))_* }
                \oplus
                \tuple{ \oplus_{i=1}^k (\cdot(g-1))_* }
                (\alpha)
                = \beta
            \]
            First of all, applying \( \phi_B \circ F(f)_* \circ \phi_A^{-1} \) to \( \alpha \) is well defined, since while \( j \) could be zero, \( k \) is always non-zero by the assumption that \( P \not\in \Pc \). In addition, coordinate \( j + 1 \) to \( j + k \) are all maps from \( M \) to \( M \).

            Secondly, it follows that the \( j + 1 \)-th coordiate of \( \beta \) is \( \cdot(g-1) \) which is a non-zero map. And hence, the map \( \phi_B \circ F(f)_* \circ \phi_A^{-1} \) is non-zero.
        }
        \item {
            For every \( X \in \Tc \) one has that \( X \cong S^j \oplus M^k \) for some \( j, k \in \Nb_0 \).

            Firstly, note that the following triangle is distinguished
            \begin{center}
                \begin{tikzpicture}
                    \diagram{m}{1cm}{1cm} {
                        S \& S \& 0 \& \Sigma(S) \\
                    };

                    \draw[math]
                        (m-1-1) edge node {\Id_s} (m-1-2)
                        (m-1-2) edge (m-1-3)
                        (m-1-3) edge (m-1-4);
                \end{tikzpicture}
            \end{center}
            And since distinguished triangles are closed under direct sums, it follows that the following triangle is also distinguished
            \begin{center}
                \begin{tikzpicture}
                    \diagram{m}{1cm}{1cm} {
                        S^j \& S^j \& 0 \& \Sigma(S)^j \\
                    };

                    \draw[math]
                        (m-1-1) edge node {( \Id_s )^j} (m-1-2)
                        (m-1-2) edge (m-1-3)
                        (m-1-3) edge (m-1-4);
                \end{tikzpicture}
            \end{center}
            And by \autoref{lem:s_m_s_distinguished} it follows that
            \begin{center}
                \begin{tikzpicture}
                    \diagram{m}{1cm}{1cm} {
                        S^k \& M^k \& S^k \& \Sigma(S)^k \\
                    };

                    \draw[math]
                        (m-1-1) edge node {( \nu )^k} (m-1-2)
                        (m-1-2) edge node {( \mu )^k} (m-1-3)
                        (m-1-3) edge node {( \nu )^k} (m-1-4);
                \end{tikzpicture}
            \end{center}
            is also distinguished.

            Taking the direct summand of these distinguished triangles yields the following distinguished triangle
            \begin{center}
                \begin{tikzpicture}
                    \diagram{m}{1cm}{2cm} {
                        S^j \oplus S^k \& S^j \oplus M^k \& S^k \& \Sigma(S)^j \oplus \Sigma(S)^k \\
                    };

                    \draw[math]
                        (m-1-1) edge node {(\Id_S)^j \oplus ( \nu )^k} (m-1-2)
                        (m-1-2) edge node {
                            \begin{psmallmatrix}
                                0 & ( \mu )^k
                            \end{psmallmatrix}
                            } (m-1-3)
                        (m-1-3) edge node {
                            \begin{psmallmatrix}
                                0 \\
                                ( \nu )^k
                            \end{psmallmatrix}
                            } (m-1-4);
                \end{tikzpicture}
            \end{center}
            Need to check if this triangle satisfies the criteria.
            
            Firstly, \( S^j \oplus S^k \cong S^{j + k} \) is in \( \Pc \). Secondly need to check that the middle map 
            \( 
                \begin{psmallmatrix}
                    0 & ( \mu )^k
                \end{psmallmatrix} 
            \)
            is in \( \Nc \).

            Using \autoref{rem:phi_and_L_connection}, it follows that
            \[
                \phi(
                    \begin{psmallmatrix}
                        0 & ( \mu )^k
                    \end{psmallmatrix}
                    ) =
                \begin{psmallmatrix}
                    0 \\
                    \vdots \\
                    0 \\
                    0 \\
                    \vdots \\
                    0 \\
                    \mu \\
                    \vdots \\
                    \mu \\
                    0 \\
                    \vdots \\
                    0
                \end{psmallmatrix}
            \]
            Which satisfies the criteria that any map in \( \Nc \) need to follow.

            Therefore the following distinguished triangle satisfies all the criteria
            \begin{center}
                \begin{tikzpicture}
                    \diagram{m}{1cm}{2cm} {
                        S^{j + k} \& S^j \oplus M^k \& S^k \& \Sigma(S)^{j + k} \\
                    };

                    \draw[math]
                        (m-1-1) edge node {(\Id_S)^j \oplus ( \nu )^k} (m-1-2)
                        (m-1-2) edge node {
                            \begin{psmallmatrix}
                                0 & ( \mu )^k
                            \end{psmallmatrix}
                            } (m-1-3)
                        (m-1-3) edge node {
                            \begin{psmallmatrix}
                                0 \\
                                ( \nu )^k
                            \end{psmallmatrix}
                            } (m-1-4);
                \end{tikzpicture}
            \end{center}
        }
    \end{enumerate}
    Therefore, \( (\Pc, \Nc) \) is a projective class.
\end{proof}

\begin{definition} \label{def:P_M}
    Let the functor \( F \) be as in \autoref{thm:F_functor}, and let \( \phi \) be as in \autoref{rem:big_iso}. Furthermore use the notation from \autoref{rem:F_properties}.

    Let \( P_M \) be defined as follows:

    For any object \( A, B \in \Mc \), and for any \( f \in \Mc(A, B) \).

    Then \( f \in P_M  \iff \)

    All of the following are true:
    \begin{enumerate}
        \item \( F(f)_{S_a}^{S_b} = 0 \) for all \( a, b \).
        \item \( F(f)_{M_a}^{S_b} = 0 \) for all \( a, b \).
        \item \( F(f)_{M_a}^{M_b} = 0 \) for all \( a, b \).
    \end{enumerate}
\end{definition}

\begin{example}
    Let \( P_M \) be as in \autoref{def:P_M}.

    Let \( \Pc = \set{M^n \mid n \in \Nb_0} \) and let \( \Nc = P_M \).

    Then \( (\Pc, \Nc) \) is a projective class.
\end{example}
\begin{proof}
    Need to show that \( (\Pc, \Nc) \) satisfies the three properties in \autoref{def:projective_class}.

    \begin{enumerate}
        \item {
            \( (\Rightarrow) \) Let \( f \in \Nc \).

            If \( f = 0 \), then the statement is true. Therefore assume that \( f \neq 0 \).

            Then \( f \in \Mc\tuple{A, B} \) for two non-zero modules \( A, B \in \Mc \), satisfying \autoref{def:P_M}.

            Let \( \tilde{P} \in \Pc \).
            
            If \( \tilde{P} = 0 \), then the statement is true. Therefore assume that \( \tilde{P} \neq 0 \). This implies that \( \tilde{P} = M^k \) for some \( k \in \Nb \).

            Then from \autoref{thm:hom_direct_sum_map_nice} one gets the following commutative diagram
            \begin{center}
                \begin{tikzpicture}
                    \diagram{m}{1cm}{1cm} {
                        \Mc\tuple{M^k, A} \& \Mc\tuple{M^k, B} \\
                        \Mc\tuple{M, A}^k \& \Mc\tuple{M, B}^k \\
                    };

                    \draw[math]
                        (m-1-1) edge node {f_*} (m-1-2)
                            edge node[marking, above] {\sim} (m-2-1)
                        (m-1-2) edge node[marking, above] {\sim} (m-2-2)

                        (m-2-1) edge node {(f_*)^k} (m-2-2);
                \end{tikzpicture}
            \end{center}
            This implies that it suffices to check that \( f_*: \Mc\tuple{M, A} \to \Mc\tuple{M, B} \) is zero.

            Using \autoref{rem:F_properties} with \( j = 0 \) and \( k = 1 \) gives the following map
            \[
                \phi_B \circ F(f)_* \circ \phi_A^{-1} =
                \begin{pmatrix}
                    L_{M, S, S} & L_{M, M, S} \\
                    L_{M, S, M} & L_{M, M, M}
                \end{pmatrix}
            \]
            From the definition of \( \Nc \) it follows that \( L_{M, S, S} = 0, L_{M, M, S} = 0 \) and \( L_{M, M, M} = 0 \). Therefore, one can write
            \[
                \phi_B \circ F(f)_* \circ \phi_A^{-1} =
                \begin{pmatrix}
                    0 & 0 \\
                    L_{M, S, M} & 0
                \end{pmatrix}
            \]
            Looking at the defintion of \( L_{M, S, M} \) one can see
            \[
                L_{M, S, M} =
                \begin{pmatrix}
                    \tuple{ F(f)_{S_1}^{M_1} }_* &
                    \tuple{ F(f)_{S_2}^{M_1} }_* &
                    \dots &
                    \tuple{ F(f)_{S_{n_A}}^{M_1} }_* \\
                    \tuple{ F(f)_{S_1}^{M_2} }_* &
                    \tuple{ F(f)_{S_2}^{M_2} }_* &
                    \dots &
                    \tuple{ F(f)_{S_{n_A}}^{M_2} }_* \\
                    \vdots & \vdots & \ddots & \vdots \\
                    \tuple{ F(f)_{S_1}^{M_{m_B}} }_* &
                    \tuple{ F(f)_{S_2}^{M_{m_B}} }_* &
                    \dots &
                    \tuple{ F(f)_{S_{n_A}}^{M_{m_B}} }_* \\
                \end{pmatrix}
            \]
            By \autoref{thm:f_3c_3_nu} it follows that
            \[
                F(f)_{S_i}^{M_j} \in \Mc\tuple{S, M} = \set{0, \pm \nu}.
            \]
            And also by \autoref{thm:f_3c_3_mu} it follows that for any
            \[
                \alpha = \tuple{a_1, a_2, \dots, a_{n_A}, b_1, b_2, \dots, b_{m_A}} \in \Mc\tuple{M, S}^{n_A} \oplus \Mc\tuple{M, M}^{m_A}
            \]
            one has that \( a_1, a_2, \dots, a_{n_A} \in \Mc\tuple{M, S} = \set{ 0, \pm \mu } \).

            Therefore by \autoref{thm:f_3c_3_nu_circ_mu_zero} one has that
            \begin{multline*}
                \phi_B \circ F(f)_* \circ \phi_A^{-1}(\alpha) =
                \phi_B \circ F(f)_* \circ \phi_A^{-1} =
                \begin{pmatrix}
                    0 & 0 \\
                    L_{M, S, M} & 0
                \end{pmatrix}
                \tuple{a_1, a_2, \dots, a_{n_A}, b_1, b_2, \dots, b_{m_A}} \\
                =
                \begin{pmatrix}
                    0 \\
                    \vdots \\
                    0 \\
                    L_{M, S, M}
                    \begin{psmallmatrix}
                        \set{0, \pm \mu} \\
                        \vdots \\
                        \set{0, \pm \mu}
                    \end{psmallmatrix}
                \end{pmatrix}
                = 0
            \end{multline*}

            \( (\Leftarrow) \) it is easier to show this implication using a contrapositive argument.

            Assume \( f \in \Mc\tuple{A, B} \), but with \( f \not\in \Nc \).

            Then first of all \( f \neq 0 \) and so \( A, B \neq 0 \).

            Want to show that there exist some \( \tilde{P} \in \Pc \) such that
            \[
                f_*: \Mc\tuple{\tilde{P}, A} \to \Mc\tuple{\tilde{P}, B}
            \]
            is non-zero.

            Assume that \( \tilde{P} = M \).

            By \autoref{rem:F_properties} it is sufficient to show that \( \phi_B \circ F(f)_* \circ \phi_A^{-1} \) is non-zero.

            In order to prove this, split the cases up by which property in \autoref{def:P_M} that one assume that \( f \) does not fulfill:
            \begin{enumerate}
                \item {
                    Assume that \( F(f)_{S_a}^{S_b} \neq 0 \) for some \( a, b \).

                    That implies that 
                    \( 
                        L_{M, S, S} \neq 0 
                    \)
                    in the 
                    \( 
                        b, a
                    \)
                    -th coordinate.
                    Therefore for
                    \[
                        \alpha = \tuple{ 0, \dots 0, \mu, 0, \dots, 0 }
                    \]
                    a
                    \(
                        (n_a + m_a)
                    \)
                    -tuple that is entirely zero, except for in coordinate
                    \(
                        a
                    \)
                    where it is
                    \(
                        \mu
                    \).

                    Taking
                    \[
                        \phi_B \circ F(f)_* \circ \phi_A^{-1} (\alpha) =
                        \begin{pmatrix}
                            L_{M, S, S} & L_{M, M, S} \\
                            L_{M, S, M} & L_{M, M, M}
                        \end{pmatrix}
                        \tuple{ 0, \dots 0, \mu, 0, \dots, 0 }
                        = \beta
                    \]
                    one gets that the \( b \)-th coordinate of \( \beta \) has the value \( F(f)_{S_a}^{S_b} \circ \mu \).
                    
                    By assumption, \( F(f)_{S_a}^{S_b} \neq 0 \), and therefore by \autoref{lem:S-to-S} \( F(f)_{S_a}^{S_b} = \pm \Id_S \). However, this implies that \( F(f)_{S_a}^{S_b} \circ \mu = \mu \neq 0 \).
                    
                    And so \( \phi_B \circ F(f)_* \circ \phi_A^{-1} \) is non-zero.
                }
                \item {
                    Assume that there is some \( a, b \) such that \( F(f)_{M_a}^{S_b} \neq 0 \).

                    Let
                    \[
                        \alpha = \tuple{0, \dots, 0, \Id_M, 0, \dots, 0}
                    \]
                    be a \( ( n_a + m_A ) \)-tuple that is all zeroes, except for \( \Id_M \) in the \( ( n_A + a ) \)-th coordinate.

                    Then
                    \[
                        \phi_B \circ F(f)_* \circ \phi_A^{-1} (\alpha) =
                        \begin{pmatrix}
                            L_{M, S, S} & L_{M, M, S} \\
                            L_{M, S, M} & L_{M, M, M}
                        \end{pmatrix}
                        \tuple{ 0, \dots 0, \Id_M, 0, \dots, 0 }
                        = \beta
                    \]
                    where in the \( b \)-th coordinate it has the value \( F(f)_{M_a}^{S_b} \circ \Id_M = F(f)_{M_a}^{S_b} \), which is non-zero by assumption, and therefore it follows that \( f \not\in \Nc \).
                }
                \item {
                    Assume that there is some \( a, b \) such that \( F(f)_{M_a}^{M_b} \neq 0 \).

                    Let
                    \[
                        \alpha = \tuple{0, \dots, 0, \Id_M, 0, \dots, 0}
                    \]
                    be a \( ( n_a + m_A ) \)-tuple that is all zeroes, except for \( \Id_M \) in the \( ( n_A + a ) \)-th coordinate.

                    Then
                    \[
                        \phi_B \circ F(f)_* \circ \phi_A^{-1} (\alpha) =
                        \begin{pmatrix}
                            L_{M, S, S} & L_{M, M, S} \\
                            L_{M, S, M} & L_{M, M, M}
                        \end{pmatrix}
                        \tuple{ 0, \dots 0, \Id_M, 0, \dots, 0 }
                        = \beta
                    \]
                    where in the \( ( n_B + b ) \)-th coordinate it has the value \( F(f)_{M_a}^{M_b} \circ \Id_M = F(f)_{M_a}^{M_b} \), which is non-zero by assumption, and therefore it follows that \( f \not\in \Nc \).
                }
            \end{enumerate}
        }
        \item {
            \( ( \Rightarrow ) \) this is implied by point 1 \( (\Rightarrow) \).

            \( ( \Leftarrow ) \) want to show this by a contrapositive argument.

            Assume \( \tilde{P} \not\in \Pc \). Then there are some \( j \in \Nb \) and \( k \in \Nb_0 \) such that \( \tilde{P} = S^j \oplus M^k \).

            Want to show that there there exists some \( A, B, \in \Mc \) and \( f \in \Mc\tuple{A, B} \intersect \Nc \) such that
            \[
                f_*: \Mc\tuple{\tilde{P}, A} \to \Mc\tuple{\tilde{P}, B}
            \]
            is non-zero.

            Let \( A \cong S \) and \( B \cong M \) with \( F(f) = \nu \).

            Then one has that \( n_a = m_b = 1 \) and \( m_a = n_b = 0 \).

            This makes
            \begin{align*}
                \phi_B \circ F(f)_* \circ \phi_A^{-1} &=
                \begin{pmatrix}
                    L_{S, S, S} & L_{S, M, S} \\
                    L_{S, S, M } & L_{S, M, M}
                \end{pmatrix}
                \oplus
                \begin{pmatrix}
                    L_{M, S, S} & L_{M, M, S} \\
                    L_{M, S, M} & L_{M, M, M}
                \end{pmatrix} \\
                &= L_{S, S, M} \oplus L_{M, S, M} \\
                &= \tuple{\nu}_*^j \oplus \tuple{\nu}_*^k
            \end{align*}

            Let \( \alpha = \tuple{\Id_S, 0, \dots, 0} \) be a \( (j + k) \)-tuple where the only non-zero coordinate is the first one, which is \( \Id_S \).

            First of all \( \alpha \) is well-defined, since by assumption \( j > 0 \), and so the first element will always exist.

            Considering
            \[
                \phi_B \circ F(f)_* \circ \phi_A^{-1} \tuple{\alpha} =
                \tuple{\nu}_*^j \oplus \tuple{\nu}_*^k \tuple{\alpha} =
                \beta
            \]

            Where the first coordinate of \( \beta \) is
            \[
                \nu \circ \Id_S = \nu
            \]
            which is non-zero, and therefore \( \phi_B \circ F(f)_* \circ \phi_A^{-1} \neq 0 \).
        }
        \item {
            From \autoref{thm:f_3c_3_decomposition} one has that for any \( X \in \Mc \), there exist \( j, k \in \Nb \) such that \( X \cong S^j \oplus M^k \).

            First, note that the following triangle is distinguished
            \begin{center}
                \begin{tikzpicture}
                    \diagram{m}{1cm}{1cm} {
                        M \& M \& 0 \& \Sigma M \\
                    };

                    \draw[math]
                        (m-1-1) edge node {\Id_M} (m-1-2)
                        (m-1-2) edge (m-1-3)
                        (m-1-3) edge (m-1-4);
                \end{tikzpicture}
            \end{center}

            And taking the direct summand of a distinguished triangle is distinguished, therefore the following triangle is also distinguished
            \begin{center}
                \begin{tikzpicture}
                    \diagram{m}{1cm}{1cm} {
                        M^k \& M^k \& 0 \& (\Sigma M)^k \\
                    };

                    \draw[math]
                        (m-1-1) edge node {\tuple{\Id_M}^k} (m-1-2)
                        (m-1-2) edge (m-1-3)
                        (m-1-3) edge (m-1-4);
                \end{tikzpicture}
            \end{center}

            By \autoref{lem:s_m_s_distinguished} one has the following distinguished triangle
            \begin{center}
                \begin{tikzpicture}
                    \diagram{m}{1cm}{1cm} {
                        S \& M \& S \& \Sigma S \\
                    };

                    \draw[math]
                        (m-1-1) edge node {\nu} (m-1-2)
                        (m-1-2) edge node {\mu} (m-1-3)
                        (m-1-3) edge node {\nu} (m-1-4);
                \end{tikzpicture}
            \end{center}

            Shifting the above triangle yields the following distinguished triangle
            \begin{center}
                \begin{tikzpicture}
                    \diagram{m}{1cm}{1cm} {
                        \Sigma^{-1} S \& S \& M \& S \\
                    };

                    \draw[math]
                        (m-1-1) edge node {-\mu} (m-1-2)
                        (m-1-2) edge node {\nu} (m-1-3)
                        (m-1-3) edge node {\mu} (m-1-4);
                \end{tikzpicture}
            \end{center}

            Taking the direct summand of this distinguished triangle with itself \( j \) times, as well as using \autoref{lem:sigma_switch_s_m} by identifying \( \Sigma^{-1} S \cong M \) and \( S \cong \Sigma M \), yields the following distinguished triangle
            \begin{center}
                \begin{tikzpicture}
                    \diagram{m}{1cm}{1cm} {
                        M^j \& S^j \& M^j \& (\Sigma M)^j \\
                    };

                    \draw[math]
                        (m-1-1) edge node {\tuple{-\mu}^j} (m-1-2)
                        (m-1-2) edge node {\tuple{\nu}^j} (m-1-3)
                        (m-1-3) edge node {\tuple{\mu}^j} (m-1-4);
                \end{tikzpicture}
            \end{center}

            % TODO: Refrence the tikzpictures
            And finally, taking the direct summand of the two biggest distinguished triangles yields the following distinguished triangle
            \begin{center}
                \begin{tikzpicture}
                    \diagram{m}{1cm}{2cm} {
                        M^{j + k} \& S^j \oplus M^k \& M^j \& (\Sigma M)^{j + k} \\
                    };

                    \draw[math]
                        (m-1-1) edge node {\tuple{-\mu}^j \oplus \tuple{\Id_M}^k} (m-1-2)
                        (m-1-2) edge node {
                            \begin{psmallmatrix}
                                \tuple{\nu}^j & 0
                            \end{psmallmatrix}
                            } (m-1-3)
                        (m-1-3) edge node {
                            \begin{psmallmatrix}
                                \tuple{\mu}^j \\
                                0
                            \end{psmallmatrix}
                            } (m-1-4);
                \end{tikzpicture}
            \end{center}
            Need to check if this triangle satisfies the conditions.

            Firstly, \( M^{j + k} \) is in \( \Pc \).

            Secondly, need to check if \( 
                \begin{psmallmatrix}
                    \tuple{\nu}^j & 0
                \end{psmallmatrix}
            \) is in \( \Nc \).

            Uisng \autoref{rem:phi_and_L_connection}, it follows that
            \[
                \phi\tuple{
                    \begin{psmallmatrix}
                        \tuple{\nu}^j & 0
                    \end{psmallmatrix}
                }
                =
                \begin{psmallmatrix}
                    0 \\
                    \vdots \\
                    0 \\
                    \nu \\
                    \vdots \\
                    \nu \\
                    0 \\
                    \vdots \\
                    0 \\
                    0 \\
                    \vdots \\
                    0
                \end{psmallmatrix}
            \]
            which satisfies the criteria that any map in \( \Nc \) need to follow.
        }
    \end{enumerate}
    
    Therefore \( \tuple{\Pc, \Nc} \) is a projective class.
\end{proof}


\subsection{Counterexample from article (TODO)}
For \( R = kG \) a group algebra, one has that in \( \StMod(R) \), the toda bracket is \emph{not neccesarily} equal to the toda bracket with the negative triangulation on \( \StMod(R) \).

\begin{definition}
    Negative triangulation TODO
\end{definition}

Let \( R = \Fb_3C_3 \) with \( g \in C_3 \) a generator.


\addcontentsline{toc}{section}{References}
\bibliography{thesis}{}
\bibliographystyle{utcaps}

\end{document}
