\documentclass[a4paper, 12pt]{article}

% Right-justify text. Creates some overfull hbox-es, and works weirdly with microtype.
% \usepackage[document]{ragged2e}

% Make language-specific tweaks, like changing section/theorem words or adding more hyphonation points.
\usepackage[english]{babel}

% For å kunna skriva æøå i tekstar MERK: Blir automatisk ubrukeleg med lualatex og fontspec
% \usepackage[utf8]{inputenc}

\usepackage[T1]{fontenc}

% Adds hyphenation points and aims to improve paragraph rendering.
% Use [activate=false] to disable some features that causes issues with ragged2e. Not sure if setting this option disables every feature in the package entirely, or if it's still worth keeping then.
\usepackage{microtype}

% Enables memoization
% extract=no means that latexmk need to do the extracting itself. This is controlled by the latexmkrc file.
\usepackage[extract=no]{memoize}

% Uncomment the following to recompile every tikz-diagram.
% \mmzset{
%     recompile,
% }

% Fiksa margin
\usepackage[margin=2cm]{geometry}

% Fiksar datoformatet på tiitelen
\usepackage[ddmmyyyy]{datetime}

\usepackage{amssymb}

% For visse mattesymbol, typ \mathbb
\usepackage{amsmath}

% Bilete
\usepackage{graphicx}

% For kodesnuttar og resultat
% \usepackage{minted}

% Kan endra på korleis listar ser ut
\usepackage{enumitem}

% For autoref
\usepackage[hidelinks,colorlinks=true]{hyperref} 

% For fargar på ting ein referer til i autoref
\hypersetup{allcolors=[rgb]{0,0.31,0.62}}

% For not needing to compile twice with hyperref (?)
\usepackage{bookmark}

% For teorem, definisjon, bevis enviornments.
\usepackage{amsthm}

% For meir avanserte teoremkonstruksjonar.
\usepackage{thmtools}

% For psmallmatrix.
\usepackage{mathtools}

% For boksar rundt tekst.
\usepackage{tcolorbox}
\tcbuselibrary{skins} % For å ha meir fancy boksar, trengs for "enhanced".
\tcbuselibrary{breakable} % For å ha breakable boksar.

% For svgar
% \usepackage{svg}

% Set svg mappo
% \svgpath{svg/}

% Fjernar indents ved nye avsnitt, men gjer linjeavstanden kortare (Kanskje)
\usepackage{parskip} 

% Lualatex font greie
\usepackage{fontspec}

\usepackage[warnings-off={mathtools-colon,mathtools-overbracket}]{unicode-math}

% TikZ!
\usepackage{tikz}

% Set fontar som blir brukt
\setmathfont{Latin Modern Math} % Dette er standardfonten
\setmathfont[range=\setminus]{Asana Math} % Somehow setting this changes tikz-cd arrow style
% \setmainfont{Atkinson Hyperlegible}
% \setmainfont{GFS Neohellenic Math}
% \setmainfont{Fira Sans}
% \setmathfont{Fira Math}
% \setmathfont[range=\setminus]{Asana Math}

\usetikzlibrary{matrix}

\usetikzlibrary{cd}

% In order to remove 1 pixel line at end of equal arrows.
\usetikzlibrary{nfold}

% From quiver:
% `pathmorphing` is necessary to draw squiggly arrows.
\usetikzlibrary{decorations.pathmorphing}

% From quiver:
% `calc` is necessary to draw curved arrows.
\usetikzlibrary{calc}

% Externalize TikZ diagrams to save compilation time.
% NOTE: Doesn't work with tikz-cd.
% \usetikzlibrary{external}
% \tikzexternalize[prefix=tikz/]

% \usepackage{memoize}
% \mmzset{memo dir}

% Marius Thaule TikZ matrix.
\newcommand{\diagram}[3]{\matrix (#1) [matrix of math nodes,row
  sep={#2},column sep={#3},text height=1.5ex,text
  depth=0.25ex]}


% Set style of TikZ pictures to tikz-cd style.
\tikzset{every picture/.append style={commutative diagrams/every diagram}}
\tikzset{math/.style = {commutative diagrams/every arrow,
  commutative diagrams/every label,
  execute at begin node=\(, execute at end node=\)}
}

% Shortcuts to tikz-cd styles.
\tikzset{hook/.style = {commutative diagrams/hook}}
\tikzset{dashed/.style = {commutative diagrams/dashed}}
\tikzset{two heads/.style = {commutative diagrams/two heads}}
\tikzset{equal/.style = {commutative diagrams/equal, nfold}}
\tikzset{squiggly/.style = {commutative diagrams/squiggly}}
\tikzset{marking/.style = {commutative diagrams/marking}}

% From quiver:
% A TikZ style for curved arrows of a fixed height, due to AndréC.
\tikzset{curve/.style={settings={#1},to path={(\tikztostart)
    .. controls ($(\tikztostart)!\pv{pos}!(\tikztotarget)!\pv{height}!270:(\tikztotarget)$)
    and ($(\tikztostart)!1-\pv{pos}!(\tikztotarget)!\pv{height}!270:(\tikztotarget)$)
    .. (\tikztotarget)\tikztonodes}},
    settings/.code={\tikzset{quiver/.cd,#1}
        \def\pv##1{\pgfkeysvalueof{/tikz/quiver/##1}}},
    quiver/.cd,pos/.initial=0.35,height/.initial=0
}

% Needed for suspension style.
\usetikzlibrary{decorations.markings}

% Marks the arrows with a suspension style.
\tikzset{
  suspension/.style = {postaction = decorate,
      decoration = {
          markings,
          mark = at position 0.3 with {\draw[-] (0,-0.075) -- (0,0.075);}
      },
  },
}
    

% Ny type lista med ganske perfekt spacing
\newlist{plist}{enumerate}{5}
\setlist[plist]{align=left, itemindent = 0cm, labelsep = 0cm, labelindent = 0cm}
\setlist[plist,1]{label=\arabic*, font=\bf\Large}
\setlist[plist,2]{label*=.\arabic*, labelwidth=1.25cm, leftmargin=1.25cm}
\setlist[plist,3]{label*=.\arabic*, labelwidth=1.5cm, leftmargin=1.5cm}

% Teoremstil
\theoremstyle{plain}
\newtheorem{theorem}{Theorem}[section]
\newtheorem{proposition}[theorem]{Proposition}
\newtheorem{corollary}[theorem]{Corollary}
\newtheorem{lemma}[theorem]{Lemma}

% Definisjonstil
\theoremstyle{definition}
\newtheorem{definition}[theorem]{Definition}
\newtheorem{example}[theorem]{Example}
\newtheorem{remark}[theorem]{Remark}
\newtheorem{construction}[theorem]{Construction}
\newtheorem{notation}[theorem]{Notation}
\newtheorem{fact}[theorem]{Fact}

% Blackboard shortcuts
\newcommand{\Fb}{{\mathbb{F}}}
\newcommand{\Nb}{{\mathbb{N}}}
\newcommand{\Qb}{{\mathbb{Q}}}
\newcommand{\Rb}{{\mathbb{R}}}
\newcommand{\Zb}{{\mathbb{Z}}}

% Caligraphy shortcuts
\newcommand{\Ac}{{\mathcal{A}}}
\newcommand{\Bc}{{\mathcal{B}}}
\newcommand{\Cc}{{\mathcal{C}}}
\newcommand{\Ic}{{\mathcal{I}}}
\newcommand{\Kc}{{\mathcal{K}}}
\newcommand{\Mc}{{\mathcal{M}}}
\newcommand{\Nc}{{\mathcal{N}}}
\newcommand{\Pc}{{\mathcal{P}}}
\newcommand{\Tc}{{\mathcal{T}}}

% Set management shortcuts
\newcommand{\intersect}{\mathop{\cap}\limits}
\newcommand{\union}{\mathop{\cup}\limits}
\newcommand{\directsum}{\mathop{\oplus}\limits}

% Shorthands
% \newcommand{\abs}[1]{ \lvert #1 \rvert }
% \newcommand{\set}[1]{ \left\{ #1 \right\} }
% \newcommand{\tuple}[1]{ \left( #1 \right) }
% \newcommand{\toda}[1]{ \langle #1 \rangle }
% \newcommand{\class}[1]{ \left[ #1 \right] }
% \newcommand{\massey}[1]{ \langle \! \langle #1 \rangle \! \rangle }
\newcommand{\Stmod}[1]{\stablemod\tuple{ #1 }}
\DeclarePairedDelimiter{\abs}{\vert}{\vert}
\DeclarePairedDelimiter{\set}{\{}{\}}
\DeclarePairedDelimiter{\tuple}{(}{)}
\DeclarePairedDelimiter{\toda}{\langle}{\rangle}
\DeclarePairedDelimiter{\class}{[}{]}
\DeclarePairedDelimiter{\massey}{\langle \! \langle}{\rangle \! \rangle}


% New math operators
\DeclareMathOperator{\Id}{Id}
\DeclareMathOperator{\StMod}{StMod}
\DeclareMathOperator{\stablemod}{Stmod}
\DeclareMathOperator{\Obj}{Obj}
\DeclareMathOperator{\Hom}{Hom}
\DeclareMathOperator{\Mod}{Mod}
% \DeclareMathOperator{\mod}{mod}
\DeclareMathOperator{\coker}{coker}
\DeclareMathOperator{\im}{im}
\DeclareMathOperator{\Ab}{Ab}
\DeclareMathOperator{\Fun}{Fun}
\DeclareMathOperator{\dg}{dg}
\DeclareMathOperator{\C}{C}
\DeclareMathOperator{\D}{D}
\DeclareMathOperator{\dgMod}{dgMod}
\DeclareMathOperator{\dgFun}{dgFun}


\title{Master thesis TODO}
\author{Håvard Skjetne Lilleheie}

\begin{document}

\maketitle

\tableofcontents

\section{Stable module category is triangulated}
\section{Stable module category is triangulated}

% TODO: Significantly rewrite the entire section. Additivity can directly show most of the results given in this section. Also, many proofs are very similar.
\begin{definition}\label{def:stable_module_category}
    Let \( G \) be a group. Let \( R \) be a \( G \) algebra over the field \( K \), i.e. \( R = KG \) with the free module structure, and normal multiplication of group and field elements.

    Then the ``stable module category over \( R \)'', denoted \( \Tc := \StMod(R) \) is defined in the following way:
    \begin{enumerate}
        \item \( \Obj(\Tc) := \Obj(\Mod(R)) \).
        \item \( \Hom_{\Tc}(A, B) := \Hom_R(A, B)/\set{\text{maps that factor through a projective}} \)
    \end{enumerate}
\end{definition}

\begin{theorem}
    The definition in \autoref{def:stable_module_category} is well defined, and it is an additive category.
\end{theorem}
\begin{proof}
    First, need to check that the set of maps that factor through a projective is an \( R \) submodule of \( \Hom_R(A, B) \).

    Let, \( f \) and \( g \) be two maps that factor through the projectives \( P \) and \( Q \) respectively. Then we have the following diagrams:

    \begin{center}
        \begin{tikzpicture}
            \diagram{m}{1cm}{1cm} {
                A & P & B \\
            };

            \draw[math]
                (m-1-1) edge node {f_1} (m-1-2)
                (m-1-2) edge node {f_2} (m-1-3);
        \end{tikzpicture}
    \end{center}

    Where \( f_2 \circ f_1 = f \), and

    \begin{center}
        \begin{tikzpicture}
            \diagram{m}{1cm}{1cm} {
                A & Q & B \\
            };

            \draw[math]
                (m-1-1) edge node {g_1} (m-1-2)
                (m-1-2) edge node {g_2} (m-1-3);
        \end{tikzpicture}
    \end{center}

    Where \( g_2 \circ g_1 = g \).

    Can then construct the map

    \begin{center}
        \begin{tikzpicture}
            \diagram{m}{1cm}{1cm} {
                A & {P \oplus Q} & B \\
            };

            \draw[math]
                (m-1-1) edge node {(f_1, g_1)^T} (m-1-2)
                (m-1-2) edge node {(f_2, g_2)} (m-1-3);
        \end{tikzpicture}
    \end{center}

    Composing these two maps, one gets the map \( f_2 \circ f_1 + g_2 \circ g_1 = f + g \). This maps factors thorugh \( P \oplus Q \), which is projective since it's a direct sum of projective modules.

    Therefore, the set of homomorphisms that factor through a projective is closed under addition. And multiplying with a ring element still factors through the same projective, since every map is an \( R \) homomorphism. Therefore the set of maps that factor through a projective is an \( R \) submodule.

    Therefore \( \Hom_{\Tc}(A, B) \) is an abelian group, and the outstanding properties of an additive category is inherited from \( \Mod(R) \) as well. (TODO: Prove the unproved properties!)
\end{proof}

\begin{definition}
    Let \( A \in \Obj(\Tc) \).

    Let \( \Omega \) an endofunctor on \( \Tc \). Where \( \Omega(A) \) is given by choosing a projecive module \( P \) for every \( A \) with an endomorphism \( \pi_A \) from \( P \) to \( A \), and taking the kernel of that map. I.e \( \Omega(A) = \ker(\pi_A) \).
\end{definition}

\begin{remark}
    From the definition of \( \Omega \), \( \Omega(f) \) is constructed as follows:

    Looking at the following commutative diagram:

    \begin{center}
        \begin{tikzpicture}
            \diagram{m}{1cm}{1cm} {
                {\Omega(A)} & {P_A} & A \\
                {\Omega(B)} & {P_B} & B \\
            };

            \draw[math]
                (m-1-1) edge[hook] node {\iota_A} (m-1-2)
                    edge node {\Omega(f)} (m-2-1)
                (m-1-2) edge[two heads] node {\pi_A} (m-1-3)
                    edge node {p_f} (m-2-2)
                (m-1-3) edge node {f} (m-2-3)

                (m-2-1) edge[hook] node {\iota_B} (m-2-2)
                (m-2-2) edge[two heads] node {\pi_B} (m-2-3);
        \end{tikzpicture}
    \end{center}

    One has that for a map \( f: A \to B \), one gets the map \( p_f \) from the lifitng property of projective modules. Please note that this map is \emph{not neccesarily} unique.

    Furthermore, since \( \pi_B \circ p_f \circ \iota_A = f \circ \pi_A \circ \iota_A = f \circ 0 = 0 \), one has from the universal kernel property that there is a \emph{unique} map (given \( p_f \)) \( \Omega(f) \) from \( \Omega(A) \) to \( \Omega(B) \), which is the map defined by the functor.
\end{remark}

\begin{lemma}
    One has that \( \Omega \) is a functor.
\end{lemma}
\begin{proof}
    % Need to check the following:
    % 1) F(f o g) = F(f) o F(g)
    % 2) F(1) = 1

    First want to show that \( \Omega \) is functorial. Let \( A, B, C \in \Obj(\Tc) \). Then one can create the following diagram using the notation from before:

    \begin{center}
        \begin{tikzpicture}
            \diagram{m}{1cm}{2cm} {
                {\Omega(A)} & {P_A} & A \\
                {\Omega(B)} & {P_B} & B \\
                {\Omega(C)} & {P_C} & C \\
            };

            \draw[math]
                (m-1-1) edge[hook] (m-1-2)
                    edge[curve={height=30pt}, swap, color={rgb,255:red,214;green,92;blue,92}] node {\Omega(f \circ g)} (m-3-1)
                    edge node {\Omega(g)} (m-2-1)
                (m-1-2) edge[two heads] (m-1-3)
                    edge[curve={height=30pt}, swap, color={rgb,255:red,214;green,92;blue,92}] node[pos=0.3] {p_{f \circ g}} (m-3-2)
                    edge node {p_g} (m-2-2)
                (m-1-3) edge node {g} (m-2-3)
                    edge[curve={height=-30pt}, color={rgb,255:red,214;green,92;blue,92}] node {f \circ g} (m-3-3)

                (m-2-1) edge[hook] (m-2-2)
                    edge node {\Omega(f)} (m-3-1)
                (m-2-2) edge[two heads] (m-2-3)
                    edge node {p_f} (m-3-2)
                (m-2-3) edge node {f} (m-3-3)

                (m-3-1) edge[hook] (m-3-2)
                (m-3-2) edge[two heads] (m-3-3);
        \end{tikzpicture}
    \end{center}

    Then, one gets the following commuting diagram:

    \begin{center}
        \begin{tikzpicture}
            \diagram{m}{1cm}{2cm} {
                {\Omega(A)} & {P_A} & A \\
                {\Omega(C)} & {P_C} & C \\
            };

            \draw[math]
                (m-1-1) edge[hook] (m-1-2)
                    edge[swap] node {\Omega(f \circ g) - \Omega(f) \circ \Omega(g)} (m-2-1)
                (m-1-2) edge[two heads] (m-1-3)
                    edge[dashed, swap] node {\phi} (m-2-1)
                    edge node {p_{f \circ g} - p_f \circ p_g} (m-2-2)
                (m-1-3) edge node {f \circ g - f \circ g = 0} (m-2-3)

                (m-2-1) edge[hook] (m-2-2)
                (m-2-2) edge[two heads] (m-2-3);
        \end{tikzpicture}
    \end{center}

    But this implies that \( \pi_C \circ (p_{f \circ g} - p_f \circ p_g) = 0 \), which inducec a map by the kernel property \( \phi: P_A \to \Omega(C) \). Such that the lower triangle commutes. And since, \( \iota_C \) is a monomorphism, one gets that the upper triangle also commutes. And therefore \( \Omega(f \circ g) - \Omega(f) \circ \Omega(g) \) factors through a projective, and therefore \( \Omega(f \circ g) \sim \Omega(f) \circ \Omega(g) \).

    Second, need to show that \( \Omega(\Id_A) = \Id_{\Omega(A)} \) in \( \Tc \).

    By the same argument like above, one can see that every square and triangle in the following diagram also commutes:

    \begin{center}
        \begin{tikzpicture}
            \diagram{m}{1cm}{2cm} {
                {\Omega(A)} & {P_A} & A \\
                {\Omega(A)} & {P_A} & A \\
            };

            \draw[math]
                (m-1-1) edge[hook] (m-1-2)
                    edge[swap] node {\Omega(Id_A) - Id_{\Omega(A)}} (m-2-1)
                (m-1-2) edge[two heads] (m-1-3)
                    edge[dashed, swap] node {\phi} (m-2-1)
                    edge node {p_{Id_A} - Id_{P_A}} (m-2-2)
                (m-1-3) edge node {Id_A - Id_A = 0} (m-2-3)

                (m-2-1) edge[hook] (m-2-2)
                (m-2-2) edge[two heads] (m-2-3);
        \end{tikzpicture}
    \end{center}

    And therefore \( \Omega(\Id_A) \sim Id_{\Omega(A)} \).

\end{proof}

\begin{lemma}
    Let \( A, B \in \Tc \), then for \( f, g \in \Hom_{\Tc}(A, B) \), one has that \( \Omega(f + g) = \Omega(f) + \Omega(g) \) in \( \Tc \). I.e. \( \Omega \) is additive.
\end{lemma}
\begin{proof}
    Want to show that \( \Omega(f + g) \sim \Omega(f) + \Omega(g) \).
    
    One has that for any morphisms \( f, g \in \Hom_{\Tc}(A, B) \), from the definition of \( \Tc \), that \( f = g \) in \( \Tc \) if \( f - g \) factors through a projective.

    With that in mind, look at the following diagram:

    \begin{center}
        \begin{tikzpicture}
            \diagram{m}{1cm}{2cm} {
                {\Omega(A)} & {P_A} & A \\
                {\Omega(B)} & {P_B} & B \\
            };

            \draw[math]
                (m-1-1) edge[hook] node {\iota_A} (m-1-2)
                    edge[swap] node {\Omega(f+g)-\Omega(f)-\Omega(g)} (m-2-1)
                (m-1-2) edge[two heads] node {\pi_A} (m-1-3)
                    edge[dashed, swap] node {\phi} (m-2-1)
                    edge node {p_{f + g} - p_f - p_g} (m-2-2)
                (m-1-3) edge node {f + g - f - g = 0} (m-2-3)

                (m-2-1) edge[hook] node {\iota_B} (m-2-2)
                (m-2-2) edge[two heads] node {\pi_B} (m-2-3);
        \end{tikzpicture}
    \end{center}

    Starting from the leftmost side, want to show that \( \Omega(f + g) - \Omega(f) - \Omega(g) \) factors through a projective.

    Firstly, one can observe that \( \iota_B \circ (\Omega(f+g)-\Omega(f)-\Omega(g)) = \iota_B \circ \Omega(f + g) - \iota_B \circ \Omega(f) - \iota_B \circ \Omega(g) = p_{f + g} \circ \iota_A - p_{f} \circ \iota_A - p_{g} \circ \iota_A = (p_{f + g} - p_f - p_g) \circ \iota_A \). So the map \( p_{f + g} - p_f - p_g: P_A \to P_B \) makes the left square commute.
    
    Secondly, one can see that \( \pi_B \circ (p_{f + g} - p_f - p_g) = \pi_B \circ p_{f + g} - \pi_B \circ p_f - \pi_B \circ  p_g = (f + g) \circ \pi_A - f \circ \pi_A - g \circ \pi_A = (f + g - f - g) \circ \pi_A = 0 \circ \pi_A = 0 \). 
    
    But then from the kernel property there is an induced and unique map \( \phi: P_A \to \Omega(B) \) such that \( \iota_B \circ \phi = p_{f + g} - p_f - p_g \). But from the commutativity of the left square, one has that \( \iota_B \circ (\Omega(f+g)-\Omega(f)-\Omega(g)) = \iota_B \circ \phi \circ \iota_A \). Furthermore, since \( \iota_A \) is a monomorphism, one gets that \( \Omega(f+g)-\Omega(f)-\Omega(g) = \phi \circ \iota_A \).

    But that implies that \( \Omega(f+g)-\Omega(f)-\Omega(g) \) factors thorugh a projective, and therefore \( \Omega(f + g) \sim \Omega(f) + \Omega(g) \).
\end{proof}

\begin{lemma}
    The definition of \( \Omega \) is well defined.
\end{lemma}
\begin{proof}
    There are two things that need to be proven. Firstly, in the construction of \( \Omega(f) \), one have to chose a map \( p_f \) from the projective property. Need to show that if one choses another projective map, that the \( \Omega \) still yields the same map in \( \Tc \). Secondly, one need to show that if \( f \sim g \), then \( \Omega(f) \sim \Omega(g) \).

    To prove the first part, let \( p_f \) and \( p_f' \) be two different projective maps that give the maps \( \Omega(f) \) and \( \Omega(f)' \) respectively. Then one has the following commutative diagram:

    \begin{center}
        \begin{tikzpicture}
            \diagram{m}{1cm}{2cm} {
                {\Omega(A)} & {P_A} & A \\
                {\Omega(B)} & {P_B} & B \\
            };

            \draw[math]
                (m-1-1) edge[hook] node {\iota_A} (m-1-2)
                    edge[swap] node {\Omega(f) - \Omega(f)'} (m-2-1)
                (m-1-2) edge[two heads] node {\pi_A} (m-1-3)
                    edge[dashed, swap] node {\phi} (m-2-1)
                    edge node {p_f - p'_f} (m-2-2)
                (m-1-3) edge node {f -f = 0} (m-2-3)

                (m-2-1) edge[hook] node {\iota_B} (m-2-2)
                (m-2-2) edge[two heads] node {\pi_B} (m-2-3);
        \end{tikzpicture}
    \end{center}

    Using the same argument as always, one gets that \( \Omega(f) \sim \Omega(f)' \). (I also think this follows directly from additivity.)

    Secondly, need to show that if \( f \sim g \), then \( \Omega(f) \sim \Omega(g) \). Look at the following diagram:

    \begin{center}
        \begin{tikzpicture}
            \diagram{m}{1cm}{3cm} {
                \Omega(A) & P_A & A \\
                && P \\
                \Omega(B) & P_B & B \\
            };

            \draw[math]
                (m-1-1) edge[hook] (m-1-2)
                    edge node {0} (m-3-1)
                (m-1-2) edge[two heads] (m-1-3)
                    edge node {\theta \circ (f - g)_1 \circ \pi_A} (m-3-2)
                (m-1-3) edge[swap] node {(f - g)_1} (m-2-3)
                    edge[curve={height=-25pt}] node {f - g} (m-3-3)

                (m-2-3) edge[swap, dashed, color={rgb,255:red,214;green,92;blue,92}] node {\theta} (m-3-2)
                    edge[swap] node {(f - g)_2} (m-3-3)

                (m-3-1) edge[hook] (m-3-2)
                (m-3-2) edge[two heads] (m-3-3);
        \end{tikzpicture}
    \end{center}

    Let \( P \) be the projective that \( f - g \) factors through. Then from the projective propertive, one gets a map \( \theta: P \to P_B \). Then the diagram commutes. But then \( \theta \circ (f-g)_1 \circ \pi_A \circ \iota_A = \theta \circ (f-g)_1 \circ 0 = 0 \), and so the diagram commutes with \( 0: \Omega(A) \to \Omega(B) \). But since, \( \theta \circ (f-g)_1 \circ \pi_A \) is another choice of \( p_{f - g} \), from the previous part of the proof since \( \Omega \) is independent of the choice of \( h \)-maps, one gets that \( \Omega(f - g) \sim 0 \). And from additivity, one has that \( \Omega(f - g) \sim \Omega(f) - \Omega(g) \), one then gets that \( \Omega(f) \sim \Omega(g) \).
\end{proof}

% Need to show the following:
    % Sigma well defined:
    % 1) Independant of choice of P/I
    % 2) Independant of choice of projective/injective map
    % 3) Given two equivalent maps, is the image the same?
    % Sigma additive.
    % Then both \( \Sigma \) and \( \Omega \), are additive automorphisms with \( \Sigma^{-1} = \Omega \).
\begin{remark}
    Since \( R \) is a \( G \)-algebra over a field \( K \), it is known that every projective module is injective and vice versa.
\end{remark}

\begin{definition}
    Let \( A \in \Obj(\Tc) \).

    Let \( \Sigma \) be an endofunctor on \( \Tc \), where \( \Sigma(A) \) is given by choosing for every \( A \) a injective module \( I \), and a monomorphism from \( \kappa_A: A \to I \). Then taking the cokernel of that map. I.e \( \Sigma(A) = \coker(\kappa_A) \).
\end{definition}

\begin{lemma}
    One has that \( \Sigma \) is a well defined and additive functor.
\end{lemma}
\begin{proof}
    Very similar proofs as for \( \Omega \), but using the injective module property as well as the cokernel property. (TODO)
\end{proof}

\begin{theorem}
    One has that \( \Omega \) is an auto equivalence with \( \Omega^{-1} = \Sigma \).
\end{theorem}
\begin{proof}
    I will only show that \( \Id_{\Tc} \) is naturally isomorphic to \( \Sigma\Omega \). I claim (TODO) that the proof of the other part is very similar, but uses many dual properties.

    Let \( A \in \Obj(\Tc) \).

    First show that there exist a (not neccesarily unique, but for any object, just choose one) isomorphism from \( A \to \Sigma\Omega(A) \). Consider the following diagram:

    \begin{center}
        \begin{tikzpicture}
            \diagram{m}{1cm}{2cm} {
                \Omega(A) & P_A & A \\
                \Omega(A) & I_{\Omega(A)} & \Sigma \circ \Omega(A) \\
                \Omega(A) & P_A & A \\
            };

            \draw[math]
                (m-1-1) edge[hook] node {\iota_A} (m-1-2)
                    edge[equal] (m-2-1)
                (m-1-2) edge[two heads] node {\pi_A} (m-1-3)
                    edge node {i_{\phi_1}} (m-2-2)
                (m-1-3) edge node {\phi_1} (m-2-3)

                (m-2-1) edge[hook] node {\kappa_{\Omega(A)}} (m-2-2)
                    edge[equal] (m-3-1)
                (m-2-2) edge[two heads] node {\rho_{\Omega(A)}} (m-2-3)
                    edge node {i_{\phi_2}} (m-3-2)
                (m-2-3) edge node {\phi_2} (m-3-3)

                (m-3-1) edge[hook] node {\iota_A} (m-3-2)
                (m-3-2) edge[two heads] node {\pi_A} (m-3-3);
        \end{tikzpicture}
    \end{center}

    Where \( i_{\phi_1} \) is the injective property map induced from \( \iota_A \). Then, since \( \rho_{\Omega(A)} \circ i_{\phi_1} \circ \iota_A = \rho_{\Omega(A)} \circ \kappa_{\Omega(A)} = 0 \), and since \( \Mod(R) \) is an abelian category, one has that the cokernel of a kernel of a epimorphism is isomorphic to the codomain of the epimorphism. Therefore one has that \( A \) is the cokernel of \( \iota_A \). Therefore, from the cokernel property, there is a uniquely induced map \( \phi_1: A \to \Sigma\Omega(A) \). Then doing the same for the lower rectangle of the diagram, using the fact that every projective is also injective, then one gets the map \( \phi_2 \).

    Then to show that \( \phi_1 \) and \( \phi_2 \) are isomorphisms, look at the following diagram:

    \begin{center}
        \begin{tikzpicture}
            \diagram{m}{1cm}{2cm} {
                \Omega(A) & P_A & A \\
                \Omega(A) & P_A & A \\
            };

            \draw[math]
                (m-1-1) edge[hook] node {\iota_A} (m-1-2)
                    edge[swap] node {\Id_{\Omega(A)} \circ \Id_{\Omega(A)} - \Id_{\Omega(A)} = 0} (m-2-1)
                (m-1-2) edge[two heads] node {\pi_A} (m-1-3)
                    edge[swap] node {p_{\phi_2} \circ p_{\phi_1} - \Id_P} (m-2-2)
                (m-1-3) edge[swap, color={rgb,255:red,214;green,92;blue,92}] node {\theta} (m-2-2)
                    edge node {\phi_2 \circ \phi_1 - \Id_A} (m-2-3)

                (m-2-1) edge[hook] node {\iota_A} (m-2-2)
                (m-2-2) edge[two heads] node {\pi_A} (m-2-3);
        \end{tikzpicture}
    \end{center}

    Using the previous commutative diagram, one gets that every square commutes. But then \( (p_{\phi_2} \circ p_{\phi_1} - \Id_{P_A}) \circ \iota_A = \iota_A \circ 0 = 0 \). Then from the cokernel property there exist a map \( \theta: A \to P_A \) such that \( \theta \circ \pi_A = p_{\theta_2} \circ p_{\theta_1} - \Id_P \). But then one has that \( (\phi_2 \circ \phi_1 - \Id_A) \circ \pi_A = \pi_A \circ (p_{\theta_2} \circ p_{\theta_1} - \Id_P) = \pi_A \circ \theta \circ \pi_A \). But since \( \pi_A \) is an epimorphism, one gets that \( \phi_2 \circ \phi_1 - \Id_A = \pi_A \circ \theta \), which means that \( \phi_2 \circ \phi_1 \sim \Id_A \).
    
    Similarly (TODO) one can show that \( \phi_1 \circ \phi_2 \sim Id_A \), which means that \( \phi_1 \) and \( \phi_2 \) are isomorphisms from \( A \) to \( \Sigma\Omega(A) \).

    For \( A, B \in \Tc \), let \( f \in \Hom_R(A, B) \).

    To show that these isomorphisms are natural, look at the following two diagrams:

    \begin{center}
        \begin{tikzpicture}
            \diagram{m}{1cm}{2cm} {
                \Omega(A) & P & A \\
                \Omega(A) & I & \Sigma \circ \Omega(A) \\
                \Omega(B) & I & \Sigma \circ \Omega(B) \\
            };

            \draw[math]
                (m-1-1) edge[hook] node {\iota_A} (m-1-2)
                    edge[equal] (m-2-1)
                (m-1-2) edge[two heads] node {\pi_A} (m-1-3)
                    edge node {p_{\phi^A_1}} (m-2-2)
                (m-1-3) edge node {\phi^A_1} (m-2-3)

                (m-2-1) edge[hook] node {\kappa_{\Omega(A)}} (m-2-2)
                    edge node {\Omega(f)} (m-3-1)
                (m-2-2) edge[two heads] node {\rho_{\Omega(A)}} (m-2-3)
                    edge node {i_{\Omega(f)}} (m-3-2)
                (m-2-3) edge node {\Sigma \circ \Omega(f)} (m-3-3)

                (m-3-1) edge[hook] node {\kappa_{\Omega(B)}} (m-3-2)
                (m-3-2) edge[two heads] node {\rho_{\Omega(B)}} (m-3-3);
        \end{tikzpicture}
    \end{center}

    And:

    \begin{center}
        \begin{tikzpicture}
            \diagram{m}{1cm}{2cm} {
                \Omega(A) & P & A \\
                \Omega(B) & P & B \\
                \Omega(B) & I & \Sigma \circ \Omega(B) \\
            };

            \draw[math]
                (m-1-1) edge[hook] node {\iota_A} (m-1-2)
                    edge node {\Omega(f)} (m-2-1)
                (m-1-2) edge[two heads] node {\pi_A} (m-1-3)
                    edge node {p_f} (m-2-2)
                (m-1-3) edge node {f} (m-2-3)

                (m-2-1) edge[hook] node {\iota_B} (m-2-2)
                    edge[equal] (m-3-1)
                (m-2-2) edge[two heads] node {\pi_B} (m-2-3)
                    edge node {i_{\phi_1^B}} (m-3-2)
                (m-2-3) edge node {\phi_1^B} (m-3-3)

                (m-3-1) edge[hook] node {\kappa_{\Omega(B)}} (m-3-2)
                (m-3-2) edge[two heads] node {\rho_{\Omega(B)}} (m-3-3);
        \end{tikzpicture}
    \end{center}

    Where every small square, and therefore rectangle, commutes.

    This gives rise to the following commutative diagram:

    \begin{center}
        \begin{tikzpicture}
            \diagram{m}{1cm}{3cm} {
                \Omega(A) & P & A \\
                \Omega(B) & I & \Sigma \circ \Omega(B) \\
            };

            \draw[math]
                (m-1-1) edge[hook] (m-1-2)
                    edge[swap] node {\Id_{\Omega(B)} \circ \Omega(f) - \Omega(f) \circ \Id_{\Omega(A)} = 0} (m-2-1)
                (m-1-2) edge[two heads] (m-1-3)
                    edge[swap] node {i_{\phi_1^B} \circ p_f - i_{\Omega(f)} \circ p_{\phi_1^A}} (m-2-2)
                (m-1-3) edge[swap, color={rgb,255:red,214;green,92;blue,92}] node {\theta} (m-2-2)
                    edge node {\phi_1^B \circ f - \Sigma\Omega(f) \circ \phi_1^A} (m-2-3)

                (m-2-1) edge[hook] (m-2-2)
                (m-2-2) edge[two heads] (m-2-3);
        \end{tikzpicture}
    \end{center}

    But from the cokernel property of \( A \), one gets an induced map \( \theta \), and from the epimorphism property, it makes the lower triangle commute, which implies \( \phi_1^B \circ f \sim \Sigma\Omega(f) \circ \phi_1^A \), which means that it is natural. And since the choice of \( \phi_1 \) is arbitrary, it is independent of the choice of \( \phi_1 \).
\end{proof}

I found two different definitions of distinguished triangles in \( \StMod(R) \) that are most likely equivalent (TODO):

\begin{definition}
    Definition from Zimmermann:

    Let \( \Delta \) be every triangle isomorphic to a triangle of the form of the top row of this triangle:

    \begin{center}
        \begin{tikzpicture}
            \diagram{m}{1cm}{1cm} {
                A & B & C & \Sigma(A) \\
                A & I_A \\
            };

            \draw[math]
                (m-1-1) edge[hook] node {\alpha} (m-1-2)
                    edge[equal] (m-2-1)
                (m-1-2) edge[two heads] node {\beta} (m-1-3)
                    edge[swap] node {\iota} (m-2-2)
                (m-1-3) edge node {\gamma} (m-1-4)

                (m-2-1) edge[hook] (m-2-2)
                (m-2-2) edge[two heads] (m-1-4);
        \end{tikzpicture}
    \end{center}

    Where \( A, B, C \) with \( \alpha, \beta \) makes a short exact sequence, and \( \iota \) is given by the injective property of \( I_A \), and \( \gamma \) is the cokernel induced map.

    Definition from Martirosian:

    Let \( \Delta \) be any triangle isomorphic to a triangle of the form of the top row of the following diagram:

    \begin{center}
        \begin{tikzpicture}
            \diagram{m}{1cm}{1cm} {
                A & B & C_\alpha & \Sigma(A) \\
                & I_A \\
            };

            \draw[math]
                (m-1-1) edge node {\alpha} (m-1-2)
                    edge[swap] node {\iota_A} (m-2-2)
                (m-1-2) edge node {\beta} (m-1-3)
                    edge[curve={height=-25pt}] node {0} (m-1-4)
                (m-1-3) edge node {\gamma} (m-1-4)

                (m-2-2) edge[curve={height=15pt}] node {\pi_A} (m-1-4)
                    edge node {\rho} (m-1-3);
        \end{tikzpicture}
    \end{center}

    Where \( C_\alpha \) is the pushout of \( A, B, I_A \), and \( \gamma \) is given by the pushout universal property.
\end{definition}

\begin{remark}
    I think (TODO) the definitions are equivalent because a pushout can be expressed as a short exact sequence \( A \to B \oplus I_A \to C_\alpha \), and since \( B \oplus I_A \simeq B \) in \( \Tc \), then it can be possible to connect the two definitions. 
\end{remark}

\begin{theorem}
    One has that \( \StMod(R) \) is a triangulated category with \( \Sigma \) as the suspension and \( \Delta \) as the distinguished triangles.
\end{theorem}
\begin{proof}
    TODO
\end{proof}

% TODO: Not boldface, the math-operator and \Fb doesnt become bold.
\section{Properties of \texorpdfstring{\( \Stmod{\Fb_3 C_3} \)}{Stmod(F\_3C\_3)}} 
\section{Properties of \texorpdfstring{\( \Stmod{\Fb_3 C_3} \)}{Stmod(F\_3C\_3)}} % TODO: Not boldface, the math-operator and \Fb doesnt become bold.
% TODO another class possible if P = M^n?

% Isomorphic to path-algebra with relations

\begin{notation}
    Let \( \Fb_3 \) be the finite field with \( 3 \) elements.
\end{notation}

\begin{notation}
    Let \( C_3 \) be the cyclic group with \( 3 \) elements.
\end{notation}

\begin{notation}
    Let \( \Fb_3 C_3 \) be the group algebra of \( \Fb_3 \) over \( C_3 \).
\end{notation}

\begin{lemma}
    Let \( \Gamma = \)
    \begin{tikzpicture}
        \diagram{m}{1cm}{1cm} {
            1 \\
        };

        \draw[math]
            (m-1-1) edge[in=150, out=30, looseness=4.8] node[swap] {\alpha} (m-1-1);
    \end{tikzpicture}
    be a quiver.
    Let \( g \in \Fb_3 C_3 \) be the generator of \( C_3 \).
    
    Then the algebra homomorphism defined as follows
    \begin{align*}
        \phi: \Fb_3 C_3 &\to \frac{\Fb_3\Gamma}{(\alpha^3)} \\
        1 &\mapsto e \\
        g &\mapsto \alpha + 1
    \end{align*}
    is an isomorphism from \( \Fb_3 C_3 \) to \( \frac{\Fb_3\Gamma}{(\alpha^3)} \).    
\end{lemma}
\begin{proof}
    TODO
\end{proof}

\begin{definition}
    Let \( S := \frac{\Fb_3C_3}{(g - 1)} \), let \( M := \frac{\Fb_3C_3}{(g - 1)^2} \), and let \( P := {\Fb_3C_3}_{\Fb_3C_3} \).
\end{definition}

\begin{lemma}
    One has that \( \Fb_3 C_3 \) has three distinct indecomposable modules up to isomorphism.
    
    They are isomorphic to \( S, M, \) and \( P \) with \( S \) a simple module, and \( P \) a projective module.

    Furthermore, neither \( S \) or \( M \) are projective modules.
\end{lemma}
\begin{proof}
    TODO
\end{proof}

\begin{definition}
    Let \( \Mc := \Stmod{\Fb_3C_3} \).
\end{definition}

\begin{lemma} \label{thm:f_3c_3_mu}
    Let \( \mu: M \to S \) be defined as TODO

    Then \( \Mc\tuple{M, S} = \set{0, \pm\mu} \)
\end{lemma}
\begin{proof}
    TODO
\end{proof}

\begin{lemma} \label{lem:classify_m}
    One has that \( M = \set{[\pm 1], [\pm g], [\pm g^2], [\pm (g - 1)]} \)
\end{lemma}
\begin{proof}
    TODO
\end{proof}

\begin{lemma} \label{thm:f_3c_3_nu} % NEEDED TODO
    Let 
    \begin{align*}
        \nu: S &\to M \\
        [\tilde{a}1] \mapsto [\tilde{a}(g - 1)]
    \end{align*}

    Then \( \nu \) is well defined, and \( \Mc\tuple{S, M} = \set{0, \pm \nu} \)
\end{lemma}
\begin{proof}
    First, want to prove that \( \nu \) is well defined.

    Class-representation:

    Let \( \bar{a} \in S \). Then one has that \( \bar{a} = [a1 + bg + cg^2] = [a1 + b1 + c1] = [(a + b + c)1] \). And so any element of \( S \) can be written as \( [\tilde{a}1] \) for som \( \tilde{a} \in \Fb_3 \).

    Group homomorphism:

    Let \( g \in \Fb_3C_3 \) be the generator of \( C_3 \). Then \( \nu([a1] + [b1]) = \nu([(a + b)1]) = [(a + b)(g - 1)] = [a(g - 1) + b(g - 1)] = [a(g - 1)] + [b(g - 1)] = \nu([a]) + \nu([b]) \).

    Module homomorphism:

    Let \( g \in \Fb_3C_3 \) be the generator of \( C_3 \). Then one has that for any \( r = a + bg + cg^2 \in \Fb_3C_3 \) that \( r[\tilde{a}1] = [r\tilde{a}1] = [(a + b + c)\tilde{a}1] \), and
    \begin{align*}
        r\nu([\tilde{a}1]) &= r[\tilde{a}(g - 1)] \\
        &= [r\tilde{a}(g - 1)] \\
        &= [(a + bg + cg^2)\tilde{a}(g - 1)] \\
        &\vdots \\
        &= [(a + b + c)\tilde{a}(g - 1)] \\
        &= \nu(r[\tilde{a}1])
    \end{align*}

    Independent of choice of class-representation:

    Let \( [a] = [b] \) in \( S \). Then \( [a] - [b] = [r(g-1)] \) for some \( r \in \Fb_3C_3 \).

    Then \( \nu([a]) - \nu([b]) = \nu([a] - [b]) = \nu([r(g - 1)]) = [r(g - 1)^2] = [0] \).

    Secondly, want to show that \( \Mc\tuple{S, M} = \set{0, \pm \nu} \).

    Let \( f \in \Mc\tuple{S, M} \), and let \( r \in \Fb_3C_3 \).
    
    Then \( f \) is a module homomorphism, and one therefore has that \( f([a1]) = a1 f([1]) \). And so \( f \) is fully determined by the value of \( f([1]) \). And from \autoref{lem:classify_m} on has that \( M = \set{[\pm (g - 1)], [\pm 1], [\pm g], [\pm g^2]} \).

    However one can see that if \( [1] \) is sent to \( [\pm 1], [\pm g] \), or \( [\pm g^2] \), then \( f([0]) = f([g - 1]) = (g - 1)f([1]) \neq [0] = [g^2 + g + 1] = [(g - 1)^2] \), which would imply that \( f \) is \emph{not} a group homomorphism.

    Therefore \( [1] \) has to be sent to either \( [(g - 1)] \) or \( [-(g - 1)] \). These maps, being \( \nu \) and \( -\nu \) respectivly.
\end{proof}

\begin{lemma} \label{thm:f_3c_3_mu_circ_nu_zero}
    Let \( \mu \) and \( \nu \) be as in \autoref{thm:f_3c_3_mu} and \autoref{thm:f_3c_3_nu} respectively.

    Then \( \mu \circ \nu = 0 \).
\end{lemma}
\begin{proof}
    TODO
\end{proof}

\begin{lemma} \label{thm:f_3c_3_decomposition}
    Any object in \( \Mc \) is isomorphic to a direct sum \( \tuple{ S }^n \oplus \tuple{ M }^m \) for \( n, m \in \Nb_0 \), where taking the power of \( 0 \) gives the zero object. \sloppy
\end{lemma}
\begin{proof}
    TODO
\end{proof}

\begin{lemma} % Is this even true?? TODO
    Let \( \Ac \) be an additive category.
    
    Then 
    \begin{enumerate}
        \item {
            The functor \( \Ac(-_1 \oplus -_2, -_3): \Ac^{op} \times \Ac^{op} \times \Ac \to \Ab \) is naturally isomorphic to the functor \( \Ac(-_1, -_3) \oplus \Ac(-_2, -_3) \). With the isomorphism

            \begin{center}
                \begin{tikzpicture}
                    \diagram{m}{1cm}{2cm} {
                        \Ac(-_1 \oplus -_2, -_3) & \Ac(-_1, -_3) \oplus \Ac(-_2, -_3) \\
                    };

                    \draw[math]
                        (m-1-1) edge[curve={height=-25pt}] node {\begin{psmallmatrix} (i_{-_1})^* \\ (i_{-_2})^* \end{psmallmatrix}} (m-1-2)
                        (m-1-2) edge[curve={height=-25pt}] node {\begin{psmallmatrix} (p_{-_1})^* & (p_{-_2})^* \end{psmallmatrix}} (m-1-1);
                \end{tikzpicture}
            \end{center}
        }
        \item {
            The functor \( \Ac(-_1, -_2 \oplus -_3): \Ac^{op} \times \Ac \times \Ac \to \Ab \) is naturally isomorphic to the functor \( \Ac(-_1, -_2) \oplus \Ac(-_1, -_3) \). With the isomorphism

            \begin{center}
                \begin{tikzpicture}
                    \diagram{m}{1cm}{2cm} {
                        \Ac(-_1, -_2 \oplus -_3) & \Ac(-_1, -_2) \oplus \Ac(-_1, -_3) \\
                    };

                    \draw[math]
                        (m-1-1) edge[curve={height=-25pt}] node {\begin{psmallmatrix} (p_{-_2})_* \\ (p_{-_3})_* \end{psmallmatrix}} (m-1-2)
                        (m-1-2) edge[curve={height=-25pt}] node {\begin{psmallmatrix} (i_{-_2})_* & (i_{-_3})_* \end{psmallmatrix}} (m-1-1);
                \end{tikzpicture}
            \end{center}
        }
    \end{enumerate}
\end{lemma}
\begin{proof}
    TODO: 
    
    https://ncatlab.org/nlab/show/additive+functor

    https://ncatlab.org/nlab/show/hom-functor+preserves+limits
\end{proof}

% Remark abuse of post-composition notation? TODO

\begin{lemma} \label{thm:hom_direct_sum_map_nice}
    Let \( \Ac \) be an additive category, and let \( A, B, C \in \Ac \). Let \( f: B \to C \). Let \( n \in \Nb \).

    Then:
    \begin{enumerate}
        \item {
            The following diagram commutes
            \begin{center}
                \begin{tikzpicture}
                    \diagram{m}{3cm}{2cm} {
                        \Ac(A, B^n) & \Ac(A, C^n) \\
                        \Ac(A, B)^n & \Ac(A, C)^n \\
                    };
        
                    \draw[math]
                        (m-1-1) edge node {(f^n)_*} (m-1-2)
                            edge node { \begin{psmallmatrix} (p_1^B)_* \\ (p_2^B)_* \\ \vdots \\ (p_n^B)_* \end{psmallmatrix} } (m-2-1)
                        (m-1-2) edge node { \begin{psmallmatrix} (p_1^C)_* \\ (p_2^C)_* \\ \vdots \\ (p_n^C)_* \end{psmallmatrix} } (m-2-2)
                        
                        (m-2-1) edge node {(f_*)^n} (m-2-2);
                \end{tikzpicture}
            \end{center}
        }
        \item {
            The following diagram commutes
            \begin{center}
                \begin{tikzpicture}
                    \diagram{m}{3cm}{2cm} {
                        \Ac(A^n, B) & \Ac(A^n, C) \\
                        \Ac(A, B)^n & \Ac(A, C)^n \\
                    };
        
                    \draw[math]
                        (m-1-1) edge node {f_*} (m-1-2)
                            edge node { \begin{psmallmatrix} (i_1^A)^* \\ (i_2^A)^* \\ \vdots \\ (i_n^A)^* \end{psmallmatrix} } (m-2-1)
                        (m-1-2) edge node { \begin{psmallmatrix} (i_1^A)^* \\ (i_2^A)^* \\ \vdots \\ (i_n^A)^* \end{psmallmatrix} } (m-2-2)
                        
                        (m-2-1) edge node {(f_*)^n} (m-2-2);
                \end{tikzpicture}
            \end{center}
        }
    \end{enumerate}
\end{lemma}
\begin{proof}
    TODO: Notes, example projective class.
\end{proof}


% Show that there only is one non-trivial map between the indecomposable non-projective modules (Might not be true)

% Show that Sigma(S) = M, Sigma(M) = S.

% Show that S -> M -> S is exact.
% Show that S -> M -> S -> M is dist.

\begin{lemma} \label{thm:F_functor} % Assignment OK? TODO
    Let \( F: \Mc \to \Mc \) be an assignment that takes any object \( A \in \Mc \) and maps it to its decomposition by \autoref{thm:f_3c_3_decomposition}, and morphisms are induced by the isomorphisms from the decomposition. I.e. there are some \( n, m \in \Nb_0 \) such that \( A \mapsto F(A) = \tuple{ S }^n \oplus \tuple{ M }^m \). And furthermore for \( f \in \Mc\tuple{A, B} \), one has \( f \mapsto F(f) = \phi_A^{-1} \circ f \circ \phi_B \), where \( \phi_A \) and \( \phi_B \) are the chosen isomorphisms between \( A \) and it's decomposition, and \( B \) and it's decomposition, respectively.

    Then \( F \) is a well defined functor.
\end{lemma}
\begin{proof}
    TODO
\end{proof}

\begin{remark} \label{rem:big_iso}
    By using \autoref{thm:hom_direct_sum_map_nice} one gets that there is a (natural) isomorphism, \( \phi \) such that
    \begin{align*}
        &\Mc\tuple{\tuple{ S }^{n_A} \oplus \tuple{ M }^{m_A}, \tuple{ S }^{n_B} \oplus \tuple{ M }^{m_B}} \\
        &\stackrel{\phi}{\cong} \Mc\tuple{ S, S }^{n_An_B} \\
        &\oplus \Mc\tuple{ S, M }^{n_Am_B} \\
        &\oplus \Mc\tuple{ M, S }^{m_An_B} \\
        &\oplus \Mc\tuple{ M, M }^{m_Am_B}
    \end{align*}

    With \( p_{S, S}, p_{S, M}, p_{M, S}, p_{M, M} \) being the projection maps down to the four main components.
\end{remark}

% Independent of choice of F, possibly, due to krull-remak-schmidt uniqe decomposition? TODO
\begin{definition} % Unholy definition.
    Let the functor \( F \) be as in \autoref{thm:F_functor}.  Let, \( \phi, p_{S, S}, p_{S, M}, p_{M, S}, p_{M, M} \) be as in \autoref{rem:big_iso}.

    Define the set \( P_B \) as follows:

    For any object \( A, B \in \Mc \), and for any \( f \in \Mc\tuple{A, B} \).

    Then \( f \in P_B \iff \)

    All of the following are true:
    \begin{enumerate}
        \item {
            \( p_{S, S} \circ \phi \circ F(f) = 0 \).
        }
        \item {
            \( p_{S, M} \circ \phi \circ F(f) = 0 \).
        }
        \item {
            \( p_{M, M} \circ \phi \circ F(f) = \set{0, \cdot(\pm (g - 1))}^{n_An_B} \).
        }
    \end{enumerate}
\end{definition}

\begin{remark}
    Definitely needf to remark on the previous definition.... 
    
    Any morphism can not have any component from S to S, or S to M, as well as they can only have one certain component from M to M.

    TODO: Fix
\end{remark}

\begin{example} % Major rewrite of first part! TODO
    Let \( \Pc = \set{ \tuple{ S }^n \mid n \in \Nb } \union \set{ 0 } \). Let \( \Nc = P_B \union \set{ 0 } \).

    Then \( \tuple{ \Pc, \Nc} \) is a projective class in \( \Mc \).
\end{example}
\begin{proof}
    Need to show that \( \tuple{ \Pc, \Nc } \) satisfies the three properties in \autoref{def:projective_class}.

    \begin{enumerate}
        \item {
            \( \tuple{ \Rightarrow } \) Let \( f \in \Nc \).

            If \( f = 0 \), then the statement is true.

            Assume \( f \neq 0 \). Then \( f = \tuple{ \mu }^n \in \Nc \) for some \( n \in \Nb \).
            
            Let \( P \in \Pc \)
            
            If \( P = 0 \), then the statement is true.

            Assume \( P = \tuple{ S }^m \) for some \( m \in \Nb \).

            Then using \autoref{thm:hom_direct_sum_map_nice}, the following map
            \[
                \tuple{ \tuple{ \mu }^n }_* : \Mc\tuple{P^m, \tuple{M}^n } \to \Mc\tuple{P^m, \tuple{S}^n }.
            \]
            can be written as
            \[
                \tuple{ \tuple{ \mu }^n }_* = \phi \circ (\mu_*)^{nm} \circ \psi
            \]
            where \( \phi \) and \( \psi \) are isomorphisms.

            From \autoref{thm:f_3c_3_nu} one has that \( \Mc\tuple{ S, M } = \set{ 0, \pm \nu } \).

            And from \autoref{thm:f_3c_3_mu_circ_nu_zero} one gets that applying \( (\mu)_* \) to either \( 0 \) or \( \pm \nu \) gives the zero map.

            Then
            \[
                \tuple{ \tuple{ \mu }^n }_* = \phi \circ (\mu_*)^{nm} \circ \psi = \phi \circ 0^{nm} \circ \psi = 0
            \]

            Since the choice of \( P \in \Pc \) was arbitrary. One must have that it is correct for every \( P \in \Pc \). And likewise for any \( f \in \Nc \).

            \( ( \Leftarrow ) \) Let \( A, B \in \Mc \) be two arbitrary objects, and let \( f \in \Mc\tuple{ A, B } \).

            If \( f = 0 \), then the statement is true. 
            
            Assume \( f \neq 0 \) (if possible). % Allowed to say this? TODO

            From \autoref{thm:f_3c_3_decomposition}, one has integers \( n_A, m_A, n_B, m_b \in \Nc_0 \) such that \( A \cong \tuple{ M }^{n_A} \oplus \tuple{ S }^{m_A} \), and \( B \cong \tuple{ M }^{n_B} \oplus \tuple{ S }^{m_B} \).

            TODO show that f is in Nc.
        }
    \end{enumerate}
\end{proof}

\section{What is Toda Brackets?}
\section{What is Toda Brackets?}

Besed on Christensen \& Frankland article (TODO).

\begin{definition}
    Given a triangulated category \( \Tc \), given the following diagram in \( \Tc \):

    % https://q.uiver.app/#q=WzAsNCxbMCwwLCJYXzAiXSxbMSwwLCJYXzEiXSxbMiwwLCJYXzIiXSxbMywwLCJYXzMiXSxbMCwxLCJmXzEiXSxbMSwyLCJmXzIiXSxbMiwzLCJmXzMiXV0=
    \[\begin{tikzcd}
        {X_0} & {X_1} & {X_2} & {X_3}
        \arrow["{f_1}", from=1-1, to=1-2]
        \arrow["{f_2}", from=1-2, to=1-3]
        \arrow["{f_3}", from=1-3, to=1-4]
    \end{tikzcd}\]

    one can define the \emph{three-fold Toda bracket of \( f_3, f_2, f_1 \)}, denoted \( \langle f_3, f_2, f_1 \rangle \), a subset of \( \Hom_{\Tc}(\Sigma(X_0), X_3) \) in three different ways:

    \begin{enumerate}
        \item {
            Iterated cofiber Toda bracket:

            Every possible \( \psi \) that makes the following diagram commute:

            % https://q.uiver.app/#q=WzAsOCxbMCwwLCJYXzAiXSxbMSwwLCJYXzEiXSxbMywwLCJcXFNpZ21hKFhfMCkiXSxbMiwwLCJDX2YiXSxbMCwxLCJYXzAiXSxbMSwxLCJYXzEiXSxbMiwxLCJYXzIiXSxbMywxLCJYXzMiXSxbMCwxLCJmXzEiXSxbMSwzXSxbMywyXSxbMCw0LCIiLDIseyJsZXZlbCI6Miwic3R5bGUiOnsiaGVhZCI6eyJuYW1lIjoibm9uZSJ9fX1dLFsxLDUsIiIsMix7ImxldmVsIjoyLCJzdHlsZSI6eyJoZWFkIjp7Im5hbWUiOiJub25lIn19fV0sWzMsNl0sWzIsNywiXFxwc2kiXSxbNCw1LCJmXzEiXSxbNSw2LCJmXzIiXSxbNiw3LCJmXzMiXV0=
            \[\begin{tikzcd}
                {X_0} & {X_1} & {Y} & {\Sigma(X_0)} \\
                {X_0} & {X_1} & {X_2} & {X_3}
                \arrow["{f_1}", from=1-1, to=1-2]
                \arrow[from=1-2, to=1-3]
                \arrow[from=1-3, to=1-4]
                \arrow[Rightarrow, no head, from=1-1, to=2-1]
                \arrow[Rightarrow, no head, from=1-2, to=2-2]
                \arrow[from=1-3, to=2-3]
                \arrow["\psi", from=1-4, to=2-4]
                \arrow["{f_1}", from=2-1, to=2-2]
                \arrow["{f_2}", from=2-2, to=2-3]
                \arrow["{f_3}", from=2-3, to=2-4]
            \end{tikzcd}\]

            Where the top row is distinguished.
        }
    \item {
            Fiber cofiber Toda bracket:

            Every composite \( \beta \circ \Sigma(\alpha) \)that makes the following diaram commute:

            % https://q.uiver.app/#q=WzAsOCxbMCwwLCJYXzAiXSxbMSwwLCJYXzEiXSxbMSwxLCJYXzEiXSxbMiwxLCJYXzIiXSxbMiwyLCJYXzIiXSxbMywyLCJYXzMiXSxbMywxLCJDX3tmXzJ9Il0sWzAsMSwiXFxTaWdtYV57LTF9KENfe2ZfMn0pIl0sWzAsMSwiZl8xIl0sWzEsMiwiIiwwLHsibGV2ZWwiOjIsInN0eWxlIjp7ImhlYWQiOnsibmFtZSI6Im5vbmUifX19XSxbMiwzLCJmXzIiXSxbMyw0LCIiLDAseyJsZXZlbCI6Miwic3R5bGUiOnsiaGVhZCI6eyJuYW1lIjoibm9uZSJ9fX1dLFs0LDUsImZfMyJdLFszLDZdLFs3LDJdLFswLDcsIlxcYWxwaGEiLDJdLFs2LDUsIlxcYmV0YSIsMl1d
            \[\begin{tikzcd}
                {X_0} & {X_1} \\
                {\Sigma^{-1}(Y)} & {X_1} & {X_2} & {Y} \\
                && {X_2} & {X_3}
                \arrow["{f_1}", from=1-1, to=1-2]
                \arrow[Rightarrow, no head, from=1-2, to=2-2]
                \arrow["{f_2}", from=2-2, to=2-3]
                \arrow[Rightarrow, no head, from=2-3, to=3-3]
                \arrow["{f_3}", from=3-3, to=3-4]
                \arrow[from=2-3, to=2-4]
                \arrow[from=2-1, to=2-2]
                \arrow["\alpha"', from=1-1, to=2-1]
                \arrow["\beta"', from=2-4, to=3-4]
            \end{tikzcd}\]

            Where the middle row is distinguished.
        }
    \item {
        Iterated fiber Toda bracket:

        Every map \( \Sigma(\delta) \), where \( \delta \) makes the following diagram commute:

        % https://q.uiver.app/#q=WzAsOCxbMCwwLCJYXzAiXSxbMSwwLCJYXzEiXSxbMiwwLCJYXzIiXSxbMywwLCJYXzMiXSxbMiwxLCJYXzIiXSxbMywxLCJYXzMiXSxbMCwxLCJcXFNpZ21hXnstMX0oWF8zKSJdLFsxLDEsIkNfe2ZfM30iXSxbMCwxLCJmXzEiXSxbMSwyLCJmXzIiXSxbMiwzLCJmXzMiXSxbNCw1LCJmXzMiXSxbNyw0XSxbNiw3XSxbMCw2LCJcXGRlbHRhIiwyXSxbMSw3XSxbMiw0LCIiLDEseyJsZXZlbCI6Miwic3R5bGUiOnsiaGVhZCI6eyJuYW1lIjoibm9uZSJ9fX1dLFszLDUsIiIsMSx7ImxldmVsIjoyLCJzdHlsZSI6eyJoZWFkIjp7Im5hbWUiOiJub25lIn19fV1d
        \[\begin{tikzcd}
            {X_0} & {X_1} & {X_2} & {X_3} \\
            {\Sigma^{-1}(X_3)} & {Y} & {X_2} & {X_3}
            \arrow["{f_1}", from=1-1, to=1-2]
            \arrow["{f_2}", from=1-2, to=1-3]
            \arrow["{f_3}", from=1-3, to=1-4]
            \arrow["{f_3}", from=2-3, to=2-4]
            \arrow[from=2-2, to=2-3]
            \arrow[from=2-1, to=2-2]
            \arrow["\delta"', from=1-1, to=2-1]
            \arrow[from=1-2, to=2-2]
            \arrow[Rightarrow, no head, from=1-3, to=2-3]
            \arrow[Rightarrow, no head, from=1-4, to=2-4]
        \end{tikzcd}\]

        Where the bottom row is distinguished.
    }
    \end{enumerate}

    Which all gives rise to the same subset of \( \Hom_{\Tc}(\Sigma(X_0), X_3) \).
\end{definition}

\begin{remark}
    Note that every \( Y \) in the above definition is isomorphic to a cone of the respective map.
\end{remark}

\subsection{Counterexample from article (TODO)}
\subsection{Counterexample from article (TODO)}

For \( R = kG \) a group algebra, one has that in \( \StMod(R) \), the toda bracket is \emph{not neccesarily} equal to the toda bracket with the negative triangulation on \( \StMod(R) \).

\begin{definition}
    Negative triangulation TODO
\end{definition}

Let \( R = \Fb_3C_3 \) with \( g \in C_3 \) a generator.



\section{Examples of Toda brackets}
\section{Examples of Toda brackets}

Let \( R = \Fb_2C_2 \), with \( g \in C_2 \) being the generator.

Then \( J = \tuple{1 + g} \) is the only ideal of \( R \).

First note that \( J \) is not projective, since the short exact sequence:

\begin{center}
	\begin{tikzpicture}
		\diagram{m}{1cm}{1cm}{
			J & R & J \\
		};
	
		\draw[math]
			(m-1-1) edge[hook] node {\iota} (m-1-2)
			(m-1-2) edge[two heads] node {\phi} (m-1-3);
	\end{tikzpicture}
\end{center}

Where, \( \iota \) is the inclusion, and \( \phi \) is the map:

\begin{align*}
    \phi: R &\to J \\
    0, 1 + g &\mapsto 0 \\
    1, g &\mapsto 1 + g
\end{align*}

But \( \phi \) does not split, since \( \iota \) is the only monomorphism of \( J \) into \( R \), but composes to \( 0 \) with \( \phi \). Therefore \( J \) cannot be projective, since every epimorphism into a projective module splits.

Furthermore, since \( \phi \) is an epimorphism with kernel \( J \), one gets from the third isomorphism theorem that \( \frac{R}{J} \cong J \).

\subsection{Example 1}

The first example I calculated was the Toda bracket of the following diagram:

\begin{center}
	\begin{tikzpicture}
		\diagram{m}{1cm}{1cm}{
				J & J & J & J \\
		};

		\draw[math]
			(m-1-1) edge node {\Id} (m-1-2)
			(m-1-2) edge node {0} (m-1-3)
			(m-1-3) edge node {\Id} (m-1-4);
	\end{tikzpicture}
\end{center}

The cone of \( \Id_J \) is the pushout of \( (\Id_J, \iota_J) \), where \( \iota_J \) is a monomorphism into the injective/projective \( R \). This is by construction \( \frac{R \oplus J}{\sim} \), where \( (0, 1+g) \sim (1+g, 0) \). This is isomorphic to \( R \).

The supension of \( J \) is the cokernel of \( \iota_J \). In \( \Mod(R) \), this is isomorphic to \( \frac{R}{J} \).

Using the cofiber-cofiber definition, one gets the following diagram:

\begin{center}
	\begin{tikzpicture}
		\diagram{m}{1cm}{1cm} {
			J & J & R & {\frac{R}{J}} \\
			J & J & J & J \\
		};

		\draw[math]
			(m-1-1) edge node {\Id} (m-1-2)
				edge[equal] (m-2-1)
			(m-1-2) edge (m-1-3)
				edge[equal] (m-2-2)
			(m-1-3) edge (m-1-4)
				edge node {\rho} (m-2-3)
			(m-1-4) edge node {\psi} (m-2-4)

			(m-2-1) edge node {\Id} (m-2-2)
			(m-2-2) edge node {0} (m-2-3)
			(m-2-3) edge node {\Id} (m-2-4);
	\end{tikzpicture}
\end{center}

However, since the diagram is in \( \StMod(R) \), one has that \( R \cong 0 \), and therefore \( \rho = 0 \). And from earlier one has that \( \frac{R}{J} \cong J \).

This gives the following diagram in \( \StMod(R) \):

\begin{center}
	\begin{tikzpicture}
		\diagram{m}{1cm}{1cm} {
			J & J & 0 & J \\
			J & J & J & J \\
		};

		\draw[math]
			(m-1-1) edge node {\Id} (m-1-2)
				edge[equal] (m-2-1)
			(m-1-2) edge node {0} (m-1-3)
				edge[equal] (m-2-2)
			(m-1-3) edge node {0} (m-1-4)
				edge node {0} (m-2-3)
			(m-1-4) edge node {\psi} (m-2-4)

			(m-2-1) edge node {\Id} (m-2-2)
			(m-2-2) edge node {0} (m-2-3)
			(m-2-3) edge node {\Id} (m-2-4);
	\end{tikzpicture}
\end{center}

But this shows that any endomorphism on \( J \) makes the rightmost square commute, and is therefore in the Toda bracket (up to pre-composition by an isomorphism \( \frac{R}{J} \to J \)) of \( \toda{\Id_J, 0, \Id_J} \). But since we only care about Toda brackets up to pre/post-composition of isomorphism, one can write that \( \toda{\Id_J, 0, \Id_J} = \StMod(R)(J, J) \).

\subsection{Example 2}

Want to show that \( \toda{\Id_J, \Id_J, \Id_J} = \emptyset \) since there is no distinguished triangle one can put in the cofiber-cofiber definition such that the squares commute.

The cone of the identity map is \( 0 \) as seen before (and in general in a triangulated category, the cone of an isomorphism is isomorphic to \( 0 \) (TODO)), and \( \Sigma(J) \cong J \):

\begin{center}
	\begin{tikzpicture}
		\diagram{m}{1cm}{1cm} {
			J & J & 0 & J \\
			J & J & J & J \\
		};

		\draw[math]
			(m-1-1) edge node {\Id} (m-1-2)
				edge[equal] (m-2-1)
			(m-1-2) edge node {0} (m-1-3)
				edge[equal] (m-2-2)
			(m-1-3) edge node {0} (m-1-4)
				edge[squiggly] node {0} (m-2-3)
			(m-1-4) edge node {\psi} (m-2-4)

			(m-2-1) edge node {\Id} (m-2-2)
			(m-2-2) edge node {\Id} (m-2-3)
			(m-2-3) edge node {\Id} (m-2-4);
	\end{tikzpicture}
\end{center}

There is no squiggly map in the diagram above that can make the middle square commute, unless \( J \cong 0 \).

Therefore \( \toda{\Id_J, \Id_J, \Id_J} = \emptyset \). And one has that (TODO: Cite, Supervisors said so) in general, for the Toda bracket \( \toda{f_3, f_2, f_1} \), if \( f_3 \circ f_2 \neq 0 \) or \( f_2 \circ f_1 \neq 0 \), then the Toda bracket will be empty:

\begin{theorem}
	Let \( f_1, f_2, f_3 \) be three composable maps in any triangulated category, \( \Tc \):

	\begin{center}
		\begin{tikzpicture}
			\diagram{m}{1cm}{1cm} {
				X_1 & X_2 & X_3 & X_4 \\
			};

			\draw[math]
				(m-1-1) edge node {f_1} (m-1-2)
				(m-1-2) edge node {f_2} (m-1-3)
				(m-1-3) edge node {f_3} (m-1-4);
		\end{tikzpicture}
	\end{center}

	such that \( f_2 \circ f_1 \neq 0 \) or \( f_3 \circ f_2 \neq 0 \).

	Then \( \toda{f_3, f_2, f_1} = \emptyset \).
\end{theorem}
\begin{proof}
	Assume that \( \toda{f_3, f_2, f_1} \neq \emptyset \). Then from the definition of Toda bracket there exists maps \(\alpha, \beta, \phi, \psi \) such that the following diagram commutes:

	\begin{center}
		\begin{tikzpicture}
			\diagram{m}{1cm}{1cm} {
				X_1 & X_2 & C_{f_1} & \Sigma(X_1) \\
				X_1 & X_2 & X_3 & X_4 \\
			};

			\draw[math]
				(m-1-1) edge node {f_1} (m-1-2)
					edge[equal]	(m-2-1)
				(m-1-2) edge node {\alpha} (m-1-3)
					edge[equal] (m-2-2)
				(m-1-3) edge node {\beta} (m-1-4)
					edge node {\phi} (m-2-3)
				(m-1-4) edge node {\psi} (m-2-4)

				(m-2-1) edge node {f_1} (m-2-2)
				(m-2-2) edge node {f_2} (m-2-3)
				(m-2-3) edge node {f_3} (m-2-4);
		\end{tikzpicture}
	\end{center}

	Split the proof into two different contradictions:

	\begin{itemize}
		\item{
			Case 1:

			Assume that \( f_2 \circ f_1 \neq 0 \). Then one has that
			\[
				\phi \circ \alpha \circ f_1 = \phi \circ 0 = 0
			\]
			since \( \alpha, f_1 \) are two composable maps from the same distinguished triangle. (TODO: Cite or prove?)

			But from commutativity of the diagram one also has
			\[
				\phi \circ \alpha \circ f_1 = f_2 \circ f_1 \neq 0.
			\]
			Which is a contradiction, so \( f_2 \circ f_1 = 0 \).
		}
		\item{
			Case 2:

			Assume that \( f_3 \circ f_2 \neq 0 \). Then one has that
			\[
				0 = \psi \circ 0 = \psi \circ \beta \circ \alpha = f_3 \circ \phi \circ \alpha = f_3 \circ f_2 \neq 0.
			\]
			Which is also a contradiction.
		}
	\end{itemize}

	Therefore both \( f_2 \circ f_1 = 0 \) and \( f_3 \circ f_2 = 0 \) if \( \toda{f_3, f_2, f_1} \neq \emptyset \), which is contrapositive to the statement in the theorem.

\end{proof}

\subsection{Example 3}

Let \( \toda{f_3, 0, \Id} \) be a well defined Toda bracket. Then one has the following diagram:

\begin{center}
	\begin{tikzpicture}
		\diagram{m}{1cm}{1cm} {
			{X_1} & {X_1} & 0 & {\Sigma(X_1)} \\
			{X_1} & {X_1} & {X_2} & {X_3} \\
		};

		\draw[math]
			(m-1-1) edge node {\Id} (m-1-2)
				edge[equal] (m-2-1)
			(m-1-2) edge (m-1-3)
				edge[equal] (m-2-2)
			(m-1-3) edge (m-1-4)
				edge (m-2-3)
			(m-1-4) edge node {\phi} (m-2-4)

			(m-2-1) edge node {\Id} (m-2-2)
			(m-2-2) edge node {0} (m-2-3)
			(m-2-3) edge node {f_3} (m-2-4);
	\end{tikzpicture}
\end{center}

Here one has that any possible \( \psi: \Sigma(X_1) \to X_3 \) will make the right square commute. Therefore \( \toda{f_3, 0, \Id} = \Tc(\Sigma(X_1), X_3) \).

\subsection{Example 4}

Want to find the Toda bracket of the following maps using the cofiber-cofiber definition:

\begin{center}
	\begin{tikzpicture}
		\diagram{m}{1cm}{1cm} {
			J & {J \oplus J} & {J \oplus J} & J \\
		};

		\draw[math]
			(m-1-1) edge node {\begin{psmallmatrix} 1 \\ 1 \end{psmallmatrix}} (m-1-2)
			(m-1-2) edge node {\begin{psmallmatrix} 1 & 1 \\ 1 & 1 \end{psmallmatrix}} (m-1-3)
			(m-1-3) edge node {\begin{psmallmatrix} 1 & 1 \end{psmallmatrix}} (m-1-4);
	\end{tikzpicture}
\end{center}

First need to find the standard triangle of \( \begin{psmallmatrix} 1 \\ 1 \end{psmallmatrix}: J \to J \oplus J \):

The cone is defined as the pushout of the following diagram:

\begin{center}
	\begin{tikzpicture}
		\diagram{m}{1cm}{1cm} {
			J & R \\
			J \oplus J \\
		};

		\draw[math]
			(m-1-1) edge[hook] node {\iota_J} (m-1-2)
				edge[swap] node {\begin{psmallmatrix} 1 \\ 1 \end{psmallmatrix}} (m-2-1);
	\end{tikzpicture}
\end{center}

Where \( \iota_J \) is a monomorphism into a injective module. Here, this map is chosen to be the only monomorphism from \( J \to R \), namely \( \iota \).

In the category of modules, the pushout becomes:

\begin{center}
	\begin{tikzpicture}
		\diagram{m}{1cm}{1cm} {
			J & R \\
			J \oplus J & \frac{J \oplus J \oplus R}{\sim} \\
		};

		\draw[math]
			(m-1-1) edge[hook] node {\iota_J} (m-1-2)
				edge[swap] node {\begin{psmallmatrix} 1 \\ 1 \end{psmallmatrix}} (m-2-1)
			(m-1-2) edge node {\gamma} (m-2-2)

			(m-2-1) edge[hook] node {\rho} (m-2-2);
	\end{tikzpicture}
\end{center}

Where \( (1 + g, 1 + g, 0) \sim (0, 0, 1 + g) \).

The map \( \rho \) is given as the composition:
\begin{center}
	\begin{tikzpicture}
		\diagram{m}{1cm}{1cm} {
			J \oplus J & J \oplus J \oplus R & \frac{J \oplus J \oplus R}{\sim} \\
		};

		\draw[math]
			(m-1-1) edge[hook] node {i} (m-1-2)
			(m-1-2) edge[two heads] node {\pi} (m-1-3);
	\end{tikzpicture}
\end{center}

Where \( i \) is the embedding, and \( \pi \) is the quotient epimorphism.

One can check that the pushout is isomorphic to \( J \oplus R \) via the map:

\begin{align*}
	\alpha: \frac{J \oplus J \oplus R}{\sim} &\to J \oplus R \\
	(0, 0, r) &\mapsto (0, r) \\
	(1 + g, 0, r) &\mapsto (1 + g, r) \\
	(0, 1 + g, r) &\mapsto (1 + g, r) \\
	(1+ + g, 1 + g, r) &\mapsto (0, r) \\
\end{align*}

Therefore, by checking the map \( \alpha \circ \rho \) one can see that it becomes the map:
\[ 
	(\begin{psmallmatrix}
		1 & 1 \\
	\end{psmallmatrix}, 0):  J \oplus J \to J \oplus R
\]

Doing a similar argument for \( \gamma \), one gets that it is simply the embedding \( \pi \).

Furthermore, the map \( J \oplus R \to \Sigma(J) \cong \frac{R}{J} \) is given as the unique pushout map \( \beta \), satisfying the following commutative diagram:

\begin{center}
	\begin{tikzpicture}
		\diagram{m}{1cm}{2cm} {
			J & R & \frac{R}{J} \\
			J \oplus J & J \oplus R \\
		};

		\draw[math]
			(m-1-1) edge[hook] node {\iota_J} (m-1-2)
				edge[swap] node {\begin{psmallmatrix} 1 \\ 1 \end{psmallmatrix}} (m-2-1)
			(m-1-2) edge[two heads] node {\pi_J} (m-1-3)
				edge node {\pi} (m-2-2)

			(m-2-1) edge[hook] node {(\begin{psmallmatrix} 1 & 1 \\ \end{psmallmatrix}, 0)} (m-2-2)
				edge[curve={height=60pt}] node {0} (m-1-3)
			(m-2-2) edge node {\beta} (m-1-3);
	\end{tikzpicture}
\end{center}

One can check that a candidate for the map \( \beta \) is \( (0, \pi) \). And since \( \beta \) is unique, this is the only map making the diagram commute.

Therefore the standard triangle becomes:

\begin{center}
	\begin{tikzpicture}
		\diagram{m}{1cm}{2cm} {
			J & J \oplus J & J \oplus R & \frac{R}{J} \\
		};

		\draw[math]
			(m-1-1) edge node {\begin{psmallmatrix} 1 \\ 1 \end{psmallmatrix}} (m-1-2)
			(m-1-2) edge node {(\begin{psmallmatrix} 1 & 1 \\ \end{psmallmatrix}, 0)} (m-1-3)
			(m-1-3) edge node {(0, \pi)} (m-1-4);
	\end{tikzpicture}
\end{center}

However, since the objects and morphisms are in \( \StMod(R) \), one has that \( R \cong 0 \), and by using the isomorphism \( \frac{R}{J} \cong J \), it becomes:

\begin{center}
	\begin{tikzpicture}
		\diagram{m}{1cm}{1cm} {
			J & J \oplus J & J & J \\
		};

		\draw[math]
			(m-1-1) edge node {\begin{psmallmatrix} 1 \\ 1 \end{psmallmatrix}} (m-1-2)
			(m-1-2) edge node {\begin{psmallmatrix} 1 & 1 \\ \end{psmallmatrix}} (m-1-3)
			(m-1-3) edge node {0} (m-1-4);
	\end{tikzpicture}
\end{center}

Using the cofiber-cofiber definition, one gets the following commutative diagram:

\begin{center}
	\begin{tikzpicture}
		\diagram{m}{1cm}{1cm} {
			J & {J \oplus J} & J & J \\
			J & {J \oplus J} & {J \oplus J} & J \\
		};

		\draw[math]
			(m-1-1) edge node {\begin{psmallmatrix} 1 \\ 1 \end{psmallmatrix}} (m-1-2)
				edge[equal] (m-2-1)
			(m-1-2) edge node {{\begin{psmallmatrix} 1 & 1 \end{psmallmatrix}}} (m-1-3)
				edge[equal] (m-2-2)
			(m-1-3) edge node {0} (m-1-4)
				edge node {\phi} (m-2-3)
			(m-1-4) edge node {\psi} (m-2-4)

			(m-2-1) edge node {\begin{psmallmatrix} 1 \\ 1 \end{psmallmatrix}} (m-2-2)
			(m-2-2) edge node {\begin{psmallmatrix} 1 & 1 \\ 1 & 1 \end{psmallmatrix}} (m-2-3)
			(m-2-3) edge node {\begin{psmallmatrix} 1 & 1 \end{psmallmatrix}} (m-2-4);
	\end{tikzpicture}
\end{center}

Where the top row is distinguished.

Firstly, using the fact that \( \begin{psmallmatrix}
		1 & 1 \\
	\end{psmallmatrix}: J \oplus J \to J \) is an epimorphism, one gets that \( \phi \circ \begin{psmallmatrix}
		1 & 1 \\
	\end{psmallmatrix} = \begin{psmallmatrix}
		1 \\ 1 \\
	\end{psmallmatrix} \begin{psmallmatrix}
		1 & 1 \\
	\end{psmallmatrix} \implies \phi = \begin{psmallmatrix}
		1 \\ 1 \\
	\end{psmallmatrix} \) by the epimorphism property.

And secondly one has that \( \begin{psmallmatrix}
		1 & 1 \\
	\end{psmallmatrix} \circ \phi = \begin{psmallmatrix}
		1 & 1 \\
	\end{psmallmatrix} \circ \begin{psmallmatrix}
		1 \\ 1 \\
	\end{psmallmatrix} = 0 \), so the Toda-bracket is non-empty.

Finally one can see that for any \( \psi: J \to J \), the right square will commute, and so the toda bracket becomes \( \StMod(R)(J, J) \).

\section{Projective and Injective classes}
\section{Projective and Injective classes}

\begin{definition}[Projective class, projective object] \label{def:projective_class}
    Let \( \Tc \) be a triangulated category.

    Let \( (\Pc, \Nc) \) be a tuple where \( \Pc \) is a class of objects, and \( \Nc \) is a class of morphisms, satisfying the following properties:

    \begin{enumerate}
        \item {Let \( f \in \Tc(X, Y) \).
        
        Then \( f \in \Nc \) if and only if for all \( P \in \Pc \) one has that \( f_*: \Tc(P, X) \to \Tc(P, Y) \) is the zero map.}

        \item {Let \( P \in \Tc \).
        
        Then \( P \in \Pc \) if and only if for all \( X, Y \in \Tc \), for all \( f \in \Tc(X, Y) \intersect \Nc \), one has that \( f_*: \Tc(P, X) \to \Tc(P, Y) \) is the zero map.}

        \item {For every \( X \in \Tc \) there exists objects \( Y \in \Tc \) and \( P \in \Pc \) along with a morphism \( f \in \Tc(X, Y) \) such that there exists a distinguished triangle on the form \( P \to X \stackrel{f}{\to} Y \to \Sigma(P) \).}
    \end{enumerate}

    Then \( (\Pc, \Nc) \) is called a \emph{projective class in \( \Tc \)}, and an object \( P \in \Pc \) is called \emph{projective}. % Projective over a projective class? TODO
\end{definition}

\begin{definition}[Stable projective class]
    Let \( (\Pc, \Nc) \) be a projective class in \( \Tc \).

    If for any \( P \in \Pc \) and \( n \in \Zb \) one has that \( \Sigma^n(P) \in \Pc \).
    
    Then \( (\Pc, \Nc) \) is called a \emph{stable projective class}.
\end{definition}

% Projective class stable under coproduct and retracts? TODO

\begin{definition}[\( \Pc \)-epic, \( \Pc \)-monic]
    Let \( f \in \Tc(X, Y) \) and let \( (\Pc, \Nc) \) be a projective class in \( \Tc \).

    Then one has the following definitions:

    \begin{enumerate} % surjective or epimorphism? Injective or monomorphism?
        \item {If for all \( P \in \Pc \), one has that \( f_*: \Tc(P, X) \to \Tc(P, Y) \) is surjective.
        
        Then \( f \) is called \emph{\( \Pc \)-epic}.}
        
        \item {If for all \( P \in \Pc \), one has that \( f_*: \Tc(P, X) \to \Tc(P, Y) \) is injective.
        
        Then \( f \) is called \emph{\( \Pc \)-monic}.}
    \end{enumerate}
\end{definition}

% Equivalent to cofiber map being P-null, or fiber map being P-null. TODO

\begin{definition}[Injective class, injective object]
    Let \( \Tc \) be a triangulated category.

    If the tuple \( (\Ic, \Nc) \) is a projective class in \( \Tc^{op} \).
    
    Then \( (\Ic, \Nc) \) is called an \emph{injective class in \( \Tc \)}, and an object \( I \in \Ic \) is called \emph{injective}.
\end{definition}

% Explicit definition. TODO

% Stable under products and retracts? TODO.

\begin{definition}[Stable injective class]
    Let \( (\Ic, \Nc) \) be an injective class in \( \Tc \).

    If for any \( I \in \Ic \) and \( n \in \Zb \) one has that \( \Sigma^n(I) \in \Ic \).
    
    Then \( (\Ic, \Nc) \) is called a \emph{stable injective class}.
\end{definition}

\begin{definition}[\( \Ic \)-monic, \( \Ic \)-epic]
    Let \( f \in \Tc(X, Y) \) and let \( (\Ic, \Nc) \) be an injective class in \( \Tc \).

    Then one has the following definitions:

    \begin{enumerate} % surjective or epimorphism? Injective or monomorphism?
        \item {If for all \( I \in \Ic \), one has that \( f_*: \Tc(I, X) \to \Tc(I, Y) \) is surjective.
        
        Then \( f \) is called \emph{\( \Ic \)-monic}.}
        
        \item {If for all \( I \in \Ic \), one has that \( f_*: \Tc(I, X) \to \Tc(I, Y) \) is injective.
        
        Then \( f \) is called \emph{\( \Ic \)-epic}.}
    \end{enumerate}
\end{definition}

% Equivalent to fiber map being I-null, or cofiber map being I-null. TODO

% Remark -> Connection to lifitng and extension property.

% Convention neccesary?

\begin{definition}[Adams resolution w.r.t. a projective class] % Need stable? TODO
    Let \( (\Pc, \Nc) \) be a projective class in \( \Tc \). Let \( X_0 \in \Tc \).

    Then given objects and morphisms in \( \Tc \) fitting in the following diagram:

    \begin{center}
        \begin{tikzpicture}
            \diagram{m}{1cm}{1cm} {
                X_0 && X_1 && X_2 && X_3 & \dots \\
                & P_0 && P_1 && P_2 \\
            };

            \draw[math]
                (m-1-1) edge node {i_0} (m-1-3)
                (m-1-3) edge node {i_1} (m-1-5)
                    edge[suspension] node {\delta_0} (m-2-2)
                (m-1-5) edge node {i_2} (m-1-7)
                    edge[suspension] node {\delta_1} (m-2-4)
                (m-1-7) edge (m-1-8)
                    edge[suspension] node {\delta_2} (m-2-6)

                (m-2-2) edge node {p_0} (m-1-1)
                (m-2-4) edge node {p_1} (m-1-3)
                (m-2-6) edge node {p_2} (m-1-5);
        \end{tikzpicture}
    \end{center}

    Where every \( i_n \in \Nc \), and \( P_n \in \Pc \), and marked arrows denote degree shifting maps, with \( \delta_n \in \Tc(X_{n + 1}, \Sigma(P_n)) \). And where every triangle on the form:

    \begin{center}
        \begin{tikzpicture}
            \diagram{m}{1cm}{1cm} {
                P_n & X_n & X_{n + 1} & \Sigma(P_n) \\
            };

            \draw[math]
                (m-1-1) edge node {p_n} (m-1-2)
                (m-1-2) edge node {i_n} (m-1-3)
                (m-1-3) edge node {\delta_n} (m-1-4);
        \end{tikzpicture}
    \end{center}

    is distinguished.

    This is called an \emph{Adams resolution of \( X_0 \) with respect to a projective class \( (\Pc, \Nc) \)}.
\end{definition}

\begin{definition}[Adams resolution w.r.t. an injective class] \label{def:adams_resolution_injective_class} % Need stable? TODO
    Let \( (\Ic, \Nc) \) be an injective class in \( \Tc \). Let \( Y_0 \in \Tc \).

    Then given objects and morphisms in \( \Tc \) fitting in the following diagram:

    \begin{center}
        \begin{tikzpicture}
            \diagram{m}{1cm}{1cm} {
                Y_0 && Y_1 && Y_2 && Y_3 & \dots \\
                & I_0 && I_1 && I_2 \\
            };

            \draw[math]
                (m-1-1) edge[swap] node {p_0} (m-2-2)
                (m-1-3) edge[swap] node {i_0} (m-1-1)
                    edge[swap] node {p_1} (m-2-4)
                (m-1-5) edge[swap] node {i_1} (m-1-3)
                    edge[swap] node {p_2} (m-2-6)
                (m-1-7) edge[swap] node {i_2} (m-1-5)
                (m-1-8) edge (m-1-7)

                (m-2-2) edge[suspension] node[swap] {\delta_0} (m-1-3)
                (m-2-4) edge[suspension] node[swap] {\delta_1} (m-1-5)
                (m-2-6) edge[suspension] node[swap] {\delta_2} (m-1-7);
        \end{tikzpicture}
    \end{center}

    Where every \( i_n \in \Nc \), and \( I_n \in \Ic \), and marked arrows denote degree shifting maps, with \( \delta_n \in \Tc(I_n, \Sigma(Y_{n + 1})) \). And where every triangle on the form:

    \begin{center}
        \begin{tikzpicture}
            \diagram{m}{1cm}{2cm} {
                \Sigma^{-1}(I_n) & Y_{n + 1} & Y_n & I_n \\
            };

            \draw[math]
                (m-1-1) edge node {\Sigma^{-1}(\delta_s)} (m-1-2)
                (m-1-2) edge node {i_n} (m-1-3)
                (m-1-3) edge node {p_n} (m-1-4);
        \end{tikzpicture}
    \end{center}

    is distinguished.

    This is called an \emph{Adams resolution of \( Y_0 \) with respect to an injective class \( (\Ic, \Nc) \)}.
\end{definition}

\begin{theorem} % Stable? TODO
    Let \( (\Pc, \Nc) \) be a projective class in \( \Tc \).

    Then for any object \( X \in \Tc \), there exists an Adams Resolution of \( X \) with respect to the projective class \( (\Pc, \Nc) \).
\end{theorem}
\begin{proof}
    Let \( X_0 = X \).

    Then by \autoref{def:projective_class} property 3, one has that there exists an object \( X_1 \in \Tc \) and object \( P_0 \in \Pc \), and three morphisms \( i_0, p_0, \delta_0 \), such that the following is a distinguished triangle:
    \[
        P_0 \stackrel{p_0}{\longrightarrow} X_0 \stackrel{i_0}{\longrightarrow} X_1 \stackrel{\delta_0}{\longrightarrow} \Sigma(P_0)
    \]

    TODO: Finish
\end{proof}

\begin{theorem} \label{thm:exists_adams_resolution_injective_class} % Stable? TODO
    Let \( (\Ic, \Nc) \) be an injective class in \( \Tc \).

    Then for any object \( Y \in \Tc \), there exists an Adams resolution of \( Y \) with respect to the injective class \( (\Ic, \Nc) \).
\end{theorem}
\begin{proof}
    TODO
\end{proof}

% Be specific of the use of the isomorphism in (\delta_s)_* TODO
\begin{construction} \label{construction:adams_spectral_sequence} 
    Let \( (\Ic, \Nc) \) be an injective class in \( \Tc \).

    For any \( Y \in \Tc \), one can construct an Adams resolution of \( Y \) with respect to the injective class \( (\Ic, \Nc) \) by \autoref{thm:exists_adams_resolution_injective_class}, using the notation from \autoref{def:adams_resolution_injective_class}.

    For any \( X \in \Tc, t \in \Zb, s \in \Nb_0 \), let:
    \[ 
        \phi: \Tc(\Sigma^{t - s}(X), \Sigma(Y_{s + 1})) \stackrel{\cong}{\longrightarrow} \Tc(\Sigma^{t - s - 1}(X), Y_{s + 1}) 
    \]
    Denote the natural isomorphism from \( \Sigma \)'s automorphism property.

    Furthermore, let
    \begin{align*}
        (i_s)_*: \Tc(\Sigma^{t - s}(X), Y_s) &\to \Tc(\Sigma^{t - s}(X), Y_{s - 1}) \\
        (p_s)_*: \Tc(\Sigma^{t - s}(X), Y_s) &\to \Tc(\Sigma^{t - s}(X), I_s) \\
        (\phi \circ \delta_s)_*: \Tc(\Sigma^{t - s}(X), I_s) &\to \Tc(\Sigma^{t - s - 1}(X), Y_{s + 1})
    \end{align*}
    be maps given by post-composition by \( i_s, p_s, \phi \circ \delta_s \) respectively.

    Let \( i, p, \delta \) be morphisms that fit in the following diagram:

    \begin{center}
        \begin{tikzpicture}
            \diagram{m}{2cm}{1cm} {
                \directsum_{s \in \Nb_0, t \in \Zb}\Tc(\Sigma^{t - s}(X), Y_s) && \directsum_{s \in \Nb_0, t \in \Zb}\Tc(\Sigma^{t - s}(X), Y_s) \\
                &  \directsum_{s \in \Nb_0, t \in \Zb}\Tc(\Sigma^{t - s}(X), I_s) \\
            };

            % Very sus 0-length arrow created when using anchors.
            % tips=proper or use "to" instead of "edge" (only on final arrow, or else there is no head).
            \draw[math, tips=proper]
                (m-1-1) edge node {i} (m-1-3)
                (m-1-3) edge node {p} (m-2-2.north east)

                (m-2-2.north west) edge node {\delta} (m-1-1);
        \end{tikzpicture}
    \end{center}

    Where \( i \) is a direct sum of either \( 0 \) on the coordinates with \( s = 0 \), or \( (i_s)_* \) for some \( s \neq 0 \), depending on what fits. Likewise, \( p \) is a direct sum of \( (p_s)_* \), and \( \delta \) is a direct sum of \( (\phi \circ \delta_s)_* \), for \( s \) such that it fits.
\end{construction}

\begin{theorem}
    The diagram in \autoref{construction:adams_spectral_sequence} is an exact couple.
\end{theorem}
\begin{proof} % Simplify proof, not dependant on Adams resolution structure at all? TODO
    Want to prove that the diagram is exact in the three corners.

    Firstly, note that since \( i, p, \delta \) are all direct summands of maps, it is sufficient to check that it is exact for every ``level'' of the direct sum.

    Therefore, fix any \( t \in \Zb \) and \( s \in \Nb_0 \).

    Focusing on the top right corner one needs to show that the maps \( i \) and \( p \) are exact given these \( t \) and \( s \).

    Firstly note that
    \begin{center}
        \begin{tikzpicture}
            \diagram{m}{1cm}{1cm} {
                \Sigma^{-1}(I_s) & Y_{s+1} & Y_s & I_s \\
            };

            \draw[math]
                (m-1-1) edge node {\Sigma^{-1}(\delta_s)} (m-1-2)
                (m-1-2) edge node {i_s} (m-1-3)
                (m-1-3) edge node {p_s} (m-1-4);
        \end{tikzpicture}
    \end{center}
    is a distinguished triangle by \autoref{def:adams_resolution_injective_class}. Then one can construct the following long exact sequence:
    \begin{center}
        \begin{tikzpicture}
            \diagram{m}{1cm}{1cm} {
                \dots & \Tc(\Sigma^{t-s}(X), Y_{s+1}) & \Tc(\Sigma^{t-s}(X), Y_s) & \Tc(\Sigma^{t-s}(X), I_s) & \dots \\
            };

            \draw[math]
                (m-1-1) edge (m-1-2)
                (m-1-2) edge node {(i_s)_*} (m-1-3)
                (m-1-3) edge node {(p_s)_*} (m-1-4)
                (m-1-4) edge (m-1-5);
        \end{tikzpicture}
    \end{center}

    Which is exactly how the maps \( i \) and \( p \) would interact on the coordinate with \( s \) and \( t \) fixed, and is therefore exact.

    Focusing on the bottom corner, it is very similar:

    The following triangle is distinguished
    \begin{center}
        \begin{tikzpicture}
            \diagram{m}{1cm}{1cm} {
                Y_{s+1} & Y_s & I_s & \Sigma(Y_{s+1}) \\
            };

            \draw[math]
                (m-1-1) edge node {i_s} (m-1-2)
                (m-1-2) edge node {p_s} (m-1-3)
                (m-1-3) edge node {\delta_s} (m-1-4);
        \end{tikzpicture}
    \end{center}
    since it is a left-rotated version of the distinguished triangle above.

    Again, this gives rise to the following long exact sequence
    \begin{center}
        \begin{tikzpicture}
            \diagram{m}{1cm}{1cm} {
                \dots & \Tc(\Sigma^{t-s}(X), Y_s) & \Tc(\Sigma^{t-s}(X), I_s) & \Tc(\Sigma^{t-s}(X), \Sigma(Y_{s + 1})) & \dots \\
            };

            \draw[math]
                (m-1-1) edge (m-1-2)
                (m-1-2) edge node {(p_s)_*} (m-1-3)
                (m-1-3) edge node {(\delta_s)_*} (m-1-4)
                (m-1-4) edge (m-1-5);
        \end{tikzpicture}
    \end{center}
    Changing the right map to \( (\phi \circ \delta_s)_* \), one gets the following sequence
    \begin{center}
        \begin{tikzpicture}
            \diagram{m}{1cm}{1cm} {
                \dots & \Tc(\Sigma^{t-s}(X), Y_s) & \Tc(\Sigma^{t-s}(X), I_s) & \Tc(\Sigma^{t-s-1}(X), Y_{s + 1}) & \dots \\
            };

            \draw[math]
                (m-1-1) edge (m-1-2)
                (m-1-2) edge node {(p_s)_*} (m-1-3)
                (m-1-3) edge node {\phi \circ (\delta_s)_*} (m-1-4)
                (m-1-4) edge (m-1-5);
        \end{tikzpicture}
    \end{center}
    Which is still exact in the middle, because \( \ker((\delta_s)_*) = \ker((\phi \circ \delta_s)_*) \), since \( \phi \) is an isomorphism.

    And since the middle part is exactly how the maps \( p \) and \( \delta \) would meet for given \( s \) and \( t \), the diagram is also exact at the bottom.

    Lastly at the top left corner one has to divide the problem into two cases:

    If \( s \neq 0 \):

    Then the following triangle is distinguished and exists since \( s \geq 1 \)
    \begin{center}
        \begin{tikzpicture}
            \diagram{m}{1cm}{1cm} {
                I_{s - 1} & \Sigma(Y_s) & \Sigma(Y_{s - 1}) & \Sigma(I_{s - 1}) \\
            };

            \draw[math]
                (m-1-1) edge node[above=5pt] {\delta_{s - 1}} (m-1-2)
                (m-1-2) edge node[above=5pt] {\Sigma(i_{s - 1})} (m-1-3)
                (m-1-3) edge node[above=5pt] {\Sigma(p_{s - 1})} (m-1-4);
        \end{tikzpicture}
    \end{center}
    because it is the same as the triangles above, but with the indexes reduced by one.

    Using this, one can construct the following long exact sequence
    \begin{center}
        \begin{tikzpicture}
            \diagram{m}{1cm}{1cm} {
                \dots & \Tc(\Sigma^{t-s}(X), I_{s - 1}) & \Tc(\Sigma^{t-s}(X), \Sigma(Y_s)) & \Tc(\Sigma^{t - s}(X), \Sigma(Y_{s - 1})) & \dots \\
            };

            \draw[math]
                (m-1-1) edge (m-1-2)
                (m-1-2) edge node[above=5pt] {(\delta_{s - 1})_*} (m-1-3)
                (m-1-3) edge node[above=5pt] {(\Sigma(i_{s - 1}))_*} (m-1-4)
                (m-1-4) edge (m-1-5);
        \end{tikzpicture}
    \end{center}

    Let \( \psi \) be the natural isomorphism
    \[
        \psi: \Tc(\Sigma^{t-s}(X), \Sigma(Y_{s - 1})) \stackrel{\cong}{\longrightarrow} \Tc(\Sigma^{t - s - 1}(X), Y_{s - 1}).
    \]

    Then by changing the maps, one gets the following sequence
    \begin{center}
        \begin{tikzpicture}
            \diagram{m}{1cm}{1cm} {
                \dots & \Tc(\Sigma^{t-s}(X), I_{s - 1}) & \Tc(\Sigma^{t - s - 1}(X), Y_s) & \Tc(\Sigma^{t - s - 1}(X), Y_{s - 1}) & \dots \\
            };

            \draw[math]
                (m-1-1) edge (m-1-2)
                (m-1-2) edge node[above=5pt] {\phi \circ (\delta_{s - 1})_*} (m-1-3)
                (m-1-3) edge node[above=5pt] {\psi \circ (\Sigma(i_{s - 1}))_* \circ \phi^{-1}} (m-1-4)
                (m-1-4) edge (m-1-5);
        \end{tikzpicture}
    \end{center}
    Where the middle part is still exact, because
    % Need lemma for last part?
    \begin{align*}
        \im((\delta_{s-1})_*) &= \ker((\Sigma(i_{s-1}))_*) \\
        &\Updownarrow \\
        \phi(\im((\delta_{s-1})_*)) &= \phi(\ker((\Sigma(i_{s-1}))_*)) \\
        \im(\phi \circ (\delta_{s-1})_*) &= \phi(\ker((\Sigma(i_{s-1}))_*)) \\
        &= \ker((\Sigma(i_{s-1}))_* \circ \phi^{-1})
    \end{align*}
    and since \( \ker(\psi \circ (\Sigma(i_{s-1}))_* \circ \phi^{-1}) = \ker((\Sigma(i_{s-1}))_* \circ \phi^{-1}) \), since \( \psi \) is an isomorphism.

    Furthermore, by the definition of \( \phi \) and \( \psi \), one has the following commutative diagram
    \begin{center}
        \begin{tikzpicture}
            \diagram{m}{1cm}{1cm} {
                \Sigma^{t - s}( X ) & \Sigma( Y_s ) & \Sigma( Y_{s - 1} ) \\
                \Sigma^{t - s - 1}( X ) & Y_s & Y_{s - 1} \\
            };

            \draw[math]
                (m-1-1) edge node {\Sigma( f )} (m-1-2)
                    edge node {\eta_1} (m-2-1)
                (m-1-2) edge node {\Sigma( i_s )} (m-1-3)
                    edge node {\eta_2} (m-2-2)
                (m-1-3) edge node {\eta_3} (m-2-3)

                (m-2-1) edge node {f} (m-2-2)
                (m-2-2) edge node {i_s} (m-2-3);
        \end{tikzpicture}
    \end{center}
    Where the \( \eta_j \)'s are the natural isomorphisms that define both \( \phi \) and \( \psi \), i.e.
    \[
        \phi: \Sigma( f ) \mapsto f = \eta_2 \circ \Sigma( f ) \circ \eta_1^{-1}
    \]
    and
    \[
        \psi: \Sigma( g ) \mapsto g = \eta_3 \circ \Sigma( g ) \circ \eta_1^{-1}.
    \]

    Then look at \( \psi \circ (\Sigma(i_{s-1}))_* \circ \phi^{-1} \) pointwise
    \begin{align*}
        \psi \circ (\Sigma(i_{s-1}))_* \circ \phi^{-1} ( f ) &= \psi \circ (\Sigma(i_{s-1}))_* ( \eta_2^{-1} \circ f \circ \eta_1 ) \\
        &= \psi ( \Sigma(i_{s - 1}) \circ \eta_2^{-1} \circ f \circ \eta_1 ) \\
        &= \eta_3 \circ \Sigma(i_{s - 1}) \circ \eta_2^{-1} \circ f \circ \eta_1 \circ \eta_1^{-1} \\
        &= \eta_3 \circ \Sigma(i_{s - 1}) \circ \eta_2^{-1} \circ f \\
        \intertext{By commutativity of the above diagram, this becomes}
        &= i_{s - 1} \circ f
    \end{align*}

    Which means that \( \psi \circ (\Sigma(i_{s-1}))_* \circ \phi^{-1} = ( i_{s - 1} )_* \). One thereforore gets the following sequence that is identical to the one above
    \begin{center}
        \begin{tikzpicture}
            \diagram{m}{1cm}{1cm} {
                \dots & \Tc(\Sigma^{t-s}(X), I_{s - 1}) & \Tc(\Sigma^{t - s - 1}(X), Y_s) & \Tc(\Sigma^{t - s - 1}(X), Y_{s - 1}) & \dots \\
            };

            \draw[math]
                (m-1-1) edge (m-1-2)
                (m-1-2) edge node[above=5pt] {\phi \circ (\delta_{s - 1})_*} (m-1-3)
                (m-1-3) edge node[above=5pt] {i_{s - 1}} (m-1-4)
                (m-1-4) edge (m-1-5);
        \end{tikzpicture}
    \end{center}
    Which is exactly how the maps \( \delta \) and \( i \) would meet for given \( s \) and \( t \), and is exact in the middle, implying the top left corner of the above diagram would be exact.

    And finally, if \( s = 0 \):

    The image of \( (\phi \circ \delta_n)_* \) for any \( n \) would never have codomain \( Y_0 \), except for the zero map by the definition, and therefore there is no image onto \( \Tc(\Sigma^t( X ), Y_0) \). There is also no map \( i_n \) with domain \( \Tc(\Sigma^t( X ), Y_0) \), except for the zero map \( i_0 = 0 \). And so, it is exact by default.

    Therefore one has that the top left corner is also exact for the given \( t \) and \( s \).

    By conclusion, since the choice of \( s \) and \( t \) was arbitrary, every corner is exact for any \( s \in \Nb_0 \) and \( t \in \Zb \), and by the argument in the beginning, the theorem follows.
\end{proof}

\begin{definition}[Adams spectral sequence]
    Let \( (\Ic, \Nc) \) be a stable injective class in \( \Tc \).

    Then

    TODO: Finish
\end{definition}


\section{Massey product on a DG-category}
\subsection{What is a DG-category?}
% MS-Question: Is it fine to define over a commutative ring? -> Yes
% TODO: Fine to define with respect to non-commutative algebras? -> We shall see!
\begin{definition}
    Let \( A \) be an (unital) associative algebra over a commutative (unital) ring \( R \). Furthermore, let \( A \) have a graded ring structure. Let \( R \cdot 1 \subseteq A_0 \).

    Then \( A \) is called a \emph{graded algebra over \( R \)}.
\end{definition}

% MS-Question: Is the degree the ``biggest'' i such that a|_{A_i} is non-zero? -> Probably only defined for homogenous elements.
% -> No! Change the definitions TODO
\begin{notation}
    Let \( A \) be a graded algebra over \( R \).

    Then the degree of a \emph{homogenous} element \( a \in A \) is denoted \emph{\( \abs{a} \)}.
\end{notation}

\begin{definition}
    Let \( A \) be a graded algebra over \( R \).

    If there is a graded algebra over \( R \) -epimorphism of degree \( 1 \), denoted \( d_A \), (i.e. for any \( i \in \Zb \), \( d_A |_{A_i}: A_i \to A_{i + 1} \))
    with the following properties:
    \begin{enumerate}
        \item One has that \( d \circ d = 0 \).
        \item {
            For \( a, b \in A \) with \( a \) homogenous, one has that
            \[
                d\tuple{a \cdot b}
                =
                d\tuple{a} \cdot b + \tuple{-1}^{\abs{a}}a \cdot d\tuple{b}.
            \]
            }
    \end{enumerate}

    Then \( A \) is called a \emph{differentially graded algebra over \( R \)}.
\end{definition}

\begin{definition}
    Let \( A, B \) be two differentially graded algebras over \( R \).

    TODO: Tensor product
\end{definition}

% MS-Question: Should I specify this or go straight to just the definition of enrichment (where the monoid in the monoidal category defines the tensor product.)? If not, does the tensor product for DG-categories have a similar universal property as it has for modules?
\begin{remark}
    The composition maps in a algebra-enriched category factors through the tensor product... TODO
\end{remark}

% MS-Question: What is the correct terminology of a degree-preserving algebra homomorphism? 
    % -> Degree 0 homomorphism.
% MS-Question: Keller doesn't use "pre-additive", but rather "k-linear", what does that mean? 
    % -> Probably the same as in the lecture notes.
% MS-Question: Both Keller and Bondav--Kapranov use tensor product in the definition on composition, shouldn't it be (regular) product? Also, the composition order isthe other way around in both texts, why so? 
    % -> Tensor product is by the lecture notes argument, order is by convention. Most likely tensor product is symmetric, so it makes little difference.
% MS-Question: Is (associative) graded algebras over commutative rings an abelian category? 
    % -> Most likely.
% MS-Question: This definition of composition wont add degrees, it will only take the max of the degrees. What is the grading of direct sum? And if composition factors through tensor products, what is the grading of the tensor product and is it different than level-wise? 
    % -> Direct sum is probably level wise, and tensor product is usually ``diagonal''. The composition is probably only defined from the tensor product.
% TODO: Is d(\Id_A) = 0 a neccesary condition? Is \Id_A = 1_A? 
    % -> It follows from enrichment that the identity morhpism is the identity element. Therefore it is implied.

% MS-Question: Is this the definition of a DG-enhanced category?
\begin{definition}
    Let \( \Cc \) be a pre-additive category, where for any \( A, B, C \in \Cc \) one has that all of the following hold:
    \begin{enumerate}
        \item {
            \[
                \Cc\tuple{A, B}
            \]
            is a differentially graded algebra over \( R \).
        }
        \item {
            The composition map
            \[
                \Cc\tuple{B, C} \otimes \Cc\tuple{A, B}  \to \Cc\tuple{A, C}
            \]
            is a graded \( R \)-algebra homomorphism of degree \( 0 \).
        }
    \end{enumerate}
    Then \( \Cc \) is called a \emph{differentially graded category over \( R \)}.
\end{definition}

\begin{remark}
    At least one author (TODO:Ref) specified an additional property when defining a differentially graded category over \( R \), namely that for any element \( A \in \Cc \), one has \( d(\Id_A) = 0 \).

    However by (TODO:Sebastian said so for enrichments) one has that \( \Id_A \) is the identity element. Therefore the property is implied by the following argument:
    
    If one has that \( \Id_A \) is the identity element in the differentially graded algebra \( \Cc\tuple{A, A} \). Then \( \abs{\Id_A} = 0 \) and it follows that
    \[
        d(\Id_A) = d(\Id_A \cdot \Id_A) = d(\Id_A) \cdot \Id_A + (-1)^0 \Id_A \cdot d(\Id_A) = 2d(\Id_A)
    \]
    which implies that \( d(\Id_A) = 0 \).
\end{remark}


\subsection{What is a massey product?}
% MS-Question: Is H^* a full functor on a DG-category? -> It should be full.
% TODO: Decide on notation, class-notation or H^*-notation?
\begin{definition}
    Let \( \Cc \) be a differentially graded category over \( R \).

    Let the following be a diagram in \( \Cc \)
    \begin{center}
        \begin{tikzpicture}
            \diagram{m}{1cm}{1cm} {
                X_1 & X_2 & X_3 & X_4 \\
            };

            \draw[math]
                (m-1-1) edge node {f_1} (m-1-2)
                (m-1-2) edge node {f_2} (m-1-3)
                (m-1-3) edge node {f_3} (m-1-4);
        \end{tikzpicture}
    \end{center}

    Furthermore, let
    \[
        [f_1] = H^* \tuple{f_1} \in H^* \tuple{\Cc \tuple{X_1, X_2}},
    \]
    \[
        [f_2] = H^* \tuple{f_2} \in H^* \tuple{\Cc \tuple{X_2, X_3}},
    \]
    and
    \[
        [f_3] = H^* \tuple{f_3} \in H^* \tuple{\Cc \tuple{X_3, X_4}}.
    \]
    Then let:
    \[
        \set{
            [(-1)^{\abs{f_3} + \abs{f_2}}s f_1 + (-1)^{\abs{f_3} + 1} f_3 t]
            \mid
            d(s) = (-1)^{f_3 + 1} f_3 f_2, \quad
            d(t) = (-1)^{f_2 + 1} f_2 f_1
        }
    \]
    This is a subset of \( H^* \tuple{\Cc \tuple{X_1, X_4}} \), called the \emph{Massey product of \( f_3, f_2 \) and \( f_1 \)}, and is denoted as \( \toda{[f_3], [f_2], [f_1]} \).
\end{definition}

\begin{remark}
    If \( \Cc \) is a differentially graded category over \( R \) with the following diagram
    \begin{center}
        \begin{tikzpicture}
            \diagram{m}{1cm}{1cm} {
                X_1 & X_2 & X_3 & X_4 \\
            };

            \draw[math]
                (m-1-1) edge node {f_1} (m-1-2)
                (m-1-2) edge node {f_2} (m-1-3)
                (m-1-3) edge node {f_3} (m-1-4);
        \end{tikzpicture}
    \end{center}
    where \( \abs{f_1} = d_1, \abs{f_2} = d_2 \), and \( \abs{f_3} = d_3 \).

    Then one gets that
    
\end{remark}

\subsection{What is a pre-triangulated category?}
There are multiple definitions of an algebraic triangulated categoery. In this thesis, the definition used will be from TODO:Cite Jasso--Muro.

% TODO Cite: Jasso--Muro
\begin{definition}[\( \C_{\dg}(\Mod(R)) \)]
    Let \( R \) be a commutative ring with identity.

    Then let \emph{\( \C_{\dg}(\Mod(R)) \)} be a DG-category defined as follows
    \begin{enumerate}
        \item {
            \( \Obj(\C_{\dg}(\Mod(R))) := \Obj(\C(\Mod(R))) \).
        }
        \item {
            For \( A, B \in \C_{\dg}(\Mod(R)) \) let
            \[ 
                \tuple{\C_{\dg}(\Mod(R))(A, B)}_i := \bigoplus_{j \in \Zb} \Mod(R)(A_j, B_{j + i})
            \]
            with
            \[
                \C_{\dg}(\Mod(R))(A, B) = \bigoplus_{i \in \Zb} \tuple{\C_{\dg}(\Mod(R))(A, B)}_i.
            \]

            Let \( d_A \) and \( d_B \) be the differential of \( A \) and \( B \) as objects of \( \C(\Mod(R)) \), respectively.

            % TODO: This is slight abuse of notation (d_B and d_A are defined on A, B, not A_j, B_{j + i}). Should I fix?
            Let the differential of \( \C_{\dg}(\Mod(R))(A, B) \) be defined as follows
            \begin{align*}
                d_i: \tuple{\C_{\dg}(\Mod(R))(A, B)}_i &\to \tuple{\C_{\dg}(\Mod(R))(A, B)}_{i + 1} \\
                f &\mapsto d_B \circ f - (-1)^i f \circ d_A
            \end{align*}
        }
        \item {
            For \( A, B, C \in \C_{\dg}(\Mod(R)) \), let
            \[
                \circ_{\C_{\dg}(\Mod(R))}: \C_{\dg}(\Mod(R))(B, C) \otimes \C_{\dg}(\Mod(R))(A, B) \to \C_{\dg}(\Mod(R))(A, C)
            \]
            % TODO: Explain more?
            be defined as expected.
        }
    \end{enumerate}
\end{definition}

% TODO Cite: Jasso--Muro
% TODO: Elaborate on enriched functor?
\begin{definition}[DG-functor]
    An enriched functor between two DG-categories is called a \emph{DG-functor}.
\end{definition}

% MS-question: Remark below. -> Probably OK.
\begin{remark}
    % Bondal--Kapranov has as definition
    Enriched functor between DG-categories implies it preserves differentials and grading? TODO
\end{remark}

% TODO: DG-categories over the same ring?
% TODO: Need to show that it's a category?
% MS-Question: Correct? Berest--Mehrle (LN) has another def. -> Subscript dg betyr enriched.
\begin{notation}
    For two DG-categories \( \Ac, \Bc \) over the same commutative ring \( R \), let \( \Fun_{\dg}(\Ac, \Bc) \) denote the DG-category over \( R \) of all DG-functors from \( \Ac \) to \( \Bc \).

    TODO: Show DG-structure.
\end{notation}

\begin{definition}[Opposite DG-category]
    Let \( \Cc \) be a DG-category.

    Then let \( \Cc^{op} \) be the DG-category defined as follows
    \begin{enumerate}
        \item {
            \( \Obj(\Cc^{op}) := \Obj(\Cc) \)
        }
        \item {
            For \( A, B \in \Cc^{op} \), let \( \Cc^{op}(A, B) := \Cc(B, A) \).
        }
        \item {
            For \( A, B, C \in \Cc^{op} \), with \( f \in \Cc^{op}(B, C) \) and \( g \in \Cc^{op}(A, B) \) homogeneous elements of degree \( d_f \) and \( d_g \) respectively.

            Let composition be defined as
            \begin{align*}
                \circ_{\Cc^{op}}: \Cc^{op}(B, C) \otimes \Cc^{op}(A, B) &\to \Cc^{op}(A, C) \\
                f \otimes g &\mapsto (-1)^{d_f d_g} \circ_{\Cc} (g \otimes f)
            \end{align*}
            % TODO: Why can I say this? Need some statement saying all elementary tensors are a sum of homogeneous elementary tensors? As well as saying that the set of elementary tensors generate all elements in the tensor product?
            and extended to all other elementary tensors.
        }
    \end{enumerate}
\end{definition}

% TODO: Cite: Jasso--Muro
% No mention of the ring in the notation? -> Implied since DG-category is over a ring!
\begin{definition}[\( \dgMod_{\dg}(\Cc) \)]
    Let \( \Cc \) be a DG-category over \( R \).

    % TODO: Why "Right"?
    Then define the \emph{DG-category of (right) DG \( \Cc \)-modules} as
    \[
        \dgMod_{\dg}(\Cc) := \Fun_{\dg}(\Cc^{op}, \C_{\dg}(\Mod(R))).
    \]
    Objects in \( \dgMod_{\dg}(\Cc) \) are called \emph{DG-modules over \( \Cc \)}.
\end{definition}

% TODO: Should be true according to Jasso--Muro
% TODO: DG-category over R?
% Probably true by the definition considering Fun_dg is a DG cat.
\begin{proposition}
    \( \dgMod_{\dg}(\Cc) \) is a DG-category.
\end{proposition}
\begin{proof}
    TODO
\end{proof}

% TODO: Why is \Cc(-, A) a functor into \C_{\dg}(\Mod(R))?
\begin{definition}[DG Yoneda embedding]
    \label{def:DG_Yoneda_embedding}
    Let \( \Cc \) be a DG-category over \( R \).
    
    Then let \( \mathbf{h} \) be the functor defined as follows
    \begin{align*}
        \mathbf{h}: \Cc &\to \dgMod_{\dg}(\Cc) \\
        A &\mapsto \Cc(-, A)
    \end{align*}

    This functor is called the \emph{DG Yoneda embedding of \( \Cc \)}.
\end{definition}

% TODO: Add Yoneda embedding identifies \Cc with a full subcategory of \dgMod_dg(\Cc)?

% TODO: Could define this for Mod(R)-enriched categories?
\begin{definition}[0th cohomology category of a DG category]
    Let \( \Cc \) be a DG category over \( R \).

    Then let \( H^0(\Cc) \) be the following (enriched over \( \Mod(R) \) TODO) category defined as follows
    \begin{enumerate}
        \item {
            Let \( \Obj(H^0(\Cc)) := \Obj(\Cc) \).
        }
        \item {
            Let \( A, B \in H^0(\Cc) \).

            Then let \( H^0(\Cc)(A, B) := H^0(\Cc(A, B)) \).
        }
        \item {
            Let \( A, B, C \in H^0(\Cc) \) with \( f_1 \in H^0(A, B) \) and \( f_2 \in H^0(B, C) \).

            Then by \autoref{lem:massey_product_in_dg_cat/massey_product_definition/exist_lifting_h_star}, there exists \( g_1 \in \Cc(A, B) \), and \( g_2 \in \Cc(B, C) \) such that \( \class{g_1} = f_1 \) and \( \class{g_2} = f_2 \).

            % TODO: Slight abuse of notation taking a "class" of an element of a chain complex.
            % TODO: Need to show that this is well defined?
            Then let composition be defined on elementary tensors as follows
            \begin{align*}
                \circ_{H^0(\Cc)}: H^0(\Cc)(B, C) \otimes H^0(\Cc)(A, B) &\to H^0(\Cc)(A, C) \\
                f_2 \otimes f_1 &\mapsto \class{ \circ_{\Cc}(g_2 \otimes g_1) }
            \end{align*}
        }
    \end{enumerate}
\end{definition}

% TODO: What are the triangles? Is the shift functor correct on maps?
% TODO: Incorrect/abuse of notation, how does the shift work on maps?
\begin{theorem}
    Let \( \Cc \) be a DG-category over \( R \). Let \( \Sigma_{\C_{\dg}(\Mod(R))} \) be the shift functor on \( \C_{\dg}(\Mod(R)) \).

    Then \( H^0(\dgMod_{\dg}(\Cc)) \) is a triangulated category with the shift functor \( \Sigma(-) = \Sigma_{\C_{\dg}(\Mod(R))} \circ - \).
\end{theorem}
\begin{proof}
    TODO
\end{proof}

% MS-Question: Have seen definition of small category that is that the class of iso classes are small, not the class of objects. What is correct? Are they equivalent? -> Essentially small.

\begin{definition}[Acyclic DG-module]
    Let \( \Cc \) be a DG-category over a commutative ring (with identity) \( R \). Furthermore, let \( A \in \dgMod_{\dg}(\Cc) \) be a DG-module over \( \Cc \).

    Then \( A \) is called \emph{acyclic} if for any \( X \in \Cc \), one has that \( A(X) \in \C_{\dg}(\Mod(R)) \) is acyclic, i.e. \( H^*(A(X)) = 0 \).
\end{definition}

% TODO: Is \dgMod_{\dg}(\Cc) abelian?/Have kernels?
\begin{definition}[DG-projective module]
    Let \( P \in \dgMod_{\dg}(\Cc) \) be a DG-module.

    Then \( P \) is called a \emph{DG-projective module over \( \Cc \)} if:
    
    For any DG-module \( A \in \dgMod_{\dg}(\Cc) \) and any epimorphism \( f \in \dgMod_{dg}(\Cc)(A, P) \) where \( \ker(f) \in \dgMod_{\dg}(\Cc) \) is acyclic. Then \( f \) is split.
\end{definition}

% TODO: Various definitions and idiosyncracies. Which is correct?
    % Acyclic kernel? Projective objects? Spanned?
    % Krause 07 -> Compact objects, maybe more.
    % Keller 94 -> Another definition of derived DG category.
% TODO: Following def from Krause 07, but not explicitly written down. Is it correct?
\begin{definition}[Derived DG-category]
    Let \( \Cc \) be a DG-category.

    Then the \emph{derived DG-category} of \( \Cc \), denoted \( \D(\Cc) \), is defined as the full subcategory of \( H^0(\dgMod_{\dg}(\Cc)) \) spanned by the objects of \( \dgMod_{\dg}(\Cc) \) that are DG-projective.
\end{definition}

% MS-Question: What is coproduct for the derived category?
% Cite: Jasso--Muro p.31, only a statement, no proof
\begin{proposition}
    \( \D(\Cc) \) is closed under arbitrary coproduct.
\end{proposition}
\begin{proof}
    TODO
\end{proof}

% MS-Quastion: Why are small categories sometimes mentioned in def of derived category? Something to do with localization being well defined?

% MS-Question: Arbitrary coproducts <=> infinite coproducts? -> Need triangulated property for this to make sense.
% Cite: Krause 07 p. 29
\begin{definition}[Compact objects of a category]
    Let \( \Cc \) be a triangulated category with arbitrary coproduct. Let \( X \in \Cc \). 
    
    Then if \( X \) has the following property:
    
    For any index set \( I \) and any morphism \( f: X \to \coprod_{i \in I} Y_i \), there is a finite index set \( J \subseteq I \) such that \( f \) factors through \( \coprod_{j \in J} Y_j \).
    
    Then define \( X \) as a \emph{compact object in \( \Cc \)}.
\end{definition}

% TODO: Is the derived category triangulated? Need to be in order for previous def to apply.
\begin{definition}[Perfect derived DG-category \( \D^c(\Cc) \)]
    Let \( \D(\Cc) \) be the derived DG-category of \( \Cc \).

    Then define \( \D^c(\Cc) \) to be the full subcategory of \( \D(\Cc) \) consisting of all compact objects in \( \D(\Cc) \). This is called the \emph{perfect derived DG-category of \( \Cc \)}.
\end{definition}

% TODO: Cite: Jasso--Muro says so
\begin{proposition}
    \( \D^c(\Cc) \) is triangulated.
\end{proposition}
\begin{proof}
    TODO
\end{proof}

% TODO: Is this even a functor? Or well defined? Probably need to show well defined and that every morphism in H^0(A) has a representative in A.
\begin{definition}[\( H^0 \)-induced functor]
    \label{def:H^0-induced_functor}
    Let \( \Ac \) and \( \Bc \) be two DG-categories, and let \( F: \Ac \to \Bc \) be a functor between them.

    Then define the functor \( H^0(F) \) as follows:
    \begin{align*}
        H^0(F): H^0(\Ac) &\to H^0(\Bc) \\
        A &\mapsto F(A) \\
        (H^0(f): A \to B) &\mapsto (H^0(F(f)): F(A) \to F(B)) 
    \end{align*}

    This is called the \( H^0 \)-induced functor of \( F \).
\end{definition}

\begin{theorem}
    \autoref{def:H^0-induced_functor} is a well-defined functor.
\end{theorem}
\begin{proof}
    TODO
\end{proof}

% Want to show that H^0(h) has codomain D^c(\Cc)
\begin{remark}
    Let \( \mathbf{h}: \Cc \to \dgMod_{\dg}(\Cc) \) be the DG-Yoneda embedding from \autoref{def:DG_Yoneda_embedding}.

    Then for any \( A \in \Cc \), one has that \( H^0(\mathbf{h})(A) \) is both DG-projective and compact.
    
    TODO: SHOW!!!

    Therefore one has that the functor \( H^0(\mathbf{h}): H^0(\Cc) \to H^0(\dgMod_{\dg}(\Cc)) \) factors through \( \D^c(\Cc) \). Denote this functor with the same notation:
    \[
        H^0(\mathbf{h}): H^0(\Cc) \to \D^c(\Cc)
    \]
\end{remark}

\begin{remark}
    \( H^0(\mathbf{h}): H^0(\Cc) \to \D^c(\Cc) \) is fully faithful.

    TODO: Prove
\end{remark}

% TODO: Why need small? Probably something with derived.
% TODO: Heilt ordrett nesten frå Jasso--Muro 2023 p. 32, burde kanskje omformulera?
\begin{definition}[pre-triangulated DG-category]
    Let \( \Cc \) be a small DG-category.

    Then \( \Cc \) is called a \emph{pre-triangulated DG-category} if the image of the (fully faithful) functor \( H^0(\mathbf{h}): H^0(\Cc) \to \D^c(\Cc) \) is a triangulated subcategory of \( \D^c(Cc) \).
\end{definition}

\begin{definition}[Algebraic triangulated category]
    Let \( \Tc \) be a triangulated category.

    Then \( \Tc \) is called an \emph{algebraic triangulated category} if there exist a pre-triangulated DG-category, \( \Cc \), such that \( H^0(\Cc) \) is equivalent to \( \Tc \).
\end{definition}

\subsection{Why do Massey product and toda brackets intersect?}
% First need to extend massey-prod definition to H^0

% MS-Question: I Jasso--Muro så er dette berre definert for element av dgMod (Ingen subscript!)
% TODO: This is also the same shift functor that makes H^0(dgmodblabla) triangulated. Probably not neccesary to specify then.
\begin{proposition}
    \label{prop:H^i_dgmod_cong_H^0_with_shift}
    Let \( \Cc \) be a DG-category over \( R \). Let \( \Sigma \) be the shift functor on \( H^0(\dgMod_{\dg}(\Cc)) \). Let \( A, B \in \dgMod_{\dg}(\Cc) \).

    Then there is an isomorphism
    \[
        \phi: H^i(\dgMod_{\dg}(\Cc)(A, B)) \stackrel{\sim}{\to} H^0(\dgMod_{\dg}(\Cc))(A, \Sigma^i(B)).
    \]
\end{proposition}
\begin{proof}
    TODO
\end{proof}

% TODO: Show this is well defined!!
\begin{remark}
    \label{rem:H^0_into_H^*_inclusion}
    Let \( \Cc \) be a DG-category.

    There is a dense and faithful functor \( \iota: H^0(\Cc) \hookrightarrow H^*(\Cc) \) given by
    \begin{align*}
        \iota: H^0(\Cc) &\to H^*(\Cc) \\
        A &\mapsto A \\
        \iota_{A, B}: H^0(\Cc)(A, B) &\to H^*(\Cc)(A, B) \\
        f &\mapsto \tuple{\dots, 0, f, 0, \dots} \quad \text{\( f \) is in degree \( 0 \)}
    \end{align*}

    TODO: Show well defined.
\end{remark}

\begin{definition}[Massey product on \( H^0(\dgMod_{\dg}(\Cc)) \)]
    \label{def:massey_product_H^0(dgMod_dg(C))}
    Let \( \Cc \) be a DG-category. Let the following be a diagram in \( H^0(\dgMod_{\dg}(\Cc)) \)
    \begin{center}
        \begin{tikzpicture}
            \diagram{m}{1cm}{1cm} {
                X_1 & X_2 & X_3 & X_4 \\
            };

            \draw[math]
                (m-1-1) edge node {f_1} (m-1-2)
                (m-1-2) edge node {f_2} (m-1-3)
                (m-1-3) edge node {f_3} (m-1-4);
        \end{tikzpicture}
    \end{center}
    Using the functor \( \iota \) in \autoref{rem:H^0_into_H^*_inclusion} one can view the above diagram as a diagram in \( H^*(\dgMod_{\dg}(\Cc)) \) as follows
    \begin{center}
        \begin{tikzpicture}
            \diagram{m}{1cm}{1cm} {
                X_1 & X_2 & X_3 & X_4 \\
            };

            \draw[math]
                (m-1-1) edge node {\iota(f_1)} (m-1-2)
                (m-1-2) edge node {\iota(f_2)} (m-1-3)
                (m-1-3) edge node {\iota(f_3)} (m-1-4);
        \end{tikzpicture}
    \end{center}
    where all the maps are of degree \( 0 \).

    % MS-Question: Overly complicated and imprecise on domains and stuff.
    By \autoref{rem:massey_product_in_dg_cat/massey_product_definition/massey_product_sum_of_degrees} the massey product of these maps \( \massey{\iota(f_3), \iota(f_2), \iota(f_1)} \) only have non-zero components in \( H^{-1}(\dgMod_{\dg}(\Cc)(X_1, X_4)) \). Then, using the ismorphism \( \phi \) in \autoref{prop:H^i_dgmod_cong_H^0_with_shift} one has that \( \phi(\massey{\iota(f_3), \iota(f_2), \iota(f_1)}) \subseteq H^0(\dgMod_{\dg}(\Cc)(X_1, \Sigma^{-1}X_4)) \). This in turn means that \( \phi(\massey{\iota(f_3), \iota(f_2), \iota(f_1)}) \subseteq \im(\iota_{X_1, \Sigma^{-1}(X_4)}) \). But since \( \iota \) is a faithful and dense funcor, it follows that there is a unique subset \( M \subseteq H^0(\dgMod_{\dg}(\Cc))(X_1, \Sigma^{-1}(X_4)) \) such that \( \iota(M) = \phi(\massey{\iota(f_3), \iota(f_2), \iota(f_1)}) \).

    Then define this \( M \) as the \emph{massey product on \( H^0(\dgMod_{\dg}(\Cc)) \)}.
\end{definition}

% TODO: Specify that the functor is exact?
\begin{definition}[Massey product in an algebraic triangulated category]
    Let \( \Tc \) be an algebraic triangulated category, and let the following be a diagram in \( \Tc \)
    \begin{center}
        \begin{tikzpicture}
            \diagram{m}{1cm}{1cm} {
                X_1 & X_2 & X_3 & X_4 \\
            };

            \draw[math]
                (m-1-1) edge node {f_1} (m-1-2)
                (m-1-2) edge node {f_2} (m-1-3)
                (m-1-3) edge node {f_3} (m-1-4);
        \end{tikzpicture}
    \end{center}

    Since \( \Tc \) is algebraic, it is equivalent to \( H^0(\Cc) \) for some pre-triangulated DG-category \( \Cc \). Furthermore, since \( \Cc \) is a pre-triangulated DG-category, one has that \( H^0(\Cc) \) is equivalent to \( \im(H^0(\mathbf{h})) \). And since \( \im(H^0(\mathbf{h})) \) is a full subcategory of \( H^0(\dgMod_{\dg}(\Cc)) \), the diagram above can be looked at as a diagram in \( H^0(\dgMod_{\dg}(\Cc)) \).

    To recap, one has the relation:
    \[
        \Tc \cong H^0(\Cc) \cong  \stackrel{full}{\subseteq} \D^c(\Cc) \stackrel{full}{\subseteq} \D(\Cc) \stackrel{full}{\subseteq} H^0(\dgMod_{\dg}(\Cc))
    \]

    Let \( F \) denote the functor that takes \( F: \Tc \hookrightarrow H^0(\dgMod_{\dg}(\Cc)) \) by the maps above. Then one has that the above diagram in \( \Tc \) can be viewed as a diagram in \( H^0(\dgMod_{\dg}(\Cc)) \) as follows
    \begin{center}
        \begin{tikzpicture}
            \diagram{m}{1cm}{1cm} {
                F(X_1) & F(X_2) & F(X_3) & F(X_4). \\
            };

            \draw[math]
                (m-1-1) edge node {F(f_1)} (m-1-2)
                (m-1-2) edge node {F(f_2)} (m-1-3)
                (m-1-3) edge node {F(f_3)} (m-1-4);
        \end{tikzpicture}
    \end{center}
    
    On the above diagram one can take the massey product (as in \autoref{def:massey_product_H^0(dgMod_dg(C))}). This yields a subset \( \massey{F(f_3), F(f_2), F(f_1)} \subseteq H^0(\dgMod_{\dg}(\Cc))(F(X_1), \Sigma^{-1}(F(X_4))) \), and since \( \im(H^0(\mathbf{h})) \) is a full subcategory of \( H^0(\dgMod_{\dg}(\Cc)) \), one has that \( \massey{F(f_3), F(f_2), F(f_1)} \subseteq \im(H^0(\mathbf{h}))(F(X_1), \Sigma^{-1}(F(X_4))) \) and therefore isomorphic as a set to a subset of \( \Tc(X_1, \Sigma^{-1}(X_4)) \cong \Tc(\Sigma(X_1), X_4) \). This subset is denoted as \( \massey{f_3, f_2, f_1} \) and is called the \emph{massey product of \( \Tc \)}.
\end{definition}

\begin{theorem}
    Let \( \Tc \) be an algebraic triangulated category. Furthermore let the following be a diagram in \( \Tc \)
    \begin{center}
        \begin{tikzpicture}
            \diagram{m}{1cm}{1cm} {
                X_1 & X_2 & X_3 & X_4 \\
            };

            \draw[math]
                (m-1-1) edge node {f_1} (m-1-2)
                (m-1-2) edge node {f_2} (m-1-3)
                (m-1-3) edge node {f_3} (m-1-4);
        \end{tikzpicture}
    \end{center}
    % MS-question: Massey product ser stygt ut.
    Then \( \toda{f_3, f_2, f_1} = (-1)^{TODO} \massey{f_3, f_2, f_1} \).
\end{theorem}


\end{document}
