\documentclass[a4paper, 10pt]{article}

% Right-justify text. Creates some overfull hbox-es, and works weirdly with microtype.
% \usepackage[document]{ragged2e}

% Make language-specific tweaks, like changing section/theorem words or adding more hyphonation points.
\usepackage[english]{babel}

% For å kunna skriva æøå i tekstar MERK: Blir automatisk ubrukeleg med lualatex og fontspec
% \usepackage[utf8]{inputenc}

\usepackage[T1]{fontenc}

% Adds hyphenation points and aims to improve paragraph rendering.
% Use [activate=false] to disable some features that causes issues with ragged2e. Not sure if setting this option disables every feature in the package entirely, or if it's still worth keeping then.
\usepackage{microtype}

% Enables memoization
% extract=no means that latexmk need to do the extracting itself. This is controlled by the latexmkrc file.
\usepackage[extract=no]{memoize}

% Uncomment the following to recompile every tikz-diagram. NOTE: Recompile is a little borked with latexmk. TODO: Write issue to memoize.
% \mmzset{
%     recompile,
% }

% Fiksa margin
\usepackage[margin=2cm]{geometry}

% Fiksar datoformatet på tiitelen
\usepackage[ddmmyyyy]{datetime}

\usepackage{amssymb}

% For visse mattesymbol, typ \mathbb
\usepackage{amsmath}

% Bilete
\usepackage{graphicx}

% For kodesnuttar og resultat
% \usepackage{minted}

% Kan endra på korleis listar ser ut
\usepackage{enumitem}

% For autoref
\usepackage[hidelinks,colorlinks=true]{hyperref} 

% For fargar på ting ein referer til i autoref
\hypersetup{allcolors=[rgb]{0,0.31,0.62}}

% For not needing to compile twice with hyperref (?)
\usepackage{bookmark}

% For teorem, definisjon, bevis enviornments.
\usepackage{amsthm}

% For meir avanserte teoremkonstruksjonar.
\usepackage{thmtools}

% For psmallmatrix.
\usepackage{mathtools}

% For boksar rundt tekst.
\usepackage{tcolorbox}
\tcbuselibrary{skins} % For å ha meir fancy boksar, trengs for "enhanced".
\tcbuselibrary{breakable} % For å ha breakable boksar.

% For svgar
% \usepackage{svg}

% Set svg mappo
% \svgpath{svg/}

% Fjernar indents ved nye avsnitt, men gjer linjeavstanden kortare (Kanskje)
\usepackage{parskip}

% Lualatex font greie
\usepackage{fontspec}

\usepackage[warnings-off={mathtools-colon,mathtools-overbracket}]{unicode-math}

% TikZ!
\usepackage{tikz}

% Set fontar som blir brukt
\setmainfont{Latin Modern Roman}
\setmathfont{Latin Modern Math} % Dette er standardfonten
\setmathfont[range={\setminus}]{Asana Math} % Somehow setting this changes tikz-cd arrow style
% \setmainfont{Atkinson Hyperlegible}
% \setmainfont{GFS Neohellenic Math}
% \setmainfont{Fira Sans}
% \setmathfont{Fira Math}
% \setmathfont[range=\setminus]{Asana Math}

\usetikzlibrary{matrix}

\usetikzlibrary{cd}

% In order to remove 1 pixel line at end of equal arrows.
\usetikzlibrary{nfold}

% From quiver:
% `pathmorphing` is necessary to draw squiggly arrows.
\usetikzlibrary{decorations.pathmorphing}

% From quiver:
% `calc` is necessary to draw curved arrows.
\usetikzlibrary{calc}

% Externalize TikZ diagrams to save compilation time.
% NOTE: Doesn't work with tikz-cd.
% \usetikzlibrary{external}
% \tikzexternalize[prefix=tikz/]

% \usepackage{memoize}
% \mmzset{memo dir}

% Marius Thaule TikZ matrix.
\newcommand{\diagram}[3]{%
  \matrix[ampersand replacement = \&] (#1) [%
  matrix of math nodes,%
  row sep={#2},%
  column sep={#3},%
  text height=1.5ex,%
  text depth=0.25ex%
  ]%
}
% Modified to fix the distance between nodes. Useful for diagrams where there are diagonal arrows that should be paralell.
\newcommand{\diagramorigin}[3]{%
  \matrix[ampersand replacement = \&] (#1) [%
    matrix of math nodes,%
    row sep={#2},%
    column sep={#3, between origins},%
    text height=1.5ex,%
    text depth=0.25ex%
  ]%
}

% Set style of TikZ pictures to tikz-cd style.
% \tikzset{%
%   every picture/.append style={%
%     commutative diagrams/every diagram%
%   },%
%   math/.style = {%
%       commutative diagrams/every arrow,%
%       commutative diagrams/every label,%
%       execute at begin node=\(, execute at end node=\)%
%     },%
% }
\tikzset{%
  math/.style = {%
    auto,%
    ->,%
    font=\scriptsize,%
    execute at begin node=\(, execute at end node=\)%
  }%
}

% Shortcuts to tikz-cd styles.
\tikzset{hooked/.style = {commutative diagrams/tailed}}
\tikzset{dashed/.style = {commutative diagrams/dashed}}
\tikzset{two headed/.style = {commutative diagrams/two heads}}
\tikzset{equality/.style = {commutative diagrams/equal, nfold}}
\tikzset{squiggly/.style = {commutative diagrams/squiggly}}
\tikzset{marked/.style = {commutative diagrams/marking}}
\tikzset{maps to/.style = {commutative diagrams/maps to}}
\tikzset{tailed/.style = {commutative diagrams/tail}}

% From quiver:
% A TikZ style for curved arrows of a fixed height, due to AndréC.
\tikzset{curve/.style={settings={#1},to path={(\tikztostart)
    .. controls ($(\tikztostart)!\pv{pos}!(\tikztotarget)!\pv{height}!270:(\tikztotarget)$)
    and ($(\tikztostart)!1-\pv{pos}!(\tikztotarget)!\pv{height}!270:(\tikztotarget)$)
    .. (\tikztotarget)\tikztonodes}},
    settings/.code={\tikzset{quiver/.cd,#1}
        \def\pv##1{\pgfkeysvalueof{/tikz/quiver/##1}}},
    quiver/.cd,pos/.initial=0.35,height/.initial=0
}

% Needed for suspension style.
\usetikzlibrary{decorations.markings}

% Marks the arrows with a suspension style.
\tikzset{
  suspension/.style = {postaction = decorate,
      decoration = {
          markings,
          mark = at position 0.3 with {\draw[-] (0,-0.075) -- (0,0.075);}
      },
  },
}
    

% Ny type lista med ganske perfekt spacing
\newlist{plist}{enumerate}{5}
\setlist[plist]{align=left, itemindent = 0cm, labelsep = 0cm, labelindent = 0cm}
\setlist[plist,1]{label=\arabic*, font=\bf\Large}
\setlist[plist,2]{label*=.\arabic*, labelwidth=1.25cm, leftmargin=1.25cm}
\setlist[plist,3]{label*=.\arabic*, labelwidth=1.5cm, leftmargin=1.5cm}

% Teoremstil
\theoremstyle{plain}
\newtheorem{theorem}{Theorem}[subsection]
\newtheorem{proposition}[theorem]{Proposition}
\newtheorem{corollary}[theorem]{Corollary}
\newtheorem{lemma}[theorem]{Lemma}

% Definisjonstil
\theoremstyle{definition}
\newtheorem{definition}[theorem]{Definition}
\newtheorem{example}[theorem]{Example}
\newtheorem{remark}[theorem]{Remark}
\newtheorem{construction}[theorem]{Construction}
\newtheorem{notation}[theorem]{Notation}
\newtheorem{fact}[theorem]{Fact}
\newtheorem{question}[theorem]{Question}

% Fancy box engine?
% \newcommand{\createbox}[1]{
%     \tcolorboxenvironment{#1}{
%         breakable, % Makes boxes breakable.
%         enhanced jigsaw, % Required for making breaks fancy.
%         arc = 0mm, % Corners.
%         % sharp corners, % make the corners sharp
%         oversize, % Makes the box not "squish" the text, but rather extend the box into the margins.
%         colback = white, % Changes the background colour.
%         % parskip, % Behave better with parskip package (not sure what it does)
%         % beforeafter skip balanced = 0pt, % Change spacing before and after boxes.
%         parbox = false, % Make text formatting similar to the outside.
%         before upper=\vspace{-2\parskip}, % Fix for extra parskip that comes in the start of the box.
%     }
% }

% Simple box engine?
\newcommand{\createbox}[1]{
    \tcolorboxenvironment{#1}{
        breakable, % Makes boxes breakable.
        empty, % Empty skin.
        arc = 0mm, % Corners.
        % sharp corners, % make the corners sharp
        oversize, % Makes the box not "squish" the text, but rather extend the box into the margins.
        % colback = white, % Changes the background colour.
        % parskip, % Behave better with parskip package (not sure what it does)
        % beforeafter skip balanced = 0pt, % Change spacing before and after boxes.
        parbox = false, % Make text formatting similar to the outside.
        before upper=\vspace{-2\parskip}, % Fix for extra parskip that comes in the start of the box.
        borderline={1pt}{0pt}{black}
    }
}

\createbox{theorem}
\createbox{proposition}
\createbox{corollary}
\createbox{lemma}
\createbox{definition}
\createbox{example}
\createbox{remark}
\createbox{construction}
\createbox{notation}
\createbox{fact}
\createbox{question}

% TODO: Make proof connect to the theorem box. Maybe it will look better? Then it would be difficult to have text in between.
\createbox{proof}

% Blackboard shortcuts
\newcommand{\Fb}{{\mathbb{F}}}
\newcommand{\Nb}{{\mathbb{N}}}
\newcommand{\Qb}{{\mathbb{Q}}}
\newcommand{\Rb}{{\mathbb{R}}}
\newcommand{\Zb}{{\mathbb{Z}}}
\newcommand{\Mb}{{\mathbb{M}}}

% Caligraphy shortcuts
\newcommand{\Ac}{{\mathcal{A}}}
\newcommand{\Bc}{{\mathcal{B}}}
\newcommand{\Cc}{{\mathcal{C}}}
\newcommand{\Ic}{{\mathcal{I}}}
\newcommand{\Kc}{{\mathcal{K}}}
\newcommand{\Mc}{{\mathcal{M}}}
\newcommand{\Nc}{{\mathcal{N}}}
\newcommand{\Pc}{{\mathcal{P}}}
\newcommand{\Tc}{{\mathcal{T}}}

% Set management shortcuts
\newcommand{\intersect}{\mathop{\cap}\limits}
\newcommand{\union}{\mathop{\cup}\limits}
\newcommand{\directsum}{\mathop{\oplus}\limits}

% \newcommand{\Stmod}[1]{\stablemod\tuple{ #1 }}

% Shorthands
% \newcommand{\abs}[1]{ \lvert #1 \rvert }
% \newcommand{\set}[1]{ \left\{ #1 \right\} }
% \newcommand{\tuple}[1]{ \left( #1 \right) }
% \newcommand{\toda}[1]{ \langle #1 \rangle }
% \newcommand{\class}[1]{ \left[ #1 \right] }
% \newcommand{\massey}[1]{ \langle \! \langle #1 \rangle \! \rangle }

% SRC: https://tex.stackexchange.com/questions/94410/easily-change-behavior-of-declarepaireddelimiter
% \NewDocumentCommand\xDeclarePairedDelimiter{mmm}
%  {%
%   \NewDocumentCommand#1{som}{%
%    \IfNoValueTF{##2}
%     {\IfBooleanTF{##1}{#2##3#3}{\mleft#2##3\mright#3}}
%     {\mathopen{##2#2}##3\mathclose{##2#3}}%
%   }%
%  }
% \xDeclarePairedDelimiter{\set}{\lbrace}{\rbrace}

% Swaps * and non * version of \DeclarePairedDelimiter, but doesnt work.
% \newcommand\swapifbranches[3]{#1{#3}{#2}}
% \makeatletter
% \MHInternalSyntaxOn
% \patchcmd{\DeclarePairedDelimiterX}{\@ifstar}{\swapifbranches\@ifstar}{}{}
% \MHInternalSyntaxOff
% \makeatother

% Double barackets: https://tex.stackexchange.com/questions/79657/how-to-get-double-angle-bracket-without-using-mnsymbol-package
% Trenge Mathtools 1.30, eg har 1.29.... Ser bra ut, då.
% \makeatletter
% \newsavebox{\@bra}
% \newsavebox{\@brb}
% \DeclarePairedDelimiterX{\massey}[1]{.}{.}{%
%   \delimsize\langle%
%   \hspace*{0.3mm}\hspace*{0.55mm}\savebox{\@bra}{\(\displaystyle\left\langle\vphantom{#1}\right.\)}\hspace*{-1.035\wd\@bra}%
%   \delimsize\langle%
%   #1% 
%   \delimsize\rangle%
%   \hspace*{0.3mm}\hspace*{0.55mm}\savebox{\@brb}{\(\displaystyle\left.\vphantom{#1}\right\rangle\)}\hspace*{-1.035\wd\@brb}%
%   \delimsize\rangle
% }
% \makeatother


\DeclarePairedDelimiterX{\abs}[1]{\vert}{\vert}{#1}
\DeclarePairedDelimiterX{\set}[1]{\{}{\}}{#1}
\DeclarePairedDelimiterX{\tuple}[1]{(}{)}{#1}
\DeclarePairedDelimiterX{\toda}[1]{\langle}{\rangle}{#1}
\DeclarePairedDelimiterX{\class}[1]{[}{]}{#1}
% \DeclarePairedDelimiterX{\massey}[1]{\lmangle}{\rmangle}{#1}
\DeclarePairedDelimiterX{\massey}[1]{\langle \! \langle}{\rangle \! \rangle}{#1}

% Maybe use \bigl og \bigr instead?
% \renewcommand*{\tuple}{\tuple*} % Not working? Tex capacity exceeded?

% New math operators
\DeclareMathOperator{\Id}{Id}
\DeclareMathOperator{\StMod}{StMod}
\DeclareMathOperator{\Stmod}{Stmod}
\DeclareMathOperator{\Obj}{Obj}
\DeclareMathOperator{\Hom}{Hom}
\DeclareMathOperator{\Mod}{Mod}
\let\mod\relax
\DeclareMathOperator{\mod}{mod}
\DeclareMathOperator{\coker}{coker}
\DeclareMathOperator{\im}{im}
\DeclareMathOperator{\Ab}{Ab}
\DeclareMathOperator{\Fun}{Fun}
\DeclareMathOperator{\dg}{dg}
\DeclareMathOperator{\op}{op}
\DeclareMathOperator{\C}{\mathbf{C}}
\DeclareMathOperator{\D}{D}
\DeclareMathOperator{\K}{K}
\DeclareMathOperator{\dgMod}{dgMod}
\DeclareMathOperator{\dgM}{dgM}
\DeclareMathOperator{\dgFun}{dgFun}
\DeclareMathOperator*{\colim}{colim}

% Toda bracket subscrips
\DeclareMathOperator{\cc}{cc}
\DeclareMathOperator{\fc}{fc}
\DeclareMathOperator{\ff}{ff}


\title{Master Thesis Draft 2}
\author{
    \begin{tabular}{rl}
        Author:& Håvard Skjetne Lilleheie\\
        Supervisor:& Marius Thaule\\
        Co-Supervisor:& Sebastian H. Martensen
    \end{tabular}
}

\begin{document}

\maketitle

\tableofcontents

\section*{Abstract}
\addcontentsline{toc}{section}{Abstract}
The main purpose of this thesis is to prove the equality (up to a sign) between Toda brackets and Massey products in an algebraic triangulated category. \autoref{section:2} through \autoref{section:5} will consist of the necessary preliminaries for the proof, while \autoref{section:6} contains the proof itself. Finally, in \autoref{section:7} we will look at potential applications where the equality could be useful.


\section{Introduction}
\subsection{Motivation}
\label{subsection:1.1-Motivation}
Triangulated categories were discovered independently by Dieter Puppe and Jean-Louis Verdier in 1962 and 1963, respectively. Verdier introduced his axioms in his doctoral thesis, supervised by Alexander Grothendieck, leading to the attribution for the discovery of triangulated category sometimes being split three ways between Puppe, Verdier, and Grothendieck. Puppe discovered triangulated categories while looking at the stable homotopy category, while Verdier and Grothendieck were motivated from a more algebraic viewpoint by the derived category of an abelian category.

% TODO: Fix claims and references when finished.
As their discovery would imply, triangulated categories are useful in both algebra and topology. Typical algebraic examples of triangulated categories are: the derived categories of abelian category as mentioned above, the stable module category, as well as the homotopy category of chain complexes. On the other hand, typical topological examples of triangulated categories are: the stable homotopy category, and the Spanier--Whitehead category. In \autoref{section:2} we will discuss triangulated categories in more detail. This thesis will touch a little upon the Spanier--Whitehead category (\autoref{subsubsection:spanier_whitehead_cat}) and the chain homotopy category (\autoref{subsubsection:chain_homotopy_cat}), but will mainly focus on the stable module category (\autoref{subsubsection:stable_module_cat}) for use in examples throughout the thesis, along with a proof that it is a triangulated category (\autoref{example:stable_module_category_triangulated}).

% MS-Question: Dwyer-Kan influence on Shipley?
% TODO: Skriv meir samanhengande.
Toda brackets were initially introduced by Hiroshi Toda in 1962 for use in calculating homotopy groups of spheres. The definition of Toda brackets in any triangulated category is based on the work of Shipley in 2002 \cite[Definition A.2]{Shipley_2002}. She bases her definition on ideas from Cohen in 1968 \cite[Definition at the bottom of p. 308]{Cohen_1968}. Shipley's definition was further refined by Sagave in 2008 \cite[Remark 4.5]{Sagave_2008}. The definitions in this thesis is a more elegant formulation of Sagave's definitions by Christensen and Frankland in 2017 \cite[Definition 3.1]{Christensen-Frankland_2017}. The definitions and some examples of calculations will be discussed in \autoref{section:3}.

Massey products were introduced by William Schumacher Massey in 1958. They can be defined in a DG-category (\autoref{def:massey_product_dg_cat}) and further generalized to be defined in any algebraic triangulated category (\autoref{def:massey_product_alg_tri_cat}). Both DG-categories (\autoref{def:dg_cat}) and algebraic triangulated categories (\autoref{def:alg_tri_cat}) will be defined in this thesis. In general, it is known that computing Massey products is easier than computing Toda brackets TODO: CITE. Luckily, it turns out that in an algebraic triangulated category, Massey products and Toda brackets coincide with a small sign difference as follows
\[
    TODO,
\]
which will be proved in \autoref{theorem:massey_equals_toda}.

In \autoref{section:7} we will discuss further applications of this equality between Massey products and Toda brackets.

By NTNU, it is mandatory to include the following subsection.
\subsubsection{Sustainability statement}
While pure mathematics seemingly has no real world impact, it has been repeatedly proven that pure mathematics appear in other fields of study in the future and provides tools to solve problems that can appear in the real world. Topology in particular, has a deep history connected to optimization problems, and in newer times have been used in data analysis and other fields. This thesis could therefore bring future benefit to the physical sciences as well as data analysis which are key fields for future innovation. Given the potential widespread applicability of pure topology research, this thesis could contribute to goal number 3 for good health and well being, by being a tool that could be used in the medical field, as well as goal 9 for industry, innovation and infrastructure, by creating tools that could potentially be used to drive new innovations.

Of course, given the potential wide applicability of topological techniques in the future, it could contribute to other goals, but the two abovementioned goals, namely goal 3 and 9, are the ones where topology is currently being used to some degree.



\subsection{Notation}
This thesis will assume that the reader is already familiar with basic homological algebra as well as some module theory.

Some assumptions in results will be implied by definitions and not be explicitly stated in theorems. However, there are some conventions that are used throughout this thesis.

First, the following is the chain complex convention.
\begin{notation}
    \label{not:chain_complex}
    Every chain complex will have \emph{ascending} order. This is sometimes called a \emph{cochain complex} in other literature.

    Differentials are indexed according to the order of their domain, i.e., for a chain complex \( A = \tuple*{A_i}_{i \in \Zb} \), the \( n \)th differential has the same index as its domain, \( d_n: A_n \to A_{n + 1} \).
    
    We try to keep to the same convention of indexing with respect to the domain when considering morphisms between graded objects.
\end{notation}

Second, the module notation is as follows.
\begin{notation}
    \( \Mod(R) \) will refer to \( R \)-modules, and \( \mod(R) \) will refer to \emph{finitely generated} \( R \)-modules.
\end{notation}

Third, the following arrow conventions will be used.
\begin{notation}
    Tailed arrows \( (\rightarrowtail) \) represent monomorphisms, and double-headed arrows \( (\twoheadrightarrow) \) represent epimorphisms.
\end{notation}

Fourth, certain canonical isomorphisms will be omitted in the notation.
\begin{notation}
    \label{not:suppress_canonical_isomorphisms}
    We will mention when there is a canonical isomorphism, but will suppress them in calculations for clarity.

    For example, given a category \( \Cc \), \( \Cc(A, B) \) is \emph{canonically isomorphic} to \( \Cc^{\op}(B, A) \), since by definition \( \Cc^{\op}(B, A) = \Cc(A, B) \).

    However, in calculations we will write, for example, \( f \in \Cc(A, B) \iff f \in \Cc^{\op}(B, A) \).
\end{notation}

Fifth, we differ between ``morphisms'' and ``maps.''
\begin{notation}
    A ``morphism'' is a morphism in some category, while a ``map'' is a function on the underlying sets.
\end{notation}

\section{Triangulated categories}
As mentioned in the introduction, we have to define triangulated categories before we can talk about Massey products or Toda brackets in triangulated categories. This section contains a definition of triangulated categories as well as three examples of triangulated categories: The Spanier--Whitehead category, the chain homotopy category and the stable module category. The stable module category will be used for examples later on, and so it is stated in more detail than the other two examples.

\subsection{Definition}
A triangulated category is closely linked to some objects and morphisms making ``triangles'' and showing that there exist some class of triangles that satisfy certain properties. We start by defining what a triangle in an additive category is, as well as what a morphism between triangles is.

\begin{definition}[Triangles]
    \label{def:triangles}
    Let \( \Tc \) be an additive category, and \( \Sigma: \Tc \to \Tc \) be an additive auto-equivalence.

    A \emph{triangle in \( \Tc \)}, is a diagram in \( \Tc \) on the form
    \begin{center}
        \begin{tikzpicture}
            \diagram{m}{1cm}{1cm} {
                A \& B \& C \& \Sigma A, \\
            };

            \draw[math]
                (m-1-1) edge node {f} (m-1-2)
                (m-1-2) edge node {g} (m-1-3)
                (m-1-3) edge node {h} (m-1-4);
        \end{tikzpicture}
    \end{center}
    or shortened to just \( \tuple{f, g, h} \).

    A \emph{triangle morphism} from
    \begin{center}
        \begin{tikzpicture}
            \diagram{m}{1cm}{1cm} {
                A \& B \& C \& \Sigma A \&[-0.8cm] \text{to} \&[-0.8cm]  A' \& B' \& C' \& \Sigma A', \\
            };

            \draw[math]
                (m-1-1) edge node {f} (m-1-2)
                (m-1-2) edge node {g} (m-1-3)
                (m-1-3) edge node {h} (m-1-4)
                (m-1-6) edge node {f'} (m-1-7)
                (m-1-7) edge node {g'} (m-1-8)
                (m-1-8) edge node {h'} (m-1-9);
        \end{tikzpicture}
    \end{center}
    is a tuple of three morphisms \( (a, b, c) \) in \( \Tc \) such that the following diagram,
    \begin{center}
        \begin{tikzpicture}
            \diagram{m}{1cm}{1cm} {
                A \& B \& C \& \Sigma A \\
                A' \& B' \& C' \& \Sigma A', \\
            };

            \draw[math]
                (m-1-1) edge node {f} (m-1-2)
                    edge node {a} (m-2-1)
                (m-1-2) edge node {g} (m-1-3)
                    edge node {b} (m-2-2)
                (m-1-3) edge node {h} (m-1-4)
                    edge node {c} (m-2-3)
                (m-1-4) edge node {\Sigma a} (m-2-4)

                (m-2-1) edge node {f'} (m-2-2)
                (m-2-2) edge node {g'} (m-2-3)
                (m-2-3) edge node {h'} (m-2-4);
        \end{tikzpicture}
    \end{center}
    commutes.
    Furthermore, if \( a, b, \) and \( c \) are isomorphisms, then \( (a, b, c) \) is called an \emph{isomorphism of triangles} and the two triangles are said to be \emph{isomorphic}.
\end{definition}

We will use the following definition of a triangulated category.

\begin{definition}[Triangulated category]
    \label{def:triangulated_category}
    Let \( \Tc \) be an additive category, \( \Sigma: \Tc \to \Tc \) be an additive auto-equivalence, and \( \Delta \) be a class of triangles (\autoref{def:triangles}) in \( \Tc \).

    Furthermore, let \( \Tc \), \( \Sigma \) and \( \Delta \) satisfy the following four axioms:
    \begin{enumerate}[label={(\bfseries TR\arabic*)}]
        \item {
            The following three conditions must be satisfied:
            \begin{enumerate}
                \item {
                    For any \( f: X \to Y \) in \( \Tc \), there exist some object \( C_f \in \Tc \), called a \emph{cone of \( f \)}, along with morphisms \( \iota_f \) and \( \pi_f \), such that there is a triangle,
                    \begin{center}
                        \begin{tikzpicture}
                            \diagram{m}{1cm}{1cm} {
                                X \& Y \& C_f \& \Sigma X \\
                            };

                            \draw[math]
                                (m-1-1) edge node {f} (m-1-2)
                                (m-1-2) edge node {\iota_f} (m-1-3)
                                (m-1-3) edge node {\pi_f} (m-1-4);
                        \end{tikzpicture}
                    \end{center}
                    in \( \Delta \).

                    Any triangle of the above form is called a \emph{standard triangle of \( f \)}.
                }
                \item {
                    For any \( X \in \Tc \), the \emph{trivial triangle of \( X \)}, which is the following triangle
                    \begin{center}
                        \begin{tikzpicture}
                            \diagram{m}{1cm}{1cm} {
                                X \& X \& 0 \& \Sigma X \\
                            };

                            \draw[math]
                                (m-1-1) edge node {\Id_X} (m-1-2)
                                (m-1-2) edge (m-1-3)
                                (m-1-3) edge (m-1-4);
                        \end{tikzpicture}
                    \end{center}
                    is in \( \Delta \).
                }
                \item {
                    \( \Delta \) is closed under isomorphisms of triangles.
                }
            \end{enumerate}
        }
        \item {
            Any triangle
            \begin{diagramlabel}[\label{eq:tri_cat_def_right_rotation}]
                \begin{tikzpicture}
                    \diagram{m}{1cm}{1cm} {
                        A \& B \& C \& \Sigma A \\
                    };
        
                    \draw[math]
                        (m-1-1) edge node {f} (m-1-2)
                        (m-1-2) edge node {g} (m-1-3)
                        (m-1-3) edge node {h} (m-1-4);
                \end{tikzpicture}
            \end{diagramlabel}
            is in \( \Delta \) if and only if
            \begin{diagramlabel}[\label{eq:tri_cat_def_left_rotation}]
                \begin{tikzpicture}
                    \diagram{m}{1cm}{1cm} {
                        B \& C \& \Sigma A \& \Sigma B \\
                    };
        
                    \draw[math]
                        (m-1-1) edge node {g} (m-1-2)
                        (m-1-2) edge node {h} (m-1-3)
                        (m-1-3) edge node {-\Sigma f} (m-1-4);
                \end{tikzpicture}
            \end{diagramlabel}
            is in \( \Delta \). The triangle \autoref{eq:tri_cat_def_left_rotation} is called the \emph{left rotation} of \autoref{eq:tri_cat_def_right_rotation}, and \autoref{eq:tri_cat_def_right_rotation} is called the \emph{right rotation} of \autoref{eq:tri_cat_def_left_rotation}.
            
            Other names for this axiom include the \emph{right rotation axiom} for the \( (\Leftarrow) \) part of the axiom, and the \emph{left rotation axiom} for the \( (\Rightarrow) \) part of the axiom.
        }
        \item {
            Let the solid part of the following diagram
            \begin{center}
                \begin{tikzpicture}
                    \diagram{m}{1cm}{1cm} {
                        A \& B \& C \& \Sigma A \\
                        A' \& B' \& C' \& \Sigma A'. \\
                    };
        
                    \draw[math]
                        (m-1-1) edge node {f} (m-1-2)
                            edge node {a} (m-2-1)
                        (m-1-2) edge node {g} (m-1-3)
                            edge node {b} (m-2-2)
                        (m-1-3) edge node {h} (m-1-4)
                            edge[dashed] node {c} (m-2-3)
                        (m-1-4) edge node {\Sigma a} (m-2-4)
        
                        (m-2-1) edge node {f'} (m-2-2)
                        (m-2-2) edge node {g'} (m-2-3)
                        (m-2-3) edge node {h'} (m-2-4);
                \end{tikzpicture}
            \end{center}
            commute, and let both of the rows be triangles in \( \Delta \).
            
            Then there exists a morphism \( c: C \to C' \) such that \( (a, b, c) \) becomes a morphism of triangles.
        }
        \item {
            Let the solid part of the following diagram commute,
            \begin{center}
                \begin{tikzpicture}
                    \diagram{m}{1cm}{1cm} {
                        A \& B \& C \& \Sigma A \\
                        A \& D \& E \& \Sigma A \\
                        \& F \& F \& \Sigma B \\
                        \& \Sigma B \& \Sigma C, \\
                    };
        
                    \draw[math]
                        (m-1-1) edge node {f} (m-1-2)
                            edge[equality] (m-2-1)
                        (m-1-2) edge node {g} (m-1-3)
                            edge node {n} (m-2-2)
                        (m-1-3) edge node {h} (m-1-4)
                            edge[dashed] node {\phi} (m-2-3)
                        (m-1-4) edge[equality] (m-2-4)
        
                        (m-2-1) edge node {n \circ f} (m-2-2)
                        (m-2-2) edge node {j} (m-2-3)
                            edge node {m} (m-3-2)
                        (m-2-3) edge node {l} (m-2-4)
                            edge[dashed] node {\psi} (m-3-3)
                        (m-2-4) edge node {\Sigma f} (m-3-4)

                        (m-3-2) edge[equality] (m-3-3)
                            edge node[swap] {k} (m-4-2)
                        (m-3-3) edge node {k} (m-3-4)
                            edge node {\Sigma(g) \circ k} (m-4-3)

                        (m-4-2) edge node {\Sigma g} (m-4-3);
                \end{tikzpicture}
            \end{center}
            and let \( \tuple{f, g, h} \), \( \tuple{n, m, k} \), and \( \tuple{n \circ f, j, l} \) be three triangles in \( \Delta \).

            Then there exist some morphisms, \( \phi \) and \( \psi \), such that \( \tuple{\phi, \psi, (\Sigma g) \circ k} \) is a triangle in \( \Delta \) and the entire diagram, including \( \phi \) and \( \psi \), commute.
        }
    \end{enumerate}

    Then \( \tuple{ \Tc, \Sigma, \Delta } \), or shortened to just \( \Tc \), is called a \emph{triangulated category}, or a \emph{triangulation} of \( \Tc \). The functor \( \Sigma \) is called the \emph{shift} (or \emph{suspension}) \emph{functor} of \( \Tc \), and every triangle in \( \Delta \) is called a \emph{distinguished} (or \emph{exact}) \emph{triangle} in \( \Tc \). 
\end{definition}

A tuple \( \tuple{\Tc, \Sigma, \Delta} \) satisfying {\bf (TR1)} -- {\bf (TR3)} is called a \emph{pre-triangulated} category. This must not be confused with the later definition of a ``pre-triangulated DG-category'' (\autoref{def:pre-tri_dg_cat}), which is not generally a pre-triangulated category.

Although the definition is axiomatic and may seem arbitrary, there are surprisingly many categories that admit the structure of a triangulated category.

There are multiple different, but equivalent, definitions of a triangulated category with different axioms. An example of this is \cite[Definition 2.1]{May_2001}, where instead of what we call {\bf (TR4)}, he instead uses another axiom, closer to Verdier's original definition, which upon close inspection is equivalent (Assuming {\bf (TR1)}) to our {\bf (TR4)}.

Some axioms of a triangulated category are superfluous in the fact that they are implied by the other axioms. An example is that the axiom {\bf (TR3)} can be skipped if we still assume {\bf (TR1), (TR2)} and {\bf (TR4)}, as was also shown by \cite[Lemma 2.2]{May_2001}.

Another example of this is the following lemma, which is useful in proving that categories are triangulated. 
\begin{lemma}
    \label{lem:triangulated_category-TR2-only_one_rotation}
    Let \( \tuple{ \Tc, \Sigma, \Delta } \) satisfy the axioms {\bf (TR1)} and {\bf (TR3)} from \autoref{def:triangulated_category} as well as the left rotation axiom of {\bf (TR2)}.

    Then the right rotation axiom of {\bf (TR2)} is also satisfied.
\end{lemma}
For a proof of this see \cite[Lemma 2.4]{May_2001}.

There is also a special type of triangulation-preserving functor between triangulated categories.
\begin{definition}[Triangulated functor]
    \label{def:tri_functor}
    For \( F: \Tc \to \Yc \) an additive functor between two triangulated categories.

    If there is some natural transformation \( \eta: F \Sigma \to \Sigma F \) such that
    for a distinguished triangle in \( \Tc \),
    \begin{center}
        \begin{tikzpicture}
            \diagram{m}{1cm}{1cm} {
                A \& B \& C \& \Sigma A, \\
            };

            \draw[math]
                (m-1-1) edge node {f} (m-1-2)
                (m-1-2) edge node {g} (m-1-3)
                (m-1-3) edge node {h} (m-1-4);
        \end{tikzpicture}
    \end{center}
    the following triangle in \( \Yc \) is distinguished,
    \begin{center}
        \begin{tikzpicture}
            \diagram{m}{1cm}{1cm} {
                F A \& F B \& F C \& \Sigma F A, \\
            };

            \draw[math]
                (m-1-1) edge node {F f} (m-1-2)
                (m-1-2) edge node {F g} (m-1-3)
                (m-1-3) edge node {\eta \circ (F h)} (m-1-4);
        \end{tikzpicture}
    \end{center}
    then \( F \) is called a \emph{triangulated functor} (or \emph{exact functor}).

    Furthermore, if \( F \) is an equivalence, it is called a \emph{triangulated equivalence}.
\end{definition}

In particular, as is mentioned in \cite[p.4 Lemma]{Happel_1988}, if \( F \) is a triangulated equivalence between two triangulated categories, then \( F^{-1} \) is also a triangulated functor.

There is also the notion of a triangulated subcategory.
\begin{definition}[Triangulated subcategory]
    For \( \Yc \) an additive subcategory of a triangulated category \( \Tc \), if the following three statements are true, then \( \tuple{\Yc, \Sigma, \Delta'} \) is called a \emph{triangulated subcategory of \( \tuple{\Tc, \Sigma, \Delta} \)}:
    \begin{itemize}
        \item {
            \( \Yc \) is closed under \( \Sigma \), and \( \Sigma \) is an auto-equivalence on \( \Yc \),
        }
        \item {
            \( \Delta' \) is a subclass of \( \Delta \) consisting of triangles in \( \Yc \), and
        }
        \item {
            \( \tuple{\Yc, \Sigma, \Delta'} \) is a triangulated category.
        }
    \end{itemize}
\end{definition}


In the following three subsections we will consider three different triangulated categories, the first of which is inspired by topology, and the last two are inspired by algebra.
\subsection{The Spanier--Whitehead category}
\label{subsubsection:spanier_whitehead_cat}
An example of a triangulated category is the Spanier--Whitehead category. It is a \emph{topological triangulated category}. The definition of topological triangulated categories is ``… any triangulated category which is equivalent to a full triangulated subcategory of the homotopy category of a stable model category'' \cite[p.\ 6]{Schwede_2010}. In the same paragraph as the above quote, Schwede also explains why the Spanier--Whitehead category is topological. It will become clear that the Spanier--Whitehead category is an example of a triangulated category inspired by topology, and so the designation as a topological triangulated category fits (morally). However, the definition of a topological triangulated category is not relevant for this thesis, and so we will not go into further details. 
% Schwede says "For us" før sitatet. Betyr det at det ikkje er ein definitsjon, men blir implisert av definisjonen? Er homotopikategori av ein cofibration category betre def?

First, some notation.
\begin{notation}
    Let \( X \) and \( Y \) be two based topological spaces, with \( x_0 \) the basepoint of \( X \).

    We write \( \class*{ X, Y } \) for the set of basepoint-preserving homotopy classes of continuous basepoint-preserving functions from \( X \) to \( Y \).

    Let \( I \) denote the closed unit interval with basepoint \( 0 \).

    We write \( \Sigma X \) for the reduced suspension of \( X \), i.e., the quotient space
    \[
        X \times I / \sim
    \]
    where \( (x, 0) \sim (x_0, t) \sim (x, 1) \) for all \( x \in X \)  and \( t \in I \).
\end{notation}

We can now define the Spanier--Whitehead category. The triangulation will be described afterwards.

The Spanier--Whitehead category is motivated by the Freudenthal suspension theorem and historically led to the definition of the stable homotopy category. The reason the Freudenthal suspension theorem is so central to the Spanier--Whitehead category is the following corollary, whose proof can be found in \cite[Remark 5.2]{Daria_Bachelor}.

\begin{corollary}
    Let \( X \) and \( Y \) be based and finite CW-complexes, and let \( n, m \in \Nb \).
    
    Then the colimit
    \[
        \colim_{q \to \infty} \class*{ \Sigma^{n + q} X, \Sigma^{m + q} Y }
    \]
    is attained for a finite \( q \in \Nb \).
\end{corollary}

The above corollary of the Freudenthal suspension theorem is crucial to understanding why the Spanier--Whitehead category is defined as it is. The following definition is based on \cite[Definition 2]{Schwede_2010}.

\begin{definition}[Spanier--Whitehead category]
    \label{def:sw-cat}
    Let \( SW \) be the category with the following properties:
    \begin{enumerate}
        \item {
            Objects in \( SW \) are pairs consisting of a based CW-complex, \( X \), and an integer, \( n \), i.e., \( \tuple*{X, n} \).
        }
        \item {
            Let \( \Sigma \) denote the reduced suspension of a topological space.

            Morphisms in \( SW \) are the following colimits of abelian groups
            \[
                SW\tuple*{ (X, n), (Y, m) } := \colim_{\stackrel{q \to \infty}{q \geq \max(|n| + 2, |m|)}} \class*{ \Sigma^{n + q}(X), \Sigma^{m + q}(Y) }
            \]
        }
        \item {
            Composition is the induced colimit morphism from ordinary composition in each degree.
        }
    \end{enumerate}

    Then \( SW \) is called the \emph{Spanier--Whitehead category}.
\end{definition}

The reason for limiting the \( q \) in the colimit in part 2, is to make sure that the colimit is well-defined and taken over abelian groups. This is because \( \class*{\Sigma^n X, Y} \) is only an abelian group if \( n \geq 2 \). 

Composition is well-defined by functoriality of the reduced suspension functor.

It can be shown that the Spanier--Whitehead category is an additive category \cite[Proposition 5.7]{Daria_Bachelor}.

To define the triangulated structure of the Spanier--Whitehead category it is necessary to define the shift functor and the class of distinguished triangles.

The definition of the shift functor is dependent on the following property.
\begin{lemma}
    \label{lem:sw_colim_canonical_iso}
    There is a unique, canonical isomorphism,
    \[
        SW\tuple*{ (X, n), (Y, m) } \stackrel{\sim}{\to} SW\tuple*{ (X, n + 1), (Y, m + 1) }.
    \]
\end{lemma}
\begin{proof}
    From the definition of the colimit, \( SW\tuple*{ (X, n), (Y, m) } \) is a colimit of the same diagram as \( SW\tuple*{ (X, n + 1), (Y, m + 1) } \), and vice versa.
\end{proof}

% Marius: Her har eg fortrengt den kanoniske isomorfien.
The following defines the shift functor.
\begin{definition}[Shift functor in \( SW \)]
    \label{def:sw-shift}
    Let \( \Sigma_{SW} \) be the following assignment of objects and morphisms in \( SW \).

    For any \( (X, n) \in SW \), let
    \[
        \Sigma_{SW} (X, n) := (X, n + 1),
    \] 
    and for \( f \in SW((X, n), (Y, m)) \), let
    \[
        \Sigma_{SW} f := f,
    \]
    where we have implicitly applied the canonical isomorphism from the previous lemma.

    We will denote \( \Sigma_{SW} \) as simply \( \Sigma \) when it is clear from the context.
\end{definition}

The shift functor in \autoref{def:sw-shift} is an automorphism.

As the notation would imply, in \( SW \), we have that \( \Sigma_{SW} (X, n) = (X, n + 1) \cong ( \Sigma X, n ) \).

Before defining the distinguished triangles, we first define what a ``mapping cone'' is.
\begin{definition}[Mapping cone \( C_f \)]
    Let \( f: X \to Y \) be a based continuous map between based topological spaces, with \( x_0 \) being the basepoint of \( X \). Let \( I \) be the closed unit interval.

    Then let the \emph{mapping cylinder}, denoted \( M_f \), be the quotient space
    \[
        (X \times I) \sqcup Y / \sim
    \]
    with the relation \( (x, 0) = f(x) \) for all \( x \in X \).

    In addition, let the \emph{mapping cone}, denoted \( C_f \), be defined as the quotient space
    \[
        M_f / \sim
    \]
    with the relation \( (x, 1) = (x', 1) \) for all \( x, x' \in X \) as well as \( (x_0, t) \sim (x_0, 0) \) for all \( t \in I \).
\end{definition}

For a continuous based map \( f: X \to Y \), let \( \iota_f: Y \rightarrowtail C_f \) be the inclusion of \( Y \) into \( C_f \). In addition, let \( \pi_f: C_f \twoheadrightarrow \Sigma X \) map the subspace \( Y \) of \( C_f \) to \( x_0 \), and any point \( (x, t) \) in the \( X \times I \) subspace is mapped to \( (x, t) \in \Sigma X \).

The following definition of the distinguished triangles in \( SW \) omits a lot of details as it is not necessary for our purposes. More details can be found in \cite[Definition 4.7 and 5.8]{Daria_Bachelor}.

\begin{definition}[Distinguished triangles in \( SW \)]
    \label{def:sw-dist_triangles}
    Let \( \Delta \) be the collection of triangles in \( SW \) satisfying the following property.
    
    A triangle
    \[
        (X, n) \to (Y, m) \to (Z, l) \to \Sigma (X, n)
    \]
    is in \( \Delta \) if and only if there is some continuous based map \( f: A \to B \) between two based CW-complexes, and an even number \( k \), such that the following triangle in the homotopy category of based CW-complexes
    \[
        \Sigma^{n + k} X \to \Sigma^{m + k} Y \to \Sigma^{l + k} Z \to \Sigma^{n + k + 1} X
    \]
    is isomorphic to
    \begin{center}
        \begin{tikzpicture}
            \diagram{m}{1cm}{1cm} {
                A \& B \& C_f \& \Sigma A. \\
            };

            \draw[math]
                (m-1-1) edge node {[f]} (m-1-2)
                (m-1-2) edge node {[\iota_f]} (m-1-3)
                (m-1-3) edge node {[\pi_f]} (m-1-4);
        \end{tikzpicture}
    \end{center}
\end{definition}

The reason for restricting \( k \) to an even number is in order to prove the right rotation axiom as well as proving that the trivial triangle is distinguished.

Finally, we can define \( SW \) as a triangulated category.

\begin{example}
    Let \( SW \) be as in \autoref{def:sw-cat}, let \( \Sigma: SW \to SW \) be as in \autoref{def:sw-shift} and let \( \Delta \) be as in \autoref{def:sw-dist_triangles}.

    Then \( \tuple*{SW, \Sigma, \Delta} \) is a triangulated category.
\end{example}
For a proof of this, see \cite[Theorem 5.9]{Daria_Bachelor}.

From this example of a triangulated category we can see where a lot of the notation in the definition of a triangulated category comes from. It is no coincidence that it is common to use the same symbol for both reduced suspension and for the shift functor (\( \Sigma \)), and the same symbol for the mapping cone as well as the cone in a triangulated category (\( C_f \)).

\subsection{The chain homotopy category}
\label{subsubsection:chain_homotopy_cat}
The chain homotopy category is one of the simplest triangulated categories to define, and is a prototypical example of an algebraic triangulated category. The definition of an algebraic triangulated category will be given later in the thesis in \autoref{def:alg_tri_cat}. It will become clear from the definition of the chain homotopy category that it is an algebra inspired triangulated category, and so it fits (morally), as an algebraic triangulated category.

First, we define the category of chain complexes.

\begin{definition}[Chain complex, \( \C(\Cc) \)]
    \label{def:chain_complex}
    Let \( \Cc \) be any additive category.
    
    Then let \( \C \tuple*{\Cc} \) denote the category of chain complexes of objects in \( \Cc \).

    As mentioned in \autoref{not:chain_complex}, the chain complexes have ascending order, and differentials are indexed according to the order of their domain.
\end{definition}

We start off by defining the category, and will later define the triangulation.

% MS-Question: Ikkje vist at dette er ein kategori...
\begin{definition}[Chain homotopy category, \( \K(\Ac) \)]
    \label{def:chain_homotopy_cat}
    Let \( \Ac \) be an abelian category.

    Then let \( \K(\Ac) \) be the following category:
    \begin{enumerate}
        \item {
            Objects in \( \K(\Ac) \) are chain complexes.
        }
        \item {
            Morphisms in \( \K(\Ac) \) are equivalence classes of chain morphisms up to chain homotopy.
        }
        \item {
            Composition in \( \K(\Ac) \) is the equivalence class of the composition of the representatives of the morphisms, i.e.,
            \[
                [g] \circ [f] := [g \circ f ].
            \]
        }
    \end{enumerate}

    Then \( \K(\Ac) \) is called the \emph{chain homotopy category} over \( \Ac \).
\end{definition}

\( \K(\Ac) \) can be shown to be an additive category.

On \( \C(\Ac) \), we can define a simple automorphism of categories called the shift functor.

\begin{definition}[Shift functor on \( \C(\Ac) \)]
    \label{def:chain_complex_shift}
    Let \( \Sigma_{\C(\Ac)} \) be the following assignment of objects and morphisms in \( \C(\Ac) \).

    \begin{itemize}
        \item {
            For \( A = \tuple*{A_i}_{i \in \Zb} \in \C(\Ac) \) a chain complex, and \( d_n^A: A_n \to A_{n + 1} \) the \( n \)th differential of \( A \), let
            \[
                \Sigma_{\C(\Ac)} A := \tuple*{A_{i + 1}}_{i \in \Zb}, \quad \text{with} \quad d_n^{\Sigma A} := -d_{n + 1}^A.
            \]
        }
        \item {
            For \( f = \tuple*{f_i}_{i \in \Zb} \in \C(\Ac)(A, B) \) a chain morphism, let
            \[
                \Sigma_{\C(\Ac)} f := \tuple*{f_{i + 1}}_{i \in \Zb} \in \C(\Ac)(\Sigma A, \Sigma B).
            \]
        }
    \end{itemize}
    
    This is called the \emph{shift functor on \( \C(\Ac) \)}.

    We will denote it as \( \Sigma \) when it is clear from the context.
\end{definition}

\( \Sigma \) can be shown to be an additive automorphism on \( \C(\Ac) \), and can be induced to an additive automorphism with the same notation and name on \( \K(\Ac) \).

Finally, we need to define what the cone of a morphism in \( \K(\Ac) \) is. This turns out to also be inherited from \( \C(\Ac) \).

\begin{definition}[Cone in \( \C(\Ac) \)]
    Let \( f \in \C(\Ac)(A, B) \).

    Then \( C_f \) is the following chain complex
    \begin{center}
        \mmznext{disable}
        \begin{tikzpicture}
            \diagram{m}{1cm}{2cm} {
                \cdots \&[-1cm] A_0 \oplus B_{-1} \& A_1 \oplus B_0 \& A_2 \oplus B_1 \&[-1cm] \cdots \\
            };

            \draw[math]
                (m-1-1) edge (m-1-2)
                (m-1-2) edge node {
                    \begin{pmatrix}
                        -d_0^A & 0 \\
                        f_0 & d_{-1}^B
                    \end{pmatrix}
                } (m-1-3)
                (m-1-3) edge node {
                    \begin{pmatrix}
                        -d_1^A & 0 \\
                        f_1 & d_0^B
                    \end{pmatrix}
                } (m-1-4)
                (m-1-4) edge (m-1-5);
        \end{tikzpicture}
    \end{center}
    called the \emph{cone of \( f \)} in \( \C(\Ac) \).
\end{definition}

It is easy to verify that \( C_f \) is a chain complex, and we can also verify that if we did not have the sign in front of \( d_{i + 1}^A \), or  \( d_i^B \), then it would not be a chain complex.

We can think of the cone as having the underlying objects of \( \Sigma A \oplus B \), but with \( f \) ``gluing'' the two components together such that the chain complex cannot be written as a direct sum.

\begin{remark}
    Let \( \Ac \) be an abelian category.

    Then the cone fits into the following level-wise short exact sequence in \( \C(\Ac) \),
    \begin{center}
        \begin{tikzpicture}
            \diagram{m}{1cm}{1cm} {
                B \& C_f \& \Sigma A. \\
            };

            \draw[math]
                (m-1-1) edge[tailed] node {\iota} (m-1-2)
                (m-1-2) edge[two headed] node {\pi} (m-1-3);
        \end{tikzpicture}
    \end{center}
    where
    \[
        \iota_i := \iota_i^B: B_i \rightarrowtail (\Sigma A)_i \oplus B_i,
    \]
    and
    \[
        \pi_i := \pi_i^{\Sigma A}: (\Sigma A)_i \oplus B_i \twoheadrightarrow (\Sigma A)_i.
    \]

    These are chain morphisms because
    \[
        d_i^{C_f} \circ \iota_i =
        \begin{pmatrix}
            -d_{i + 1}^A & 0 \\
            f_{i + 1} & d_i^B
        \end{pmatrix}
        \begin{pmatrix}
            0 \\
            1
        \end{pmatrix}
        =
        \begin{pmatrix}
            0 \\
            d_i^B
        \end{pmatrix}
        =
        \iota_{i + 1} \circ d_i^B
    \]
    and
    \[
        \pi_i \circ d_i^{C_f} =
        \begin{pmatrix}
            1 & 0
        \end{pmatrix}
        \begin{pmatrix}
            -d_{i + 1}^A & 0 \\
            f_{i + 1} & d_i^B
        \end{pmatrix}
        =
        \begin{pmatrix}
            -d_{i + 1}^A & 0
        \end{pmatrix}
        = d_i^{\Sigma A} \circ \pi_i.
    \]
\end{remark}

The above remark is the reason we have a sign in front of the differential in \( \Sigma A \) compared to \( A \). If we did not have the sign, then \( \pi \) would not be a chain morphism.

The distinguished triangles are the triangles which seem to satisfy just the first triangulated axiom, {\bf (TR1)}, but ends up satisfying every axiom.

\begin{definition}
    \label{def:chain_homotopy_dist}
    Let \( \Delta \) be the collection of triangles in \( \K(\Ac) \) consisting of triangles that are isomorphic to any triangle of the form

    \begin{center}
        \begin{tikzpicture}
            \diagram{m}{1cm}{1cm} {
                A \& B \& C_f \& \Sigma A. \\
            };

            \draw[math]
                (m-1-1) edge node {f} (m-1-2)
                (m-1-2) edge[tailed] node {\iota} (m-1-3)
                (m-1-3) edge[two headed] node {\pi} (m-1-4);
        \end{tikzpicture}
    \end{center}
\end{definition}

Combining, \( \K(\Ac), \Sigma, \) and \( \Delta \) we get our second example of a triangulated category.

\begin{example}
    Let \( \K(\Ac) \) be as in \autoref{def:chain_homotopy_cat}, let \( \Sigma \) be as in \autoref{def:chain_complex_shift}, and let \( \Delta \) be as in \autoref{def:chain_homotopy_dist}.

    Then \( \tuple*{\K(\Ac), \Sigma, \Delta} \) is a triangulated category.
\end{example}
For a proof of \( \K(\Ac) \) being triangulated, see \cite[Proposition 3.5.25]{Zimmermann_2014}.

By \cite[Section 7.5]{Krause_2007}, \( \K(\Ac) \) is an ``algebraic triangulated category,'' by his definition. Note that his definition might not be equivalent to our definition of an algebraic triangulated category in general, which we will define in \autoref{section:alg_tri_cats}.

Similar to the Spanier--Whitehead category, the triangulated category terminology of a \emph{shift functor} is derived from the chain homotopy category. We can see from \autoref{def:chain_complex_shift} that \( \Sigma \) takes the chain complex and shifts it to the left (given that differentials are pointing to the right). The terminology used for triangulated categories can be seen as a mixture of terms from topology and algebra, which fits well since it sits right in the intersection between the two fields.

\subsection{The stable module category}
\label{subsubsection:stable_module_cat}
A triangulated category that will be central in this thesis is the stable module category. Therefore, the definition will be given in more details than the previous two examples of triangulated categories.

Before defining the stable module category, we prove a lemma that will be used to define the morphisms.
\begin{lemma}
    \label{lem:morphisms_factoring_through_projectives_r-submodule}
    Let \( R \) be a commutative ring with identity.

    Let \( G \) be the subset of \( \Mod(R)(A, B) \) consisting of module morphisms that factor through a projective module.

    Then \( G \) is an \( R \)-submodule of \( \Mod(R)(A, B) \).
\end{lemma}
\begin{proof}
    Let \( f \) and \( g \) be two morphisms that factor through the projective objects \( P \) and \( Q \), respectively. Then we have the following commutative diagrams,
    \[
        \begin{aligned}
            \begin{tikzpicture}
                \diagram{m}{1cm}{1cm} {
                    A \& P \& B, \\
                };
    
                \draw[math]
                    (m-1-1) edge node {f_1} (m-1-2)
                        edge[curve={height=25pt}] node[swap] {f} (m-1-3)
                    (m-1-2) edge node {f_2} (m-1-3);
            \end{tikzpicture}
        \end{aligned}
        \hspace{0.5cm}
        \text{ and }
        \hspace{0.5cm}
        \begin{aligned}
            \begin{tikzpicture}
                \diagram{m}{1cm}{1cm} {
                    A \& Q \& B. \\
                };
    
                \draw[math]
                    (m-1-1) edge node {g_1} (m-1-2)
                        edge[curve={height=25pt}] node[swap] {g} (m-1-3)
                    (m-1-2) edge node {g_2} (m-1-3);
            \end{tikzpicture}
        \end{aligned}
    \]

    We can then construct the morphism
    \begin{center}
        \begin{tikzpicture}
            \diagram{m}{1cm}{1cm} {
                A \& {P \oplus Q} \& B. \\
            };

            \draw[math]
                (m-1-1) edge node {
                    \begin{pmatrix}
                        f_1 \\
                        g_1
                    \end{pmatrix}
                } (m-1-2)
                (m-1-2) edge node {
                    \begin{pmatrix}
                        f_2 & g_2
                    \end{pmatrix}
                } (m-1-3);
        \end{tikzpicture}
    \end{center}

    Composing these two morphisms, we get the morphism \( f_2 \circ f_1 + g_2 \circ g_1 = f + g \). This morphism factors through \( P \oplus Q \), which is projective since it is a direct sum of projective modules. This implies that \( G \) is closed under addition.
    
    Let \( r \in R \). Then the following diagram commutes,
    \begin{center}
        \begin{tikzpicture}
            \diagram{m}{1cm}{1cm} {
                A \& P \& B, \\
            };

            \draw[math]
                (m-1-1) edge node {rf_1} (m-1-2)
                    edge[curve={height=25pt}] node[swap] {rf} (m-1-3)
                (m-1-2) edge node {f_2} (m-1-3);
        \end{tikzpicture}
    \end{center}
    which implies that \( G \) is also closed under multiplication in \( R \).
\end{proof}

In order to get a triangulated category, we have to use a specific type of ring, as defined below.
\begin{definition}
    A commutative ring with identity is called a \emph{Frobenius ring}, if every injective module is projective, and vice versa.
\end{definition}

The following is the definition of the stable module category.
\begin{definition}
    \label{def:stable_module_category}
    Let \( R \) be a \emph{Frobenius ring}.

    Then the \emph{stable module category over \( R \)}, denoted \( \Mc \), is defined in the following way:
    \begin{enumerate}
        \item {
            The objects are modules over \( R \).
        }
        \item {
            For \( A, B \in \Mc \), let \( G \) be the \( R \)-submodule from \autoref{lem:morphisms_factoring_through_projectives_r-submodule}.
        
            Then let
            \[
                \Mc(A, B) := \Mod(R)(A, B)/G.
            \]
        }
        \item {
            For \( [g] \in \Mc(B, C) \), and \( [f] \in \Mc(A, B) \), let composition be defined as follows
            \[
                [g] \circ [f] := [g \circ f].
            \]
        }
    \end{enumerate}
    This category is called the \emph{stable module category over \( R \)}.
\end{definition}

Other authors typically use the notation \( \StMod(R) \) or \( \underline{\Mod}(R) \) for the stable module category. However, for the sake of brevity this is reduced to simply \( \Mc \) in this thesis.

\begin{lemma}
    Composition in \autoref{def:stable_module_category} is well-defined.
\end{lemma}
\begin{proof}
    We need to check that two different choices of representatives of \( [f] \) and \( [g] \) yield the same value.

    Let \( f + \widetilde{f} \) and \( g + \widetilde{g} \) be two different representatives of \( [f] \) and \( [g] \), with \( \widetilde{f} \) and \( \widetilde{g} \) factoring through some projective modules.

    Then it follows that
    \[
         [g + \widetilde{g}] \circ [f + \widetilde{f}] = [g \circ f] + [\widetilde{g} \circ f] + [g \circ \widetilde{f}] + [\widetilde{g} \circ \widetilde{f}].
    \]
    
    Every term other than \( [g \circ f] \) factors through a projective and is therefore equal to \( 0 \) in \( \Mc(A, C) \).
\end{proof}

In the general definition of the stable module category, it is not required that the ring is a Frobenius ring. However, as will become apparent later on, it is a requirement to have a triangulation. Since we will only be considering stable module categories which are triangulated, this assumption is made from the beginning.

In order to admit a triangulation, we need to prove that the stable module category is additive.

% TODO: Forenkla beviset, biprodukt argumentet spesielt. Ikkje 100% sikker på at Mac Lane seier det eg ynskjer, berre 80%ish.
\begin{lemma}
    \( \Mc \) is an additive category.
\end{lemma}
\begin{proof}
    There are two parts to this proof. First, we show that \( \Mc \) is pre-additive and second, we show that \( \Mc \) has finite products. By \cite[p.\ 251]{Mac_Lane_1995}, this implies that \( \Mc \) is additive.

    To show that \( \Mc \) is pre-additive there are two properties that need to be shown
    \begin{enumerate}
        \item {
            We show that for any \( A, B \in \Mc \) that \( \Mc(A, B) \) is an abelian group.

            This follows immediately from the definition, since \( \Mc(A, B) \) is a quotient module, and is therefore an \( R \)-module and, hence, an abelian group.
        }
        \item {
            We show that composition is bilinear.

            Let \( [g], [g'] \in \Mc(B, C) \) and \( [f], [f'] \in \Mc(A, B) \). Consider the following equation,
            \begin{align*}
                ([g] + [g']) \circ ([f] + [f']) &= [g + g'] \circ [f + f'] \\
                &= \class*{(g + g') \circ (f + f')} \\
                &= \class*{g \circ f + g \circ f' + g' \circ f + g' \circ f'} \\
                &= [g] \circ [f] + [g] \circ [f'] + [g'] \circ [f] + [g'] \circ [f'],
            \end{align*}
            which is precisely bilinearity.
        }
    \end{enumerate}
    Thus, \( \Mc \) is pre-additive.

    Now we want to prove that the usual biproduct in \( \Mod(R) \) induces the product in \( \Mc \).
    
    Consider the commutative diagram for the universal property of \( A \oplus B \) as a product in \( \Mod(R) \). Taking residue classes of the morphisms yields the following commutative diagram in \( \Mc \),
    \begin{center}
        \begin{tikzpicture}
            \diagram{m}{1.5cm}{1.5cm} {
                \& X \\
                A \& A \oplus B \& B. \\
            };

            \draw[math]
                (m-1-2) edge node[swap] {[f_A]} (m-2-1)
                    edge[dashed] node {[f]} (m-2-2)
                    edge node {[f_B]} (m-2-3)

                (m-2-2) edge node[swap] {[\pi_B]} (m-2-3)
                    edge node {[\pi_A]} (m-2-1);
        \end{tikzpicture}
    \end{center}
    Let \( [g]: X \to A \oplus B \) be another morphism that satisfies the universal property.

    Then the following diagram
    \begin{center}
        \begin{tikzpicture}
            \diagram{m}{1.5cm}{1.5cm} {
                \& X \\
                A \& A \oplus B \& B \\
            };

            \draw[math]
                (m-1-2) edge node[swap] {[0]} (m-2-1)
                    edge[dashed] node {[f - g]} (m-2-2)
                    edge node {[0]} (m-2-3)

                (m-2-2) edge node[swap] {[\pi_B]} (m-2-3)
                    edge node {[\pi_A]} (m-2-1);
        \end{tikzpicture}
    \end{center}
    commutes.

    Let \( \iota_A \), and \( \iota_B \) denote the canonical split monomorphisms from the universal property of the coproduct. Since \( A \oplus B \) is a biproduct in \( \Mod(R) \), we have
    \[
        \Id_{A \oplus B} = \iota_A \circ \pi_A + \iota_B \circ \pi_B.
    \]
    Thus,
    \begin{align*}
        [f - g] &= [\Id_{A \oplus B} \circ (f - g)] \\
        &= [\iota_A \circ \pi_A \circ (f - g) + \iota_B \circ \pi_B \circ (f - g)] \\
        &= [\iota_A] \circ [\pi_A \circ (f - g)] + [\iota_B] \circ [\pi_B \circ (f - g)] \\
        &= [0],
    \end{align*}
    which implies that \( [f] = [g] \), and that \( A \oplus B \) is a product in \( \Mc \), and by the statement at the start, also a biproduct.
\end{proof}

The next step in creating a triangulation for \( \Mc \) is to define the shift functor. In the stable module category, the shift functor is the inverse to the ``syzygy functor.''

The syzygy functor as well as the shift functor exists in the category of modules over a commutative ring with identity (e.g., a Frobenius ring) because it has \emph{enough projectives} and \emph{enough injectives}. Having enough projectives means that for any module \( A \), there exists some projective module \( P_A \), as well as an epimorphism \( \pi_A \) from \( P_A \) to \( A \). The choice of \( P_A \) and consequently \( \pi_A \) is not necessarily unique up to isomorphism, and two different choices of \( P_A \) could be non-isomorphic. Similarly, having enough injectives means that for any module \( A \) there exists some injective module \( I_A \) along with a monomorphism \( \iota_A \) from \( A \) to \( I_A \). Equal to the projective case, \( I_A \) is not necessarily unique up to isomorphism and \( \iota_A \) is not necessarily unique.

The definition of the syzygy functor is closely tied to a choice of \( P_A \) and \( \pi_A \) for every object \( A \).

\begin{definition}[The syzygy functor \( \Omega \)]
    \label{def:stmod_omega}
    Let \( R \) be a Frobenius ring.

    Let \( P = \set*{\tuple*{P_A, \pi_A, \Omega A}}_{A \in \Mod(R)} \) be a collection of tuples for every object \( A \), where \( P_A \) is a projective module, \( \pi_A: P_A \twoheadrightarrow A \) an epimorphism, and \( \Omega A \) is a kernel of \( \pi_A \). This exists because \( \Mod(R) \) has enough projectives.

    Then define \( \Omega \) as the assignment of objects and morphisms in \( \Mc \) as follows:
    \begin{itemize}
        \item {
            For any object \( A \) in \( \Mc \), let \( \Omega A \) be as above.
        }
        \item {
            For any \( [f] \in \Mc(A, B) \), let \( \Omega [f] \) be constructed as follows:

            Consider the following diagram in \( \Mod(R) \) excluding the dashed arrows,
            \begin{center}
                \begin{tikzpicture}
                    \diagram{m}{1cm}{1cm} {
                        \Omega A \& P_A \& A \\
                        \Omega B \& P_B \& B. \\
                    };

                    \draw[math]
                        (m-1-1) edge[tailed] node {\iota_A} (m-1-2)
                            edge[dashed] node[swap] {\Omega f} (m-2-1)
                        (m-1-2) edge[two headed] node {\pi_A} (m-1-3)
                            edge[dashed] node {p_f} (m-2-2)
                        (m-1-3) edge node {f} (m-2-3)

                        (m-2-1) edge[tailed] node {\iota_B} (m-2-2)
                        (m-2-2) edge[two headed] node {\pi_B} (m-2-3);
                \end{tikzpicture}
            \end{center}
            Let \( \Omega f: \Omega A \to \Omega B \) be any morphism that makes the above diagram commute for some \( p_f: P_A \to P_B \).

            Then define \( \Omega [f] := \class*{\Omega f} \).
        }
    \end{itemize}
    This is called the \emph{syzygy functor}.
\end{definition}

The goal is to show that \( \Omega: \Mc \to \Mc \) is a well-defined endofunctor, and later showing that it is in fact an additive auto-equivalence of categories.

\begin{lemma}
    \label{lem:stmod_omega_f_is_well_defined}
    \( \Omega \) is a well-defined assignment of morphisms.
\end{lemma}
\begin{proof}
    There are three things that need to be proven. First, the existence of an \( \Omega f \) for any \( f \). Second, we need to show that for any \( p_f \), then \( \Omega \) still yields the same morphism in \( \Mc \). Third, we need to show that if \( [f] = [g] \), then \( \Omega [f] = \Omega [g] \).

    First, we have that for a morphism \( f: A \to B \), there exists a morphism \( p_f \) from the lifting property of projective modules such that the right square in the definition commutes.

    Furthermore, since
    \[
        \pi_B \circ p_f \circ \iota_A = f \circ \pi_A \circ \iota_A = f \circ 0 = 0,
    \]
    we have from the universal kernel property that there exists a morphism, \( \Omega f \), dependent on the choice of \( p_f \), such that the left square in the definition commutes. This is therefore a valid choice of \( \Omega f \).

    Second, let \( p_f \) and \( \widetilde{p_f} \) be two different projective morphisms that yield morphisms \( \Omega f \) and \( \widetilde{\Omega f} \), respectively. Then we have the following commutative diagram excluding the dashed arrow,
    \begin{center}
        \begin{tikzpicture}
            \diagram{m}{1cm}{2cm} {
                \Omega A \& P_A \& A \\
                \Omega B \& P_B \& B. \\
            };

            \draw[math]
                (m-1-1) edge[tailed] node {\iota_A} (m-1-2)
                    edge[swap] node {\Omega f - \widetilde{\Omega f}} (m-2-1)
                (m-1-2) edge[two headed] node {\pi_A} (m-1-3)
                    edge[dashed, swap] node {\phi} (m-2-1)
                    edge node {p_f - \widetilde{p_f}} (m-2-2)
                (m-1-3) edge node {f -f = 0} (m-2-3)

                (m-2-1) edge[tailed] node {\iota_B} (m-2-2)
                (m-2-2) edge[two headed] node {\pi_B} (m-2-3);
        \end{tikzpicture}
    \end{center}

    Since
    \[
        \pi_B \circ \tuple*{p_f - \widetilde{p_f}} = \tuple*{f - f} \circ \pi_A = 0
    \]
    there exists a morphism \( \phi \) induced by the kernel property of \( \Omega B \), such that the lower triangle in the diagram commutes. Then because of the monomorphism property of \( \iota_B \), we also get that the upper triangle commutes. This implies that the morphism \( \Omega f - \widetilde{\Omega f} \) factors through \( P_A \), a projective module. Therefore,
    \[
        [0] = \class*{\Omega f - \widetilde{\Omega f}} = \class*{\Omega f} - \class*{\widetilde{\Omega f}}
    \]
    which implies that \( \Omega [f] \) is independent of the choice of \( p_f \).

    Third, we need to show that if \( [f] = [g] \), then \( \Omega [f] = \Omega [g] \). Consider the following commutative diagram excluding the dashed arrow,
    \begin{center}
        \begin{tikzpicture}
            \diagram{m}{1cm}{3cm} {
                \Omega A \& P_A \& A \\
                \&\& P \\
                \Omega B \& P_B \& B. \\
            };

            \draw[math]
                (m-1-1) edge[tailed] node {\iota_A} (m-1-2)
                    edge node {\Omega f - \Omega g} (m-3-1)
                (m-1-2) edge[two headed] node {\pi_A} (m-1-3)
                    edge node {p_f - p_g} (m-3-2)
                (m-1-3) edge[swap] node {(f - g)_1} (m-2-3)
                    edge[curve={height=-25pt}] node {f - g} (m-3-3)

                (m-2-3) edge[swap, dashed] node {\theta} (m-3-2)
                    edge[swap] node {(f - g)_2} (m-3-3)

                (m-3-1) edge[tailed] node {\iota_B} (m-3-2)
                (m-3-2) edge[two headed] node[swap] {\pi_B} (m-3-3);
        \end{tikzpicture}
    \end{center}

    Let \( P \) be the projective module that \( f - g \) factors through. Then from the projective property, there exists a morphism \( \theta: P \to P_B \), which causes the lower triangle to commute.

    Let \( p_{f - g} := \theta \circ (f - g)_1 \circ \pi_A \). By construction, we have that both \( p_f - p_g \) and \( p'_{f - g} \) are morphisms that would make the right hand square commute.
    
    But since
    \[
        p_{f - g} \circ \iota_A = \theta \circ (f-g)_1 \circ \pi_A \circ \iota_A = \theta \circ (f-g)_1 \circ 0 = 0,
    \]
    the following diagram,
    \begin{center}
        \begin{tikzpicture}
            \diagram{m}{1cm}{1cm} {
                \Omega A \& P_A \& A \\
                \Omega B \& P_B \& B, \\
            };

            \draw[math]
                (m-1-1) edge[tailed] node {\iota_A} (m-1-2)
                    edge node {0} (m-2-1)
                (m-1-2) edge[two headed] node {\pi_A} (m-1-3)
                    edge node {p_{f - g}} (m-2-2)
                (m-1-3) edge node {f - g} (m-2-3)

                (m-2-1) edge[tailed] node {\iota_B} (m-2-2)
                (m-2-2) edge[two headed] node {\pi_B} (m-2-3);
        \end{tikzpicture}
    \end{center}
    commutes.

    However, by the second part of this proof, this implies that \( [0] = [\Omega f - \Omega g] \), and therefore \( \Omega [f] = \Omega [g] \).
\end{proof}

Now that \( \Omega \) is a well-defined assignment of objects and morphisms, it only remains to prove functoriality.

\begin{lemma}
    \label{lem:stmod_omega_endofunctor}
    \( \Omega \) is an endofunctor on \( \Mc \).
\end{lemma}
\begin{proof}
    To show that \( \Omega \) is a functor we must prove functoriality.

    First, we show that \( \Omega \) is preserves composition. Let \( A, B, C \in \Mc \). Then by the definition of \( \Omega \), we have the following commutative diagram,
    \begin{center}
        \begin{tikzpicture}
            \diagram{m}{1cm}{2cm} {
                \Omega A \& P_A \& A \\
                \Omega B \& P_B \& B \\
                \Omega C \& P_C \& C. \\
            };

            \draw[math]
                (m-1-1) edge[tailed] node {\iota_A} (m-1-2)
                    % edge[curve={height=30pt}, swap] node {\Omega (g \circ f)} (m-3-1)
                    edge node {\Omega f} (m-2-1)
                (m-1-2) edge[two headed] node {\pi_A} (m-1-3)
                    edge node {p_f} (m-2-2)
                (m-1-3) edge node {f} (m-2-3)

                (m-2-1) edge[tailed] node {\iota_B} (m-2-2)
                    edge node {\Omega g} (m-3-1)
                (m-2-2) edge[two headed] node {\pi_B} (m-2-3)
                    edge node {p_g} (m-3-2)
                (m-2-3) edge node {g} (m-3-3)

                (m-3-1) edge[tailed] node {\iota_A} (m-3-2)
                (m-3-2) edge[two headed] node {\pi_B} (m-3-3);
        \end{tikzpicture}
    \end{center}

    Considering the composition of the vertical morphisms, we end up with the following commutative diagram
    \begin{center}
        \begin{tikzpicture}
            \diagram{m}{1cm}{2cm} {
                \Omega A \& P_A \& A \\
                \Omega C \& P_C \& C. \\
            };

            \draw[math]
                (m-1-1) edge[tailed] node {\iota_A} (m-1-2)
                    edge[swap] node {\Omega g \circ \Omega f} (m-2-1)
                (m-1-2) edge[two headed] node {\pi_A} (m-1-3)
                    edge node {p_g \circ p_f} (m-2-2)
                (m-1-3) edge node {g \circ f} (m-2-3)

                (m-2-1) edge[tailed] node {\iota_C} (m-2-2)
                (m-2-2) edge[two headed] node {\pi_C} (m-2-3);
        \end{tikzpicture}
    \end{center}
    Since \( \Omega \) is a well-defined assignment of morphisms, this implies
    \[
        (\Omega [g]) \circ (\Omega [f]) = \class*{(\Omega g) \circ (\Omega f)} = \class*{\Omega (g \circ f)} = \Omega \class*{g \circ f}.
    \]

    Second, we need to show that \( \Omega [\Id_A] = [\Id_{\Omega A}] \) in \( \Mc \).

    We can see that every square and triangle in the following diagram,
    \begin{center}
        \begin{tikzpicture}
            \diagram{m}{1cm}{2cm} {
                \Omega A \& P_A \& A \\
                \Omega A \& P_A \& A, \\
            };

            \draw[math]
                (m-1-1) edge[tailed] node {\iota_A} (m-1-2)
                    edge[swap] node {\Id_{\Omega A}} (m-2-1)
                (m-1-2) edge[two headed] node {\pi_A} (m-1-3)
                    edge node {\Id_{P_A}} (m-2-2)
                (m-1-3) edge node {\Id_A} (m-2-3)

                (m-2-1) edge[tailed] node {\iota_A} (m-2-2)
                (m-2-2) edge[two headed] node {\pi_A} (m-2-3);
        \end{tikzpicture}
    \end{center}
    commutes.

    Therefore, since \( \Omega \) is a well-defined assignment of morphisms, \( \Omega \class*{\Id_A} = \class*{\Id_{\Omega A}} \).
\end{proof}

Finally, we need the following result to show that \( \Omega \) is additive.

\begin{lemma}
    \label{lem:stmod_omega_additive_functor}
    \( \Omega \) is an additive functor.
\end{lemma}
% TODO: Ikkje fullstendig bevis. Må visa at \Omega (A \oplus B) er nat iso med \Omega A \oplus \Omega B, og at embedding og projeskjon morfiane er "bevart". Sidan \Omega har invers-funktor, så er den høgre/venstre adjungert, og må difor bevara produkt/koprodukt.
\begin{proof}
    We must show that \( \Omega [f + g] = \Omega [f] + \Omega [g] \).
    
    Since the diagrams
    \[
        \begin{aligned}
            \begin{tikzpicture}
                \diagram{m}{1cm}{2cm} {
                    {\Omega A} \& {P_A} \& A \\
                    {\Omega B} \& {P_B} \& B \\
                };

                \draw[math]
                    (m-1-1) edge[tailed] node {\iota_A} (m-1-2)
                        edge node {\Omega (f + g)} (m-2-1)
                    (m-1-2) edge[two headed] node {\pi_A} (m-1-3)
                        edge node {p_{f + g}} (m-2-2)
                    (m-1-3) edge node[swap] {f + g} (m-2-3)

                    (m-2-1) edge[tailed] node {\iota_B} (m-2-2)
                    (m-2-2) edge[two headed] node {\pi_B} (m-2-3);
            \end{tikzpicture}
        \end{aligned}
        \quad
        \text{and}
        \quad
        \begin{aligned}
            \begin{tikzpicture}
                \diagram{m}{1cm}{2cm} {
                    {\Omega A} \& {P_A} \& A \\
                    {\Omega B} \& {P_B} \& B \\
                };

                \draw[math]
                    (m-1-1) edge[tailed] node {\iota_A} (m-1-2)
                        edge node {\Omega f + \Omega g} (m-2-1)
                    (m-1-2) edge[two headed] node {\pi_A} (m-1-3)
                        edge node {p_f + p_g} (m-2-2)
                    (m-1-3) edge node[swap] {f + g} (m-2-3)

                    (m-2-1) edge[tailed] node {\iota_B} (m-2-2)
                    (m-2-2) edge[two headed] node {\pi_B} (m-2-3);
            \end{tikzpicture}
        \end{aligned}
    \]
    both commute, and since \( \Omega \) is a well-defined assignment of morphisms, this implies \( \Omega [f + g] = \Omega [f] + \Omega [g] \).
\end{proof}

We can define another functor, the cosyzygy functor \( \Sigma \), which will turn out to be the inverse of the syzygy functor \( \Omega \) and the shift functor in the triangulation of \( \Mc \).

\begin{definition}[The cosyzygy functor \( \Sigma \)]
    \label{def:stmod_sigma}
    Let \( R \) be a Frobenius ring.

    Let \( I = \set*{\tuple*{I_A, \kappa_A, \Sigma A}}_{A \in \Mod(R)} \) be a collection of tuples for every object \( A \), where \( I_A \) is an injective module, \( \kappa_A: A \rightarrowtail I_A \) a monomorphism, and \( \Sigma A \) is a cokernel of \( \kappa_A \). This exists because \( \Mod(R) \) has enough injectives.

    Then define \( \Sigma \) as the assignment of objects and morphisms in \( \Mc \) as follows:
    \begin{itemize}
        \item {
            For any object \( A \) in \( \Mc \), let \( \Sigma A \) be as above.
        }
        \item {
            For \( [f] \in \Mc(A, B) \), let \( \Sigma [f] \) be constructed as follows:

            Consider the following diagram in \( \Mod(R) \) excluding the dashed arrows,
            \begin{center}
                \begin{tikzpicture}
                    \diagram{m}{1cm}{1cm} {
                        A \& I_A \& \Sigma A \\
                        B \& I_B \& \Sigma B. \\
                    };

                    \draw[math]
                        (m-1-1) edge[tailed] node {\kappa_A} (m-1-2)
                            edge node {f} (m-2-1)
                        (m-1-2) edge[two headed] node {\rho_A} (m-1-3)
                            edge[dashed] node {i_f} (m-2-2)
                        (m-1-3) edge[dashed] node {\Sigma f} (m-2-3)

                        (m-2-1) edge[tailed] node {\kappa_B} (m-2-2)
                        (m-2-2) edge[two headed] node {\rho_B} (m-2-3);
                \end{tikzpicture}
            \end{center}

            Let \( \Sigma f: \Sigma A \to \Sigma B \) be any morphism that makes the above diagram commute for some \( i_f: I_A \to I_B \).

            Then define \( \Sigma [f] := \class*{\Sigma f} \).
        }
    \end{itemize}
    This is called the \emph{cosyzygy functor.}
\end{definition}

It remains to show that every desired property of \( \Sigma \), like being well-defined, an endofunctor, and additive, translates to \( \Sigma \).
\begin{lemma}
    \label{lem:stmod_sigma_well-defined_additive_endofunctor}
    \( \Sigma \) is a well-defined and additive endofunctor on \( \Mc \).
\end{lemma}
\begin{proof}
    The proofs of the various statements are entirely dual to the proofs of \autoref{lem:stmod_omega_f_is_well_defined}, \autoref{lem:stmod_omega_endofunctor} and \autoref{lem:stmod_omega_additive_functor}.
\end{proof}

Now we can finally show that \( \Sigma = \Omega^{-1} \).

\begin{theorem}
    \label{thm:stmod_omega_autoeq}
    \( \Omega \) is an auto-equivalence with inverse \( \Sigma \).
\end{theorem}
\begin{proof}
    We will only show that \( \Id_{\Mc} \) is naturally isomorphic to \( \Sigma \Omega \). The omitted part that \( \Id_{\Mc} \) is naturally isomorphic to \( \Omega \Sigma \) is very similar, and uses many dual properties.

    First, we show that for any \( A \in \Mc \), there exists an isomorphism \( A \to \Sigma \Omega A \). Consider the following diagram excluding the dashed arrows, where the rows come from the definitions of \( \Omega A \) and \( \Sigma \Omega A \),
    \begin{center}
        \begin{tikzpicture}
            \diagram{m}{1cm}{2cm} {
                \Omega A \& P_A \& A \\
                \Omega A \& I_{\Omega A} \& \Sigma \Omega A \\
                \Omega A \& P_A \& A. \\
            };

            \draw[math]
                (m-1-1) edge[tailed] node {\iota_A} (m-1-2)
                    edge[equality] (m-2-1)
                (m-1-2) edge[two headed] node {\pi_A} (m-1-3)
                    edge[dashed] node {i_{\phi_1}} (m-2-2)
                (m-1-3) edge[dashed] node {\phi_1} (m-2-3)

                (m-2-1) edge[tailed] node {\kappa_{\Omega A}} (m-2-2)
                    edge[equality] (m-3-1)
                (m-2-2) edge[two headed] node {\rho_{\Omega A}} (m-2-3)
                    edge[dashed] node {i_{\phi_2}} (m-3-2)
                (m-2-3) edge[dashed] node {\phi_2} (m-3-3)

                (m-3-1) edge[tailed] node {\iota_A} (m-3-2)
                (m-3-2) edge[two headed] node {\pi_A} (m-3-3);
        \end{tikzpicture}
    \end{center}
    There exists some \( i_{\phi_1}: P_A \to I_{\Omega A} \) from the injective property of \( I_{\Omega A} \), induced by \( \iota_A \), which would make the top left square commute. In addition, since \( \Mod(R) \) is an abelian category, \( A \) is a cokernel of \( \iota_A \), and by
    \[
        \rho_{\Omega A} \circ i_{\phi_1} \circ \iota_A = \rho_{\Omega A} \circ \kappa_{\Omega A} = 0,
    \]
    we get from the cokernel property that there is a uniquely induced morphism \( \phi_1: A \to \Sigma\Omega A \), which makes the top right square commute. Then by doing the same for the lower rectangle of the diagram, using the fact that every projective module is also injective, we get the morphisms \( i_{\phi_2} \) and \( \phi_2 \) by similar arguments. This makes the entire diagram, including the dashed arrows, commute.

    To show that \( \phi_1 \) and \( \phi_2 \) are isomorphisms, consider the following commutative diagram excluding the dashed arrow,
    \begin{center}
        \begin{tikzpicture}
            \diagram{m}{1cm}{2cm} {
                \Omega A \& P_A \& A \\
                \Omega A \& P_A \& A. \\
            };

            \draw[math]
                (m-1-1) edge[tailed] node {\iota_A} (m-1-2)
                    edge[swap] node {\Id_{\Omega A} \circ \Id_{\Omega A} - \Id_{\Omega A} = 0} (m-2-1)
                (m-1-2) edge[two headed] node {\pi_A} (m-1-3)
                    edge[swap] node {i_{\phi_2} \circ i_{\phi_1} - \Id_{P_A}} (m-2-2)
                (m-1-3) edge[swap, dashed] node {\theta} (m-2-2)
                    edge node {\phi_2 \circ \phi_1 - \Id_A} (m-2-3)

                (m-2-1) edge[tailed] node {\iota_A} (m-2-2)
                (m-2-2) edge[two headed] node {\pi_A} (m-2-3);
        \end{tikzpicture}
    \end{center}

    Since
    \[
        (i_{\phi_2} \circ i_{\phi_1} - \Id_{P_A}) \circ \iota_A = \iota_A \circ 0 = 0,
    \]
    we have from the cokernel property that there exists a morphism \( \theta: A \to P_A \) such that the upper triangle commutes. Thus,
    \[
        (\phi_2 \circ \phi_1 - \Id_A) \circ \pi_A = \pi_A \circ (i_{\theta_2} \circ i_{\theta_1} - \Id_{P_A}) = \pi_A \circ \theta \circ \pi_A,
    \]
    and since \( \pi_A \) is an epimorphism, we get that the lower triangle commutes. This implies that \( [\phi_2 \circ \phi_1 - \Id_A] = [0] \), which implies \( [\phi_2 \circ \phi_1] = [\Id_A] \).
    
    By a similar argument that is omitted for brevity, we can show that \( [\phi_1 \circ \phi_2] = [\Id_{\Sigma\Omega A}] \), which means that \( [\phi_1] \) and \( [\phi_2] \) are isomorphisms from \( A \) to \( \Sigma\Omega A \).

    Finally, to show that these isomorphisms are natural, let \( [f] \in \Mc(A, B) \) and consider the following two commutative diagrams
    \begin{center}
        \begin{tikzpicture}
            \diagram{m}{1cm}{2cm} {
                \Omega A \& P_A \& A \\
                \Omega A \& I_{\Omega A} \& \Sigma \Omega A \\
                \Omega B \& I_{\Omega B} \& \Sigma \Omega B, \\
            };

            \draw[math]
                (m-1-1) edge[tailed] node {\iota_A} (m-1-2)
                    edge[equality] (m-2-1)
                (m-1-2) edge[two headed] node {\pi_A} (m-1-3)
                    edge node {i_{\phi^A_1}} (m-2-2)
                (m-1-3) edge node {\phi^A_1} (m-2-3)

                (m-2-1) edge[tailed] node {\kappa_{\Omega A}} (m-2-2)
                    edge node {\Omega f} (m-3-1)
                (m-2-2) edge[two headed] node {\rho_{\Omega A}} (m-2-3)
                    edge node {i_{\Omega f}} (m-3-2)
                (m-2-3) edge node {\Sigma \Omega f} (m-3-3)

                (m-3-1) edge[tailed] node {\kappa_{\Omega B}} (m-3-2)
                (m-3-2) edge[two headed] node {\rho_{\Omega B}} (m-3-3);
        \end{tikzpicture}
    \end{center}
    and
    \begin{center}
        \begin{tikzpicture}
            \diagram{m}{1cm}{2cm} {
                \Omega A \& P_A \& A \\
                \Omega B \& P_B \& B \\
                \Omega B \& I_{\Omega B} \& \Sigma \Omega B. \\
            };

            \draw[math]
                (m-1-1) edge[tailed] node {\iota_A} (m-1-2)
                    edge node {\Omega f} (m-2-1)
                (m-1-2) edge[two headed] node {\pi_A} (m-1-3)
                    edge node {p_f} (m-2-2)
                (m-1-3) edge node {f} (m-2-3)

                (m-2-1) edge[tailed] node {\iota_B} (m-2-2)
                    edge[equality] (m-3-1)
                (m-2-2) edge[two headed] node {\pi_B} (m-2-3)
                    edge node {i_{\phi_1^B}} (m-3-2)
                (m-2-3) edge node {\phi_1^B} (m-3-3)

                (m-3-1) edge[tailed] node {\kappa_{\Omega B}} (m-3-2)
                (m-3-2) edge[two headed] node {\rho_{\Omega B}} (m-3-3);
        \end{tikzpicture}
    \end{center}

    These diagrams give rise to the following commutative diagram, excluding the dashed arrow,
    \begin{center}
        \begin{tikzpicture}
            \diagram{m}{1cm}{2.7cm} {
                \Omega A \& P_A \& A \\
                \Omega B \& I_{\Omega B} \& \Sigma \Omega B, \\
            };

            \draw[math]
                (m-1-1) edge[tailed] node {\iota_A} (m-1-2)
                    edge[swap] node {\Id_{\Omega B} \circ (\Omega f) - (\Omega f) \circ \Id_{\Omega A} = 0} (m-2-1)
                (m-1-2) edge[two headed] node {\pi_A} (m-1-3)
                    edge[swap] node {i_{\phi_1^B} \circ p_f - i_{\Omega f} \circ i_{\phi_1^A}} (m-2-2)
                (m-1-3) edge[swap, dashed] node {\theta} (m-2-2)
                    edge node {\phi_1^B \circ f - (\Sigma \Omega f) \circ \phi_1^A} (m-2-3)

                (m-2-1) edge[tailed] node[swap] {\kappa_{\Omega B}} (m-2-2)
                (m-2-2) edge[two headed] node[swap] {\rho_{\Omega B}} (m-2-3);
        \end{tikzpicture}
    \end{center}
    where from the cokernel property of \( A \), we get an induced morphism \( \theta \). Furthermore, from the epimorphism property of \( \pi_A \), the lower triangle commutes. Since \( R \) is assumed to be a Frobenius ring, \( I_{\Omega B} \) is also projective. This implies
    \[
        [\phi_1^B \circ f] = [(\Sigma \Omega f) \circ \phi_1^A],
    \]
    which means that \( [\phi_1] \) is a natural isomorphism from \( \Id_{\Mc} \) to \( \Sigma \Omega \), with inverse \( [\phi_2] \).

    As mentioned at the start, the proof that \( \Id_{\Mc} \) is naturally isomorphic to \( \Omega \Sigma \) is very similar to the above proof, but using dual properties.
\end{proof}

An interesting consequence of how \( \Omega \) and \( \Sigma \) are defined is that if we were to choose a different \( P \) or \( I \) in their definitions, then it would yield different, but naturally isomorphic functors. A proof of this statement for \( \Sigma \) can be found in \cite[p.\ 13]{Happel_1988}, with the proof for \( \Omega \) being dual.

Before we can show the triangulation of \( \Mc \), we first need to define the cone of a morphism. The definition of a cone, as well as the definition of the distinguished triangles leans heavily upon \cite[Section 1.2.5]{Happel_1988}.

\begin{definition}
    \label{def:stmod_cone}
    Let \( f \in \Mod(R)(A, B) \).
    
    Then define a \emph{cone of \( f \)} to be a pushout object of the following diagram
    \begin{center}
        \begin{tikzpicture}
            \diagram{m}{1cm}{1cm} {
                A \& B \\
                I_A, \\
            };

            \draw[math]
                (m-1-1) edge node {f} (m-1-2)
                    edge[tailed] node[swap] {\kappa_A} (m-2-1);
        \end{tikzpicture}
    \end{center}
    and denote it by \( C_f \).

    This pushout also defines two morphisms \( g \) and \( \gamma_f \) which fit into the following pushout square,
    \begin{center}
        \begin{tikzpicture}
            \diagram{m}{1cm}{1cm} {
                A \& B \\
                I_A \& C_f. \\
            };

            \draw[math]
                (m-1-1) edge node {f} (m-1-2)
                    edge[tailed] node[swap] {\kappa_A} (m-2-1)
                (m-1-2) edge node {g} (m-2-2)

                (m-2-1) edge node {\gamma_f} (m-2-2);
        \end{tikzpicture}
    \end{center}
\end{definition}

Following the definition of a cone, we can define the standard triangles of \( \Mc \) as induced from some triangles in \( \Mod(R) \). The following remark walks through the construction, which will be very useful in proofs.

\begin{remark}
    \label{rem:stmod_cone}
    Let \( f \in \Mod(R)(A, B) \), and let \( C_f, g \) and \( \gamma_f \) be as in \autoref{def:stmod_cone}.
    
    Consider the following commutative diagram excluding the dashed arrows,
    \begin{center}
        \begin{tikzpicture}
            \diagram{m}{1cm}{1cm} {
                A \& B \\
                I_A \& C_f \\
                \& \Sigma A. \\
            };

            \draw[math]
                (m-1-1) edge node {f} (m-1-2)
                    edge[tailed] node[swap] {\kappa_A} (m-2-1)
                (m-1-2) edge node {g} (m-2-2)
                    edge[curve={height=-25pt}] node {0} (m-3-2)

                (m-2-1) edge node {\gamma_f} (m-2-2)
                    edge[two headed] node {\rho_A} (m-3-2)
                (m-2-2) edge[dashed] node {h} (m-3-2);
        \end{tikzpicture}
    \end{center}
    Then there exists some morphism \( h: C_f \to \Sigma A \) from the pushout property of \( C_f \).

    This diagram will form the backbone of the distinguished triangles in \( \Mc \).
\end{remark}

Now, we can finally define the triangulation on \( \Mc \).

\begin{definition}
    \label{def:stmod_delta}
    Let \( \Delta \) be the collection of triangles in \( \Mc \) isomorphic to any triangle of the form
    \begin{center}
        \begin{tikzpicture}
            \diagram{m}{1cm}{1cm} {
                A \& B \& C_f \& \Sigma A \\
            };

            \draw[math]
                (m-1-1) edge node {[f]} (m-1-2)
                (m-1-2) edge node {[g]} (m-1-3)
                (m-1-3) edge node {[h]} (m-1-4);
        \end{tikzpicture}
    \end{center}
    for any \( f \in \Mod(R)(A, B) \), and where \( C_f \), \( g \), and \( h \) are as defined in \autoref{rem:stmod_cone}.
\end{definition}

There are some important details from the definition of standard triangles that will be important in proving that \( \Mc \) has a triangulation.

\begin{remark}
    \label{rem:stmod_cone_pushout_properties}
    Consider the objects and morphisms in the above remark.

    Since \( \Mod(R) \) is an abelian category, it follows that since \( \kappa_A \) is a monomorphism, then \( g \) is also a monomorphism as \( \ker(g) = \ker(\kappa_A) = 0 \). Likewise, it follows that \( \coker(g) \cong \coker(\kappa_A) \cong \Sigma A \), and therefore \( h \) is a cokernel morphism of \( g \).
\end{remark}

Since morphisms in \( \Mc \) are residue classes and can therefore have multiple representatives, there are multiple cones for each morphism. Each cone yields a different standard triangle, and there are therefore multiple standard triangles for each morphism. This turns out to not be a problem since it does not conflict with the definition of a triangulated category. In particular, the use of the word ``standard triangle'' and ``cone'' above, still aligns with the definitions given in the definition of a triangulated category.

The following lemma is needed to prove {\bf (TR4)}, however, it also hints to the fact that we could have chosen \( \Delta \) differently, as is done in \cite[Definition 4.16]{Johan_Bachelor}.
\begin{lemma}
    \label{lem:stmod_pushout_different_injectives_isomorphic}
    Let
    \[
        \begin{aligned}
            \begin{tikzpicture}
                \diagram{m}{1cm}{1cm} {
                    A \& B \\
                    I \& C \\
                };
    
                \draw[math]
                    (m-1-1) edge node {f} (m-1-2)
                        edge[tailed] node[swap] {\kappa} (m-2-1)
                    (m-1-2) edge[tailed] node {g} (m-2-2)
    
                    (m-2-1) edge node {\gamma} (m-2-2);
            \end{tikzpicture}
        \end{aligned}
        \hspace{0.5cm}
        \text{ and }
        \hspace{0.5cm}
        \begin{aligned}
            \begin{tikzpicture}
                \diagram{m}{1cm}{1cm} {
                    A \& B \\
                    I' \& C' \\
                };
    
                \draw[math]
                    (m-1-1) edge node {f} (m-1-2)
                        edge[tailed] node[swap] {\kappa'} (m-2-1)
                    (m-1-2) edge[tailed] node {g'} (m-2-2)
    
                    (m-2-1) edge node {\gamma'} (m-2-2);
            \end{tikzpicture}
        \end{aligned}
    \]
    be two pushout squares in \( \Mod(R) \), with \( \kappa \) and \( \kappa' \) monomorphisms into injective objects \( I \) and \( I' \).

    Then there exists an isomorphism \( [\alpha]: C \to C' \) in \( \Mc \).
    
    In addition, given a morphism \( i: I \to I' \), such that \( i \circ \kappa = \kappa' \), then \( \alpha \) has the following properties:
    \begin{itemize}
        \item \( \alpha \circ g = g' \), and
        \item \( \alpha \circ \gamma = \gamma' \circ i \).
    \end{itemize}
\end{lemma}
\begin{proof}
    Since both \( I \) and \( I' \) are injective objects and \( \kappa \) and \( \kappa' \) are monomorphisms, there exists morphisms \( i \) and \( i' \) such that the following two diagrams,
    \[
        \begin{aligned}
            \begin{tikzpicture}
                \diagram{m}{1cm}{0.5cm} {
                    \& A \\
                    I \& \& I', \\
                };

                \draw[math]
                    (m-1-2) edge[tailed] node[swap] {\kappa} (m-2-1)
                        edge[tailed] node {\kappa'} (m-2-3)

                    (m-2-1) edge node[swap] {i} (m-2-3);
            \end{tikzpicture}
        \end{aligned}
        \hspace{0.5cm}
        \text{ and }
        \hspace{0.5cm}
        \begin{aligned}
            \begin{tikzpicture}
                \diagram{m}{1cm}{0.5cm} {
                    \& A \\
                    I \& \& I', \\
                };

                \draw[math]
                    (m-1-2) edge[tailed] node[swap] {\kappa} (m-2-1)
                        edge[tailed] node {\kappa'} (m-2-3)

                    (m-2-3) edge node {i'} (m-2-1);
            \end{tikzpicture}
        \end{aligned}
    \]
    commute.

    Using those morphisms, we can create the following commutative diagram, where \( \alpha \) and \( \beta \) are the induced morphisms from the pushout property of \( C \) and \( C' \) respectively, 
    \begin{center}
        \begin{tikzpicture}
            \diagramorigin{m}{1cm}{1.5cm} {
                A \& B \\
                I \& C \\
                I' \& \& C' \\
                I \& \& \& C. \\
            };

            \draw[math]
                (m-1-1) edge node {f} (m-1-2)
                    edge[curve={height=50pt}, tailed] node {\kappa} (m-4-1)
                    edge[curve={height=25pt}, tailed] node {\kappa'} (m-3-1)
                    edge[tailed] node {\kappa} (m-2-1)
                (m-1-2) edge[tailed] node {g} (m-2-2)
                    edge[curve={height=-25pt}, tailed] node {g'} (m-3-3)
                    edge[curve={height=-50pt}, tailed] node {g} (m-4-4)

                (m-2-1) edge node {\gamma} (m-2-2)
                    edge node {i} (m-3-1)
                (m-2-2) edge[dashed] node {\alpha} (m-3-3)

                (m-3-1) edge node {\gamma'} (m-3-3)
                    edge node {i'} (m-4-1)
                (m-3-3) edge[dashed] node {\beta} (m-4-4)

                (m-4-1) edge node {\gamma} (m-4-4);
        \end{tikzpicture}
    \end{center}
    We want to show that \( [\beta \circ \alpha] = [\Id_C] \).

    By the definition of a pushout, the following sequence is exact
    \begin{center}
        \begin{tikzpicture}
            \diagram{m}{1cm}{1cm} {
                A \& B \oplus I \& C. \\
            };

            \draw[math]
                (m-1-1) edge[tailed] node {
                    \begin{pmatrix}
                        f \\
                        \kappa
                    \end{pmatrix}
                } (m-1-2)
                (m-1-2) edge[two headed] node {
                    \begin{pmatrix}
                        g & \gamma
                    \end{pmatrix}
                } (m-1-3);
        \end{tikzpicture}
    \end{center}
    Consider the following commutative diagram, excluding the dashed arrow,
    \begin{diagramlabel}[\label{diag:C-iso}]
        \begin{tikzpicture}
            \diagram{m}{1cm}{1cm} {
                A \& B \oplus I \& C \\
                \& I \& C. \\
            };

            \draw[math]
                (m-1-1) edge[tailed] node {
                    \begin{pmatrix}
                        f \\
                        \kappa
                    \end{pmatrix}
                } (m-1-2)
                (m-1-2) edge[two headed] node {
                    \begin{pmatrix}
                        g & \gamma
                    \end{pmatrix}
                } (m-1-3)
                (m-1-2) edge node[swap] {
                    \begin{pmatrix}
                        0 & i' \circ i - \Id_I
                    \end{pmatrix}
                } (m-2-2)
                (m-1-3) edge[dashed] node[swap] {\delta} (m-2-2)
                    edge node {\beta \circ \alpha - \Id_C} (m-2-3)

                (m-2-2) edge node[swap] {\gamma} (m-2-3);
        \end{tikzpicture}
    \end{diagramlabel}

    Since
    \[
        \begin{pmatrix}
            0 & i' \circ i - \Id_I
        \end{pmatrix}
        \begin{pmatrix}
            f \\
            \kappa
        \end{pmatrix}
        = 0,
    \]
    then by the cokernel property of \( C \) there exists a morphism \( \delta \) such that the top triangle in \autoref{diag:C-iso} commutes. However, since \( 
        \begin{pmatrix}
            g & \gamma
        \end{pmatrix}
    \) is an epimorphism by definition, it follows that the bottom triangle also commutes, which implies \( [\beta \circ \alpha] = [\Id_C] \).

    We can prove that \( [\alpha \circ \beta] = [\Id_{C'}] \) in a similar way.
\end{proof}

We can now prove that \( \tuple*{\Mc, \Sigma, \Delta} \) is triangulated, where our proof is inspired by \cite[p.\ 16]{Happel_1988} and \cite[Theorem 4.18]{Johan_Bachelor}.

The details of the proof will not be important for the rest of the thesis, but is included mainly because we will be using \( \Mc \) for examples later on.

\begin{theorem}
    \label{example:stable_module_category_triangulated}
    The tuple \( \tuple*{\Mc, \Sigma, \Delta} \) is a triangulated category.
\end{theorem}
% TODO: Kanskje formuler pushout-unikheit eigenskapen på ein kortare måte, kanskje eit lemma?
% TODO: Legg til referansar til dei ulike diagramma i likningane for å gjere det enklare å finna ut kor likskapane kjem frå.
\begin{proof}
    We need to prove {\bf (TR1)} -- {\bf (TR4)} from \autoref{def:triangulated_category}.
    \begin{enumerate}[label={(\bfseries TR\arabic*)}]
        \item {
            \begin{enumerate}
                \item {
                    Let \( [f] \in \Mc(A, B) \).
                    
                    Then by the definition of the distinguished triangles in \( \Mc \), the following triangle,
                    \begin{center}
                        \begin{tikzpicture}
                            \diagram{m}{1cm}{1cm} {
                                A \& B \& C_f \& \Sigma A, \\
                            };

                            \draw[math]
                                (m-1-1) edge node {[f]} (m-1-2)
                                (m-1-2) edge node {[g]} (m-1-3)
                                (m-1-3) edge node {[h]} (m-1-4);
                        \end{tikzpicture}
                    \end{center}
                    is distinguished.
                }
                \item {
                    Let \( A \in \Mc \).
                    
                    By definition, \( C_{\Id_A} \) is the pushout
                    \begin{center}
                        \begin{tikzpicture}
                            \diagram{m}{1cm}{1cm} {
                                A \& A \\
                                I_A \& C_{\Id_A}. \\
                            };

                            \draw[math]
                                (m-1-1) edge node {\Id_A} (m-1-2)
                                    edge[tailed] node {\kappa_A} (m-2-1)
                                (m-1-2) edge[tailed] (m-2-2)

                                (m-2-1) edge node {\gamma_{\Id_A}} (m-2-2);
                        \end{tikzpicture}
                    \end{center}
                    Since the pushout of an isomorphism is an isomorphism, \( \gamma_{\Id_A} \) is an isomorphism, which implies \( C_{\Id_A} \cong 0 \) in \( \Mc \) because all injective modules are projective in \( \Mc \). This yields the trivial triangle.
                }
                \item {
                    \( \Delta \) is closed under isomorphisms of triangles by definition.
                }
            \end{enumerate}
        }
        \item {
            We need to show \( (\Leftarrow) \) and \( (\Rightarrow) \). By \autoref{lem:triangulated_category-TR2-only_one_rotation}, assuming the upcoming proof of {\bf (TR3)} as well as the previous proof of {\bf (TR1)}, then \( (\Leftarrow) \) is implied by \( (\Rightarrow) \).
            
            Therefore, it is sufficient to only prove \( (\Rightarrow) \), as long as we do not assume {\bf (TR2)} in the proof of {\bf (TR3)}.

            First note that a left rotated distinguished triangle will be isomorphic to a left rotated standard triangle. Therefore, it suffices to check that every left rotated standard triangle is distinguished.
            
            We will prove this by picking an arbitrary distinguished triangle, and use it to create a new distinguished triangle which is isomorphic to the left rotated standard triangle.

            Consider the following standard triangle
            \begin{center}
                \begin{tikzpicture}
                    \diagram{m}{1cm}{1cm} {
                        A \& B \& C_f \& \Sigma A \\
                    };

                    \draw[math]
                        (m-1-1) edge node {[f]} (m-1-2)
                        (m-1-2) edge node {[g]} (m-1-3)
                        (m-1-3) edge node {[h]} (m-1-4);
                \end{tikzpicture}
            \end{center}

            Recall the following commutative diagrams given by the definition of \( \Sigma [f] \) (\autoref{def:stmod_sigma}) and the construction of the above standard triangle (\autoref{rem:stmod_cone}),
            \[
                \begin{aligned}
                    \begin{tikzpicture}
                        \diagram{m}{1cm}{1cm} {
                            A \& I_A \& \Sigma A \\
                            B \& I_B \& \Sigma B, \\
                        };

                        \draw[math]
                            (m-1-1) edge[tailed] node {\kappa_A} (m-1-2)
                                edge node {f} (m-2-1)
                            (m-1-2) edge[two headed] node {\rho_A} (m-1-3)
                                edge node {i_f} (m-2-2)
                            (m-1-3) edge node {\Sigma f} (m-2-3)

                            (m-2-1) edge[tailed] node {\kappa_B} (m-2-2)
                            (m-2-2) edge[two headed] node {\rho_B} (m-2-3);
                    \end{tikzpicture}
                \end{aligned}
                \hspace{0.5cm}
                \text{and}
                \hspace{0.5cm}
                \begin{aligned}
                    \begin{tikzpicture}
                        \diagram{m}{1cm}{1cm} {
                            A \& B \\
                            I_A \& C_f \\
                            \& \Sigma A. \\
                        };

                        \draw[math]
                            (m-1-1) edge node {f} (m-1-2)
                                edge[tailed] node {\kappa_A} (m-2-1)
                            (m-1-2) edge[tailed] node {g} (m-2-2)

                            (m-2-1) edge node {\gamma_f} (m-2-2)
                                edge[two headed] node {\rho_A} (m-3-2)
                            (m-2-2) edge[two headed] node {h} (m-3-2);
                    \end{tikzpicture}
                \end{aligned}
            \]
            With the above diagrams in mind, consider the following commutative diagram excluding the dashed arrow,
            \begin{center}
                \begin{tikzpicture}
                    \diagram{m}{1cm}{1cm} {
                        A \& B \\
                        I_A \& C_f \& I_B \& \Sigma B, \\
                    };

                    \draw[math]
                        (m-1-1) edge node {f} (m-1-2)
                            edge[tailed] node {\kappa_A} (m-2-1)
                        (m-1-2) edge[tailed] node {g} (m-2-2)
                            edge[tailed] node {\kappa_B} (m-2-3)

                        (m-2-1) edge node {\gamma_f} (m-2-2)
                            edge[curve={height=1cm}] node {i_f} (m-2-3)
                        (m-2-2) edge[dashed] node {\phi} (m-2-3)

                        (m-2-3) edge[two headed] node {\rho_B} (m-2-4);
                \end{tikzpicture}
            \end{center}
            where there exists some \( \phi \), given by the pushout universal property.

            By the commutativity of the above diagrams, we have
            \[
                \rho_B \circ \phi \circ g = \rho_B \circ \kappa_B = 0 = (\Sigma f) \circ h \circ g,
            \]
            and
            \[
                \rho_B \circ \phi \circ \gamma_f = \rho_B \circ i_f = (\Sigma f) \circ \rho_A = (\Sigma f) \circ h \circ \gamma_f.
            \]
            This implies that the following pushout diagram
            \begin{center}
                \begin{tikzpicture}
                    \diagram{m}{1cm}{1cm} {
                        A \& B \\
                        I_A \& C_f \& \Sigma B \\
                    };

                    \draw[math]
                        (m-1-1) edge node {f} (m-1-2)
                            edge[tailed] node {\kappa_A} (m-2-1)
                        (m-1-2) edge[tailed] node {g} (m-2-2)
                            edge node {0} (m-2-3)

                        (m-2-1) edge node {\gamma_f} (m-2-2)
                            edge[curve={height=1cm}] node {\rho_B \circ i_f} (m-2-3)
                        (m-2-2) edge[dashed] (m-2-3);
                \end{tikzpicture}
            \end{center}
            has two different morphisms, \( \rho_B \circ \phi \) and \( (\Sigma f) \circ h \), that could satisfy the pushout universal property. By uniqueness, they have to be the same morphism, i.e.,
            \[
                \rho_B \circ \phi = (\Sigma f) \circ h.
            \]
            Finally, consider the following commutative diagram
            \begin{center}
                \begin{tikzpicture}
                    \diagram{m}{1cm}{1.5cm} {
                        0 \& B \& C_f \& \Sigma A \& 0 \\
                        0 \& I_B \& I_B \oplus \Sigma A \& \Sigma A \& 0 \\
                        \& \& \Sigma B \\
                    };

                    \draw[math]
                        (m-1-1) edge (m-1-2)
                            edge[equality] (m-2-1)
                        (m-1-2) edge[tailed] node {g} (m-1-3)
                            edge[tailed] node {\kappa_B} (m-2-2)
                        (m-1-3) edge[two headed] node {h} (m-1-4)
                            edge node {
                                \begin{pmatrix}
                                    \phi \\
                                    h
                                \end{pmatrix}
                            } (m-2-3)
                        (m-1-4) edge (m-1-5)
                                edge[equality] (m-2-4)

                        (m-2-1) edge (m-2-2)
                        (m-2-2) edge[tailed] node {
                            \begin{pmatrix}
                                1 \\
                                0
                            \end{pmatrix}
                        } (m-2-3)
                            edge[two headed] node {\rho_B} (m-3-3)
                        (m-2-3) edge[two headed] node {
                            \begin{pmatrix}
                                0 & 1
                            \end{pmatrix}
                        } (m-2-4)
                            edge node {
                                \begin{pmatrix}
                                    \rho_B & -\Sigma f
                                \end{pmatrix}
                            } (m-3-3)
                        (m-2-4) edge (m-2-5);
                \end{tikzpicture}
            \end{center}
            where the top row is exact by \autoref{rem:stmod_cone_pushout_properties}, and the second row is split-exact. It follows that the middle square is in fact a pushout, which implies that \( I_B \oplus \Sigma A  \) is uniquely isomorphic to \( C_g \).
            
            Since
            \[
                \begin{pmatrix}
                    \rho_B & -\Sigma f
                \end{pmatrix}
                \begin{pmatrix}
                    \phi \\
                    h
                \end{pmatrix}
                =
                \rho_B \circ \phi - (\Sigma f) \circ h = 0,
            \]
            the triangle
            \begin{center}
                \begin{tikzpicture}
                    \diagram{m}{1cm}{2cm} {
                        B \& C_f \& I_B \oplus \Sigma A \& \Sigma B \\
                    };

                    \draw[math]
                        (m-1-1) edge node {[g]} (m-1-2)
                        (m-1-2) edge node {\class*{
                            \begin{pmatrix}
                                \phi \\
                                h
                            \end{pmatrix}
                        }} (m-1-3)
                        (m-1-3) edge node {\class*{
                            \begin{pmatrix}
                                \rho_B & -\Sigma f
                            \end{pmatrix}
                        }} (m-1-4);
                \end{tikzpicture}
            \end{center} 
            is a standard triangle of \( g \), and therefore distinguished.

            Finally, it remains to check if the triangle is isomorphic to the expected triangle. Consider the following diagram in \( \Mc \)
            \begin{center}
                \begin{tikzpicture}
                    \diagram{m}{1cm}{2cm} {
                        B \& C_f \& I_B \oplus \Sigma A \& \Sigma B \\
                        B \& C_f \& \Sigma A \& \Sigma B. \\
                    };

                    \draw[math]
                        (m-1-1) edge node {[g]} (m-1-2)
                            edge[equality] (m-2-1)
                        (m-1-2) edge node {\class*{
                            \begin{pmatrix}
                                \phi \\
                                h
                            \end{pmatrix}
                        }} (m-1-3)
                            edge[equality] (m-2-2)
                        (m-1-3) edge node {\class*{
                            \begin{pmatrix}
                                \rho_B & -\Sigma f
                            \end{pmatrix}
                        }} (m-1-4)
                            edge node {\class*{
                                \begin{pmatrix}
                                    0 & 1
                                \end{pmatrix}
                            }} (m-2-3)
                        (m-1-4) edge[equality] (m-2-4)

                        (m-2-1) edge node {[g]} (m-2-2)
                        (m-2-2) edge node {[h]} (m-2-3)
                        (m-2-3) edge node {[-\Sigma f]} (m-2-4);
                \end{tikzpicture}
            \end{center}
            The left and the middle square commute directly, but it remains to check if the right square commutes. Additionally, we need to check that \( \class*{
                \begin{psmallmatrix}
                    0 & 1
                \end{psmallmatrix}
            } \) is an isomorphism.

            First, we check the commutativity. Consider the difference
            \[
                \begin{pmatrix}
                    \rho_B & -\Sigma f
                \end{pmatrix}
                -
                (-\Sigma f) \circ
                \begin{pmatrix}
                    0 & 1
                \end{pmatrix}
                =
                \begin{pmatrix}
                    \rho_B & -\Sigma f
                \end{pmatrix}
                +
                \begin{pmatrix}
                    0 & \Sigma f
                \end{pmatrix}
                =
                \begin{pmatrix}
                    \rho_B & 0
                \end{pmatrix}.
            \]
            We can see that the diagram,
            \begin{center}
                \begin{tikzpicture}
                    \diagram{m}{1cm}{1cm} {
                        I_B \oplus \Sigma A \& \& \Sigma B \\
                        \& I_B, \\
                    };

                    \draw[math]
                        (m-1-1) edge[two headed] node {
                            \begin{pmatrix}
                                \rho_B & 0
                            \end{pmatrix}
                        } (m-1-3)
                            edge[two headed] node[swap] {
                                \begin{pmatrix}
                                    1 & 0
                                \end{pmatrix}
                            } (m-2-2)

                        (m-2-2) edge[two headed] node[swap] {\rho_B} (m-1-3);
                \end{tikzpicture}
            \end{center}
            commutes, and since every injective module is also projective,
            \[
                \class*{
                    \begin{pmatrix}
                        \rho_B & -\Sigma f
                    \end{pmatrix}
                }
                =
                \class*{
                    -(\Sigma f) \circ
                    \begin{pmatrix}
                        0 & 1
                    \end{pmatrix}
                }.
            \]

            Second, we check that \( \class*{
                \begin{psmallmatrix}
                    0 & 1
                \end{psmallmatrix}
            } \) is an isomorphism.

            Consider the morphism \( \class*{
                \begin{psmallmatrix}
                    0 \\
                    1
                \end{psmallmatrix}
            } \).
            Note that \( \class*{
                \begin{psmallmatrix}
                    0 & 1
                \end{psmallmatrix}
            } \circ \class*{
                \begin{psmallmatrix}
                    0 \\
                    1
                \end{psmallmatrix}
            } = [\Id_A] \).
            It remains to check if
            \[
                \Id_{I_B \oplus \Sigma A} -
                \begin{pmatrix}
                    0 \\
                    1
                \end{pmatrix}
                \begin{pmatrix}
                    0 & 1
                \end{pmatrix}
                =
                \begin{pmatrix}
                    1 & 0 \\
                    0 & 0
                \end{pmatrix}
            \]
            factors through a projective module.

            This follows from the following commutative diagram,
            \begin{center}
                \begin{tikzpicture}
                    \diagram{m}{1cm}{1cm} {
                        I_B \oplus \Sigma A \& \& I_B \oplus \Sigma A \\
                        \& I_B. \\
                    };

                    \draw[math]
                        (m-1-1) edge node {
                            \begin{pmatrix}
                                1 & 0 \\
                                0 & 0
                            \end{pmatrix}
                        } (m-1-3)
                            edge[two headed] node[swap] {
                                \begin{pmatrix}
                                    1 & 0
                                \end{pmatrix}
                            } (m-2-2)

                        (m-2-2) edge[tailed] node[swap] {
                            \begin{pmatrix}
                                1 \\
                                0
                            \end{pmatrix}
                        } (m-1-3);
                \end{tikzpicture}
            \end{center}
        }
        \item {
            By considering two arbitrary distinguished triangles, any morphism between their components will induce a unique morphism between the component of their standard triangles that they are isomorphic to, and vice versa. Therefore, it suffices to only check {\bf (TR3)} for standard triangles.

            In addition, by the argument made in {\bf (TR2)}, we can not assume {\bf (TR2)} in this proof, as that would yield a circular argument.

            We need to show that given the following commutative diagram, excluding the dashed arrow, where the top and bottom row are standard triangles,
            \begin{diagramlabel}[\label{eq:stablemod}]
                \begin{tikzpicture}
                    \diagram{m}{1cm}{1cm} {
                        A \& B  \& C_f \& \Sigma A \\
                        D \& E \& C_l \& \Sigma D, \\
                    };

                    \draw[math]
                        (m-1-1) edge node {[f]} (m-1-2)
                            edge node {[\alpha]} (m-2-1)
                        (m-1-2) edge node {[g]} (m-1-3)
                            edge node {[\beta]} (m-2-2)
                        (m-1-3) edge node {[h]} (m-1-4)
                            edge[dashed] node {[\phi]} (m-2-3)
                        (m-1-4) edge node {\Sigma [\alpha]} (m-2-4)

                        (m-2-1) edge node {[l]} (m-2-2)
                        (m-2-2) edge node {[m]} (m-2-3)
                        (m-2-3) edge node {[n]} (m-2-4);
                \end{tikzpicture}
            \end{diagramlabel}
            that there exists some \( \phi: C_f \to C_l \) such that the entire diagram, including \( \phi \), commutes.

            Consider the following commutative diagram in \( \Mod(R) \) from the definition of the top standard triangle from the diagram above,
            \[
                \begin{aligned}
                    \begin{tikzpicture}
                        \diagram{m}{1cm}{1cm} {
                            A \& B \\
                            I_A \& C_f \\
                            \& \Sigma A. \\
                        };
    
                        \draw[math]
                            (m-1-1) edge node {f} (m-1-2)
                                edge[tailed] node {\kappa_A} (m-2-1)
                            (m-1-2) edge[tailed] node {g} (m-2-2)
    
                            (m-2-1) edge node {\gamma_f} (m-2-2)
                                edge[two headed] node {\rho_A} (m-3-2)
                            (m-2-2) edge[two headed] node {h} (m-3-2);
                    \end{tikzpicture}
                \end{aligned}
            \]

            Since \( [l] \circ [\alpha] = [\beta] \circ [f] \) in \( \Mc \), we have that \( l \circ \alpha - \beta \circ f \) factors through a projective module, \( Q \). However, since projective modules are injective modules, then by the injective module universal property of \( Q \) applied to the monomorphism \( \kappa_A \), it follows that there is some morphism \( \xi: I_A \to E \) such that
            \[
                l \circ \alpha - \beta \circ f = \xi \circ \kappa_A,
            \]
            which is equivalent to
            \[
                \beta \circ f = l \circ \alpha - \xi \circ \kappa_A.
            \]

            Consider the following commutative diagrams from the definition of \( \Sigma [\alpha] \) and from the definition of the lower standard triangle in \autoref{eq:stablemod},
            \[
                \begin{aligned}
                    \begin{tikzpicture}
                        \diagram{m}{1cm}{1cm} {
                            A \& I_A \& \Sigma A \\
                            D \& I_{D} \& \Sigma D, \\
                        };

                        \draw[math]
                            (m-1-1) edge[tailed] node {\kappa_A} (m-1-2)
                                edge node {\alpha} (m-2-1)
                            (m-1-2) edge[two headed] node {\rho_A} (m-1-3)
                                edge node {i_{\alpha}} (m-2-2)
                            (m-1-3) edge node {\Sigma \alpha} (m-2-3)

                            (m-2-1) edge[tailed] node {\kappa_{D}} (m-2-2)
                            (m-2-2) edge[two headed] node {\rho_{D}} (m-2-3);
                    \end{tikzpicture}
                \end{aligned}
                \quad
                \text{and}
                \quad
                \begin{aligned}
                    \begin{aligned}
                        \begin{tikzpicture}
                            \diagram{m}{1cm}{1cm} {
                                D \& E \\
                                I_{D} \& C_{l} \\
                                \& \Sigma D. \\
                            };
        
                            \draw[math]
                                (m-1-1) edge node {l} (m-1-2)
                                    edge[tailed] node {\kappa_{D}} (m-2-1)
                                (m-1-2) edge[tailed] node {m} (m-2-2)
        
                                (m-2-1) edge node {\gamma_{l}} (m-2-2)
                                    edge[two headed] node {\rho_{D}} (m-3-2)
                                (m-2-2) edge[two headed] node {n} (m-3-2);
                        \end{tikzpicture}
                    \end{aligned}
                \end{aligned}
            \]

            Since
            \begin{align*}
                m \circ \beta \circ f &= m \circ l \circ \alpha - m \circ \xi \circ \kappa_A \\
                &= \gamma_{l} \circ \kappa_{D} \circ \alpha - m \circ \xi \circ \kappa_A \\
                &= \gamma_{l} \circ i_{\alpha} \circ \kappa_A - m \circ \xi \circ \kappa_A \\
                &= (\gamma_{l} \circ i_{\alpha} - m \circ \xi) \circ \kappa_A,
            \end{align*}
            then by the pushout property of \( C_f \) it follows that there exists some unique \( \phi \) such that the following diagram,
            \begin{center}
                \begin{tikzpicture}
                    \diagram{m}{1cm}{1cm} {
                        A \& B \\
                        I_A \& C_f \\
                        \& \& C_{l}, \\
                    };

                    \draw[math]
                        (m-1-1) edge node {f} (m-1-2)
                            edge[tailed] node {\kappa_A} (m-2-1)
                        (m-1-2) edge[tailed] node {g} (m-2-2)
                            edge[curve={height=-25pt}] node {m \circ \beta} (m-3-3)

                        (m-2-1) edge node {\gamma_f} (m-2-2)
                            edge[curve={height=25pt}] node[swap] {\gamma_{l} \circ i_{\alpha} - m \circ \xi} (m-3-3)
                        (m-2-2) edge[dashed] node {\phi} (m-3-3);
                \end{tikzpicture}
            \end{center}
            commutes.

            In particular, note that \( \phi \circ \gamma_f = \gamma_{l} \circ i_{\alpha} - m \circ \xi \).

            Finally, we check that this \( \phi \) makes \autoref{eq:stablemod} commute.

            By the commutativity of the pushout diagram, the middle square of \autoref{eq:stablemod} commutes. Then it remains to check \( [n \circ \phi] = [(\Sigma \alpha) \circ h] \). Consider the following diagram excluding the dashed arrow,
            \begin{center}
                \begin{tikzpicture}
                    \diagram{m}{1cm}{1cm} {
                        A \& B \\
                        I_A \& C_f \\
                        \& \& \Sigma D. \\
                    };

                    \draw[math]
                        (m-1-1) edge node {f} (m-1-2)
                            edge[tailed] node {\kappa_A} (m-2-1)
                        (m-1-2) edge[tailed] node {g} (m-2-2)
                            edge[curve={height=-25pt}] node {0} (m-3-3)

                        (m-2-1) edge node {\gamma_f} (m-2-2)
                            edge[curve={height=25pt}] node[swap] {(\Sigma \alpha) \circ h \circ \gamma_f} (m-3-3)
                        (m-2-2) edge[dashed] (m-3-3);
                \end{tikzpicture}
            \end{center}
            The above diagram commutes because \( (\Sigma \alpha) \circ h \circ \gamma_f \circ \kappa_A = (\Sigma \alpha) \circ \rho_A \circ \kappa_A = 0 \).

            We want to show that both \( (\Sigma \alpha) \circ h \) and \( n \circ \phi \) could fit as the dashed line in the above diagram. Then by uniqueness of the pushout, this would imply that they are equal, which is what we wish to prove.

            Consider the following equations
            \[
                n \circ \phi \circ g = n \circ m \circ \beta = 0 = (\Sigma \alpha) \circ h \circ g,
            \]
            and
            \begin{align*}
                (\Sigma \alpha) \circ h \circ \gamma_f &= (\Sigma \alpha) \circ \rho_A \\
                &= \rho_{D} \circ i_{\alpha} \\
                &= n \circ \gamma_{l} \circ i_{\alpha} \\
                &= n \circ (\phi \circ \gamma_f + m \circ \xi) \\
                &= n \circ \phi \circ \gamma_f.
            \end{align*}

            These imply that both could fit as the dashed arrow. Therefore,
            \[
                (\Sigma \alpha) \circ h = n \circ \phi,
            \]
            and \autoref{eq:stablemod} commutes.
        }
        \item {
            % TODO: Swap m og n?
            By a similar argument as in {\bf (TR3)} it is sufficient to only check {\bf (TR4)} for standard triangles.

            Consider three standard triangles
            \begin{center}
                \begin{tikzpicture}
                    \diagram{m}{0.7cm}{1cm} {
                        A \& B \& C_f \& \Sigma A, \\
                        B \& D \& C_n \& \Sigma B, \\
                    };

                    \draw[math]
                        (m-1-1) edge node {[f]} (m-1-2)
                        (m-1-2) edge node {[g]} (m-1-3)
                        (m-1-3) edge node {[h]} (m-1-4)

                        (m-2-1) edge node {[n]} (m-2-2)
                        (m-2-2) edge node {[m]} (m-2-3)
                        (m-2-3) edge node {[k]} (m-2-4);
                \end{tikzpicture}
            \end{center}
            and
            \begin{center}
                \begin{tikzpicture}
                    \diagram{m}{0.7cm}{1cm} {
                        A \& D \& C_{n \circ f} \& \Sigma A, \\
                    };
                    
                    \draw[math]
                        (m-1-1) edge node {[n \circ f]} (m-1-2)
                        (m-1-2) edge node {[j]} (m-1-3)
                        (m-1-3) edge node {[l]} (m-1-4);
                \end{tikzpicture}
            \end{center}
            which fit into the following commutative diagram excluding the dashed arrows,
            \begin{center}
                \begin{tikzpicture}
                    \diagram{m}{1cm}{1cm} {
                        A \& B \& C_f \& \Sigma A \\
                        A \& D \& C_{n \circ f} \& \Sigma A \\
                        \& C_n \& C_n \& \Sigma B \\
                        \& \Sigma B \& \Sigma C_f. \\
                    };
        
                    \draw[math]
                        (m-1-1) edge node {[f]} (m-1-2)
                            edge[equality] (m-2-1)
                        (m-1-2) edge node {[g]} (m-1-3)
                            edge node {[n]} (m-2-2)
                        (m-1-3) edge node {[h]} (m-1-4)
                            edge[dashed] node {[\phi]} (m-2-3)
                        (m-1-4) edge[equality] (m-2-4)
        
                        (m-2-1) edge node {[n \circ f]} (m-2-2)
                        (m-2-2) edge node {[j]} (m-2-3)
                            edge node {[m]} (m-3-2)
                        (m-2-3) edge node {[l]} (m-2-4)
                            edge[dashed] node {[\psi]} (m-3-3)
                        (m-2-4) edge node {\Sigma [f]} (m-3-4)

                        (m-3-2) edge[equality] (m-3-3)
                            edge node[swap] {[k]} (m-4-2)
                        (m-3-3) edge node {[k]} (m-3-4)
                            edge node {[(\Sigma g) \circ k]} (m-4-3)

                        (m-4-2) edge node {\Sigma [g]} (m-4-3);
                \end{tikzpicture}
            \end{center}
            We need to show that there exist morphisms \( [\phi] \) and \( [\psi] \) that fit into the above commutative diagram, such that
            \begin{diagramlabel}[\label{tri:stmod_tr4}]
                \begin{tikzpicture}
                    \diagram{m}{1cm}{1.5cm} {
                        C_f \& C_{n \circ f} \& C_n \& \Sigma C_f \\
                    };

                    \draw[math]
                        (m-1-1) edge node {[\phi]} (m-1-2)
                        (m-1-2) edge node {[\psi]} (m-1-3)
                        (m-1-3) edge node {[(\Sigma g) \circ k]} (m-1-4);
                \end{tikzpicture}
            \end{diagramlabel}
            is a distinguished triangle.

            Consider the following commutative diagrams from the constructions of the three standard triangles mentioned above (\autoref{rem:stmod_cone}),
            \begin{center}
                \begin{tikzpicture}
                    \diagram{m}{1cm}{1cm} {
                        A \& B \& B \& D \& \& A \& D \\
                        I_A \& C_f \& I_B \& C_n \& \text{and} \& I_A \& C_{n \circ f} \\
                        \& \Sigma A, \& \& \Sigma B, \& \& \& \Sigma A. \\
                    };

                    \draw[math]
                        (m-1-1) edge node {f} (m-1-2)
                            edge[tailed] node {\kappa_A} (m-2-1)
                        (m-1-2) edge[tailed] node {g} (m-2-2)
                        (m-1-3) edge node {n} (m-1-4)
                            edge[tailed] node {\kappa_B} (m-2-3)
                        (m-1-4) edge[tailed] node {m} (m-2-4)
                        (m-1-6) edge node {n \circ f} (m-1-7)
                            edge[tailed] node {\kappa_A} (m-2-6)
                        (m-1-7) edge[tailed] node {j} (m-2-7)

                        (m-2-1) edge node {\gamma_f} (m-2-2)
                            edge[two headed] node[swap] {\rho_A} (m-3-2)
                        (m-2-2) edge[two headed] node {h} (m-3-2)
                        (m-2-3) edge node {\gamma_n} (m-2-4)
                            edge[two headed] node[swap] {\rho_B} (m-3-4)
                        (m-2-4) edge[two headed] node {k} (m-3-4)
                        (m-2-6) edge node {\gamma_{n \circ f}} (m-2-7)
                            edge[two headed] node[swap] {\rho_A} (m-3-7)
                        (m-2-7) edge[two headed] node {l} (m-3-7);
                \end{tikzpicture}
            \end{center}

            In order to construct \( [\phi] \) and \( [\psi] \), we will work with the following commutative diagram in \( \Mod(R) \), excluding the dashed arrows,
            \begin{diagramlabel}[\label{diag:stmod_tr4}]
                \begin{tikzpicture}
                    \diagram{m}{1cm}{1cm} {
                        A \& B \& C_f \& \Sigma A \\
                        A \& D \& C_{n \circ f} \& \Sigma A \\
                        \& C_n \& C_n \& \Sigma B \\
                        \& \Sigma B \& \Sigma C_f. \\
                    };
        
                    \draw[math]
                        (m-1-1) edge node {f} (m-1-2)
                            edge[equality] (m-2-1)
                        (m-1-2) edge[tailed] node {g} (m-1-3)
                            edge node {n} (m-2-2)
                        (m-1-3) edge[two headed] node {h} (m-1-4)
                            edge[dashed] node {\phi} (m-2-3)
                        (m-1-4) edge[equality] (m-2-4)
        
                        (m-2-1) edge node {n \circ f} (m-2-2)
                        (m-2-2) edge[tailed] node {j} (m-2-3)
                            edge[tailed] node {m} (m-3-2)
                        (m-2-3) edge[two headed] node {l} (m-2-4)
                            edge[dashed] node {\psi} (m-3-3)
                        (m-2-4) edge node {\Sigma f} (m-3-4)

                        (m-3-2) edge[equality] (m-3-3)
                            edge[two headed] node[swap] {k} (m-4-2)
                        (m-3-3) edge[two headed] node {k} (m-3-4)
                            edge node {(\Sigma g) \circ k} (m-4-3)

                        (m-4-2) edge node {\Sigma g} (m-4-3);
                \end{tikzpicture}
            \end{diagramlabel}

            If we can find some \( \phi \) and \( \psi \) along with appropriate \( \Sigma f \) such that the above diagram commutes, then all we would have left to prove is that \autoref{tri:stmod_tr4} is a distinguished triangle for some \( \Sigma g \).

            Similar to what was done in the proof of {\bf (TR3)}, we choose \( \phi \) to be the (dashed) pushout morphism completing the following commutative diagram,
            \begin{center}
                \begin{tikzpicture}
                    \diagram{m}{1cm}{1cm} {
                        A \& B \\
                        I_A \& C_f \\
                        \& \& C_{n \circ f}. \\
                    };

                    \draw[math]
                        (m-1-1) edge node {f} (m-1-2)
                            edge[tailed] node {\kappa_A} (m-2-1)
                        (m-1-2) edge[tailed] node {g} (m-2-2)
                            edge[curve={height=-25pt}] node {j \circ n} (m-3-3)
                        
                        (m-2-1) edge node {\gamma_f} (m-2-2)
                            edge[curve={height=25pt}] node[swap] {\gamma_{n \circ f}} (m-3-3)
                        (m-2-2) edge[dashed] node {\phi} (m-3-3);
                \end{tikzpicture}
            \end{center}
            From this, we can see that the square left of \( \phi \) in \autoref{diag:stmod_tr4} commutes. In order to check if the square to the right commutes, consider the following commutative diagram excluding the dashed arrow,
            \begin{center}
                \begin{tikzpicture}
                    \diagram{m}{1cm}{1cm} {
                        A \& B \\
                        I_A \& C_f \\
                        \& \& \Sigma A. \\
                    };

                    \draw[math]
                        (m-1-1) edge node {f} (m-1-2)
                            edge node {\kappa_A} (m-2-1)
                        (m-1-2) edge node {g} (m-2-2)
                            edge[curve={height=-25pt}] node {l \circ j \circ n} (m-3-3)
                        
                        (m-2-1) edge node {\gamma_f} (m-2-2)
                            edge[curve={height=25pt}] node[swap] {l \circ \gamma_{n \circ f}} (m-3-3)
                        (m-2-2) edge[dashed] (m-3-3);
                \end{tikzpicture}
            \end{center}
            Consider the following equations
            \[
                l \circ \phi \circ g = l \circ j \circ n = 0 \circ n = 0 = h \circ g,
            \]
            and
            \[
                l \circ \phi \circ \gamma_f = l \circ \gamma_{n \circ f} = \rho_A = h \circ \gamma_f,
            \]
            which, by the pushout morphism uniqueness, implies that
            \[
                l \circ \phi = h.
            \]
            Therefore, the square to the right of \( \phi \) in \autoref{diag:stmod_tr4} also commutes.

            In order to construct \( \psi \) we have to first define two morphisms which will be used in its definition.

            Consider the following short exact sequence from the definition of \( \Sigma C_f \),
            \begin{center}
                \begin{tikzpicture}
                    \diagram{m}{1cm}{1cm} {
                        C_f \& I_{C_f} \& \Sigma C_f. \\
                    };

                    \draw[math]
                        (m-1-1) edge[tailed] node {\kappa_{C_f}} (m-1-2)
                        (m-1-2) edge[two headed] node {\rho_{C_f}} (m-1-3); 
                \end{tikzpicture}
            \end{center}

            By \autoref{rem:stmod_cone_pushout_properties}, \( g \) is a monomorphism. Since \( \kappa_{C_f} \) is also a monomorphism, it follows that
            \[
                \kappa_{C_f} \circ g: B \to I_{C_f}
            \]
            is also a monomorphism. Then by the injective object property of \( I_B \), there exists some morphism \( i: I_{C_f} \to I_B \) which makes the following diagram commute,
            \begin{center}
                \begin{tikzpicture}
                    \diagram{m}{1cm}{0.5cm} {
                        \& B \\
                        I_{C_f} \& \& I_B. \\
                    };

                    \draw[math]
                        (m-1-2) edge[tailed] node[swap] {\kappa_{C_f} \circ g} (m-2-1)
                            edge[tailed] node {\kappa_B} (m-2-3)

                        (m-2-1) edge node[swap] {i} (m-2-3);
                \end{tikzpicture}
            \end{center}
            By the injective property of \( I_{C_f} \), there also exists some \( \tilde{i} \) such that the following diagram commutes,
            \begin{center}
                \begin{tikzpicture}
                    \diagram{m}{1cm}{0.5cm} {
                        \& B \\
                        I_{C_f} \& \& I_B. \\
                    };

                    \draw[math]
                        (m-1-2) edge[tailed] node[swap] {\kappa_{C_f} \circ g} (m-2-1)
                            edge[tailed] node {\kappa_B} (m-2-3)

                        (m-2-3) edge node {\tilde{i}} (m-2-1);
                \end{tikzpicture}
            \end{center}
            
            Now we can move on to defining \( \psi \).

            Consider the following diagram excluding the dashed arrow,
            \begin{center}
                \begin{tikzpicture}
                    \diagram{m}{1cm}{1cm} {
                        A \& D \\
                        I_A \& C_{n \circ f} \\
                        \& \& C_n, \\
                    };

                    \draw[math]
                        (m-1-1) edge node {n \circ f} (m-1-2)
                            edge[tailed] node {\kappa_A} (m-2-1)
                        (m-1-2) edge[tailed] node {j} (m-2-2)
                            edge[curve={height=-25pt}, tailed] node {m} (m-3-3)

                        (m-2-1) edge node {\gamma_{n \circ f}} (m-2-2)
                            edge[curve={height=25pt}] node[swap] {\gamma_n \circ i \circ \kappa_{C_f} \circ \gamma_f} (m-3-3)
                        (m-2-2) edge[dashed] node {\psi} (m-3-3);
                \end{tikzpicture}
            \end{center}
            which commutes because
            \begin{equation}
                \label{diag:stmod_tr4_m_n_f_equal_something_else}
                \begin{aligned}
                    m \circ n \circ f &= \gamma_n \circ \kappa_B \circ f \\
                    &= \gamma_n \circ i \circ \kappa_{C_f} \circ g \circ f \\
                    &= \gamma_n \circ i \circ \kappa_{C_f} \circ \gamma_f \circ \kappa_A.
                \end{aligned}
            \end{equation}
            Thus, by the pushout property, there exists a morphism \( \psi: C_{n \circ f} \to C_n \) such that the above diagram including the dashed arrow commutes.

            To show that \( \psi \) makes \autoref{diag:stmod_tr4} commute in \( \Mc \), we can see that the square to the left of \( \psi \) commutes by definition, and so it remains to choose some \( \Sigma f \) such that the square to the right also commutes.

            Consider the following diagram excluding the dashed arrow,
            \begin{center}
                \begin{tikzpicture}
                    \diagram{m}{1cm}{1cm} {
                        A \& I_A \& \Sigma A \\
                        \& C_f \\
                        \& I_{C_f} \\
                        B \& I_B \& \Sigma B, \\
                    };

                    \draw[math]
                        (m-1-1) edge[tailed] node {\kappa_A} (m-1-2)
                            edge node[swap] {f} (m-4-1)
                        (m-1-2) edge[two headed] node {\rho_A} (m-1-3)
                            edge node {\gamma_f} (m-2-2)
                        (m-1-3) edge[dashed] node {\Sigma f} (m-4-3)

                        (m-2-2) edge[tailed] node {\kappa_{C_f}} (m-3-2)

                        (m-3-2) edge node {i} (m-4-2)

                        (m-4-1) edge[tailed] node {\kappa_B} (m-4-2)
                            edge[tailed] node {g} (m-2-2)
                        (m-4-2) edge[two headed] node {\rho_B} (m-4-3);
                \end{tikzpicture}
            \end{center}
            which commutes because the top left ``trapezoid'' and the bottom left ``triangle'' in the diagram commute. Then since
            \[
                \rho_B \circ i \circ \kappa_{C_f} \circ \gamma_f \circ \kappa_A = \rho_B \circ \kappa_B \circ f = 0,
            \]
            it follows by the cokernel property of \( \Sigma A \) that there exists some morphism, \( \Sigma f \), which is denoted as such because it fits with \autoref{def:stmod_sigma}.

            Then consider the following commutative diagram excluding the dashed arrow,
            \begin{center}
                \begin{tikzpicture}
                    \diagram{m}{1cm}{1cm} {
                        A \& D \\
                        I_A \& C_{n \circ f} \\
                        \& \& \Sigma B, \\
                    };

                    \draw[math]
                        (m-1-1) edge node {n \circ f} (m-1-2)
                            edge[tailed] node {\kappa_A} (m-2-1)
                        (m-1-2) edge[tailed] node {j} (m-2-2)
                            edge[curve={height=-25pt}] node {0} (m-3-3)

                        (m-2-1) edge node {\gamma_{n \circ f}} (m-2-2)
                            edge[curve={height=25pt}] node[swap] {k \circ \gamma_n \circ i \circ \kappa_{C_f} \circ \gamma_f} (m-3-3)
                        (m-2-2) edge[dashed] (m-3-3);
                \end{tikzpicture}
            \end{center}
            where
            \[
                (\Sigma f) \circ l \circ j = (\Sigma f) \circ 0 = 0 = k \circ m = k \circ \psi \circ j
            \]
            and
            \begin{align*}
                k \circ \psi \circ \gamma_{n \circ f} &= k \circ \gamma_n \circ i \circ \kappa_{C_f} \circ \gamma_f \\
                &= \rho_B \circ i \circ \kappa_{C_f} \circ \gamma_f \\
                &= (\Sigma f) \circ \rho_A \\
                &= (\Sigma f) \circ l \circ \gamma_{n \circ f}.
            \end{align*}
            Thus, by uniqueness of the pushout property it follows that \( k \circ \psi = (\Sigma f) \circ l \), and the square to the right of \( \psi \) in \autoref{diag:stmod_tr4} commutes.

            Finally, it remains to check that
            \begin{center}
                \begin{tikzpicture}
                    \diagram{m}{1cm}{1.5cm} {
                        C_f \& C_{n \circ f} \& C_n \& \Sigma C_f \\
                    };

                    \draw[math]
                        (m-1-1) edge node {[\phi]} (m-1-2)
                        (m-1-2) edge node {[\psi]} (m-1-3)
                        (m-1-3) edge node {(\Sigma[g]) \circ [k]} (m-1-4);
                \end{tikzpicture}
            \end{center}
            is a distinguished triangle.

            We will construct a standard triangle of \( [\phi] \) and then show that this triangle is isomorphic to the above triangle.

            Consider the following commutative diagram
            \begin{center}
                \begin{tikzpicture}
                    \diagram{m}{1cm}{1cm} {
                        A \& B \& D \\
                        I_A \& C_f \& C_{n \circ f}. \\
                    };

                    \draw[math]
                        (m-1-1) edge node {f} (m-1-2)
                            edge[tailed] node {\kappa_A} (m-2-1)
                        (m-1-2) edge node {n} (m-1-3)
                            edge[tailed] node {g} (m-2-2)
                        (m-1-3) edge[tailed] node {j} (m-2-3)

                        (m-2-1) edge node {\gamma_f} (m-2-2)
                            edge[curve={height=25pt}] node[swap] {\gamma_{n \circ f}} (m-2-3)
                        (m-2-2) edge node {\phi} (m-2-3);
                \end{tikzpicture}
            \end{center}
            Since the left and the outer square are pushouts this implies that the right square is also a pushout.

            Consider the following pushout diagram of \( \tuple*{ B, n, \kappa_{C_f} \circ g } \),
            \begin{diagramlabel}[\label{diag:stmod_tr4_C}]
                \begin{tikzpicture}
                    \diagram{m}{1cm}{1cm} {
                        B \& D \\
                        I_{C_f} \& C. \\
                    };

                    \draw[math]
                        (m-1-1) edge node {n} (m-1-2)
                            edge[tailed] node[swap] {\kappa_{C_f} \circ g} (m-2-1)
                        (m-1-2) edge[tailed] node {\delta} (m-2-2)

                        (m-2-1) edge node {\gamma} (m-2-2);
                \end{tikzpicture}
            \end{diagramlabel}

            Using the new pushout, consider the following commutative diagram excluding the dashed arrow,
            \begin{center}
                \begin{tikzpicture}
                    \diagram{m}{1cm}{1cm} {
                        B \& D \\
                        C_f \& C_{n \circ f} \\
                        \& \& C. \\
                    };

                    \draw[math]
                        (m-1-1) edge node {n} (m-1-2)
                            edge[tailed] node[swap] {g} (m-2-1)
                        (m-1-2) edge[tailed] node {j} (m-2-2)
                            edge[curve={height=-25pt}, tailed] node {\delta} (m-3-3)

                        (m-2-1) edge node {\phi} (m-2-2)
                            edge[curve={height=25pt}] node[swap] {\gamma \circ \kappa_{C_f}} (m-3-3)
                        (m-2-2) edge[dashed] node {x} (m-3-3);
                \end{tikzpicture}
            \end{center}
            By the pushout property, there exists a morphism \( x: C_{n \circ f} \to C \) such that the above diagram including \( x \) commutes.
            
            Consider the following commutative diagram,
            \begin{diagramlabel}[\label{diag:stmod_tr4_show_psi_phi_equal_something_else}]
                \begin{tikzpicture}
                    \diagram{m}{1cm}{1cm} {
                        B \& D \\
                        C_f \& C_{n \circ f} \\
                        I_{C_f} \& C. \\
                    };
                    
                    \draw[math]
                        (m-1-1) edge node {n} (m-1-2)
                            edge[tailed] node[swap] {g} (m-2-1)
                        (m-1-2) edge[tailed] node[swap] {j} (m-2-2)
                            edge[curve={height=-25pt}, tailed] node {\delta} (m-3-2)

                        (m-2-1) edge node {\phi} (m-2-2)
                            edge[tailed] node[swap] {\kappa_{C_f}} (m-3-1)
                        (m-2-2) edge node[swap] {x} (m-3-2)

                        (m-3-1) edge[tailed] node {\gamma} (m-3-2);
                \end{tikzpicture}
            \end{diagramlabel}

            Since the outer rectangle and upper square of \autoref{diag:stmod_tr4_show_psi_phi_equal_something_else} are pushouts, then the lower square is also a pushout.

            This implies that there exists a morphism \( p \) by the pushout property, such that the following diagram
            \begin{center}
                \begin{tikzpicture}
                    \diagram{m}{1cm}{1cm} {
                        C_f \& C_{n \circ f} \\
                        I_{C_f} \& C \\
                        \& \Sigma C_f, \\
                    };

                    \draw[math]
                        (m-1-1) edge node {\phi} (m-1-2)
                            edge[tailed] node[swap] {\kappa_{C_f}} (m-2-1)
                        (m-1-2) edge[tailed] node[swap] {x} (m-2-2)
                            edge[curve={height=-25pt}] node {0} (m-3-2)

                        (m-2-1) edge node {\gamma} (m-2-2)
                            edge[two headed] node[swap] {\rho_{C_f}} (m-3-2)
                        (m-2-2) edge[two headed] node[swap] {p} (m-3-2);
                \end{tikzpicture}
            \end{center}
            commutes and yields a standard triangle
            \begin{center}
                \begin{tikzpicture}
                    \diagram{m}{1cm}{1cm} {
                        C_f \& C_{n \circ f} \& C \& \Sigma C_f. \\
                    };

                    \draw[math]
                        (m-1-1) edge node {[\phi]} (m-1-2)
                        (m-1-2) edge node {[x]} (m-1-3)
                        (m-1-3) edge node {[p]} (m-1-4);
                \end{tikzpicture}
            \end{center}
            
            Consider the two pushout diagrams:
            \[
                \begin{aligned}
                    \begin{tikzpicture}
                        \diagram{m}{1cm}{1cm} {
                            B \& D \\
                            I_B \& C_n, \\
                        };
    
                        \draw[math]
                            (m-1-1) edge node {n} (m-1-2)
                                edge[tailed] node[swap] {\kappa_B} (m-2-1)
                            (m-1-2) edge[tailed] node {m} (m-2-2)
    
                            (m-2-1) edge node {\gamma_n} (m-2-2);
                    \end{tikzpicture}
                \end{aligned}
                \hspace{0.5cm}
                \text{ and }
                \hspace{0.5cm}
                \begin{aligned}
                    \begin{tikzpicture}
                        \diagram{m}{1cm}{1cm} {
                            B \& D \\
                            I_{C_f} \& C. \\
                        };
    
                        \draw[math]
                            (m-1-1) edge node {n} (m-1-2)
                                edge[tailed] node[swap] {\kappa_{C_f} \circ g} (m-2-1)
                            (m-1-2) edge[tailed] node {\delta} (m-2-2)
    
                            (m-2-1) edge node {\gamma} (m-2-2);
                    \end{tikzpicture}
                \end{aligned}  
            \]
            By \autoref{lem:stmod_pushout_different_injectives_isomorphic} it follows that there exists some \( \alpha: C \to C_n \) such that \( [\alpha] \) is an isomorphism in \( \Mc \), and
            \begin{itemize}
                \item \( \alpha \circ \delta = m \), and
                \item \( \alpha \circ \gamma = \gamma_n \circ i \).
            \end{itemize}

            The final step is then to show that the following diagram is an isomorphism of triangles
            \begin{diagramlabel}[\label{diag:stmod_tr4_iso}]
                \begin{tikzpicture}
                    \diagram{m}{1cm}{1cm} {
                        C_f \& C_{n \circ f} \& C \& \Sigma C_f \\
                        C_f \& C_{n \circ f} \& C_n \& \Sigma C_f. \\
                    };

                    \draw[math]
                        (m-1-1) edge node {[\phi]} (m-1-2)
                            edge[equality] (m-2-1)
                        (m-1-2) edge node {[x]} (m-1-3)
                            edge[equality] (m-2-2)
                        (m-1-3) edge node {[p]} (m-1-4)
                            edge[tailed, two headed] node[swap] {[\alpha]} (m-2-3)
                        (m-1-4) edge[equality] (m-2-4)

                        (m-2-1) edge node {[\phi]} (m-2-2)
                        (m-2-2) edge node {[\psi]} (m-2-3)
                        (m-2-3) edge node {[(\Sigma g) \circ k]} (m-2-4);
                \end{tikzpicture}
            \end{diagramlabel}

            The leftmost square clearly commutes.

            In order to prove that the middle square commutes, consider the following commutative diagram excluding the dashed arrow,
            \begin{center}
                \begin{tikzpicture}
                    \diagram{m}{1cm}{1cm} {
                        A \& B \\
                        I_A \& C_f \\
                        \& \& C_n, \\
                    };

                    \draw[math]
                        (m-1-1) edge node {f} (m-1-2)
                            edge[tailed] node[swap] {\kappa_A} (m-2-1)
                        (m-1-2) edge[tailed] node {g} (m-2-2)
                            edge[curve={height=-25pt}] node {m \circ n} (m-3-3)

                        (m-2-1) edge node {\gamma_f} (m-2-2)
                            edge[curve={height=25pt}] node[swap] {\gamma_n \circ i \circ \kappa_{C_f} \circ \gamma_f} (m-3-3)
                        (m-2-2) edge[dashed] (m-3-3);
                \end{tikzpicture}
            \end{center}
            where the inner square is a pushout and the outer ``square'' commutes by \eqref{diag:stmod_tr4_m_n_f_equal_something_else}.

            Note,
            \begin{align*}
                \psi \circ \phi \circ g &= m \circ n \\
                &= \gamma_n \circ \kappa_B \\
                &= \gamma_n \circ i \circ \kappa_{C_f} \circ g,
            \end{align*}
            and
            \begin{align*}
                \psi \circ \phi \circ \gamma_f &= \psi \circ \gamma_{n \circ f} \\
                &= \gamma_n \circ i \circ \kappa_{C_f} \circ \gamma_f,
            \end{align*}
            which implies by the uniqueness of the pushout morphism that
            \[
                \psi \circ \phi = \gamma_n \circ i \circ \kappa_{C_f}.
            \]

            With the above equality in mind, consider the following commutative diagram excluding the dashed arrow,
            \begin{center}
                \begin{tikzpicture}
                    \diagram{m}{1cm}{1cm} {
                        B \& D \\
                        C_f \& C_{n \circ f} \\
                        \& \& C_n. \\
                    };

                    \draw[math]
                        (m-1-1) edge node {n} (m-1-2)
                            edge[tailed] node[swap] {g} (m-2-1)
                        (m-1-2) edge[tailed] node {j} (m-2-2) 
                            edge[curve={height=-25pt}, tailed] node {m} (m-3-3)

                        (m-2-1) edge node {\phi} (m-2-2)
                            edge[curve={height=25pt}] node[swap] {\alpha \circ \gamma \circ \kappa_{C_f}} (m-3-3)
                        (m-2-2) edge[dashed] (m-3-3);
                \end{tikzpicture}
            \end{center}
            Consider the following two equations,
            \[
                \alpha \circ x \circ j = \alpha \circ \delta = m = \psi \circ j,
            \]
            and
            \[
                \alpha \circ x \circ \phi = \alpha \circ \gamma \circ \kappa_{C_f} = \gamma_n \circ i \circ \kappa_{C_f} = \psi \circ \phi,
            \]
            it follows that by the uniqueness of the pushout that \( \alpha \circ x = \psi \), and so the middle square of \autoref{diag:stmod_tr4_iso} commutes.
            
            It remains to show that the rightmost square commutes, which requires some additional tools.

            Let \( E := \coker(\kappa_{C_f} \circ g) \). Let \( c: I_{C_f} \twoheadrightarrow E \) be the cokernel morphism of \( \kappa_{C_f} \circ g \), and let \( \hat{d}: C \twoheadrightarrow \coker(\delta) \) be a cokernel morphism of \( \delta \). Since \autoref{diag:stmod_tr4_C} is a pushout, there is an isomorphism \( \tilde{d}: \coker(\delta) \to E \), such that \( \tilde{d} \circ \hat{d} = c \circ \gamma \). Let \( d := \tilde{d} \circ \hat{d} \). Then the following diagram,
            \begin{center}
                \begin{tikzpicture}
                    \diagram{m}{1cm}{1cm} {
                        B \& D \\
                        I_{C_f} \& C \\
                        E \& E, \\
                    };

                    \draw[math]
                        (m-1-1) edge node {n} (m-1-2)
                            edge[tailed] node[swap] {\kappa_{C_f} \circ g} (m-2-1)
                        (m-1-2) edge[tailed] node {\delta} (m-2-2)

                        (m-2-1) edge node {\gamma} (m-2-2)
                            edge[two headed] node {c} (m-3-1)
                        (m-2-2) edge[two headed] node {d} (m-3-2)

                        (m-3-1) edge[equality] (m-3-2);
                \end{tikzpicture}
            \end{center}
            commutes, and \( d \) is another choice of cokernel morphism of \( \delta \).

            Let \( e \) and \( e^{-1} \) be the (dashed) morphisms induced by the cokernel properties which fit in the following commutative diagram,
            \begin{center}
                \begin{tikzpicture}
                    \diagram{m}{1cm}{1cm} {
                        B \& I_B \& \Sigma B \\
                        B \& I_{C_f} \& E \\
                        B \& I_B \& \Sigma B. \\
                    };

                    \draw[math]
                        (m-1-1) edge[tailed] node {\kappa_B} (m-1-2)
                            edge[equality] (m-2-1)
                        (m-1-2) edge[two headed] node {\rho_B} (m-1-3)
                            edge node {\tilde{i}} (m-2-2)
                        (m-1-3) edge[dashed] node {e^{-1}} (m-2-3)

                        (m-2-1) edge[tailed] node {\kappa_{C_f} \circ g} (m-2-2)
                            edge[equality] (m-3-1)
                        (m-2-2) edge[two headed] node {c} (m-2-3)
                            edge node {i} (m-3-2)
                        (m-2-3) edge[dashed] node {e} (m-3-3)

                        (m-3-1) edge[tailed] node {\kappa_B} (m-3-2)
                        (m-3-2) edge[two headed] node {\rho_B} (m-3-3);
                \end{tikzpicture}
            \end{center}
            
            We have \( [e^{-1} \circ e] = [\Id_E] \) by the following argument:
            
            Consider the following commutative diagram excluding the dashed arrow,
            \begin{center}
                \begin{tikzpicture}
                    \diagram{m}{1cm}{2cm} {
                        B \& I_{C_f} \& E \\
                        B \& I_{C_f} \& E. \\
                    };

                    \draw[math]
                        (m-1-1) edge[tailed] node {\kappa_{C_f} \circ g} (m-1-2)
                            edge node[swap] {\Id_B - \Id_B = 0} (m-2-1)
                        (m-1-2) edge[two headed] node {\rho_B} (m-1-3)
                            edge node[swap] {\tilde{i} \circ i - \Id_{I_{C_f}}} (m-2-2)
                        (m-1-3) edge[dashed] (m-2-2)
                            edge node {e^{-1} \circ e - \Id_E} (m-2-3)

                        (m-2-1) edge[tailed] node[swap] {\kappa_{C_f} \circ g} (m-2-2)
                        (m-2-2) edge[two headed] node[swap] {\rho_B} (m-2-3);
                \end{tikzpicture}
            \end{center}
            By the cokernel property of \( E \) there exists some morphism such that the top right triangle including the dashed arrow commutes. Furthermore, since \( \rho_B \) is an epimorphism, the bottom right triangle commutes, which implies \( [e^{-1} \circ e] = [\Id_E] \), since \( I_{C_f} \) is a projective module.

            % The above morphisms fit into the following commutative diagram,
            % \begin{center}
            %     \begin{tikzpicture}
            %         \diagram{m}{1cm}{1cm} {
            %             B \& D \\
            %             I_{C_f} \& C \\
            %             E \& E \\
            %             \& \Sigma B. \\
            %         };

            %         \draw[math]
            %             (m-1-1) edge node {n} (m-1-2)
            %                 edge[tailed] node {\kappa_{C_f} \circ g} (m-2-1)
            %             (m-1-2) edge[tailed] node {\delta} (m-2-2)
                        
            %             (m-2-1) edge node {\gamma} (m-2-2)
            %                 edge[two headed] node {c} (m-3-1)
            %             (m-2-2) edge[two headed] node {d} (m-3-2)

            %             (m-3-1) edge[equality] (m-3-2)
            %             (m-3-2) edge node {e} (m-4-2);
            %     \end{tikzpicture}
            % \end{center}
            
            Consider the following commutative diagram excluding the dashed arrows,
            \begin{center}
                \begin{tikzpicture}
                    \diagram{m}{1cm}{1cm} {
                        B \& I_B \& \Sigma B \\
                        B \& I_{C_f} \& E \\
                        C_f \& I_{C_f} \& \Sigma C_f, \\
                    };

                    \draw[math]
                        (m-1-1) edge[tailed] node {\kappa_B} (m-1-2)
                            edge[equality] (m-2-1)
                        (m-1-2) edge[two headed] node {\rho_B} (m-1-3)
                            edge node {\tilde{i}} (m-2-2)
                        (m-1-3) edge node[swap] {e^{-1}} (m-2-3)
                            edge[dashed, curve={height=-25pt}] node {\Sigma g} (m-3-3)

                        (m-2-1) edge[tailed] node {\kappa_{C_f} \circ g} (m-2-2)
                            edge[tailed] node {g} (m-3-1)
                        (m-2-2) edge[two headed] node {c} (m-2-3)
                            edge[equality] (m-3-2)
                        (m-2-3) edge[dashed] node[swap] {\mu} (m-3-3)

                        (m-3-1) edge[tailed] node {\kappa_{C_f}} (m-3-2)
                        (m-3-2) edge[two headed] node {\rho_{C_f}} (m-3-3);
                \end{tikzpicture}
            \end{center}
            and let \( \mu \) be the morphism induced by the cokernel property of \( E \), and let \( \Sigma g := \mu \circ e^{-1} \).

            Before proving \( [(\Sigma g) \circ k \circ \alpha] = [p] \), we first have to prove some relations between the new morphisms and the old ones.

            Since
            \[
                \rho_B \circ i \circ \kappa_{C_f} \circ g = \rho_B \circ \kappa_B = 0,
            \]
            we have the following commutative diagram excluding the dashed arrow,
            \begin{center}
                \begin{tikzpicture}
                    \diagram{m}{1cm}{1cm} {
                        B \& D \\
                        I_{C_f} \& C \\
                        \& \& \Sigma B. \\
                    };

                    \draw[math]
                        (m-1-1) edge node {n} (m-1-2)
                            edge[tailed] node[swap] {\kappa_{C_f} \circ g} (m-2-1)
                        (m-1-2) edge[tailed] node {\delta} (m-2-2)
                            edge[curve={height=-25pt}] node {0} (m-3-3)

                        (m-2-1) edge node {\gamma} (m-2-2)
                            edge[curve={height=25pt}] node {\rho_B \circ i} (m-3-3)
                        (m-2-2) edge[dashed] (m-3-3);
                \end{tikzpicture}
            \end{center}

            Both \( e \circ d \) and \( k \circ \alpha \) could fit as the dashed arrow, as
            \[
                k \circ \alpha \circ \delta = k \circ m = 0 = e \circ 0 = e \circ d \circ \delta,
            \]
            and
            \[
                k \circ \alpha \circ \gamma = k \circ \gamma_n \circ i = \rho_B \circ i = e \circ c = e \circ d \circ \gamma.
            \]
            Therefore, by uniqueness of the universal property of the pushout, \( e \circ d = k \circ \alpha \).

            Similarly, consider the following commutative diagram excluding the dashed arrow,
            \begin{center}
                \begin{tikzpicture}
                    \diagram{m}{1cm}{1cm} {
                        B \& D \\
                        I_{C_f} \& C \\
                        \& \& \Sigma C_f. \\
                    };

                    \draw[math]
                        (m-1-1) edge node {n} (m-1-2)
                            edge[tailed] node[swap] {\kappa_{C_f} \circ g} (m-2-1)
                        (m-1-2) edge[tailed] node {\delta} (m-2-2)
                            edge[curve={height=-25pt}] node {0} (m-3-3)

                        (m-2-1) edge node {\gamma} (m-2-2)
                            edge[curve={height=25pt}, two headed] node {\rho_{C_f}} (m-3-3)
                        (m-2-2) edge[dashed] (m-3-3);
                \end{tikzpicture}
            \end{center}

            We can verify that \( p \) already fits as a dashed arrow, since \( p \circ \delta = p \circ x \circ j = 0 \circ j = 0 \). However, \( \mu \circ d \) also fits, as
            \[
                \mu \circ d \circ \delta = \mu \circ 0 = 0,
            \]
            and
            \[
               \mu \circ d \circ \gamma = \mu \circ c = \rho_{C_f}.
            \]
            Therefore, by the uniqueness of the universal property of the pushout, \( p = \mu \circ d \).

            Combining everything, we have
            \[
                [(\Sigma g) \circ k \circ \alpha] = [(\Sigma g) \circ e \circ d] =[\mu \circ e^{-1} \circ e \circ d] = [\mu \circ d] = [p],
            \]
            which finishes the proof, as \( [\alpha] \) is an isomorphism. \qedhere
        }
    \end{enumerate}
\end{proof}

Thus, we have proven that the stable (infinitely generated) module category is triangulated. A natural follow-up question is therefore if there exists a stable finitely generated module category.
\begin{remark}
    This subsection has only mentioned (infinitely generated) modules. However, the same definitions, functors, and proofs work for the finitely generated modules, \( \mod(R) \). This is because a Frobenius ring is Noetherian, and \( \mod(R) \) is therefore an abelian category. In addition, \( \mod(R) \) also has enough projectives and injectives, and therefore no proof or definition uses any properties of \( \Mod(R) \), that \( \mod(R) \) does not have. This yields a triangulated category, \( \Stmod(R) \) which is the stable module category over \( \mod(R) \).
\end{remark}

By \cite[Lemma, Section 7.5]{Krause_2007} it follows that \( \StMod(R) \) and \( \Stmod(R) \) are in fact what we call an ``algebraic triangulated category.'' Note that his definition might not be equivalent to our definition of an algebraic triangulated category in general, which we will define in \autoref{section:alg_tri_cats}.

\section{Toda brackets}
In this thesis, the focus is solely on Toda brackets definition for triangulated categories. As mentioned in the introduction, the definitions presented are based upon Christensen and Franklands definition from 2017 \cite[Definition 3.1]{Christensen-Frankland_2017}.

\subsection{Definition}
Let \( \Tc \) be a triangulated category. Given the following diagram in \( \Tc \),
\begin{center}
    \begin{tikzpicture}
        \diagram{m}{1cm}{1cm} {
            X_1 \& X_2 \& X_3 \& X_4, \\
        };

        \draw[math]
            (m-1-1) edge node {f_1} (m-1-2)
            (m-1-2) edge node {f_2} (m-1-3)
            (m-1-3) edge node {f_3} (m-1-4);
    \end{tikzpicture}
\end{center}
we can define the \emph{three-fold Toda bracket of \( f_1, f_2, \) and \( f_3 \)} in three different (but actually identical, see \autoref{prop:toda-bracket-definitions-coincide}) ways:

% MS-Question: Problem med at venstre rotasjon ikkje blir skriven på den måten?
\begin{definition}[Toda brackets]
    \label{def:toda_bracket}
    \phantom{hei}

    \begin{enumerate}
        \item {
            Let
            \begin{center}
                \begin{tikzpicture}
                    \diagram{m}{1cm}{1cm} {
                        X_1 \& X_2 \& Y \& \Sigma X_1 \\
                    };

                    \draw[math]
                        (m-1-1) edge node {f_1} (m-1-2)
                        (m-1-2) edge (m-1-3)
                        (m-1-3) edge (m-1-4);
                \end{tikzpicture}
            \end{center}
            be any distinguished triangle in \( \Tc \).

            The set of every possible \( \psi \in \Tc(\Sigma X_1, X_4) \) that makes the following diagram commute,
            \begin{center}
                \begin{tikzpicture}
                    \diagram{m}{1cm}{1cm} {
                        X_1 \& X_2 \& Y \& \Sigma X_1 \\
                        X_1 \& X_2 \& X_3 \& X_4, \\
                    };

                    \draw[math]
                        (m-1-1) edge node {f_1} (m-1-2)
                            edge[equality] (m-2-1)
                        (m-1-2) edge (m-1-3)
                            edge[equality] (m-2-2)
                        (m-1-3) edge (m-1-4)
                            edge (m-2-3)
                        (m-1-4) edge node {\psi} (m-2-4)

                        (m-2-1) edge node {f_1} (m-2-2)
                        (m-2-2) edge node {f_2} (m-2-3)
                        (m-2-3) edge node {f_3} (m-2-4);
                \end{tikzpicture}
            \end{center}
            is denoted as \( \toda{f_3, f_2, f_1}_{\cc} \). This is called the \emph{three-fold iterated cofiber Toda Bracket of \( f_1, f_2, \) and \( f_3 \)}.
        }
        \item {
            Let
            \begin{center}
                \begin{tikzpicture}
                    \diagram{m}{1cm}{1cm} {
                        \Sigma^{-1} Y \& X_2 \& X_3 \& \Sigma \Sigma^{-1} Y \\
                    };

                    \draw[math]
                        (m-1-1) edge (m-1-2)
                        (m-1-2) edge node {f_2} (m-1-3)
                        (m-1-3) edge (m-1-4);
                \end{tikzpicture}
            \end{center}
            be any distinguished triangle in \( \Tc \).

            The set of every composite \( \beta \circ (\Sigma \alpha) \in \Tc(\Sigma X_1, X_4) \) that makes the following diagram commute,
            \begin{center}
                \begin{tikzpicture}
                    \diagram{m}{1cm}{1cm} {
                        X_1 \& X_2 \\
                        \Sigma^{-1} Y \& X_2 \& X_3 \& \Sigma \Sigma^{-1} Y \\
                        \& \& X_3 \& X_4, \\
                    };

                    \draw[math]
                        (m-1-1) edge node {f_1} (m-1-2)
                            edge node {\alpha} (m-2-1)
                        (m-1-2) edge[equality] (m-2-2)

                        (m-2-1) edge (m-2-2)
                        (m-2-2) edge node {f_2} (m-2-3)
                        (m-2-3) edge (m-2-4)
                            edge[equality] (m-3-3)
                        (m-2-4) edge node {\beta} (m-3-4)

                        (m-3-3) edge node {f_3} (m-3-4);
                \end{tikzpicture}
            \end{center}
            is denoted as \( \toda{f_3, f_2, f_1}_{\fc} \). This is called the \emph{three-fold fiber-cofiber Toda Bracket of \( f_1, f_2, \) and \( f_3 \)}.
        }
        \item {
            Let \( \phi = \set*{\phi_A}_{A \in \Tc} \) denote the natural transformation from \( \Sigma \Sigma^{-1} \) to \( \Id_{\Tc} \), and let
            \begin{center}
                \begin{tikzpicture}
                    \diagram{m}{1cm}{1cm} {
                        \Sigma^{-1} X_4 \& Y \& X_3 \& \Sigma \Sigma^{-1} X_4 \\
                    };

                    \draw[math]
                        (m-1-1) edge (m-1-2)
                        (m-1-2) edge (m-1-3)
                        (m-1-3) edge node {\phi_{X_3}^{-1} \circ f_3} (m-1-4);
                \end{tikzpicture}
            \end{center}
            be any distinguished triangle in \( \Tc \).
            
            The set of every morphism \( \phi_{X_3} \circ (\Sigma \delta) \in \Tc(\Sigma X_1, X_4) \), where \( \delta \) makes the following diagram commute
            \begin{center}
                \begin{tikzpicture}
                    \diagram{m}{1cm}{1cm} {
                        X_1 \& X_2 \& X_3 \& X_4 \\
                        \Sigma^{-1} X_4 \& Y \& X_3 \& \Sigma \Sigma^{-1} X_4 \\
                    };

                    \draw[math]
                        (m-1-1) edge node {f_1} (m-1-2)
                            edge node {\delta} (m-2-1)
                        (m-1-2) edge node {f_2} (m-1-3)
                            edge (m-2-2)
                        (m-1-3) edge node {f_3} (m-1-4)
                            edge[equality] (m-2-3)
                        (m-1-4) edge node {\phi_{X_3}^{-1}} (m-2-4)

                        (m-2-1) edge (m-2-2)
                        (m-2-2) edge (m-2-3)
                        (m-2-3) edge node {\phi_{X_3}^{-1} \circ f_3} (m-2-4);
                \end{tikzpicture}
            \end{center}
            is denoted as \( \toda{f_3, f_2, f_1}_{\ff} \). This is called the \emph{three-fold iterated fiber Toda Bracket of \( f_1, f_2, \) and \( f_3 \)}.
        }
    \end{enumerate}
\end{definition}

These are in fact equivalent ways to describe the same subset of \( \Tc(\Sigma X_1, X_4) \) which is clear by the following proposition.

\begin{proposition}
    \label{prop:toda-bracket-definitions-coincide}
    All three of the Toda brackets definitions in \autoref{def:toda_bracket} are equal.
\end{proposition}

For a proof of \autoref{prop:toda-bracket-definitions-coincide}, see \cite[Proposition 3.3]{Christensen-Frankland_2017}. Note that they specify the natural transformations between \( \Sigma \Sigma^{-1} \) and \( \Id_{\Tc} \), however, they use them implicit when it fits as mentioned in \cite[p.\ 2690]{Christensen-Frankland_2017}.

As a consequence of \autoref{prop:toda-bracket-definitions-coincide} we can uniquely define the \emph{Toda bracket of \( f_1, f_2, \) and \( f_3 \)} as either \( \toda{f_3, f_2, f_1}_{\cc} \), \( \toda{f_3, f_2, f_1}_{\fc} \), or \( \toda{f_3, f_2, f_1}_{\ff} \), and it is denoted simply as \emph{\( \toda{f_3, f_2, f_1} \)}.

Note that Toda brackets are well-defined in the sense that different choices of distinguished triangles yields the same subset. We will be using a (rotated) standard triangle in our calculations.

% TODO: Bevis? Skjønne ikkje korleis Daria sitt bevis er fullstendig.
% TODO: Trur det må vere non-empty.
Another property helpful in calculations, is the ``indeterminacy'' of a Toda bracket. It is the result of the following lemma, which we will omit the proof of.
\begin{lemma}
    \label{lem:indeterminacy}
    Let the following be a diagram in a triangulated category \( \Tc \),
    \begin{center}
        \begin{tikzpicture}
            \diagram{m}{1cm}{1cm} {
                X_1 \& X_2 \& X_3 \& X_4. \\
            };
    
            \draw[math]
                (m-1-1) edge node {f_1} (m-1-2)
                (m-1-2) edge node {f_2} (m-1-3)
                (m-1-3) edge node {f_3} (m-1-4);
        \end{tikzpicture}
    \end{center}

    If \( \toda{f_3, f_2, f_1} \) is non-empty, then the Toda bracket \( \toda{f_3, f_2, f_1} \) is a coset of the subgroup,
    \[
        f_3 \circ \Tc(\Sigma X_1, X_3)  + \Tc(\Sigma X_2, X_4) \circ (\Sigma f_1) \subseteq \Tc(\Sigma X_1, X_4),
    \]
    which is called the \emph{indeterminacy of \( \toda{f_3, f_2, f_1} \)}.
\end{lemma}

\subsection{Examples}
To get a better understanding of how to compute Toda brackets, as well as illustrate some properties they have, this section contains some examples for \( \Mc := \Stmod(R) \) where \( R := \Fb_2 C_2 \), which is a known Frobenius ring.

The following remark contains some preliminary information that is helpful in the following computations:

\begin{remark}
	\label{rem:toda_bracket_examples_properties}
    Let \( g \) denote the generator of \( C_2 \).

    The ideal \( J := \tuple*{1 + g} \) is the only non-trivial ideal in \( R \) (up to isomorphism).

    The ideal \( J \) is not projective, since the short exact sequence
    \begin{center}
        \begin{tikzpicture}
            \diagram{m}{1cm}{1cm}{
                J \& R \& J, \\
            };
        
            \draw[math]
                (m-1-1) edge[tailed] node {\kappa_J} (m-1-2)
                (m-1-2) edge[two headed] node {\rho_J} (m-1-3);
        \end{tikzpicture}
    \end{center}
    where \( \kappa_J \) is the inclusion and \( \rho_J \) is the morphism
    \begin{align*}
        \rho_J: R &\to J \\
        0, 1 + g &\mapsto 0 \\
        1, g &\mapsto 1 + g,
    \end{align*}
    is not split.

    This is because \( \kappa_J \) is the only monomorphism of \( J \) into \( R \), but composes to \( 0 \) with \( \rho_J \). Therefore, \( J \) cannot be projective, since every epimorphism into a projective module splits.

    The suspension \( \Sigma J \) is assumed to be the cokernel of \( \kappa_J \), which by the short exact sequence above, can be assumed to be \( J \).

	% Furthermore, since \( \rho_J \) is an epimorphism with kernel \( J \), we get from the third isomorphism theorem that \( \frac{R}{J} \cong J \). Let the isomorphism from \( \frac{R}{J} \) to \( J \) also be denoted as \( \rho_J \).
\end{remark}

The following example gives the idea of how to calculate Toda brackets in the simplest cases.

\begin{example}
	\label{ex:toda_bracket_1}
	Let the following be a diagram in \( \Mc \)
	\begin{center}
		\begin{tikzpicture}
			\diagram{m}{1cm}{1cm}{
					J \& J \& J \& J. \\
			};

			\draw[math]
				(m-1-1) edge node {[\Id_J]} (m-1-2)
				(m-1-2) edge node {[0]} (m-1-3)
				(m-1-3) edge node {[\Id_J]} (m-1-4);
		\end{tikzpicture}
	\end{center}
	
	The goal is to calculate the Toda bracket \( \toda{[\Id_J], [0], [\Id_J]} \).

	Using the iterated cofiber definition of Toda brackets, we need to choose a distinguished triangle. We will use a standard triangle of \( [\Id_J] \), which is the trivial triangle,
	\begin{center}
		\begin{tikzpicture}
			\diagram{m}{1cm}{1cm} {
				J \& J \& 0 \& J. \\
			};

			\draw[math]
				(m-1-1) edge node {[\Id_J]} (m-1-2)
				(m-1-2) edge node {[0]} (m-1-3)
				(m-1-3) edge node {[0]} (m-1-4);
		\end{tikzpicture}
	\end{center}

	By the definition of \( \toda{f_3, f_2, f_1}_{cc} \), we get the following diagram
	\begin{center}
		\begin{tikzpicture}
			\diagram{m}{1cm}{1cm} {
				J \& J \& 0 \& J \\
				J \& J \& J \& J. \\
			};

			\draw[math]
				(m-1-1) edge node {[\Id_J]} (m-1-2)
					edge[equality] (m-2-1)
				(m-1-2) edge node {[0]} (m-1-3)
					edge[equality] (m-2-2)
				(m-1-3) edge node {[0]} (m-1-4)
					edge node {[\rho]} (m-2-3)
				(m-1-4) edge node {[\psi]} (m-2-4)

				(m-2-1) edge node {[\Id_J]} (m-2-2)
				(m-2-2) edge node {[0]} (m-2-3)
				(m-2-3) edge node {[\Id_J]} (m-2-4);
		\end{tikzpicture}
	\end{center}

	Since \( [\rho] = [0] \), any endomorphism on \( J \) makes the rightmost square commute. In addition, since \( \Mod(R)(J, J) \) is simply \( \set*{[0], [\Id_J]} \), this implies that \( \Mc(J, J) = \set*{[0], [\Id_J]} \).

	Therefore, \( \toda{[\Id_J], [0], [\Id_J]} = \set*{[0], [\Id_J]} \).

	Another way we could have proven this, is by calculating the indeterminacy (\autoref{lem:indeterminacy}), which would be
	\[
		[\Id_J] \circ \Mc(J, J) + \Mc(J, J) \circ [\Id_J] = \Mc(J, J),
	\]
	which only has one coset, namely \( \Mc(J, J) \).
\end{example}

The previous example demonstrates that the Toda bracket is dependent on the choice of \( I \) in \autoref{def:stmod_sigma}. In the previous example, we could have assumed \( \Sigma J \) to be \( R/J \), which would have yielded the Toda bracket \( \toda{[\Id_J], [0], [\Id_J]} = \set*{[0], [\mu]} \), where \( \mu \) is the unique isomorphism \( R/J \cong J \). Therefore, for applications of Toda brackets on stable module categories, it is important to keep in mind that the choice of \( I \) in the definition of \( \Sigma \) is consistent.

\begin{example}
	\label{ex:toda_bracket_2}
	We want to calculate \( \toda{[\Id_J], [\Id_J], [\Id_J]} \).

	We have the following diagram by the iterated cofiber definition of Toda brackets using the assumptions made in \autoref{rem:toda_bracket_examples_properties}
	\begin{center}
		\begin{tikzpicture}
			\diagram{m}{1cm}{1cm} {
				J \& J \& 0 \& J \\
				J \& J \& J \& J. \\
			};

			\draw[math]
				(m-1-1) edge node {[\Id_J]} (m-1-2)
					edge[equality] (m-2-1)
				(m-1-2) edge node {[0]} (m-1-3)
					edge[equality] (m-2-2)
				(m-1-3) edge node {[0]} (m-1-4)
					edge[squiggly] node {[\rho]} (m-2-3)
				(m-1-4) edge node {[\psi]} (m-2-4)

				(m-2-1) edge node {[\Id_J]} (m-2-2)
				(m-2-2) edge node {[\Id_J]} (m-2-3)
				(m-2-3) edge node {[\Id_J]} (m-2-4);
		\end{tikzpicture}
	\end{center}

	Here there is an issue. There exists no \( [\rho]: 0 \to J \) such that \( [\rho] \circ [0] = [0]: J \to J \) is equal to \( [\Id_J] \). It could only be true if \( J \cong 0 \), which it is not, since \( J \) is not projective by the argument in \autoref{rem:toda_bracket_examples_properties}.

	Therefore, the Toda bracket is \emph{empty}, i.e., \( \toda{[\Id_J], [\Id_J], [\Id_J]} = \emptyset \).
\end{example}

The previous example shows another property of Toda brackets, namely that it can be empty. The following proposition and proof shows this property for a general case.

\begin{proposition}
	\label{prop:toda_pairwise_non_vanishing_is_empty}
	Let \( f_1, f_2, \) and \( f_3 \) be three composable morphisms in any triangulated category \( \Tc \),
	\begin{center}
		\begin{tikzpicture}
			\diagram{m}{1cm}{1cm} {
				X_1 \& X_2 \& X_3 \& X_4, \\
			};

			\draw[math]
				(m-1-1) edge node {f_1} (m-1-2)
				(m-1-2) edge node {f_2} (m-1-3)
				(m-1-3) edge node {f_3} (m-1-4);
		\end{tikzpicture}
	\end{center}
	such that \( f_2 \circ f_1 \neq 0 \) or \( f_3 \circ f_2 \neq 0 \).

	Then \( \toda{f_3, f_2, f_1} = \emptyset \).
\end{proposition}
\begin{proof}
	Assume that \( \toda{f_3, f_2, f_1} \neq \emptyset \). Then from the iterated cofiber definition of Toda brackets, there exists morphisms \( \alpha, \beta, \phi, \) and \( \psi \) such that the following diagram commutes, and the top row is distinguished,
	\begin{center}
		\begin{tikzpicture}
			\diagram{m}{1cm}{1cm} {
				X_1 \& X_2 \& C_{f_1} \& \Sigma X_1 \\
				X_1 \& X_2 \& X_3 \& X_4. \\
			};

			\draw[math]
				(m-1-1) edge node {f_1} (m-1-2)
					edge[equality]	(m-2-1)
				(m-1-2) edge node {\alpha} (m-1-3)
					edge[equality] (m-2-2)
				(m-1-3) edge node {\beta} (m-1-4)
					edge node {\phi} (m-2-3)
				(m-1-4) edge node {\psi} (m-2-4)

				(m-2-1) edge node {f_1} (m-2-2)
				(m-2-2) edge node {f_2} (m-2-3)
				(m-2-3) edge node {f_3} (m-2-4);
		\end{tikzpicture}
	\end{center}

	We split the proof into two different contradictions:

	\begin{itemize}
		\item{
			Case 1:

			Assume that \( f_2 \circ f_1 \neq 0 \). Since \( \alpha, f_1 \) are two composable morphisms from the same distinguished triangle, we have
			\[
				\phi \circ \alpha \circ f_1 = \phi \circ 0 = 0.
			\]

			But from commutativity of the diagram we also have
			\[
				\phi \circ \alpha \circ f_1 = f_2 \circ f_1 \neq 0,
			\]
			which is a contradiction.
		}
		\item{
			Case 2:

			Assume that \( f_3 \circ f_2 \neq 0 \). Then we have that
			\[
				0 = \psi \circ 0 = \psi \circ \beta \circ \alpha = f_3 \circ \phi \circ \alpha = f_3 \circ f_2 \neq 0,
			\]
			which is also a contradiction.
		}
	\end{itemize}

	Therefore, both \( f_2 \circ f_1 = 0 \) and \( f_3 \circ f_2 = 0 \) if \( \toda{f_3, f_2, f_1} \neq \emptyset \), which is contrapositive to the statement in the theorem.
\end{proof}

The following example is a generalization of \autoref{ex:toda_bracket_1} to any triangulated category \( \Tc \).

\begin{example}
	\label{ex:toda_bracket_3}
	We want to compute \( \toda{f, 0, \Id} \) for any triangulated category \( \Tc \) and for any \( f \in \Tc(X_3, X_4) \).

	We use the iterated cofiber definition of Toda brackets.

	Using the trivial triangle as the distinguished triangle, we get the following diagram
	\begin{center}
		\begin{tikzpicture}
			\diagram{m}{1cm}{1cm} {
				X_1 \& X_1 \& 0 \& \Sigma X_1 \\
				X_1 \& X_1 \& X_3 \& X_4. \\
			};

			\draw[math]
				(m-1-1) edge node {\Id} (m-1-2)
					edge[equality] (m-2-1)
				(m-1-2) edge (m-1-3)
					edge[equality] (m-2-2)
				(m-1-3) edge (m-1-4)
					edge (m-2-3)
				(m-1-4) edge node {\psi} (m-2-4)

				(m-2-1) edge node {\Id} (m-2-2)
				(m-2-2) edge node {0} (m-2-3)
				(m-2-3) edge node {f} (m-2-4);
		\end{tikzpicture}
	\end{center}

	Here we have that any possible \( \psi: \Sigma X_1 \to X_4 \) will make the right square commute. Therefore, \( \toda{f, 0, \Id} = \Tc(\Sigma X_1, X_4) \).

	Calculating the indeterminacy,
	\[
		f \circ \Tc(\Sigma X_1, X_3) + \Tc(\Sigma X_1, X_4) \circ \Id_{\Sigma X_1} = \Tc(\Sigma X_1, X_4),
	\]
	yields the same result.
\end{example}

Finally, the last example will be in \( \Mc \), but where the computations are a bit more complicated.

\begin{example}
	\label{ex:toda_bracket_4}
	We want to compute the Toda bracket of the following morphisms using the iterated cofiber definition first, and then using the indeterminacy afterwards,
	\begin{center}
		\begin{tikzpicture}
			\diagram{m}{1cm}{1cm} {
				J \& {J \oplus J} \& {J \oplus J} \& J. \\
			};

			\draw[math]
				(m-1-1) edge node {\class*{\begin{psmallmatrix} 1 \\ 1 \end{psmallmatrix}}} (m-1-2)
				(m-1-2) edge node {\class*{\begin{psmallmatrix} 1 & 1 \\ 1 & 1 \end{psmallmatrix}}} (m-1-3)
				(m-1-3) edge node {\class*{\begin{psmallmatrix} 1 & 1 \end{psmallmatrix}}} (m-1-4);
		\end{tikzpicture}
	\end{center}

	First, we find a standard triangle of \( \class*{\begin{psmallmatrix} 1 \\ 1 \end{psmallmatrix}}: J \to J \oplus J \).

	The cone is defined as the pushout of the following diagram,
	\begin{center}
		\begin{tikzpicture}
			\diagram{m}{1cm}{1cm} {
				J \& J \oplus J \\
				R, \\
			};

			\draw[math]
				(m-1-1) edge node {\begin{psmallmatrix} 1 \\ 1 \end{psmallmatrix}} (m-1-2)
					edge[tailed] node[swap] {\kappa_J} (m-2-1);
		\end{tikzpicture}
	\end{center}
	where \( \kappa_J \) is assumed to be as in \autoref{rem:toda_bracket_examples_properties}.

	In the category of \( \Mod(R) \), the pushout becomes
	\begin{center}
		\begin{tikzpicture}
			\diagram{m}{1cm}{1cm} {
				J \& J \oplus J \\
				R \& (J \oplus J \oplus R)/\sim, \\
			};

			\draw[math]
				(m-1-1) edge node {\begin{psmallmatrix} 1 \\ 1 \end{psmallmatrix}} (m-1-2)
					edge[tailed] node[swap] {\kappa_J} (m-2-1)
				(m-1-2) edge[tailed] node {\rho} (m-2-2)

				(m-2-1) edge node {\gamma} (m-2-2);
		\end{tikzpicture}
	\end{center}
	where \( (1 + g, 1 + g, 0) \sim (0, 0, 1 + g) \).

	The morphism \( \rho \) is given by the composition
	\begin{center}
		\begin{tikzpicture}
			\diagram{m}{1cm}{1cm} {
				J \oplus J \& J \oplus J \oplus R \& (J \oplus J \oplus R)/\sim \\
			};

			\draw[math]
				(m-1-1) edge[tailed] node {i} (m-1-2)
				(m-1-2) edge[two headed] node {\pi} (m-1-3);
		\end{tikzpicture}
	\end{center}
	where \( i \) is the embedding, and \( \pi \) is the quotient epimorphism.

	We can check that the cone is isomorphic to \( J \oplus R \) via the morphism
	\begin{align*}
		\alpha: (J \oplus J \oplus R)/\sim &\to J \oplus R \\
		(0, 0, r) &\mapsto (0, r) \\
		(1 + g, 0, r) &\mapsto (1 + g, r + 1 + g) \\
		(0, 1 + g, r) &\mapsto (1 + g, r) \\
		(1 + g, 1 + g, r) &\mapsto (0, r + 1 + g). \\
	\end{align*}

	Therefore, by checking the morphism \( \alpha \circ \rho \) we can see that it becomes
	\[
		\begin{pmatrix}
			1 & 1 \\
			\kappa_J & 0 \\
		\end{pmatrix}
		: J \oplus J \to J \oplus R.
	\]

	By a similar argument for \( \gamma \), we get that it is simply the embedding into \( J \oplus R \), here denoted as \( \begin{psmallmatrix} 0 \\ 1 \\ \end{psmallmatrix}. \)

	Thus, we can rewrite the pushout to the form
	\begin{center}
		\begin{tikzpicture}
			\diagram{m}{1cm}{1cm} {
				J \& J \oplus J \\
				R \& J \oplus R. \\
			};

			\draw[math]
				(m-1-1) edge node {\begin{psmallmatrix} 1 \\ 1 \end{psmallmatrix}} (m-1-2)
					edge[tailed] node[swap] {\kappa_J} (m-2-1)
				(m-1-2) edge[tailed] node {\begin{psmallmatrix} 1 & 1 \\ \kappa_J & 0 \\ \end{psmallmatrix}} (m-2-2)

				(m-2-1) edge node {\begin{psmallmatrix} 0 \\ 1 \\ \end{psmallmatrix}} (m-2-2);
		\end{tikzpicture}
	\end{center}
	
	Furthermore, the morphism \( J \oplus R \to \Sigma J = J \) is given as the unique pushout morphism \( \beta \), satisfying the following commutative diagram
	\begin{center}
		\begin{tikzpicture}
			\diagram{m}{1cm}{1cm} {
				J \& J \oplus J \\
				R \& J \oplus R \\
				\& J. \\
			};

			\draw[math]
				(m-1-1) edge node {\begin{psmallmatrix} 1 \\ 1 \end{psmallmatrix}} (m-1-2)
					edge[tailed] node[swap] {\kappa_J} (m-2-1)
				(m-1-2) edge node[swap] {\begin{psmallmatrix} 1 & 1 \\ \kappa_J & 0 \\ \end{psmallmatrix}} (m-2-2)
					edge[curve={height=-25pt}] node {0} (m-3-2) 

				(m-2-1) edge[tailed] node {\begin{psmallmatrix} 0 \\ 1 \\ \end{psmallmatrix}} (m-2-2)
					edge node {\rho_J} (m-3-2)
				(m-2-2) edge node {\beta} (m-3-2);
		\end{tikzpicture}
	\end{center}

	We can check that a candidate for the morphism \( \beta \) is \( \begin{psmallmatrix} 0 & \rho_J \\ \end{psmallmatrix}. \) By uniqueness of the pushout property, this morphism is \( \beta \).

	Therefore, the standard triangle becomes
	\begin{center}
		\begin{tikzpicture}
			\diagram{m}{1cm}{2cm} {
				J \& J \oplus J \& J \oplus R \& J. \\
			};

			\draw[math]
				(m-1-1) edge node {\class*{\begin{psmallmatrix} 1 \\ 1 \end{psmallmatrix}}} (m-1-2)
				(m-1-2) edge node {\class*{\begin{psmallmatrix} 1 & 1 \\ \kappa_J & 0 \\ \end{psmallmatrix}}} (m-1-3)
				(m-1-3) edge node {\class*{\begin{psmallmatrix} 0 & \rho_J \\ \end{psmallmatrix}}} (m-1-4);
		\end{tikzpicture}
	\end{center}

	In \( \Mc \), \( R \cong 0 \), and the morphism \( \class*{\begin{psmallmatrix} 1 & 0 \\ \end{psmallmatrix}}: J \oplus R \to J \) becomes an isomorphism which yields the following distinguished triangle
	\begin{center}
		\begin{tikzpicture}
			\diagram{m}{1cm}{2cm} {
				J \& J \oplus J \& J \& J. \\
			};

			\draw[math]
				(m-1-1) edge node {\class*{\begin{psmallmatrix} 1 \\ 1 \end{psmallmatrix}}} (m-1-2)
				(m-1-2) edge node {\class*{\begin{psmallmatrix} 1 & 1 \\ \end{psmallmatrix}}} (m-1-3)
				(m-1-3) edge node {\class*{0}} (m-1-4);
		\end{tikzpicture}
	\end{center}

	Using the iterated cofiber definition of Toda brackets, we get the following commutative diagram
	\begin{center}
		\begin{tikzpicture}
			\diagram{m}{1cm}{1cm} {
				J \& {J \oplus J} \& J \& J \\
				J \& {J \oplus J} \& {J \oplus J} \& J, \\
			};

			\draw[math]
				(m-1-1) edge node {\class*{\begin{psmallmatrix} 1 \\ 1 \end{psmallmatrix}}} (m-1-2)
					edge[equality] (m-2-1)
				(m-1-2) edge node {\class*{\begin{psmallmatrix} 1 & 1 \\ \end{psmallmatrix}}} (m-1-3)
					edge[equality] (m-2-2)
				(m-1-3) edge node {[0]} (m-1-4)
					edge node {\phi} (m-2-3)
				(m-1-4) edge node {\psi} (m-2-4)

				(m-2-1) edge node {\class*{\begin{psmallmatrix} 1 \\ 1 \end{psmallmatrix}}} (m-2-2)
				(m-2-2) edge node {\class*{\begin{psmallmatrix} 1 & 1 \\ 1 & 1 \end{psmallmatrix}}} (m-2-3)
				(m-2-3) edge node {\class*{\begin{psmallmatrix} 1 & 1 \end{psmallmatrix}}} (m-2-4);
		\end{tikzpicture}
	\end{center}
	where the top row is distinguished.

	Now, we have to find every possible \( \psi \) such that the above diagram commutes for some \( \phi \).

	Start by assuming
	\[ 
		\phi = \class*{
			\begin{psmallmatrix}
				1 \\
				1
			\end{psmallmatrix}
		},
	\]
	this makes the square to the left of \( \phi \) commute, in addition, it makes the square to the right of \( \phi \) also commute, because
	\[
		\class*{\begin{psmallmatrix} 1 & 1 \end{psmallmatrix}} \circ
		\class*{\begin{psmallmatrix}
			1 \\
			1
		\end{psmallmatrix}}
		=
		[2] = [0].
	\]

	Thus, with the assumed \( \phi \), any \( \psi \) will make the diagram commute. Therefore,
	\[ 
		\toda{\class*{\begin{psmallmatrix} 1 & 1 \end{psmallmatrix}}, \class*{\begin{psmallmatrix} 1 & 1 \\ 1 & 1 \end{psmallmatrix}}, \class*{\begin{psmallmatrix} 1 \\ 1 \end{psmallmatrix}}} = \Mc(J, J).
	\]
	
	Then we want to try the same calculations using indeterminacy:
	
	Consider the subgroup
	\[
		\class*{\begin{psmallmatrix} 1 & 1 \end{psmallmatrix}} \circ \Mc(J, J \oplus J) + \Mc(J \oplus J, J) \circ \tuple*{\Sigma \class*{\begin{psmallmatrix} 1 \\ 1 \end{psmallmatrix}}}.
	\]
	For any \( [f] \in \Mc(J, J) \),
	\[
		\class*{\begin{psmallmatrix} 1 & 1 \end{psmallmatrix}} \circ \class*{\begin{psmallmatrix} f \\ 0 \end{psmallmatrix}} = [f],
	\]
	we get that
	\[
		\class*{\begin{psmallmatrix} 1 & 1 \end{psmallmatrix}} \circ \Mc(J, J \oplus J) = \Mc(J, J),
	\]
	and so
	\[
		\toda{\class*{\begin{psmallmatrix} 1 & 1 \end{psmallmatrix}}, \class*{\begin{psmallmatrix} 1 & 1 \\ 1 & 1 \end{psmallmatrix}}, \class*{\begin{psmallmatrix} 1 \\ 1 \end{psmallmatrix}}} = \Mc(J, J).
	\]
\end{example}

If, only using the Toda bracket definition, at the end of the final example we were not so lucky and could not have chosen a \( \phi \), such that every \( \psi \) was expressed, the calculations would have become much more tedious. Then we would have to find every possible \( \phi \), and from every possible \( \phi \) we would then find every possible \( \psi \). In addition, working with morphisms in \( \Mc \) is not simple, and we could end up with a very complicated and annoying calculation. Combining this with the fact that simply calculating an appropriate distinguished triangle is tedious, and the calculations become even worse. Calculating the Toda bracket using the indeterminacy, at least in the above cases turned out to be much simpler.

This illustrates the reason for wanting to combine Massey products and Toda brackets as mentioned in the introduction, as additional tools, such as indeterminacy, might significantly simplify the calculations in certain cases.

\section{Massey products}
\subsection{DG-categories}
Before defining Massey products on the cohomology category of a DG-category, we first have to define what a DG-category is.

We will use the definition of DG-categories based on enriched category theory, as it is both modern, and is helpful for defining the DG-functor category in subsequent sections. The enriched category theory in this section is mainly inspired by \cite[Section 6.2]{Borceux_1994}.

The first definition necessary to understanding DG-categories, is the tensor product of chain complexes over modules. However, this definition uses coproducts, and we will use special notation for elements in the coproduct.

\begin{notation}
    \label{not:coprod}
    Let \( R \) be a commutative ring with identity. Let \( A_i \in \Mod(R) \) and let
    \[
        \iota_i: A_i \rightarrowtail \coprod_{i \in \Zb} A_i
    \]
    denote the canonical split monomorphism by the universal property of the coproduct in \( \Mod(R) \).

    Then for any \( a_i \in A_i \), the element
    \[
        \iota_i(a_i) \in \coprod_{i \in \Zb} A_i
    \]
    is just denoted as
    \[
        a_i \in \coprod_{i \in \Zb} A_i.
    \]
    % Furthermore, let 
    % \[
    %     \pi_i: \prod_{i \in \Zb} A_i \twoheadrightarrow A_i
    % \]
    % be the universal split epimorphism by the universal property of the product in \( \Mod(R) \).
    
    % Then for any element \( a \in \prod_{i \in \Zb} A_i \), denote \( a \) as
    % \[
    %     a = \tuple{a_i}_{i \in \Zb}, \: \text{where} \: a_i := \pi_i(a) \in A_i.
    % \]
    
    % The reasoning behind this notation is twofold. First, the product in \( \Mod(R) \) is the direct product, and direct products are usually denoted this way. Second, by the universal property of the product, a morphism out of \( \prod_{i \in \Zb} A_i \) is fully dependent on what the resulting value in each degree is, and the above notation captures that property.

    % However, there is another issue with products specifically, and that is changing the base, and how it relates with equality. The convention that we will use in this thesis is best explained with an example.
    
    % Consider two chain complexes \( A \), and \( B \), we can construct the set \( \prod_{j \in \Zb} \Mod(R)(A_{j - 5}, B_j) \). In this thesis, this set is \emph{equal} to \( \prod_{j \in \Zb} \Mod(R)(A_j, B_{j + 5}) \) even though the \( j \)-s don't match. This is because we consider the product to have no order. This simplifies some results, but may be unintuitive when looking at what happens at specific degrees (as is common in proofs).
\end{notation}

The reasoning for the above notation is twofold. First, it reduces notation while not being ambiguous. Second, we never use a general element of a coproduct. Almost always when considering what a morphism does to an element of the coproduct, it is what happens to the \( \iota_i(a_i) \)'s, which is a consequence of the universal property of the coproduct.

The following is the definition of the tensor product of chain complexes.

\begin{definition}[Tensor product of \( \C \)]
    \label{def:tensor_product_of_chain_complexes_over_Mod(R)}
    Let \( R \) be a commutative ring with identity. Furthermore, let \( A, B \in \C \).

    Then, for any  \( n \in \Zb \) define the modules
    \[
        (A \otimes B)_n := \coprod_{p + q = n} A_p \otimes B_q
    \]
    which are a part of the chain complex
    \begin{center}
        \begin{tikzpicture}
            \diagram{m}{1cm}{1cm} {
                A \otimes B: \\
            };
        \end{tikzpicture}
        %
        \begin{tikzpicture}
            \diagram{m}{1cm}{1cm} {
                \cdots \& \tuple*{A \otimes B}_{-1} \& \tuple*{A \otimes B}_0 \& \tuple*{A \otimes B}_1 \& \cdots \\
            };

            \draw[math]
                (m-1-1) edge (m-1-2)
                (m-1-2) edge node {d_{-1}} (m-1-3)
                (m-1-3) edge node {d_0} (m-1-4)
                (m-1-4) edge (m-1-5);
        \end{tikzpicture}
    \end{center}
    where the differentials, \( d_n \), are defined as follows:
    
    Let \( i + j = n \) and \( a \otimes b \in A_i \otimes B_j \) be an elementary tensor.

    Then the differential is (uniquely) defined by the following assignments
    \[
        d_n(a \otimes b) := d_{A, i}(a) \otimes b + (-1)^{i} a \otimes d_{B, j}(b).
    \]

    This is called the \emph{tensor product of \( \C \)}.
\end{definition}

In order to see that the above definition of the differential on the chain complex is well-defined, we need the following lemma, which will be useful later as well.

WIP: Marius tilbakemelding. Hugs å flytta notasjonen over.

\begin{lemma}
    \label{lem:map_out_of_tensor_unique}
    Let \( A, B \in \C \) and let \( C \in \Mod(R) \). Furthermore, let \( i, j \in \Zb \) with \( i + j = n \).

    Then for any
    \[
        f: (A \otimes B)_n \to C
    \]
    where for any \( a \in A_i, b \in B_j \) we have
    \[
        f(a \otimes b) = g_{i, j}(a, b)
    \]
    for some \( R \)-bilinear morphisms
    \[
        g_{i, j}: A_i \times B_j \to C.
    \]

    Then \( f \) is uniquely defined by the \( g_{i, j} \)'s.
\end{lemma}
\begin{proof}
    Consider at the following diagram where \( \iota_{i, j} \) is the canocial split monomorphism by the universal property of the coproduct, and where the \( g_{i, j} \)'s are \( R \)-bilienar
    \begin{diagramlabel}[\label{tikz:differential_of_tensor_product_of_chain_complexes_over_Mod(R)}]
        \begin{tikzpicture}
            \diagram{m}{2cm}{2cm} {
                A_i \otimes B_j \& \coprod\limits_{p + q = n} A_p \otimes B_q \\
                A_i \times B_j \& C. \\
            };

            \draw[math]
                (m-1-1) edge[tailed] node {\iota_{i, j}} (m-1-2)
                    edge[dashed] node {\alpha_{i, j}} (m-2-2)
                (m-1-2) edge[dashed] node {\beta} (m-2-2)

                (m-2-1) edge node {g_{i, j}} (m-2-2)
                    edge node {\otimes} (m-1-1);
        \end{tikzpicture}
    \end{diagramlabel}

    Then by the universal property of tensor product in \( \Mod(R) \), \( g_{i, j} \) induces a unique morphism, \( \alpha_{i, j} \), which is induced from the elementary tensors as follows
    \[
        a \otimes b \mapsto g_{i, j}(a, b).
    \]

    Since this works for any \( i, j \) as long as \( i + j = n \), we can construct \( \alpha_{i, j} \) for every valid \( i, j \) pair.

    Then by using the universal property of the coproduct we get the unique map \( \beta \) which is by \autoref{tikz:differential_of_tensor_product_of_chain_complexes_over_Mod(R)} uniquely determined by its actions on elementary tensors \( a \otimes b \in A_i \otimes B_j \) in the following way
    \[
        \beta: a \otimes b \mapsto g_{i, j}(a, b)
    \]
    which is exactly equal to \( f \), and \( f \) is therefore uniquely deterined by the \( g_{i, j} \)'s.
\end{proof}

The following remark shows that \autoref{def:tensor_product_of_chain_complexes_over_Mod(R)} is well-defined.

\begin{remark}
    The definition of the differentials in \autoref{def:tensor_product_of_chain_complexes_over_Mod(R)} is well-defined and unique by the following argument.

    We can check that for \( i + j = n \) that
    \begin{align*}
        g_{i, j}: A_i \times B_j &\to (A \otimes B)_{n + 1} \\
        (a, b) &\mapsto d_{A, i}(a) \otimes b + (-1)^i a \otimes d_{B, j}(b)
    \end{align*}
    is \( R \)-bilinear.

    Then by \autoref{lem:map_out_of_tensor_unique} it follows that \( d_n \) is uniquely defined.

    Similarly, by seeing that
    \[
        d_{n + 1} \circ d_n: a \otimes b \mapsto 0
    \]
    and the \( 0 \) map is \( R \)-bilinear, so \( d_{n + 1} \circ d_n \) is uniquely defined. Since the zero map would also send \( a \otimes b \) to \( 0 \), then by uniquenes \( d_{n + 1} \circ d_n = 0 \).
\end{remark}

An important property of the tensor product of chain complexes is symmetry. This is shown in the following remark.

\begin{remark}[Symmetry of tensor product in \( \C \)]
    \label{rem:symmetry_tensor_product_of_chain_complex}
    For two \( A, B \in \C \), there exist an isomorphism
    \[
        s = \tuple{s_n}_{n \in \Zb}: A \otimes B \to B \otimes A
    \]
    that for any \( n \) and \( i + j = n \) with \( a \in A_i \) and \( b \in B_j \) is defined on the \( n \)-th component as follows,
    \[
        s_n: a \otimes b \mapsto (-1)^{ij} b \otimes a.
    \]

    By \autoref{lem:map_out_of_tensor_unique} every \( s_n \) is well-defined. It is also clear that every \( s_n \) is an isomorphism by checking injectivity and surjectivity. It only remains to verify that \( s = \tuple{s_n}_{n \in \Zb} \) is a chain morphism.

    Consider the following difference
    \begin{align*}
        &s_{n + 1} \circ d_n(a \otimes b) - d_n \circ s_n (a \otimes b) = s_{n + 1} \tuple{d_{A, i}(a) \otimes b + (-1)^i a \otimes d_{B, j}(b)} - d_n \tuple{ (-1)^{ij} b \otimes a } \\
        &= (-1)^{(i + 1)j} b \otimes d_{A, i}(a) + (-1)^{i + i(j + 1)} d_{B, j}(b) \otimes a - (-1)^{ij} d_{B, j}(b) \otimes a - (-1)^{j + ij} b \otimes d_{A, i}(a) \\
        &= (-1)^{(i + 1)j} b \otimes d_{A, i}(a) - (-1)^{(i + 1)j} b \otimes d_{A, i}(a) + (-1)^{ij} d_{B, j}(b) \otimes a - (-1)^{ij} d_{B, j}(b) \otimes a \\
        &= 0,
    \end{align*}
    which implies that the morphism \( s_{n + 1} \circ d_n - d_n \circ s_n \) sends every elementary tensor to \( 0 \), which again implies that the morphism is equal to \( 0 \).
\end{remark}

Another definition closely linked to the tensor product in \( \C \) is the internal hom.

\begin{definition}[Internal hom of \( \C \)]
    \label{def:internal_hom_of_chain_complexes_over_Mod(R)}
    Let \( R \) be a commutative ring with identity. Furthermore, let \( A, B \in \C \).

    Then, for any \( n \in \Zb \) define the modules
    \[
        \class*{A, B}_n := \prod_{i \in \Zb} \Mod(R)(A_i, B_{i + n})
    \]
    which are a part of the chain complex
    \begin{center}
        \begin{tikzpicture}
            \diagram{m}{1cm}{1cm} {
                \class*{A, B}: \&[-0.5cm] \cdots \& \class*{A, B}_{-1} \& \class*{A, B}_0 \& \class*{A, B}_1 \& \cdots \\
            };

            \draw[math]
                (m-1-2) edge (m-1-3)
                (m-1-3) edge node {d_{-1}} (m-1-4)
                (m-1-4) edge node {d_0} (m-1-5)
                (m-1-5) edge (m-1-6);
        \end{tikzpicture}
    \end{center}

    where the differentials, \( d_n \), are defined as follows,
    \begin{align*}
        d_n : \class*{A, B}_n &\to \class*{A, B}_{n + 1} \\
        \tuple*{f_i}_{i \in \Zb} &\mapsto \tuple*{d_{B, i + n} \circ f_i - (-1)^n f_{i + 1} \circ d_{A, i}}_{i \in \Zb}.
    \end{align*}
    This is called the \emph{internal hom of chain complexes over \( \Mod(R) \)}.
\end{definition}

The following remark explains why the differentials in the previous definition is well-defined.

\begin{remark}
    The definition of the differential in \autoref{def:internal_hom_of_chain_complexes_over_Mod(R)} is well-defined because the only thing that needs to be checked is if the \( d_n \)'s are differentials, which can be checked straight forwardly.
\end{remark}

As the names would imply, the tensor product and internal hom of chain complexes over modules are adjoint in the usual sense. This is shown in the following remark.

\begin{remark}[Tensor product and internal hom adjunction in \( \C \)]
    \label{rem:tensor_prod_internal_hom_adjoint}
    Let \( A, B, C \) be chain complexes in \( \C \) for some commutative ring \( R \), and let
    \[
        f \in \C\tuple*{A \otimes B, C}.
    \]
    
    The goal of this remark is to understand what the adjoint of \( f \) is.

    Let \( f = \tuple*{f_n}_{n \in \Zb} \) where \( f_n \in \Mod(R)\tuple*{ \tuple*{A \otimes B}_n, C_n} \) are the individual level-wise morphisms of the chain morphism \( f \).

    Then, unpacking the definitions, we have that \( f \) looks like the following diagram
    \begin{center}
        \begin{tikzpicture}
            \diagram{m}{1cm}{1cm} {
                \cdots \& \coprod\limits_{i + j = -1} A_i \otimes B_j \& \coprod\limits_{i + j = 0} A_i \otimes B_j \& \coprod\limits_{i + j = 1} A_i \otimes B_j \& \cdots \\
                \cdots \& C_{-1} \& C_0 \& C_1 \& \cdots \\
            };

            \draw[math]
                (m-1-1) edge (m-1-2)
                (m-1-2) edge node {d_{A \otimes B, -1}} (m-1-3)
                    edge node {f_{-1}} (m-2-2)
                (m-1-3) edge node {d_{A \otimes B, 0}} (m-1-4)
                    edge node {f_0} (m-2-3)
                (m-1-4) edge (m-1-5)
                    edge node {f_1} (m-2-4)

                (m-2-1) edge (m-2-2)
                (m-2-2) edge node {d_{C, -1}} (m-2-3)
                (m-2-3) edge node {d_{C, 0}} (m-2-4)
                (m-2-4) edge (m-2-5);
        \end{tikzpicture}
    \end{center}
    Likewise, the adjoint has to look like the following diagram
    \begin{center}
        \begin{tikzpicture}
            \diagram{m}{1cm}{0.70cm} {
                \cdots \&[-0.5cm] A_{-1} \& A_0 \& A_1 \&[-0.5cm] \cdots \\
                \cdots \& \prod\limits_{j \in \Zb} \Mod(R)(B_j, C_{j - 1}) \& \prod\limits_{j \in \Zb} \Mod(R)(B_j, C_j) \& \prod\limits_{j \in \Zb} \Mod(R)(B_j, C_{j + 1}) \& \cdots \\
            };

            \draw[math]
                (m-1-1) edge (m-1-2)
                (m-1-2) edge node {d_{A, -1}} (m-1-3)
                    edge node {?_{-1}} (m-2-2)
                (m-1-3) edge node {d_{A, 0}} (m-1-4)
                    edge node {?_0} (m-2-3)
                (m-1-4) edge (m-1-5)
                    edge node {?_1} (m-2-4)

                (m-2-1) edge (m-2-2)
                (m-2-2) edge node {d_{\class*{B, C}, -1}} (m-2-3)
                (m-2-3) edge node {d_{\class*{B, C}, 0}} (m-2-4)
                (m-2-4) edge (m-2-5);
        \end{tikzpicture}
    \end{center}

    % For any \( n \in \Zb \), let \( i', j' \in \Zb \) with \( i' + j' = n \) let
    % \[
    %     \iota_{i', j'}: A_{i'} \otimes B_{j'} \rightarrowtail \coprod_{i + j = n} A_i \otimes B_j
    % \]
    % be the canonical split monomorphism by the definintion of the coproduct \( \coprod_{i + j = n} A_i \otimes B_j \).
    Let \( \iota_{i, j} \) be as defined in \autoref{not:coprod}.

    Then take the hom-tensor adjoint in \( \Mod(R) \) of the morphism
    \[
        f_{i + j} \circ \iota_{i, j}: A_i \otimes B_j \to C_{i + j}.
    \]
    This yields a morphism
    \begin{align*}
        \phi_{f, i, j}: A_i &\to \Mod(R)(B_j, C_{j + i}) \\
        a &\mapsto f_{i + j}\tuple*{ a \otimes ? }.
    \end{align*}
    Then by the universal property of the product there is some morphism
    \[
        \phi_{f, i} := \prod_{j \in \Zb} \phi_{f, i, j}: A_i \to \prod_{j \in \Zb} \Mod(R)\tuple*{B_j, C_{j + i}}.
    \]
    Collecting these morphisms yields a morphism, which is a candidate for the adjoint of \( f \), namely
    \[
        \phi_f := \tuple*{\phi_{f, i}}_{i \in \Zb}.
    \]
    In order to show that this is the proper adjoint definition, we need to show the following properties:
    \begin{enumerate}
        \item {
            First, \( \phi_f \) is a chain morphism.
        }
        \item {
            Second, the assignment \( f \mapsto \phi_f \) is an isomorphism of groups.
        }
        \item {
            Third, there is a natural transformation
            \[
                \C(?_1 \otimes ?_2, ?_3) \cong \C(?_1, \left[ ?_2, ?_3 \right])
            \]
            where the natural morphisms are \( f \mapsto \phi_f \).
        }
    \end{enumerate}
    % TODO: Prove or SRC.
    In this thesis only the first statement will be proven.

    Want to show that \( \phi_f \) is a chain morphism.

    Need to check that for any \( i \in \Zb \) that the following diagram commutes
    \begin{center}
        \begin{tikzpicture}
            \diagram{m}{1cm}{1cm} {
                A_i \& A_{i + 1} \\
                \class*{B, C}_i \& \class*{B, C}_{i + 1} \\
            };

            \draw[math]
                (m-1-1) edge node {d_{A, i}} (m-1-2)
                    edge node {\phi_{f, i}} (m-2-1)
                (m-1-2) edge node {\phi_{f, i + 1}} (m-2-2)

                (m-2-1) edge node {d_{\class*{B, C}, i}} (m-2-2);
        \end{tikzpicture}
    \end{center}
    Pick an arbitrary \( a \in A_i \) and consider the following equation
    \begin{align*}
        \phi_{f, i + 1} \circ d_{A, i}(a) &- d_{\class*{B, C}, i} \circ \phi_{f, i}(a)
        = \tuple*{ f_{i + j + 1}\tuple*{d_{A, i}(a) \otimes ?} }_{j \in \Zb}
        - d_{\class*{B, C}, i} \tuple*{ \tuple*{ f_{i + j}\tuple*{a \otimes ?} }_{j \in \Zb} } \\
        \intertext{by expanding out the definition of \( d_{\class*{B, C}, i} \) it follows that}
        &= \tuple*{ f_{i + j + 1}\tuple*{d_{A, i}(a) \otimes ?}
        - d_{C, i + j} \circ f_{i + j}\tuple*{a \otimes ?}
        + (-1)^i f_{i + j + 1}\tuple*{a \otimes d_{B, j}(?)} }_{j \in \Zb} \\
        \intertext{by consolodating the two terms that post-compose by \( f_{i + j + 1} \) it follows that}
        &= \tuple*{ f_{i + j + 1}\bigl( \tuple*{d_{A, i}(a) \otimes ?}
        + (-1)^i\tuple*{ a \otimes d_{B, j}(?) } \bigr)
        - d_{C, i + j} \circ f_{i + j}\tuple*{ a \otimes ? } }_{j \in \Zb} \\
        \intertext{by the definition of the differential of \( A \otimes B \) it follows that}
        &= \tuple*{ f_{i + j + 1} \circ d_{A \otimes B, i + j} \tuple*{ a \otimes ? }
        - d_{C, i + j} \circ f_{i + j} ( a \otimes ? ) }_{j \in \Zb} \\
        \intertext{by \( f \) being a chain homomorphism from \( A \otimes B \) to \( C \) it follows that}
        &= 0.
    \end{align*}
\end{remark}

Together the above definitions and remarks gives the structure of a symmetric monoidal category, which will not be shown in this thesis. For a definition of symmetric monoidal categories, see \cite[Definition 6.1.1]{Borceux_1994}.

\begin{fact}[\( \C \) is symmetric monoidal]
    Let \( R \) be a commutative ring with identity, and let \( \otimes \) denote the tensor product on \( \C \). Furthermore let \( I \) be the chain complex in \( \C \) consisting solely of \( 0 \)-objects exept for the \( R \)-module \( R \) in index \( 0 \).

    Then \( \tuple*{\C, \otimes, I} \) is a symmetric closed monoidal category.
\end{fact}

Finally we can define what a DG-category is.

\begin{definition}[DG-category]
    \label{def:dg_cat}
    Let \( R \) be a commutative ring with identity.

    Then \( \Cc \) is a \emph{DG-category over \( R \)} if it is a category enriched over \( \C \).
\end{definition}

This definition also appears in \cite[p. 29]{Jasso-Muro_2023}, except they define it for a field and not a commutative ring with identity as is done in this thesis.


\subsection{Definitions}
In this subsection, the goal is to define what a Massey product is.

Start off with defining the notation that will be used for the cohomology functor.

\begin{notation}
    Let \( R \) be a commutative ring with identity.

    Then let
    \[
        H^\bullet: \C \to \C
    \]
    be the cohomology functor, with the differentials on the right hand side all being \( 0 \).
\end{notation}

Now there is an issue with usual category theory notation of a ``diagram'' for DG-categories. This is because a DG-category doesn't have the notion of a morphism since there is no element of a chain complex. Therefore there can't be a diagram since the arrows would correspond to an elements of a chain complex. However, having a notion of a diagram in a DG-category would make certain future results easier to both state and understand, at the cost of some difficulty in the proofs. And that is why the following notation convention will be used.

% TODO: Maybe move this notation above the definition of H-bullet of a DG category as it is used in the definition.
% TODO: Remove all mentions of iota_i, j's in every massety section. They are implied.
% TODO: Maybe a separate notation for composition of morphisms, since morphisms are not always a apart of a dg-diagram.
\begin{notation}
    It is prudent to properly define what a ``diagram'' in a DG-category is by the discussion above.

    Let \( \Cc \) be a DG-category. A \emph{diagram in a DG-category} (or a \emph{DG-diagram}) is a quiver \( \Gamma = \tuple*{V, E, s, t} \) where every vertex in \( V \) corresponds to an object in \( \Cc \), and every edge \( e \in E \) corresponds to an element in \( \Cc\tuple*{s(e), t(e)}_n \) for some \( n \in \Zb \). These edges are called \emph{DG-morphisms} and \( n \) is called the \emph{degree} of \( e \), and is also denoted as \( |e| \).

    For two morphisms \( f, g \in E \) with \( s(g) = t(f) \), they are denoted as \emph{composable}. And their composition, denoted \( g \circ f \), is the morphism
    \[
        g \circ f := c_{\Cc, |g| + |f|}(g \otimes f) \in \Cc\tuple*{s(f), t(g)}_{|g| + |f|},
    \]
    where
    \[
        c_{\Cc} := \set*{c_{\Cc, i}}_{i \in \Zb} \text{ with } c_{\Cc, i}: \tuple*{ \Cc\tuple*{s(g), t(g)} \otimes \Cc\tuple*{s(f), t(f)} }_i \to \Cc\tuple*{s(f), t(g)}_i
    \]
    is the composition chain morphism for \( \Cc \).
\end{notation}

Composition in a DG-diagram is essentially usual composition in the DG-category, but restricted to just one component of the coproduct.

In order to justify using this notation, there are som properties that are helpful to know. First, that composition as defined above is associative.

\begin{lemma}[Associativity of composition in a DG-diagram]
    \label{lem:dg-composition_associative}
    Let the following be a DG-diagram in a DG-category \( \Cc \)
    \begin{center}
        \begin{tikzpicture}
            \diagram{m}{1cm}{1cm} {
                A \& B \& C \& D. \\
            };

            \draw[math]
                (m-1-1) edge node {f} (m-1-2)
                (m-1-2) edge node {g} (m-1-3)
                (m-1-3) edge node {h} (m-1-4);
        \end{tikzpicture}
    \end{center}

    Then \( h \circ (g \circ f) = (h \circ g) \circ f \).
\end{lemma}
\begin{proof}
    Expanding the definitions, it is necessary to show that the following equation holds (the category is omitted for readability)
    \begin{equation}
        \label{eq:dg-composition_associative}
        c_{|h| + |g| + |f|}\tuple*{h \otimes \tuple*{c_{|g| + |f|}(g \otimes f)}} = c_{|h| + |g| + |f|}\tuple*{\tuple*{c_{|h| + |g|}(h \otimes g)} \otimes f}.
    \end{equation}
    % TODO: This needs a proof. In RM: "Associativity of dg composition" p. 2, there is a diagram of how the diagram could exist. Would need to show that the stated composition of (natural?) isomorphisms are in fact exactly the associativity map.
    Assume (without proof) that the following diagram in \( \Mod(R) \), where the top horizontal morphism is the usual associativity morphism of the tensor product of \( R \)-modules, and \( a \) is the associativity morphism for the tensor product of chain complexes (in degree \( |h| + |g| + |f| \) ),
    \begin{center}
        \begin{tikzpicture}
            \diagram{m}{1cm}{0.45cm} {
                \Cc(C, D)_{|h|} \otimes \tuple*{ \Cc(B, C)_{|g|} \otimes \Cc(A, B)_{|f|} } \& \tuple*{ \Cc(C, D)_{|h|} \otimes \Cc(B, C)_{|g|} } \otimes \Cc(A, B)_{|f|} \\
                \Cc(C, D)_{|h|} \otimes \tuple*{ \Cc(B, C) \otimes \Cc(A, B) }_{|g| + |f|} \& \tuple*{ \Cc(C, D) \otimes \Cc(B, C) }_{|h| + |g|} \otimes \Cc(A, B)_{|f|} \\
                \tuple*{ \Cc(C, D) \otimes ( \Cc(B, C) \otimes \Cc(A, B) ) }_{|h| + |g| + |f|} \& \tuple*{ ( \Cc(C, D) \otimes \Cc(B, C) ) \otimes \Cc(A, B) }_{|h| + |g| + |f|}, \\
            };

            \draw[math]
                (m-1-1) edge node {\sim} (m-1-2)
                    edge node {\Id \otimes \iota_{|g|, |f|}} (m-2-1)
                (m-1-2) edge node {\iota_{|h|, |g|} \otimes \Id} (m-2-2)

                (m-2-1) edge node {\iota_{|h|, |g| + |f|}} (m-3-1)
                (m-2-2) edge node {\iota_{|h| + |g|, |f|}} (m-3-2)

                (m-3-1) edge node {\sim} node[swap] {a} (m-3-2);
        \end{tikzpicture}
    \end{center}
    commutes.

    And by \cite[Definition 6.2.1]{Borceux_1994} the following diagram of chain morphisms commute
    \begin{center}
        \begin{tikzpicture}
            \diagram{m}{1cm}{1cm} {
                \Cc(C, D) \otimes \tuple*{ \Cc(B, C) \otimes \Cc(A, B) } \& \& \tuple*{ \Cc(C, D) \otimes \Cc(B, C) } \otimes \Cc(A, B) \\
                \Cc(C, D) \otimes \Cc(A, C) \& \& \Cc(B, D) \otimes \Cc(A, B) \\
                \& \Cc(A, D). \\
            };
            
            \draw[math]
                (m-1-1) edge node {\sim} node[swap] {a} (m-1-3)
                    edge node {\Id \otimes c} (m-2-1)
                (m-1-3) edge node {c \otimes \Id} (m-2-3)

                (m-2-1) edge node {c} (m-3-2)
                (m-2-3) edge node[swap] {c} (m-3-2);
        \end{tikzpicture}
    \end{center}
    Gluing together the top diagram with the bottom diagram restricted to degree \( |h| + |g| + |f| \), and looking at where the element \( h \otimes (g \otimes f) \) is sent, yields exactly \autoref{eq:dg-composition_associative}.
\end{proof}

In addition to being associative, it would also be nice to prove that composition is \( R \)-bilinear, which is proven in the following lemma.

\begin{lemma}
    Composition in a DG-diagram is a morphism in \( \Mod(R) \), in particular, it is \( R \)-linear in both components.
\end{lemma}
\begin{proof}
    By definition, composition in a DG-diagram is the composition of two \( \Mod(R) \) morphisms, \( \iota_{i, j} \) and \( c_{\Cc, i + j} \), and so it is a \( \Mod(R) \) morphism.
\end{proof}

Now the notation should be sufficient for working with DG-categories.

Before defining the Massey product, we first have to define the category the product is taken in. That is the following category, the cohomology category of any DG-category.

\begin{definition}[Cohomology category, \( H^\bullet(\Cc) \)]
    \label{def:H_bullet_dg_category}
    Let \( \Cc \) be a differentially graded category over  \( R \).

    Let \( H^\bullet(\Cc) \) be the following (enriched over \( C\tuple*{\Mod(R)} \)) category:
    \begin{enumerate}
        \item Let \( \Obj(H^\bullet(\Cc)) := \Obj(\Cc) \).
        \item For any \( A, B \in H^\bullet(\Cc) \), let \( H^\bullet(\Cc)(A, B) := H^\bullet \tuple*{\Cc(A, B)} \).
        \item {
            For any \( A, B, C \in H^\bullet(\Cc) \), define the composition morphism
            \begin{align*}
                c_{H^\bullet(\Cc)}: H^\bullet(\Cc)(B, C) \otimes H^\bullet(\Cc)(A, B) &\to H^\bullet(\Cc)(A, C)
            \end{align*}
            to be the chain morphism with the \( n \)-th component being
            \begin{align*}
                c_n: (H^\bullet(\Cc)(B, C) \otimes H^\bullet(\Cc)(A, B))_n &\to H^n(\Cc(A, C)) \\
                [g_i] \otimes [f_j] &\mapsto \class*{g_i \circ f_j}
            \end{align*}
            for any \( i, j \in \Zb \) with \( i + j = n \) and \( [g_i] \in H^i(\Cc)(B, C) \) and \( [f_j] \in H^j(\Cc)(A, B) \).
        }
        \item {
            Let \( u: I_A \to \Cc(A, A) \) be the unit morphism for \( \Cc \).
            
            For \( A \in H^\bullet(\Cc) \), the unit morphism of \( H^\bullet(\Cc) \) is defined as the chain morphism
            \[
                H^\bullet(u): I \to H^\bullet(\Cc)(A, A).
            \]
        }
    \end{enumerate}

    Then \( H^\bullet(\Cc) \) is called the \emph{cohomology category of \( \Cc \)}.
\end{definition}

In order to show that composition is well-defined the following remark can make the proof a bit simpler.

\begin{remark}
    \label{rem:H_bullet_composition_alpha}
    The composition definition in \autoref{def:H_bullet_dg_category} is actually the composition of two different morphisms.

    Consider the maps
    \begin{align*}
        \alpha_{i, j}: H^i(\Cc(B, C)) \times H^j(\Cc(A, B)) &\to H^n(\Cc(B, C) \otimes \Cc(A, B)) \\
        ([g_i], [f_j]) &\mapsto \class*{g_i \otimes f_j}.
    \end{align*}
    Assuming these are well-defined \( R \)-bilinear morphisms, denote the unique morphism they define by \autoref{lem:map_out_of_tensor_unique} by \( \alpha_n \), and the entire chain morphism by \( \alpha := \set*{\alpha_i}_{i \in \Zb} \). \footnote{
        The map \( \alpha \) is know at the cross product morphism, as explained in \cite[p. 273]{Hatcher_2002}. In addition for \( R \) a field (as is assumed in \cite{Jasso-Muro_2023}) it is known that by the Algebraic Künneth Theorem that \( \alpha \) is an isomorphism \cite[Theorem 3B.5]{Hatcher_2002}.
    }
    
    Then we can see that
    \[
        c_{H^\bullet(\Cc)} := H^\bullet(c_{\Cc}) \circ \alpha.
    \] 
\end{remark}

Using the above remark, since \( H^{\bullet}(c_{\Cc}) \) is already a well-defined morphism, it is only neccesary to show that \( \alpha \) exists, is well-defined and is unique.

\begin{remark}
    \label{rem:composition_in_H_bullet_is_well_defined}
    The composition definition in \autoref{def:H_bullet_dg_category} is well-defined and unique by the following argument.

    By \autoref{rem:H_bullet_composition_alpha} it is sufficient to only have to verify that the \( \alpha_{i, j} \)'s are well-defined and \( R \)-balanced in order to show that \( c_{H^\bullet(\Cc)} \) is well-defined and unique.

    We can check that the maps are \( R \)-balanced, but we still need to check if the maps are well-defined, which is a bit more difficult. There are two points that need to be shown:
    \begin{enumerate}
        \item {
            Firstly, is
            \[
                g_i \otimes f_j \in H^n(\Cc(B, C) \otimes \Cc(A, B))?
            \]
            To show this, we have to verify that
            \[
                g_i \otimes f_j \in \ker(d_{\Cc(B, C) \otimes \Cc(A, B), n}).
            \]
            This is true because by assumption, both \( g_i \) and \( f_j \) are cycles, and by definition of the differential of the tensor product
            \[
                d_{\Cc(B, C) \otimes \Cc(A, B), n}(g_i \otimes f_j) = 0.
            \]
        }
        \item {
            Secondly, are the values of the \( \alpha_{i, j} \)'s independent of the choice of representative?

            Let \( b_g \) be a boundary in \( \Cc(B, C)_i \), and let \( b_f \) be a boundary in \( \Cc(A, B)_j \).
            \begin{align*}
                \alpha_{i, j}([g_i + b_g], [f_j + b_f]) &= [(g_i + b_g) \otimes (f_j + b_f)] \\
                &= [g_i \otimes f_j + g_i \otimes b_f + b_g \otimes f_j + b_g \otimes b_f] \\
                &= [g_i \otimes f_j + g_i \otimes b_f + b_g \otimes f_j + b_g \otimes b_f]
            \end{align*}
            By the definition of the differential of the tensor product, we have that the tensor product betweeen a boundary and a cycle, a cycle and a boundary, and a boundary and a boundary are all boundaries in the tensor product.

            As an example, let's take the case above of \( g_i \otimes b_f \). Since \( b_f \) is a boundary in \( \Cc(A, B)_j \), there is some \( b_f' \in \Cc(A, B)_{j - 1} \) such that \( d_{\Cc(A, B), j - 1}(b_f') = b_f \). Consider the following equation
            \begin{align*}
                d_{\Cc(B, C) \otimes \Cc(A, B)}(&(-1)^i g_i \otimes b_f') \\
                &= (-1)^i d_{\Cc(B, C), i}(g_i) \otimes b_f' + (-1)^i (-1)^i g_i \otimes d_{\Cc(A, B), j - 1}(b_f') \\
                &= (-1)^i 0 \otimes b_f' + g_i \otimes b_f \\
                &= g_i \otimes b_f.
            \end{align*}
            A similar argument can be made for the other cases.

            Therefore is follows that
            \[
                \alpha_{i, j}([g_i + b_g], [f_j + b_f]) = [g_i \otimes f_j + g_i \otimes b_f + b_g \otimes f_j + b_g \otimes b_f] = [g_i \otimes f_j]
            \]
            And \( \alpha_{i, j} \) is therefore well-defined.
        }
    \end{enumerate}
    There is no need to verify if the composition is a chain morphism, since the differential of \( H^\bullet(\Cc)(A, C) \) is zero by definition.
\end{remark}

This thesis will refrain from proving that the above definition of \( H^\bullet(\Cc) \) and any future alleged DG-category actually satisfy the associativity axiom and the unit axiom as defined by \cite[Definition 6.2.1]{Borceux_1994}. This is because doing so would require defining the associativity morphism (properly), as well as the left and right tensor unit, only to do proofs that are straight forward to prove using the lemmas already defined. Defining the morphisms and doing the proofs would require a lot of boilerplate for something that, in the end, is not necessary for the goal of this thesis.

Therefore in this thesis, both the associativity axiom and the unit axiom is assumed to be true for every defined DG-category.

The following is the definition of the Massey product on the cohomology category of any DG-category. The definition is the 3-fold variant of \cite[Definition 4.2.1]{Jasso-Muro_2023}.

% SRC: Jasso-Muro
% TODO: What is the relation between different reprersentatives and different choices of s & t? Does it matter, and how does it matter?
% TODO: Write definition clearer? A lot of implied domains and codomains.
\begin{definition}
    \label{def:massey_product_dg_cat}
    Let \( \Cc \) be a differentially graded category over a commutative ring with identity \( R \).

    Let the following be a DG-diagram in \( H^\bullet(\Cc) \)
    \begin{center}
        \begin{tikzpicture}
            \diagram{m}{1cm}{1cm} {
                X_1 \& X_2 \& X_3 \& X_4 \\
            };

            \draw[math]
                (m-1-1) edge node {[f_1]} (m-1-2)
                (m-1-2) edge node {[f_2]} (m-1-3)
                (m-1-3) edge node {[f_3]} (m-1-4);
        \end{tikzpicture}
    \end{center}

    Furthermore for an element, \( h \in \C_{|h|} \) (like a morphism in a DG-diagram), let \( \bar{h} := (-1)^{|h| + 1}h \).

    Then let
    \begin{multline*}
        \massey{[f_3], [f_2], [f_1]} :=
        \{
            \class*{
                \bar{s} \circ g_1 + \bar{g_3} \circ t
            }
            \mid [g_i] = [f_i], i = 1, 2, 3 \quad \\
            d_{\Cc, |f_3| + |f_2| - 1}(s) = \bar{g_3} \circ g_2, \,
            d_{\Cc, |f_2| + |f_1| - 1}(t) = \bar{g_2} \circ g_1
        \}.
    \end{multline*}

    This is a subset of
    \[
        H^{|f_1| + |f_2| + |f_3| - 1}\tuple*{\Cc\tuple*{X_1, X_4}}
    \]
    and is called the \emph{Massey product of \( [f_3], [f_2] \) and \( [f_1] \)}.
\end{definition}

The following remark shows that \autoref{def:massey_product_dg_cat} is well-defined.

\begin{remark}
    The definition of Massey product in \autoref{def:massey_product_dg_cat} is well-defined by the following argument:

    Want to show that \( \bar{s} \circ g_1 + \bar{g_3} \circ t \) is a cocyle in \( \Cc(X_1, X_4)_{|g_1| + |g_2| + |g_3| - 1} \).

    Let \( n := |g_1| + |g_2| + |g_3| \), and omit the degrees of the differentials for readability. Consider the following equation
    \begin{align*}
        d_{\Cc(X_1, X_4)}(\bar{s} \circ g_1 &+ \bar{g_3} \circ t) = d_{\Cc(X_1, X_4)}(\bar{s} \circ g_1) + d_{\Cc(X_1, X_4)}(\bar{g_3} \circ t) \\
        \intertext{by the definition of composition of morphisms in a DG-diagram, it follows that}
        &= d_{\Cc(X_1, X_4)}(c_{\Cc, n - 1}(\bar{s} \otimes g_1)) + d_{\Cc(X_1, X_4)}(c_{\Cc, n - 1}(\bar{g_3} \otimes t)) \\
        \intertext{by the fact that composition is a chain morphism, it follows that}
        &= c_{\Cc, n}(d_{\Cc(X_2, X_4) \otimes \Cc(X_1, X_2)}(\bar{s} \otimes g_1)) + c_{\Cc, n}(d_{\Cc(X_3, X_4) \otimes \Cc(X_1, X_3)}(\bar{g}_3 \otimes t))
        \intertext{by the definition of the differential of the tensor products, it follows that}
        &= c_{\Cc, n}((-1)^{|g_2| + |g_3|}(\bar{g}_3 \circ g_2) \otimes g_1) + c_{\Cc, n}((-1)^{|g_3|}\bar{g}_3 \otimes (\bar{g}_2 \circ g_1)) \\
        \intertext{by pulling out every sign, it follows that}
        &= (-1)^{|g_2| + 2|g_3| - 1}c_{\Cc, n}((g_3 \circ g_2) \otimes g_1) + (-1)^{|g_2| + 2|g_3| - 2}c_{\Cc, n}(g_3 \otimes (g_2 \circ g_1)) \\
        \intertext{by the definition of composition of morphisms in a DG-diagram, it follows that}
        &= (-1)^{|g_2| + 2|g_3| - 1}(g_3 \circ g_2) \circ g_1 + (-1)^{|g_2| + 2|g_3| - 2}g_3 \circ (g_2 \circ g_1) \\
        \intertext{by simplifying signs, it follows that}
        &= (-1)^{|g_2|}\tuple*{ g_3 \circ (g_2 \circ g_1) - (g_3 \circ g_2) \circ g_1 } \\
        \intertext{by \autoref{lem:dg-composition_associative} it follows that}
        &= 0.
    \end{align*}
    % Secondly, what to show that the massey product is independent of the choice of cocyle-representatives.

    % Let \( b_i \in \Cc(X_i, X_{i + 1}) \) for \( i = 1, 2 \) and \( 3 \) be boundaries.

    % Then look at
    % \[
    %     \set*{ \class*{ \overline{s} \circ (g_1 + b_1) + (\overline{g_3 + b_3}) \circ t } \mid d(s) = (\overline{g_3 + b_3}) \circ (g_2 + b_2), \quad d(t) = (\overline{g_2 + b_2}) \circ (g_1 + b_1) }.
    % \]
    
    % By linearity, the first part can be written as 
    % \[
    %     [\overline{s} \circ g_1 + \overline{s} \circ b_1 + g_3 \circ t + b_3 \circ t].
    % \]
    % But since \( b_1 \) is a boundary, by a similar argument as in \autoref{rem:composition_in_H_bullet_is_well_defined} item 2, we have that
    % \[
    %     \overline{s} \circ b_1 = c_{|g_1| + |g_2| + |g_3| - 1}( \iota_{|g_2| + |g_3| - 1, |g_1|} ( \overline{s} \otimes b_1 ) )
    % \]
    % is in fact
    % \[
    %     \overline{s} \circ b_1 = c_{|g_1| + |g_2| + |g_3| - 1}( d_{\Cc(X_2, X_4) \otimes \Cc(X_1, X_2), |g_1| + |g_2| + |g_3| - 2} (\tilde{g}) )
    % \]
    % for some \( \tilde{g} \in (\Cc(X_2, X_4) \otimes \Cc(X_1, X_2))_{|g_1| + |g_2| + |g_3| - 2} \).

    % And since \( c \) is a chain morphism, it follows that
    % \[
    %     \overline{s} \circ b_1 = d_{\Cc(X_1, X_4), |g_1| + |g_2| + |g_3| - 2}(c_{|g_1| + |g_2| + |g_3| - 2}(\tilde{g}))
    % \]
    % and so \( \overline{s} \circ b_1 \) is a boundary, and a similar argument can be made of \( b_3 \circ t \).

    % It remains to show that
    % \[
    %     \set*{ \class*{ \overline{s} \circ g_1 + \overline{g_3} \circ t } \mid d(s) = (\overline{g_3 + b_3}) \circ (g_2 + b_2), \quad d(t) = (\overline{g_2 + b_2}) \circ (g_1 + b_1) }.
    % \]
    % is independent of the \( b_i \)'s.
\end{remark}

Astute readers may notice that nowhere in the definition of a Massey product does it state or imply that the cohomology category of \( \Cc \), \( H^\bullet(\Cc) \), is triangulated. Which is necessary in order to compare Massey products to Toda brackets. And even if it were triangulated, the scope of categories that could be equivalent to a category consisiting of chain complexes with zero differential is very small, which would make applicability an issue. However, by moving over to the category of ``DG-modules'' which will be defined in the next section, we can define the Massey product on a big class of triangulated categories, namely the ``algebraic triangulated categories''.


\subsection{Examples}
TODO: MS-Question: Skal laga eksempel for generelle Massey produkt eller venta til Massey porodukt på alg. tri. kat?

\section{Toda brackets equal Massey products}
\subsection{Proof}
\input{sections/5.1_proof.tex}

\section{Applications on Adams spectral sequences}
\subsection{Projective and injective classes}
\input{sections/6.1_projective_and_injective_classes.tex}

\subsection{Adams resolution}
\input{sections/6.2_adams_resolutions.tex}

\subsection{Adams spectral sequence and the page 2 differential}
\input{sections/6.3_adams_spectral_sequence_and_page_2_differential.tex}

\subsection{Application of Massey products}
\input{sections/6.4_application_of_massey_products.tex}

\appendix
\section{Appendix}
\subsection{Properties of \texorpdfstring{\( \Stmod\tuple{ \Fb_3 C_3 } \)}{Stmod(F\_3C\_3)}}
\input{sections/A1_properties_of_stmod(F_3C_3).tex}


\addcontentsline{toc}{section}{References}
\bibliography{thesis}{}
\bibliographystyle{utcaps}

\end{document}
